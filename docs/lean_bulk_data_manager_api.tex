% Lean Bulk Data Manager Module API Documentation (Accurate Analysis)
% To be included in master LaTeX document
%
% Usage: % Lean Bulk Data Manager Module API Documentation (Accurate Analysis)
% To be included in master LaTeX document
%
% Usage: % Lean Bulk Data Manager Module API Documentation (Accurate Analysis)
% To be included in master LaTeX document
%
% Usage: % Lean Bulk Data Manager Module API Documentation (Accurate Analysis)
% To be included in master LaTeX document
%
% Usage: \input{docs/lean_bulk_data_manager_api}

\section{Lean Bulk Data Manager Module API Reference}
\label{sec:lean_bulk_data_manager_api}

This section provides an exact reference for the BulkDataManager class (\texttt{ooc1d.core.lean\_bulk\_data\_manager.BulkDataManager}) based on detailed analysis of the actual implementation. This is an ultra-lean coordinator that minimizes memory usage by storing only essential domain data and accepting framework objects as method parameters.

\subsection{Module Overview}

The lean bulk data manager provides memory-efficient coordination for bulk operations by:
\begin{itemize}
    \item Storing only extracted essential domain data
    \item Accepting framework objects as method parameters
    \item Validating framework object compatibility
    \item Providing flexible bulk data operations without memory overhead
\end{itemize}

\subsection{Module Imports and Dependencies}

\begin{lstlisting}[language=Python, caption=Module Dependencies]
import numpy as np
from typing import List, Optional, Callable
from ooc1d.core.bulk_data import BulkData
from ooc1d.core.domain_data import DomainData
\end{lstlisting}

\subsection{BulkDataManager Class Definition}
\label{subsec:lean_bulk_data_manager_class}

\begin{lstlisting}[language=Python, caption=Class Declaration]
class BulkDataManager:
    """
    Ultra-lean coordinator for bulk operations in HDG method.
    
    This class stores only essential extracted domain data and accepts
    framework objects as parameters to methods that need them. This approach
    minimizes memory usage and increases flexibility.
    """
\end{lstlisting}

\subsection{Constructor}
\label{subsec:lean_constructor}

\paragraph{\_\_init\_\_()}\leavevmode
\begin{lstlisting}[language=Python, caption=Lean BulkDataManager Constructor]
def __init__(self, domain_data_list: List[DomainData])
\end{lstlisting}

\textbf{Parameters:}
\begin{itemize}
    \item \texttt{domain\_data\_list}: List of DomainData objects with essential extracted information
\end{itemize}

\textbf{Side Effects:} Sets \texttt{self.domain\_data\_list} attribute

\textbf{Usage:}
\begin{lstlisting}[language=Python, caption=Constructor Usage]
# Extract domain data first (see static factory method)
domain_data_list = BulkDataManager.extract_domain_data_list(
    problems, discretizations, static_condensations
)

# Create lean manager with extracted data only
lean_manager = BulkDataManager(domain_data_list)
\end{lstlisting}

\subsection{Core Attributes}
\label{subsec:lean_attributes}

\begin{longtable}{|p{3.5cm}|p{2.5cm}|p{7cm}|}
\hline
\textbf{Attribute} & \textbf{Type} & \textbf{Description} \\
\hline
\endhead

\texttt{domain\_data\_list} & \texttt{List[DomainData]} & List of essential domain data objects (only stored attribute) \\
\hline

\end{longtable}

\textbf{Memory Efficiency:} The lean manager stores \textbf{only} the domain data list, avoiding storage of large framework objects.

\subsection{Validation Methods}
\label{subsec:validation_methods}

\paragraph{\_validate\_framework\_objects()}\leavevmode
\begin{lstlisting}[language=Python, caption=Framework Validation Method]
def _validate_framework_objects(self, 
                               problems: List = None,
                               discretizations: List = None, 
                               static_condensations: List = None,
                               operation_name: str = "operation") -> None
\end{lstlisting}

\textbf{Parameters:}
\begin{itemize}
    \item \texttt{problems}: List of Problem objects to validate (optional)
    \item \texttt{discretizations}: List of discretization objects to validate (optional)
    \item \texttt{static\_condensations}: List of static condensation objects to validate (optional)
    \item \texttt{operation\_name}: Name of operation for error messages (default: "operation")
\end{itemize}

\textbf{Returns:} \texttt{None}

\textbf{Raises:} \texttt{ValueError} for incompatible framework objects

\textbf{Validation Checks:}
\begin{enumerate}
    \item \textbf{List Length Validation}: All provided lists must match domain count
    \item \textbf{Problem Validation}: \texttt{neq} attribute must match stored domain data
    \item \textbf{Discretization Validation}: \texttt{n\_elements}, nodes, and \texttt{element\_length} compatibility
    \item \textbf{Static Condensation Validation}: Matrix compatibility and method availability
\end{enumerate}

\textbf{Usage:}
\begin{lstlisting}[language=Python, caption=Validation Usage]
try:
    lean_manager._validate_framework_objects(
        problems=problems,
        discretizations=discretizations,
        operation_name="my_operation"
    )
    print("✓ Framework objects are compatible")
except ValueError as e:
    print(f"✗ Validation failed: {e}")
\end{lstlisting}

\subsection{Static Factory Methods}
\label{subsec:static_factory_methods}

\paragraph{extract\_domain\_data\_list()}\leavevmode
\begin{lstlisting}[language=Python, caption=Domain Data Extraction Method]
@staticmethod
def extract_domain_data_list(problems: List, 
                            discretizations: List, 
                            static_condensations: List) -> List[DomainData]
\end{lstlisting}

\textbf{Parameters:}
\begin{itemize}
    \item \texttt{problems}: List of Problem instances
    \item \texttt{discretizations}: List of discretization instances
    \item \texttt{static\_condensations}: List of static condensation instances
\end{itemize}

\textbf{Returns:} \texttt{List[DomainData]} - Extracted essential domain information

\textbf{Purpose:} Static factory method to extract and store essential data once for reuse

\textbf{Usage:}
\begin{lstlisting}[language=Python, caption=Domain Data Extraction Usage]
# One-time extraction of essential domain data
domain_data_list = BulkDataManager.extract_domain_data_list(
    problems=problems,
    discretizations=discretizations,
    static_condensations=static_condensations
)

# Can create multiple lean managers with same extracted data
lean_manager_1 = BulkDataManager(domain_data_list)
lean_manager_2 = BulkDataManager(domain_data_list)
\end{lstlisting}

\paragraph{\_extract\_single\_domain\_data()}\leavevmode
\begin{lstlisting}[language=Python, caption=Single Domain Extraction Method]
@staticmethod
def _extract_single_domain_data(problem, discretization, sc, domain_idx: int) -> DomainData
\end{lstlisting}

\textbf{Parameters:}
\begin{itemize}
    \item \texttt{problem}: Problem object for single domain
    \item \texttt{discretization}: Discretization object for single domain
    \item \texttt{sc}: Static condensation object for single domain
    \item \texttt{domain\_idx}: Domain index for error reporting
\end{itemize}

\textbf{Returns:} \texttt{DomainData} - Extracted essential data for single domain

\textbf{Extraction Process:}
\begin{enumerate}
    \item Extract matrices from static condensation (\texttt{M}, \texttt{T})
    \item Extract initial conditions with multiple access pattern support
    \item Extract forcing functions with multiple access pattern support
    \item Create DomainData object with essential information
\end{enumerate}

\subsection{BulkData Creation Methods}
\label{subsec_bulk_data_creation}

\paragraph{create\_bulk\_data()}\leavevmode
\begin{lstlisting}[language=Python, caption=Create BulkData Method]
def create_bulk_data(self, 
                    domain_index: int, 
                    problem, 
                    discretization, 
                    dual: bool = False) -> BulkData
\end{lstlisting}

\textbf{Parameters:}
\begin{itemize}
    \item \texttt{domain\_index}: Index of the domain (0 to \texttt{n\_domains-1})
    \item \texttt{problem}: Problem object for this domain
    \item \texttt{discretization}: Discretization object for this domain
    \item \texttt{dual}: Whether to use dual formulation (default: False)
\end{itemize}

\textbf{Returns:} \texttt{BulkData} - New BulkData object

\textbf{Raises:} \texttt{ValueError} for invalid domain index

\textbf{Usage:}
\begin{lstlisting}[language=Python, caption=BulkData Creation Usage]
# Create primal BulkData for domain 0
bulk_data_primal = lean_manager.create_bulk_data(
    domain_index=0,
    problem=problems[0],
    discretization=discretizations[0],
    dual=False
)

# Create dual BulkData for forcing terms
bulk_data_dual = lean_manager.create_bulk_data(
    domain_index=0,
    problem=problems[0],
    discretization=discretizations[0],
    dual=True
)
\end{lstlisting}

\subsection{Bulk Operations Methods}
\label{subsec:bulk_operations}

\paragraph{compute\_source\_terms()}\leavevmode
\begin{lstlisting}[language=Python, caption=Compute Source Terms Method]
def compute_source_terms(self,
                        problems: List,
                        discretizations: List,
                        time: float) -> List[BulkData]
\end{lstlisting}

\textbf{Parameters:}
\begin{itemize}
    \item \texttt{problems}: List of Problem objects
    \item \texttt{discretizations}: List of discretization objects
    \item \texttt{time}: Current time for evaluation
\end{itemize}

\textbf{Returns:} \texttt{List[BulkData]} - Source terms for all domains using dual formulation

\textbf{Process:}
\begin{enumerate}
    \item Validate framework objects against stored domain data
    \item Create dual BulkData for each domain
    \item Set forcing functions from problems at specified time
    \item Return list of integrated source terms
\end{enumerate}

\textbf{Usage:}
\begin{lstlisting}[language=Python, caption=Source Terms Usage]
# Compute source terms at time t=0.5
source_terms = lean_manager.compute_source_terms(
    problems=problems,
    discretizations=discretizations,
    time=0.5
)

print(f"Computed {len(source_terms)} source term BulkData objects")
\end{lstlisting}

\paragraph{compute\_forcing\_terms()}\leavevmode
\begin{lstlisting}[language=Python, caption=Compute Forcing Terms Method]
def compute_forcing_terms(self, 
                          bulk_data_list: List[BulkData],
                          problems: List,
                          discretizations: List, 
                          time: float, 
                          dt: float) -> List[np.ndarray]
\end{lstlisting}

\textbf{Parameters:}
\begin{itemize}
    \item \texttt{bulk\_data\_list}: List of current BulkData solutions
    \item \texttt{problems}: List of Problem objects
    \item \texttt{discretizations}: List of discretization objects
    \item \texttt{time}: Current time
    \item \texttt{dt}: Time step size
\end{itemize}

\textbf{Returns:} \texttt{List[np.ndarray]} - Forcing term arrays for implicit Euler

\textbf{Computation:} For each domain: \texttt{forcing\_term = dt * force\_contrib + M * U\_old}

\textbf{Usage:}
\begin{lstlisting}[language=Python, caption=Forcing Terms Usage]
# Compute forcing terms for implicit Euler step
current_bulk_data = [...]  # Current solution
forcing_terms = lean_manager.compute_forcing_terms(
    bulk_data_list=current_bulk_data,
    problems=problems,
    discretizations=discretizations,
    time=0.1,
    dt=0.01
)

# forcing_terms[i] has shape (2*neq, n_elements) for domain i
\end{lstlisting}

\subsection{Initialization Methods}
\label{subsec:initialization_methods}

\paragraph{initialize\_all\_bulk\_data()}\leavevmode
\begin{lstlisting}[language=Python, caption=Initialize All BulkData Method]
def initialize_all_bulk_data(self, 
                            problems: List,
                            discretizations: List,
                            time: float = 0.0) -> List[BulkData]
\end{lstlisting}

\textbf{Parameters:}
\begin{itemize}
    \item \texttt{problems}: List of Problem objects
    \item \texttt{discretizations}: List of discretization objects
    \item \texttt{time}: Initial time (default: 0.0)
\end{itemize}

\textbf{Returns:} \texttt{List[BulkData]} - Initialized BulkData objects for all domains

\textbf{Usage:}
\begin{lstlisting}[language=Python, caption=Initialize All Usage]
# Initialize all domains with initial conditions
initial_bulk_data = lean_manager.initialize_all_bulk_data(
    problems=problems,
    discretizations=discretizations,
    time=0.0
)

print(f"Initialized {len(initial_bulk_data)} BulkData objects")
\end{lstlisting}

\paragraph{initialize\_bulk\_data\_from\_initial\_conditions()}\leavevmode
\begin{lstlisting}[language=Python, caption=Initialize Single Domain Method]
def initialize_bulk_data_from_initial_conditions(self, 
                                                domain_index: int,
                                                problem,
                                                discretization,
                                                time: float = 0.0) -> BulkData
\end{lstlisting}

\textbf{Parameters:}
\begin{itemize}
    \item \texttt{domain\_index}: Index of domain to initialize
    \item \texttt{problem}: Problem object for this domain
    \item \texttt{discretization}: Discretization object for this domain
    \item \texttt{time}: Initial time (default: 0.0)
\end{itemize}

\textbf{Returns:} \texttt{BulkData} - Initialized BulkData object

\textbf{Process:}
\begin{enumerate}
    \item Validate domain index and framework objects
    \item Create BulkData with primal formulation
    \item Set initial conditions from stored domain data
    \item Default to zero if no initial conditions available
\end{enumerate}

\textbf{Usage:}
\begin{lstlisting}[language=Python, caption=Initialize Single Domain Usage]
# Initialize specific domain
domain_0_bulk = lean_manager.initialize_bulk_data_from_initial_conditions(
    domain_index=0,
    problem=problems[0],
    discretization=discretizations[0],
    time=0.0
)
\end{lstlisting}

\subsection{Data Management Methods}
\label{subsec:data_management}

\paragraph{update\_bulk\_data()}\leavevmode
\begin{lstlisting}[language=Python, caption=Update BulkData Method]
def update_bulk_data(self, bulk_data_list: List[BulkData], new_data_list: List[np.ndarray])
\end{lstlisting}

\textbf{Parameters:}
\begin{itemize}
    \item \texttt{bulk\_data\_list}: List of BulkData objects to update
    \item \texttt{new\_data\_list}: List of new bulk solution arrays
\end{itemize}

\textbf{Returns:} \texttt{None}

\textbf{Side Effects:} Updates data in all BulkData objects

\textbf{Validation:}
\begin{itemize}
    \item Lists must have matching lengths
    \item New data must have compatible shapes
    \item New data must not contain NaN or infinite values
\end{itemize}

\textbf{Usage:}
\begin{lstlisting}[language=Python, caption=Update BulkData Usage]
# Update bulk data with new solution
new_solutions = [...]  # List of numpy arrays
lean_manager.update_bulk_data(
    bulk_data_list=current_bulk_data,
    new_data_list=new_solutions
)
\end{lstlisting}

\paragraph{get\_bulk\_data\_arrays()}\leavevmode
\begin{lstlisting}[language=Python, caption=Get Data Arrays Method]
def get_bulk_data_arrays(self, bulk_data_list: List[BulkData]) -> List[np.ndarray]
\end{lstlisting}

\textbf{Parameters:}
\begin{itemize}
    \item \texttt{bulk\_data\_list}: List of BulkData objects
\end{itemize}

\textbf{Returns:} \texttt{List[np.ndarray]} - Data arrays from all BulkData objects

\textbf{Usage:}
\begin{lstlisting}[language=Python, caption=Get Arrays Usage]
# Extract data arrays for external processing
data_arrays = lean_manager.get_bulk_data_arrays(bulk_data_list)
for i, array in enumerate(data_arrays):
    print(f"Domain {i} data shape: {array.shape}")
\end{lstlisting}

\subsection{Analysis Methods}
\label{subsec:analysis_methods}

\paragraph{compute\_total\_mass()}\leavevmode
\begin{lstlisting}[language=Python, caption=Compute Total Mass Method]
def compute_total_mass(self, bulk_data_list: List[BulkData]) -> float
\end{lstlisting}

\textbf{Parameters:}
\begin{itemize}
    \item \texttt{bulk\_data\_list}: List of BulkData instances
\end{itemize}

\textbf{Returns:} \texttt{float} - Total mass across all domains

\textbf{Computation:} Sums mass from each domain using stored mass matrices

\textbf{Usage:}
\begin{lstlisting}[language=Python, caption=Mass Computation Usage]
# Monitor mass conservation
initial_mass = lean_manager.compute_total_mass(initial_bulk_data)
current_mass = lean_manager.compute_total_mass(current_bulk_data)

mass_change = abs(current_mass - initial_mass) / initial_mass
print(f"Relative mass change: {mass_change:.6e}")
\end{lstlisting}

\paragraph{compute\_mass\_conservation()}\leavevmode
\begin{lstlisting}[language=Python, caption=Mass Conservation Method]
def compute_mass_conservation(self, bulk_data_list: List[BulkData]) -> float
\end{lstlisting}

\textbf{Parameters:}
\begin{itemize}
    \item \texttt{bulk\_data\_list}: List of BulkData instances
\end{itemize}

\textbf{Returns:} \texttt{float} - Total mass (alias for \texttt{compute\_total\_mass})

\textbf{Note:} This method is an alias for consistency with other interfaces

\subsection{Utility Methods}
\label{subsec:utility_methods}

\paragraph{get\_num\_domains()}\leavevmode
\begin{lstlisting}[language=Python, caption=Get Number of Domains Method]
def get_num_domains(self) -> int
\end{lstlisting}

\textbf{Returns:} \texttt{int} - Number of domains managed

\textbf{Usage:}
\begin{lstlisting}[language=Python, caption=Get Domains Count Usage]
n_domains = lean_manager.get_num_domains()
print(f"Managing {n_domains} domains")
\end{lstlisting}

\paragraph{get\_domain\_info()}\leavevmode
\begin{lstlisting}[language=Python, caption=Get Domain Info Method]
def get_domain_info(self, domain_idx: int) -> DomainData
\end{lstlisting}

\textbf{Parameters:}
\begin{itemize}
    \item \texttt{domain\_idx}: Domain index
\end{itemize}

\textbf{Returns:} \texttt{DomainData} - Domain data object for inspection

\textbf{Raises:} \texttt{IndexError} for invalid domain index

\textbf{Usage:}
\begin{lstlisting}[language=Python, caption=Domain Info Usage]
# Inspect domain properties
domain_info = lean_manager.get_domain_info(0)
print(f"Domain 0: {domain_info.neq} equations, {domain_info.n_elements} elements")
print(f"Element length: {domain_info.element_length}")
\end{lstlisting}

\subsection{Testing and Validation}
\label{subsec:testing_validation}

\paragraph{test()}\leavevmode
\begin{lstlisting}[language=Python, caption=Test Method]
def test(self, 
         problems: List = None,
         discretizations: List = None,
         static_condensations: List = None) -> bool
\end{lstlisting}

\textbf{Parameters:}
\begin{itemize}
    \item \texttt{problems}: List of Problem objects for testing (optional)
    \item \texttt{discretizations}: List of discretization objects for testing (optional)
    \item \texttt{static\_condensations}: List of static condensation objects for testing (optional)
\end{itemize}

\textbf{Returns:} \texttt{bool} - True if all tests pass, False otherwise

\textbf{Test Suite:}
\begin{enumerate}
    \item \textbf{Framework Object Validation Test}: Validates provided framework objects
    \item \textbf{Domain Data Structure Test}: Validates stored domain data integrity
    \item \textbf{BulkData Creation Test}: Tests creation of primal and dual BulkData
    \item \textbf{Initialization Test}: Tests bulk data initialization
    \item \textbf{Forcing Term Computation Test}: Tests forcing term calculations
    \item \textbf{Mass Computation Test}: Tests mass conservation calculations
    \item \textbf{Bounds Checking Test}: Tests error handling for invalid indices
    \item \textbf{Utility Methods Test}: Tests helper methods
    \item \textbf{Parameter Mismatch Test}: Tests validation error detection
\end{enumerate}

\textbf{Usage:}
\begin{lstlisting}[language=Python, caption=Test Method Usage]
# Comprehensive testing with framework objects
if lean_manager.test(
    problems=problems,
    discretizations=discretizations,
    static_condensations=static_condensations
):
    print("✓ Lean BulkDataManager is fully functional")
else:
    print("✗ Issues detected in Lean BulkDataManager")

# Minimal testing without framework objects
if lean_manager.test():
    print("✓ Basic structure validation passed")
\end{lstlisting}

\textbf{Sample Test Output:}
\begin{lstlisting}[language=Python, caption=Sample Test Output]
Testing Lean BulkDataManager with 3 domains
PASS: Framework object validation passed
PASS: All domain data validated
PASS: BulkData creation tests passed
PASS: Initialization tests passed
PASS: Forcing term computation tests passed
PASS: Mass computation test passed (total_mass=1.234567e+00)
PASS: ValueError raised for negative domain index
PASS: get_num_domains() returned correct value
PASS: get_domain_info() test passed
PASS: Correctly detected wrong number of problems
PASS: Correctly detected incompatible problem neq
PASS: Parameter mismatch detection tests passed
✓ All Lean BulkDataManager tests passed!
\end{lstlisting}

\subsection{Special Methods}
\label{subsec:special_methods}

\paragraph{\_\_str\_\_()}\leavevmode
\begin{lstlisting}[language=Python, caption=String Representation Method]
def __str__(self) -> str
\end{lstlisting}

\textbf{Returns:} \texttt{str} - Human-readable summary

\textbf{Format:} \texttt{"LeanBulkDataManager(domains=N, total\_elements=M, total\_equations=K)"}

\paragraph{\_\_repr\_\_()}\leavevmode
\begin{lstlisting}[language=Python, caption=Repr Method]
def __repr__(self) -> str
\end{lstlisting}

\textbf{Returns:} \texttt{str} - Developer-oriented representation

\textbf{Format:} \texttt{"LeanBulkDataManager(n\_domains=N, domain\_elements=[...], domain\_equations=[...])"}

\textbf{Usage:}
\begin{lstlisting}[language=Python, caption=String Methods Usage]
print(str(lean_manager))
# Output: LeanBulkDataManager(domains=3, total_elements=60, total_equations=6)

print(repr(lean_manager))
# Output: LeanBulkDataManager(n_domains=3, domain_elements=[20, 20, 20], domain_equations=[2, 2, 2])
\end{lstlisting}

\subsection{Complete Usage Examples}
\label{subsec:complete_usage_examples}

\subsubsection{Standard Workflow Example}

\begin{lstlisting}[language=Python, caption=Complete Lean Manager Workflow]
from ooc1d.core.lean_bulk_data_manager import BulkDataManager
from ooc1d.core.problem import Problem
from ooc1d.core.discretization import Discretization
import numpy as np

# Step 1: Create framework objects (problems, discretizations, static_condensations)
problems = [...]  # List of Problem instances
discretizations = [...]  # List of Discretization instances  
static_condensations = [...]  # List of static condensation instances

# Step 2: Extract essential data once (memory-efficient)
domain_data_list = BulkDataManager.extract_domain_data_list(
    problems=problems,
    discretizations=discretizations,
    static_condensations=static_condensations
)

# Step 3: Create lean manager with extracted data only
lean_manager = BulkDataManager(domain_data_list)

# Step 4: Validate compatibility (optional but recommended)
if not lean_manager.test(problems, discretizations, static_condensations):
    raise RuntimeError("Framework objects incompatible with extracted data")

# Step 5: Initialize bulk data for all domains
bulk_data_list = lean_manager.initialize_all_bulk_data(
    problems=problems,
    discretizations=discretizations,
    time=0.0
)

# Step 6: Time evolution loop
dt = 0.01
for time_step in range(100):
    current_time = time_step * dt
    
    # Compute forcing terms for implicit Euler
    forcing_terms = lean_manager.compute_forcing_terms(
        bulk_data_list=bulk_data_list,
        problems=problems,
        discretizations=discretizations,
        time=current_time,
        dt=dt
    )
    
    # Solve system (external solver)
    new_solutions = solve_system(forcing_terms)  # User-defined solver
    
    # Update bulk data with new solutions
    lean_manager.update_bulk_data(bulk_data_list, new_solutions)
    
    # Monitor mass conservation
    current_mass = lean_manager.compute_total_mass(bulk_data_list)
    if time_step % 10 == 0:
        print(f"Time {current_time:.3f}: Mass = {current_mass:.6e}")

print("✓ Time evolution completed with lean manager")
\end{lstlisting}

\subsubsection{Memory Comparison Example}

\begin{lstlisting}[language=Python, caption=Memory Usage Comparison]
import psutil
import os

# Measure memory before
process = psutil.Process(os.getpid())
memory_before = process.memory_info().rss / 1024 / 1024  # MB

# Traditional approach (stores framework objects)
# traditional_manager = FullBulkDataManager(problems, discretizations, static_condensations)

# Lean approach (stores only essential data)
domain_data_list = BulkDataManager.extract_domain_data_list(
    problems, discretizations, static_condensations
)
lean_manager = BulkDataManager(domain_data_list)

# Measure memory after
memory_after = process.memory_info().rss / 1024 / 1024  # MB
memory_used = memory_after - memory_before

print(f"Memory used by lean manager: {memory_used:.2f} MB")
print(f"Domains managed: {lean_manager.get_num_domains()}")
print(f"Memory per domain: {memory_used / lean_manager.get_num_domains():.2f} MB")
\end{lstlisting}

\subsubsection{Multi-Manager Example}

\begin{lstlisting}[language=Python, caption=Multiple Lean Managers from Same Data]
# Extract domain data once
domain_data_list = BulkDataManager.extract_domain_data_list(
    problems, discretizations, static_condensations
)

# Create multiple lean managers for different purposes
# (all sharing the same extracted data - no additional memory cost)

# Manager for time evolution
evolution_manager = BulkDataManager(domain_data_list)

# Manager for forcing term computation
forcing_manager = BulkDataManager(domain_data_list)

# Manager for mass conservation tracking  
conservation_manager = BulkDataManager(domain_data_list)

# Each manager can operate independently but uses same base data
initial_bulk = evolution_manager.initialize_all_bulk_data(problems, discretizations)
source_terms = forcing_manager.compute_source_terms(problems, discretizations, time=0.0)
total_mass = conservation_manager.compute_total_mass(initial_bulk)

print(f"Created 3 independent managers sharing {len(domain_data_list)} domain data objects")
\end{lstlisting}

\subsection{Method Summary Table}
\label{subsec:lean_method_summary}

\begin{longtable}{|p{5.3cm}|p{3.2cm}|p{5cm}|}
\hline
\textbf{Method} & \textbf{Returns} & \textbf{Purpose} \\
\hline
\endhead

\texttt{\_\_init\_\_} & \texttt{None} & Initialize with extracted domain data only \\
\hline

\texttt{extract\_domain\_data\_list} & \texttt{List[DomainData]} & Static factory for one-time data extraction \\
\hline

\texttt{create\_bulk\_data} & \texttt{BulkData} & Create BulkData using external framework objects \\
\hline

\texttt{initialize\_all\_bulk\_data} & \texttt{List[BulkData]} & Initialize all domains with initial conditions \\
\hline

\texttt{compute\_source\_terms} & \texttt{List[BulkData]} & Compute source terms using dual formulation \\
\hline

\texttt{compute\_forcing\_terms} & \texttt{List[np.ndarray]} & Compute forcing terms for implicit Euler \\
\hline

\texttt{update\_bulk\_data} & \texttt{None} & Update BulkData objects with new solutions \\
\hline

\texttt{compute\_total\_mass} & \texttt{float} & Calculate total mass for conservation \\
\hline

\texttt{get\_bulk\_data\_arrays} & \texttt{List[np.ndarray]} & Extract data arrays from BulkData objects \\
\hline

\texttt{get\_num\_domains} & \texttt{int} & Get number of managed domains \\
\hline

\texttt{get\_domain\_info} & \texttt{DomainData} & Access domain data for inspection \\
\hline

\texttt{test} & \texttt{bool} & Comprehensive validation and testing \\
\hline

\texttt{\_validate\_framework\_objects} & \texttt{None} & Validate framework object compatibility \\
\hline

\end{longtable}

This documentation provides an exact reference for the lean BulkDataManager class, emphasizing its memory-efficient design and parameter-based approach to framework object usage. The lean architecture minimizes memory overhead while maintaining full functionality through external object validation and flexible method interfaces.

% End of lean bulk data manager module API documentation


\section{Lean Bulk Data Manager Module API Reference}
\label{sec:lean_bulk_data_manager_api}

This section provides an exact reference for the BulkDataManager class (\texttt{ooc1d.core.lean\_bulk\_data\_manager.BulkDataManager}) based on detailed analysis of the actual implementation. This is an ultra-lean coordinator that minimizes memory usage by storing only essential domain data and accepting framework objects as method parameters.

\subsection{Module Overview}

The lean bulk data manager provides memory-efficient coordination for bulk operations by:
\begin{itemize}
    \item Storing only extracted essential domain data
    \item Accepting framework objects as method parameters
    \item Validating framework object compatibility
    \item Providing flexible bulk data operations without memory overhead
\end{itemize}

\subsection{Module Imports and Dependencies}

\begin{lstlisting}[language=Python, caption=Module Dependencies]
import numpy as np
from typing import List, Optional, Callable
from ooc1d.core.bulk_data import BulkData
from ooc1d.core.domain_data import DomainData
\end{lstlisting}

\subsection{BulkDataManager Class Definition}
\label{subsec:lean_bulk_data_manager_class}

\begin{lstlisting}[language=Python, caption=Class Declaration]
class BulkDataManager:
    """
    Ultra-lean coordinator for bulk operations in HDG method.
    
    This class stores only essential extracted domain data and accepts
    framework objects as parameters to methods that need them. This approach
    minimizes memory usage and increases flexibility.
    """
\end{lstlisting}

\subsection{Constructor}
\label{subsec:lean_constructor}

\paragraph{\_\_init\_\_()}\leavevmode
\begin{lstlisting}[language=Python, caption=Lean BulkDataManager Constructor]
def __init__(self, domain_data_list: List[DomainData])
\end{lstlisting}

\textbf{Parameters:}
\begin{itemize}
    \item \texttt{domain\_data\_list}: List of DomainData objects with essential extracted information
\end{itemize}

\textbf{Side Effects:} Sets \texttt{self.domain\_data\_list} attribute

\textbf{Usage:}
\begin{lstlisting}[language=Python, caption=Constructor Usage]
# Extract domain data first (see static factory method)
domain_data_list = BulkDataManager.extract_domain_data_list(
    problems, discretizations, static_condensations
)

# Create lean manager with extracted data only
lean_manager = BulkDataManager(domain_data_list)
\end{lstlisting}

\subsection{Core Attributes}
\label{subsec:lean_attributes}

\begin{longtable}{|p{3.5cm}|p{2.5cm}|p{7cm}|}
\hline
\textbf{Attribute} & \textbf{Type} & \textbf{Description} \\
\hline
\endhead

\texttt{domain\_data\_list} & \texttt{List[DomainData]} & List of essential domain data objects (only stored attribute) \\
\hline

\end{longtable}

\textbf{Memory Efficiency:} The lean manager stores \textbf{only} the domain data list, avoiding storage of large framework objects.

\subsection{Validation Methods}
\label{subsec:validation_methods}

\paragraph{\_validate\_framework\_objects()}\leavevmode
\begin{lstlisting}[language=Python, caption=Framework Validation Method]
def _validate_framework_objects(self, 
                               problems: List = None,
                               discretizations: List = None, 
                               static_condensations: List = None,
                               operation_name: str = "operation") -> None
\end{lstlisting}

\textbf{Parameters:}
\begin{itemize}
    \item \texttt{problems}: List of Problem objects to validate (optional)
    \item \texttt{discretizations}: List of discretization objects to validate (optional)
    \item \texttt{static\_condensations}: List of static condensation objects to validate (optional)
    \item \texttt{operation\_name}: Name of operation for error messages (default: "operation")
\end{itemize}

\textbf{Returns:} \texttt{None}

\textbf{Raises:} \texttt{ValueError} for incompatible framework objects

\textbf{Validation Checks:}
\begin{enumerate}
    \item \textbf{List Length Validation}: All provided lists must match domain count
    \item \textbf{Problem Validation}: \texttt{neq} attribute must match stored domain data
    \item \textbf{Discretization Validation}: \texttt{n\_elements}, nodes, and \texttt{element\_length} compatibility
    \item \textbf{Static Condensation Validation}: Matrix compatibility and method availability
\end{enumerate}

\textbf{Usage:}
\begin{lstlisting}[language=Python, caption=Validation Usage]
try:
    lean_manager._validate_framework_objects(
        problems=problems,
        discretizations=discretizations,
        operation_name="my_operation"
    )
    print("✓ Framework objects are compatible")
except ValueError as e:
    print(f"✗ Validation failed: {e}")
\end{lstlisting}

\subsection{Static Factory Methods}
\label{subsec:static_factory_methods}

\paragraph{extract\_domain\_data\_list()}\leavevmode
\begin{lstlisting}[language=Python, caption=Domain Data Extraction Method]
@staticmethod
def extract_domain_data_list(problems: List, 
                            discretizations: List, 
                            static_condensations: List) -> List[DomainData]
\end{lstlisting}

\textbf{Parameters:}
\begin{itemize}
    \item \texttt{problems}: List of Problem instances
    \item \texttt{discretizations}: List of discretization instances
    \item \texttt{static\_condensations}: List of static condensation instances
\end{itemize}

\textbf{Returns:} \texttt{List[DomainData]} - Extracted essential domain information

\textbf{Purpose:} Static factory method to extract and store essential data once for reuse

\textbf{Usage:}
\begin{lstlisting}[language=Python, caption=Domain Data Extraction Usage]
# One-time extraction of essential domain data
domain_data_list = BulkDataManager.extract_domain_data_list(
    problems=problems,
    discretizations=discretizations,
    static_condensations=static_condensations
)

# Can create multiple lean managers with same extracted data
lean_manager_1 = BulkDataManager(domain_data_list)
lean_manager_2 = BulkDataManager(domain_data_list)
\end{lstlisting}

\paragraph{\_extract\_single\_domain\_data()}\leavevmode
\begin{lstlisting}[language=Python, caption=Single Domain Extraction Method]
@staticmethod
def _extract_single_domain_data(problem, discretization, sc, domain_idx: int) -> DomainData
\end{lstlisting}

\textbf{Parameters:}
\begin{itemize}
    \item \texttt{problem}: Problem object for single domain
    \item \texttt{discretization}: Discretization object for single domain
    \item \texttt{sc}: Static condensation object for single domain
    \item \texttt{domain\_idx}: Domain index for error reporting
\end{itemize}

\textbf{Returns:} \texttt{DomainData} - Extracted essential data for single domain

\textbf{Extraction Process:}
\begin{enumerate}
    \item Extract matrices from static condensation (\texttt{M}, \texttt{T})
    \item Extract initial conditions with multiple access pattern support
    \item Extract forcing functions with multiple access pattern support
    \item Create DomainData object with essential information
\end{enumerate}

\subsection{BulkData Creation Methods}
\label{subsec_bulk_data_creation}

\paragraph{create\_bulk\_data()}\leavevmode
\begin{lstlisting}[language=Python, caption=Create BulkData Method]
def create_bulk_data(self, 
                    domain_index: int, 
                    problem, 
                    discretization, 
                    dual: bool = False) -> BulkData
\end{lstlisting}

\textbf{Parameters:}
\begin{itemize}
    \item \texttt{domain\_index}: Index of the domain (0 to \texttt{n\_domains-1})
    \item \texttt{problem}: Problem object for this domain
    \item \texttt{discretization}: Discretization object for this domain
    \item \texttt{dual}: Whether to use dual formulation (default: False)
\end{itemize}

\textbf{Returns:} \texttt{BulkData} - New BulkData object

\textbf{Raises:} \texttt{ValueError} for invalid domain index

\textbf{Usage:}
\begin{lstlisting}[language=Python, caption=BulkData Creation Usage]
# Create primal BulkData for domain 0
bulk_data_primal = lean_manager.create_bulk_data(
    domain_index=0,
    problem=problems[0],
    discretization=discretizations[0],
    dual=False
)

# Create dual BulkData for forcing terms
bulk_data_dual = lean_manager.create_bulk_data(
    domain_index=0,
    problem=problems[0],
    discretization=discretizations[0],
    dual=True
)
\end{lstlisting}

\subsection{Bulk Operations Methods}
\label{subsec:bulk_operations}

\paragraph{compute\_source\_terms()}\leavevmode
\begin{lstlisting}[language=Python, caption=Compute Source Terms Method]
def compute_source_terms(self,
                        problems: List,
                        discretizations: List,
                        time: float) -> List[BulkData]
\end{lstlisting}

\textbf{Parameters:}
\begin{itemize}
    \item \texttt{problems}: List of Problem objects
    \item \texttt{discretizations}: List of discretization objects
    \item \texttt{time}: Current time for evaluation
\end{itemize}

\textbf{Returns:} \texttt{List[BulkData]} - Source terms for all domains using dual formulation

\textbf{Process:}
\begin{enumerate}
    \item Validate framework objects against stored domain data
    \item Create dual BulkData for each domain
    \item Set forcing functions from problems at specified time
    \item Return list of integrated source terms
\end{enumerate}

\textbf{Usage:}
\begin{lstlisting}[language=Python, caption=Source Terms Usage]
# Compute source terms at time t=0.5
source_terms = lean_manager.compute_source_terms(
    problems=problems,
    discretizations=discretizations,
    time=0.5
)

print(f"Computed {len(source_terms)} source term BulkData objects")
\end{lstlisting}

\paragraph{compute\_forcing\_terms()}\leavevmode
\begin{lstlisting}[language=Python, caption=Compute Forcing Terms Method]
def compute_forcing_terms(self, 
                          bulk_data_list: List[BulkData],
                          problems: List,
                          discretizations: List, 
                          time: float, 
                          dt: float) -> List[np.ndarray]
\end{lstlisting}

\textbf{Parameters:}
\begin{itemize}
    \item \texttt{bulk\_data\_list}: List of current BulkData solutions
    \item \texttt{problems}: List of Problem objects
    \item \texttt{discretizations}: List of discretization objects
    \item \texttt{time}: Current time
    \item \texttt{dt}: Time step size
\end{itemize}

\textbf{Returns:} \texttt{List[np.ndarray]} - Forcing term arrays for implicit Euler

\textbf{Computation:} For each domain: \texttt{forcing\_term = dt * force\_contrib + M * U\_old}

\textbf{Usage:}
\begin{lstlisting}[language=Python, caption=Forcing Terms Usage]
# Compute forcing terms for implicit Euler step
current_bulk_data = [...]  # Current solution
forcing_terms = lean_manager.compute_forcing_terms(
    bulk_data_list=current_bulk_data,
    problems=problems,
    discretizations=discretizations,
    time=0.1,
    dt=0.01
)

# forcing_terms[i] has shape (2*neq, n_elements) for domain i
\end{lstlisting}

\subsection{Initialization Methods}
\label{subsec:initialization_methods}

\paragraph{initialize\_all\_bulk\_data()}\leavevmode
\begin{lstlisting}[language=Python, caption=Initialize All BulkData Method]
def initialize_all_bulk_data(self, 
                            problems: List,
                            discretizations: List,
                            time: float = 0.0) -> List[BulkData]
\end{lstlisting}

\textbf{Parameters:}
\begin{itemize}
    \item \texttt{problems}: List of Problem objects
    \item \texttt{discretizations}: List of discretization objects
    \item \texttt{time}: Initial time (default: 0.0)
\end{itemize}

\textbf{Returns:} \texttt{List[BulkData]} - Initialized BulkData objects for all domains

\textbf{Usage:}
\begin{lstlisting}[language=Python, caption=Initialize All Usage]
# Initialize all domains with initial conditions
initial_bulk_data = lean_manager.initialize_all_bulk_data(
    problems=problems,
    discretizations=discretizations,
    time=0.0
)

print(f"Initialized {len(initial_bulk_data)} BulkData objects")
\end{lstlisting}

\paragraph{initialize\_bulk\_data\_from\_initial\_conditions()}\leavevmode
\begin{lstlisting}[language=Python, caption=Initialize Single Domain Method]
def initialize_bulk_data_from_initial_conditions(self, 
                                                domain_index: int,
                                                problem,
                                                discretization,
                                                time: float = 0.0) -> BulkData
\end{lstlisting}

\textbf{Parameters:}
\begin{itemize}
    \item \texttt{domain\_index}: Index of domain to initialize
    \item \texttt{problem}: Problem object for this domain
    \item \texttt{discretization}: Discretization object for this domain
    \item \texttt{time}: Initial time (default: 0.0)
\end{itemize}

\textbf{Returns:} \texttt{BulkData} - Initialized BulkData object

\textbf{Process:}
\begin{enumerate}
    \item Validate domain index and framework objects
    \item Create BulkData with primal formulation
    \item Set initial conditions from stored domain data
    \item Default to zero if no initial conditions available
\end{enumerate}

\textbf{Usage:}
\begin{lstlisting}[language=Python, caption=Initialize Single Domain Usage]
# Initialize specific domain
domain_0_bulk = lean_manager.initialize_bulk_data_from_initial_conditions(
    domain_index=0,
    problem=problems[0],
    discretization=discretizations[0],
    time=0.0
)
\end{lstlisting}

\subsection{Data Management Methods}
\label{subsec:data_management}

\paragraph{update\_bulk\_data()}\leavevmode
\begin{lstlisting}[language=Python, caption=Update BulkData Method]
def update_bulk_data(self, bulk_data_list: List[BulkData], new_data_list: List[np.ndarray])
\end{lstlisting}

\textbf{Parameters:}
\begin{itemize}
    \item \texttt{bulk\_data\_list}: List of BulkData objects to update
    \item \texttt{new\_data\_list}: List of new bulk solution arrays
\end{itemize}

\textbf{Returns:} \texttt{None}

\textbf{Side Effects:} Updates data in all BulkData objects

\textbf{Validation:}
\begin{itemize}
    \item Lists must have matching lengths
    \item New data must have compatible shapes
    \item New data must not contain NaN or infinite values
\end{itemize}

\textbf{Usage:}
\begin{lstlisting}[language=Python, caption=Update BulkData Usage]
# Update bulk data with new solution
new_solutions = [...]  # List of numpy arrays
lean_manager.update_bulk_data(
    bulk_data_list=current_bulk_data,
    new_data_list=new_solutions
)
\end{lstlisting}

\paragraph{get\_bulk\_data\_arrays()}\leavevmode
\begin{lstlisting}[language=Python, caption=Get Data Arrays Method]
def get_bulk_data_arrays(self, bulk_data_list: List[BulkData]) -> List[np.ndarray]
\end{lstlisting}

\textbf{Parameters:}
\begin{itemize}
    \item \texttt{bulk\_data\_list}: List of BulkData objects
\end{itemize}

\textbf{Returns:} \texttt{List[np.ndarray]} - Data arrays from all BulkData objects

\textbf{Usage:}
\begin{lstlisting}[language=Python, caption=Get Arrays Usage]
# Extract data arrays for external processing
data_arrays = lean_manager.get_bulk_data_arrays(bulk_data_list)
for i, array in enumerate(data_arrays):
    print(f"Domain {i} data shape: {array.shape}")
\end{lstlisting}

\subsection{Analysis Methods}
\label{subsec:analysis_methods}

\paragraph{compute\_total\_mass()}\leavevmode
\begin{lstlisting}[language=Python, caption=Compute Total Mass Method]
def compute_total_mass(self, bulk_data_list: List[BulkData]) -> float
\end{lstlisting}

\textbf{Parameters:}
\begin{itemize}
    \item \texttt{bulk\_data\_list}: List of BulkData instances
\end{itemize}

\textbf{Returns:} \texttt{float} - Total mass across all domains

\textbf{Computation:} Sums mass from each domain using stored mass matrices

\textbf{Usage:}
\begin{lstlisting}[language=Python, caption=Mass Computation Usage]
# Monitor mass conservation
initial_mass = lean_manager.compute_total_mass(initial_bulk_data)
current_mass = lean_manager.compute_total_mass(current_bulk_data)

mass_change = abs(current_mass - initial_mass) / initial_mass
print(f"Relative mass change: {mass_change:.6e}")
\end{lstlisting}

\paragraph{compute\_mass\_conservation()}\leavevmode
\begin{lstlisting}[language=Python, caption=Mass Conservation Method]
def compute_mass_conservation(self, bulk_data_list: List[BulkData]) -> float
\end{lstlisting}

\textbf{Parameters:}
\begin{itemize}
    \item \texttt{bulk\_data\_list}: List of BulkData instances
\end{itemize}

\textbf{Returns:} \texttt{float} - Total mass (alias for \texttt{compute\_total\_mass})

\textbf{Note:} This method is an alias for consistency with other interfaces

\subsection{Utility Methods}
\label{subsec:utility_methods}

\paragraph{get\_num\_domains()}\leavevmode
\begin{lstlisting}[language=Python, caption=Get Number of Domains Method]
def get_num_domains(self) -> int
\end{lstlisting}

\textbf{Returns:} \texttt{int} - Number of domains managed

\textbf{Usage:}
\begin{lstlisting}[language=Python, caption=Get Domains Count Usage]
n_domains = lean_manager.get_num_domains()
print(f"Managing {n_domains} domains")
\end{lstlisting}

\paragraph{get\_domain\_info()}\leavevmode
\begin{lstlisting}[language=Python, caption=Get Domain Info Method]
def get_domain_info(self, domain_idx: int) -> DomainData
\end{lstlisting}

\textbf{Parameters:}
\begin{itemize}
    \item \texttt{domain\_idx}: Domain index
\end{itemize}

\textbf{Returns:} \texttt{DomainData} - Domain data object for inspection

\textbf{Raises:} \texttt{IndexError} for invalid domain index

\textbf{Usage:}
\begin{lstlisting}[language=Python, caption=Domain Info Usage]
# Inspect domain properties
domain_info = lean_manager.get_domain_info(0)
print(f"Domain 0: {domain_info.neq} equations, {domain_info.n_elements} elements")
print(f"Element length: {domain_info.element_length}")
\end{lstlisting}

\subsection{Testing and Validation}
\label{subsec:testing_validation}

\paragraph{test()}\leavevmode
\begin{lstlisting}[language=Python, caption=Test Method]
def test(self, 
         problems: List = None,
         discretizations: List = None,
         static_condensations: List = None) -> bool
\end{lstlisting}

\textbf{Parameters:}
\begin{itemize}
    \item \texttt{problems}: List of Problem objects for testing (optional)
    \item \texttt{discretizations}: List of discretization objects for testing (optional)
    \item \texttt{static\_condensations}: List of static condensation objects for testing (optional)
\end{itemize}

\textbf{Returns:} \texttt{bool} - True if all tests pass, False otherwise

\textbf{Test Suite:}
\begin{enumerate}
    \item \textbf{Framework Object Validation Test}: Validates provided framework objects
    \item \textbf{Domain Data Structure Test}: Validates stored domain data integrity
    \item \textbf{BulkData Creation Test}: Tests creation of primal and dual BulkData
    \item \textbf{Initialization Test}: Tests bulk data initialization
    \item \textbf{Forcing Term Computation Test}: Tests forcing term calculations
    \item \textbf{Mass Computation Test}: Tests mass conservation calculations
    \item \textbf{Bounds Checking Test}: Tests error handling for invalid indices
    \item \textbf{Utility Methods Test}: Tests helper methods
    \item \textbf{Parameter Mismatch Test}: Tests validation error detection
\end{enumerate}

\textbf{Usage:}
\begin{lstlisting}[language=Python, caption=Test Method Usage]
# Comprehensive testing with framework objects
if lean_manager.test(
    problems=problems,
    discretizations=discretizations,
    static_condensations=static_condensations
):
    print("✓ Lean BulkDataManager is fully functional")
else:
    print("✗ Issues detected in Lean BulkDataManager")

# Minimal testing without framework objects
if lean_manager.test():
    print("✓ Basic structure validation passed")
\end{lstlisting}

\textbf{Sample Test Output:}
\begin{lstlisting}[language=Python, caption=Sample Test Output]
Testing Lean BulkDataManager with 3 domains
PASS: Framework object validation passed
PASS: All domain data validated
PASS: BulkData creation tests passed
PASS: Initialization tests passed
PASS: Forcing term computation tests passed
PASS: Mass computation test passed (total_mass=1.234567e+00)
PASS: ValueError raised for negative domain index
PASS: get_num_domains() returned correct value
PASS: get_domain_info() test passed
PASS: Correctly detected wrong number of problems
PASS: Correctly detected incompatible problem neq
PASS: Parameter mismatch detection tests passed
✓ All Lean BulkDataManager tests passed!
\end{lstlisting}

\subsection{Special Methods}
\label{subsec:special_methods}

\paragraph{\_\_str\_\_()}\leavevmode
\begin{lstlisting}[language=Python, caption=String Representation Method]
def __str__(self) -> str
\end{lstlisting}

\textbf{Returns:} \texttt{str} - Human-readable summary

\textbf{Format:} \texttt{"LeanBulkDataManager(domains=N, total\_elements=M, total\_equations=K)"}

\paragraph{\_\_repr\_\_()}\leavevmode
\begin{lstlisting}[language=Python, caption=Repr Method]
def __repr__(self) -> str
\end{lstlisting}

\textbf{Returns:} \texttt{str} - Developer-oriented representation

\textbf{Format:} \texttt{"LeanBulkDataManager(n\_domains=N, domain\_elements=[...], domain\_equations=[...])"}

\textbf{Usage:}
\begin{lstlisting}[language=Python, caption=String Methods Usage]
print(str(lean_manager))
# Output: LeanBulkDataManager(domains=3, total_elements=60, total_equations=6)

print(repr(lean_manager))
# Output: LeanBulkDataManager(n_domains=3, domain_elements=[20, 20, 20], domain_equations=[2, 2, 2])
\end{lstlisting}

\subsection{Complete Usage Examples}
\label{subsec:complete_usage_examples}

\subsubsection{Standard Workflow Example}

\begin{lstlisting}[language=Python, caption=Complete Lean Manager Workflow]
from ooc1d.core.lean_bulk_data_manager import BulkDataManager
from ooc1d.core.problem import Problem
from ooc1d.core.discretization import Discretization
import numpy as np

# Step 1: Create framework objects (problems, discretizations, static_condensations)
problems = [...]  # List of Problem instances
discretizations = [...]  # List of Discretization instances  
static_condensations = [...]  # List of static condensation instances

# Step 2: Extract essential data once (memory-efficient)
domain_data_list = BulkDataManager.extract_domain_data_list(
    problems=problems,
    discretizations=discretizations,
    static_condensations=static_condensations
)

# Step 3: Create lean manager with extracted data only
lean_manager = BulkDataManager(domain_data_list)

# Step 4: Validate compatibility (optional but recommended)
if not lean_manager.test(problems, discretizations, static_condensations):
    raise RuntimeError("Framework objects incompatible with extracted data")

# Step 5: Initialize bulk data for all domains
bulk_data_list = lean_manager.initialize_all_bulk_data(
    problems=problems,
    discretizations=discretizations,
    time=0.0
)

# Step 6: Time evolution loop
dt = 0.01
for time_step in range(100):
    current_time = time_step * dt
    
    # Compute forcing terms for implicit Euler
    forcing_terms = lean_manager.compute_forcing_terms(
        bulk_data_list=bulk_data_list,
        problems=problems,
        discretizations=discretizations,
        time=current_time,
        dt=dt
    )
    
    # Solve system (external solver)
    new_solutions = solve_system(forcing_terms)  # User-defined solver
    
    # Update bulk data with new solutions
    lean_manager.update_bulk_data(bulk_data_list, new_solutions)
    
    # Monitor mass conservation
    current_mass = lean_manager.compute_total_mass(bulk_data_list)
    if time_step % 10 == 0:
        print(f"Time {current_time:.3f}: Mass = {current_mass:.6e}")

print("✓ Time evolution completed with lean manager")
\end{lstlisting}

\subsubsection{Memory Comparison Example}

\begin{lstlisting}[language=Python, caption=Memory Usage Comparison]
import psutil
import os

# Measure memory before
process = psutil.Process(os.getpid())
memory_before = process.memory_info().rss / 1024 / 1024  # MB

# Traditional approach (stores framework objects)
# traditional_manager = FullBulkDataManager(problems, discretizations, static_condensations)

# Lean approach (stores only essential data)
domain_data_list = BulkDataManager.extract_domain_data_list(
    problems, discretizations, static_condensations
)
lean_manager = BulkDataManager(domain_data_list)

# Measure memory after
memory_after = process.memory_info().rss / 1024 / 1024  # MB
memory_used = memory_after - memory_before

print(f"Memory used by lean manager: {memory_used:.2f} MB")
print(f"Domains managed: {lean_manager.get_num_domains()}")
print(f"Memory per domain: {memory_used / lean_manager.get_num_domains():.2f} MB")
\end{lstlisting}

\subsubsection{Multi-Manager Example}

\begin{lstlisting}[language=Python, caption=Multiple Lean Managers from Same Data]
# Extract domain data once
domain_data_list = BulkDataManager.extract_domain_data_list(
    problems, discretizations, static_condensations
)

# Create multiple lean managers for different purposes
# (all sharing the same extracted data - no additional memory cost)

# Manager for time evolution
evolution_manager = BulkDataManager(domain_data_list)

# Manager for forcing term computation
forcing_manager = BulkDataManager(domain_data_list)

# Manager for mass conservation tracking  
conservation_manager = BulkDataManager(domain_data_list)

# Each manager can operate independently but uses same base data
initial_bulk = evolution_manager.initialize_all_bulk_data(problems, discretizations)
source_terms = forcing_manager.compute_source_terms(problems, discretizations, time=0.0)
total_mass = conservation_manager.compute_total_mass(initial_bulk)

print(f"Created 3 independent managers sharing {len(domain_data_list)} domain data objects")
\end{lstlisting}

\subsection{Method Summary Table}
\label{subsec:lean_method_summary}

\begin{longtable}{|p{5.3cm}|p{3.2cm}|p{5cm}|}
\hline
\textbf{Method} & \textbf{Returns} & \textbf{Purpose} \\
\hline
\endhead

\texttt{\_\_init\_\_} & \texttt{None} & Initialize with extracted domain data only \\
\hline

\texttt{extract\_domain\_data\_list} & \texttt{List[DomainData]} & Static factory for one-time data extraction \\
\hline

\texttt{create\_bulk\_data} & \texttt{BulkData} & Create BulkData using external framework objects \\
\hline

\texttt{initialize\_all\_bulk\_data} & \texttt{List[BulkData]} & Initialize all domains with initial conditions \\
\hline

\texttt{compute\_source\_terms} & \texttt{List[BulkData]} & Compute source terms using dual formulation \\
\hline

\texttt{compute\_forcing\_terms} & \texttt{List[np.ndarray]} & Compute forcing terms for implicit Euler \\
\hline

\texttt{update\_bulk\_data} & \texttt{None} & Update BulkData objects with new solutions \\
\hline

\texttt{compute\_total\_mass} & \texttt{float} & Calculate total mass for conservation \\
\hline

\texttt{get\_bulk\_data\_arrays} & \texttt{List[np.ndarray]} & Extract data arrays from BulkData objects \\
\hline

\texttt{get\_num\_domains} & \texttt{int} & Get number of managed domains \\
\hline

\texttt{get\_domain\_info} & \texttt{DomainData} & Access domain data for inspection \\
\hline

\texttt{test} & \texttt{bool} & Comprehensive validation and testing \\
\hline

\texttt{\_validate\_framework\_objects} & \texttt{None} & Validate framework object compatibility \\
\hline

\end{longtable}

This documentation provides an exact reference for the lean BulkDataManager class, emphasizing its memory-efficient design and parameter-based approach to framework object usage. The lean architecture minimizes memory overhead while maintaining full functionality through external object validation and flexible method interfaces.

% End of lean bulk data manager module API documentation


\section{Lean Bulk Data Manager Module API Reference}
\label{sec:lean_bulk_data_manager_api}

This section provides an exact reference for the BulkDataManager class (\texttt{ooc1d.core.lean\_bulk\_data\_manager.BulkDataManager}) based on detailed analysis of the actual implementation. This is an ultra-lean coordinator that minimizes memory usage by storing only essential domain data and accepting framework objects as method parameters.

\subsection{Module Overview}

The lean bulk data manager provides memory-efficient coordination for bulk operations by:
\begin{itemize}
    \item Storing only extracted essential domain data
    \item Accepting framework objects as method parameters
    \item Validating framework object compatibility
    \item Providing flexible bulk data operations without memory overhead
\end{itemize}

\subsection{Module Imports and Dependencies}

\begin{lstlisting}[language=Python, caption=Module Dependencies]
import numpy as np
from typing import List, Optional, Callable
from ooc1d.core.bulk_data import BulkData
from ooc1d.core.domain_data import DomainData
\end{lstlisting}

\subsection{BulkDataManager Class Definition}
\label{subsec:lean_bulk_data_manager_class}

\begin{lstlisting}[language=Python, caption=Class Declaration]
class BulkDataManager:
    """
    Ultra-lean coordinator for bulk operations in HDG method.
    
    This class stores only essential extracted domain data and accepts
    framework objects as parameters to methods that need them. This approach
    minimizes memory usage and increases flexibility.
    """
\end{lstlisting}

\subsection{Constructor}
\label{subsec:lean_constructor}

\paragraph{\_\_init\_\_()}\leavevmode
\begin{lstlisting}[language=Python, caption=Lean BulkDataManager Constructor]
def __init__(self, domain_data_list: List[DomainData])
\end{lstlisting}

\textbf{Parameters:}
\begin{itemize}
    \item \texttt{domain\_data\_list}: List of DomainData objects with essential extracted information
\end{itemize}

\textbf{Side Effects:} Sets \texttt{self.domain\_data\_list} attribute

\textbf{Usage:}
\begin{lstlisting}[language=Python, caption=Constructor Usage]
# Extract domain data first (see static factory method)
domain_data_list = BulkDataManager.extract_domain_data_list(
    problems, discretizations, static_condensations
)

# Create lean manager with extracted data only
lean_manager = BulkDataManager(domain_data_list)
\end{lstlisting}

\subsection{Core Attributes}
\label{subsec:lean_attributes}

\begin{longtable}{|p{3.5cm}|p{2.5cm}|p{7cm}|}
\hline
\textbf{Attribute} & \textbf{Type} & \textbf{Description} \\
\hline
\endhead

\texttt{domain\_data\_list} & \texttt{List[DomainData]} & List of essential domain data objects (only stored attribute) \\
\hline

\end{longtable}

\textbf{Memory Efficiency:} The lean manager stores \textbf{only} the domain data list, avoiding storage of large framework objects.

\subsection{Validation Methods}
\label{subsec:validation_methods}

\paragraph{\_validate\_framework\_objects()}\leavevmode
\begin{lstlisting}[language=Python, caption=Framework Validation Method]
def _validate_framework_objects(self, 
                               problems: List = None,
                               discretizations: List = None, 
                               static_condensations: List = None,
                               operation_name: str = "operation") -> None
\end{lstlisting}

\textbf{Parameters:}
\begin{itemize}
    \item \texttt{problems}: List of Problem objects to validate (optional)
    \item \texttt{discretizations}: List of discretization objects to validate (optional)
    \item \texttt{static\_condensations}: List of static condensation objects to validate (optional)
    \item \texttt{operation\_name}: Name of operation for error messages (default: "operation")
\end{itemize}

\textbf{Returns:} \texttt{None}

\textbf{Raises:} \texttt{ValueError} for incompatible framework objects

\textbf{Validation Checks:}
\begin{enumerate}
    \item \textbf{List Length Validation}: All provided lists must match domain count
    \item \textbf{Problem Validation}: \texttt{neq} attribute must match stored domain data
    \item \textbf{Discretization Validation}: \texttt{n\_elements}, nodes, and \texttt{element\_length} compatibility
    \item \textbf{Static Condensation Validation}: Matrix compatibility and method availability
\end{enumerate}

\textbf{Usage:}
\begin{lstlisting}[language=Python, caption=Validation Usage]
try:
    lean_manager._validate_framework_objects(
        problems=problems,
        discretizations=discretizations,
        operation_name="my_operation"
    )
    print("✓ Framework objects are compatible")
except ValueError as e:
    print(f"✗ Validation failed: {e}")
\end{lstlisting}

\subsection{Static Factory Methods}
\label{subsec:static_factory_methods}

\paragraph{extract\_domain\_data\_list()}\leavevmode
\begin{lstlisting}[language=Python, caption=Domain Data Extraction Method]
@staticmethod
def extract_domain_data_list(problems: List, 
                            discretizations: List, 
                            static_condensations: List) -> List[DomainData]
\end{lstlisting}

\textbf{Parameters:}
\begin{itemize}
    \item \texttt{problems}: List of Problem instances
    \item \texttt{discretizations}: List of discretization instances
    \item \texttt{static\_condensations}: List of static condensation instances
\end{itemize}

\textbf{Returns:} \texttt{List[DomainData]} - Extracted essential domain information

\textbf{Purpose:} Static factory method to extract and store essential data once for reuse

\textbf{Usage:}
\begin{lstlisting}[language=Python, caption=Domain Data Extraction Usage]
# One-time extraction of essential domain data
domain_data_list = BulkDataManager.extract_domain_data_list(
    problems=problems,
    discretizations=discretizations,
    static_condensations=static_condensations
)

# Can create multiple lean managers with same extracted data
lean_manager_1 = BulkDataManager(domain_data_list)
lean_manager_2 = BulkDataManager(domain_data_list)
\end{lstlisting}

\paragraph{\_extract\_single\_domain\_data()}\leavevmode
\begin{lstlisting}[language=Python, caption=Single Domain Extraction Method]
@staticmethod
def _extract_single_domain_data(problem, discretization, sc, domain_idx: int) -> DomainData
\end{lstlisting}

\textbf{Parameters:}
\begin{itemize}
    \item \texttt{problem}: Problem object for single domain
    \item \texttt{discretization}: Discretization object for single domain
    \item \texttt{sc}: Static condensation object for single domain
    \item \texttt{domain\_idx}: Domain index for error reporting
\end{itemize}

\textbf{Returns:} \texttt{DomainData} - Extracted essential data for single domain

\textbf{Extraction Process:}
\begin{enumerate}
    \item Extract matrices from static condensation (\texttt{M}, \texttt{T})
    \item Extract initial conditions with multiple access pattern support
    \item Extract forcing functions with multiple access pattern support
    \item Create DomainData object with essential information
\end{enumerate}

\subsection{BulkData Creation Methods}
\label{subsec_bulk_data_creation}

\paragraph{create\_bulk\_data()}\leavevmode
\begin{lstlisting}[language=Python, caption=Create BulkData Method]
def create_bulk_data(self, 
                    domain_index: int, 
                    problem, 
                    discretization, 
                    dual: bool = False) -> BulkData
\end{lstlisting}

\textbf{Parameters:}
\begin{itemize}
    \item \texttt{domain\_index}: Index of the domain (0 to \texttt{n\_domains-1})
    \item \texttt{problem}: Problem object for this domain
    \item \texttt{discretization}: Discretization object for this domain
    \item \texttt{dual}: Whether to use dual formulation (default: False)
\end{itemize}

\textbf{Returns:} \texttt{BulkData} - New BulkData object

\textbf{Raises:} \texttt{ValueError} for invalid domain index

\textbf{Usage:}
\begin{lstlisting}[language=Python, caption=BulkData Creation Usage]
# Create primal BulkData for domain 0
bulk_data_primal = lean_manager.create_bulk_data(
    domain_index=0,
    problem=problems[0],
    discretization=discretizations[0],
    dual=False
)

# Create dual BulkData for forcing terms
bulk_data_dual = lean_manager.create_bulk_data(
    domain_index=0,
    problem=problems[0],
    discretization=discretizations[0],
    dual=True
)
\end{lstlisting}

\subsection{Bulk Operations Methods}
\label{subsec:bulk_operations}

\paragraph{compute\_source\_terms()}\leavevmode
\begin{lstlisting}[language=Python, caption=Compute Source Terms Method]
def compute_source_terms(self,
                        problems: List,
                        discretizations: List,
                        time: float) -> List[BulkData]
\end{lstlisting}

\textbf{Parameters:}
\begin{itemize}
    \item \texttt{problems}: List of Problem objects
    \item \texttt{discretizations}: List of discretization objects
    \item \texttt{time}: Current time for evaluation
\end{itemize}

\textbf{Returns:} \texttt{List[BulkData]} - Source terms for all domains using dual formulation

\textbf{Process:}
\begin{enumerate}
    \item Validate framework objects against stored domain data
    \item Create dual BulkData for each domain
    \item Set forcing functions from problems at specified time
    \item Return list of integrated source terms
\end{enumerate}

\textbf{Usage:}
\begin{lstlisting}[language=Python, caption=Source Terms Usage]
# Compute source terms at time t=0.5
source_terms = lean_manager.compute_source_terms(
    problems=problems,
    discretizations=discretizations,
    time=0.5
)

print(f"Computed {len(source_terms)} source term BulkData objects")
\end{lstlisting}

\paragraph{compute\_forcing\_terms()}\leavevmode
\begin{lstlisting}[language=Python, caption=Compute Forcing Terms Method]
def compute_forcing_terms(self, 
                          bulk_data_list: List[BulkData],
                          problems: List,
                          discretizations: List, 
                          time: float, 
                          dt: float) -> List[np.ndarray]
\end{lstlisting}

\textbf{Parameters:}
\begin{itemize}
    \item \texttt{bulk\_data\_list}: List of current BulkData solutions
    \item \texttt{problems}: List of Problem objects
    \item \texttt{discretizations}: List of discretization objects
    \item \texttt{time}: Current time
    \item \texttt{dt}: Time step size
\end{itemize}

\textbf{Returns:} \texttt{List[np.ndarray]} - Forcing term arrays for implicit Euler

\textbf{Computation:} For each domain: \texttt{forcing\_term = dt * force\_contrib + M * U\_old}

\textbf{Usage:}
\begin{lstlisting}[language=Python, caption=Forcing Terms Usage]
# Compute forcing terms for implicit Euler step
current_bulk_data = [...]  # Current solution
forcing_terms = lean_manager.compute_forcing_terms(
    bulk_data_list=current_bulk_data,
    problems=problems,
    discretizations=discretizations,
    time=0.1,
    dt=0.01
)

# forcing_terms[i] has shape (2*neq, n_elements) for domain i
\end{lstlisting}

\subsection{Initialization Methods}
\label{subsec:initialization_methods}

\paragraph{initialize\_all\_bulk\_data()}\leavevmode
\begin{lstlisting}[language=Python, caption=Initialize All BulkData Method]
def initialize_all_bulk_data(self, 
                            problems: List,
                            discretizations: List,
                            time: float = 0.0) -> List[BulkData]
\end{lstlisting}

\textbf{Parameters:}
\begin{itemize}
    \item \texttt{problems}: List of Problem objects
    \item \texttt{discretizations}: List of discretization objects
    \item \texttt{time}: Initial time (default: 0.0)
\end{itemize}

\textbf{Returns:} \texttt{List[BulkData]} - Initialized BulkData objects for all domains

\textbf{Usage:}
\begin{lstlisting}[language=Python, caption=Initialize All Usage]
# Initialize all domains with initial conditions
initial_bulk_data = lean_manager.initialize_all_bulk_data(
    problems=problems,
    discretizations=discretizations,
    time=0.0
)

print(f"Initialized {len(initial_bulk_data)} BulkData objects")
\end{lstlisting}

\paragraph{initialize\_bulk\_data\_from\_initial\_conditions()}\leavevmode
\begin{lstlisting}[language=Python, caption=Initialize Single Domain Method]
def initialize_bulk_data_from_initial_conditions(self, 
                                                domain_index: int,
                                                problem,
                                                discretization,
                                                time: float = 0.0) -> BulkData
\end{lstlisting}

\textbf{Parameters:}
\begin{itemize}
    \item \texttt{domain\_index}: Index of domain to initialize
    \item \texttt{problem}: Problem object for this domain
    \item \texttt{discretization}: Discretization object for this domain
    \item \texttt{time}: Initial time (default: 0.0)
\end{itemize}

\textbf{Returns:} \texttt{BulkData} - Initialized BulkData object

\textbf{Process:}
\begin{enumerate}
    \item Validate domain index and framework objects
    \item Create BulkData with primal formulation
    \item Set initial conditions from stored domain data
    \item Default to zero if no initial conditions available
\end{enumerate}

\textbf{Usage:}
\begin{lstlisting}[language=Python, caption=Initialize Single Domain Usage]
# Initialize specific domain
domain_0_bulk = lean_manager.initialize_bulk_data_from_initial_conditions(
    domain_index=0,
    problem=problems[0],
    discretization=discretizations[0],
    time=0.0
)
\end{lstlisting}

\subsection{Data Management Methods}
\label{subsec:data_management}

\paragraph{update\_bulk\_data()}\leavevmode
\begin{lstlisting}[language=Python, caption=Update BulkData Method]
def update_bulk_data(self, bulk_data_list: List[BulkData], new_data_list: List[np.ndarray])
\end{lstlisting}

\textbf{Parameters:}
\begin{itemize}
    \item \texttt{bulk\_data\_list}: List of BulkData objects to update
    \item \texttt{new\_data\_list}: List of new bulk solution arrays
\end{itemize}

\textbf{Returns:} \texttt{None}

\textbf{Side Effects:} Updates data in all BulkData objects

\textbf{Validation:}
\begin{itemize}
    \item Lists must have matching lengths
    \item New data must have compatible shapes
    \item New data must not contain NaN or infinite values
\end{itemize}

\textbf{Usage:}
\begin{lstlisting}[language=Python, caption=Update BulkData Usage]
# Update bulk data with new solution
new_solutions = [...]  # List of numpy arrays
lean_manager.update_bulk_data(
    bulk_data_list=current_bulk_data,
    new_data_list=new_solutions
)
\end{lstlisting}

\paragraph{get\_bulk\_data\_arrays()}\leavevmode
\begin{lstlisting}[language=Python, caption=Get Data Arrays Method]
def get_bulk_data_arrays(self, bulk_data_list: List[BulkData]) -> List[np.ndarray]
\end{lstlisting}

\textbf{Parameters:}
\begin{itemize}
    \item \texttt{bulk\_data\_list}: List of BulkData objects
\end{itemize}

\textbf{Returns:} \texttt{List[np.ndarray]} - Data arrays from all BulkData objects

\textbf{Usage:}
\begin{lstlisting}[language=Python, caption=Get Arrays Usage]
# Extract data arrays for external processing
data_arrays = lean_manager.get_bulk_data_arrays(bulk_data_list)
for i, array in enumerate(data_arrays):
    print(f"Domain {i} data shape: {array.shape}")
\end{lstlisting}

\subsection{Analysis Methods}
\label{subsec:analysis_methods}

\paragraph{compute\_total\_mass()}\leavevmode
\begin{lstlisting}[language=Python, caption=Compute Total Mass Method]
def compute_total_mass(self, bulk_data_list: List[BulkData]) -> float
\end{lstlisting}

\textbf{Parameters:}
\begin{itemize}
    \item \texttt{bulk\_data\_list}: List of BulkData instances
\end{itemize}

\textbf{Returns:} \texttt{float} - Total mass across all domains

\textbf{Computation:} Sums mass from each domain using stored mass matrices

\textbf{Usage:}
\begin{lstlisting}[language=Python, caption=Mass Computation Usage]
# Monitor mass conservation
initial_mass = lean_manager.compute_total_mass(initial_bulk_data)
current_mass = lean_manager.compute_total_mass(current_bulk_data)

mass_change = abs(current_mass - initial_mass) / initial_mass
print(f"Relative mass change: {mass_change:.6e}")
\end{lstlisting}

\paragraph{compute\_mass\_conservation()}\leavevmode
\begin{lstlisting}[language=Python, caption=Mass Conservation Method]
def compute_mass_conservation(self, bulk_data_list: List[BulkData]) -> float
\end{lstlisting}

\textbf{Parameters:}
\begin{itemize}
    \item \texttt{bulk\_data\_list}: List of BulkData instances
\end{itemize}

\textbf{Returns:} \texttt{float} - Total mass (alias for \texttt{compute\_total\_mass})

\textbf{Note:} This method is an alias for consistency with other interfaces

\subsection{Utility Methods}
\label{subsec:utility_methods}

\paragraph{get\_num\_domains()}\leavevmode
\begin{lstlisting}[language=Python, caption=Get Number of Domains Method]
def get_num_domains(self) -> int
\end{lstlisting}

\textbf{Returns:} \texttt{int} - Number of domains managed

\textbf{Usage:}
\begin{lstlisting}[language=Python, caption=Get Domains Count Usage]
n_domains = lean_manager.get_num_domains()
print(f"Managing {n_domains} domains")
\end{lstlisting}

\paragraph{get\_domain\_info()}\leavevmode
\begin{lstlisting}[language=Python, caption=Get Domain Info Method]
def get_domain_info(self, domain_idx: int) -> DomainData
\end{lstlisting}

\textbf{Parameters:}
\begin{itemize}
    \item \texttt{domain\_idx}: Domain index
\end{itemize}

\textbf{Returns:} \texttt{DomainData} - Domain data object for inspection

\textbf{Raises:} \texttt{IndexError} for invalid domain index

\textbf{Usage:}
\begin{lstlisting}[language=Python, caption=Domain Info Usage]
# Inspect domain properties
domain_info = lean_manager.get_domain_info(0)
print(f"Domain 0: {domain_info.neq} equations, {domain_info.n_elements} elements")
print(f"Element length: {domain_info.element_length}")
\end{lstlisting}

\subsection{Testing and Validation}
\label{subsec:testing_validation}

\paragraph{test()}\leavevmode
\begin{lstlisting}[language=Python, caption=Test Method]
def test(self, 
         problems: List = None,
         discretizations: List = None,
         static_condensations: List = None) -> bool
\end{lstlisting}

\textbf{Parameters:}
\begin{itemize}
    \item \texttt{problems}: List of Problem objects for testing (optional)
    \item \texttt{discretizations}: List of discretization objects for testing (optional)
    \item \texttt{static\_condensations}: List of static condensation objects for testing (optional)
\end{itemize}

\textbf{Returns:} \texttt{bool} - True if all tests pass, False otherwise

\textbf{Test Suite:}
\begin{enumerate}
    \item \textbf{Framework Object Validation Test}: Validates provided framework objects
    \item \textbf{Domain Data Structure Test}: Validates stored domain data integrity
    \item \textbf{BulkData Creation Test}: Tests creation of primal and dual BulkData
    \item \textbf{Initialization Test}: Tests bulk data initialization
    \item \textbf{Forcing Term Computation Test}: Tests forcing term calculations
    \item \textbf{Mass Computation Test}: Tests mass conservation calculations
    \item \textbf{Bounds Checking Test}: Tests error handling for invalid indices
    \item \textbf{Utility Methods Test}: Tests helper methods
    \item \textbf{Parameter Mismatch Test}: Tests validation error detection
\end{enumerate}

\textbf{Usage:}
\begin{lstlisting}[language=Python, caption=Test Method Usage]
# Comprehensive testing with framework objects
if lean_manager.test(
    problems=problems,
    discretizations=discretizations,
    static_condensations=static_condensations
):
    print("✓ Lean BulkDataManager is fully functional")
else:
    print("✗ Issues detected in Lean BulkDataManager")

# Minimal testing without framework objects
if lean_manager.test():
    print("✓ Basic structure validation passed")
\end{lstlisting}

\textbf{Sample Test Output:}
\begin{lstlisting}[language=Python, caption=Sample Test Output]
Testing Lean BulkDataManager with 3 domains
PASS: Framework object validation passed
PASS: All domain data validated
PASS: BulkData creation tests passed
PASS: Initialization tests passed
PASS: Forcing term computation tests passed
PASS: Mass computation test passed (total_mass=1.234567e+00)
PASS: ValueError raised for negative domain index
PASS: get_num_domains() returned correct value
PASS: get_domain_info() test passed
PASS: Correctly detected wrong number of problems
PASS: Correctly detected incompatible problem neq
PASS: Parameter mismatch detection tests passed
✓ All Lean BulkDataManager tests passed!
\end{lstlisting}

\subsection{Special Methods}
\label{subsec:special_methods}

\paragraph{\_\_str\_\_()}\leavevmode
\begin{lstlisting}[language=Python, caption=String Representation Method]
def __str__(self) -> str
\end{lstlisting}

\textbf{Returns:} \texttt{str} - Human-readable summary

\textbf{Format:} \texttt{"LeanBulkDataManager(domains=N, total\_elements=M, total\_equations=K)"}

\paragraph{\_\_repr\_\_()}\leavevmode
\begin{lstlisting}[language=Python, caption=Repr Method]
def __repr__(self) -> str
\end{lstlisting}

\textbf{Returns:} \texttt{str} - Developer-oriented representation

\textbf{Format:} \texttt{"LeanBulkDataManager(n\_domains=N, domain\_elements=[...], domain\_equations=[...])"}

\textbf{Usage:}
\begin{lstlisting}[language=Python, caption=String Methods Usage]
print(str(lean_manager))
# Output: LeanBulkDataManager(domains=3, total_elements=60, total_equations=6)

print(repr(lean_manager))
# Output: LeanBulkDataManager(n_domains=3, domain_elements=[20, 20, 20], domain_equations=[2, 2, 2])
\end{lstlisting}

\subsection{Complete Usage Examples}
\label{subsec:complete_usage_examples}

\subsubsection{Standard Workflow Example}

\begin{lstlisting}[language=Python, caption=Complete Lean Manager Workflow]
from ooc1d.core.lean_bulk_data_manager import BulkDataManager
from ooc1d.core.problem import Problem
from ooc1d.core.discretization import Discretization
import numpy as np

# Step 1: Create framework objects (problems, discretizations, static_condensations)
problems = [...]  # List of Problem instances
discretizations = [...]  # List of Discretization instances  
static_condensations = [...]  # List of static condensation instances

# Step 2: Extract essential data once (memory-efficient)
domain_data_list = BulkDataManager.extract_domain_data_list(
    problems=problems,
    discretizations=discretizations,
    static_condensations=static_condensations
)

# Step 3: Create lean manager with extracted data only
lean_manager = BulkDataManager(domain_data_list)

# Step 4: Validate compatibility (optional but recommended)
if not lean_manager.test(problems, discretizations, static_condensations):
    raise RuntimeError("Framework objects incompatible with extracted data")

# Step 5: Initialize bulk data for all domains
bulk_data_list = lean_manager.initialize_all_bulk_data(
    problems=problems,
    discretizations=discretizations,
    time=0.0
)

# Step 6: Time evolution loop
dt = 0.01
for time_step in range(100):
    current_time = time_step * dt
    
    # Compute forcing terms for implicit Euler
    forcing_terms = lean_manager.compute_forcing_terms(
        bulk_data_list=bulk_data_list,
        problems=problems,
        discretizations=discretizations,
        time=current_time,
        dt=dt
    )
    
    # Solve system (external solver)
    new_solutions = solve_system(forcing_terms)  # User-defined solver
    
    # Update bulk data with new solutions
    lean_manager.update_bulk_data(bulk_data_list, new_solutions)
    
    # Monitor mass conservation
    current_mass = lean_manager.compute_total_mass(bulk_data_list)
    if time_step % 10 == 0:
        print(f"Time {current_time:.3f}: Mass = {current_mass:.6e}")

print("✓ Time evolution completed with lean manager")
\end{lstlisting}

\subsubsection{Memory Comparison Example}

\begin{lstlisting}[language=Python, caption=Memory Usage Comparison]
import psutil
import os

# Measure memory before
process = psutil.Process(os.getpid())
memory_before = process.memory_info().rss / 1024 / 1024  # MB

# Traditional approach (stores framework objects)
# traditional_manager = FullBulkDataManager(problems, discretizations, static_condensations)

# Lean approach (stores only essential data)
domain_data_list = BulkDataManager.extract_domain_data_list(
    problems, discretizations, static_condensations
)
lean_manager = BulkDataManager(domain_data_list)

# Measure memory after
memory_after = process.memory_info().rss / 1024 / 1024  # MB
memory_used = memory_after - memory_before

print(f"Memory used by lean manager: {memory_used:.2f} MB")
print(f"Domains managed: {lean_manager.get_num_domains()}")
print(f"Memory per domain: {memory_used / lean_manager.get_num_domains():.2f} MB")
\end{lstlisting}

\subsubsection{Multi-Manager Example}

\begin{lstlisting}[language=Python, caption=Multiple Lean Managers from Same Data]
# Extract domain data once
domain_data_list = BulkDataManager.extract_domain_data_list(
    problems, discretizations, static_condensations
)

# Create multiple lean managers for different purposes
# (all sharing the same extracted data - no additional memory cost)

# Manager for time evolution
evolution_manager = BulkDataManager(domain_data_list)

# Manager for forcing term computation
forcing_manager = BulkDataManager(domain_data_list)

# Manager for mass conservation tracking  
conservation_manager = BulkDataManager(domain_data_list)

# Each manager can operate independently but uses same base data
initial_bulk = evolution_manager.initialize_all_bulk_data(problems, discretizations)
source_terms = forcing_manager.compute_source_terms(problems, discretizations, time=0.0)
total_mass = conservation_manager.compute_total_mass(initial_bulk)

print(f"Created 3 independent managers sharing {len(domain_data_list)} domain data objects")
\end{lstlisting}

\subsection{Method Summary Table}
\label{subsec:lean_method_summary}

\begin{longtable}{|p{5.3cm}|p{3.2cm}|p{5cm}|}
\hline
\textbf{Method} & \textbf{Returns} & \textbf{Purpose} \\
\hline
\endhead

\texttt{\_\_init\_\_} & \texttt{None} & Initialize with extracted domain data only \\
\hline

\texttt{extract\_domain\_data\_list} & \texttt{List[DomainData]} & Static factory for one-time data extraction \\
\hline

\texttt{create\_bulk\_data} & \texttt{BulkData} & Create BulkData using external framework objects \\
\hline

\texttt{initialize\_all\_bulk\_data} & \texttt{List[BulkData]} & Initialize all domains with initial conditions \\
\hline

\texttt{compute\_source\_terms} & \texttt{List[BulkData]} & Compute source terms using dual formulation \\
\hline

\texttt{compute\_forcing\_terms} & \texttt{List[np.ndarray]} & Compute forcing terms for implicit Euler \\
\hline

\texttt{update\_bulk\_data} & \texttt{None} & Update BulkData objects with new solutions \\
\hline

\texttt{compute\_total\_mass} & \texttt{float} & Calculate total mass for conservation \\
\hline

\texttt{get\_bulk\_data\_arrays} & \texttt{List[np.ndarray]} & Extract data arrays from BulkData objects \\
\hline

\texttt{get\_num\_domains} & \texttt{int} & Get number of managed domains \\
\hline

\texttt{get\_domain\_info} & \texttt{DomainData} & Access domain data for inspection \\
\hline

\texttt{test} & \texttt{bool} & Comprehensive validation and testing \\
\hline

\texttt{\_validate\_framework\_objects} & \texttt{None} & Validate framework object compatibility \\
\hline

\end{longtable}

This documentation provides an exact reference for the lean BulkDataManager class, emphasizing its memory-efficient design and parameter-based approach to framework object usage. The lean architecture minimizes memory overhead while maintaining full functionality through external object validation and flexible method interfaces.

% End of lean bulk data manager module API documentation


\section{Lean Bulk Data Manager Module API Reference}
\label{sec:lean_bulk_data_manager_api}

This section provides an exact reference for the BulkDataManager class (\texttt{ooc1d.core.lean\_bulk\_data\_manager.BulkDataManager}) based on detailed analysis of the actual implementation. This is an ultra-lean coordinator that minimizes memory usage by storing only essential domain data and accepting framework objects as method parameters.

\subsection{Module Overview}

The lean bulk data manager provides memory-efficient coordination for bulk operations by:
\begin{itemize}
    \item Storing only extracted essential domain data
    \item Accepting framework objects as method parameters
    \item Validating framework object compatibility
    \item Providing flexible bulk data operations without memory overhead
\end{itemize}

\subsection{Module Imports and Dependencies}

\begin{lstlisting}[language=Python, caption=Module Dependencies]
import numpy as np
from typing import List, Optional, Callable
from ooc1d.core.bulk_data import BulkData
from ooc1d.core.domain_data import DomainData
\end{lstlisting}

\subsection{BulkDataManager Class Definition}
\label{subsec:lean_bulk_data_manager_class}

\begin{lstlisting}[language=Python, caption=Class Declaration]
class BulkDataManager:
    """
    Ultra-lean coordinator for bulk operations in HDG method.
    
    This class stores only essential extracted domain data and accepts
    framework objects as parameters to methods that need them. This approach
    minimizes memory usage and increases flexibility.
    """
\end{lstlisting}

\subsection{Constructor}
\label{subsec:lean_constructor}

\paragraph{\_\_init\_\_()}\leavevmode
\begin{lstlisting}[language=Python, caption=Lean BulkDataManager Constructor]
def __init__(self, domain_data_list: List[DomainData])
\end{lstlisting}

\textbf{Parameters:}
\begin{itemize}
    \item \texttt{domain\_data\_list}: List of DomainData objects with essential extracted information
\end{itemize}

\textbf{Side Effects:} Sets \texttt{self.domain\_data\_list} attribute

\textbf{Usage:}
\begin{lstlisting}[language=Python, caption=Constructor Usage]
# Extract domain data first (see static factory method)
domain_data_list = BulkDataManager.extract_domain_data_list(
    problems, discretizations, static_condensations
)

# Create lean manager with extracted data only
lean_manager = BulkDataManager(domain_data_list)
\end{lstlisting}

\subsection{Core Attributes}
\label{subsec:lean_attributes}

\begin{longtable}{|p{3.5cm}|p{2.5cm}|p{7cm}|}
\hline
\textbf{Attribute} & \textbf{Type} & \textbf{Description} \\
\hline
\endhead

\texttt{domain\_data\_list} & \texttt{List[DomainData]} & List of essential domain data objects (only stored attribute) \\
\hline

\end{longtable}

\textbf{Memory Efficiency:} The lean manager stores \textbf{only} the domain data list, avoiding storage of large framework objects.

\subsection{Validation Methods}
\label{subsec:validation_methods}

\paragraph{\_validate\_framework\_objects()}\leavevmode
\begin{lstlisting}[language=Python, caption=Framework Validation Method]
def _validate_framework_objects(self, 
                               problems: List = None,
                               discretizations: List = None, 
                               static_condensations: List = None,
                               operation_name: str = "operation") -> None
\end{lstlisting}

\textbf{Parameters:}
\begin{itemize}
    \item \texttt{problems}: List of Problem objects to validate (optional)
    \item \texttt{discretizations}: List of discretization objects to validate (optional)
    \item \texttt{static\_condensations}: List of static condensation objects to validate (optional)
    \item \texttt{operation\_name}: Name of operation for error messages (default: "operation")
\end{itemize}

\textbf{Returns:} \texttt{None}

\textbf{Raises:} \texttt{ValueError} for incompatible framework objects

\textbf{Validation Checks:}
\begin{enumerate}
    \item \textbf{List Length Validation}: All provided lists must match domain count
    \item \textbf{Problem Validation}: \texttt{neq} attribute must match stored domain data
    \item \textbf{Discretization Validation}: \texttt{n\_elements}, nodes, and \texttt{element\_length} compatibility
    \item \textbf{Static Condensation Validation}: Matrix compatibility and method availability
\end{enumerate}

\textbf{Usage:}
\begin{lstlisting}[language=Python, caption=Validation Usage]
try:
    lean_manager._validate_framework_objects(
        problems=problems,
        discretizations=discretizations,
        operation_name="my_operation"
    )
    print("✓ Framework objects are compatible")
except ValueError as e:
    print(f"✗ Validation failed: {e}")
\end{lstlisting}

\subsection{Static Factory Methods}
\label{subsec:static_factory_methods}

\paragraph{extract\_domain\_data\_list()}\leavevmode
\begin{lstlisting}[language=Python, caption=Domain Data Extraction Method]
@staticmethod
def extract_domain_data_list(problems: List, 
                            discretizations: List, 
                            static_condensations: List) -> List[DomainData]
\end{lstlisting}

\textbf{Parameters:}
\begin{itemize}
    \item \texttt{problems}: List of Problem instances
    \item \texttt{discretizations}: List of discretization instances
    \item \texttt{static\_condensations}: List of static condensation instances
\end{itemize}

\textbf{Returns:} \texttt{List[DomainData]} - Extracted essential domain information

\textbf{Purpose:} Static factory method to extract and store essential data once for reuse

\textbf{Usage:}
\begin{lstlisting}[language=Python, caption=Domain Data Extraction Usage]
# One-time extraction of essential domain data
domain_data_list = BulkDataManager.extract_domain_data_list(
    problems=problems,
    discretizations=discretizations,
    static_condensations=static_condensations
)

# Can create multiple lean managers with same extracted data
lean_manager_1 = BulkDataManager(domain_data_list)
lean_manager_2 = BulkDataManager(domain_data_list)
\end{lstlisting}

\paragraph{\_extract\_single\_domain\_data()}\leavevmode
\begin{lstlisting}[language=Python, caption=Single Domain Extraction Method]
@staticmethod
def _extract_single_domain_data(problem, discretization, sc, domain_idx: int) -> DomainData
\end{lstlisting}

\textbf{Parameters:}
\begin{itemize}
    \item \texttt{problem}: Problem object for single domain
    \item \texttt{discretization}: Discretization object for single domain
    \item \texttt{sc}: Static condensation object for single domain
    \item \texttt{domain\_idx}: Domain index for error reporting
\end{itemize}

\textbf{Returns:} \texttt{DomainData} - Extracted essential data for single domain

\textbf{Extraction Process:}
\begin{enumerate}
    \item Extract matrices from static condensation (\texttt{M}, \texttt{T})
    \item Extract initial conditions with multiple access pattern support
    \item Extract forcing functions with multiple access pattern support
    \item Create DomainData object with essential information
\end{enumerate}

\subsection{BulkData Creation Methods}
\label{subsec_bulk_data_creation}

\paragraph{create\_bulk\_data()}\leavevmode
\begin{lstlisting}[language=Python, caption=Create BulkData Method]
def create_bulk_data(self, 
                    domain_index: int, 
                    problem, 
                    discretization, 
                    dual: bool = False) -> BulkData
\end{lstlisting}

\textbf{Parameters:}
\begin{itemize}
    \item \texttt{domain\_index}: Index of the domain (0 to \texttt{n\_domains-1})
    \item \texttt{problem}: Problem object for this domain
    \item \texttt{discretization}: Discretization object for this domain
    \item \texttt{dual}: Whether to use dual formulation (default: False)
\end{itemize}

\textbf{Returns:} \texttt{BulkData} - New BulkData object

\textbf{Raises:} \texttt{ValueError} for invalid domain index

\textbf{Usage:}
\begin{lstlisting}[language=Python, caption=BulkData Creation Usage]
# Create primal BulkData for domain 0
bulk_data_primal = lean_manager.create_bulk_data(
    domain_index=0,
    problem=problems[0],
    discretization=discretizations[0],
    dual=False
)

# Create dual BulkData for forcing terms
bulk_data_dual = lean_manager.create_bulk_data(
    domain_index=0,
    problem=problems[0],
    discretization=discretizations[0],
    dual=True
)
\end{lstlisting}

\subsection{Bulk Operations Methods}
\label{subsec:bulk_operations}

\paragraph{compute\_source\_terms()}\leavevmode
\begin{lstlisting}[language=Python, caption=Compute Source Terms Method]
def compute_source_terms(self,
                        problems: List,
                        discretizations: List,
                        time: float) -> List[BulkData]
\end{lstlisting}

\textbf{Parameters:}
\begin{itemize}
    \item \texttt{problems}: List of Problem objects
    \item \texttt{discretizations}: List of discretization objects
    \item \texttt{time}: Current time for evaluation
\end{itemize}

\textbf{Returns:} \texttt{List[BulkData]} - Source terms for all domains using dual formulation

\textbf{Process:}
\begin{enumerate}
    \item Validate framework objects against stored domain data
    \item Create dual BulkData for each domain
    \item Set forcing functions from problems at specified time
    \item Return list of integrated source terms
\end{enumerate}

\textbf{Usage:}
\begin{lstlisting}[language=Python, caption=Source Terms Usage]
# Compute source terms at time t=0.5
source_terms = lean_manager.compute_source_terms(
    problems=problems,
    discretizations=discretizations,
    time=0.5
)

print(f"Computed {len(source_terms)} source term BulkData objects")
\end{lstlisting}

\paragraph{compute\_forcing\_terms()}\leavevmode
\begin{lstlisting}[language=Python, caption=Compute Forcing Terms Method]
def compute_forcing_terms(self, 
                          bulk_data_list: List[BulkData],
                          problems: List,
                          discretizations: List, 
                          time: float, 
                          dt: float) -> List[np.ndarray]
\end{lstlisting}

\textbf{Parameters:}
\begin{itemize}
    \item \texttt{bulk\_data\_list}: List of current BulkData solutions
    \item \texttt{problems}: List of Problem objects
    \item \texttt{discretizations}: List of discretization objects
    \item \texttt{time}: Current time
    \item \texttt{dt}: Time step size
\end{itemize}

\textbf{Returns:} \texttt{List[np.ndarray]} - Forcing term arrays for implicit Euler

\textbf{Computation:} For each domain: \texttt{forcing\_term = dt * force\_contrib + M * U\_old}

\textbf{Usage:}
\begin{lstlisting}[language=Python, caption=Forcing Terms Usage]
# Compute forcing terms for implicit Euler step
current_bulk_data = [...]  # Current solution
forcing_terms = lean_manager.compute_forcing_terms(
    bulk_data_list=current_bulk_data,
    problems=problems,
    discretizations=discretizations,
    time=0.1,
    dt=0.01
)

# forcing_terms[i] has shape (2*neq, n_elements) for domain i
\end{lstlisting}

\subsection{Initialization Methods}
\label{subsec:initialization_methods}

\paragraph{initialize\_all\_bulk\_data()}\leavevmode
\begin{lstlisting}[language=Python, caption=Initialize All BulkData Method]
def initialize_all_bulk_data(self, 
                            problems: List,
                            discretizations: List,
                            time: float = 0.0) -> List[BulkData]
\end{lstlisting}

\textbf{Parameters:}
\begin{itemize}
    \item \texttt{problems}: List of Problem objects
    \item \texttt{discretizations}: List of discretization objects
    \item \texttt{time}: Initial time (default: 0.0)
\end{itemize}

\textbf{Returns:} \texttt{List[BulkData]} - Initialized BulkData objects for all domains

\textbf{Usage:}
\begin{lstlisting}[language=Python, caption=Initialize All Usage]
# Initialize all domains with initial conditions
initial_bulk_data = lean_manager.initialize_all_bulk_data(
    problems=problems,
    discretizations=discretizations,
    time=0.0
)

print(f"Initialized {len(initial_bulk_data)} BulkData objects")
\end{lstlisting}

\paragraph{initialize\_bulk\_data\_from\_initial\_conditions()}\leavevmode
\begin{lstlisting}[language=Python, caption=Initialize Single Domain Method]
def initialize_bulk_data_from_initial_conditions(self, 
                                                domain_index: int,
                                                problem,
                                                discretization,
                                                time: float = 0.0) -> BulkData
\end{lstlisting}

\textbf{Parameters:}
\begin{itemize}
    \item \texttt{domain\_index}: Index of domain to initialize
    \item \texttt{problem}: Problem object for this domain
    \item \texttt{discretization}: Discretization object for this domain
    \item \texttt{time}: Initial time (default: 0.0)
\end{itemize}

\textbf{Returns:} \texttt{BulkData} - Initialized BulkData object

\textbf{Process:}
\begin{enumerate}
    \item Validate domain index and framework objects
    \item Create BulkData with primal formulation
    \item Set initial conditions from stored domain data
    \item Default to zero if no initial conditions available
\end{enumerate}

\textbf{Usage:}
\begin{lstlisting}[language=Python, caption=Initialize Single Domain Usage]
# Initialize specific domain
domain_0_bulk = lean_manager.initialize_bulk_data_from_initial_conditions(
    domain_index=0,
    problem=problems[0],
    discretization=discretizations[0],
    time=0.0
)
\end{lstlisting}

\subsection{Data Management Methods}
\label{subsec:data_management}

\paragraph{update\_bulk\_data()}\leavevmode
\begin{lstlisting}[language=Python, caption=Update BulkData Method]
def update_bulk_data(self, bulk_data_list: List[BulkData], new_data_list: List[np.ndarray])
\end{lstlisting}

\textbf{Parameters:}
\begin{itemize}
    \item \texttt{bulk\_data\_list}: List of BulkData objects to update
    \item \texttt{new\_data\_list}: List of new bulk solution arrays
\end{itemize}

\textbf{Returns:} \texttt{None}

\textbf{Side Effects:} Updates data in all BulkData objects

\textbf{Validation:}
\begin{itemize}
    \item Lists must have matching lengths
    \item New data must have compatible shapes
    \item New data must not contain NaN or infinite values
\end{itemize}

\textbf{Usage:}
\begin{lstlisting}[language=Python, caption=Update BulkData Usage]
# Update bulk data with new solution
new_solutions = [...]  # List of numpy arrays
lean_manager.update_bulk_data(
    bulk_data_list=current_bulk_data,
    new_data_list=new_solutions
)
\end{lstlisting}

\paragraph{get\_bulk\_data\_arrays()}\leavevmode
\begin{lstlisting}[language=Python, caption=Get Data Arrays Method]
def get_bulk_data_arrays(self, bulk_data_list: List[BulkData]) -> List[np.ndarray]
\end{lstlisting}

\textbf{Parameters:}
\begin{itemize}
    \item \texttt{bulk\_data\_list}: List of BulkData objects
\end{itemize}

\textbf{Returns:} \texttt{List[np.ndarray]} - Data arrays from all BulkData objects

\textbf{Usage:}
\begin{lstlisting}[language=Python, caption=Get Arrays Usage]
# Extract data arrays for external processing
data_arrays = lean_manager.get_bulk_data_arrays(bulk_data_list)
for i, array in enumerate(data_arrays):
    print(f"Domain {i} data shape: {array.shape}")
\end{lstlisting}

\subsection{Analysis Methods}
\label{subsec:analysis_methods}

\paragraph{compute\_total\_mass()}\leavevmode
\begin{lstlisting}[language=Python, caption=Compute Total Mass Method]
def compute_total_mass(self, bulk_data_list: List[BulkData]) -> float
\end{lstlisting}

\textbf{Parameters:}
\begin{itemize}
    \item \texttt{bulk\_data\_list}: List of BulkData instances
\end{itemize}

\textbf{Returns:} \texttt{float} - Total mass across all domains

\textbf{Computation:} Sums mass from each domain using stored mass matrices

\textbf{Usage:}
\begin{lstlisting}[language=Python, caption=Mass Computation Usage]
# Monitor mass conservation
initial_mass = lean_manager.compute_total_mass(initial_bulk_data)
current_mass = lean_manager.compute_total_mass(current_bulk_data)

mass_change = abs(current_mass - initial_mass) / initial_mass
print(f"Relative mass change: {mass_change:.6e}")
\end{lstlisting}

\paragraph{compute\_mass\_conservation()}\leavevmode
\begin{lstlisting}[language=Python, caption=Mass Conservation Method]
def compute_mass_conservation(self, bulk_data_list: List[BulkData]) -> float
\end{lstlisting}

\textbf{Parameters:}
\begin{itemize}
    \item \texttt{bulk\_data\_list}: List of BulkData instances
\end{itemize}

\textbf{Returns:} \texttt{float} - Total mass (alias for \texttt{compute\_total\_mass})

\textbf{Note:} This method is an alias for consistency with other interfaces

\subsection{Utility Methods}
\label{subsec:utility_methods}

\paragraph{get\_num\_domains()}\leavevmode
\begin{lstlisting}[language=Python, caption=Get Number of Domains Method]
def get_num_domains(self) -> int
\end{lstlisting}

\textbf{Returns:} \texttt{int} - Number of domains managed

\textbf{Usage:}
\begin{lstlisting}[language=Python, caption=Get Domains Count Usage]
n_domains = lean_manager.get_num_domains()
print(f"Managing {n_domains} domains")
\end{lstlisting}

\paragraph{get\_domain\_info()}\leavevmode
\begin{lstlisting}[language=Python, caption=Get Domain Info Method]
def get_domain_info(self, domain_idx: int) -> DomainData
\end{lstlisting}

\textbf{Parameters:}
\begin{itemize}
    \item \texttt{domain\_idx}: Domain index
\end{itemize}

\textbf{Returns:} \texttt{DomainData} - Domain data object for inspection

\textbf{Raises:} \texttt{IndexError} for invalid domain index

\textbf{Usage:}
\begin{lstlisting}[language=Python, caption=Domain Info Usage]
# Inspect domain properties
domain_info = lean_manager.get_domain_info(0)
print(f"Domain 0: {domain_info.neq} equations, {domain_info.n_elements} elements")
print(f"Element length: {domain_info.element_length}")
\end{lstlisting}

\subsection{Testing and Validation}
\label{subsec:testing_validation}

\paragraph{test()}\leavevmode
\begin{lstlisting}[language=Python, caption=Test Method]
def test(self, 
         problems: List = None,
         discretizations: List = None,
         static_condensations: List = None) -> bool
\end{lstlisting}

\textbf{Parameters:}
\begin{itemize}
    \item \texttt{problems}: List of Problem objects for testing (optional)
    \item \texttt{discretizations}: List of discretization objects for testing (optional)
    \item \texttt{static\_condensations}: List of static condensation objects for testing (optional)
\end{itemize}

\textbf{Returns:} \texttt{bool} - True if all tests pass, False otherwise

\textbf{Test Suite:}
\begin{enumerate}
    \item \textbf{Framework Object Validation Test}: Validates provided framework objects
    \item \textbf{Domain Data Structure Test}: Validates stored domain data integrity
    \item \textbf{BulkData Creation Test}: Tests creation of primal and dual BulkData
    \item \textbf{Initialization Test}: Tests bulk data initialization
    \item \textbf{Forcing Term Computation Test}: Tests forcing term calculations
    \item \textbf{Mass Computation Test}: Tests mass conservation calculations
    \item \textbf{Bounds Checking Test}: Tests error handling for invalid indices
    \item \textbf{Utility Methods Test}: Tests helper methods
    \item \textbf{Parameter Mismatch Test}: Tests validation error detection
\end{enumerate}

\textbf{Usage:}
\begin{lstlisting}[language=Python, caption=Test Method Usage]
# Comprehensive testing with framework objects
if lean_manager.test(
    problems=problems,
    discretizations=discretizations,
    static_condensations=static_condensations
):
    print("✓ Lean BulkDataManager is fully functional")
else:
    print("✗ Issues detected in Lean BulkDataManager")

# Minimal testing without framework objects
if lean_manager.test():
    print("✓ Basic structure validation passed")
\end{lstlisting}

\textbf{Sample Test Output:}
\begin{lstlisting}[language=Python, caption=Sample Test Output]
Testing Lean BulkDataManager with 3 domains
PASS: Framework object validation passed
PASS: All domain data validated
PASS: BulkData creation tests passed
PASS: Initialization tests passed
PASS: Forcing term computation tests passed
PASS: Mass computation test passed (total_mass=1.234567e+00)
PASS: ValueError raised for negative domain index
PASS: get_num_domains() returned correct value
PASS: get_domain_info() test passed
PASS: Correctly detected wrong number of problems
PASS: Correctly detected incompatible problem neq
PASS: Parameter mismatch detection tests passed
✓ All Lean BulkDataManager tests passed!
\end{lstlisting}

\subsection{Special Methods}
\label{subsec:special_methods}

\paragraph{\_\_str\_\_()}\leavevmode
\begin{lstlisting}[language=Python, caption=String Representation Method]
def __str__(self) -> str
\end{lstlisting}

\textbf{Returns:} \texttt{str} - Human-readable summary

\textbf{Format:} \texttt{"LeanBulkDataManager(domains=N, total\_elements=M, total\_equations=K)"}

\paragraph{\_\_repr\_\_()}\leavevmode
\begin{lstlisting}[language=Python, caption=Repr Method]
def __repr__(self) -> str
\end{lstlisting}

\textbf{Returns:} \texttt{str} - Developer-oriented representation

\textbf{Format:} \texttt{"LeanBulkDataManager(n\_domains=N, domain\_elements=[...], domain\_equations=[...])"}

\textbf{Usage:}
\begin{lstlisting}[language=Python, caption=String Methods Usage]
print(str(lean_manager))
# Output: LeanBulkDataManager(domains=3, total_elements=60, total_equations=6)

print(repr(lean_manager))
# Output: LeanBulkDataManager(n_domains=3, domain_elements=[20, 20, 20], domain_equations=[2, 2, 2])
\end{lstlisting}

\subsection{Complete Usage Examples}
\label{subsec:complete_usage_examples}

\subsubsection{Standard Workflow Example}

\begin{lstlisting}[language=Python, caption=Complete Lean Manager Workflow]
from ooc1d.core.lean_bulk_data_manager import BulkDataManager
from ooc1d.core.problem import Problem
from ooc1d.core.discretization import Discretization
import numpy as np

# Step 1: Create framework objects (problems, discretizations, static_condensations)
problems = [...]  # List of Problem instances
discretizations = [...]  # List of Discretization instances  
static_condensations = [...]  # List of static condensation instances

# Step 2: Extract essential data once (memory-efficient)
domain_data_list = BulkDataManager.extract_domain_data_list(
    problems=problems,
    discretizations=discretizations,
    static_condensations=static_condensations
)

# Step 3: Create lean manager with extracted data only
lean_manager = BulkDataManager(domain_data_list)

# Step 4: Validate compatibility (optional but recommended)
if not lean_manager.test(problems, discretizations, static_condensations):
    raise RuntimeError("Framework objects incompatible with extracted data")

# Step 5: Initialize bulk data for all domains
bulk_data_list = lean_manager.initialize_all_bulk_data(
    problems=problems,
    discretizations=discretizations,
    time=0.0
)

# Step 6: Time evolution loop
dt = 0.01
for time_step in range(100):
    current_time = time_step * dt
    
    # Compute forcing terms for implicit Euler
    forcing_terms = lean_manager.compute_forcing_terms(
        bulk_data_list=bulk_data_list,
        problems=problems,
        discretizations=discretizations,
        time=current_time,
        dt=dt
    )
    
    # Solve system (external solver)
    new_solutions = solve_system(forcing_terms)  # User-defined solver
    
    # Update bulk data with new solutions
    lean_manager.update_bulk_data(bulk_data_list, new_solutions)
    
    # Monitor mass conservation
    current_mass = lean_manager.compute_total_mass(bulk_data_list)
    if time_step % 10 == 0:
        print(f"Time {current_time:.3f}: Mass = {current_mass:.6e}")

print("✓ Time evolution completed with lean manager")
\end{lstlisting}

\subsubsection{Memory Comparison Example}

\begin{lstlisting}[language=Python, caption=Memory Usage Comparison]
import psutil
import os

# Measure memory before
process = psutil.Process(os.getpid())
memory_before = process.memory_info().rss / 1024 / 1024  # MB

# Traditional approach (stores framework objects)
# traditional_manager = FullBulkDataManager(problems, discretizations, static_condensations)

# Lean approach (stores only essential data)
domain_data_list = BulkDataManager.extract_domain_data_list(
    problems, discretizations, static_condensations
)
lean_manager = BulkDataManager(domain_data_list)

# Measure memory after
memory_after = process.memory_info().rss / 1024 / 1024  # MB
memory_used = memory_after - memory_before

print(f"Memory used by lean manager: {memory_used:.2f} MB")
print(f"Domains managed: {lean_manager.get_num_domains()}")
print(f"Memory per domain: {memory_used / lean_manager.get_num_domains():.2f} MB")
\end{lstlisting}

\subsubsection{Multi-Manager Example}

\begin{lstlisting}[language=Python, caption=Multiple Lean Managers from Same Data]
# Extract domain data once
domain_data_list = BulkDataManager.extract_domain_data_list(
    problems, discretizations, static_condensations
)

# Create multiple lean managers for different purposes
# (all sharing the same extracted data - no additional memory cost)

# Manager for time evolution
evolution_manager = BulkDataManager(domain_data_list)

# Manager for forcing term computation
forcing_manager = BulkDataManager(domain_data_list)

# Manager for mass conservation tracking  
conservation_manager = BulkDataManager(domain_data_list)

# Each manager can operate independently but uses same base data
initial_bulk = evolution_manager.initialize_all_bulk_data(problems, discretizations)
source_terms = forcing_manager.compute_source_terms(problems, discretizations, time=0.0)
total_mass = conservation_manager.compute_total_mass(initial_bulk)

print(f"Created 3 independent managers sharing {len(domain_data_list)} domain data objects")
\end{lstlisting}

\subsection{Method Summary Table}
\label{subsec:lean_method_summary}

\begin{longtable}{|p{5.3cm}|p{3.2cm}|p{5cm}|}
\hline
\textbf{Method} & \textbf{Returns} & \textbf{Purpose} \\
\hline
\endhead

\texttt{\_\_init\_\_} & \texttt{None} & Initialize with extracted domain data only \\
\hline

\texttt{extract\_domain\_data\_list} & \texttt{List[DomainData]} & Static factory for one-time data extraction \\
\hline

\texttt{create\_bulk\_data} & \texttt{BulkData} & Create BulkData using external framework objects \\
\hline

\texttt{initialize\_all\_bulk\_data} & \texttt{List[BulkData]} & Initialize all domains with initial conditions \\
\hline

\texttt{compute\_source\_terms} & \texttt{List[BulkData]} & Compute source terms using dual formulation \\
\hline

\texttt{compute\_forcing\_terms} & \texttt{List[np.ndarray]} & Compute forcing terms for implicit Euler \\
\hline

\texttt{update\_bulk\_data} & \texttt{None} & Update BulkData objects with new solutions \\
\hline

\texttt{compute\_total\_mass} & \texttt{float} & Calculate total mass for conservation \\
\hline

\texttt{get\_bulk\_data\_arrays} & \texttt{List[np.ndarray]} & Extract data arrays from BulkData objects \\
\hline

\texttt{get\_num\_domains} & \texttt{int} & Get number of managed domains \\
\hline

\texttt{get\_domain\_info} & \texttt{DomainData} & Access domain data for inspection \\
\hline

\texttt{test} & \texttt{bool} & Comprehensive validation and testing \\
\hline

\texttt{\_validate\_framework\_objects} & \texttt{None} & Validate framework object compatibility \\
\hline

\end{longtable}

This documentation provides an exact reference for the lean BulkDataManager class, emphasizing its memory-efficient design and parameter-based approach to framework object usage. The lean architecture minimizes memory overhead while maintaining full functionality through external object validation and flexible method interfaces.

% End of lean bulk data manager module API documentation
