\documentclass[11pt,a4paper]{article}
\usepackage[utf8]{inputenc}
\usepackage[english]{babel}
\usepackage{amsmath,amsfonts,amssymb}
\usepackage{graphicx}
\usepackage{geometry}
\usepackage{fancyhdr}
\usepackage{listings}
\usepackage{xcolor}
\usepackage{hyperref}
\usepackage{tocloft}
\usepackage{titlesec}
\usepackage{float}
\usepackage{booktabs}
\usepackage{array}
\usepackage{longtable}
\usepackage{mathtools}
\usepackage{bm}

% Page setup
\geometry{margin=2.5cm}
\pagestyle{fancy}
\fancyhf{}
\fancyhead[L]{\textsc{Mathematical Background}}
\fancyhead[R]{\thepage}
\fancyfoot[C]{\textit{BioNetFlux Applications}}

% Hyperlink setup
\hypersetup{
    colorlinks=true,
    linkcolor=blue,
    filecolor=magenta,      
    urlcolor=cyan,
    pdftitle={Mathematical Background for BioNetFlux Applications},
    pdfauthor={BioNetFlux Development Team},
}

% Code listing setup
\definecolor{codegreen}{rgb}{0,0.6,0}
\definecolor{codegray}{rgb}{0.5,0.5,0.5}
\definecolor{codepurple}{rgb}{0.58,0,0.82}
\definecolor{backcolour}{rgb}{0.95,0.95,0.92}

\lstdefinestyle{mystyle}{
    backgroundcolor=\color{backcolour},   
    commentstyle=\color{codegreen},
    keywordstyle=\color{magenta},
    numberstyle=\tiny\color{codegray},
    stringstyle=\color{codepurple},
    basicstyle=\ttfamily\footnotesize,
    breakatwhitespace=false,         
    breaklines=true,                 
    captionpos=b,                    
    keepspaces=true,                 
    numbers=left,                    
    numbersep=5pt,                  
    showspaces=false,                
    showstringspaces=false,
    showtabs=false,                  
    tabsize=2
}

\lstset{style=mystyle}

% Custom commands
\newcommand{\code}[1]{\texttt{#1}}
\newcommand{\bionetflux}{\textsc{BioNetFlux}}

% Title page customization
\title{\Huge \textbf{Mathematical Background} \\[0.5cm]
       \Large for \bionetflux{} Applications}
\author{BioNetFlux Development Team}
\date{\today}

\begin{document}

% Title page
\begin{titlepage}
    \centering
    
    % BioNetFlux Logo
    \includegraphics[width=0.6\textwidth]{../Logos/BioNetFlux.png}\\[1cm]
    
    {\Huge \textbf{Mathematical Background} \\[0.5cm]}
    {\Large \textbf{for \bionetflux{} Applications} \\[1cm]}
    
    {\large Comprehensive Mathematical Foundation \\[0.5cm]}
    {\large Keller-Segel Chemotaxis and Organ-on-Chip Systems \\[2cm]}
    
    {\Large BioNetFlux Development Team \\[0.5cm]}
    {\large \today}
    
    \vfill
    
    {\footnotesize 
    \textit{Mathematical theory and numerical methods} \\
    \textit{for multi-domain biological transport simulations}
    }
        
    \vskip3cm
        
    % Barra bar
    \includegraphics[width=0.8\textwidth]{../Logos/Barra.png}\\[2cm]
\end{titlepage}

% Table of contents
\tableofcontents
\clearpage

\section{Introduction}

This document provides comprehensive mathematical background for the two primary application domains of \bionetflux{}: Keller-Segel chemotaxis models and organ-on-chip transport systems. The mathematical formulations, physical interpretations, and numerical methods presented here form the theoretical foundation for understanding and implementing these models in the \bionetflux{} framework.

\section{Keller-Segel Chemotaxis Model}

\subsection{Overview}

The Keller-Segel model describes the movement of cells in response to chemical gradients, a phenomenon known as chemotaxis. First proposed by Keller and Segel in 1970, this system has become fundamental in mathematical biology for modeling bacterial aggregation, tumor invasion, and immune cell migration.

\subsection{Governing Equations}

The classical Keller-Segel system in one dimension consists of two coupled partial differential equations:

\begin{align}
\frac{\partial u}{\partial t} &= D_u \frac{\partial^2 u}{\partial x^2} - \frac{\partial}{\partial x}\left[\chi(u,\varphi) u \frac{\partial \varphi}{\partial x}\right] + f_u(u,\varphi,x,t) \label{eq:ks_cell}\\
\frac{\partial \varphi}{\partial t} &= D_\varphi \frac{\partial^2 \varphi}{\partial x^2} + g(u,\varphi) - \delta\varphi + f_\varphi(u,\varphi,x,t) \label{eq:ks_chemical}
\end{align}

where:
\begin{itemize}
    \item $u(x,t)$ is the cell density
    \item $\varphi(x,t)$ is the chemoattractant concentration
    \item $D_u, D_\varphi > 0$ are diffusion coefficients
    \item $\chi(u,\varphi)$ is the chemotaxis sensitivity function
    \item $g(u,\varphi)$ represents chemoattractant production by cells
    \item $\delta \geq 0$ is the chemoattractant decay rate
    \item $f_u, f_\varphi$ are external source terms
\end{itemize}

\subsection{Physical Interpretation}

\subsubsection{Cell Equation \eqref{eq:ks_cell}}

\begin{itemize}
    \item \textbf{Diffusion term}: $D_u \frac{\partial^2 u}{\partial x^2}$ represents random cell movement (Brownian motion)
    \item \textbf{Chemotactic flux}: $-\frac{\partial}{\partial x}\left[\chi(u,\varphi) u \frac{\partial \varphi}{\partial x}\right]$ describes directed movement along chemical gradients
    \item \textbf{Source term}: $f_u(u,\varphi,x,t)$ accounts for cell proliferation, death, or external injection
\end{itemize}

\subsubsection{Chemical Equation \eqref{eq:ks_chemical}}

\begin{itemize}
    \item \textbf{Diffusion term}: $D_\varphi \frac{\partial^2 \varphi}{\partial x^2}$ represents molecular diffusion of the chemoattractant
    \item \textbf{Production term}: $g(u,\varphi)$ typically linear $g(u,\varphi) = \alpha u$ or saturated $g(u,\varphi) = \frac{\alpha u}{1+\beta u}$
    \item \textbf{Decay term}: $-\delta\varphi$ represents natural degradation
    \item \textbf{Source term}: $f_\varphi(u,\varphi,x,t)$ accounts for external chemical sources
\end{itemize}

\subsection{Chemotaxis Sensitivity Functions}

The chemotaxis sensitivity function $\chi(u,\varphi)$ is crucial for model behavior:

\subsubsection{Constant Sensitivity}
\begin{equation}
\chi(u,\varphi) = \chi_0 = \text{constant}
\end{equation}

\subsubsection{Signal-Dependent Sensitivity}
\begin{equation}
\chi(u,\varphi) = \frac{\chi_0}{(K + \varphi)^n}
\end{equation}
where $\chi_0, K > 0$ and $n \geq 0$ control sensitivity strength and saturation.

\subsubsection{Receptor Kinetics Model}
\begin{equation}
\chi(u,\varphi) = \chi_0 \frac{K}{(K + \varphi)^2}
\end{equation}
derived from receptor binding kinetics, where $K$ is the dissociation constant.

\subsubsection{Log-Sensing Model}
\begin{equation}
\chi(u,\varphi) = \frac{\chi_0}{1 + \varphi}
\end{equation}
captures logarithmic sensing behavior observed in bacterial chemotaxis.

\subsection{Boundary Conditions}

\subsubsection{No-Flux Conditions (Closed System)}
\begin{align}
D_u \frac{\partial u}{\partial x} - \chi(u,\varphi) u \frac{\partial \varphi}{\partial x} &= 0 \quad \text{at boundaries}\\
D_\varphi \frac{\partial \varphi}{\partial x} &= 0 \quad \text{at boundaries}
\end{align}

\subsubsection{Prescribed Flux Conditions}
\begin{align}
D_u \frac{\partial u}{\partial x} - \chi(u,\varphi) u \frac{\partial \varphi}{\partial x} &= j_u(t)\\
D_\varphi \frac{\partial \varphi}{\partial x} &= j_\varphi(t)
\end{align}

\subsubsection{Mixed Conditions}
\begin{align}
u &= u_0(t) \quad \text{at inlet}\\
\frac{\partial u}{\partial x} &= 0 \quad \text{at outlet}\\
\varphi &= \varphi_0(t) \quad \text{at inlet}\\
\frac{\partial \varphi}{\partial x} &= 0 \quad \text{at outlet}
\end{align}

\subsection{Analytical Solutions}

For specific parameter choices, traveling wave solutions exist. Consider:
\begin{itemize}
    \item Constant chemotaxis: $\chi(u,\varphi) = \chi_0$
    \item Linear chemical production: $g(u,\varphi) = \alpha u$
    \item No decay: $\delta = 0$
\end{itemize}

\subsubsection{Traveling Wave Ansatz}
\begin{align}
u(x,t) &= U(\xi), \quad \xi = x - ct\\
\varphi(x,t) &= \Phi(\xi), \quad \xi = x - ct
\end{align}

For the parameter set $D_u = \nu$, $D_\varphi = \mu$, $\chi_0 = 1/\nu$, and wave speed $c = 1/2$, analytical solutions are:

\begin{align}
U(\xi) &= \frac{5e^{\xi/2}}{e^{\xi/2} - 1} - \frac{4e^\xi}{(e^{\xi/2} - 1)^2} - \frac{5}{8}\\
\Phi(\xi) &= \frac{5\xi}{4} - 2\ln(e^{\xi/2} - 1)
\end{align}

These solutions are used in \bionetflux{} for validation and testing.

\subsection{Dimensionless Analysis}

\subsubsection{Characteristic Scales}
\begin{itemize}
    \item Length: $L$ (domain size)
    \item Time: $T = L^2/D_u$ (diffusion time)
    \item Cell density: $U_0$ (initial density)
    \item Chemical concentration: $\Phi_0$ (reference concentration)
\end{itemize}

\subsubsection{Dimensionless Variables}
\begin{equation}
x^* = \frac{x}{L}, \quad t^* = \frac{t}{T}, \quad u^* = \frac{u}{U_0}, \quad \varphi^* = \frac{\varphi}{\Phi_0}
\end{equation}

\subsubsection{Dimensionless Parameters}
\begin{align}
\text{Pe}_c &= \frac{\chi_0 \Phi_0 L}{D_u} \quad \text{(chemotactic Péclet number)}\\
D &= \frac{D_\varphi}{D_u} \quad \text{(diffusivity ratio)}\\
\alpha^* &= \frac{\alpha U_0 T}{\Phi_0} \quad \text{(production parameter)}\\
\delta^* &= \delta T \quad \text{(decay parameter)}
\end{align}

\subsubsection{Dimensionless Equations}
\begin{align}
\frac{\partial u^*}{\partial t^*} &= \frac{\partial^2 u^*}{\partial x^{*2}} - \text{Pe}_c \frac{\partial}{\partial x^*}\left[\chi^*(u^*,\varphi^*) u^* \frac{\partial \varphi^*}{\partial x^*}\right] + f_u^*\\
\frac{\partial \varphi^*}{\partial t^*} &= D \frac{\partial^2 \varphi^*}{\partial x^{*2}} + \alpha^* g^*(u^*,\varphi^*) - \delta^* \varphi^* + f_\varphi^*
\end{align}

\subsection{Model Variants}

\subsubsection{Volume-Filling Effect}
\begin{equation}
\frac{\partial u}{\partial t} = \nabla \cdot \left[D_u\left(1-\frac{u}{u_{\max}}\right)\nabla u - \chi(\varphi)u\left(1-\frac{u}{u_{\max}}\right)\nabla\varphi\right] + f_u
\end{equation}

\subsubsection{Cross-Diffusion}
\begin{equation}
\frac{\partial u}{\partial t} = \nabla \cdot \left[D_u\nabla u + D_{u\varphi}u\nabla\varphi - \chi(\varphi)u\nabla\varphi\right] + f_u
\end{equation}

\subsubsection{Multiple Species}
\begin{align}
\frac{\partial u_i}{\partial t} &= D_i\nabla^2 u_i - \nabla \cdot [\chi_i(\varphi)u_i\nabla\varphi] + R_i(u_1,\ldots,u_n,\varphi)\\
\frac{\partial \varphi}{\partial t} &= D_\varphi\nabla^2\varphi + \sum_i g_i(u_i,\varphi) - \delta\varphi
\end{align}

\section{Organ-on-Chip Transport Model}

\subsection{Overview}

Organ-on-chip (OoC) systems are microfluidic devices that simulate human organ functions. The mathematical model combines fluid flow, species transport, and cellular interactions within multi-compartment geometries.

\subsection{Governing Equations}

The OoC transport model consists of coupled advection-diffusion-reaction equations:

\begin{align}
\frac{\partial c_i}{\partial t} + \bm{v} \cdot \nabla c_i &= D_i \nabla^2 c_i + R_i(\bm{c},\bm{\varphi}) + S_i(\bm{x},t) \label{eq:ooc_transport}\\
\frac{\partial \varphi_j}{\partial t} &= D_{\varphi_j} \nabla^2 \varphi_j + P_j(\bm{c},\bm{\varphi}) - \delta_j \varphi_j + Q_j(\bm{x},t) \label{eq:ooc_signaling}
\end{align}

where:
\begin{itemize}
    \item $c_i(\bm{x},t)$ represents concentration of species $i$ (nutrients, drugs, metabolites)
    \item $\varphi_j(\bm{x},t)$ represents signaling molecules or cellular markers
    \item $\bm{v}(\bm{x})$ is the fluid velocity field
    \item $D_i, D_{\varphi_j}$ are diffusion coefficients
    \item $R_i(\bm{c},\bm{\varphi})$ represents reaction kinetics for species $i$
    \item $P_j(\bm{c},\bm{\varphi})$ represents production of signaling molecules
    \item $\delta_j$ is degradation rate of signaling molecule $j$
    \item $S_i, Q_j$ are external source/sink terms
\end{itemize}

\subsection{Multi-Compartment Structure}

OoC devices typically consist of multiple interconnected compartments:

\subsubsection{Flow Channel (High Flow Rate, Advection-Dominated)}
\begin{equation}
\frac{\partial c}{\partial t} + v \frac{\partial c}{\partial x} = D \frac{\partial^2 c}{\partial x^2} + S(x,t)
\end{equation}

Péclet number: $\text{Pe} = \frac{vL}{D} \gg 1$

\subsubsection{Cell Culture Chamber (Low Flow, Reaction-Dominated)}
\begin{equation}
\frac{\partial c}{\partial t} = D \frac{\partial^2 c}{\partial x^2} + R(c,\varphi) + S(x,t)
\end{equation}

Damköhler number: $\text{Da} = \frac{kL^2}{D} \gg 1$

\subsubsection{Membrane Interface (Selective Permeability)}
\begin{equation}
J = P(c_1 - c_2) + \sigma\varphi \frac{\partial c}{\partial x}
\end{equation}
where $P$ is permeability and $\sigma$ is reflection coefficient.

\subsection{Cellular Reaction Kinetics}

\subsubsection{Michaelis-Menten Kinetics}
\begin{equation}
R(c) = -\frac{V_{\max} c}{K_m + c}
\end{equation}
Parameters: $V_{\max}$ (maximum rate), $K_m$ (Michaelis constant)

\subsubsection{Hill Kinetics (Cooperative Binding)}
\begin{equation}
R(c) = -\frac{V_{\max} c^n}{K^n + c^n}
\end{equation}
Parameters: $n$ (Hill coefficient), $K$ (half-saturation constant)

\subsubsection{Competitive Inhibition}
\begin{equation}
R(c,I) = -\frac{V_{\max} c}{(K_m + c)\left(1 + \frac{I}{K_i}\right)}
\end{equation}
Parameters: $I$ (inhibitor concentration), $K_i$ (inhibition constant)

\subsubsection{Non-Competitive Inhibition}
\begin{equation}
R(c,I) = -\frac{V_{\max} c}{(K_m + c)\left(1 + \frac{I}{K_i}\right)}
\end{equation}

\subsubsection{Substrate Inhibition}
\begin{equation}
R(c) = -\frac{V_{\max} c}{K_m + c + \frac{c^2}{K_i}}
\end{equation}

\subsection{Interface Conditions}

At compartment interfaces, various junction conditions are implemented:

\subsubsection{Kedem-Katchalsky Equations (Membrane Transport)}
\begin{align}
J_v &= L_p(\Delta P - \sigma \Delta \pi) \quad \text{(volume flux)}\\
J_s &= \omega \Delta \pi + (1-\sigma) \bar{c} J_v \quad \text{(solute flux)}
\end{align}
where:
\begin{itemize}
    \item $L_p$ is hydraulic permeability
    \item $\omega$ is solute permeability
    \item $\sigma$ is reflection coefficient
    \item $\Delta P, \Delta \pi$ are pressure and osmotic pressure differences
    \item $\bar{c}$ is mean concentration
\end{itemize}

\subsubsection{Simplified Interface Conditions}
\begin{equation}
J_i = P_i(c_{i,1} - c_{i,2}) + \sigma_i \varphi \frac{\partial c_i}{\partial n}
\end{equation}

\subsubsection{Trace Continuity (Perfect Mixing)}
\begin{equation}
c_1 = c_2 \quad \text{at interface}
\end{equation}

\subsubsection{Flux Continuity}
\begin{equation}
D_1 \frac{\partial c_1}{\partial n} = D_2 \frac{\partial c_2}{\partial n} \quad \text{at interface}
\end{equation}

\subsection{Dimensionless Analysis}

\subsubsection{Characteristic Scales}
\begin{itemize}
    \item Length: $L$ (channel length)
    \item Velocity: $V$ (average flow velocity)
    \item Time: $T = L/V$ (convection time)
    \item Concentration: $C_0$ (inlet concentration)
\end{itemize}

\subsubsection{Dimensionless Variables}
\begin{equation}
x^* = \frac{x}{L}, \quad t^* = \frac{t}{T}, \quad c^* = \frac{c}{C_0}, \quad v^* = \frac{v}{V}
\end{equation}

\subsubsection{Dimensionless Parameters}
\begin{align}
\text{Pe} &= \frac{VL}{D} \quad \text{(Péclet number - advection vs diffusion)}\\
\text{Da} &= \frac{kL}{V} \quad \text{(Damköhler number - reaction vs advection)}\\
\text{Re} &= \frac{\rho VL}{\mu} \quad \text{(Reynolds number - inertia vs viscosity)}\\
\text{Sc} &= \frac{\mu}{\rho D} \quad \text{(Schmidt number - momentum vs mass diffusion)}
\end{align}

\subsubsection{Dimensionless Equation}
\begin{equation}
\frac{\partial c^*}{\partial t^*} + \text{Pe} \, \bm{v}^* \cdot \nabla^* c^* = \nabla^{*2} c^* + \text{Da} \, R^*(c^*) + S^*
\end{equation}

\subsection{Typical Parameter Ranges}

\begin{table}[H]
\centering
\begin{tabular}{@{}llll@{}}
\toprule
Parameter & Range & Units & Application \\
\midrule
Channel length & 1--10 & mm & Flow channels \\
Channel width & 10--1000 & μm & Microchannels \\
Cell chamber size & 100--5000 & μm & Culture chambers \\
Flow velocity & 0.1--10 & mm/s & Physiological flow \\
Diffusion coefficient & $10^{-11}$--$10^{-9}$ & m²/s & Small molecules \\
Permeability & $10^{-8}$--$10^{-4}$ & m/s & Membrane transport \\
Reaction rate & $10^{-6}$--$10^{-2}$ & s⁻¹ & Enzymatic reactions \\
Cell density & $10^6$--$10^8$ & cells/mL & Typical cultures \\
\bottomrule
\end{tabular}
\caption{Typical parameter ranges for organ-on-chip systems}
\end{table}

\subsection{Common OoC Applications}

\subsubsection{Drug Screening}
\begin{align}
\frac{\partial c_{\text{drug}}}{\partial t} + \bm{v} \cdot \nabla c_{\text{drug}} &= D\nabla^2 c_{\text{drug}} - k_{\text{uptake}} c_{\text{drug}} + S_{\text{inlet}}\\
\frac{\partial c_{\text{metabolite}}}{\partial t} &= D_m\nabla^2 c_{\text{metabolite}} + k_{\text{metabolism}} c_{\text{drug}} - k_{\text{clearance}} c_{\text{metabolite}}
\end{align}

\subsubsection{Barrier Function Studies}
\begin{align}
\frac{\partial c}{\partial t} &= D\nabla^2 c + R(c,\text{TEER})\\
\frac{d\text{TEER}}{dt} &= f(c_{\text{inflammatory}}, c_{\text{protective}})
\end{align}

\subsubsection{Angiogenesis Models}
\begin{align}
\frac{\partial c_{\text{VEGF}}}{\partial t} &= D\nabla^2 c_{\text{VEGF}} + \alpha_{\text{production}} - \delta c_{\text{VEGF}}\\
\frac{\partial \rho_{\text{vessel}}}{\partial t} &= k_{\text{sprouting}} \nabla \cdot (\rho_{\text{vessel}} \nabla c_{\text{VEGF}}) + k_{\text{growth}} \rho_{\text{vessel}}
\end{align}

\section{Numerical Methods}

\subsection{Finite Element Discretization}

\bionetflux{} employs linear finite elements for spatial discretization. For a generic transport equation:
\begin{equation}
\frac{\partial u}{\partial t} = \nabla \cdot (D\nabla u) + \bm{v} \cdot \nabla u + R(u) + S
\end{equation}

\subsubsection{Weak Form}
\begin{equation}
\int_\Omega \frac{\partial u}{\partial t} w \, d\Omega = -\int_\Omega D\nabla u \cdot \nabla w \, d\Omega + \int_\Omega (\bm{v} \cdot \nabla u + R + S) w \, d\Omega + \int_{\partial\Omega} \text{flux} \cdot w \, d\Gamma
\end{equation}

\subsubsection{Matrix Form}
\begin{equation}
\bm{M} \frac{d\bm{u}}{dt} + \bm{K} \bm{u} = \bm{f}
\end{equation}
where $\bm{M}$ is mass matrix, $\bm{K}$ is stiffness matrix, $\bm{f}$ is load vector.

\subsection{Time Integration}

\subsubsection{Backward Euler (Implicit)}
\begin{equation}
\bm{M} \frac{\bm{u}^{n+1} - \bm{u}^n}{\Delta t} + \bm{K} \bm{u}^{n+1} = \bm{f}^{n+1}
\end{equation}

\textbf{Advantages:}
\begin{itemize}
    \item Unconditionally stable
    \item Suitable for stiff problems
    \item Handles large time steps
\end{itemize}

\subsubsection{Crank-Nicolson (Semi-Implicit)}
\begin{equation}
\bm{M} \frac{\bm{u}^{n+1} - \bm{u}^n}{\Delta t} + \bm{K} \frac{\bm{u}^{n+1} + \bm{u}^n}{2} = \frac{\bm{f}^{n+1} + \bm{f}^n}{2}
\end{equation}

\subsection{Newton-Raphson Method}

For nonlinear problems, Newton-Raphson iteration:
\begin{align}
\bm{R}(\bm{u}) &= \bm{M} \frac{\bm{u} - \bm{u}^n}{\Delta t} + \bm{K} \bm{u} - \bm{f} = \bm{0}\\
\bm{J} \Delta \bm{u} &= -\bm{R}(\bm{u}^k)\\
\bm{u}^{k+1} &= \bm{u}^k + \Delta \bm{u}
\end{align}
where $\bm{J} = \frac{\partial \bm{R}}{\partial \bm{u}}$ is the Jacobian matrix.

\subsubsection{Convergence Criteria}
\begin{align}
\|\bm{R}(\bm{u}^k)\| &< \text{tol}_{\text{abs}}\\
\frac{\|\Delta \bm{u}\|}{\|\bm{u}^k\|} &< \text{tol}_{\text{rel}}
\end{align}

\subsection{Static Condensation}

To reduce computational cost, interior degrees of freedom are eliminated:
\begin{equation}
\begin{pmatrix}
\bm{K}_{ii} & \bm{K}_{ib} \\
\bm{K}_{bi} & \bm{K}_{bb}
\end{pmatrix}
\begin{pmatrix}
\bm{u}_i \\
\bm{u}_b
\end{pmatrix}
=
\begin{pmatrix}
\bm{f}_i \\
\bm{f}_b
\end{pmatrix}
\end{equation}

\subsubsection{After Condensation}
\begin{align}
\tilde{\bm{K}}_{bb} \bm{u}_b &= \tilde{\bm{f}}_b\\
\tilde{\bm{K}}_{bb} &= \bm{K}_{bb} - \bm{K}_{bi} \bm{K}_{ii}^{-1} \bm{K}_{ib}\\
\tilde{\bm{f}}_b &= \bm{f}_b - \bm{K}_{bi} \bm{K}_{ii}^{-1} \bm{f}_i
\end{align}

\subsubsection{Recovery}
\begin{equation}
\bm{u}_i = \bm{K}_{ii}^{-1}(\bm{f}_i - \bm{K}_{ib} \bm{u}_b)
\end{equation}

\section{Implementation Notes}

\subsection{Stability Considerations}

\subsubsection{CFL Condition for Advection}
\begin{equation}
\Delta t \leq \frac{\Delta x}{|\bm{v}|}
\end{equation}

\subsubsection{Diffusion Stability}
\begin{equation}
\Delta t \leq \frac{\Delta x^2}{2D}
\end{equation}

\subsubsection{Chemotaxis Stability}
\begin{equation}
\Delta t \leq \frac{\Delta x^2}{2\chi\varphi_{\max}}
\end{equation}

\subsection{Mesh Requirements}

\subsubsection{Boundary Layers}
\begin{equation}
\Delta x_{\text{boundary}} \leq \sqrt{\frac{D}{v}} \quad \text{(for advection-diffusion)}
\end{equation}

\subsubsection{Reaction Zones}
\begin{equation}
\Delta x_{\text{reaction}} \leq \sqrt{\frac{D}{k}} \quad \text{(for reaction-diffusion)}
\end{equation}

\subsubsection{Chemotactic Focusing}
\begin{equation}
\Delta x \leq \frac{1}{\sqrt{\chi|\nabla\varphi|}}
\end{equation}

\subsection{Parameter Estimation}

\textbf{Typical workflows:}
\begin{enumerate}
    \item \textbf{Literature values}: Start with published parameters
    \item \textbf{Dimensional analysis}: Ensure parameter scaling is correct
    \item \textbf{Sensitivity analysis}: Identify critical parameters
    \item \textbf{Calibration}: Fit to experimental data
    \item \textbf{Validation}: Test on independent data sets
\end{enumerate}

\subsection{Common Pitfalls}

\begin{enumerate}
    \item \textbf{Units consistency}: Always check dimensional analysis
    \item \textbf{Mesh resolution}: Insufficient resolution can cause instabilities
    \item \textbf{Time step size}: Too large steps can cause convergence issues
    \item \textbf{Boundary conditions}: Incorrect BCs can dominate solution
    \item \textbf{Parameter ranges}: Unphysical values can cause numerical issues
\end{enumerate}

\section{Conclusion}

This mathematical background provides the theoretical foundation for understanding and implementing Keller-Segel chemotaxis and organ-on-chip transport models in \bionetflux{}. The comprehensive coverage of governing equations, physical interpretations, analytical solutions, dimensionless analysis, and numerical methods enables users to properly configure, validate, and interpret their simulations.

For specific implementation details and example problems, refer to the main \bionetflux{} documentation and the provided problem templates in the framework.

\vspace{2cm}

\begin{center}
\textbf{\bionetflux{} Development Team} \\
\textit{Mathematical Foundation for Multi-Domain Biological Transport Simulations}
\end{center}

\end{document}
