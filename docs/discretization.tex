% filepath: /Users/silviabertoluzza/GIT/BioNetFlux/docs/discretization_module.tex

\section{Discretization Module}
\label{sec:discretization_module}

The discretization module in BioNetFlux handles the spatial and temporal discretization of network domains. Based on the framework architecture, discretization is integrated into the global assembly process rather than being a standalone comprehensive finite element library.

\subsection{Overview}
\label{subsec:discretization_overview}

The discretization approach in BioNetFlux is designed for networks of 1D segments where:

\begin{itemize}
	\item Each domain represents a single 1D segment (vessel, channel, etc.)
	\item Domains are discretized independently using simple 1D meshes
	\item Inter-domain coupling occurs through trace variables at boundaries
	\item Global assembly coordinates multiple domains through the \texttt{GlobalAssembler}
	\item Time discretization is handled through implicit/explicit schemes in the Newton solver
\end{itemize}

\subsection{Domain Discretization Structure}
\label{subsec:domain_discretization}

From the \texttt{lean\_global\_assembly.py} implementation, we can see that discretization information is managed through domain data structures:

\subsubsection{Domain Data Management}

\begin{lstlisting}[language=Python, caption=Domain Discretization from Global Assembly]
	class GlobalAssembler:
	def __init__(self, domain_data_list: List, constraint_manager=None):
	self.bulk_manager = BulkDataManager(domain_data_list)
	self.n_domains = len(domain_data_list)
	
	# Each domain_data contains discretization information:
	# - n_elements: Number of finite elements
	# - neq: Number of equations per domain
	# - nodes: Mesh node information
	
	self._compute_dof_structure()
	
	def _compute_dof_structure(self):
	"""Compute DOF structure from domain data."""
	self.domain_trace_sizes = []
	self.domain_trace_offsets = []
	
	total_trace_dofs = 0
	for domain_data in self.bulk_manager.domain_data_list:
	# Nodes = elements + 1 for 1D linear elements
	n_nodes = domain_data.n_elements + 1
	# Trace size = equations × nodes
	trace_size = domain_data.neq * n_nodes
	
	self.domain_trace_sizes.append(trace_size)
	self.domain_trace_offsets.append(total_trace_dofs)
	total_trace_dofs += trace_size
\end{lstlisting}

Key observations:
\begin{itemize}
	\item \textbf{1D Linear Elements}: The framework uses $n\_nodes = n\_elements + 1$, indicating linear finite elements
	\item \textbf{Multi-equation Systems}: Each domain supports \texttt{neq} equations
	\item \textbf{Trace-based Coupling}: Primary DOFs are trace variables on domain boundaries
	\item \textbf{Global DOF Management}: Automatic assignment of global DOF indices
\end{itemize}

\subsubsection{Global Discretization Management}

The framework manages discretization across multiple domains:

\begin{lstlisting}[language=Python, caption=Global Discretization Coordination]
	# From the factory method in GlobalAssembler
	@classmethod
	def from_framework_objects(cls, problems: List, global_discretization, 
	static_condensations: List, constraint_manager=None):
	"""Create assembler from framework objects."""
	
	# Extract domain data from framework objects
	domain_data_list = BulkDataManager.extract_domain_data_list(
	problems, 
	global_discretization.spatial_discretizations, 
	static_condensations
	)
	
	return cls(domain_data_list, constraint_manager)
\end{lstlisting}

This indicates:
\begin{itemize}
	\item \textbf{GlobalDiscretization class}: Exists and contains \texttt{spatial\_discretizations}
	\item \textbf{Framework integration}: Problems, discretizations, and static condensations work together
	\item \textbf{Data extraction}: Domain data is extracted from discretization objects
\end{itemize}

\subsection{Spatial Discretization Features}
\label{subsec:spatial_features}

Based on the code structure, the spatial discretization supports:

\subsubsection{1D Finite Element Discretization}

\begin{lstlisting}[language=Python, caption=1D Element Structure (inferred)]
	# Each spatial discretization likely contains:
	class SpatialDiscretization:
	def __init__(self, domain_start, domain_end, n_elements):
	self.n_elements = n_elements
	self.nodes = self._generate_nodes(domain_start, domain_end, n_elements)
	# Linear elements: n_nodes = n_elements + 1
	
	def _generate_nodes(self, start, end, n_elements):
	"""Generate uniform node distribution."""
	return np.linspace(start, end, n_elements + 1)
\end{lstlisting}

\subsubsection{Multi-Equation Support}

From the DOF structure computation, each domain supports multiple equations:

\begin{lstlisting}[language=Python, caption=Multi-Equation DOF Layout]
	# For a domain with neq equations and n_nodes:
	# DOF layout: [eq0_node0, eq0_node1, ..., eq0_nodeN, 
	#              eq1_node0, eq1_node1, ..., eq1_nodeN,
	#              ...]
	# Total DOFs per domain = neq * n_nodes
	
	def extract_equation_dofs(trace_solution, equation_idx, n_nodes):
	"""Extract DOFs for specific equation."""
	eq_start = equation_idx * n_nodes
	eq_end = eq_start + n_nodes
	return trace_solution[eq_start:eq_end]
\end{lstlisting}

\subsection{Constraint-Based Inter-Domain Coupling}
\label{subsec:interdomain_coupling}

The discretization integrates with the constraint system for domain coupling:

\subsubsection{Trace Space Coupling}

\begin{lstlisting}[language=Python, caption=Constraint-Based Coupling]
	# From constraint Jacobian contributions in GlobalAssembler
	def _add_constraint_jacobian_contributions(self, jacobian, trace_solutions, 
	multipliers, time):
	"""Add constraint coupling terms."""
	
	for constraint in self.constraint_manager.constraints:
	if constraint.is_boundary_condition:
	# Single domain constraint
	domain_idx = constraint.domains[0]
	# Calculate global trace index
	domain_data = self.bulk_manager.domain_data_list[domain_idx]
	n_nodes = domain_data.n_elements + 1
	trace_idx = (domain_offset + 
	constraint.equation_index * n_nodes + 
	node_idx)
	else:
	# Junction constraint - couples two domains
	domain1_idx, domain2_idx = constraint.domains
	# Calculate trace indices for both domains
	# Add coupling terms to Jacobian
\end{lstlisting}

Key coupling features:
\begin{itemize}
	\item \textbf{Boundary Conditions}: Dirichlet, Neumann, Robin conditions on domain boundaries
	\item \textbf{Junction Conditions}: Trace continuity and Kedem-Katchalsky conditions between domains
	\item \textbf{Lagrange Multipliers}: Used for constraint enforcement
	\item \textbf{Automatic Indexing}: Global DOF indices computed automatically
\end{itemize}

\subsection{Temporal Discretization}
\label{subsec:temporal_discretization}

While not explicitly shown in the assembly code, temporal discretization is handled through:

\subsubsection{Implicit Time Integration}

\begin{lstlisting}[language=Python, caption=Time Integration Framework]
	# From the Newton solver approach in examples
	def advance_solution(current_solution, dt, time):
	"""Advance solution by one time step."""
	
	# Newton iteration for implicit time step
	newton_solution = current_solution.copy()
	
	for newton_iter in range(max_newton_iterations):
	# Assemble residual and Jacobian at current time
	residual, jacobian = global_assembler.assemble_residual_and_jacobian(
	global_solution=newton_solution,
	forcing_terms=forcing_terms,
	static_condensations=static_condensations,
	time=time
	)
	
	# Check convergence
	if np.linalg.norm(residual) < tolerance:
	break
	
	# Newton update
	delta = np.linalg.solve(jacobian, -residual)
	newton_solution += delta
	
	return newton_solution
\end{lstlisting}

\subsubsection{Static Condensation Integration}

The framework uses static condensation to eliminate bulk DOFs:

\begin{lstlisting}[language=Python, caption=Static Condensation for Time Integration]
	# From bulk_by_static_condensation method
	def bulk_by_static_condensation(self, global_solution, forcing_terms, 
	static_condensations, time):
	"""Recover bulk solution from trace solution."""
	
	trace_solutions = self._extract_trace_solutions(global_solution)
	bulk_solution = []
	
	for i in range(self.n_domains):
	# Use static condensation to compute bulk from trace
	U, F, JF = domain_flux_jump(
	trace_solutions[i].reshape(-1, 1),
	forcing_terms[i],
	None, None,
	static_condensations[i]
	)
	bulk_solution.append(U)
	
	return bulk_solution
\end{lstlisting}

\subsection{Integration with Framework Components}
\label{subsec:framework_integration}

The discretization integrates with other framework components:

\subsubsection{Problem Module Integration}

\begin{lstlisting}[language=Python, caption=Problem-Discretization Interface]
	# Problems provide equation count and parameters
	def create_initial_guess_from_problems(self, problems, discretizations, time=0.0):
	"""Create initial conditions from problem definitions."""
	
	for i in range(self.n_domains):
	problem = problems[i]
	discretization = discretizations[i]
	
	# Use problem.neq to determine equation count
	for eq in range(problem.neq):
	# Use problem initial conditions if available
	if hasattr(domain_data, 'initial_conditions'):
	initial_values = domain_data.initial_conditions[eq](nodes, time)
\end{lstlisting}

\subsubsection{Bulk Data Manager Integration}

\begin{lstlisting}[language=Python, caption=Bulk Data Integration]
	# BulkDataManager handles discretization data
	def initialize_bulk_data(self, problems, discretizations, time=0.0):
	"""Initialize bulk data using discretization information."""
	return self.bulk_manager.initialize_all_bulk_data(
	problems, discretizations, time
	)
	
	def compute_forcing_terms(self, bulk_data_list, problems, 
	discretizations, time, dt):
	"""Compute forcing terms using discretization."""
	return self.bulk_manager.compute_forcing_terms(
	bulk_data_list, problems, discretizations, time, dt
	)
\end{lstlisting}

\subsection{Limitations and Extensions}
\label{subsec:limitations}

Based on the current implementation:

\subsubsection{Current Limitations}

\begin{itemize}
	\item \textbf{Linear Elements Only}: The $n\_nodes = n\_elements + 1$ relationship suggests only linear elements
	\item \textbf{Uniform Meshes}: No evidence of adaptive mesh refinement
	\item \textbf{Simple 1D}: Limited to one-dimensional segments
	\item \textbf{No Higher-Order}: No quadratic or higher-order elements visible
\end{itemize}

\subsubsection{Extension Points}

The framework architecture allows for extensions:

\begin{itemize}
	\item \textbf{Domain Data Structure}: Can be extended with more discretization information
	\item \textbf{Constraint System}: Already supports complex coupling conditions
	\item \textbf{Static Condensation}: Framework for eliminating internal DOFs
	\item \textbf{Modular Assembly}: Components can be replaced or extended
\end{itemize}

The discretization module in BioNetFlux provides a focused, efficient approach to handling 1D network problems with multi-equation systems and complex inter-domain coupling through constraints and static condensation.