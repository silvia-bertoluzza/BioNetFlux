% Problem Module API Documentation
% To be included in master LaTeX document
%
% Usage: % Problem Module API Documentation
% To be included in master LaTeX document
%
% Usage: % Problem Module API Documentation
% To be included in master LaTeX document
%
% Usage: % Problem Module API Documentation
% To be included in master LaTeX document
%
% Usage: \input{docs/problem_module_api}

\section{Problem Module API Reference}
\label{sec:problem_module_api}

This section provides a comprehensive reference for the Problem class (\texttt{ooc1d.core.problem}), which encapsulates the mathematical problem specification for BioNetFlux solver including parameters, boundary conditions, initial conditions, and forcing terms.

\subsection{Module Overview}

The Problem class serves as the central container for:
\begin{itemize}
    \item Physical and mathematical parameters
    \item Domain geometry specification
    \item Initial condition functions
    \item Forcing term (source) functions
    \item Analytical solution functions (when available)
    \item Problem-specific nonlinear functions (e.g., chemotaxis, OrganOnChip)
\end{itemize}

\subsection{Problem Class}
\label{subsec:problem_class}

The main class for problem specification, supporting both Keller-Segel chemotaxis (2 equations) and OrganOnChip (4 equations) systems based on MATLAB reference implementations.

\subsubsection{Constructor}

\paragraph{\_\_init\_\_()}
\begin{lstlisting}[language=Python, caption=Problem Constructor]
def __init__(self, 
             neq: int,
             domain_start: float,
             domain_length: float,
             parameters: np.ndarray,
             problem_type: str = "default",
             name: str = "unnamed_problem")
\end{lstlisting}

\textbf{Parameters:}
\begin{itemize}
    \item \texttt{neq}: Number of equations in the system (2 for Keller-Segel, 4 for OrganOnChip)
    \item \texttt{domain\_start}: Start coordinate of the domain (corresponding to MATLAB \texttt{A})
    \item \texttt{domain\_length}: Length of the domain (corresponding to MATLAB \texttt{L})
    \item \texttt{parameters}: NumPy array of physical parameters
    \item \texttt{problem\_type}: Problem identifier string ("keller\_segel", "organ\_on\_chip", etc.)
    \item \texttt{name}: Descriptive name for the problem instance
\end{itemize}

\textbf{Parameter Array Structures:}

For \textbf{Keller-Segel} problems (\texttt{neq=2}):
\begin{lstlisting}[language=Python, caption=Keller-Segel Parameters]
# parameters = [mu, nu, a, b]
# mu: diffusion coefficient for phi equation  
# nu: diffusion coefficient for u equation
# a, b: reaction/coupling parameters
parameters = np.array([2.0, 1.0, 0.0, 1.0])
\end{lstlisting}

For \textbf{OrganOnChip} problems (\texttt{neq=4}) - from MATLAB \texttt{TestProblem.m}:
\begin{lstlisting}[language=Python, caption=OrganOnChip Parameters]
# parameters = [nu, mu, epsilon, sigma, a, b, c, d, chi]
# nu, mu: viscosity parameters
# epsilon, sigma: coupling parameters
# a, c: reaction parameters  
# b, d: inter-equation coupling
# chi: nonlinear coupling strength
parameters = np.array([1.0, 2.0, 1.0, 1.0, 0.0, 1.0, 0.0, 1.0, 1.0])
\end{lstlisting}

\textbf{Usage Examples:}
\begin{lstlisting}[language=Python, caption=Problem Creation Examples]
# Keller-Segel chemotaxis problem
ks_problem = Problem(
    neq=2,
    domain_start=0.0,
    domain_length=1.0,
    parameters=np.array([2.0, 1.0, 0.0, 1.0]),
    problem_type="keller_segel",
    name="chemotaxis_migration"
)

# OrganOnChip problem (matching MATLAB TestProblem.m)
ooc_problem = Problem(
    neq=4,
    domain_start=0.0,    # MATLAB: A = 0
    domain_length=1.0,   # MATLAB: L = 1
    parameters=np.array([1.0, 2.0, 1.0, 1.0, 0.0, 1.0, 0.0, 1.0, 1.0]),
    problem_type="organ_on_chip",
    name="microfluidic_device"
)
\end{lstlisting}

\subsubsection{Core Attributes}

\begin{longtable}{|p{3.7cm}|p{2.5cm}|p{7cm}|}
\hline
\textbf{Attribute} & \textbf{Type} & \textbf{Description} \\
\hline
\endhead

\texttt{neq} & \texttt{int} & Number of equations in the system \\
\hline

\texttt{domain\_start} & \texttt{float} & Start coordinate of the domain (MATLAB: \texttt{A}) \\
\hline

\texttt{domain\_length} & \texttt{float} & Domain length (MATLAB: \texttt{L}) \\
\hline

\texttt{domain\_end} & \texttt{float} & End coordinate: \texttt{domain\_start + domain\_length} \\
\hline

\texttt{parameters} & \texttt{np.ndarray} & Physical/mathematical parameters array \\
\hline

\texttt{problem\_type} & \texttt{str} & Problem type identifier \\
\hline

\texttt{name} & \texttt{str} & Descriptive problem name \\
\hline

\texttt{type} & \texttt{str} & Alias for \texttt{problem\_type} (legacy compatibility) \\
\hline

\texttt{initial\_conditions} & \texttt{List} & Initial condition functions for each equation \\
\hline

\texttt{forcing\_functions} & \texttt{List} & Source/forcing functions for each equation \\
\hline

\texttt{solution} & \texttt{List} & Analytical solution functions (when available) \\
\hline

\end{longtable}

\subsubsection{Initial Condition Management}

\paragraph{set\_initial\_condition()}
\begin{lstlisting}[language=Python, caption=Set Initial Condition Method]
def set_initial_condition(self, equation_index: int, 
                         initial_function: Callable[[np.ndarray, float], np.ndarray])
\end{lstlisting}

\textbf{Parameters:}
\begin{itemize}
    \item \texttt{equation\_index}: Equation number (0 to \texttt{neq-1})
    \item \texttt{initial\_function}: Function \texttt{f(x, t)} returning initial values
\end{itemize}

\textbf{Function Signature:} \texttt{f(x: np.ndarray, t: float) -> np.ndarray}

\textbf{Usage Examples:}
\begin{lstlisting}[language=Python, caption=Initial Condition Examples]
# Based on MATLAB TestProblem.m initial conditions
# Equation 0 (u): sin wave initial condition
ooc_problem.set_initial_condition(0, lambda x, t: np.sin(2*np.pi*x))

# Equations 1,2,3 (omega, v, phi): zero initial conditions
for eq_idx in [1, 2, 3]:
    ooc_problem.set_initial_condition(eq_idx, lambda x, t: np.zeros_like(x))

# Keller-Segel example: Gaussian initial distribution
ks_problem.set_initial_condition(0, lambda x, t: np.exp(-(x-0.5)**2/0.1))
ks_problem.set_initial_condition(1, lambda x, t: np.ones_like(x))
\end{lstlisting}

\paragraph{get\_initial\_condition()}
\begin{lstlisting}[language=Python, caption=Get Initial Condition Method]
def get_initial_condition(self, equation_index: int, 
                         coordinates: np.ndarray, 
                         time: float = 0.0) -> np.ndarray
\end{lstlisting}

\textbf{Parameters:}
\begin{itemize}
    \item \texttt{equation\_index}: Equation number
    \item \texttt{coordinates}: Spatial coordinate array
    \item \texttt{time}: Time value (default: 0.0)
\end{itemize}

\textbf{Returns:} \texttt{np.ndarray} - Initial condition values at given coordinates

\textbf{Usage:}
\begin{lstlisting}[language=Python]
# Evaluate initial condition at specific points
x_points = np.linspace(0, 1, 11)
u_initial = ooc_problem.get_initial_condition(0, x_points, t=0.0)
print(f"u(x, t=0) = {u_initial}")
\end{lstlisting}

\subsubsection{Forcing Function Management}

\paragraph{set\_force()}
\begin{lstlisting}[language=Python, caption=Set Force Method]
def set_force(self, equation_index: int,
              force_function: Callable[[np.ndarray, float], np.ndarray])
\end{lstlisting}

\textbf{Parameters:}
\begin{itemize}
    \item \texttt{equation\_index}: Equation number (0 to \texttt{neq-1})
    \item \texttt{force\_function}: Source term function \texttt{f(x, t)}
\end{itemize}

\textbf{Usage Examples:}
\begin{lstlisting}[language=Python, caption=Forcing Function Examples]
# Based on MATLAB TestProblem.m: all forcing terms are zero
for eq_idx in range(4):
    ooc_problem.set_force(eq_idx, lambda x, t: np.zeros_like(x))

# Non-zero forcing example: time-dependent source
def time_dependent_source(x, t):
    return 0.1 * np.exp(-t) * np.sin(np.pi * x)

ks_problem.set_force(0, time_dependent_source)
ks_problem.set_force(1, lambda x, t: np.zeros_like(x))  # No source for phi
\end{lstlisting}

\paragraph{get\_force()}
\begin{lstlisting}[language=Python, caption=Get Force Method]
def get_force(self, equation_index: int,
              coordinates: np.ndarray,
              time: float) -> np.ndarray
\end{lstlisting}

\textbf{Parameters:}
\begin{itemize}
    \item \texttt{equation\_index}: Equation number
    \item \texttt{coordinates}: Spatial coordinate array
    \item \texttt{time}: Time value
\end{itemize}

\textbf{Returns:} \texttt{np.ndarray} - Force values at given coordinates and time

\textbf{Usage:}
\begin{lstlisting}[language=Python]
# Evaluate forcing term at current time
x_points = np.linspace(0, 1, 11)
current_time = 0.5
f_values = ooc_problem.get_force(0, x_points, current_time)
\end{lstlisting}

\subsubsection{Analytical Solution Management}

\paragraph{set\_solution()}
\begin{lstlisting}[language=Python, caption=Set Solution Method]
def set_solution(self, equation_index: int,
                 solution_function: Callable[[np.ndarray, float], np.ndarray])
\end{lstlisting}

\textbf{Parameters:}
\begin{itemize}
    \item \texttt{equation\_index}: Equation number
    \item \texttt{solution\_function}: Analytical solution function \texttt{u(x, t)}
\end{itemize}

\textbf{Usage:}
\begin{lstlisting}[language=Python, caption=Analytical Solution Example]
# Set analytical solution for validation (when known)
def analytical_u(x, t):
    return np.exp(-t) * np.sin(np.pi * x)

def analytical_phi(x, t):
    return np.ones_like(x) + 0.1 * np.cos(np.pi * x) * np.exp(-0.5 * t)

ks_problem.set_solution(0, analytical_u)
ks_problem.set_solution(1, analytical_phi)
\end{lstlisting}

\paragraph{get\_solution()}
\begin{lstlisting}[language=Python, caption=Get Solution Method]
def get_solution(self, equation_index: int,
                 coordinates: np.ndarray,
                 time: float) -> np.ndarray
\end{lstlisting}

\textbf{Parameters:}
\begin{itemize}
    \item \texttt{equation\_index}: Equation number
    \item \texttt{coordinates}: Spatial coordinate array
    \item \texttt{time}: Time value
\end{itemize}

\textbf{Returns:} \texttt{np.ndarray} - Analytical solution values

\textbf{Usage:}
\begin{lstlisting}[language=Python]
# Compare numerical with analytical solution
x_test = np.linspace(0, 1, 21)
t_test = 1.0
analytical_values = ks_problem.get_solution(0, x_test, t_test)
# numerical_values = ... (from solver)
# error = np.abs(numerical_values - analytical_values)
\end{lstlisting}

\subsubsection{Problem-Specific Extensions}

\paragraph{set\_chemotaxis() - Keller-Segel Specific}
\begin{lstlisting}[language=Python, caption=Set Chemotaxis Functions]
def set_chemotaxis(self, chi_function: Callable[[np.ndarray], np.ndarray],
                   dchi_function: Callable[[np.ndarray], np.ndarray])
\end{lstlisting}

\textbf{Parameters:}
\begin{itemize}
    \item \texttt{chi\_function}: Chemotactic sensitivity function $\chi(\phi)$
    \item \texttt{dchi\_function}: Derivative function $\chi'(\phi)$
\end{itemize}

\textbf{Usage:}
\begin{lstlisting}[language=Python, caption=Chemotaxis Function Example]
# Nonlinear chemotaxis with saturation
def chi_function(phi):
    k1, k2 = 3.9e-9, 5e-6
    return k1 / (k2 + phi)**2

def dchi_function(phi):
    k1, k2 = 3.9e-9, 5e-6
    return -2 * k1 / (k2 + phi)**3

ks_problem.set_chemotaxis(chi_function, dchi_function)
\end{lstlisting}

\paragraph{set\_lambda\_function() - OrganOnChip Specific}
\begin{lstlisting}[language=Python, caption=Set Lambda Function]
def set_lambda_function(self, lambda_function: Callable[[np.ndarray], np.ndarray],
                       dlambda_function: Callable[[np.ndarray], np.ndarray])
\end{lstlisting}

\textbf{Parameters:}
\begin{itemize}
    \item \texttt{lambda\_function}: Nonlinear response function $\lambda(\omega)$
    \item \texttt{dlambda\_function}: Derivative function $\lambda'(\omega)$
\end{itemize}

\textbf{Usage (from MATLAB TestProblem.m):}
\begin{lstlisting}[language=Python, caption=Lambda Function Example]
# Constant lambda function (as in MATLAB reference)
def constant_lambda(omega):
    return np.ones_like(omega)

def constant_dlambda(omega):
    return np.zeros_like(omega)

ooc_problem.set_lambda_function(constant_lambda, constant_dlambda)
\end{lstlisting}

\subsubsection{Utility Methods}

\paragraph{validate\_setup()}
\begin{lstlisting}[language=Python, caption=Validate Setup Method]
def validate_setup(self) -> bool
\end{lstlisting}

\textbf{Returns:} \texttt{bool} - True if problem setup is complete and consistent

\textbf{Validation Checks:}
\begin{itemize}
    \item All equations have initial conditions defined
    \item All equations have forcing functions defined
    \item Parameter array has correct length for problem type
    \item Domain geometry is valid (positive length)
    \item Problem-specific functions are set when required
\end{itemize}

\textbf{Usage:}
\begin{lstlisting}[language=Python]
if ooc_problem.validate_setup():
    print("Problem setup is complete")
else:
    print("Problem setup incomplete - check missing components")
\end{lstlisting}

\paragraph{summary()}
\begin{lstlisting}[language=Python, caption=Problem Summary Method]
def summary(self) -> str
\end{lstlisting}

\textbf{Returns:} \texttt{str} - Multi-line summary of problem configuration

\textbf{Usage:}
\begin{lstlisting}[language=Python]
print(ooc_problem.summary())
# Output:
# Problem: microfluidic_device
# Type: organ_on_chip
# Equations: 4
# Domain: [0.000, 1.000] (length: 1.000)
# Parameters: [1.0, 2.0, 1.0, 1.0, 0.0, 1.0, 0.0, 1.0, 1.0]
# Initial conditions: 4/4 set
# Forcing functions: 4/4 set
# Analytical solutions: 0/4 set
\end{lstlisting}

\paragraph{copy()}
\begin{lstlisting}[language=Python, caption=Problem Copy Method]
def copy(self) -> 'Problem'
\end{lstlisting}

\textbf{Returns:} \texttt{Problem} - Deep copy of the problem instance

\textbf{Usage:}
\begin{lstlisting}[language=Python]
# Create variant with different parameters
variant_problem = ooc_problem.copy()
variant_problem.parameters[0] = 2.0  # Change nu parameter
variant_problem.name = "high_viscosity_variant"
\end{lstlisting}

\subsection{Integration with BioNetFlux Components}
\label{subsec:problem_integration}

\subsubsection{Integration with Static Condensation}

\begin{lstlisting}[language=Python, caption=Static Condensation Integration]
from ooc1d.core.static_condensation_ooc import StaticCondensationOOC
from ooc1d.core.discretization import Discretization
from ooc1d.utils.elementary_matrices import ElementaryMatrices

# Create problem
problem = Problem(
    neq=4,
    domain_start=0.0,
    domain_length=1.0,
    parameters=np.array([1.0, 2.0, 1.0, 1.0, 0.0, 1.0, 0.0, 1.0, 1.0]),
    problem_type="organ_on_chip"
)

# Set up discretization
discretization = Discretization(
    n_elements=20,
    domain_start=problem.domain_start,
    domain_length=problem.domain_length,
    stab_constant=1.0
)

# Create static condensation
elementary_matrices = ElementaryMatrices()
static_condensation = StaticCondensationOOC(
    problem=problem,
    discretization=discretization, 
    elementary_matrices=elementary_matrices
)
\end{lstlisting}

\subsubsection{Integration with BulkDataManager}

\begin{lstlisting}[language=Python, caption=BulkDataManager Integration]
from ooc1d.solver.bulk_data_manager import BulkDataManager

# Initialize bulk data from problem
bulk_manager = BulkDataManager()
bulk_data = bulk_manager.create_bulk_data(
    problem=problem,
    discretization=discretization,
    time=0.0,  # Initial time
    initialization_method="initial_conditions"
)

# Compute forcing terms at specific time
force_data = bulk_manager.create_bulk_data(
    problem=problem,
    discretization=discretization, 
    time=0.5,  # Current time
    initialization_method="forcing_functions"
)
\end{lstlisting}

\subsection{Factory Methods and Problem Creation Patterns}
\label{subsec:problem_factories}

\subsubsection{Standard Problem Factory}

\begin{lstlisting}[language=Python, caption=Problem Factory Functions]
def create_keller_segel_problem(domain_length: float = 1.0,
                               mu: float = 2.0,
                               nu: float = 1.0,
                               chemotaxis_strength: float = 1.0) -> Problem:
    """Factory for Keller-Segel chemotaxis problems."""
    parameters = np.array([mu, nu, 0.0, chemotaxis_strength])
    
    problem = Problem(
        neq=2,
        domain_start=0.0,
        domain_length=domain_length,
        parameters=parameters,
        problem_type="keller_segel",
        name="keller_segel_chemotaxis"
    )
    
    # Set standard chemotaxis functions
    problem.set_chemotaxis(
        chi_function=lambda phi: np.ones_like(phi),
        dchi_function=lambda phi: np.zeros_like(phi)
    )
    
    return problem

def create_organ_on_chip_problem(domain_length: float = 1.0,
                                viscosity_params: Tuple[float, float] = (1.0, 2.0),
                                coupling_params: Tuple[float, float] = (1.0, 1.0)) -> Problem:
    """Factory for OrganOnChip problems matching MATLAB TestProblem.m."""
    nu, mu = viscosity_params
    epsilon, sigma = coupling_params
    
    # Parameters matching MATLAB: [nu, mu, epsilon, sigma, a, b, c, d, chi]
    parameters = np.array([nu, mu, epsilon, sigma, 0.0, 1.0, 0.0, 1.0, 1.0])
    
    problem = Problem(
        neq=4,
        domain_start=0.0,  # MATLAB: A = 0
        domain_length=domain_length,  # MATLAB: L = 1
        parameters=parameters,
        problem_type="organ_on_chip",
        name="organ_on_chip_device"
    )
    
    # Set constant lambda function (as in MATLAB)
    problem.set_lambda_function(
        lambda_function=lambda omega: np.ones_like(omega),
        dlambda_function=lambda omega: np.zeros_like(omega)
    )
    
    return problem
\end{lstlisting}

\subsubsection{Complete Problem Setup Example}

\begin{lstlisting}[language=Python, caption=Complete Problem Setup]
def setup_complete_ooc_problem():
    """Complete OrganOnChip problem setup matching MATLAB TestProblem.m."""
    
    # Create problem with MATLAB parameters
    problem = create_organ_on_chip_problem(domain_length=1.0)
    
    # Set initial conditions (matching MATLAB TestProblem.m)
    problem.set_initial_condition(0, lambda x, t: np.sin(2*np.pi*x))  # u
    problem.set_initial_condition(1, lambda x, t: np.zeros_like(x))   # omega
    problem.set_initial_condition(2, lambda x, t: np.zeros_like(x))   # v  
    problem.set_initial_condition(3, lambda x, t: np.zeros_like(x))   # phi
    
    # Set zero forcing functions (matching MATLAB)
    for eq_idx in range(4):
        problem.set_force(eq_idx, lambda x, t: np.zeros_like(x))
    
    # Validate setup
    if not problem.validate_setup():
        raise RuntimeError("Problem setup validation failed")
    
    print(f"✓ OrganOnChip problem setup complete:")
    print(problem.summary())
    
    return problem

# Usage
ooc_problem = setup_complete_ooc_problem()
\end{lstlisting}

\subsection{Advanced Usage Patterns}
\label{subsec:advanced_usage}

\subsubsection{Time-Dependent Parameters}

\begin{lstlisting}[language=Python, caption=Time-Dependent Problem Setup]
class TimeDependentProblem(Problem):
    """Extension for time-dependent parameters."""
    
    def __init__(self, *args, **kwargs):
        super().__init__(*args, **kwargs)
        self.parameter_functions = {}
    
    def set_time_dependent_parameter(self, param_index: int,
                                   param_function: Callable[[float], float]):
        """Set time-dependent parameter function."""
        self.parameter_functions[param_index] = param_function
    
    def get_parameters_at_time(self, time: float) -> np.ndarray:
        """Get parameters evaluated at specific time."""
        params = self.parameters.copy()
        for idx, func in self.parameter_functions.items():
            params[idx] = func(time)
        return params

# Usage example: time-dependent viscosity
td_problem = TimeDependentProblem(
    neq=4, domain_start=0.0, domain_length=1.0,
    parameters=np.array([1.0, 2.0, 1.0, 1.0, 0.0, 1.0, 0.0, 1.0, 1.0]),
    problem_type="organ_on_chip"
)

# Viscosity increases with time
td_problem.set_time_dependent_parameter(0, lambda t: 1.0 + 0.1*t)
\end{lstlisting}

The Problem class provides a comprehensive interface for mathematical problem specification in BioNetFlux, with full compatibility with both Keller-Segel and OrganOnChip systems as demonstrated in the MATLAB reference implementations.

% End of problem module API documentation


\section{Problem Module API Reference}
\label{sec:problem_module_api}

This section provides a comprehensive reference for the Problem class (\texttt{ooc1d.core.problem}), which encapsulates the mathematical problem specification for BioNetFlux solver including parameters, boundary conditions, initial conditions, and forcing terms.

\subsection{Module Overview}

The Problem class serves as the central container for:
\begin{itemize}
    \item Physical and mathematical parameters
    \item Domain geometry specification
    \item Initial condition functions
    \item Forcing term (source) functions
    \item Analytical solution functions (when available)
    \item Problem-specific nonlinear functions (e.g., chemotaxis, OrganOnChip)
\end{itemize}

\subsection{Problem Class}
\label{subsec:problem_class}

The main class for problem specification, supporting both Keller-Segel chemotaxis (2 equations) and OrganOnChip (4 equations) systems based on MATLAB reference implementations.

\subsubsection{Constructor}

\paragraph{\_\_init\_\_()}
\begin{lstlisting}[language=Python, caption=Problem Constructor]
def __init__(self, 
             neq: int,
             domain_start: float,
             domain_length: float,
             parameters: np.ndarray,
             problem_type: str = "default",
             name: str = "unnamed_problem")
\end{lstlisting}

\textbf{Parameters:}
\begin{itemize}
    \item \texttt{neq}: Number of equations in the system (2 for Keller-Segel, 4 for OrganOnChip)
    \item \texttt{domain\_start}: Start coordinate of the domain (corresponding to MATLAB \texttt{A})
    \item \texttt{domain\_length}: Length of the domain (corresponding to MATLAB \texttt{L})
    \item \texttt{parameters}: NumPy array of physical parameters
    \item \texttt{problem\_type}: Problem identifier string ("keller\_segel", "organ\_on\_chip", etc.)
    \item \texttt{name}: Descriptive name for the problem instance
\end{itemize}

\textbf{Parameter Array Structures:}

For \textbf{Keller-Segel} problems (\texttt{neq=2}):
\begin{lstlisting}[language=Python, caption=Keller-Segel Parameters]
# parameters = [mu, nu, a, b]
# mu: diffusion coefficient for phi equation  
# nu: diffusion coefficient for u equation
# a, b: reaction/coupling parameters
parameters = np.array([2.0, 1.0, 0.0, 1.0])
\end{lstlisting}

For \textbf{OrganOnChip} problems (\texttt{neq=4}) - from MATLAB \texttt{TestProblem.m}:
\begin{lstlisting}[language=Python, caption=OrganOnChip Parameters]
# parameters = [nu, mu, epsilon, sigma, a, b, c, d, chi]
# nu, mu: viscosity parameters
# epsilon, sigma: coupling parameters
# a, c: reaction parameters  
# b, d: inter-equation coupling
# chi: nonlinear coupling strength
parameters = np.array([1.0, 2.0, 1.0, 1.0, 0.0, 1.0, 0.0, 1.0, 1.0])
\end{lstlisting}

\textbf{Usage Examples:}
\begin{lstlisting}[language=Python, caption=Problem Creation Examples]
# Keller-Segel chemotaxis problem
ks_problem = Problem(
    neq=2,
    domain_start=0.0,
    domain_length=1.0,
    parameters=np.array([2.0, 1.0, 0.0, 1.0]),
    problem_type="keller_segel",
    name="chemotaxis_migration"
)

# OrganOnChip problem (matching MATLAB TestProblem.m)
ooc_problem = Problem(
    neq=4,
    domain_start=0.0,    # MATLAB: A = 0
    domain_length=1.0,   # MATLAB: L = 1
    parameters=np.array([1.0, 2.0, 1.0, 1.0, 0.0, 1.0, 0.0, 1.0, 1.0]),
    problem_type="organ_on_chip",
    name="microfluidic_device"
)
\end{lstlisting}

\subsubsection{Core Attributes}

\begin{longtable}{|p{3.7cm}|p{2.5cm}|p{7cm}|}
\hline
\textbf{Attribute} & \textbf{Type} & \textbf{Description} \\
\hline
\endhead

\texttt{neq} & \texttt{int} & Number of equations in the system \\
\hline

\texttt{domain\_start} & \texttt{float} & Start coordinate of the domain (MATLAB: \texttt{A}) \\
\hline

\texttt{domain\_length} & \texttt{float} & Domain length (MATLAB: \texttt{L}) \\
\hline

\texttt{domain\_end} & \texttt{float} & End coordinate: \texttt{domain\_start + domain\_length} \\
\hline

\texttt{parameters} & \texttt{np.ndarray} & Physical/mathematical parameters array \\
\hline

\texttt{problem\_type} & \texttt{str} & Problem type identifier \\
\hline

\texttt{name} & \texttt{str} & Descriptive problem name \\
\hline

\texttt{type} & \texttt{str} & Alias for \texttt{problem\_type} (legacy compatibility) \\
\hline

\texttt{initial\_conditions} & \texttt{List} & Initial condition functions for each equation \\
\hline

\texttt{forcing\_functions} & \texttt{List} & Source/forcing functions for each equation \\
\hline

\texttt{solution} & \texttt{List} & Analytical solution functions (when available) \\
\hline

\end{longtable}

\subsubsection{Initial Condition Management}

\paragraph{set\_initial\_condition()}
\begin{lstlisting}[language=Python, caption=Set Initial Condition Method]
def set_initial_condition(self, equation_index: int, 
                         initial_function: Callable[[np.ndarray, float], np.ndarray])
\end{lstlisting}

\textbf{Parameters:}
\begin{itemize}
    \item \texttt{equation\_index}: Equation number (0 to \texttt{neq-1})
    \item \texttt{initial\_function}: Function \texttt{f(x, t)} returning initial values
\end{itemize}

\textbf{Function Signature:} \texttt{f(x: np.ndarray, t: float) -> np.ndarray}

\textbf{Usage Examples:}
\begin{lstlisting}[language=Python, caption=Initial Condition Examples]
# Based on MATLAB TestProblem.m initial conditions
# Equation 0 (u): sin wave initial condition
ooc_problem.set_initial_condition(0, lambda x, t: np.sin(2*np.pi*x))

# Equations 1,2,3 (omega, v, phi): zero initial conditions
for eq_idx in [1, 2, 3]:
    ooc_problem.set_initial_condition(eq_idx, lambda x, t: np.zeros_like(x))

# Keller-Segel example: Gaussian initial distribution
ks_problem.set_initial_condition(0, lambda x, t: np.exp(-(x-0.5)**2/0.1))
ks_problem.set_initial_condition(1, lambda x, t: np.ones_like(x))
\end{lstlisting}

\paragraph{get\_initial\_condition()}
\begin{lstlisting}[language=Python, caption=Get Initial Condition Method]
def get_initial_condition(self, equation_index: int, 
                         coordinates: np.ndarray, 
                         time: float = 0.0) -> np.ndarray
\end{lstlisting}

\textbf{Parameters:}
\begin{itemize}
    \item \texttt{equation\_index}: Equation number
    \item \texttt{coordinates}: Spatial coordinate array
    \item \texttt{time}: Time value (default: 0.0)
\end{itemize}

\textbf{Returns:} \texttt{np.ndarray} - Initial condition values at given coordinates

\textbf{Usage:}
\begin{lstlisting}[language=Python]
# Evaluate initial condition at specific points
x_points = np.linspace(0, 1, 11)
u_initial = ooc_problem.get_initial_condition(0, x_points, t=0.0)
print(f"u(x, t=0) = {u_initial}")
\end{lstlisting}

\subsubsection{Forcing Function Management}

\paragraph{set\_force()}
\begin{lstlisting}[language=Python, caption=Set Force Method]
def set_force(self, equation_index: int,
              force_function: Callable[[np.ndarray, float], np.ndarray])
\end{lstlisting}

\textbf{Parameters:}
\begin{itemize}
    \item \texttt{equation\_index}: Equation number (0 to \texttt{neq-1})
    \item \texttt{force\_function}: Source term function \texttt{f(x, t)}
\end{itemize}

\textbf{Usage Examples:}
\begin{lstlisting}[language=Python, caption=Forcing Function Examples]
# Based on MATLAB TestProblem.m: all forcing terms are zero
for eq_idx in range(4):
    ooc_problem.set_force(eq_idx, lambda x, t: np.zeros_like(x))

# Non-zero forcing example: time-dependent source
def time_dependent_source(x, t):
    return 0.1 * np.exp(-t) * np.sin(np.pi * x)

ks_problem.set_force(0, time_dependent_source)
ks_problem.set_force(1, lambda x, t: np.zeros_like(x))  # No source for phi
\end{lstlisting}

\paragraph{get\_force()}
\begin{lstlisting}[language=Python, caption=Get Force Method]
def get_force(self, equation_index: int,
              coordinates: np.ndarray,
              time: float) -> np.ndarray
\end{lstlisting}

\textbf{Parameters:}
\begin{itemize}
    \item \texttt{equation\_index}: Equation number
    \item \texttt{coordinates}: Spatial coordinate array
    \item \texttt{time}: Time value
\end{itemize}

\textbf{Returns:} \texttt{np.ndarray} - Force values at given coordinates and time

\textbf{Usage:}
\begin{lstlisting}[language=Python]
# Evaluate forcing term at current time
x_points = np.linspace(0, 1, 11)
current_time = 0.5
f_values = ooc_problem.get_force(0, x_points, current_time)
\end{lstlisting}

\subsubsection{Analytical Solution Management}

\paragraph{set\_solution()}
\begin{lstlisting}[language=Python, caption=Set Solution Method]
def set_solution(self, equation_index: int,
                 solution_function: Callable[[np.ndarray, float], np.ndarray])
\end{lstlisting}

\textbf{Parameters:}
\begin{itemize}
    \item \texttt{equation\_index}: Equation number
    \item \texttt{solution\_function}: Analytical solution function \texttt{u(x, t)}
\end{itemize}

\textbf{Usage:}
\begin{lstlisting}[language=Python, caption=Analytical Solution Example]
# Set analytical solution for validation (when known)
def analytical_u(x, t):
    return np.exp(-t) * np.sin(np.pi * x)

def analytical_phi(x, t):
    return np.ones_like(x) + 0.1 * np.cos(np.pi * x) * np.exp(-0.5 * t)

ks_problem.set_solution(0, analytical_u)
ks_problem.set_solution(1, analytical_phi)
\end{lstlisting}

\paragraph{get\_solution()}
\begin{lstlisting}[language=Python, caption=Get Solution Method]
def get_solution(self, equation_index: int,
                 coordinates: np.ndarray,
                 time: float) -> np.ndarray
\end{lstlisting}

\textbf{Parameters:}
\begin{itemize}
    \item \texttt{equation\_index}: Equation number
    \item \texttt{coordinates}: Spatial coordinate array
    \item \texttt{time}: Time value
\end{itemize}

\textbf{Returns:} \texttt{np.ndarray} - Analytical solution values

\textbf{Usage:}
\begin{lstlisting}[language=Python]
# Compare numerical with analytical solution
x_test = np.linspace(0, 1, 21)
t_test = 1.0
analytical_values = ks_problem.get_solution(0, x_test, t_test)
# numerical_values = ... (from solver)
# error = np.abs(numerical_values - analytical_values)
\end{lstlisting}

\subsubsection{Problem-Specific Extensions}

\paragraph{set\_chemotaxis() - Keller-Segel Specific}
\begin{lstlisting}[language=Python, caption=Set Chemotaxis Functions]
def set_chemotaxis(self, chi_function: Callable[[np.ndarray], np.ndarray],
                   dchi_function: Callable[[np.ndarray], np.ndarray])
\end{lstlisting}

\textbf{Parameters:}
\begin{itemize}
    \item \texttt{chi\_function}: Chemotactic sensitivity function $\chi(\phi)$
    \item \texttt{dchi\_function}: Derivative function $\chi'(\phi)$
\end{itemize}

\textbf{Usage:}
\begin{lstlisting}[language=Python, caption=Chemotaxis Function Example]
# Nonlinear chemotaxis with saturation
def chi_function(phi):
    k1, k2 = 3.9e-9, 5e-6
    return k1 / (k2 + phi)**2

def dchi_function(phi):
    k1, k2 = 3.9e-9, 5e-6
    return -2 * k1 / (k2 + phi)**3

ks_problem.set_chemotaxis(chi_function, dchi_function)
\end{lstlisting}

\paragraph{set\_lambda\_function() - OrganOnChip Specific}
\begin{lstlisting}[language=Python, caption=Set Lambda Function]
def set_lambda_function(self, lambda_function: Callable[[np.ndarray], np.ndarray],
                       dlambda_function: Callable[[np.ndarray], np.ndarray])
\end{lstlisting}

\textbf{Parameters:}
\begin{itemize}
    \item \texttt{lambda\_function}: Nonlinear response function $\lambda(\omega)$
    \item \texttt{dlambda\_function}: Derivative function $\lambda'(\omega)$
\end{itemize}

\textbf{Usage (from MATLAB TestProblem.m):}
\begin{lstlisting}[language=Python, caption=Lambda Function Example]
# Constant lambda function (as in MATLAB reference)
def constant_lambda(omega):
    return np.ones_like(omega)

def constant_dlambda(omega):
    return np.zeros_like(omega)

ooc_problem.set_lambda_function(constant_lambda, constant_dlambda)
\end{lstlisting}

\subsubsection{Utility Methods}

\paragraph{validate\_setup()}
\begin{lstlisting}[language=Python, caption=Validate Setup Method]
def validate_setup(self) -> bool
\end{lstlisting}

\textbf{Returns:} \texttt{bool} - True if problem setup is complete and consistent

\textbf{Validation Checks:}
\begin{itemize}
    \item All equations have initial conditions defined
    \item All equations have forcing functions defined
    \item Parameter array has correct length for problem type
    \item Domain geometry is valid (positive length)
    \item Problem-specific functions are set when required
\end{itemize}

\textbf{Usage:}
\begin{lstlisting}[language=Python]
if ooc_problem.validate_setup():
    print("Problem setup is complete")
else:
    print("Problem setup incomplete - check missing components")
\end{lstlisting}

\paragraph{summary()}
\begin{lstlisting}[language=Python, caption=Problem Summary Method]
def summary(self) -> str
\end{lstlisting}

\textbf{Returns:} \texttt{str} - Multi-line summary of problem configuration

\textbf{Usage:}
\begin{lstlisting}[language=Python]
print(ooc_problem.summary())
# Output:
# Problem: microfluidic_device
# Type: organ_on_chip
# Equations: 4
# Domain: [0.000, 1.000] (length: 1.000)
# Parameters: [1.0, 2.0, 1.0, 1.0, 0.0, 1.0, 0.0, 1.0, 1.0]
# Initial conditions: 4/4 set
# Forcing functions: 4/4 set
# Analytical solutions: 0/4 set
\end{lstlisting}

\paragraph{copy()}
\begin{lstlisting}[language=Python, caption=Problem Copy Method]
def copy(self) -> 'Problem'
\end{lstlisting}

\textbf{Returns:} \texttt{Problem} - Deep copy of the problem instance

\textbf{Usage:}
\begin{lstlisting}[language=Python]
# Create variant with different parameters
variant_problem = ooc_problem.copy()
variant_problem.parameters[0] = 2.0  # Change nu parameter
variant_problem.name = "high_viscosity_variant"
\end{lstlisting}

\subsection{Integration with BioNetFlux Components}
\label{subsec:problem_integration}

\subsubsection{Integration with Static Condensation}

\begin{lstlisting}[language=Python, caption=Static Condensation Integration]
from ooc1d.core.static_condensation_ooc import StaticCondensationOOC
from ooc1d.core.discretization import Discretization
from ooc1d.utils.elementary_matrices import ElementaryMatrices

# Create problem
problem = Problem(
    neq=4,
    domain_start=0.0,
    domain_length=1.0,
    parameters=np.array([1.0, 2.0, 1.0, 1.0, 0.0, 1.0, 0.0, 1.0, 1.0]),
    problem_type="organ_on_chip"
)

# Set up discretization
discretization = Discretization(
    n_elements=20,
    domain_start=problem.domain_start,
    domain_length=problem.domain_length,
    stab_constant=1.0
)

# Create static condensation
elementary_matrices = ElementaryMatrices()
static_condensation = StaticCondensationOOC(
    problem=problem,
    discretization=discretization, 
    elementary_matrices=elementary_matrices
)
\end{lstlisting}

\subsubsection{Integration with BulkDataManager}

\begin{lstlisting}[language=Python, caption=BulkDataManager Integration]
from ooc1d.solver.bulk_data_manager import BulkDataManager

# Initialize bulk data from problem
bulk_manager = BulkDataManager()
bulk_data = bulk_manager.create_bulk_data(
    problem=problem,
    discretization=discretization,
    time=0.0,  # Initial time
    initialization_method="initial_conditions"
)

# Compute forcing terms at specific time
force_data = bulk_manager.create_bulk_data(
    problem=problem,
    discretization=discretization, 
    time=0.5,  # Current time
    initialization_method="forcing_functions"
)
\end{lstlisting}

\subsection{Factory Methods and Problem Creation Patterns}
\label{subsec:problem_factories}

\subsubsection{Standard Problem Factory}

\begin{lstlisting}[language=Python, caption=Problem Factory Functions]
def create_keller_segel_problem(domain_length: float = 1.0,
                               mu: float = 2.0,
                               nu: float = 1.0,
                               chemotaxis_strength: float = 1.0) -> Problem:
    """Factory for Keller-Segel chemotaxis problems."""
    parameters = np.array([mu, nu, 0.0, chemotaxis_strength])
    
    problem = Problem(
        neq=2,
        domain_start=0.0,
        domain_length=domain_length,
        parameters=parameters,
        problem_type="keller_segel",
        name="keller_segel_chemotaxis"
    )
    
    # Set standard chemotaxis functions
    problem.set_chemotaxis(
        chi_function=lambda phi: np.ones_like(phi),
        dchi_function=lambda phi: np.zeros_like(phi)
    )
    
    return problem

def create_organ_on_chip_problem(domain_length: float = 1.0,
                                viscosity_params: Tuple[float, float] = (1.0, 2.0),
                                coupling_params: Tuple[float, float] = (1.0, 1.0)) -> Problem:
    """Factory for OrganOnChip problems matching MATLAB TestProblem.m."""
    nu, mu = viscosity_params
    epsilon, sigma = coupling_params
    
    # Parameters matching MATLAB: [nu, mu, epsilon, sigma, a, b, c, d, chi]
    parameters = np.array([nu, mu, epsilon, sigma, 0.0, 1.0, 0.0, 1.0, 1.0])
    
    problem = Problem(
        neq=4,
        domain_start=0.0,  # MATLAB: A = 0
        domain_length=domain_length,  # MATLAB: L = 1
        parameters=parameters,
        problem_type="organ_on_chip",
        name="organ_on_chip_device"
    )
    
    # Set constant lambda function (as in MATLAB)
    problem.set_lambda_function(
        lambda_function=lambda omega: np.ones_like(omega),
        dlambda_function=lambda omega: np.zeros_like(omega)
    )
    
    return problem
\end{lstlisting}

\subsubsection{Complete Problem Setup Example}

\begin{lstlisting}[language=Python, caption=Complete Problem Setup]
def setup_complete_ooc_problem():
    """Complete OrganOnChip problem setup matching MATLAB TestProblem.m."""
    
    # Create problem with MATLAB parameters
    problem = create_organ_on_chip_problem(domain_length=1.0)
    
    # Set initial conditions (matching MATLAB TestProblem.m)
    problem.set_initial_condition(0, lambda x, t: np.sin(2*np.pi*x))  # u
    problem.set_initial_condition(1, lambda x, t: np.zeros_like(x))   # omega
    problem.set_initial_condition(2, lambda x, t: np.zeros_like(x))   # v  
    problem.set_initial_condition(3, lambda x, t: np.zeros_like(x))   # phi
    
    # Set zero forcing functions (matching MATLAB)
    for eq_idx in range(4):
        problem.set_force(eq_idx, lambda x, t: np.zeros_like(x))
    
    # Validate setup
    if not problem.validate_setup():
        raise RuntimeError("Problem setup validation failed")
    
    print(f"✓ OrganOnChip problem setup complete:")
    print(problem.summary())
    
    return problem

# Usage
ooc_problem = setup_complete_ooc_problem()
\end{lstlisting}

\subsection{Advanced Usage Patterns}
\label{subsec:advanced_usage}

\subsubsection{Time-Dependent Parameters}

\begin{lstlisting}[language=Python, caption=Time-Dependent Problem Setup]
class TimeDependentProblem(Problem):
    """Extension for time-dependent parameters."""
    
    def __init__(self, *args, **kwargs):
        super().__init__(*args, **kwargs)
        self.parameter_functions = {}
    
    def set_time_dependent_parameter(self, param_index: int,
                                   param_function: Callable[[float], float]):
        """Set time-dependent parameter function."""
        self.parameter_functions[param_index] = param_function
    
    def get_parameters_at_time(self, time: float) -> np.ndarray:
        """Get parameters evaluated at specific time."""
        params = self.parameters.copy()
        for idx, func in self.parameter_functions.items():
            params[idx] = func(time)
        return params

# Usage example: time-dependent viscosity
td_problem = TimeDependentProblem(
    neq=4, domain_start=0.0, domain_length=1.0,
    parameters=np.array([1.0, 2.0, 1.0, 1.0, 0.0, 1.0, 0.0, 1.0, 1.0]),
    problem_type="organ_on_chip"
)

# Viscosity increases with time
td_problem.set_time_dependent_parameter(0, lambda t: 1.0 + 0.1*t)
\end{lstlisting}

The Problem class provides a comprehensive interface for mathematical problem specification in BioNetFlux, with full compatibility with both Keller-Segel and OrganOnChip systems as demonstrated in the MATLAB reference implementations.

% End of problem module API documentation


\section{Problem Module API Reference}
\label{sec:problem_module_api}

This section provides a comprehensive reference for the Problem class (\texttt{ooc1d.core.problem}), which encapsulates the mathematical problem specification for BioNetFlux solver including parameters, boundary conditions, initial conditions, and forcing terms.

\subsection{Module Overview}

The Problem class serves as the central container for:
\begin{itemize}
    \item Physical and mathematical parameters
    \item Domain geometry specification
    \item Initial condition functions
    \item Forcing term (source) functions
    \item Analytical solution functions (when available)
    \item Problem-specific nonlinear functions (e.g., chemotaxis, OrganOnChip)
\end{itemize}

\subsection{Problem Class}
\label{subsec:problem_class}

The main class for problem specification, supporting both Keller-Segel chemotaxis (2 equations) and OrganOnChip (4 equations) systems based on MATLAB reference implementations.

\subsubsection{Constructor}

\paragraph{\_\_init\_\_()}
\begin{lstlisting}[language=Python, caption=Problem Constructor]
def __init__(self, 
             neq: int,
             domain_start: float,
             domain_length: float,
             parameters: np.ndarray,
             problem_type: str = "default",
             name: str = "unnamed_problem")
\end{lstlisting}

\textbf{Parameters:}
\begin{itemize}
    \item \texttt{neq}: Number of equations in the system (2 for Keller-Segel, 4 for OrganOnChip)
    \item \texttt{domain\_start}: Start coordinate of the domain (corresponding to MATLAB \texttt{A})
    \item \texttt{domain\_length}: Length of the domain (corresponding to MATLAB \texttt{L})
    \item \texttt{parameters}: NumPy array of physical parameters
    \item \texttt{problem\_type}: Problem identifier string ("keller\_segel", "organ\_on\_chip", etc.)
    \item \texttt{name}: Descriptive name for the problem instance
\end{itemize}

\textbf{Parameter Array Structures:}

For \textbf{Keller-Segel} problems (\texttt{neq=2}):
\begin{lstlisting}[language=Python, caption=Keller-Segel Parameters]
# parameters = [mu, nu, a, b]
# mu: diffusion coefficient for phi equation  
# nu: diffusion coefficient for u equation
# a, b: reaction/coupling parameters
parameters = np.array([2.0, 1.0, 0.0, 1.0])
\end{lstlisting}

For \textbf{OrganOnChip} problems (\texttt{neq=4}) - from MATLAB \texttt{TestProblem.m}:
\begin{lstlisting}[language=Python, caption=OrganOnChip Parameters]
# parameters = [nu, mu, epsilon, sigma, a, b, c, d, chi]
# nu, mu: viscosity parameters
# epsilon, sigma: coupling parameters
# a, c: reaction parameters  
# b, d: inter-equation coupling
# chi: nonlinear coupling strength
parameters = np.array([1.0, 2.0, 1.0, 1.0, 0.0, 1.0, 0.0, 1.0, 1.0])
\end{lstlisting}

\textbf{Usage Examples:}
\begin{lstlisting}[language=Python, caption=Problem Creation Examples]
# Keller-Segel chemotaxis problem
ks_problem = Problem(
    neq=2,
    domain_start=0.0,
    domain_length=1.0,
    parameters=np.array([2.0, 1.0, 0.0, 1.0]),
    problem_type="keller_segel",
    name="chemotaxis_migration"
)

# OrganOnChip problem (matching MATLAB TestProblem.m)
ooc_problem = Problem(
    neq=4,
    domain_start=0.0,    # MATLAB: A = 0
    domain_length=1.0,   # MATLAB: L = 1
    parameters=np.array([1.0, 2.0, 1.0, 1.0, 0.0, 1.0, 0.0, 1.0, 1.0]),
    problem_type="organ_on_chip",
    name="microfluidic_device"
)
\end{lstlisting}

\subsubsection{Core Attributes}

\begin{longtable}{|p{3.7cm}|p{2.5cm}|p{7cm}|}
\hline
\textbf{Attribute} & \textbf{Type} & \textbf{Description} \\
\hline
\endhead

\texttt{neq} & \texttt{int} & Number of equations in the system \\
\hline

\texttt{domain\_start} & \texttt{float} & Start coordinate of the domain (MATLAB: \texttt{A}) \\
\hline

\texttt{domain\_length} & \texttt{float} & Domain length (MATLAB: \texttt{L}) \\
\hline

\texttt{domain\_end} & \texttt{float} & End coordinate: \texttt{domain\_start + domain\_length} \\
\hline

\texttt{parameters} & \texttt{np.ndarray} & Physical/mathematical parameters array \\
\hline

\texttt{problem\_type} & \texttt{str} & Problem type identifier \\
\hline

\texttt{name} & \texttt{str} & Descriptive problem name \\
\hline

\texttt{type} & \texttt{str} & Alias for \texttt{problem\_type} (legacy compatibility) \\
\hline

\texttt{initial\_conditions} & \texttt{List} & Initial condition functions for each equation \\
\hline

\texttt{forcing\_functions} & \texttt{List} & Source/forcing functions for each equation \\
\hline

\texttt{solution} & \texttt{List} & Analytical solution functions (when available) \\
\hline

\end{longtable}

\subsubsection{Initial Condition Management}

\paragraph{set\_initial\_condition()}
\begin{lstlisting}[language=Python, caption=Set Initial Condition Method]
def set_initial_condition(self, equation_index: int, 
                         initial_function: Callable[[np.ndarray, float], np.ndarray])
\end{lstlisting}

\textbf{Parameters:}
\begin{itemize}
    \item \texttt{equation\_index}: Equation number (0 to \texttt{neq-1})
    \item \texttt{initial\_function}: Function \texttt{f(x, t)} returning initial values
\end{itemize}

\textbf{Function Signature:} \texttt{f(x: np.ndarray, t: float) -> np.ndarray}

\textbf{Usage Examples:}
\begin{lstlisting}[language=Python, caption=Initial Condition Examples]
# Based on MATLAB TestProblem.m initial conditions
# Equation 0 (u): sin wave initial condition
ooc_problem.set_initial_condition(0, lambda x, t: np.sin(2*np.pi*x))

# Equations 1,2,3 (omega, v, phi): zero initial conditions
for eq_idx in [1, 2, 3]:
    ooc_problem.set_initial_condition(eq_idx, lambda x, t: np.zeros_like(x))

# Keller-Segel example: Gaussian initial distribution
ks_problem.set_initial_condition(0, lambda x, t: np.exp(-(x-0.5)**2/0.1))
ks_problem.set_initial_condition(1, lambda x, t: np.ones_like(x))
\end{lstlisting}

\paragraph{get\_initial\_condition()}
\begin{lstlisting}[language=Python, caption=Get Initial Condition Method]
def get_initial_condition(self, equation_index: int, 
                         coordinates: np.ndarray, 
                         time: float = 0.0) -> np.ndarray
\end{lstlisting}

\textbf{Parameters:}
\begin{itemize}
    \item \texttt{equation\_index}: Equation number
    \item \texttt{coordinates}: Spatial coordinate array
    \item \texttt{time}: Time value (default: 0.0)
\end{itemize}

\textbf{Returns:} \texttt{np.ndarray} - Initial condition values at given coordinates

\textbf{Usage:}
\begin{lstlisting}[language=Python]
# Evaluate initial condition at specific points
x_points = np.linspace(0, 1, 11)
u_initial = ooc_problem.get_initial_condition(0, x_points, t=0.0)
print(f"u(x, t=0) = {u_initial}")
\end{lstlisting}

\subsubsection{Forcing Function Management}

\paragraph{set\_force()}
\begin{lstlisting}[language=Python, caption=Set Force Method]
def set_force(self, equation_index: int,
              force_function: Callable[[np.ndarray, float], np.ndarray])
\end{lstlisting}

\textbf{Parameters:}
\begin{itemize}
    \item \texttt{equation\_index}: Equation number (0 to \texttt{neq-1})
    \item \texttt{force\_function}: Source term function \texttt{f(x, t)}
\end{itemize}

\textbf{Usage Examples:}
\begin{lstlisting}[language=Python, caption=Forcing Function Examples]
# Based on MATLAB TestProblem.m: all forcing terms are zero
for eq_idx in range(4):
    ooc_problem.set_force(eq_idx, lambda x, t: np.zeros_like(x))

# Non-zero forcing example: time-dependent source
def time_dependent_source(x, t):
    return 0.1 * np.exp(-t) * np.sin(np.pi * x)

ks_problem.set_force(0, time_dependent_source)
ks_problem.set_force(1, lambda x, t: np.zeros_like(x))  # No source for phi
\end{lstlisting}

\paragraph{get\_force()}
\begin{lstlisting}[language=Python, caption=Get Force Method]
def get_force(self, equation_index: int,
              coordinates: np.ndarray,
              time: float) -> np.ndarray
\end{lstlisting}

\textbf{Parameters:}
\begin{itemize}
    \item \texttt{equation\_index}: Equation number
    \item \texttt{coordinates}: Spatial coordinate array
    \item \texttt{time}: Time value
\end{itemize}

\textbf{Returns:} \texttt{np.ndarray} - Force values at given coordinates and time

\textbf{Usage:}
\begin{lstlisting}[language=Python]
# Evaluate forcing term at current time
x_points = np.linspace(0, 1, 11)
current_time = 0.5
f_values = ooc_problem.get_force(0, x_points, current_time)
\end{lstlisting}

\subsubsection{Analytical Solution Management}

\paragraph{set\_solution()}
\begin{lstlisting}[language=Python, caption=Set Solution Method]
def set_solution(self, equation_index: int,
                 solution_function: Callable[[np.ndarray, float], np.ndarray])
\end{lstlisting}

\textbf{Parameters:}
\begin{itemize}
    \item \texttt{equation\_index}: Equation number
    \item \texttt{solution\_function}: Analytical solution function \texttt{u(x, t)}
\end{itemize}

\textbf{Usage:}
\begin{lstlisting}[language=Python, caption=Analytical Solution Example]
# Set analytical solution for validation (when known)
def analytical_u(x, t):
    return np.exp(-t) * np.sin(np.pi * x)

def analytical_phi(x, t):
    return np.ones_like(x) + 0.1 * np.cos(np.pi * x) * np.exp(-0.5 * t)

ks_problem.set_solution(0, analytical_u)
ks_problem.set_solution(1, analytical_phi)
\end{lstlisting}

\paragraph{get\_solution()}
\begin{lstlisting}[language=Python, caption=Get Solution Method]
def get_solution(self, equation_index: int,
                 coordinates: np.ndarray,
                 time: float) -> np.ndarray
\end{lstlisting}

\textbf{Parameters:}
\begin{itemize}
    \item \texttt{equation\_index}: Equation number
    \item \texttt{coordinates}: Spatial coordinate array
    \item \texttt{time}: Time value
\end{itemize}

\textbf{Returns:} \texttt{np.ndarray} - Analytical solution values

\textbf{Usage:}
\begin{lstlisting}[language=Python]
# Compare numerical with analytical solution
x_test = np.linspace(0, 1, 21)
t_test = 1.0
analytical_values = ks_problem.get_solution(0, x_test, t_test)
# numerical_values = ... (from solver)
# error = np.abs(numerical_values - analytical_values)
\end{lstlisting}

\subsubsection{Problem-Specific Extensions}

\paragraph{set\_chemotaxis() - Keller-Segel Specific}
\begin{lstlisting}[language=Python, caption=Set Chemotaxis Functions]
def set_chemotaxis(self, chi_function: Callable[[np.ndarray], np.ndarray],
                   dchi_function: Callable[[np.ndarray], np.ndarray])
\end{lstlisting}

\textbf{Parameters:}
\begin{itemize}
    \item \texttt{chi\_function}: Chemotactic sensitivity function $\chi(\phi)$
    \item \texttt{dchi\_function}: Derivative function $\chi'(\phi)$
\end{itemize}

\textbf{Usage:}
\begin{lstlisting}[language=Python, caption=Chemotaxis Function Example]
# Nonlinear chemotaxis with saturation
def chi_function(phi):
    k1, k2 = 3.9e-9, 5e-6
    return k1 / (k2 + phi)**2

def dchi_function(phi):
    k1, k2 = 3.9e-9, 5e-6
    return -2 * k1 / (k2 + phi)**3

ks_problem.set_chemotaxis(chi_function, dchi_function)
\end{lstlisting}

\paragraph{set\_lambda\_function() - OrganOnChip Specific}
\begin{lstlisting}[language=Python, caption=Set Lambda Function]
def set_lambda_function(self, lambda_function: Callable[[np.ndarray], np.ndarray],
                       dlambda_function: Callable[[np.ndarray], np.ndarray])
\end{lstlisting}

\textbf{Parameters:}
\begin{itemize}
    \item \texttt{lambda\_function}: Nonlinear response function $\lambda(\omega)$
    \item \texttt{dlambda\_function}: Derivative function $\lambda'(\omega)$
\end{itemize}

\textbf{Usage (from MATLAB TestProblem.m):}
\begin{lstlisting}[language=Python, caption=Lambda Function Example]
# Constant lambda function (as in MATLAB reference)
def constant_lambda(omega):
    return np.ones_like(omega)

def constant_dlambda(omega):
    return np.zeros_like(omega)

ooc_problem.set_lambda_function(constant_lambda, constant_dlambda)
\end{lstlisting}

\subsubsection{Utility Methods}

\paragraph{validate\_setup()}
\begin{lstlisting}[language=Python, caption=Validate Setup Method]
def validate_setup(self) -> bool
\end{lstlisting}

\textbf{Returns:} \texttt{bool} - True if problem setup is complete and consistent

\textbf{Validation Checks:}
\begin{itemize}
    \item All equations have initial conditions defined
    \item All equations have forcing functions defined
    \item Parameter array has correct length for problem type
    \item Domain geometry is valid (positive length)
    \item Problem-specific functions are set when required
\end{itemize}

\textbf{Usage:}
\begin{lstlisting}[language=Python]
if ooc_problem.validate_setup():
    print("Problem setup is complete")
else:
    print("Problem setup incomplete - check missing components")
\end{lstlisting}

\paragraph{summary()}
\begin{lstlisting}[language=Python, caption=Problem Summary Method]
def summary(self) -> str
\end{lstlisting}

\textbf{Returns:} \texttt{str} - Multi-line summary of problem configuration

\textbf{Usage:}
\begin{lstlisting}[language=Python]
print(ooc_problem.summary())
# Output:
# Problem: microfluidic_device
# Type: organ_on_chip
# Equations: 4
# Domain: [0.000, 1.000] (length: 1.000)
# Parameters: [1.0, 2.0, 1.0, 1.0, 0.0, 1.0, 0.0, 1.0, 1.0]
# Initial conditions: 4/4 set
# Forcing functions: 4/4 set
# Analytical solutions: 0/4 set
\end{lstlisting}

\paragraph{copy()}
\begin{lstlisting}[language=Python, caption=Problem Copy Method]
def copy(self) -> 'Problem'
\end{lstlisting}

\textbf{Returns:} \texttt{Problem} - Deep copy of the problem instance

\textbf{Usage:}
\begin{lstlisting}[language=Python]
# Create variant with different parameters
variant_problem = ooc_problem.copy()
variant_problem.parameters[0] = 2.0  # Change nu parameter
variant_problem.name = "high_viscosity_variant"
\end{lstlisting}

\subsection{Integration with BioNetFlux Components}
\label{subsec:problem_integration}

\subsubsection{Integration with Static Condensation}

\begin{lstlisting}[language=Python, caption=Static Condensation Integration]
from ooc1d.core.static_condensation_ooc import StaticCondensationOOC
from ooc1d.core.discretization import Discretization
from ooc1d.utils.elementary_matrices import ElementaryMatrices

# Create problem
problem = Problem(
    neq=4,
    domain_start=0.0,
    domain_length=1.0,
    parameters=np.array([1.0, 2.0, 1.0, 1.0, 0.0, 1.0, 0.0, 1.0, 1.0]),
    problem_type="organ_on_chip"
)

# Set up discretization
discretization = Discretization(
    n_elements=20,
    domain_start=problem.domain_start,
    domain_length=problem.domain_length,
    stab_constant=1.0
)

# Create static condensation
elementary_matrices = ElementaryMatrices()
static_condensation = StaticCondensationOOC(
    problem=problem,
    discretization=discretization, 
    elementary_matrices=elementary_matrices
)
\end{lstlisting}

\subsubsection{Integration with BulkDataManager}

\begin{lstlisting}[language=Python, caption=BulkDataManager Integration]
from ooc1d.solver.bulk_data_manager import BulkDataManager

# Initialize bulk data from problem
bulk_manager = BulkDataManager()
bulk_data = bulk_manager.create_bulk_data(
    problem=problem,
    discretization=discretization,
    time=0.0,  # Initial time
    initialization_method="initial_conditions"
)

# Compute forcing terms at specific time
force_data = bulk_manager.create_bulk_data(
    problem=problem,
    discretization=discretization, 
    time=0.5,  # Current time
    initialization_method="forcing_functions"
)
\end{lstlisting}

\subsection{Factory Methods and Problem Creation Patterns}
\label{subsec:problem_factories}

\subsubsection{Standard Problem Factory}

\begin{lstlisting}[language=Python, caption=Problem Factory Functions]
def create_keller_segel_problem(domain_length: float = 1.0,
                               mu: float = 2.0,
                               nu: float = 1.0,
                               chemotaxis_strength: float = 1.0) -> Problem:
    """Factory for Keller-Segel chemotaxis problems."""
    parameters = np.array([mu, nu, 0.0, chemotaxis_strength])
    
    problem = Problem(
        neq=2,
        domain_start=0.0,
        domain_length=domain_length,
        parameters=parameters,
        problem_type="keller_segel",
        name="keller_segel_chemotaxis"
    )
    
    # Set standard chemotaxis functions
    problem.set_chemotaxis(
        chi_function=lambda phi: np.ones_like(phi),
        dchi_function=lambda phi: np.zeros_like(phi)
    )
    
    return problem

def create_organ_on_chip_problem(domain_length: float = 1.0,
                                viscosity_params: Tuple[float, float] = (1.0, 2.0),
                                coupling_params: Tuple[float, float] = (1.0, 1.0)) -> Problem:
    """Factory for OrganOnChip problems matching MATLAB TestProblem.m."""
    nu, mu = viscosity_params
    epsilon, sigma = coupling_params
    
    # Parameters matching MATLAB: [nu, mu, epsilon, sigma, a, b, c, d, chi]
    parameters = np.array([nu, mu, epsilon, sigma, 0.0, 1.0, 0.0, 1.0, 1.0])
    
    problem = Problem(
        neq=4,
        domain_start=0.0,  # MATLAB: A = 0
        domain_length=domain_length,  # MATLAB: L = 1
        parameters=parameters,
        problem_type="organ_on_chip",
        name="organ_on_chip_device"
    )
    
    # Set constant lambda function (as in MATLAB)
    problem.set_lambda_function(
        lambda_function=lambda omega: np.ones_like(omega),
        dlambda_function=lambda omega: np.zeros_like(omega)
    )
    
    return problem
\end{lstlisting}

\subsubsection{Complete Problem Setup Example}

\begin{lstlisting}[language=Python, caption=Complete Problem Setup]
def setup_complete_ooc_problem():
    """Complete OrganOnChip problem setup matching MATLAB TestProblem.m."""
    
    # Create problem with MATLAB parameters
    problem = create_organ_on_chip_problem(domain_length=1.0)
    
    # Set initial conditions (matching MATLAB TestProblem.m)
    problem.set_initial_condition(0, lambda x, t: np.sin(2*np.pi*x))  # u
    problem.set_initial_condition(1, lambda x, t: np.zeros_like(x))   # omega
    problem.set_initial_condition(2, lambda x, t: np.zeros_like(x))   # v  
    problem.set_initial_condition(3, lambda x, t: np.zeros_like(x))   # phi
    
    # Set zero forcing functions (matching MATLAB)
    for eq_idx in range(4):
        problem.set_force(eq_idx, lambda x, t: np.zeros_like(x))
    
    # Validate setup
    if not problem.validate_setup():
        raise RuntimeError("Problem setup validation failed")
    
    print(f"✓ OrganOnChip problem setup complete:")
    print(problem.summary())
    
    return problem

# Usage
ooc_problem = setup_complete_ooc_problem()
\end{lstlisting}

\subsection{Advanced Usage Patterns}
\label{subsec:advanced_usage}

\subsubsection{Time-Dependent Parameters}

\begin{lstlisting}[language=Python, caption=Time-Dependent Problem Setup]
class TimeDependentProblem(Problem):
    """Extension for time-dependent parameters."""
    
    def __init__(self, *args, **kwargs):
        super().__init__(*args, **kwargs)
        self.parameter_functions = {}
    
    def set_time_dependent_parameter(self, param_index: int,
                                   param_function: Callable[[float], float]):
        """Set time-dependent parameter function."""
        self.parameter_functions[param_index] = param_function
    
    def get_parameters_at_time(self, time: float) -> np.ndarray:
        """Get parameters evaluated at specific time."""
        params = self.parameters.copy()
        for idx, func in self.parameter_functions.items():
            params[idx] = func(time)
        return params

# Usage example: time-dependent viscosity
td_problem = TimeDependentProblem(
    neq=4, domain_start=0.0, domain_length=1.0,
    parameters=np.array([1.0, 2.0, 1.0, 1.0, 0.0, 1.0, 0.0, 1.0, 1.0]),
    problem_type="organ_on_chip"
)

# Viscosity increases with time
td_problem.set_time_dependent_parameter(0, lambda t: 1.0 + 0.1*t)
\end{lstlisting}

The Problem class provides a comprehensive interface for mathematical problem specification in BioNetFlux, with full compatibility with both Keller-Segel and OrganOnChip systems as demonstrated in the MATLAB reference implementations.

% End of problem module API documentation


\section{Problem Module API Reference}
\label{sec:problem_module_api}

This section provides a comprehensive reference for the Problem class (\texttt{ooc1d.core.problem}), which encapsulates the mathematical problem specification for BioNetFlux solver including parameters, boundary conditions, initial conditions, and forcing terms.

\subsection{Module Overview}

The Problem class serves as the central container for:
\begin{itemize}
    \item Physical and mathematical parameters
    \item Domain geometry specification
    \item Initial condition functions
    \item Forcing term (source) functions
    \item Analytical solution functions (when available)
    \item Problem-specific nonlinear functions (e.g., chemotaxis, OrganOnChip)
\end{itemize}

\subsection{Problem Class}
\label{subsec:problem_class}

The main class for problem specification, supporting both Keller-Segel chemotaxis (2 equations) and OrganOnChip (4 equations) systems based on MATLAB reference implementations.

\subsubsection{Constructor}

\paragraph{\_\_init\_\_()}
\begin{lstlisting}[language=Python, caption=Problem Constructor]
def __init__(self, 
             neq: int,
             domain_start: float,
             domain_length: float,
             parameters: np.ndarray,
             problem_type: str = "default",
             name: str = "unnamed_problem")
\end{lstlisting}

\textbf{Parameters:}
\begin{itemize}
    \item \texttt{neq}: Number of equations in the system (2 for Keller-Segel, 4 for OrganOnChip)
    \item \texttt{domain\_start}: Start coordinate of the domain (corresponding to MATLAB \texttt{A})
    \item \texttt{domain\_length}: Length of the domain (corresponding to MATLAB \texttt{L})
    \item \texttt{parameters}: NumPy array of physical parameters
    \item \texttt{problem\_type}: Problem identifier string ("keller\_segel", "organ\_on\_chip", etc.)
    \item \texttt{name}: Descriptive name for the problem instance
\end{itemize}

\textbf{Parameter Array Structures:}

For \textbf{Keller-Segel} problems (\texttt{neq=2}):
\begin{lstlisting}[language=Python, caption=Keller-Segel Parameters]
# parameters = [mu, nu, a, b]
# mu: diffusion coefficient for phi equation  
# nu: diffusion coefficient for u equation
# a, b: reaction/coupling parameters
parameters = np.array([2.0, 1.0, 0.0, 1.0])
\end{lstlisting}

For \textbf{OrganOnChip} problems (\texttt{neq=4}) - from MATLAB \texttt{TestProblem.m}:
\begin{lstlisting}[language=Python, caption=OrganOnChip Parameters]
# parameters = [nu, mu, epsilon, sigma, a, b, c, d, chi]
# nu, mu: viscosity parameters
# epsilon, sigma: coupling parameters
# a, c: reaction parameters  
# b, d: inter-equation coupling
# chi: nonlinear coupling strength
parameters = np.array([1.0, 2.0, 1.0, 1.0, 0.0, 1.0, 0.0, 1.0, 1.0])
\end{lstlisting}

\textbf{Usage Examples:}
\begin{lstlisting}[language=Python, caption=Problem Creation Examples]
# Keller-Segel chemotaxis problem
ks_problem = Problem(
    neq=2,
    domain_start=0.0,
    domain_length=1.0,
    parameters=np.array([2.0, 1.0, 0.0, 1.0]),
    problem_type="keller_segel",
    name="chemotaxis_migration"
)

# OrganOnChip problem (matching MATLAB TestProblem.m)
ooc_problem = Problem(
    neq=4,
    domain_start=0.0,    # MATLAB: A = 0
    domain_length=1.0,   # MATLAB: L = 1
    parameters=np.array([1.0, 2.0, 1.0, 1.0, 0.0, 1.0, 0.0, 1.0, 1.0]),
    problem_type="organ_on_chip",
    name="microfluidic_device"
)
\end{lstlisting}

\subsubsection{Core Attributes}

\begin{longtable}{|p{3.7cm}|p{2.5cm}|p{7cm}|}
\hline
\textbf{Attribute} & \textbf{Type} & \textbf{Description} \\
\hline
\endhead

\texttt{neq} & \texttt{int} & Number of equations in the system \\
\hline

\texttt{domain\_start} & \texttt{float} & Start coordinate of the domain (MATLAB: \texttt{A}) \\
\hline

\texttt{domain\_length} & \texttt{float} & Domain length (MATLAB: \texttt{L}) \\
\hline

\texttt{domain\_end} & \texttt{float} & End coordinate: \texttt{domain\_start + domain\_length} \\
\hline

\texttt{parameters} & \texttt{np.ndarray} & Physical/mathematical parameters array \\
\hline

\texttt{problem\_type} & \texttt{str} & Problem type identifier \\
\hline

\texttt{name} & \texttt{str} & Descriptive problem name \\
\hline

\texttt{type} & \texttt{str} & Alias for \texttt{problem\_type} (legacy compatibility) \\
\hline

\texttt{initial\_conditions} & \texttt{List} & Initial condition functions for each equation \\
\hline

\texttt{forcing\_functions} & \texttt{List} & Source/forcing functions for each equation \\
\hline

\texttt{solution} & \texttt{List} & Analytical solution functions (when available) \\
\hline

\end{longtable}

\subsubsection{Initial Condition Management}

\paragraph{set\_initial\_condition()}
\begin{lstlisting}[language=Python, caption=Set Initial Condition Method]
def set_initial_condition(self, equation_index: int, 
                         initial_function: Callable[[np.ndarray, float], np.ndarray])
\end{lstlisting}

\textbf{Parameters:}
\begin{itemize}
    \item \texttt{equation\_index}: Equation number (0 to \texttt{neq-1})
    \item \texttt{initial\_function}: Function \texttt{f(x, t)} returning initial values
\end{itemize}

\textbf{Function Signature:} \texttt{f(x: np.ndarray, t: float) -> np.ndarray}

\textbf{Usage Examples:}
\begin{lstlisting}[language=Python, caption=Initial Condition Examples]
# Based on MATLAB TestProblem.m initial conditions
# Equation 0 (u): sin wave initial condition
ooc_problem.set_initial_condition(0, lambda x, t: np.sin(2*np.pi*x))

# Equations 1,2,3 (omega, v, phi): zero initial conditions
for eq_idx in [1, 2, 3]:
    ooc_problem.set_initial_condition(eq_idx, lambda x, t: np.zeros_like(x))

# Keller-Segel example: Gaussian initial distribution
ks_problem.set_initial_condition(0, lambda x, t: np.exp(-(x-0.5)**2/0.1))
ks_problem.set_initial_condition(1, lambda x, t: np.ones_like(x))
\end{lstlisting}

\paragraph{get\_initial\_condition()}
\begin{lstlisting}[language=Python, caption=Get Initial Condition Method]
def get_initial_condition(self, equation_index: int, 
                         coordinates: np.ndarray, 
                         time: float = 0.0) -> np.ndarray
\end{lstlisting}

\textbf{Parameters:}
\begin{itemize}
    \item \texttt{equation\_index}: Equation number
    \item \texttt{coordinates}: Spatial coordinate array
    \item \texttt{time}: Time value (default: 0.0)
\end{itemize}

\textbf{Returns:} \texttt{np.ndarray} - Initial condition values at given coordinates

\textbf{Usage:}
\begin{lstlisting}[language=Python]
# Evaluate initial condition at specific points
x_points = np.linspace(0, 1, 11)
u_initial = ooc_problem.get_initial_condition(0, x_points, t=0.0)
print(f"u(x, t=0) = {u_initial}")
\end{lstlisting}

\subsubsection{Forcing Function Management}

\paragraph{set\_force()}
\begin{lstlisting}[language=Python, caption=Set Force Method]
def set_force(self, equation_index: int,
              force_function: Callable[[np.ndarray, float], np.ndarray])
\end{lstlisting}

\textbf{Parameters:}
\begin{itemize}
    \item \texttt{equation\_index}: Equation number (0 to \texttt{neq-1})
    \item \texttt{force\_function}: Source term function \texttt{f(x, t)}
\end{itemize}

\textbf{Usage Examples:}
\begin{lstlisting}[language=Python, caption=Forcing Function Examples]
# Based on MATLAB TestProblem.m: all forcing terms are zero
for eq_idx in range(4):
    ooc_problem.set_force(eq_idx, lambda x, t: np.zeros_like(x))

# Non-zero forcing example: time-dependent source
def time_dependent_source(x, t):
    return 0.1 * np.exp(-t) * np.sin(np.pi * x)

ks_problem.set_force(0, time_dependent_source)
ks_problem.set_force(1, lambda x, t: np.zeros_like(x))  # No source for phi
\end{lstlisting}

\paragraph{get\_force()}
\begin{lstlisting}[language=Python, caption=Get Force Method]
def get_force(self, equation_index: int,
              coordinates: np.ndarray,
              time: float) -> np.ndarray
\end{lstlisting}

\textbf{Parameters:}
\begin{itemize}
    \item \texttt{equation\_index}: Equation number
    \item \texttt{coordinates}: Spatial coordinate array
    \item \texttt{time}: Time value
\end{itemize}

\textbf{Returns:} \texttt{np.ndarray} - Force values at given coordinates and time

\textbf{Usage:}
\begin{lstlisting}[language=Python]
# Evaluate forcing term at current time
x_points = np.linspace(0, 1, 11)
current_time = 0.5
f_values = ooc_problem.get_force(0, x_points, current_time)
\end{lstlisting}

\subsubsection{Analytical Solution Management}

\paragraph{set\_solution()}
\begin{lstlisting}[language=Python, caption=Set Solution Method]
def set_solution(self, equation_index: int,
                 solution_function: Callable[[np.ndarray, float], np.ndarray])
\end{lstlisting}

\textbf{Parameters:}
\begin{itemize}
    \item \texttt{equation\_index}: Equation number
    \item \texttt{solution\_function}: Analytical solution function \texttt{u(x, t)}
\end{itemize}

\textbf{Usage:}
\begin{lstlisting}[language=Python, caption=Analytical Solution Example]
# Set analytical solution for validation (when known)
def analytical_u(x, t):
    return np.exp(-t) * np.sin(np.pi * x)

def analytical_phi(x, t):
    return np.ones_like(x) + 0.1 * np.cos(np.pi * x) * np.exp(-0.5 * t)

ks_problem.set_solution(0, analytical_u)
ks_problem.set_solution(1, analytical_phi)
\end{lstlisting}

\paragraph{get\_solution()}
\begin{lstlisting}[language=Python, caption=Get Solution Method]
def get_solution(self, equation_index: int,
                 coordinates: np.ndarray,
                 time: float) -> np.ndarray
\end{lstlisting}

\textbf{Parameters:}
\begin{itemize}
    \item \texttt{equation\_index}: Equation number
    \item \texttt{coordinates}: Spatial coordinate array
    \item \texttt{time}: Time value
\end{itemize}

\textbf{Returns:} \texttt{np.ndarray} - Analytical solution values

\textbf{Usage:}
\begin{lstlisting}[language=Python]
# Compare numerical with analytical solution
x_test = np.linspace(0, 1, 21)
t_test = 1.0
analytical_values = ks_problem.get_solution(0, x_test, t_test)
# numerical_values = ... (from solver)
# error = np.abs(numerical_values - analytical_values)
\end{lstlisting}

\subsubsection{Problem-Specific Extensions}

\paragraph{set\_chemotaxis() - Keller-Segel Specific}
\begin{lstlisting}[language=Python, caption=Set Chemotaxis Functions]
def set_chemotaxis(self, chi_function: Callable[[np.ndarray], np.ndarray],
                   dchi_function: Callable[[np.ndarray], np.ndarray])
\end{lstlisting}

\textbf{Parameters:}
\begin{itemize}
    \item \texttt{chi\_function}: Chemotactic sensitivity function $\chi(\phi)$
    \item \texttt{dchi\_function}: Derivative function $\chi'(\phi)$
\end{itemize}

\textbf{Usage:}
\begin{lstlisting}[language=Python, caption=Chemotaxis Function Example]
# Nonlinear chemotaxis with saturation
def chi_function(phi):
    k1, k2 = 3.9e-9, 5e-6
    return k1 / (k2 + phi)**2

def dchi_function(phi):
    k1, k2 = 3.9e-9, 5e-6
    return -2 * k1 / (k2 + phi)**3

ks_problem.set_chemotaxis(chi_function, dchi_function)
\end{lstlisting}

\paragraph{set\_lambda\_function() - OrganOnChip Specific}
\begin{lstlisting}[language=Python, caption=Set Lambda Function]
def set_lambda_function(self, lambda_function: Callable[[np.ndarray], np.ndarray],
                       dlambda_function: Callable[[np.ndarray], np.ndarray])
\end{lstlisting}

\textbf{Parameters:}
\begin{itemize}
    \item \texttt{lambda\_function}: Nonlinear response function $\lambda(\omega)$
    \item \texttt{dlambda\_function}: Derivative function $\lambda'(\omega)$
\end{itemize}

\textbf{Usage (from MATLAB TestProblem.m):}
\begin{lstlisting}[language=Python, caption=Lambda Function Example]
# Constant lambda function (as in MATLAB reference)
def constant_lambda(omega):
    return np.ones_like(omega)

def constant_dlambda(omega):
    return np.zeros_like(omega)

ooc_problem.set_lambda_function(constant_lambda, constant_dlambda)
\end{lstlisting}

\subsubsection{Utility Methods}

\paragraph{validate\_setup()}
\begin{lstlisting}[language=Python, caption=Validate Setup Method]
def validate_setup(self) -> bool
\end{lstlisting}

\textbf{Returns:} \texttt{bool} - True if problem setup is complete and consistent

\textbf{Validation Checks:}
\begin{itemize}
    \item All equations have initial conditions defined
    \item All equations have forcing functions defined
    \item Parameter array has correct length for problem type
    \item Domain geometry is valid (positive length)
    \item Problem-specific functions are set when required
\end{itemize}

\textbf{Usage:}
\begin{lstlisting}[language=Python]
if ooc_problem.validate_setup():
    print("Problem setup is complete")
else:
    print("Problem setup incomplete - check missing components")
\end{lstlisting}

\paragraph{summary()}
\begin{lstlisting}[language=Python, caption=Problem Summary Method]
def summary(self) -> str
\end{lstlisting}

\textbf{Returns:} \texttt{str} - Multi-line summary of problem configuration

\textbf{Usage:}
\begin{lstlisting}[language=Python]
print(ooc_problem.summary())
# Output:
# Problem: microfluidic_device
# Type: organ_on_chip
# Equations: 4
# Domain: [0.000, 1.000] (length: 1.000)
# Parameters: [1.0, 2.0, 1.0, 1.0, 0.0, 1.0, 0.0, 1.0, 1.0]
# Initial conditions: 4/4 set
# Forcing functions: 4/4 set
# Analytical solutions: 0/4 set
\end{lstlisting}

\paragraph{copy()}
\begin{lstlisting}[language=Python, caption=Problem Copy Method]
def copy(self) -> 'Problem'
\end{lstlisting}

\textbf{Returns:} \texttt{Problem} - Deep copy of the problem instance

\textbf{Usage:}
\begin{lstlisting}[language=Python]
# Create variant with different parameters
variant_problem = ooc_problem.copy()
variant_problem.parameters[0] = 2.0  # Change nu parameter
variant_problem.name = "high_viscosity_variant"
\end{lstlisting}

\subsection{Integration with BioNetFlux Components}
\label{subsec:problem_integration}

\subsubsection{Integration with Static Condensation}

\begin{lstlisting}[language=Python, caption=Static Condensation Integration]
from ooc1d.core.static_condensation_ooc import StaticCondensationOOC
from ooc1d.core.discretization import Discretization
from ooc1d.utils.elementary_matrices import ElementaryMatrices

# Create problem
problem = Problem(
    neq=4,
    domain_start=0.0,
    domain_length=1.0,
    parameters=np.array([1.0, 2.0, 1.0, 1.0, 0.0, 1.0, 0.0, 1.0, 1.0]),
    problem_type="organ_on_chip"
)

# Set up discretization
discretization = Discretization(
    n_elements=20,
    domain_start=problem.domain_start,
    domain_length=problem.domain_length,
    stab_constant=1.0
)

# Create static condensation
elementary_matrices = ElementaryMatrices()
static_condensation = StaticCondensationOOC(
    problem=problem,
    discretization=discretization, 
    elementary_matrices=elementary_matrices
)
\end{lstlisting}

\subsubsection{Integration with BulkDataManager}

\begin{lstlisting}[language=Python, caption=BulkDataManager Integration]
from ooc1d.solver.bulk_data_manager import BulkDataManager

# Initialize bulk data from problem
bulk_manager = BulkDataManager()
bulk_data = bulk_manager.create_bulk_data(
    problem=problem,
    discretization=discretization,
    time=0.0,  # Initial time
    initialization_method="initial_conditions"
)

# Compute forcing terms at specific time
force_data = bulk_manager.create_bulk_data(
    problem=problem,
    discretization=discretization, 
    time=0.5,  # Current time
    initialization_method="forcing_functions"
)
\end{lstlisting}

\subsection{Factory Methods and Problem Creation Patterns}
\label{subsec:problem_factories}

\subsubsection{Standard Problem Factory}

\begin{lstlisting}[language=Python, caption=Problem Factory Functions]
def create_keller_segel_problem(domain_length: float = 1.0,
                               mu: float = 2.0,
                               nu: float = 1.0,
                               chemotaxis_strength: float = 1.0) -> Problem:
    """Factory for Keller-Segel chemotaxis problems."""
    parameters = np.array([mu, nu, 0.0, chemotaxis_strength])
    
    problem = Problem(
        neq=2,
        domain_start=0.0,
        domain_length=domain_length,
        parameters=parameters,
        problem_type="keller_segel",
        name="keller_segel_chemotaxis"
    )
    
    # Set standard chemotaxis functions
    problem.set_chemotaxis(
        chi_function=lambda phi: np.ones_like(phi),
        dchi_function=lambda phi: np.zeros_like(phi)
    )
    
    return problem

def create_organ_on_chip_problem(domain_length: float = 1.0,
                                viscosity_params: Tuple[float, float] = (1.0, 2.0),
                                coupling_params: Tuple[float, float] = (1.0, 1.0)) -> Problem:
    """Factory for OrganOnChip problems matching MATLAB TestProblem.m."""
    nu, mu = viscosity_params
    epsilon, sigma = coupling_params
    
    # Parameters matching MATLAB: [nu, mu, epsilon, sigma, a, b, c, d, chi]
    parameters = np.array([nu, mu, epsilon, sigma, 0.0, 1.0, 0.0, 1.0, 1.0])
    
    problem = Problem(
        neq=4,
        domain_start=0.0,  # MATLAB: A = 0
        domain_length=domain_length,  # MATLAB: L = 1
        parameters=parameters,
        problem_type="organ_on_chip",
        name="organ_on_chip_device"
    )
    
    # Set constant lambda function (as in MATLAB)
    problem.set_lambda_function(
        lambda_function=lambda omega: np.ones_like(omega),
        dlambda_function=lambda omega: np.zeros_like(omega)
    )
    
    return problem
\end{lstlisting}

\subsubsection{Complete Problem Setup Example}

\begin{lstlisting}[language=Python, caption=Complete Problem Setup]
def setup_complete_ooc_problem():
    """Complete OrganOnChip problem setup matching MATLAB TestProblem.m."""
    
    # Create problem with MATLAB parameters
    problem = create_organ_on_chip_problem(domain_length=1.0)
    
    # Set initial conditions (matching MATLAB TestProblem.m)
    problem.set_initial_condition(0, lambda x, t: np.sin(2*np.pi*x))  # u
    problem.set_initial_condition(1, lambda x, t: np.zeros_like(x))   # omega
    problem.set_initial_condition(2, lambda x, t: np.zeros_like(x))   # v  
    problem.set_initial_condition(3, lambda x, t: np.zeros_like(x))   # phi
    
    # Set zero forcing functions (matching MATLAB)
    for eq_idx in range(4):
        problem.set_force(eq_idx, lambda x, t: np.zeros_like(x))
    
    # Validate setup
    if not problem.validate_setup():
        raise RuntimeError("Problem setup validation failed")
    
    print(f"✓ OrganOnChip problem setup complete:")
    print(problem.summary())
    
    return problem

# Usage
ooc_problem = setup_complete_ooc_problem()
\end{lstlisting}

\subsection{Advanced Usage Patterns}
\label{subsec:advanced_usage}

\subsubsection{Time-Dependent Parameters}

\begin{lstlisting}[language=Python, caption=Time-Dependent Problem Setup]
class TimeDependentProblem(Problem):
    """Extension for time-dependent parameters."""
    
    def __init__(self, *args, **kwargs):
        super().__init__(*args, **kwargs)
        self.parameter_functions = {}
    
    def set_time_dependent_parameter(self, param_index: int,
                                   param_function: Callable[[float], float]):
        """Set time-dependent parameter function."""
        self.parameter_functions[param_index] = param_function
    
    def get_parameters_at_time(self, time: float) -> np.ndarray:
        """Get parameters evaluated at specific time."""
        params = self.parameters.copy()
        for idx, func in self.parameter_functions.items():
            params[idx] = func(time)
        return params

# Usage example: time-dependent viscosity
td_problem = TimeDependentProblem(
    neq=4, domain_start=0.0, domain_length=1.0,
    parameters=np.array([1.0, 2.0, 1.0, 1.0, 0.0, 1.0, 0.0, 1.0, 1.0]),
    problem_type="organ_on_chip"
)

# Viscosity increases with time
td_problem.set_time_dependent_parameter(0, lambda t: 1.0 + 0.1*t)
\end{lstlisting}

The Problem class provides a comprehensive interface for mathematical problem specification in BioNetFlux, with full compatibility with both Keller-Segel and OrganOnChip systems as demonstrated in the MATLAB reference implementations.

% End of problem module API documentation
