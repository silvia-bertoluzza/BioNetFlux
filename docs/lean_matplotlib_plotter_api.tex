% Lean Matplotlib Plotter Module API Documentation (Template/Specification)
% To be included in master LaTeX document
%
% Usage: % Lean Matplotlib Plotter Module API Documentation (Template/Specification)
% To be included in master LaTeX document
%
% Usage: % Lean Matplotlib Plotter Module API Documentation (Template/Specification)
% To be included in master LaTeX document
%
% Usage: % Lean Matplotlib Plotter Module API Documentation (Template/Specification)
% To be included in master LaTeX document
%
% Usage: \input{docs/lean_matplotlib_plotter_api}

\section{Lean Matplotlib Plotter Module API Reference}
\label{sec:lean_matplotlib_plotter_api}

\textbf{Note:} This documentation describes the expected API for a lean matplotlib plotter module that follows the BioNetFlux lean architecture pattern. The module is not yet implemented but this serves as a specification.

\subsection{Module Overview}

The lean matplotlib plotter module would provide:
\begin{itemize}
    \item Lean plotting interface using external framework objects
    \item Specialized visualization for HDG trace and bulk solutions
    \item Multi-domain network visualization capabilities
    \item Time evolution animation and comparison plots
    \item Integration with BulkData and BulkDataManager
    \item Memory-efficient plotting without storing framework objects
\end{itemize}

\subsection{Expected Module Structure}

\begin{lstlisting}[language=Python, caption=Expected Module Dependencies]
# Expected imports for lean matplotlib plotter
import numpy as np
import matplotlib.pyplot as plt
from matplotlib.animation import FuncAnimation
from typing import List, Optional, Dict, Any, Tuple
from pathlib import Path

# BioNetFlux imports
from ooc1d.core.bulk_data import BulkData
from ooc1d.core.lean_bulk_data_manager import BulkDataManager
from ooc1d.core.discretization import Discretization
from ooc1d.core.problem import Problem
\end{lstlisting}

\subsection{Expected LeanMatplotlibPlotter Class}
\label{subsec:lean_matplotlib_plotter_class}

\subsubsection{Expected Constructor}

\paragraph{\_\_init\_\_()}\leavevmode
\begin{lstlisting}[language=Python, caption=Expected Constructor Signature]
def __init__(self, 
             figsize: Tuple[float, float] = (12, 8),
             style: str = 'seaborn-v0_8',
             dpi: int = 100)
\end{lstlisting}

\textbf{Parameters:}
\begin{itemize}
    \item \texttt{figsize}: Default figure size for plots (default: (12, 8))
    \item \texttt{style}: Matplotlib style to use (default: 'seaborn-v0\_8')
    \item \texttt{dpi}: Resolution for saved figures (default: 100)
\end{itemize}

\textbf{Expected Attributes:}
\begin{longtable}{|p{3cm}|p{4cm}|p{7cm}|}
\hline
\textbf{Attribute} & \textbf{Type} & \textbf{Description} \\
\hline
\endhead

\texttt{figsize} & \texttt{Tuple[float, float]} & Default figure dimensions \\
\hline

\texttt{style} & \texttt{str} & Matplotlib style configuration \\
\hline

\texttt{dpi} & \texttt{int} & Default resolution for saved figures \\
\hline

\texttt{color\_palette} & \texttt{List[str]} & Color scheme for multi-domain plots \\
\hline

\texttt{plot\_config} & \texttt{Dict} & Default plotting configuration options \\
\hline

\end{longtable}

\subsubsection{Expected Core Plotting Methods}

\paragraph{plot\_bulk\_solution()}\leavevmode
\begin{lstlisting}[language=Python, caption=Expected Bulk Solution Plotting Method]
def plot_bulk_solution(self,
                       bulk_data: BulkData,
                       discretization: Discretization,
                       problem: Problem,
                       domain_idx: int = 0,
                       equations: Optional[List[int]] = None,
                       title: Optional[str] = None,
                       savepath: Optional[Path] = None) -> plt.Figure
\end{lstlisting}

\textbf{Parameters:}
\begin{itemize}
    \item \texttt{bulk\_data}: BulkData instance containing solution
    \item \texttt{discretization}: Discretization for spatial coordinates
    \item \texttt{problem}: Problem instance for metadata and labeling
    \item \texttt{domain\_idx}: Domain index for title generation (default: 0)
    \item \texttt{equations}: List of equation indices to plot (default: all)
    \item \texttt{title}: Custom plot title (default: auto-generated)
    \item \texttt{savepath}: Path to save figure (default: display only)
\end{itemize}

\textbf{Returns:} \texttt{plt.Figure} - Matplotlib figure object

\textbf{Expected Usage:}
\begin{lstlisting}[language=Python, caption=Bulk Solution Plotting Usage]
from ooc1d.utils.lean_matplotlib_plotter import LeanMatplotlibPlotter

plotter = LeanMatplotlibPlotter()

# Plot all equations
fig = plotter.plot_bulk_solution(
    bulk_data=bulk_solution,
    discretization=discretization,
    problem=problem,
    domain_idx=0,
    title="Bulk Solution at t=1.0"
)

# Plot specific equations only
fig = plotter.plot_bulk_solution(
    bulk_data=bulk_solution,
    discretization=discretization,
    problem=problem,
    equations=[0, 2],  # Only plot equations 0 and 2
    savepath=Path("results/bulk_solution.png")
)
\end{lstlisting}

\paragraph{plot\_trace\_solution()}\leavevmode
\begin{lstlisting}[language=Python, caption=Expected Trace Solution Plotting Method]
def plot_trace_solution(self,
                        trace_solution: np.ndarray,
                        discretization: Discretization,
                        problem: Problem,
                        domain_idx: int = 0,
                        equations: Optional[List[int]] = None,
                        title: Optional[str] = None,
                        savepath: Optional[Path] = None) -> plt.Figure
\end{lstlisting}

\textbf{Parameters:}
\begin{itemize}
    \item \texttt{trace\_solution}: Array of trace values (neq*(N+1),)
    \item \texttt{discretization}: Discretization for node coordinates
    \item \texttt{problem}: Problem instance for metadata
    \item \texttt{domain\_idx}: Domain index for labeling (default: 0)
    \item \texttt{equations}: Equation indices to plot (default: all)
    \item \texttt{title}: Custom title (default: auto-generated)
    \item \texttt{savepath}: Save path (default: display only)
\end{itemize}

\textbf{Returns:} \texttt{plt.Figure} - Matplotlib figure object

\paragraph{plot\_multi\_domain\_network()}\leavevmode
\begin{lstlisting}[language=Python, caption=Expected Multi-Domain Network Plotting Method]
def plot_multi_domain_network(self,
                              bulk_data_list: List[BulkData],
                              problems: List[Problem],
                              discretizations: List[Discretization],
                              constraint_manager = None,
                              equation_idx: int = 0,
                              time: float = 0.0,
                              title: Optional[str] = None,
                              savepath: Optional[Path] = None) -> plt.Figure
\end{lstlisting}

\textbf{Parameters:}
\begin{itemize}
    \item \texttt{bulk\_data\_list}: List of BulkData for each domain
    \item \texttt{problems}: List of Problem instances
    \item \texttt{discretizations}: List of Discretization instances
    \item \texttt{constraint\_manager}: ConstraintManager for junction visualization
    \item \texttt{equation\_idx}: Which equation to visualize (default: 0)
    \item \texttt{time}: Current time for title (default: 0.0)
    \item \texttt{title}: Custom title (default: auto-generated)
    \item \texttt{savepath}: Save path (default: display only)
\end{itemize}

\textbf{Returns:} \texttt{plt.Figure} - Matplotlib figure with network visualization

\textbf{Expected Network Visualization Features:}
\begin{itemize}
    \item Different colors for each domain
    \item Junction points marked clearly
    \item Constraint type indicators (continuity, flux jump, etc.)
    \item Domain boundaries and connections
    \item Equation-specific color mapping
\end{itemize}

\subsubsection{Expected Comparison and Analysis Methods}

\paragraph{plot\_solution\_comparison()}\leavevmode
\begin{lstlisting}[language=Python, caption=Expected Solution Comparison Method]
def plot_solution_comparison(self,
                            solutions: Dict[str, BulkData],
                            discretization: Discretization,
                            problem: Problem,
                            domain_idx: int = 0,
                            equation_idx: int = 0,
                            title: Optional[str] = None,
                            savepath: Optional[Path] = None) -> plt.Figure
\end{lstlisting}

\textbf{Parameters:}
\begin{itemize}
    \item \texttt{solutions}: Dictionary mapping labels to BulkData solutions
    \item \texttt{discretization}: Discretization for coordinates
    \item \texttt{problem}: Problem instance
    \item \texttt{domain\_idx}: Domain index (default: 0)
    \item \texttt{equation\_idx}: Equation index to compare (default: 0)
    \item \texttt{title}: Custom title (default: auto-generated)
    \item \texttt{savepath}: Save path (default: display only)
\end{itemize}

\textbf{Returns:} \texttt{plt.Figure} - Comparison plot figure

\textbf{Expected Usage:}
\begin{lstlisting}[language=Python, caption=Solution Comparison Usage]
solutions = {
    "Initial": initial_bulk_data,
    "t=0.5": intermediate_bulk_data,
    "Final": final_bulk_data,
    "Analytical": analytical_bulk_data
}

fig = plotter.plot_solution_comparison(
    solutions=solutions,
    discretization=discretization,
    problem=problem,
    equation_idx=0,
    title="Solution Evolution Comparison"
)
\end{lstlisting}

\paragraph{plot\_mass\_conservation()}\leavevmode
\begin{lstlisting}[language=Python, caption=Expected Mass Conservation Plotting Method]
def plot_mass_conservation(self,
                          mass_history: List[float],
                          time_points: np.ndarray,
                          relative: bool = True,
                          title: Optional[str] = None,
                          savepath: Optional[Path] = None) -> plt.Figure
\end{lstlisting}

\textbf{Parameters:}
\begin{itemize}
    \item \texttt{mass\_history}: List of total mass values over time
    \item \texttt{time\_points}: Corresponding time points
    \item \texttt{relative}: Plot relative change if True, absolute values if False (default: True)
    \item \texttt{title}: Custom title (default: auto-generated)
    \item \texttt{savepath}: Save path (default: display only)
\end{itemize}

\textbf{Returns:} \texttt{plt.Figure} - Mass conservation plot

\subsubsection{Expected Animation Methods}

\paragraph{create\_time\_evolution\_animation()}\leavevmode
\begin{lstlisting}[language=Python, caption=Expected Animation Creation Method]
def create_time_evolution_animation(self,
                                   bulk_data_history: List[List[BulkData]],
                                   problems: List[Problem],
                                   discretizations: List[Discretization],
                                   time_points: np.ndarray,
                                   equation_idx: int = 0,
                                   fps: int = 10,
                                   title_template: str = "t = {:.3f}",
                                   savepath: Optional[Path] = None) -> FuncAnimation
\end{lstlisting}

\textbf{Parameters:}
\begin{itemize}
    \item \texttt{bulk\_data\_history}: List of BulkData lists for each time step
    \item \texttt{problems}: List of Problem instances
    \item \texttt{discretizations}: List of Discretization instances
    \item \texttt{time\_points}: Time values for each frame
    \item \texttt{equation\_idx}: Which equation to animate (default: 0)
    \item \texttt{fps}: Frames per second (default: 10)
    \item \texttt{title\_template}: Title format string (default: "t = \{:.3f\}")
    \item \texttt{savepath}: Path to save animation (default: display only)
\end{itemize}

\textbf{Returns:} \texttt{FuncAnimation} - Matplotlib animation object

\subsubsection{Expected Convergence and Error Analysis Methods}

\paragraph{plot\_newton\_convergence()}\leavevmode
\begin{lstlisting}[language=Python, caption=Expected Newton Convergence Plotting Method]
def plot_newton_convergence(self,
                           residual_history: List[float],
                           tolerance: float = 1e-10,
                           log_scale: bool = True,
                           title: Optional[str] = None,
                           savepath: Optional[Path] = None) -> plt.Figure
\end{lstlisting}

\textbf{Parameters:}
\begin{itemize}
    \item \texttt{residual\_history}: List of residual norms from Newton iterations
    \item \texttt{tolerance}: Convergence tolerance to mark on plot (default: 1e-10)
    \item \texttt{log\_scale}: Use logarithmic y-axis (default: True)
    \item \texttt{title}: Custom title (default: auto-generated)
    \item \texttt{savepath}: Save path (default: display only)
\end{itemize}

\textbf{Returns:} \texttt{plt.Figure} - Convergence history plot

\paragraph{plot\_error\_analysis()}\leavevmode
\begin{lstlisting}[language=Python, caption=Expected Error Analysis Method]
def plot_error_analysis(self,
                       numerical_solution: BulkData,
                       analytical_solution: BulkData,
                       discretization: Discretization,
                       problem: Problem,
                       equation_idx: int = 0,
                       error_type: str = 'absolute',
                       title: Optional[str] = None,
                       savepath: Optional[Path] = None) -> plt.Figure
\end{lstlisting}

\textbf{Parameters:}
\begin{itemize}
    \item \texttt{numerical\_solution}: Computed BulkData solution
    \item \texttt{analytical\_solution}: Reference BulkData solution
    \item \texttt{discretization}: Discretization for coordinates
    \item \texttt{problem}: Problem instance
    \item \texttt{equation\_idx}: Equation index to analyze (default: 0)
    \item \texttt{error\_type}: 'absolute', 'relative', or 'both' (default: 'absolute')
    \item \texttt{title}: Custom title (default: auto-generated)
    \item \texttt{savepath}: Save path (default: display only)
\end{itemize}

\textbf{Returns:} \texttt{plt.Figure} - Error analysis plot with subplots

\subsubsection{Expected Utility Methods}

\paragraph{set\_default\_style()}\leavevmode
\begin{lstlisting}[language=Python, caption=Expected Style Configuration Method]
def set_default_style(self,
                     style: str = 'seaborn-v0_8',
                     color_palette: Optional[List[str]] = None,
                     font_size: int = 12) -> None
\end{lstlisting}

\textbf{Parameters:}
\begin{itemize}
    \item \texttt{style}: Matplotlib style name (default: 'seaborn-v0\_8')
    \item \texttt{color\_palette}: Custom color scheme (default: automatic)
    \item \texttt{font\_size}: Base font size (default: 12)
\end{itemize}

\textbf{Side Effects:} Updates default plotting configuration

\paragraph{save\_all\_figures()}\leavevmode
\begin{lstlisting}[language=Python, caption=Expected Batch Save Method]
def save_all_figures(self,
                    figures: List[plt.Figure],
                    base_path: Path,
                    prefix: str = "figure",
                    format: str = "png") -> List[Path]
\end{lstlisting}

\textbf{Parameters:}
\begin{itemize}
    \item \texttt{figures}: List of matplotlib Figure objects
    \item \texttt{base\_path}: Directory to save figures
    \item \texttt{prefix}: Filename prefix (default: "figure")
    \item \texttt{format}: File format (default: "png")
\end{itemize}

\textbf{Returns:} \texttt{List[Path]} - List of saved file paths

\subsection{Expected Complete Usage Examples}
\label{subsec:expected_complete_usage_examples}

\subsubsection{Single Domain Visualization}

\begin{lstlisting}[language=Python, caption=Expected Single Domain Usage]
from ooc1d.utils.lean_matplotlib_plotter import LeanMatplotlibPlotter
from ooc1d.core.bulk_data import BulkData
from pathlib import Path

# Create plotter
plotter = LeanMatplotlibPlotter(figsize=(14, 10))

# Load or compute solution
bulk_solution = BulkData(problem, discretization, dual=False)
# ... populate bulk_solution with data ...

# Plot bulk solution
fig1 = plotter.plot_bulk_solution(
    bulk_data=bulk_solution,
    discretization=discretization,
    problem=problem,
    title="Keller-Segel Solution at t=1.0",
    savepath=Path("results/keller_segel_bulk.png")
)

# Plot trace solution
trace_values = bulk_solution.get_trace_values()
fig2 = plotter.plot_trace_solution(
    trace_solution=trace_values,
    discretization=discretization,
    problem=problem,
    equations=[0, 1],  # Cell density and chemical concentration
    title="HDG Trace Values",
    savepath=Path("results/keller_segel_trace.png")
)

print("Single domain plots created successfully")
\end{lstlisting}

\subsubsection{Multi-Domain Network Visualization}

\begin{lstlisting}[language=Python, caption=Expected Multi-Domain Usage]
# Multi-domain network setup
problems = [problem1, problem2, problem3]  # Three domains
discretizations = [disc1, disc2, disc3]
bulk_data_list = [bulk1, bulk2, bulk3]

# Visualize network solution
fig = plotter.plot_multi_domain_network(
    bulk_data_list=bulk_data_list,
    problems=problems,
    discretizations=discretizations,
    constraint_manager=constraint_manager,
    equation_idx=0,  # Cell density
    time=2.5,
    title="Vascular Network Cell Migration",
    savepath=Path("results/network_solution.png")
)

# Create animation of time evolution
# Assume we have solutions at multiple time points
bulk_history = []  # List[List[BulkData]] for each time step
time_points = np.linspace(0, 5, 50)

animation = plotter.create_time_evolution_animation(
    bulk_data_history=bulk_history,
    problems=problems,
    discretizations=discretizations,
    time_points=time_points,
    equation_idx=0,
    fps=5,
    savepath=Path("results/network_evolution.mp4")
)

print("Multi-domain visualization completed")
\end{lstlisting}

\subsubsection{Analysis and Comparison Workflows}

\begin{lstlisting}[language=Python, caption=Expected Analysis Workflow]
# Comparison of different solutions
solutions_dict = {
    "Initial Condition": initial_bulk,
    "Numerical t=1": numerical_bulk_t1,
    "Numerical t=2": numerical_bulk_t2,
    "Analytical": analytical_bulk
}

fig_comp = plotter.plot_solution_comparison(
    solutions=solutions_dict,
    discretization=discretization,
    problem=problem,
    equation_idx=0,
    title="Solution Evolution Comparison",
    savepath=Path("results/solution_comparison.png")
)

# Error analysis
fig_error = plotter.plot_error_analysis(
    numerical_solution=numerical_bulk,
    analytical_solution=analytical_bulk,
    discretization=discretization,
    problem=problem,
    equation_idx=0,
    error_type='both',
    title="Error Analysis vs Analytical Solution",
    savepath=Path("results/error_analysis.png")
)

# Newton convergence analysis
residual_norms = [1e-1, 1e-3, 1e-6, 1e-9, 1e-12]  # Example data
fig_newton = plotter.plot_newton_convergence(
    residual_history=residual_norms,
    tolerance=1e-10,
    title="Newton Solver Convergence",
    savepath=Path("results/newton_convergence.png")
)

# Mass conservation tracking
mass_values = [100.0, 99.99, 99.98, 99.99, 100.01]  # Example
time_vals = np.linspace(0, 2, len(mass_values))

fig_mass = plotter.plot_mass_conservation(
    mass_history=mass_values,
    time_points=time_vals,
    relative=True,
    title="Mass Conservation Check",
    savepath=Path("results/mass_conservation.png")
)

print("Analysis workflow completed")
\end{lstlisting}

\subsection{Expected Integration with BioNetFlux Framework}

\begin{lstlisting}[language=Python, caption=Expected Framework Integration]
from setup_solver import quick_setup

# Setup complete solver framework
setup = quick_setup("ooc1d.problems.keller_segel_network")

# Create plotter
plotter = LeanMatplotlibPlotter()

# Initial conditions
trace_solutions, multipliers = setup.create_initial_conditions()
global_solution = setup.create_global_solution_vector(trace_solutions, multipliers)

# Initialize bulk data
bulk_data_list = setup.bulk_data_manager.initialize_all_bulk_data(
    setup.problems,
    setup.global_discretization.spatial_discretizations,
    time=0.0
)

# Plot initial network state
fig_initial = plotter.plot_multi_domain_network(
    bulk_data_list=bulk_data_list,
    problems=setup.problems,
    discretizations=setup.global_discretization.spatial_discretizations,
    constraint_manager=setup.constraint_manager,
    equation_idx=0,
    time=0.0,
    title="Initial Network State"
)

# Time evolution with plotting
time_history = []
bulk_history = []
mass_history = []

for time_step in range(setup.global_discretization.n_time_steps):
    current_time = time_step * setup.global_discretization.dt
    
    # Solve time step (simplified)
    # ... Newton iteration and solution update ...
    
    # Extract current bulk solutions
    current_bulk_list = setup.bulk_data_manager.initialize_all_bulk_data(
        setup.problems,
        setup.global_discretization.spatial_discretizations,
        time=current_time
    )
    
    # Track mass conservation
    total_mass = setup.bulk_data_manager.compute_total_mass(current_bulk_list)
    
    # Store for history
    time_history.append(current_time)
    bulk_history.append(current_bulk_list.copy())
    mass_history.append(total_mass)
    
    # Plot at specific intervals
    if time_step % 10 == 0:
        fig_step = plotter.plot_multi_domain_network(
            bulk_data_list=current_bulk_list,
            problems=setup.problems,
            discretizations=setup.global_discretization.spatial_discretizations,
            constraint_manager=setup.constraint_manager,
            equation_idx=0,
            time=current_time,
            savepath=Path(f"results/network_t_{time_step:04d}.png")
        )

# Final analysis plots
fig_mass_final = plotter.plot_mass_conservation(
    mass_history=mass_history,
    time_points=np.array(time_history),
    title="Mass Conservation Over Time"
)

# Create final animation
animation = plotter.create_time_evolution_animation(
    bulk_data_history=bulk_history,
    problems=setup.problems,
    discretizations=setup.global_discretization.spatial_discretizations,
    time_points=np.array(time_history),
    equation_idx=0,
    savepath=Path("results/complete_evolution.mp4")
)

print("Complete framework integration visualization finished")
\end{lstlisting}

\subsection{Expected Method Summary Table}
\label{subsec:expected_method_summary}

\begin{longtable}{|p{5cm}|p{3cm}|p{6.5cm}|}
\hline
\textbf{Expected Method} & \textbf{Returns} & \textbf{Purpose} \\
\hline
\endhead

\texttt{plot\_bulk\_solution} & \texttt{Figure} & Visualize bulk solution for single domain \\
\hline

\texttt{plot\_trace\_solution} & \texttt{Figure} & Plot HDG trace values at element boundaries \\
\hline

\texttt{plot\_multi\_domain\_network} & \texttt{Figure} & Visualize solution across network topology \\
\hline

\texttt{plot\_solution\_comparison} & \texttt{Figure} & Compare multiple solutions side-by-side \\
\hline

\texttt{plot\_mass\_conservation} & \texttt{Figure} & Track mass conservation over time \\
\hline

\texttt{create\_time\_animation} & \texttt{FuncAnimation} & Animate solution evolution \\
\hline

\texttt{plot\_newton\_convergence} & \texttt{Figure} & Show Newton solver convergence history \\
\hline

\texttt{plot\_error\_analysis} & \texttt{Figure} & Compare numerical vs analytical solutions \\
\hline

\texttt{set\_default\_style} & \texttt{None} & Configure plotting appearance \\
\hline

\texttt{save\_all\_figures} & \texttt{List[Path]} & Batch save multiple figures \\
\hline

\end{longtable}

\subsection{Expected Key Features}

\begin{itemize}
    \item \textbf{Lean Architecture}: No framework object storage, all passed as parameters
    \item \textbf{BulkData Integration}: Native support for BulkData solution format
    \item \textbf{Multi-Domain Networks}: Specialized visualization for connected domains
    \item \textbf{HDG-Specific Plots}: Trace and bulk solution visualization
    \item \textbf{Animation Support}: Time evolution and dynamic visualization
    \item \textbf{Analysis Tools}: Error analysis, convergence tracking, mass conservation
    \item \textbf{Flexible Styling}: Customizable appearance and output formats
    \item \textbf{Framework Integration}: Seamless integration with SolverSetup and BulkDataManager
\end{itemize}

\textbf{Note:} This documentation serves as a specification for implementing the lean matplotlib plotter module. The actual implementation would follow these interfaces and provide the described functionality.

% End of lean matplotlib plotter module API documentation (specification)


\section{Lean Matplotlib Plotter Module API Reference}
\label{sec:lean_matplotlib_plotter_api}

\textbf{Note:} This documentation describes the expected API for a lean matplotlib plotter module that follows the BioNetFlux lean architecture pattern. The module is not yet implemented but this serves as a specification.

\subsection{Module Overview}

The lean matplotlib plotter module would provide:
\begin{itemize}
    \item Lean plotting interface using external framework objects
    \item Specialized visualization for HDG trace and bulk solutions
    \item Multi-domain network visualization capabilities
    \item Time evolution animation and comparison plots
    \item Integration with BulkData and BulkDataManager
    \item Memory-efficient plotting without storing framework objects
\end{itemize}

\subsection{Expected Module Structure}

\begin{lstlisting}[language=Python, caption=Expected Module Dependencies]
# Expected imports for lean matplotlib plotter
import numpy as np
import matplotlib.pyplot as plt
from matplotlib.animation import FuncAnimation
from typing import List, Optional, Dict, Any, Tuple
from pathlib import Path

# BioNetFlux imports
from ooc1d.core.bulk_data import BulkData
from ooc1d.core.lean_bulk_data_manager import BulkDataManager
from ooc1d.core.discretization import Discretization
from ooc1d.core.problem import Problem
\end{lstlisting}

\subsection{Expected LeanMatplotlibPlotter Class}
\label{subsec:lean_matplotlib_plotter_class}

\subsubsection{Expected Constructor}

\paragraph{\_\_init\_\_()}\leavevmode
\begin{lstlisting}[language=Python, caption=Expected Constructor Signature]
def __init__(self, 
             figsize: Tuple[float, float] = (12, 8),
             style: str = 'seaborn-v0_8',
             dpi: int = 100)
\end{lstlisting}

\textbf{Parameters:}
\begin{itemize}
    \item \texttt{figsize}: Default figure size for plots (default: (12, 8))
    \item \texttt{style}: Matplotlib style to use (default: 'seaborn-v0\_8')
    \item \texttt{dpi}: Resolution for saved figures (default: 100)
\end{itemize}

\textbf{Expected Attributes:}
\begin{longtable}{|p{3cm}|p{4cm}|p{7cm}|}
\hline
\textbf{Attribute} & \textbf{Type} & \textbf{Description} \\
\hline
\endhead

\texttt{figsize} & \texttt{Tuple[float, float]} & Default figure dimensions \\
\hline

\texttt{style} & \texttt{str} & Matplotlib style configuration \\
\hline

\texttt{dpi} & \texttt{int} & Default resolution for saved figures \\
\hline

\texttt{color\_palette} & \texttt{List[str]} & Color scheme for multi-domain plots \\
\hline

\texttt{plot\_config} & \texttt{Dict} & Default plotting configuration options \\
\hline

\end{longtable}

\subsubsection{Expected Core Plotting Methods}

\paragraph{plot\_bulk\_solution()}\leavevmode
\begin{lstlisting}[language=Python, caption=Expected Bulk Solution Plotting Method]
def plot_bulk_solution(self,
                       bulk_data: BulkData,
                       discretization: Discretization,
                       problem: Problem,
                       domain_idx: int = 0,
                       equations: Optional[List[int]] = None,
                       title: Optional[str] = None,
                       savepath: Optional[Path] = None) -> plt.Figure
\end{lstlisting}

\textbf{Parameters:}
\begin{itemize}
    \item \texttt{bulk\_data}: BulkData instance containing solution
    \item \texttt{discretization}: Discretization for spatial coordinates
    \item \texttt{problem}: Problem instance for metadata and labeling
    \item \texttt{domain\_idx}: Domain index for title generation (default: 0)
    \item \texttt{equations}: List of equation indices to plot (default: all)
    \item \texttt{title}: Custom plot title (default: auto-generated)
    \item \texttt{savepath}: Path to save figure (default: display only)
\end{itemize}

\textbf{Returns:} \texttt{plt.Figure} - Matplotlib figure object

\textbf{Expected Usage:}
\begin{lstlisting}[language=Python, caption=Bulk Solution Plotting Usage]
from ooc1d.utils.lean_matplotlib_plotter import LeanMatplotlibPlotter

plotter = LeanMatplotlibPlotter()

# Plot all equations
fig = plotter.plot_bulk_solution(
    bulk_data=bulk_solution,
    discretization=discretization,
    problem=problem,
    domain_idx=0,
    title="Bulk Solution at t=1.0"
)

# Plot specific equations only
fig = plotter.plot_bulk_solution(
    bulk_data=bulk_solution,
    discretization=discretization,
    problem=problem,
    equations=[0, 2],  # Only plot equations 0 and 2
    savepath=Path("results/bulk_solution.png")
)
\end{lstlisting}

\paragraph{plot\_trace\_solution()}\leavevmode
\begin{lstlisting}[language=Python, caption=Expected Trace Solution Plotting Method]
def plot_trace_solution(self,
                        trace_solution: np.ndarray,
                        discretization: Discretization,
                        problem: Problem,
                        domain_idx: int = 0,
                        equations: Optional[List[int]] = None,
                        title: Optional[str] = None,
                        savepath: Optional[Path] = None) -> plt.Figure
\end{lstlisting}

\textbf{Parameters:}
\begin{itemize}
    \item \texttt{trace\_solution}: Array of trace values (neq*(N+1),)
    \item \texttt{discretization}: Discretization for node coordinates
    \item \texttt{problem}: Problem instance for metadata
    \item \texttt{domain\_idx}: Domain index for labeling (default: 0)
    \item \texttt{equations}: Equation indices to plot (default: all)
    \item \texttt{title}: Custom title (default: auto-generated)
    \item \texttt{savepath}: Save path (default: display only)
\end{itemize}

\textbf{Returns:} \texttt{plt.Figure} - Matplotlib figure object

\paragraph{plot\_multi\_domain\_network()}\leavevmode
\begin{lstlisting}[language=Python, caption=Expected Multi-Domain Network Plotting Method]
def plot_multi_domain_network(self,
                              bulk_data_list: List[BulkData],
                              problems: List[Problem],
                              discretizations: List[Discretization],
                              constraint_manager = None,
                              equation_idx: int = 0,
                              time: float = 0.0,
                              title: Optional[str] = None,
                              savepath: Optional[Path] = None) -> plt.Figure
\end{lstlisting}

\textbf{Parameters:}
\begin{itemize}
    \item \texttt{bulk\_data\_list}: List of BulkData for each domain
    \item \texttt{problems}: List of Problem instances
    \item \texttt{discretizations}: List of Discretization instances
    \item \texttt{constraint\_manager}: ConstraintManager for junction visualization
    \item \texttt{equation\_idx}: Which equation to visualize (default: 0)
    \item \texttt{time}: Current time for title (default: 0.0)
    \item \texttt{title}: Custom title (default: auto-generated)
    \item \texttt{savepath}: Save path (default: display only)
\end{itemize}

\textbf{Returns:} \texttt{plt.Figure} - Matplotlib figure with network visualization

\textbf{Expected Network Visualization Features:}
\begin{itemize}
    \item Different colors for each domain
    \item Junction points marked clearly
    \item Constraint type indicators (continuity, flux jump, etc.)
    \item Domain boundaries and connections
    \item Equation-specific color mapping
\end{itemize}

\subsubsection{Expected Comparison and Analysis Methods}

\paragraph{plot\_solution\_comparison()}\leavevmode
\begin{lstlisting}[language=Python, caption=Expected Solution Comparison Method]
def plot_solution_comparison(self,
                            solutions: Dict[str, BulkData],
                            discretization: Discretization,
                            problem: Problem,
                            domain_idx: int = 0,
                            equation_idx: int = 0,
                            title: Optional[str] = None,
                            savepath: Optional[Path] = None) -> plt.Figure
\end{lstlisting}

\textbf{Parameters:}
\begin{itemize}
    \item \texttt{solutions}: Dictionary mapping labels to BulkData solutions
    \item \texttt{discretization}: Discretization for coordinates
    \item \texttt{problem}: Problem instance
    \item \texttt{domain\_idx}: Domain index (default: 0)
    \item \texttt{equation\_idx}: Equation index to compare (default: 0)
    \item \texttt{title}: Custom title (default: auto-generated)
    \item \texttt{savepath}: Save path (default: display only)
\end{itemize}

\textbf{Returns:} \texttt{plt.Figure} - Comparison plot figure

\textbf{Expected Usage:}
\begin{lstlisting}[language=Python, caption=Solution Comparison Usage]
solutions = {
    "Initial": initial_bulk_data,
    "t=0.5": intermediate_bulk_data,
    "Final": final_bulk_data,
    "Analytical": analytical_bulk_data
}

fig = plotter.plot_solution_comparison(
    solutions=solutions,
    discretization=discretization,
    problem=problem,
    equation_idx=0,
    title="Solution Evolution Comparison"
)
\end{lstlisting}

\paragraph{plot\_mass\_conservation()}\leavevmode
\begin{lstlisting}[language=Python, caption=Expected Mass Conservation Plotting Method]
def plot_mass_conservation(self,
                          mass_history: List[float],
                          time_points: np.ndarray,
                          relative: bool = True,
                          title: Optional[str] = None,
                          savepath: Optional[Path] = None) -> plt.Figure
\end{lstlisting}

\textbf{Parameters:}
\begin{itemize}
    \item \texttt{mass\_history}: List of total mass values over time
    \item \texttt{time\_points}: Corresponding time points
    \item \texttt{relative}: Plot relative change if True, absolute values if False (default: True)
    \item \texttt{title}: Custom title (default: auto-generated)
    \item \texttt{savepath}: Save path (default: display only)
\end{itemize}

\textbf{Returns:} \texttt{plt.Figure} - Mass conservation plot

\subsubsection{Expected Animation Methods}

\paragraph{create\_time\_evolution\_animation()}\leavevmode
\begin{lstlisting}[language=Python, caption=Expected Animation Creation Method]
def create_time_evolution_animation(self,
                                   bulk_data_history: List[List[BulkData]],
                                   problems: List[Problem],
                                   discretizations: List[Discretization],
                                   time_points: np.ndarray,
                                   equation_idx: int = 0,
                                   fps: int = 10,
                                   title_template: str = "t = {:.3f}",
                                   savepath: Optional[Path] = None) -> FuncAnimation
\end{lstlisting}

\textbf{Parameters:}
\begin{itemize}
    \item \texttt{bulk\_data\_history}: List of BulkData lists for each time step
    \item \texttt{problems}: List of Problem instances
    \item \texttt{discretizations}: List of Discretization instances
    \item \texttt{time\_points}: Time values for each frame
    \item \texttt{equation\_idx}: Which equation to animate (default: 0)
    \item \texttt{fps}: Frames per second (default: 10)
    \item \texttt{title\_template}: Title format string (default: "t = \{:.3f\}")
    \item \texttt{savepath}: Path to save animation (default: display only)
\end{itemize}

\textbf{Returns:} \texttt{FuncAnimation} - Matplotlib animation object

\subsubsection{Expected Convergence and Error Analysis Methods}

\paragraph{plot\_newton\_convergence()}\leavevmode
\begin{lstlisting}[language=Python, caption=Expected Newton Convergence Plotting Method]
def plot_newton_convergence(self,
                           residual_history: List[float],
                           tolerance: float = 1e-10,
                           log_scale: bool = True,
                           title: Optional[str] = None,
                           savepath: Optional[Path] = None) -> plt.Figure
\end{lstlisting}

\textbf{Parameters:}
\begin{itemize}
    \item \texttt{residual\_history}: List of residual norms from Newton iterations
    \item \texttt{tolerance}: Convergence tolerance to mark on plot (default: 1e-10)
    \item \texttt{log\_scale}: Use logarithmic y-axis (default: True)
    \item \texttt{title}: Custom title (default: auto-generated)
    \item \texttt{savepath}: Save path (default: display only)
\end{itemize}

\textbf{Returns:} \texttt{plt.Figure} - Convergence history plot

\paragraph{plot\_error\_analysis()}\leavevmode
\begin{lstlisting}[language=Python, caption=Expected Error Analysis Method]
def plot_error_analysis(self,
                       numerical_solution: BulkData,
                       analytical_solution: BulkData,
                       discretization: Discretization,
                       problem: Problem,
                       equation_idx: int = 0,
                       error_type: str = 'absolute',
                       title: Optional[str] = None,
                       savepath: Optional[Path] = None) -> plt.Figure
\end{lstlisting}

\textbf{Parameters:}
\begin{itemize}
    \item \texttt{numerical\_solution}: Computed BulkData solution
    \item \texttt{analytical\_solution}: Reference BulkData solution
    \item \texttt{discretization}: Discretization for coordinates
    \item \texttt{problem}: Problem instance
    \item \texttt{equation\_idx}: Equation index to analyze (default: 0)
    \item \texttt{error\_type}: 'absolute', 'relative', or 'both' (default: 'absolute')
    \item \texttt{title}: Custom title (default: auto-generated)
    \item \texttt{savepath}: Save path (default: display only)
\end{itemize}

\textbf{Returns:} \texttt{plt.Figure} - Error analysis plot with subplots

\subsubsection{Expected Utility Methods}

\paragraph{set\_default\_style()}\leavevmode
\begin{lstlisting}[language=Python, caption=Expected Style Configuration Method]
def set_default_style(self,
                     style: str = 'seaborn-v0_8',
                     color_palette: Optional[List[str]] = None,
                     font_size: int = 12) -> None
\end{lstlisting}

\textbf{Parameters:}
\begin{itemize}
    \item \texttt{style}: Matplotlib style name (default: 'seaborn-v0\_8')
    \item \texttt{color\_palette}: Custom color scheme (default: automatic)
    \item \texttt{font\_size}: Base font size (default: 12)
\end{itemize}

\textbf{Side Effects:} Updates default plotting configuration

\paragraph{save\_all\_figures()}\leavevmode
\begin{lstlisting}[language=Python, caption=Expected Batch Save Method]
def save_all_figures(self,
                    figures: List[plt.Figure],
                    base_path: Path,
                    prefix: str = "figure",
                    format: str = "png") -> List[Path]
\end{lstlisting}

\textbf{Parameters:}
\begin{itemize}
    \item \texttt{figures}: List of matplotlib Figure objects
    \item \texttt{base\_path}: Directory to save figures
    \item \texttt{prefix}: Filename prefix (default: "figure")
    \item \texttt{format}: File format (default: "png")
\end{itemize}

\textbf{Returns:} \texttt{List[Path]} - List of saved file paths

\subsection{Expected Complete Usage Examples}
\label{subsec:expected_complete_usage_examples}

\subsubsection{Single Domain Visualization}

\begin{lstlisting}[language=Python, caption=Expected Single Domain Usage]
from ooc1d.utils.lean_matplotlib_plotter import LeanMatplotlibPlotter
from ooc1d.core.bulk_data import BulkData
from pathlib import Path

# Create plotter
plotter = LeanMatplotlibPlotter(figsize=(14, 10))

# Load or compute solution
bulk_solution = BulkData(problem, discretization, dual=False)
# ... populate bulk_solution with data ...

# Plot bulk solution
fig1 = plotter.plot_bulk_solution(
    bulk_data=bulk_solution,
    discretization=discretization,
    problem=problem,
    title="Keller-Segel Solution at t=1.0",
    savepath=Path("results/keller_segel_bulk.png")
)

# Plot trace solution
trace_values = bulk_solution.get_trace_values()
fig2 = plotter.plot_trace_solution(
    trace_solution=trace_values,
    discretization=discretization,
    problem=problem,
    equations=[0, 1],  # Cell density and chemical concentration
    title="HDG Trace Values",
    savepath=Path("results/keller_segel_trace.png")
)

print("Single domain plots created successfully")
\end{lstlisting}

\subsubsection{Multi-Domain Network Visualization}

\begin{lstlisting}[language=Python, caption=Expected Multi-Domain Usage]
# Multi-domain network setup
problems = [problem1, problem2, problem3]  # Three domains
discretizations = [disc1, disc2, disc3]
bulk_data_list = [bulk1, bulk2, bulk3]

# Visualize network solution
fig = plotter.plot_multi_domain_network(
    bulk_data_list=bulk_data_list,
    problems=problems,
    discretizations=discretizations,
    constraint_manager=constraint_manager,
    equation_idx=0,  # Cell density
    time=2.5,
    title="Vascular Network Cell Migration",
    savepath=Path("results/network_solution.png")
)

# Create animation of time evolution
# Assume we have solutions at multiple time points
bulk_history = []  # List[List[BulkData]] for each time step
time_points = np.linspace(0, 5, 50)

animation = plotter.create_time_evolution_animation(
    bulk_data_history=bulk_history,
    problems=problems,
    discretizations=discretizations,
    time_points=time_points,
    equation_idx=0,
    fps=5,
    savepath=Path("results/network_evolution.mp4")
)

print("Multi-domain visualization completed")
\end{lstlisting}

\subsubsection{Analysis and Comparison Workflows}

\begin{lstlisting}[language=Python, caption=Expected Analysis Workflow]
# Comparison of different solutions
solutions_dict = {
    "Initial Condition": initial_bulk,
    "Numerical t=1": numerical_bulk_t1,
    "Numerical t=2": numerical_bulk_t2,
    "Analytical": analytical_bulk
}

fig_comp = plotter.plot_solution_comparison(
    solutions=solutions_dict,
    discretization=discretization,
    problem=problem,
    equation_idx=0,
    title="Solution Evolution Comparison",
    savepath=Path("results/solution_comparison.png")
)

# Error analysis
fig_error = plotter.plot_error_analysis(
    numerical_solution=numerical_bulk,
    analytical_solution=analytical_bulk,
    discretization=discretization,
    problem=problem,
    equation_idx=0,
    error_type='both',
    title="Error Analysis vs Analytical Solution",
    savepath=Path("results/error_analysis.png")
)

# Newton convergence analysis
residual_norms = [1e-1, 1e-3, 1e-6, 1e-9, 1e-12]  # Example data
fig_newton = plotter.plot_newton_convergence(
    residual_history=residual_norms,
    tolerance=1e-10,
    title="Newton Solver Convergence",
    savepath=Path("results/newton_convergence.png")
)

# Mass conservation tracking
mass_values = [100.0, 99.99, 99.98, 99.99, 100.01]  # Example
time_vals = np.linspace(0, 2, len(mass_values))

fig_mass = plotter.plot_mass_conservation(
    mass_history=mass_values,
    time_points=time_vals,
    relative=True,
    title="Mass Conservation Check",
    savepath=Path("results/mass_conservation.png")
)

print("Analysis workflow completed")
\end{lstlisting}

\subsection{Expected Integration with BioNetFlux Framework}

\begin{lstlisting}[language=Python, caption=Expected Framework Integration]
from setup_solver import quick_setup

# Setup complete solver framework
setup = quick_setup("ooc1d.problems.keller_segel_network")

# Create plotter
plotter = LeanMatplotlibPlotter()

# Initial conditions
trace_solutions, multipliers = setup.create_initial_conditions()
global_solution = setup.create_global_solution_vector(trace_solutions, multipliers)

# Initialize bulk data
bulk_data_list = setup.bulk_data_manager.initialize_all_bulk_data(
    setup.problems,
    setup.global_discretization.spatial_discretizations,
    time=0.0
)

# Plot initial network state
fig_initial = plotter.plot_multi_domain_network(
    bulk_data_list=bulk_data_list,
    problems=setup.problems,
    discretizations=setup.global_discretization.spatial_discretizations,
    constraint_manager=setup.constraint_manager,
    equation_idx=0,
    time=0.0,
    title="Initial Network State"
)

# Time evolution with plotting
time_history = []
bulk_history = []
mass_history = []

for time_step in range(setup.global_discretization.n_time_steps):
    current_time = time_step * setup.global_discretization.dt
    
    # Solve time step (simplified)
    # ... Newton iteration and solution update ...
    
    # Extract current bulk solutions
    current_bulk_list = setup.bulk_data_manager.initialize_all_bulk_data(
        setup.problems,
        setup.global_discretization.spatial_discretizations,
        time=current_time
    )
    
    # Track mass conservation
    total_mass = setup.bulk_data_manager.compute_total_mass(current_bulk_list)
    
    # Store for history
    time_history.append(current_time)
    bulk_history.append(current_bulk_list.copy())
    mass_history.append(total_mass)
    
    # Plot at specific intervals
    if time_step % 10 == 0:
        fig_step = plotter.plot_multi_domain_network(
            bulk_data_list=current_bulk_list,
            problems=setup.problems,
            discretizations=setup.global_discretization.spatial_discretizations,
            constraint_manager=setup.constraint_manager,
            equation_idx=0,
            time=current_time,
            savepath=Path(f"results/network_t_{time_step:04d}.png")
        )

# Final analysis plots
fig_mass_final = plotter.plot_mass_conservation(
    mass_history=mass_history,
    time_points=np.array(time_history),
    title="Mass Conservation Over Time"
)

# Create final animation
animation = plotter.create_time_evolution_animation(
    bulk_data_history=bulk_history,
    problems=setup.problems,
    discretizations=setup.global_discretization.spatial_discretizations,
    time_points=np.array(time_history),
    equation_idx=0,
    savepath=Path("results/complete_evolution.mp4")
)

print("Complete framework integration visualization finished")
\end{lstlisting}

\subsection{Expected Method Summary Table}
\label{subsec:expected_method_summary}

\begin{longtable}{|p{5cm}|p{3cm}|p{6.5cm}|}
\hline
\textbf{Expected Method} & \textbf{Returns} & \textbf{Purpose} \\
\hline
\endhead

\texttt{plot\_bulk\_solution} & \texttt{Figure} & Visualize bulk solution for single domain \\
\hline

\texttt{plot\_trace\_solution} & \texttt{Figure} & Plot HDG trace values at element boundaries \\
\hline

\texttt{plot\_multi\_domain\_network} & \texttt{Figure} & Visualize solution across network topology \\
\hline

\texttt{plot\_solution\_comparison} & \texttt{Figure} & Compare multiple solutions side-by-side \\
\hline

\texttt{plot\_mass\_conservation} & \texttt{Figure} & Track mass conservation over time \\
\hline

\texttt{create\_time\_animation} & \texttt{FuncAnimation} & Animate solution evolution \\
\hline

\texttt{plot\_newton\_convergence} & \texttt{Figure} & Show Newton solver convergence history \\
\hline

\texttt{plot\_error\_analysis} & \texttt{Figure} & Compare numerical vs analytical solutions \\
\hline

\texttt{set\_default\_style} & \texttt{None} & Configure plotting appearance \\
\hline

\texttt{save\_all\_figures} & \texttt{List[Path]} & Batch save multiple figures \\
\hline

\end{longtable}

\subsection{Expected Key Features}

\begin{itemize}
    \item \textbf{Lean Architecture}: No framework object storage, all passed as parameters
    \item \textbf{BulkData Integration}: Native support for BulkData solution format
    \item \textbf{Multi-Domain Networks}: Specialized visualization for connected domains
    \item \textbf{HDG-Specific Plots}: Trace and bulk solution visualization
    \item \textbf{Animation Support}: Time evolution and dynamic visualization
    \item \textbf{Analysis Tools}: Error analysis, convergence tracking, mass conservation
    \item \textbf{Flexible Styling}: Customizable appearance and output formats
    \item \textbf{Framework Integration}: Seamless integration with SolverSetup and BulkDataManager
\end{itemize}

\textbf{Note:} This documentation serves as a specification for implementing the lean matplotlib plotter module. The actual implementation would follow these interfaces and provide the described functionality.

% End of lean matplotlib plotter module API documentation (specification)


\section{Lean Matplotlib Plotter Module API Reference}
\label{sec:lean_matplotlib_plotter_api}

\textbf{Note:} This documentation describes the expected API for a lean matplotlib plotter module that follows the BioNetFlux lean architecture pattern. The module is not yet implemented but this serves as a specification.

\subsection{Module Overview}

The lean matplotlib plotter module would provide:
\begin{itemize}
    \item Lean plotting interface using external framework objects
    \item Specialized visualization for HDG trace and bulk solutions
    \item Multi-domain network visualization capabilities
    \item Time evolution animation and comparison plots
    \item Integration with BulkData and BulkDataManager
    \item Memory-efficient plotting without storing framework objects
\end{itemize}

\subsection{Expected Module Structure}

\begin{lstlisting}[language=Python, caption=Expected Module Dependencies]
# Expected imports for lean matplotlib plotter
import numpy as np
import matplotlib.pyplot as plt
from matplotlib.animation import FuncAnimation
from typing import List, Optional, Dict, Any, Tuple
from pathlib import Path

# BioNetFlux imports
from ooc1d.core.bulk_data import BulkData
from ooc1d.core.lean_bulk_data_manager import BulkDataManager
from ooc1d.core.discretization import Discretization
from ooc1d.core.problem import Problem
\end{lstlisting}

\subsection{Expected LeanMatplotlibPlotter Class}
\label{subsec:lean_matplotlib_plotter_class}

\subsubsection{Expected Constructor}

\paragraph{\_\_init\_\_()}\leavevmode
\begin{lstlisting}[language=Python, caption=Expected Constructor Signature]
def __init__(self, 
             figsize: Tuple[float, float] = (12, 8),
             style: str = 'seaborn-v0_8',
             dpi: int = 100)
\end{lstlisting}

\textbf{Parameters:}
\begin{itemize}
    \item \texttt{figsize}: Default figure size for plots (default: (12, 8))
    \item \texttt{style}: Matplotlib style to use (default: 'seaborn-v0\_8')
    \item \texttt{dpi}: Resolution for saved figures (default: 100)
\end{itemize}

\textbf{Expected Attributes:}
\begin{longtable}{|p{3cm}|p{4cm}|p{7cm}|}
\hline
\textbf{Attribute} & \textbf{Type} & \textbf{Description} \\
\hline
\endhead

\texttt{figsize} & \texttt{Tuple[float, float]} & Default figure dimensions \\
\hline

\texttt{style} & \texttt{str} & Matplotlib style configuration \\
\hline

\texttt{dpi} & \texttt{int} & Default resolution for saved figures \\
\hline

\texttt{color\_palette} & \texttt{List[str]} & Color scheme for multi-domain plots \\
\hline

\texttt{plot\_config} & \texttt{Dict} & Default plotting configuration options \\
\hline

\end{longtable}

\subsubsection{Expected Core Plotting Methods}

\paragraph{plot\_bulk\_solution()}\leavevmode
\begin{lstlisting}[language=Python, caption=Expected Bulk Solution Plotting Method]
def plot_bulk_solution(self,
                       bulk_data: BulkData,
                       discretization: Discretization,
                       problem: Problem,
                       domain_idx: int = 0,
                       equations: Optional[List[int]] = None,
                       title: Optional[str] = None,
                       savepath: Optional[Path] = None) -> plt.Figure
\end{lstlisting}

\textbf{Parameters:}
\begin{itemize}
    \item \texttt{bulk\_data}: BulkData instance containing solution
    \item \texttt{discretization}: Discretization for spatial coordinates
    \item \texttt{problem}: Problem instance for metadata and labeling
    \item \texttt{domain\_idx}: Domain index for title generation (default: 0)
    \item \texttt{equations}: List of equation indices to plot (default: all)
    \item \texttt{title}: Custom plot title (default: auto-generated)
    \item \texttt{savepath}: Path to save figure (default: display only)
\end{itemize}

\textbf{Returns:} \texttt{plt.Figure} - Matplotlib figure object

\textbf{Expected Usage:}
\begin{lstlisting}[language=Python, caption=Bulk Solution Plotting Usage]
from ooc1d.utils.lean_matplotlib_plotter import LeanMatplotlibPlotter

plotter = LeanMatplotlibPlotter()

# Plot all equations
fig = plotter.plot_bulk_solution(
    bulk_data=bulk_solution,
    discretization=discretization,
    problem=problem,
    domain_idx=0,
    title="Bulk Solution at t=1.0"
)

# Plot specific equations only
fig = plotter.plot_bulk_solution(
    bulk_data=bulk_solution,
    discretization=discretization,
    problem=problem,
    equations=[0, 2],  # Only plot equations 0 and 2
    savepath=Path("results/bulk_solution.png")
)
\end{lstlisting}

\paragraph{plot\_trace\_solution()}\leavevmode
\begin{lstlisting}[language=Python, caption=Expected Trace Solution Plotting Method]
def plot_trace_solution(self,
                        trace_solution: np.ndarray,
                        discretization: Discretization,
                        problem: Problem,
                        domain_idx: int = 0,
                        equations: Optional[List[int]] = None,
                        title: Optional[str] = None,
                        savepath: Optional[Path] = None) -> plt.Figure
\end{lstlisting}

\textbf{Parameters:}
\begin{itemize}
    \item \texttt{trace\_solution}: Array of trace values (neq*(N+1),)
    \item \texttt{discretization}: Discretization for node coordinates
    \item \texttt{problem}: Problem instance for metadata
    \item \texttt{domain\_idx}: Domain index for labeling (default: 0)
    \item \texttt{equations}: Equation indices to plot (default: all)
    \item \texttt{title}: Custom title (default: auto-generated)
    \item \texttt{savepath}: Save path (default: display only)
\end{itemize}

\textbf{Returns:} \texttt{plt.Figure} - Matplotlib figure object

\paragraph{plot\_multi\_domain\_network()}\leavevmode
\begin{lstlisting}[language=Python, caption=Expected Multi-Domain Network Plotting Method]
def plot_multi_domain_network(self,
                              bulk_data_list: List[BulkData],
                              problems: List[Problem],
                              discretizations: List[Discretization],
                              constraint_manager = None,
                              equation_idx: int = 0,
                              time: float = 0.0,
                              title: Optional[str] = None,
                              savepath: Optional[Path] = None) -> plt.Figure
\end{lstlisting}

\textbf{Parameters:}
\begin{itemize}
    \item \texttt{bulk\_data\_list}: List of BulkData for each domain
    \item \texttt{problems}: List of Problem instances
    \item \texttt{discretizations}: List of Discretization instances
    \item \texttt{constraint\_manager}: ConstraintManager for junction visualization
    \item \texttt{equation\_idx}: Which equation to visualize (default: 0)
    \item \texttt{time}: Current time for title (default: 0.0)
    \item \texttt{title}: Custom title (default: auto-generated)
    \item \texttt{savepath}: Save path (default: display only)
\end{itemize}

\textbf{Returns:} \texttt{plt.Figure} - Matplotlib figure with network visualization

\textbf{Expected Network Visualization Features:}
\begin{itemize}
    \item Different colors for each domain
    \item Junction points marked clearly
    \item Constraint type indicators (continuity, flux jump, etc.)
    \item Domain boundaries and connections
    \item Equation-specific color mapping
\end{itemize}

\subsubsection{Expected Comparison and Analysis Methods}

\paragraph{plot\_solution\_comparison()}\leavevmode
\begin{lstlisting}[language=Python, caption=Expected Solution Comparison Method]
def plot_solution_comparison(self,
                            solutions: Dict[str, BulkData],
                            discretization: Discretization,
                            problem: Problem,
                            domain_idx: int = 0,
                            equation_idx: int = 0,
                            title: Optional[str] = None,
                            savepath: Optional[Path] = None) -> plt.Figure
\end{lstlisting}

\textbf{Parameters:}
\begin{itemize}
    \item \texttt{solutions}: Dictionary mapping labels to BulkData solutions
    \item \texttt{discretization}: Discretization for coordinates
    \item \texttt{problem}: Problem instance
    \item \texttt{domain\_idx}: Domain index (default: 0)
    \item \texttt{equation\_idx}: Equation index to compare (default: 0)
    \item \texttt{title}: Custom title (default: auto-generated)
    \item \texttt{savepath}: Save path (default: display only)
\end{itemize}

\textbf{Returns:} \texttt{plt.Figure} - Comparison plot figure

\textbf{Expected Usage:}
\begin{lstlisting}[language=Python, caption=Solution Comparison Usage]
solutions = {
    "Initial": initial_bulk_data,
    "t=0.5": intermediate_bulk_data,
    "Final": final_bulk_data,
    "Analytical": analytical_bulk_data
}

fig = plotter.plot_solution_comparison(
    solutions=solutions,
    discretization=discretization,
    problem=problem,
    equation_idx=0,
    title="Solution Evolution Comparison"
)
\end{lstlisting}

\paragraph{plot\_mass\_conservation()}\leavevmode
\begin{lstlisting}[language=Python, caption=Expected Mass Conservation Plotting Method]
def plot_mass_conservation(self,
                          mass_history: List[float],
                          time_points: np.ndarray,
                          relative: bool = True,
                          title: Optional[str] = None,
                          savepath: Optional[Path] = None) -> plt.Figure
\end{lstlisting}

\textbf{Parameters:}
\begin{itemize}
    \item \texttt{mass\_history}: List of total mass values over time
    \item \texttt{time\_points}: Corresponding time points
    \item \texttt{relative}: Plot relative change if True, absolute values if False (default: True)
    \item \texttt{title}: Custom title (default: auto-generated)
    \item \texttt{savepath}: Save path (default: display only)
\end{itemize}

\textbf{Returns:} \texttt{plt.Figure} - Mass conservation plot

\subsubsection{Expected Animation Methods}

\paragraph{create\_time\_evolution\_animation()}\leavevmode
\begin{lstlisting}[language=Python, caption=Expected Animation Creation Method]
def create_time_evolution_animation(self,
                                   bulk_data_history: List[List[BulkData]],
                                   problems: List[Problem],
                                   discretizations: List[Discretization],
                                   time_points: np.ndarray,
                                   equation_idx: int = 0,
                                   fps: int = 10,
                                   title_template: str = "t = {:.3f}",
                                   savepath: Optional[Path] = None) -> FuncAnimation
\end{lstlisting}

\textbf{Parameters:}
\begin{itemize}
    \item \texttt{bulk\_data\_history}: List of BulkData lists for each time step
    \item \texttt{problems}: List of Problem instances
    \item \texttt{discretizations}: List of Discretization instances
    \item \texttt{time\_points}: Time values for each frame
    \item \texttt{equation\_idx}: Which equation to animate (default: 0)
    \item \texttt{fps}: Frames per second (default: 10)
    \item \texttt{title\_template}: Title format string (default: "t = \{:.3f\}")
    \item \texttt{savepath}: Path to save animation (default: display only)
\end{itemize}

\textbf{Returns:} \texttt{FuncAnimation} - Matplotlib animation object

\subsubsection{Expected Convergence and Error Analysis Methods}

\paragraph{plot\_newton\_convergence()}\leavevmode
\begin{lstlisting}[language=Python, caption=Expected Newton Convergence Plotting Method]
def plot_newton_convergence(self,
                           residual_history: List[float],
                           tolerance: float = 1e-10,
                           log_scale: bool = True,
                           title: Optional[str] = None,
                           savepath: Optional[Path] = None) -> plt.Figure
\end{lstlisting}

\textbf{Parameters:}
\begin{itemize}
    \item \texttt{residual\_history}: List of residual norms from Newton iterations
    \item \texttt{tolerance}: Convergence tolerance to mark on plot (default: 1e-10)
    \item \texttt{log\_scale}: Use logarithmic y-axis (default: True)
    \item \texttt{title}: Custom title (default: auto-generated)
    \item \texttt{savepath}: Save path (default: display only)
\end{itemize}

\textbf{Returns:} \texttt{plt.Figure} - Convergence history plot

\paragraph{plot\_error\_analysis()}\leavevmode
\begin{lstlisting}[language=Python, caption=Expected Error Analysis Method]
def plot_error_analysis(self,
                       numerical_solution: BulkData,
                       analytical_solution: BulkData,
                       discretization: Discretization,
                       problem: Problem,
                       equation_idx: int = 0,
                       error_type: str = 'absolute',
                       title: Optional[str] = None,
                       savepath: Optional[Path] = None) -> plt.Figure
\end{lstlisting}

\textbf{Parameters:}
\begin{itemize}
    \item \texttt{numerical\_solution}: Computed BulkData solution
    \item \texttt{analytical\_solution}: Reference BulkData solution
    \item \texttt{discretization}: Discretization for coordinates
    \item \texttt{problem}: Problem instance
    \item \texttt{equation\_idx}: Equation index to analyze (default: 0)
    \item \texttt{error\_type}: 'absolute', 'relative', or 'both' (default: 'absolute')
    \item \texttt{title}: Custom title (default: auto-generated)
    \item \texttt{savepath}: Save path (default: display only)
\end{itemize}

\textbf{Returns:} \texttt{plt.Figure} - Error analysis plot with subplots

\subsubsection{Expected Utility Methods}

\paragraph{set\_default\_style()}\leavevmode
\begin{lstlisting}[language=Python, caption=Expected Style Configuration Method]
def set_default_style(self,
                     style: str = 'seaborn-v0_8',
                     color_palette: Optional[List[str]] = None,
                     font_size: int = 12) -> None
\end{lstlisting}

\textbf{Parameters:}
\begin{itemize}
    \item \texttt{style}: Matplotlib style name (default: 'seaborn-v0\_8')
    \item \texttt{color\_palette}: Custom color scheme (default: automatic)
    \item \texttt{font\_size}: Base font size (default: 12)
\end{itemize}

\textbf{Side Effects:} Updates default plotting configuration

\paragraph{save\_all\_figures()}\leavevmode
\begin{lstlisting}[language=Python, caption=Expected Batch Save Method]
def save_all_figures(self,
                    figures: List[plt.Figure],
                    base_path: Path,
                    prefix: str = "figure",
                    format: str = "png") -> List[Path]
\end{lstlisting}

\textbf{Parameters:}
\begin{itemize}
    \item \texttt{figures}: List of matplotlib Figure objects
    \item \texttt{base\_path}: Directory to save figures
    \item \texttt{prefix}: Filename prefix (default: "figure")
    \item \texttt{format}: File format (default: "png")
\end{itemize}

\textbf{Returns:} \texttt{List[Path]} - List of saved file paths

\subsection{Expected Complete Usage Examples}
\label{subsec:expected_complete_usage_examples}

\subsubsection{Single Domain Visualization}

\begin{lstlisting}[language=Python, caption=Expected Single Domain Usage]
from ooc1d.utils.lean_matplotlib_plotter import LeanMatplotlibPlotter
from ooc1d.core.bulk_data import BulkData
from pathlib import Path

# Create plotter
plotter = LeanMatplotlibPlotter(figsize=(14, 10))

# Load or compute solution
bulk_solution = BulkData(problem, discretization, dual=False)
# ... populate bulk_solution with data ...

# Plot bulk solution
fig1 = plotter.plot_bulk_solution(
    bulk_data=bulk_solution,
    discretization=discretization,
    problem=problem,
    title="Keller-Segel Solution at t=1.0",
    savepath=Path("results/keller_segel_bulk.png")
)

# Plot trace solution
trace_values = bulk_solution.get_trace_values()
fig2 = plotter.plot_trace_solution(
    trace_solution=trace_values,
    discretization=discretization,
    problem=problem,
    equations=[0, 1],  # Cell density and chemical concentration
    title="HDG Trace Values",
    savepath=Path("results/keller_segel_trace.png")
)

print("Single domain plots created successfully")
\end{lstlisting}

\subsubsection{Multi-Domain Network Visualization}

\begin{lstlisting}[language=Python, caption=Expected Multi-Domain Usage]
# Multi-domain network setup
problems = [problem1, problem2, problem3]  # Three domains
discretizations = [disc1, disc2, disc3]
bulk_data_list = [bulk1, bulk2, bulk3]

# Visualize network solution
fig = plotter.plot_multi_domain_network(
    bulk_data_list=bulk_data_list,
    problems=problems,
    discretizations=discretizations,
    constraint_manager=constraint_manager,
    equation_idx=0,  # Cell density
    time=2.5,
    title="Vascular Network Cell Migration",
    savepath=Path("results/network_solution.png")
)

# Create animation of time evolution
# Assume we have solutions at multiple time points
bulk_history = []  # List[List[BulkData]] for each time step
time_points = np.linspace(0, 5, 50)

animation = plotter.create_time_evolution_animation(
    bulk_data_history=bulk_history,
    problems=problems,
    discretizations=discretizations,
    time_points=time_points,
    equation_idx=0,
    fps=5,
    savepath=Path("results/network_evolution.mp4")
)

print("Multi-domain visualization completed")
\end{lstlisting}

\subsubsection{Analysis and Comparison Workflows}

\begin{lstlisting}[language=Python, caption=Expected Analysis Workflow]
# Comparison of different solutions
solutions_dict = {
    "Initial Condition": initial_bulk,
    "Numerical t=1": numerical_bulk_t1,
    "Numerical t=2": numerical_bulk_t2,
    "Analytical": analytical_bulk
}

fig_comp = plotter.plot_solution_comparison(
    solutions=solutions_dict,
    discretization=discretization,
    problem=problem,
    equation_idx=0,
    title="Solution Evolution Comparison",
    savepath=Path("results/solution_comparison.png")
)

# Error analysis
fig_error = plotter.plot_error_analysis(
    numerical_solution=numerical_bulk,
    analytical_solution=analytical_bulk,
    discretization=discretization,
    problem=problem,
    equation_idx=0,
    error_type='both',
    title="Error Analysis vs Analytical Solution",
    savepath=Path("results/error_analysis.png")
)

# Newton convergence analysis
residual_norms = [1e-1, 1e-3, 1e-6, 1e-9, 1e-12]  # Example data
fig_newton = plotter.plot_newton_convergence(
    residual_history=residual_norms,
    tolerance=1e-10,
    title="Newton Solver Convergence",
    savepath=Path("results/newton_convergence.png")
)

# Mass conservation tracking
mass_values = [100.0, 99.99, 99.98, 99.99, 100.01]  # Example
time_vals = np.linspace(0, 2, len(mass_values))

fig_mass = plotter.plot_mass_conservation(
    mass_history=mass_values,
    time_points=time_vals,
    relative=True,
    title="Mass Conservation Check",
    savepath=Path("results/mass_conservation.png")
)

print("Analysis workflow completed")
\end{lstlisting}

\subsection{Expected Integration with BioNetFlux Framework}

\begin{lstlisting}[language=Python, caption=Expected Framework Integration]
from setup_solver import quick_setup

# Setup complete solver framework
setup = quick_setup("ooc1d.problems.keller_segel_network")

# Create plotter
plotter = LeanMatplotlibPlotter()

# Initial conditions
trace_solutions, multipliers = setup.create_initial_conditions()
global_solution = setup.create_global_solution_vector(trace_solutions, multipliers)

# Initialize bulk data
bulk_data_list = setup.bulk_data_manager.initialize_all_bulk_data(
    setup.problems,
    setup.global_discretization.spatial_discretizations,
    time=0.0
)

# Plot initial network state
fig_initial = plotter.plot_multi_domain_network(
    bulk_data_list=bulk_data_list,
    problems=setup.problems,
    discretizations=setup.global_discretization.spatial_discretizations,
    constraint_manager=setup.constraint_manager,
    equation_idx=0,
    time=0.0,
    title="Initial Network State"
)

# Time evolution with plotting
time_history = []
bulk_history = []
mass_history = []

for time_step in range(setup.global_discretization.n_time_steps):
    current_time = time_step * setup.global_discretization.dt
    
    # Solve time step (simplified)
    # ... Newton iteration and solution update ...
    
    # Extract current bulk solutions
    current_bulk_list = setup.bulk_data_manager.initialize_all_bulk_data(
        setup.problems,
        setup.global_discretization.spatial_discretizations,
        time=current_time
    )
    
    # Track mass conservation
    total_mass = setup.bulk_data_manager.compute_total_mass(current_bulk_list)
    
    # Store for history
    time_history.append(current_time)
    bulk_history.append(current_bulk_list.copy())
    mass_history.append(total_mass)
    
    # Plot at specific intervals
    if time_step % 10 == 0:
        fig_step = plotter.plot_multi_domain_network(
            bulk_data_list=current_bulk_list,
            problems=setup.problems,
            discretizations=setup.global_discretization.spatial_discretizations,
            constraint_manager=setup.constraint_manager,
            equation_idx=0,
            time=current_time,
            savepath=Path(f"results/network_t_{time_step:04d}.png")
        )

# Final analysis plots
fig_mass_final = plotter.plot_mass_conservation(
    mass_history=mass_history,
    time_points=np.array(time_history),
    title="Mass Conservation Over Time"
)

# Create final animation
animation = plotter.create_time_evolution_animation(
    bulk_data_history=bulk_history,
    problems=setup.problems,
    discretizations=setup.global_discretization.spatial_discretizations,
    time_points=np.array(time_history),
    equation_idx=0,
    savepath=Path("results/complete_evolution.mp4")
)

print("Complete framework integration visualization finished")
\end{lstlisting}

\subsection{Expected Method Summary Table}
\label{subsec:expected_method_summary}

\begin{longtable}{|p{5cm}|p{3cm}|p{6.5cm}|}
\hline
\textbf{Expected Method} & \textbf{Returns} & \textbf{Purpose} \\
\hline
\endhead

\texttt{plot\_bulk\_solution} & \texttt{Figure} & Visualize bulk solution for single domain \\
\hline

\texttt{plot\_trace\_solution} & \texttt{Figure} & Plot HDG trace values at element boundaries \\
\hline

\texttt{plot\_multi\_domain\_network} & \texttt{Figure} & Visualize solution across network topology \\
\hline

\texttt{plot\_solution\_comparison} & \texttt{Figure} & Compare multiple solutions side-by-side \\
\hline

\texttt{plot\_mass\_conservation} & \texttt{Figure} & Track mass conservation over time \\
\hline

\texttt{create\_time\_animation} & \texttt{FuncAnimation} & Animate solution evolution \\
\hline

\texttt{plot\_newton\_convergence} & \texttt{Figure} & Show Newton solver convergence history \\
\hline

\texttt{plot\_error\_analysis} & \texttt{Figure} & Compare numerical vs analytical solutions \\
\hline

\texttt{set\_default\_style} & \texttt{None} & Configure plotting appearance \\
\hline

\texttt{save\_all\_figures} & \texttt{List[Path]} & Batch save multiple figures \\
\hline

\end{longtable}

\subsection{Expected Key Features}

\begin{itemize}
    \item \textbf{Lean Architecture}: No framework object storage, all passed as parameters
    \item \textbf{BulkData Integration}: Native support for BulkData solution format
    \item \textbf{Multi-Domain Networks}: Specialized visualization for connected domains
    \item \textbf{HDG-Specific Plots}: Trace and bulk solution visualization
    \item \textbf{Animation Support}: Time evolution and dynamic visualization
    \item \textbf{Analysis Tools}: Error analysis, convergence tracking, mass conservation
    \item \textbf{Flexible Styling}: Customizable appearance and output formats
    \item \textbf{Framework Integration}: Seamless integration with SolverSetup and BulkDataManager
\end{itemize}

\textbf{Note:} This documentation serves as a specification for implementing the lean matplotlib plotter module. The actual implementation would follow these interfaces and provide the described functionality.

% End of lean matplotlib plotter module API documentation (specification)


\section{Lean Matplotlib Plotter Module API Reference}
\label{sec:lean_matplotlib_plotter_api}

\textbf{Note:} This documentation describes the expected API for a lean matplotlib plotter module that follows the BioNetFlux lean architecture pattern. The module is not yet implemented but this serves as a specification.

\subsection{Module Overview}

The lean matplotlib plotter module would provide:
\begin{itemize}
    \item Lean plotting interface using external framework objects
    \item Specialized visualization for HDG trace and bulk solutions
    \item Multi-domain network visualization capabilities
    \item Time evolution animation and comparison plots
    \item Integration with BulkData and BulkDataManager
    \item Memory-efficient plotting without storing framework objects
\end{itemize}

\subsection{Expected Module Structure}

\begin{lstlisting}[language=Python, caption=Expected Module Dependencies]
# Expected imports for lean matplotlib plotter
import numpy as np
import matplotlib.pyplot as plt
from matplotlib.animation import FuncAnimation
from typing import List, Optional, Dict, Any, Tuple
from pathlib import Path

# BioNetFlux imports
from ooc1d.core.bulk_data import BulkData
from ooc1d.core.lean_bulk_data_manager import BulkDataManager
from ooc1d.core.discretization import Discretization
from ooc1d.core.problem import Problem
\end{lstlisting}

\subsection{Expected LeanMatplotlibPlotter Class}
\label{subsec:lean_matplotlib_plotter_class}

\subsubsection{Expected Constructor}

\paragraph{\_\_init\_\_()}\leavevmode
\begin{lstlisting}[language=Python, caption=Expected Constructor Signature]
def __init__(self, 
             figsize: Tuple[float, float] = (12, 8),
             style: str = 'seaborn-v0_8',
             dpi: int = 100)
\end{lstlisting}

\textbf{Parameters:}
\begin{itemize}
    \item \texttt{figsize}: Default figure size for plots (default: (12, 8))
    \item \texttt{style}: Matplotlib style to use (default: 'seaborn-v0\_8')
    \item \texttt{dpi}: Resolution for saved figures (default: 100)
\end{itemize}

\textbf{Expected Attributes:}
\begin{longtable}{|p{3cm}|p{4cm}|p{7cm}|}
\hline
\textbf{Attribute} & \textbf{Type} & \textbf{Description} \\
\hline
\endhead

\texttt{figsize} & \texttt{Tuple[float, float]} & Default figure dimensions \\
\hline

\texttt{style} & \texttt{str} & Matplotlib style configuration \\
\hline

\texttt{dpi} & \texttt{int} & Default resolution for saved figures \\
\hline

\texttt{color\_palette} & \texttt{List[str]} & Color scheme for multi-domain plots \\
\hline

\texttt{plot\_config} & \texttt{Dict} & Default plotting configuration options \\
\hline

\end{longtable}

\subsubsection{Expected Core Plotting Methods}

\paragraph{plot\_bulk\_solution()}\leavevmode
\begin{lstlisting}[language=Python, caption=Expected Bulk Solution Plotting Method]
def plot_bulk_solution(self,
                       bulk_data: BulkData,
                       discretization: Discretization,
                       problem: Problem,
                       domain_idx: int = 0,
                       equations: Optional[List[int]] = None,
                       title: Optional[str] = None,
                       savepath: Optional[Path] = None) -> plt.Figure
\end{lstlisting}

\textbf{Parameters:}
\begin{itemize}
    \item \texttt{bulk\_data}: BulkData instance containing solution
    \item \texttt{discretization}: Discretization for spatial coordinates
    \item \texttt{problem}: Problem instance for metadata and labeling
    \item \texttt{domain\_idx}: Domain index for title generation (default: 0)
    \item \texttt{equations}: List of equation indices to plot (default: all)
    \item \texttt{title}: Custom plot title (default: auto-generated)
    \item \texttt{savepath}: Path to save figure (default: display only)
\end{itemize}

\textbf{Returns:} \texttt{plt.Figure} - Matplotlib figure object

\textbf{Expected Usage:}
\begin{lstlisting}[language=Python, caption=Bulk Solution Plotting Usage]
from ooc1d.utils.lean_matplotlib_plotter import LeanMatplotlibPlotter

plotter = LeanMatplotlibPlotter()

# Plot all equations
fig = plotter.plot_bulk_solution(
    bulk_data=bulk_solution,
    discretization=discretization,
    problem=problem,
    domain_idx=0,
    title="Bulk Solution at t=1.0"
)

# Plot specific equations only
fig = plotter.plot_bulk_solution(
    bulk_data=bulk_solution,
    discretization=discretization,
    problem=problem,
    equations=[0, 2],  # Only plot equations 0 and 2
    savepath=Path("results/bulk_solution.png")
)
\end{lstlisting}

\paragraph{plot\_trace\_solution()}\leavevmode
\begin{lstlisting}[language=Python, caption=Expected Trace Solution Plotting Method]
def plot_trace_solution(self,
                        trace_solution: np.ndarray,
                        discretization: Discretization,
                        problem: Problem,
                        domain_idx: int = 0,
                        equations: Optional[List[int]] = None,
                        title: Optional[str] = None,
                        savepath: Optional[Path] = None) -> plt.Figure
\end{lstlisting}

\textbf{Parameters:}
\begin{itemize}
    \item \texttt{trace\_solution}: Array of trace values (neq*(N+1),)
    \item \texttt{discretization}: Discretization for node coordinates
    \item \texttt{problem}: Problem instance for metadata
    \item \texttt{domain\_idx}: Domain index for labeling (default: 0)
    \item \texttt{equations}: Equation indices to plot (default: all)
    \item \texttt{title}: Custom title (default: auto-generated)
    \item \texttt{savepath}: Save path (default: display only)
\end{itemize}

\textbf{Returns:} \texttt{plt.Figure} - Matplotlib figure object

\paragraph{plot\_multi\_domain\_network()}\leavevmode
\begin{lstlisting}[language=Python, caption=Expected Multi-Domain Network Plotting Method]
def plot_multi_domain_network(self,
                              bulk_data_list: List[BulkData],
                              problems: List[Problem],
                              discretizations: List[Discretization],
                              constraint_manager = None,
                              equation_idx: int = 0,
                              time: float = 0.0,
                              title: Optional[str] = None,
                              savepath: Optional[Path] = None) -> plt.Figure
\end{lstlisting}

\textbf{Parameters:}
\begin{itemize}
    \item \texttt{bulk\_data\_list}: List of BulkData for each domain
    \item \texttt{problems}: List of Problem instances
    \item \texttt{discretizations}: List of Discretization instances
    \item \texttt{constraint\_manager}: ConstraintManager for junction visualization
    \item \texttt{equation\_idx}: Which equation to visualize (default: 0)
    \item \texttt{time}: Current time for title (default: 0.0)
    \item \texttt{title}: Custom title (default: auto-generated)
    \item \texttt{savepath}: Save path (default: display only)
\end{itemize}

\textbf{Returns:} \texttt{plt.Figure} - Matplotlib figure with network visualization

\textbf{Expected Network Visualization Features:}
\begin{itemize}
    \item Different colors for each domain
    \item Junction points marked clearly
    \item Constraint type indicators (continuity, flux jump, etc.)
    \item Domain boundaries and connections
    \item Equation-specific color mapping
\end{itemize}

\subsubsection{Expected Comparison and Analysis Methods}

\paragraph{plot\_solution\_comparison()}\leavevmode
\begin{lstlisting}[language=Python, caption=Expected Solution Comparison Method]
def plot_solution_comparison(self,
                            solutions: Dict[str, BulkData],
                            discretization: Discretization,
                            problem: Problem,
                            domain_idx: int = 0,
                            equation_idx: int = 0,
                            title: Optional[str] = None,
                            savepath: Optional[Path] = None) -> plt.Figure
\end{lstlisting}

\textbf{Parameters:}
\begin{itemize}
    \item \texttt{solutions}: Dictionary mapping labels to BulkData solutions
    \item \texttt{discretization}: Discretization for coordinates
    \item \texttt{problem}: Problem instance
    \item \texttt{domain\_idx}: Domain index (default: 0)
    \item \texttt{equation\_idx}: Equation index to compare (default: 0)
    \item \texttt{title}: Custom title (default: auto-generated)
    \item \texttt{savepath}: Save path (default: display only)
\end{itemize}

\textbf{Returns:} \texttt{plt.Figure} - Comparison plot figure

\textbf{Expected Usage:}
\begin{lstlisting}[language=Python, caption=Solution Comparison Usage]
solutions = {
    "Initial": initial_bulk_data,
    "t=0.5": intermediate_bulk_data,
    "Final": final_bulk_data,
    "Analytical": analytical_bulk_data
}

fig = plotter.plot_solution_comparison(
    solutions=solutions,
    discretization=discretization,
    problem=problem,
    equation_idx=0,
    title="Solution Evolution Comparison"
)
\end{lstlisting}

\paragraph{plot\_mass\_conservation()}\leavevmode
\begin{lstlisting}[language=Python, caption=Expected Mass Conservation Plotting Method]
def plot_mass_conservation(self,
                          mass_history: List[float],
                          time_points: np.ndarray,
                          relative: bool = True,
                          title: Optional[str] = None,
                          savepath: Optional[Path] = None) -> plt.Figure
\end{lstlisting}

\textbf{Parameters:}
\begin{itemize}
    \item \texttt{mass\_history}: List of total mass values over time
    \item \texttt{time\_points}: Corresponding time points
    \item \texttt{relative}: Plot relative change if True, absolute values if False (default: True)
    \item \texttt{title}: Custom title (default: auto-generated)
    \item \texttt{savepath}: Save path (default: display only)
\end{itemize}

\textbf{Returns:} \texttt{plt.Figure} - Mass conservation plot

\subsubsection{Expected Animation Methods}

\paragraph{create\_time\_evolution\_animation()}\leavevmode
\begin{lstlisting}[language=Python, caption=Expected Animation Creation Method]
def create_time_evolution_animation(self,
                                   bulk_data_history: List[List[BulkData]],
                                   problems: List[Problem],
                                   discretizations: List[Discretization],
                                   time_points: np.ndarray,
                                   equation_idx: int = 0,
                                   fps: int = 10,
                                   title_template: str = "t = {:.3f}",
                                   savepath: Optional[Path] = None) -> FuncAnimation
\end{lstlisting}

\textbf{Parameters:}
\begin{itemize}
    \item \texttt{bulk\_data\_history}: List of BulkData lists for each time step
    \item \texttt{problems}: List of Problem instances
    \item \texttt{discretizations}: List of Discretization instances
    \item \texttt{time\_points}: Time values for each frame
    \item \texttt{equation\_idx}: Which equation to animate (default: 0)
    \item \texttt{fps}: Frames per second (default: 10)
    \item \texttt{title\_template}: Title format string (default: "t = \{:.3f\}")
    \item \texttt{savepath}: Path to save animation (default: display only)
\end{itemize}

\textbf{Returns:} \texttt{FuncAnimation} - Matplotlib animation object

\subsubsection{Expected Convergence and Error Analysis Methods}

\paragraph{plot\_newton\_convergence()}\leavevmode
\begin{lstlisting}[language=Python, caption=Expected Newton Convergence Plotting Method]
def plot_newton_convergence(self,
                           residual_history: List[float],
                           tolerance: float = 1e-10,
                           log_scale: bool = True,
                           title: Optional[str] = None,
                           savepath: Optional[Path] = None) -> plt.Figure
\end{lstlisting}

\textbf{Parameters:}
\begin{itemize}
    \item \texttt{residual\_history}: List of residual norms from Newton iterations
    \item \texttt{tolerance}: Convergence tolerance to mark on plot (default: 1e-10)
    \item \texttt{log\_scale}: Use logarithmic y-axis (default: True)
    \item \texttt{title}: Custom title (default: auto-generated)
    \item \texttt{savepath}: Save path (default: display only)
\end{itemize}

\textbf{Returns:} \texttt{plt.Figure} - Convergence history plot

\paragraph{plot\_error\_analysis()}\leavevmode
\begin{lstlisting}[language=Python, caption=Expected Error Analysis Method]
def plot_error_analysis(self,
                       numerical_solution: BulkData,
                       analytical_solution: BulkData,
                       discretization: Discretization,
                       problem: Problem,
                       equation_idx: int = 0,
                       error_type: str = 'absolute',
                       title: Optional[str] = None,
                       savepath: Optional[Path] = None) -> plt.Figure
\end{lstlisting}

\textbf{Parameters:}
\begin{itemize}
    \item \texttt{numerical\_solution}: Computed BulkData solution
    \item \texttt{analytical\_solution}: Reference BulkData solution
    \item \texttt{discretization}: Discretization for coordinates
    \item \texttt{problem}: Problem instance
    \item \texttt{equation\_idx}: Equation index to analyze (default: 0)
    \item \texttt{error\_type}: 'absolute', 'relative', or 'both' (default: 'absolute')
    \item \texttt{title}: Custom title (default: auto-generated)
    \item \texttt{savepath}: Save path (default: display only)
\end{itemize}

\textbf{Returns:} \texttt{plt.Figure} - Error analysis plot with subplots

\subsubsection{Expected Utility Methods}

\paragraph{set\_default\_style()}\leavevmode
\begin{lstlisting}[language=Python, caption=Expected Style Configuration Method]
def set_default_style(self,
                     style: str = 'seaborn-v0_8',
                     color_palette: Optional[List[str]] = None,
                     font_size: int = 12) -> None
\end{lstlisting}

\textbf{Parameters:}
\begin{itemize}
    \item \texttt{style}: Matplotlib style name (default: 'seaborn-v0\_8')
    \item \texttt{color\_palette}: Custom color scheme (default: automatic)
    \item \texttt{font\_size}: Base font size (default: 12)
\end{itemize}

\textbf{Side Effects:} Updates default plotting configuration

\paragraph{save\_all\_figures()}\leavevmode
\begin{lstlisting}[language=Python, caption=Expected Batch Save Method]
def save_all_figures(self,
                    figures: List[plt.Figure],
                    base_path: Path,
                    prefix: str = "figure",
                    format: str = "png") -> List[Path]
\end{lstlisting}

\textbf{Parameters:}
\begin{itemize}
    \item \texttt{figures}: List of matplotlib Figure objects
    \item \texttt{base\_path}: Directory to save figures
    \item \texttt{prefix}: Filename prefix (default: "figure")
    \item \texttt{format}: File format (default: "png")
\end{itemize}

\textbf{Returns:} \texttt{List[Path]} - List of saved file paths

\subsection{Expected Complete Usage Examples}
\label{subsec:expected_complete_usage_examples}

\subsubsection{Single Domain Visualization}

\begin{lstlisting}[language=Python, caption=Expected Single Domain Usage]
from ooc1d.utils.lean_matplotlib_plotter import LeanMatplotlibPlotter
from ooc1d.core.bulk_data import BulkData
from pathlib import Path

# Create plotter
plotter = LeanMatplotlibPlotter(figsize=(14, 10))

# Load or compute solution
bulk_solution = BulkData(problem, discretization, dual=False)
# ... populate bulk_solution with data ...

# Plot bulk solution
fig1 = plotter.plot_bulk_solution(
    bulk_data=bulk_solution,
    discretization=discretization,
    problem=problem,
    title="Keller-Segel Solution at t=1.0",
    savepath=Path("results/keller_segel_bulk.png")
)

# Plot trace solution
trace_values = bulk_solution.get_trace_values()
fig2 = plotter.plot_trace_solution(
    trace_solution=trace_values,
    discretization=discretization,
    problem=problem,
    equations=[0, 1],  # Cell density and chemical concentration
    title="HDG Trace Values",
    savepath=Path("results/keller_segel_trace.png")
)

print("Single domain plots created successfully")
\end{lstlisting}

\subsubsection{Multi-Domain Network Visualization}

\begin{lstlisting}[language=Python, caption=Expected Multi-Domain Usage]
# Multi-domain network setup
problems = [problem1, problem2, problem3]  # Three domains
discretizations = [disc1, disc2, disc3]
bulk_data_list = [bulk1, bulk2, bulk3]

# Visualize network solution
fig = plotter.plot_multi_domain_network(
    bulk_data_list=bulk_data_list,
    problems=problems,
    discretizations=discretizations,
    constraint_manager=constraint_manager,
    equation_idx=0,  # Cell density
    time=2.5,
    title="Vascular Network Cell Migration",
    savepath=Path("results/network_solution.png")
)

# Create animation of time evolution
# Assume we have solutions at multiple time points
bulk_history = []  # List[List[BulkData]] for each time step
time_points = np.linspace(0, 5, 50)

animation = plotter.create_time_evolution_animation(
    bulk_data_history=bulk_history,
    problems=problems,
    discretizations=discretizations,
    time_points=time_points,
    equation_idx=0,
    fps=5,
    savepath=Path("results/network_evolution.mp4")
)

print("Multi-domain visualization completed")
\end{lstlisting}

\subsubsection{Analysis and Comparison Workflows}

\begin{lstlisting}[language=Python, caption=Expected Analysis Workflow]
# Comparison of different solutions
solutions_dict = {
    "Initial Condition": initial_bulk,
    "Numerical t=1": numerical_bulk_t1,
    "Numerical t=2": numerical_bulk_t2,
    "Analytical": analytical_bulk
}

fig_comp = plotter.plot_solution_comparison(
    solutions=solutions_dict,
    discretization=discretization,
    problem=problem,
    equation_idx=0,
    title="Solution Evolution Comparison",
    savepath=Path("results/solution_comparison.png")
)

# Error analysis
fig_error = plotter.plot_error_analysis(
    numerical_solution=numerical_bulk,
    analytical_solution=analytical_bulk,
    discretization=discretization,
    problem=problem,
    equation_idx=0,
    error_type='both',
    title="Error Analysis vs Analytical Solution",
    savepath=Path("results/error_analysis.png")
)

# Newton convergence analysis
residual_norms = [1e-1, 1e-3, 1e-6, 1e-9, 1e-12]  # Example data
fig_newton = plotter.plot_newton_convergence(
    residual_history=residual_norms,
    tolerance=1e-10,
    title="Newton Solver Convergence",
    savepath=Path("results/newton_convergence.png")
)

# Mass conservation tracking
mass_values = [100.0, 99.99, 99.98, 99.99, 100.01]  # Example
time_vals = np.linspace(0, 2, len(mass_values))

fig_mass = plotter.plot_mass_conservation(
    mass_history=mass_values,
    time_points=time_vals,
    relative=True,
    title="Mass Conservation Check",
    savepath=Path("results/mass_conservation.png")
)

print("Analysis workflow completed")
\end{lstlisting}

\subsection{Expected Integration with BioNetFlux Framework}

\begin{lstlisting}[language=Python, caption=Expected Framework Integration]
from setup_solver import quick_setup

# Setup complete solver framework
setup = quick_setup("ooc1d.problems.keller_segel_network")

# Create plotter
plotter = LeanMatplotlibPlotter()

# Initial conditions
trace_solutions, multipliers = setup.create_initial_conditions()
global_solution = setup.create_global_solution_vector(trace_solutions, multipliers)

# Initialize bulk data
bulk_data_list = setup.bulk_data_manager.initialize_all_bulk_data(
    setup.problems,
    setup.global_discretization.spatial_discretizations,
    time=0.0
)

# Plot initial network state
fig_initial = plotter.plot_multi_domain_network(
    bulk_data_list=bulk_data_list,
    problems=setup.problems,
    discretizations=setup.global_discretization.spatial_discretizations,
    constraint_manager=setup.constraint_manager,
    equation_idx=0,
    time=0.0,
    title="Initial Network State"
)

# Time evolution with plotting
time_history = []
bulk_history = []
mass_history = []

for time_step in range(setup.global_discretization.n_time_steps):
    current_time = time_step * setup.global_discretization.dt
    
    # Solve time step (simplified)
    # ... Newton iteration and solution update ...
    
    # Extract current bulk solutions
    current_bulk_list = setup.bulk_data_manager.initialize_all_bulk_data(
        setup.problems,
        setup.global_discretization.spatial_discretizations,
        time=current_time
    )
    
    # Track mass conservation
    total_mass = setup.bulk_data_manager.compute_total_mass(current_bulk_list)
    
    # Store for history
    time_history.append(current_time)
    bulk_history.append(current_bulk_list.copy())
    mass_history.append(total_mass)
    
    # Plot at specific intervals
    if time_step % 10 == 0:
        fig_step = plotter.plot_multi_domain_network(
            bulk_data_list=current_bulk_list,
            problems=setup.problems,
            discretizations=setup.global_discretization.spatial_discretizations,
            constraint_manager=setup.constraint_manager,
            equation_idx=0,
            time=current_time,
            savepath=Path(f"results/network_t_{time_step:04d}.png")
        )

# Final analysis plots
fig_mass_final = plotter.plot_mass_conservation(
    mass_history=mass_history,
    time_points=np.array(time_history),
    title="Mass Conservation Over Time"
)

# Create final animation
animation = plotter.create_time_evolution_animation(
    bulk_data_history=bulk_history,
    problems=setup.problems,
    discretizations=setup.global_discretization.spatial_discretizations,
    time_points=np.array(time_history),
    equation_idx=0,
    savepath=Path("results/complete_evolution.mp4")
)

print("Complete framework integration visualization finished")
\end{lstlisting}

\subsection{Expected Method Summary Table}
\label{subsec:expected_method_summary}

\begin{longtable}{|p{5cm}|p{3cm}|p{6.5cm}|}
\hline
\textbf{Expected Method} & \textbf{Returns} & \textbf{Purpose} \\
\hline
\endhead

\texttt{plot\_bulk\_solution} & \texttt{Figure} & Visualize bulk solution for single domain \\
\hline

\texttt{plot\_trace\_solution} & \texttt{Figure} & Plot HDG trace values at element boundaries \\
\hline

\texttt{plot\_multi\_domain\_network} & \texttt{Figure} & Visualize solution across network topology \\
\hline

\texttt{plot\_solution\_comparison} & \texttt{Figure} & Compare multiple solutions side-by-side \\
\hline

\texttt{plot\_mass\_conservation} & \texttt{Figure} & Track mass conservation over time \\
\hline

\texttt{create\_time\_animation} & \texttt{FuncAnimation} & Animate solution evolution \\
\hline

\texttt{plot\_newton\_convergence} & \texttt{Figure} & Show Newton solver convergence history \\
\hline

\texttt{plot\_error\_analysis} & \texttt{Figure} & Compare numerical vs analytical solutions \\
\hline

\texttt{set\_default\_style} & \texttt{None} & Configure plotting appearance \\
\hline

\texttt{save\_all\_figures} & \texttt{List[Path]} & Batch save multiple figures \\
\hline

\end{longtable}

\subsection{Expected Key Features}

\begin{itemize}
    \item \textbf{Lean Architecture}: No framework object storage, all passed as parameters
    \item \textbf{BulkData Integration}: Native support for BulkData solution format
    \item \textbf{Multi-Domain Networks}: Specialized visualization for connected domains
    \item \textbf{HDG-Specific Plots}: Trace and bulk solution visualization
    \item \textbf{Animation Support}: Time evolution and dynamic visualization
    \item \textbf{Analysis Tools}: Error analysis, convergence tracking, mass conservation
    \item \textbf{Flexible Styling}: Customizable appearance and output formats
    \item \textbf{Framework Integration}: Seamless integration with SolverSetup and BulkDataManager
\end{itemize}

\textbf{Note:} This documentation serves as a specification for implementing the lean matplotlib plotter module. The actual implementation would follow these interfaces and provide the described functionality.

% End of lean matplotlib plotter module API documentation (specification)
