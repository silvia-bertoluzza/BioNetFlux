% Setup Solver Module Detailed API Documentation (Accurate Analysis)
% To be included in master LaTeX document
%
% Usage: % Setup Solver Module Detailed API Documentation (Accurate Analysis)
% To be included in master LaTeX document
%
% Usage: % Setup Solver Module Detailed API Documentation (Accurate Analysis)
% To be included in master LaTeX document
%
% Usage: % Setup Solver Module Detailed API Documentation (Accurate Analysis)
% To be included in master LaTeX document
%
% Usage: \input{docs/setup_solver_detailed_api}

\section{Setup Solver Module Detailed API Reference}
\label{sec:setup_solver_detailed_api}

This section provides an exact reference for the setup solver module (\texttt{setup\_solver.py}) based on detailed analysis of the actual implementation. The module provides lean solver orchestration for OOC1D problems on networks, minimizing data redundancy while providing clean interfaces for different problem types.

\subsection{Module Overview}

The setup solver module provides:
\begin{itemize}
    \item Lean solver setup orchestration with minimal data storage
    \item Dynamic problem module loading and initialization
    \item On-demand component creation with caching
    \item Clean interfaces for accessing all solver components
    \item Comprehensive validation and testing capabilities
    \item Factory functions for common setup patterns
\end{itemize}

\subsection{Module Imports and Dependencies}

\begin{lstlisting}[language=Python, caption=Module Dependencies]
import numpy as np
from typing import List, Tuple, Optional, Dict, Any
import importlib

# Core imports
from ooc1d.core.discretization import Discretization, GlobalDiscretization
from ooc1d.utils.elementary_matrices import ElementaryMatrices
from ooc1d.core.static_condensation_factory import StaticCondensationFactory
from ooc1d.core.constraints import ConstraintManager
from ooc1d.core.lean_global_assembly import GlobalAssembler
from ooc1d.core.lean_bulk_data_manager import BulkDataManager
\end{lstlisting}

\subsection{SolverSetup Class}
\label{subsec:solver_setup_class_detailed}

Main class that orchestrates initialization of all solver components with lean data storage.

\subsubsection{Constructor}

\paragraph{\_\_init\_\_()}
\begin{lstlisting}[language=Python, caption=SolverSetup Constructor]
def __init__(self, problem_module: str = "ooc1d.problems.test_problem2")
\end{lstlisting}

\textbf{Parameters:}
\begin{itemize}
    \item \texttt{problem\_module}: String path to problem module containing \texttt{create\_global\_framework} (default: "ooc1d.problems.test\_problem2")
\end{itemize}

\textbf{Side Effects:} 
\begin{itemize}
    \item Sets \texttt{self.problem\_module} to specified module path
    \item Initializes \texttt{self.\_initialized} to False
    \item Sets all framework object attributes to None
    \item Sets all cached component attributes to None
\end{itemize}

\textbf{Usage:}
\begin{lstlisting}[language=Python, caption=Constructor Usage Examples]
# Default problem module
setup = SolverSetup()

# Custom problem module
setup = SolverSetup("ooc1d.problems.ooc_test_problem")

# OrganOnChip problem
setup = SolverSetup("ooc1d.problems.organ_on_chip_example")
\end{lstlisting}

\subsubsection{Core Attributes}

\begin{longtable}{|p{3.5cm}|p{2.5cm}|p{7cm}|}
\hline
\textbf{Attribute} & \textbf{Type} & \textbf{Description} \\
\hline
\endhead

\texttt{problem\_module} & \texttt{str} & String path to problem module \\
\hline

\texttt{\_initialized} & \texttt{bool} & Flag tracking initialization state \\
\hline

\texttt{problems} & \texttt{Optional[List]} & List of Problem instances (loaded on initialization) \\
\hline

\texttt{global\_discretization} & \texttt{Optional[GlobalDiscretization]} & Global discretization instance \\
\hline

\texttt{constraint\_manager} & \texttt{Optional[ConstraintManager]} & Constraint manager for boundary/junction conditions \\
\hline

\texttt{problem\_name} & \texttt{Optional[str]} & Descriptive name of the problem \\
\hline

\texttt{constraints} & \texttt{Optional[ConstraintManager]} & Alias for \texttt{constraint\_manager} (backward compatibility) \\
\hline

\end{longtable}

\subsubsection{Cached Component Attributes (Private)}

\begin{longtable}{|p{3.5cm}|p{2.5cm}|p{7cm}|}
\hline
\textbf{Attribute} & \textbf{Type} & \textbf{Description} \\
\hline
\endhead

\texttt{\_elementary\_matrices} & \texttt{Optional[ElementaryMatrices]} & Cached elementary matrices instance \\
\hline

\texttt{\_static\_condensations} & \texttt{Optional[List]} & Cached static condensation implementations \\
\hline

\texttt{\_global\_assembler} & \texttt{Optional[GlobalAssembler]} & Cached global assembler instance \\
\hline

\texttt{\_bulk\_data\_manager} & \texttt{Optional[BulkDataManager]} & Cached bulk data manager instance \\
\hline

\texttt{\_domain\_data} & \texttt{Optional[List]} & Cached extracted domain data \\
\hline

\end{longtable}

\subsubsection{Initialization Methods}

\paragraph{initialize()}
\begin{lstlisting}[language=Python, caption=Initialize Method]
def initialize(self) -> None
\end{lstlisting}

\textbf{Returns:} \texttt{None}

\textbf{Algorithm:}
\begin{enumerate}
    \item Check if already initialized (early return if True)
    \item Import problem module using \texttt{importlib.import\_module()}
    \item Get \texttt{create\_global\_framework} function from module
    \item Call function to get: problems, global\_discretization, constraint\_manager, problem\_name
    \item Set \texttt{self.constraints} as alias to \texttt{constraint\_manager}
    \item Set \texttt{self.\_initialized} to True
\end{enumerate}

\textbf{Side Effects:}
\begin{itemize}
    \item Imports specified problem module
    \item Calls \texttt{create\_global\_framework()} from module
    \item Sets core problem attributes
    \item Sets \texttt{\_initialized} flag to True
\end{itemize}

\textbf{Raises:} \texttt{ImportError} if problem module or function not found

\textbf{Usage:}
\begin{lstlisting}[language=Python, caption=Initialize Usage]
setup = SolverSetup("ooc1d.problems.ooc_test_problem")
setup.initialize()  # Explicit initialization

# Or use properties that auto-initialize
print(setup.problem_name)  # Triggers initialization if needed
\end{lstlisting}

\paragraph{\_ensure\_initialized()}
\begin{lstlisting}[language=Python, caption=Ensure Initialized Method]
def _ensure_initialized(self) -> None
\end{lstlisting}

\textbf{Purpose:} Internal method ensuring initialization has occurred

\textbf{Algorithm:}
\begin{lstlisting}[language=Python, caption=Ensure Initialized Implementation]
if not self._initialized:
    self.initialize()
\end{lstlisting}

\textbf{Usage:} Called automatically by property accessors

\subsubsection{Component Properties (Cached)}

\paragraph{elementary\_matrices}
\begin{lstlisting}[language=Python, caption=Elementary Matrices Property]
@property
def elementary_matrices(self) -> ElementaryMatrices
\end{lstlisting}

\textbf{Returns:} \texttt{ElementaryMatrices} - Elementary matrices instance (created once, cached)

\textbf{Algorithm:}
\begin{lstlisting}[language=Python, caption=Elementary Matrices Property Implementation]
if self._elementary_matrices is None:
    self._elementary_matrices = ElementaryMatrices(orthonormal_basis=False)
return self._elementary_matrices
\end{lstlisting}

\textbf{Configuration:} Uses Lagrange basis (\texttt{orthonormal\_basis=False})

\textbf{Usage:}
\begin{lstlisting}[language=Python, caption=Elementary Matrices Usage]
setup = SolverSetup()
elem_matrices = setup.elementary_matrices

# Get specific matrices
M = elem_matrices.get_matrix('M')        # Mass matrix
T = elem_matrices.get_matrix('T')        # Trace matrix
D = elem_matrices.get_matrix('D')        # Derivative matrix
QUAD = elem_matrices.get_matrix('QUAD')  # Quadrature matrix
\end{lstlisting}

\paragraph{static\_condensations}
\begin{lstlisting}[language=Python, caption=Static Condensations Property]
@property
def static_condensations(self) -> List
\end{lstlisting}

\textbf{Returns:} \texttt{List} - Static condensation implementations for all domains (created once, cached)

\textbf{Algorithm:}
\begin{lstlisting}[language=Python, caption=Static Condensations Implementation]
if self._static_condensations is None:
    self._ensure_initialized()
    self._static_condensations = []
    
    for domain_idx in range(len(self.problems)):
        sc = StaticCondensationFactory.create(
            self.problems[domain_idx],
            self.global_discretization,
            self.elementary_matrices,
            domain_idx
        )
        # Build matrices immediately to cache them
        sc.build_matrices()
        self._static_condensations.append(sc)
        
return self._static_condensations
\end{lstlisting}

\textbf{Process:}
\begin{enumerate}
    \item Creates static condensation for each domain using factory
    \item Immediately builds matrices for caching
    \item Returns list of configured implementations
\end{enumerate}

\textbf{Usage:}
\begin{lstlisting}[language=Python, caption=Static Condensations Usage]
static_condensations = setup.static_condensations

# Access specific domain
sc_domain_0 = static_condensations[0]
matrices = sc_domain_0.build_matrices()  # Already built and cached

# Use in computations
for i, sc in enumerate(static_condensations):
    print(f"Domain {i}: {type(sc).__name__}")
\end{lstlisting}

\paragraph{global\_assembler}
\begin{lstlisting}[language=Python, caption=Global Assembler Property]
@property
def global_assembler(self) -> GlobalAssembler
\end{lstlisting}

\textbf{Returns:} \texttt{GlobalAssembler} - Global assembler instance (created once, cached)

\textbf{Algorithm:}
\begin{lstlisting}[language=Python, caption=Global Assembler Implementation]
if self._global_assembler is None:
    self._ensure_initialized()
    # Check the actual constructor signature and use correct parameters
    self._global_assembler = GlobalAssembler.from_framework_objects(
        self.problems,
        self.global_discretization,
        self.static_condensations,
        self.constraint_manager
    )
return self._global_assembler
\end{lstlisting}

\textbf{Creation:} Uses factory method \texttt{GlobalAssembler.from\_framework\_objects()}

\textbf{Usage:}
\begin{lstlisting}[language=Python, caption=Global Assembler Usage]
assembler = setup.global_assembler

print(f"Total DOFs: {assembler.total_dofs}")
print(f"Trace DOFs: {assembler.total_trace_dofs}")
print(f"Multipliers: {assembler.n_multipliers}")

# Use for system assembly
global_solution = np.random.rand(assembler.total_dofs)
residual, jacobian = assembler.assemble_residual_and_jacobian(
    global_solution, forcing_terms, setup.static_condensations, time=0.0
)
\end{lstlisting}

\paragraph{bulk\_data\_manager}
\begin{lstlisting}[language=Python, caption=Bulk Data Manager Property]
@property
def bulk_data_manager(self) -> BulkDataManager
\end{lstlisting}

\textbf{Returns:} \texttt{BulkDataManager} - Bulk data manager instance (created once, cached)

\textbf{Algorithm:}
\begin{lstlisting}[language=Python, caption=Bulk Data Manager Implementation]
if self._bulk_data_manager is None:
    self._ensure_initialized()
    discretizations = self.global_discretization.spatial_discretizations
    self._domain_data = BulkDataManager.extract_domain_data_list(
        self.problems, discretizations, self.static_condensations
    )
    self._bulk_data_manager = BulkDataManager(
        self._domain_data
    )
return self._bulk_data_manager
\end{lstlisting}

\textbf{Process:}
\begin{enumerate}
    \item Extracts domain data using \texttt{BulkDataManager.extract\_domain\_data\_list()}
    \item Creates lean BulkDataManager with extracted data
    \item Caches both domain data and manager
\end{enumerate}

\textbf{Usage:}
\begin{lstlisting}[language=Python, caption=Bulk Data Manager Usage]
bulk_manager = setup.bulk_data_manager

# Initialize bulk data
bulk_data_list = bulk_manager.initialize_all_bulk_data(
    setup.problems,
    setup.global_discretization.spatial_discretizations,
    time=0.0
)

# Compute forcing terms
forcing_terms = bulk_manager.compute_forcing_terms(
    bulk_data_list, setup.problems,
    setup.global_discretization.spatial_discretizations,
    time=0.1, dt=0.01
)
\end{lstlisting}

\subsubsection{Information and Analysis Methods}

\paragraph{get\_problem\_info()}
\begin{lstlisting}[language=Python, caption=Get Problem Info Method]
def get_problem_info(self) -> Dict[str, Any]
\end{lstlisting}

\textbf{Returns:} \texttt{Dict[str, Any]} - Comprehensive problem information dictionary

\textbf{Algorithm:}
\begin{lstlisting}[language=Python, caption=Problem Info Algorithm]
self._ensure_initialized()

info = {
    'problem_name': self.problem_name,
    'num_domains': len(self.problems),
    'total_elements': sum(disc.n_elements for disc in self.global_discretization.spatial_discretizations),
    'total_trace_dofs': self.global_assembler.total_trace_dofs,
    'num_constraints': self.constraint_manager.n_multipliers if self.constraint_manager else 0,
    'time_discretization': {
        'dt': self.global_discretization.dt,
        'T': self.global_discretization.T,
        'n_steps': self.global_discretization.n_time_steps
    },
    'domains': []
}

for i, (problem, discretization) in enumerate(zip(self.problems, self.global_discretization.spatial_discretizations)):
    domain_info = {
        'index': i,
        'type': problem.type,
        'domain': [problem.domain_start, problem.domain_end],
        'n_elements': discretization.n_elements,
        'n_equations': problem.neq,
        'trace_size': problem.neq * (discretization.n_elements + 1)
    }
    info['domains'].append(domain_info)

return info
\end{lstlisting}

\textbf{Information Structure:}
\begin{lstlisting}[language=Python, caption=Problem Info Structure]
{
    'problem_name': str,
    'num_domains': int,
    'total_elements': int,
    'total_trace_dofs': int,
    'num_constraints': int,
    'time_discretization': {
        'dt': float,
        'T': float,
        'n_steps': int
    },
    'domains': [
        {
            'index': int,
            'type': str,
            'domain': [float, float],  # [start, end]
            'n_elements': int,
            'n_equations': int,
            'trace_size': int
        },
        # ... for each domain
    ]
}
\end{lstlisting}

\textbf{Usage:}
\begin{lstlisting}[language=Python, caption=Problem Info Usage]
info = setup.get_problem_info()

print(f"Problem: {info['problem_name']}")
print(f"Domains: {info['num_domains']}")
print(f"Total DOFs: {info['total_trace_dofs']}")
print(f"Time stepping: dt={info['time_discretization']['dt']}, T={info['time_discretization']['T']}")

# Domain details
for domain_info in info['domains']:
    print(f"  Domain {domain_info['index']}: {domain_info['type']}")
    print(f"    Range: {domain_info['domain']}")
    print(f"    Elements: {domain_info['n_elements']}")
    print(f"    Equations: {domain_info['n_equations']}")
\end{lstlisting}

\subsubsection{Initial Condition and Solution Vector Management}

\paragraph{create\_initial\_conditions()}
\begin{lstlisting}[language=Python, caption=Create Initial Conditions Method]
def create_initial_conditions(self) -> Tuple[List[np.ndarray], np.ndarray]
\end{lstlisting}

\textbf{Returns:} \texttt{Tuple[List[np.ndarray], np.ndarray]} - (trace\_solutions, initial\_multipliers)

\textbf{Algorithm:}
\begin{lstlisting}[language=Python, caption=Initial Conditions Algorithm]
self._ensure_initialized()

trace_solutions = []

for i, (problem, discretization) in enumerate(zip(self.problems, self.global_discretization.spatial_discretizations)):
    n_nodes = discretization.n_elements + 1
    trace_size = problem.neq * n_nodes
    nodes = discretization.nodes
    
    trace_solution = np.zeros(trace_size)
    
    # Apply initial conditions if available
    for eq in range(problem.neq):
        for j in range(n_nodes):
            node_idx = eq * n_nodes + j
            if hasattr(problem, 'u0') and len(problem.u0) > eq and callable(problem.u0[eq]):
                trace_solution[node_idx] = problem.u0[eq](nodes[j])
            elif hasattr(problem, 'initial_conditions') and len(problem.initial_conditions) > eq:
                if callable(problem.initial_conditions[eq]):
                    trace_solution[node_idx] = problem.initial_conditions[eq](nodes[j])
    
    trace_solutions.append(trace_solution)

# Initialize multipliers to zero
n_multipliers = self.constraint_manager.n_multipliers if self.constraint_manager else 0
initial_multipliers = np.zeros(n_multipliers)

return trace_solutions, initial_multipliers
\end{lstlisting}

\textbf{Process:}
\begin{enumerate}
    \item For each domain: create trace solution array
    \item For each equation: evaluate initial condition at mesh nodes
    \item Initialize multipliers to zero
    \item Return domain traces and multiplier arrays
\end{enumerate}

\textbf{Initial Condition Access Patterns:}
\begin{itemize}
    \item \texttt{problem.u0[eq]} (primary pattern)
    \item \texttt{problem.initial\_conditions[eq]} (fallback)
    \item Zero initialization if no initial conditions found
\end{itemize}

\textbf{Usage:}
\begin{lstlisting}[language=Python, caption=Initial Conditions Usage]
trace_solutions, multipliers = setup.create_initial_conditions()

print(f"Created {len(trace_solutions)} domain trace solutions")
for i, trace_sol in enumerate(trace_solutions):
    print(f"  Domain {i}: shape {trace_sol.shape}")

print(f"Initial multipliers: {len(multipliers)} (all zero)")
\end{lstlisting}

\paragraph{create\_global\_solution\_vector()}
\begin{lstlisting}[language=Python, caption=Create Global Solution Vector Method]
def create_global_solution_vector(self, trace_solutions: List[np.ndarray], 
                                 multipliers: np.ndarray) -> np.ndarray
\end{lstlisting}

\textbf{Parameters:}
\begin{itemize}
    \item \texttt{trace\_solutions}: List of trace solution arrays for each domain
    \item \texttt{multipliers}: Array of Lagrange multiplier values
\end{itemize}

\textbf{Returns:} \texttt{np.ndarray} - Global solution vector

\textbf{Algorithm:}
\begin{lstlisting}[language=Python, caption=Global Vector Assembly Algorithm]
global_assembler = self.global_assembler

# Calculate total size
total_trace_size = sum(len(trace) for trace in trace_solutions)
total_size = total_trace_size + len(multipliers)

# Create global solution vector
global_solution = np.zeros(total_size)

# Fill trace solutions
offset = 0
for trace in trace_solutions:
    trace_flat = trace.flatten() if trace.ndim > 1 else trace
    global_solution[offset:offset+len(trace_flat)] = trace_flat
    offset += len(trace_flat)

# Fill multipliers
if len(multipliers) > 0:
    global_solution[offset:offset+len(multipliers)] = multipliers

return global_solution
\end{lstlisting}

\textbf{Assembly Structure:} \texttt{[trace\_domain\_0, trace\_domain\_1, ..., multipliers]}

\textbf{Usage:}
\begin{lstlisting}[language=Python, caption=Global Vector Assembly Usage]
# Create initial conditions
trace_solutions, multipliers = setup.create_initial_conditions()

# Assemble global vector
global_solution = setup.create_global_solution_vector(trace_solutions, multipliers)

print(f"Global solution vector shape: {global_solution.shape}")
print(f"Total DOFs: {len(global_solution)}")

# Verify size matches assembler
expected_size = setup.global_assembler.total_dofs
print(f"Matches assembler DOFs: {len(global_solution) == expected_size}")
\end{lstlisting}

\paragraph{extract\_domain\_solutions()}
\begin{lstlisting}[language=Python, caption=Extract Domain Solutions Method]
def extract_domain_solutions(self, global_solution: np.ndarray) -> Tuple[List[np.ndarray], np.ndarray]
\end{lstlisting}

\textbf{Parameters:}
\begin{itemize}
    \item \texttt{global\_solution}: Global solution vector
\end{itemize}

\textbf{Returns:} \texttt{Tuple[List[np.ndarray], np.ndarray]} - (trace\_solutions, multipliers)

\textbf{Algorithm:}
\begin{lstlisting}[language=Python, caption=Domain Solution Extraction Algorithm]
self._ensure_initialized()

trace_solutions = []
offset = 0

# Extract trace solutions for each domain
for i, (problem, discretization) in enumerate(zip(self.problems, self.global_discretization.spatial_discretizations)):
    n_nodes = discretization.n_elements + 1
    trace_size = problem.neq * n_nodes
    
    trace_solution = global_solution[offset:offset+trace_size]
    trace_solutions.append(trace_solution)
    offset += trace_size

# Extract multipliers
n_multipliers = self.constraint_manager.n_multipliers if self.constraint_manager else 0
if n_multipliers > 0:
    multipliers = global_solution[offset:offset+n_multipliers]
else:
    multipliers = np.array([])

return trace_solutions, multipliers
\end{lstlisting}

\textbf{Purpose:} Inverse operation of \texttt{create\_global\_solution\_vector()}

\textbf{Usage:}
\begin{lstlisting}[language=Python, caption=Domain Solution Extraction Usage]
# Extract from global solution
extracted_traces, extracted_multipliers = setup.extract_domain_solutions(global_solution)

# Verify round-trip consistency
for i, (orig, extracted) in enumerate(zip(trace_solutions, extracted_traces)):
    consistent = np.allclose(orig, extracted)
    print(f"Domain {i} round-trip consistent: {consistent}")

multiplier_consistent = np.allclose(multipliers, extracted_multipliers)
print(f"Multiplier round-trip consistent: {multiplier_consistent}")
\end{lstlisting}

\subsubsection{Validation and Testing}

\paragraph{validate\_setup()}
\begin{lstlisting}[language=Python, caption=Validate Setup Method]
def validate_setup(self, verbose: bool = False) -> bool
\end{lstlisting}

\textbf{Parameters:}
\begin{itemize}
    \item \texttt{verbose}: If True, print detailed validation information (default: False)
\end{itemize}

\textbf{Returns:} \texttt{bool} - True if all validation tests pass, False otherwise

\textbf{Validation Tests:}
\begin{enumerate}
    \item \textbf{Initial Condition Creation}: Tests creation of trace solutions and multipliers
    \item \textbf{Global Vector Round-Trip}: Tests assembly and extraction consistency
    \item \textbf{Multiplier Round-Trip}: Validates multiplier handling
    \item \textbf{Bulk Solution Creation}: Tests BulkData creation for all domains
    \item \textbf{Forcing Term Computation}: Validates forcing term calculation
    \item \textbf{Residual/Jacobian Assembly}: Tests global system assembly
\end{enumerate}

\textbf{Algorithm:}
\begin{lstlisting}[language=Python, caption=Validate Setup Algorithm]
try:
    self._ensure_initialized()
    
    if verbose:
        print(f"Validating setup for problem: {self.problem_name}")
        print(f"Number of domains: {len(self.problems)}")
        print(f"Total DOFs: {self.global_assembler.total_dofs}")
    
    # Test initial conditions
    trace_solutions, multipliers = self.create_initial_conditions()
    if verbose:
        print(f"✓ Initial conditions created")
    
    # Test global vector assembly/extraction
    global_solution = self.create_global_solution_vector(trace_solutions, multipliers)
    extracted_traces, extracted_multipliers = self.extract_domain_solutions(global_solution)
    
    # Verify round-trip consistency
    for i, (orig, extracted) in enumerate(zip(trace_solutions, extracted_traces)):
        if not np.allclose(orig, extracted):
            if verbose:
                print(f"✗ Round-trip test failed for domain {i}")
            return False
    
    if not np.allclose(multipliers, extracted_multipliers):
        if verbose:
            print(f"✗ Round-trip test failed for multipliers")
        return False
    
    if verbose:
        print(f"✓ Global vector round-trip test passed")
    
    # Test bulk data manager
    bulk_solutions = []
    for i in range(len(self.problems)):
        problem = self.problems[i]
        discretization = self.global_discretization.spatial_discretizations[i]
        bulk_sol = self.bulk_data_manager.create_bulk_data(i, problem, discretization)
        bulk_solutions.append(bulk_sol)
    
    if verbose:
        print(f"✓ Bulk solutions created")
    
    # Test forcing term computation
    forcing_terms = self.bulk_data_manager.compute_forcing_terms(bulk_solutions, 
                                                                 self.problems, 
                                                                 self.global_discretization.spatial_discretizations, 
                                                                 0.0, 
                                                                 self.global_discretization.dt
                                                                 )

    if verbose:
        print(f"✓ Forcing terms computed")
    
    # Test residual/jacobian computation
    global_residual, global_jacobian = self.global_assembler.assemble_residual_and_jacobian(
        global_solution=global_solution,
        forcing_terms=forcing_terms,
        static_condensations=self._static_condensations,
        time=0.0
    )
    
    if verbose:
        print(f"✓ Global residual and Jacobian assembled")
        print(f"  Residual norm: {np.linalg.norm(global_residual):.6e}")
        print(f"  Jacobian condition number: {np.linalg.cond(global_jacobian):.2e}")
    
    return True
    
except Exception as e:
    if verbose:
        print(f"✗ Validation failed: {e}")
        import traceback
        traceback.print_exc()
    return False
\end{lstlisting}

\textbf{Usage:}
\begin{lstlisting}[language=Python, caption=Validation Usage]
# Quick validation
is_valid = setup.validate_setup()
if is_valid:
    print("✓ Setup validation passed")
else:
    print("✗ Setup validation failed")

# Detailed validation with output
is_valid = setup.validate_setup(verbose=True)
\end{lstlisting}

\textbf{Sample Validation Output:}
\begin{lstlisting}[language=Python, caption=Sample Validation Output]
Validating setup for problem: OrganOnChip Test Problem
Number of domains: 1
Total DOFs: 86
✓ Initial conditions created
✓ Global vector round-trip test passed
✓ Bulk solutions created
✓ Forcing terms computed
✓ Global residual and Jacobian assembled
  Residual norm: 1.234567e-02
  Jacobian condition number: 2.34e+03
\end{lstlisting}

\subsection{Factory Functions}
\label{subsec:factory_functions_detailed}

\paragraph{create\_solver\_setup()}
\begin{lstlisting}[language=Python, caption=Create Solver Setup Function]
def create_solver_setup(problem_module: str = "ooc1d.problems.test_problem2") -> SolverSetup
\end{lstlisting}

\textbf{Parameters:}
\begin{itemize}
    \item \texttt{problem\_module}: String path to problem module (default: "ooc1d.problems.test\_problem2")
\end{itemize}

\textbf{Returns:} \texttt{SolverSetup} - Initialized SolverSetup instance

\textbf{Algorithm:}
\begin{lstlisting}[language=Python, caption=Factory Function Implementation]
setup = SolverSetup(problem_module)
setup.initialize()
return setup
\end{lstlisting}

\textbf{Usage:}
\begin{lstlisting}[language=Python, caption=Factory Function Usage]
# Create and initialize setup
setup = create_solver_setup("ooc1d.problems.ooc_test_problem")

# Immediate access to components
assembler = setup.global_assembler
bulk_manager = setup.bulk_data_manager
\end{lstlisting}

\paragraph{quick\_setup()}
\begin{lstlisting}[language=Python, caption=Quick Setup Function]
def quick_setup(problem_module: str = "ooc1d.problems.test_problem2", 
               validate: bool = True) -> SolverSetup
\end{lstlisting}

\textbf{Parameters:}
\begin{itemize}
    \item \texttt{problem\_module}: String path to problem module (default: "ooc1d.problems.test\_problem2")
    \item \texttt{validate}: If True, run validation tests (default: True)
\end{itemize}

\textbf{Returns:} \texttt{SolverSetup} - Validated SolverSetup instance

\textbf{Raises:} \texttt{RuntimeError} if validation fails

\textbf{Algorithm:}
\begin{lstlisting}[language=Python, caption=Quick Setup Implementation]
setup = create_solver_setup(problem_module)
if validate:
    if not setup.validate_setup(verbose=True):
        raise RuntimeError("Setup validation failed")

return setup
\end{lstlisting}

\textbf{Usage:}
\begin{lstlisting}[language=Python, caption=Quick Setup Usage]
# Quick setup with validation
setup = quick_setup("ooc1d.problems.ooc_test_problem", validate=True)

# Quick setup without validation (faster)
setup = quick_setup("ooc1d.problems.test_problem", validate=False)
\end{lstlisting}

\subsection{Complete Usage Examples}
\label{subsec:complete_usage_examples_detailed}

\subsubsection{Complete Solver Setup and Newton Iteration}

\begin{lstlisting}[language=Python, caption=Complete Solver Setup Example]
from setup_solver import quick_setup, SolverSetup
import numpy as np

def complete_solver_example():
    """Complete example of solver setup and Newton iteration."""
    
    # Step 1: Initialize solver with validation
    print("Setting up solver...")
    setup = quick_setup("ooc1d.problems.ooc_test_problem", validate=True)
    
    # Step 2: Get problem information
    info = setup.get_problem_info()
    print(f"\nProblem: {info['problem_name']}")
    print(f"Domains: {info['num_domains']}")
    print(f"Total DOFs: {info['total_trace_dofs'] + info['num_constraints']}")
    
    # Step 3: Create initial conditions
    trace_solutions, multipliers = setup.create_initial_conditions()
    global_solution = setup.create_global_solution_vector(trace_solutions, multipliers)
    
    print(f"\nInitial conditions created:")
    print(f"  Global solution shape: {global_solution.shape}")
    print(f"  Initial residual norm: {np.linalg.norm(global_solution):.6e}")
    
    # Step 4: Setup Newton iteration
    assembler = setup.global_assembler
    bulk_manager = setup.bulk_data_manager
    static_condensations = setup.static_condensations
    
    # Step 5: Newton iteration loop
    tolerance = 1e-10
    max_iterations = 20
    current_time = 0.0
    dt = setup.global_discretization.dt
    
    print(f"\nStarting Newton iterations (tol={tolerance:.0e})...")
    
    for iteration in range(max_iterations):
        # Create bulk data for forcing terms
        bulk_data_list = bulk_manager.initialize_all_bulk_data(
            setup.problems,
            setup.global_discretization.spatial_discretizations,
            time=current_time
        )
        
        # Compute forcing terms
        forcing_terms = bulk_manager.compute_forcing_terms(
            bulk_data_list, setup.problems,
            setup.global_discretization.spatial_discretizations,
            current_time, dt
        )
        
        # Assemble system
        residual, jacobian = assembler.assemble_residual_and_jacobian(
            global_solution=global_solution,
            forcing_terms=forcing_terms,
            static_condensations=static_condensations,
            time=current_time
        )
        
        # Check convergence
        residual_norm = np.linalg.norm(residual)
        print(f"  Iteration {iteration}: ||R|| = {residual_norm:.6e}")
        
        if residual_norm < tolerance:
            print("  ✓ Newton solver converged")
            break
        
        # Newton update
        try:
            delta = np.linalg.solve(jacobian, -residual)
            global_solution += delta
        except np.linalg.LinAlgError:
            print("  ✗ Newton solver failed: singular Jacobian")
            break
    
    # Step 6: Extract final solutions
    final_traces, final_multipliers = setup.extract_domain_solutions(global_solution)
    
    print(f"\nSolver completed:")
    print(f"  Final residual norm: {residual_norm:.6e}")
    print(f"  Domain solutions: {len(final_traces)}")
    print(f"  Constraint multipliers: {len(final_multipliers)}")
    
    return setup, global_solution, final_traces, final_multipliers

# Usage
setup, solution, traces, multipliers = complete_solver_example()
\end{lstlisting}

\subsection{Method Summary Table}
\label{subsec:method_summary_detailed}

\begin{longtable}{|p{4cm}|p{2.5cm}|p{6.5cm}|}
\hline
\textbf{Method/Property} & \textbf{Returns} & \textbf{Purpose} \\
\hline
\endhead

\texttt{\_\_init\_\_} & \texttt{None} & Initialize setup with problem module path \\
\hline

\texttt{initialize} & \texttt{None} & Load and initialize problem configuration \\
\hline

\texttt{elementary\_matrices} & \texttt{ElementaryMatrices} & Get cached elementary matrices \\
\hline

\texttt{static\_condensations} & \texttt{List} & Get cached static condensation implementations \\
\hline

\texttt{global\_assembler} & \texttt{GlobalAssembler} & Get cached global assembler \\
\hline

\texttt{bulk\_data\_manager} & \texttt{BulkDataManager} & Get cached bulk data manager \\
\hline

\texttt{get\_problem\_info} & \texttt{Dict} & Get comprehensive problem information \\
\hline

\texttt{create\_initial\_conditions} & \texttt{Tuple} & Create initial trace solutions and multipliers \\
\hline

\texttt{create\_global\_solution\_vector} & \texttt{np.ndarray} & Assemble global solution from components \\
\hline

\texttt{extract\_domain\_solutions} & \texttt{Tuple} & Extract components from global solution \\
\hline

\texttt{validate\_setup} & \texttt{bool} & Run comprehensive validation tests \\
\hline

\texttt{create\_solver\_setup} & \texttt{SolverSetup} & Factory function for setup creation \\
\hline

\texttt{quick\_setup} & \texttt{SolverSetup} & Factory with optional validation \\
\hline

\end{longtable}

\subsection{Key Features and Design Principles}

\begin{itemize}
    \item \textbf{Lean Architecture}: Minimal data storage with on-demand component creation
    \item \textbf{Component Caching}: Expensive objects created once and cached
    \item \textbf{Dynamic Loading}: Problem modules loaded at runtime via importlib
    \item \textbf{Lazy Initialization}: Components created only when accessed
    \item \textbf{Comprehensive Validation}: Built-in testing of all components and round-trip consistency
    \item \textbf{Clean Interfaces}: Simple property-based access to complex components
    \item \textbf{Factory Patterns}: Convenience functions for common setup scenarios
    \item \textbf{Backward Compatibility}: Aliases and fallbacks for existing code
    \item \textbf{Error Handling}: Robust error reporting and fallback strategies
    \item \textbf{Flexible Architecture}: Support for arbitrary problem modules and configurations
\end{itemize}

This documentation provides an exact reference for the setup solver module, emphasizing its role as the central orchestrator for BioNetFlux solver initialization and its lean, efficient approach to component management.

% End of setup solver module detailed API documentation


\section{Setup Solver Module Detailed API Reference}
\label{sec:setup_solver_detailed_api}

This section provides an exact reference for the setup solver module (\texttt{setup\_solver.py}) based on detailed analysis of the actual implementation. The module provides lean solver orchestration for OOC1D problems on networks, minimizing data redundancy while providing clean interfaces for different problem types.

\subsection{Module Overview}

The setup solver module provides:
\begin{itemize}
    \item Lean solver setup orchestration with minimal data storage
    \item Dynamic problem module loading and initialization
    \item On-demand component creation with caching
    \item Clean interfaces for accessing all solver components
    \item Comprehensive validation and testing capabilities
    \item Factory functions for common setup patterns
\end{itemize}

\subsection{Module Imports and Dependencies}

\begin{lstlisting}[language=Python, caption=Module Dependencies]
import numpy as np
from typing import List, Tuple, Optional, Dict, Any
import importlib

# Core imports
from ooc1d.core.discretization import Discretization, GlobalDiscretization
from ooc1d.utils.elementary_matrices import ElementaryMatrices
from ooc1d.core.static_condensation_factory import StaticCondensationFactory
from ooc1d.core.constraints import ConstraintManager
from ooc1d.core.lean_global_assembly import GlobalAssembler
from ooc1d.core.lean_bulk_data_manager import BulkDataManager
\end{lstlisting}

\subsection{SolverSetup Class}
\label{subsec:solver_setup_class_detailed}

Main class that orchestrates initialization of all solver components with lean data storage.

\subsubsection{Constructor}

\paragraph{\_\_init\_\_()}
\begin{lstlisting}[language=Python, caption=SolverSetup Constructor]
def __init__(self, problem_module: str = "ooc1d.problems.test_problem2")
\end{lstlisting}

\textbf{Parameters:}
\begin{itemize}
    \item \texttt{problem\_module}: String path to problem module containing \texttt{create\_global\_framework} (default: "ooc1d.problems.test\_problem2")
\end{itemize}

\textbf{Side Effects:} 
\begin{itemize}
    \item Sets \texttt{self.problem\_module} to specified module path
    \item Initializes \texttt{self.\_initialized} to False
    \item Sets all framework object attributes to None
    \item Sets all cached component attributes to None
\end{itemize}

\textbf{Usage:}
\begin{lstlisting}[language=Python, caption=Constructor Usage Examples]
# Default problem module
setup = SolverSetup()

# Custom problem module
setup = SolverSetup("ooc1d.problems.ooc_test_problem")

# OrganOnChip problem
setup = SolverSetup("ooc1d.problems.organ_on_chip_example")
\end{lstlisting}

\subsubsection{Core Attributes}

\begin{longtable}{|p{3.5cm}|p{2.5cm}|p{7cm}|}
\hline
\textbf{Attribute} & \textbf{Type} & \textbf{Description} \\
\hline
\endhead

\texttt{problem\_module} & \texttt{str} & String path to problem module \\
\hline

\texttt{\_initialized} & \texttt{bool} & Flag tracking initialization state \\
\hline

\texttt{problems} & \texttt{Optional[List]} & List of Problem instances (loaded on initialization) \\
\hline

\texttt{global\_discretization} & \texttt{Optional[GlobalDiscretization]} & Global discretization instance \\
\hline

\texttt{constraint\_manager} & \texttt{Optional[ConstraintManager]} & Constraint manager for boundary/junction conditions \\
\hline

\texttt{problem\_name} & \texttt{Optional[str]} & Descriptive name of the problem \\
\hline

\texttt{constraints} & \texttt{Optional[ConstraintManager]} & Alias for \texttt{constraint\_manager} (backward compatibility) \\
\hline

\end{longtable}

\subsubsection{Cached Component Attributes (Private)}

\begin{longtable}{|p{3.5cm}|p{2.5cm}|p{7cm}|}
\hline
\textbf{Attribute} & \textbf{Type} & \textbf{Description} \\
\hline
\endhead

\texttt{\_elementary\_matrices} & \texttt{Optional[ElementaryMatrices]} & Cached elementary matrices instance \\
\hline

\texttt{\_static\_condensations} & \texttt{Optional[List]} & Cached static condensation implementations \\
\hline

\texttt{\_global\_assembler} & \texttt{Optional[GlobalAssembler]} & Cached global assembler instance \\
\hline

\texttt{\_bulk\_data\_manager} & \texttt{Optional[BulkDataManager]} & Cached bulk data manager instance \\
\hline

\texttt{\_domain\_data} & \texttt{Optional[List]} & Cached extracted domain data \\
\hline

\end{longtable}

\subsubsection{Initialization Methods}

\paragraph{initialize()}
\begin{lstlisting}[language=Python, caption=Initialize Method]
def initialize(self) -> None
\end{lstlisting}

\textbf{Returns:} \texttt{None}

\textbf{Algorithm:}
\begin{enumerate}
    \item Check if already initialized (early return if True)
    \item Import problem module using \texttt{importlib.import\_module()}
    \item Get \texttt{create\_global\_framework} function from module
    \item Call function to get: problems, global\_discretization, constraint\_manager, problem\_name
    \item Set \texttt{self.constraints} as alias to \texttt{constraint\_manager}
    \item Set \texttt{self.\_initialized} to True
\end{enumerate}

\textbf{Side Effects:}
\begin{itemize}
    \item Imports specified problem module
    \item Calls \texttt{create\_global\_framework()} from module
    \item Sets core problem attributes
    \item Sets \texttt{\_initialized} flag to True
\end{itemize}

\textbf{Raises:} \texttt{ImportError} if problem module or function not found

\textbf{Usage:}
\begin{lstlisting}[language=Python, caption=Initialize Usage]
setup = SolverSetup("ooc1d.problems.ooc_test_problem")
setup.initialize()  # Explicit initialization

# Or use properties that auto-initialize
print(setup.problem_name)  # Triggers initialization if needed
\end{lstlisting}

\paragraph{\_ensure\_initialized()}
\begin{lstlisting}[language=Python, caption=Ensure Initialized Method]
def _ensure_initialized(self) -> None
\end{lstlisting}

\textbf{Purpose:} Internal method ensuring initialization has occurred

\textbf{Algorithm:}
\begin{lstlisting}[language=Python, caption=Ensure Initialized Implementation]
if not self._initialized:
    self.initialize()
\end{lstlisting}

\textbf{Usage:} Called automatically by property accessors

\subsubsection{Component Properties (Cached)}

\paragraph{elementary\_matrices}
\begin{lstlisting}[language=Python, caption=Elementary Matrices Property]
@property
def elementary_matrices(self) -> ElementaryMatrices
\end{lstlisting}

\textbf{Returns:} \texttt{ElementaryMatrices} - Elementary matrices instance (created once, cached)

\textbf{Algorithm:}
\begin{lstlisting}[language=Python, caption=Elementary Matrices Property Implementation]
if self._elementary_matrices is None:
    self._elementary_matrices = ElementaryMatrices(orthonormal_basis=False)
return self._elementary_matrices
\end{lstlisting}

\textbf{Configuration:} Uses Lagrange basis (\texttt{orthonormal\_basis=False})

\textbf{Usage:}
\begin{lstlisting}[language=Python, caption=Elementary Matrices Usage]
setup = SolverSetup()
elem_matrices = setup.elementary_matrices

# Get specific matrices
M = elem_matrices.get_matrix('M')        # Mass matrix
T = elem_matrices.get_matrix('T')        # Trace matrix
D = elem_matrices.get_matrix('D')        # Derivative matrix
QUAD = elem_matrices.get_matrix('QUAD')  # Quadrature matrix
\end{lstlisting}

\paragraph{static\_condensations}
\begin{lstlisting}[language=Python, caption=Static Condensations Property]
@property
def static_condensations(self) -> List
\end{lstlisting}

\textbf{Returns:} \texttt{List} - Static condensation implementations for all domains (created once, cached)

\textbf{Algorithm:}
\begin{lstlisting}[language=Python, caption=Static Condensations Implementation]
if self._static_condensations is None:
    self._ensure_initialized()
    self._static_condensations = []
    
    for domain_idx in range(len(self.problems)):
        sc = StaticCondensationFactory.create(
            self.problems[domain_idx],
            self.global_discretization,
            self.elementary_matrices,
            domain_idx
        )
        # Build matrices immediately to cache them
        sc.build_matrices()
        self._static_condensations.append(sc)
        
return self._static_condensations
\end{lstlisting}

\textbf{Process:}
\begin{enumerate}
    \item Creates static condensation for each domain using factory
    \item Immediately builds matrices for caching
    \item Returns list of configured implementations
\end{enumerate}

\textbf{Usage:}
\begin{lstlisting}[language=Python, caption=Static Condensations Usage]
static_condensations = setup.static_condensations

# Access specific domain
sc_domain_0 = static_condensations[0]
matrices = sc_domain_0.build_matrices()  # Already built and cached

# Use in computations
for i, sc in enumerate(static_condensations):
    print(f"Domain {i}: {type(sc).__name__}")
\end{lstlisting}

\paragraph{global\_assembler}
\begin{lstlisting}[language=Python, caption=Global Assembler Property]
@property
def global_assembler(self) -> GlobalAssembler
\end{lstlisting}

\textbf{Returns:} \texttt{GlobalAssembler} - Global assembler instance (created once, cached)

\textbf{Algorithm:}
\begin{lstlisting}[language=Python, caption=Global Assembler Implementation]
if self._global_assembler is None:
    self._ensure_initialized()
    # Check the actual constructor signature and use correct parameters
    self._global_assembler = GlobalAssembler.from_framework_objects(
        self.problems,
        self.global_discretization,
        self.static_condensations,
        self.constraint_manager
    )
return self._global_assembler
\end{lstlisting}

\textbf{Creation:} Uses factory method \texttt{GlobalAssembler.from\_framework\_objects()}

\textbf{Usage:}
\begin{lstlisting}[language=Python, caption=Global Assembler Usage]
assembler = setup.global_assembler

print(f"Total DOFs: {assembler.total_dofs}")
print(f"Trace DOFs: {assembler.total_trace_dofs}")
print(f"Multipliers: {assembler.n_multipliers}")

# Use for system assembly
global_solution = np.random.rand(assembler.total_dofs)
residual, jacobian = assembler.assemble_residual_and_jacobian(
    global_solution, forcing_terms, setup.static_condensations, time=0.0
)
\end{lstlisting}

\paragraph{bulk\_data\_manager}
\begin{lstlisting}[language=Python, caption=Bulk Data Manager Property]
@property
def bulk_data_manager(self) -> BulkDataManager
\end{lstlisting}

\textbf{Returns:} \texttt{BulkDataManager} - Bulk data manager instance (created once, cached)

\textbf{Algorithm:}
\begin{lstlisting}[language=Python, caption=Bulk Data Manager Implementation]
if self._bulk_data_manager is None:
    self._ensure_initialized()
    discretizations = self.global_discretization.spatial_discretizations
    self._domain_data = BulkDataManager.extract_domain_data_list(
        self.problems, discretizations, self.static_condensations
    )
    self._bulk_data_manager = BulkDataManager(
        self._domain_data
    )
return self._bulk_data_manager
\end{lstlisting}

\textbf{Process:}
\begin{enumerate}
    \item Extracts domain data using \texttt{BulkDataManager.extract\_domain\_data\_list()}
    \item Creates lean BulkDataManager with extracted data
    \item Caches both domain data and manager
\end{enumerate}

\textbf{Usage:}
\begin{lstlisting}[language=Python, caption=Bulk Data Manager Usage]
bulk_manager = setup.bulk_data_manager

# Initialize bulk data
bulk_data_list = bulk_manager.initialize_all_bulk_data(
    setup.problems,
    setup.global_discretization.spatial_discretizations,
    time=0.0
)

# Compute forcing terms
forcing_terms = bulk_manager.compute_forcing_terms(
    bulk_data_list, setup.problems,
    setup.global_discretization.spatial_discretizations,
    time=0.1, dt=0.01
)
\end{lstlisting}

\subsubsection{Information and Analysis Methods}

\paragraph{get\_problem\_info()}
\begin{lstlisting}[language=Python, caption=Get Problem Info Method]
def get_problem_info(self) -> Dict[str, Any]
\end{lstlisting}

\textbf{Returns:} \texttt{Dict[str, Any]} - Comprehensive problem information dictionary

\textbf{Algorithm:}
\begin{lstlisting}[language=Python, caption=Problem Info Algorithm]
self._ensure_initialized()

info = {
    'problem_name': self.problem_name,
    'num_domains': len(self.problems),
    'total_elements': sum(disc.n_elements for disc in self.global_discretization.spatial_discretizations),
    'total_trace_dofs': self.global_assembler.total_trace_dofs,
    'num_constraints': self.constraint_manager.n_multipliers if self.constraint_manager else 0,
    'time_discretization': {
        'dt': self.global_discretization.dt,
        'T': self.global_discretization.T,
        'n_steps': self.global_discretization.n_time_steps
    },
    'domains': []
}

for i, (problem, discretization) in enumerate(zip(self.problems, self.global_discretization.spatial_discretizations)):
    domain_info = {
        'index': i,
        'type': problem.type,
        'domain': [problem.domain_start, problem.domain_end],
        'n_elements': discretization.n_elements,
        'n_equations': problem.neq,
        'trace_size': problem.neq * (discretization.n_elements + 1)
    }
    info['domains'].append(domain_info)

return info
\end{lstlisting}

\textbf{Information Structure:}
\begin{lstlisting}[language=Python, caption=Problem Info Structure]
{
    'problem_name': str,
    'num_domains': int,
    'total_elements': int,
    'total_trace_dofs': int,
    'num_constraints': int,
    'time_discretization': {
        'dt': float,
        'T': float,
        'n_steps': int
    },
    'domains': [
        {
            'index': int,
            'type': str,
            'domain': [float, float],  # [start, end]
            'n_elements': int,
            'n_equations': int,
            'trace_size': int
        },
        # ... for each domain
    ]
}
\end{lstlisting}

\textbf{Usage:}
\begin{lstlisting}[language=Python, caption=Problem Info Usage]
info = setup.get_problem_info()

print(f"Problem: {info['problem_name']}")
print(f"Domains: {info['num_domains']}")
print(f"Total DOFs: {info['total_trace_dofs']}")
print(f"Time stepping: dt={info['time_discretization']['dt']}, T={info['time_discretization']['T']}")

# Domain details
for domain_info in info['domains']:
    print(f"  Domain {domain_info['index']}: {domain_info['type']}")
    print(f"    Range: {domain_info['domain']}")
    print(f"    Elements: {domain_info['n_elements']}")
    print(f"    Equations: {domain_info['n_equations']}")
\end{lstlisting}

\subsubsection{Initial Condition and Solution Vector Management}

\paragraph{create\_initial\_conditions()}
\begin{lstlisting}[language=Python, caption=Create Initial Conditions Method]
def create_initial_conditions(self) -> Tuple[List[np.ndarray], np.ndarray]
\end{lstlisting}

\textbf{Returns:} \texttt{Tuple[List[np.ndarray], np.ndarray]} - (trace\_solutions, initial\_multipliers)

\textbf{Algorithm:}
\begin{lstlisting}[language=Python, caption=Initial Conditions Algorithm]
self._ensure_initialized()

trace_solutions = []

for i, (problem, discretization) in enumerate(zip(self.problems, self.global_discretization.spatial_discretizations)):
    n_nodes = discretization.n_elements + 1
    trace_size = problem.neq * n_nodes
    nodes = discretization.nodes
    
    trace_solution = np.zeros(trace_size)
    
    # Apply initial conditions if available
    for eq in range(problem.neq):
        for j in range(n_nodes):
            node_idx = eq * n_nodes + j
            if hasattr(problem, 'u0') and len(problem.u0) > eq and callable(problem.u0[eq]):
                trace_solution[node_idx] = problem.u0[eq](nodes[j])
            elif hasattr(problem, 'initial_conditions') and len(problem.initial_conditions) > eq:
                if callable(problem.initial_conditions[eq]):
                    trace_solution[node_idx] = problem.initial_conditions[eq](nodes[j])
    
    trace_solutions.append(trace_solution)

# Initialize multipliers to zero
n_multipliers = self.constraint_manager.n_multipliers if self.constraint_manager else 0
initial_multipliers = np.zeros(n_multipliers)

return trace_solutions, initial_multipliers
\end{lstlisting}

\textbf{Process:}
\begin{enumerate}
    \item For each domain: create trace solution array
    \item For each equation: evaluate initial condition at mesh nodes
    \item Initialize multipliers to zero
    \item Return domain traces and multiplier arrays
\end{enumerate}

\textbf{Initial Condition Access Patterns:}
\begin{itemize}
    \item \texttt{problem.u0[eq]} (primary pattern)
    \item \texttt{problem.initial\_conditions[eq]} (fallback)
    \item Zero initialization if no initial conditions found
\end{itemize}

\textbf{Usage:}
\begin{lstlisting}[language=Python, caption=Initial Conditions Usage]
trace_solutions, multipliers = setup.create_initial_conditions()

print(f"Created {len(trace_solutions)} domain trace solutions")
for i, trace_sol in enumerate(trace_solutions):
    print(f"  Domain {i}: shape {trace_sol.shape}")

print(f"Initial multipliers: {len(multipliers)} (all zero)")
\end{lstlisting}

\paragraph{create\_global\_solution\_vector()}
\begin{lstlisting}[language=Python, caption=Create Global Solution Vector Method]
def create_global_solution_vector(self, trace_solutions: List[np.ndarray], 
                                 multipliers: np.ndarray) -> np.ndarray
\end{lstlisting}

\textbf{Parameters:}
\begin{itemize}
    \item \texttt{trace\_solutions}: List of trace solution arrays for each domain
    \item \texttt{multipliers}: Array of Lagrange multiplier values
\end{itemize}

\textbf{Returns:} \texttt{np.ndarray} - Global solution vector

\textbf{Algorithm:}
\begin{lstlisting}[language=Python, caption=Global Vector Assembly Algorithm]
global_assembler = self.global_assembler

# Calculate total size
total_trace_size = sum(len(trace) for trace in trace_solutions)
total_size = total_trace_size + len(multipliers)

# Create global solution vector
global_solution = np.zeros(total_size)

# Fill trace solutions
offset = 0
for trace in trace_solutions:
    trace_flat = trace.flatten() if trace.ndim > 1 else trace
    global_solution[offset:offset+len(trace_flat)] = trace_flat
    offset += len(trace_flat)

# Fill multipliers
if len(multipliers) > 0:
    global_solution[offset:offset+len(multipliers)] = multipliers

return global_solution
\end{lstlisting}

\textbf{Assembly Structure:} \texttt{[trace\_domain\_0, trace\_domain\_1, ..., multipliers]}

\textbf{Usage:}
\begin{lstlisting}[language=Python, caption=Global Vector Assembly Usage]
# Create initial conditions
trace_solutions, multipliers = setup.create_initial_conditions()

# Assemble global vector
global_solution = setup.create_global_solution_vector(trace_solutions, multipliers)

print(f"Global solution vector shape: {global_solution.shape}")
print(f"Total DOFs: {len(global_solution)}")

# Verify size matches assembler
expected_size = setup.global_assembler.total_dofs
print(f"Matches assembler DOFs: {len(global_solution) == expected_size}")
\end{lstlisting}

\paragraph{extract\_domain\_solutions()}
\begin{lstlisting}[language=Python, caption=Extract Domain Solutions Method]
def extract_domain_solutions(self, global_solution: np.ndarray) -> Tuple[List[np.ndarray], np.ndarray]
\end{lstlisting}

\textbf{Parameters:}
\begin{itemize}
    \item \texttt{global\_solution}: Global solution vector
\end{itemize}

\textbf{Returns:} \texttt{Tuple[List[np.ndarray], np.ndarray]} - (trace\_solutions, multipliers)

\textbf{Algorithm:}
\begin{lstlisting}[language=Python, caption=Domain Solution Extraction Algorithm]
self._ensure_initialized()

trace_solutions = []
offset = 0

# Extract trace solutions for each domain
for i, (problem, discretization) in enumerate(zip(self.problems, self.global_discretization.spatial_discretizations)):
    n_nodes = discretization.n_elements + 1
    trace_size = problem.neq * n_nodes
    
    trace_solution = global_solution[offset:offset+trace_size]
    trace_solutions.append(trace_solution)
    offset += trace_size

# Extract multipliers
n_multipliers = self.constraint_manager.n_multipliers if self.constraint_manager else 0
if n_multipliers > 0:
    multipliers = global_solution[offset:offset+n_multipliers]
else:
    multipliers = np.array([])

return trace_solutions, multipliers
\end{lstlisting}

\textbf{Purpose:} Inverse operation of \texttt{create\_global\_solution\_vector()}

\textbf{Usage:}
\begin{lstlisting}[language=Python, caption=Domain Solution Extraction Usage]
# Extract from global solution
extracted_traces, extracted_multipliers = setup.extract_domain_solutions(global_solution)

# Verify round-trip consistency
for i, (orig, extracted) in enumerate(zip(trace_solutions, extracted_traces)):
    consistent = np.allclose(orig, extracted)
    print(f"Domain {i} round-trip consistent: {consistent}")

multiplier_consistent = np.allclose(multipliers, extracted_multipliers)
print(f"Multiplier round-trip consistent: {multiplier_consistent}")
\end{lstlisting}

\subsubsection{Validation and Testing}

\paragraph{validate\_setup()}
\begin{lstlisting}[language=Python, caption=Validate Setup Method]
def validate_setup(self, verbose: bool = False) -> bool
\end{lstlisting}

\textbf{Parameters:}
\begin{itemize}
    \item \texttt{verbose}: If True, print detailed validation information (default: False)
\end{itemize}

\textbf{Returns:} \texttt{bool} - True if all validation tests pass, False otherwise

\textbf{Validation Tests:}
\begin{enumerate}
    \item \textbf{Initial Condition Creation}: Tests creation of trace solutions and multipliers
    \item \textbf{Global Vector Round-Trip}: Tests assembly and extraction consistency
    \item \textbf{Multiplier Round-Trip}: Validates multiplier handling
    \item \textbf{Bulk Solution Creation}: Tests BulkData creation for all domains
    \item \textbf{Forcing Term Computation}: Validates forcing term calculation
    \item \textbf{Residual/Jacobian Assembly}: Tests global system assembly
\end{enumerate}

\textbf{Algorithm:}
\begin{lstlisting}[language=Python, caption=Validate Setup Algorithm]
try:
    self._ensure_initialized()
    
    if verbose:
        print(f"Validating setup for problem: {self.problem_name}")
        print(f"Number of domains: {len(self.problems)}")
        print(f"Total DOFs: {self.global_assembler.total_dofs}")
    
    # Test initial conditions
    trace_solutions, multipliers = self.create_initial_conditions()
    if verbose:
        print(f"✓ Initial conditions created")
    
    # Test global vector assembly/extraction
    global_solution = self.create_global_solution_vector(trace_solutions, multipliers)
    extracted_traces, extracted_multipliers = self.extract_domain_solutions(global_solution)
    
    # Verify round-trip consistency
    for i, (orig, extracted) in enumerate(zip(trace_solutions, extracted_traces)):
        if not np.allclose(orig, extracted):
            if verbose:
                print(f"✗ Round-trip test failed for domain {i}")
            return False
    
    if not np.allclose(multipliers, extracted_multipliers):
        if verbose:
            print(f"✗ Round-trip test failed for multipliers")
        return False
    
    if verbose:
        print(f"✓ Global vector round-trip test passed")
    
    # Test bulk data manager
    bulk_solutions = []
    for i in range(len(self.problems)):
        problem = self.problems[i]
        discretization = self.global_discretization.spatial_discretizations[i]
        bulk_sol = self.bulk_data_manager.create_bulk_data(i, problem, discretization)
        bulk_solutions.append(bulk_sol)
    
    if verbose:
        print(f"✓ Bulk solutions created")
    
    # Test forcing term computation
    forcing_terms = self.bulk_data_manager.compute_forcing_terms(bulk_solutions, 
                                                                 self.problems, 
                                                                 self.global_discretization.spatial_discretizations, 
                                                                 0.0, 
                                                                 self.global_discretization.dt
                                                                 )

    if verbose:
        print(f"✓ Forcing terms computed")
    
    # Test residual/jacobian computation
    global_residual, global_jacobian = self.global_assembler.assemble_residual_and_jacobian(
        global_solution=global_solution,
        forcing_terms=forcing_terms,
        static_condensations=self._static_condensations,
        time=0.0
    )
    
    if verbose:
        print(f"✓ Global residual and Jacobian assembled")
        print(f"  Residual norm: {np.linalg.norm(global_residual):.6e}")
        print(f"  Jacobian condition number: {np.linalg.cond(global_jacobian):.2e}")
    
    return True
    
except Exception as e:
    if verbose:
        print(f"✗ Validation failed: {e}")
        import traceback
        traceback.print_exc()
    return False
\end{lstlisting}

\textbf{Usage:}
\begin{lstlisting}[language=Python, caption=Validation Usage]
# Quick validation
is_valid = setup.validate_setup()
if is_valid:
    print("✓ Setup validation passed")
else:
    print("✗ Setup validation failed")

# Detailed validation with output
is_valid = setup.validate_setup(verbose=True)
\end{lstlisting}

\textbf{Sample Validation Output:}
\begin{lstlisting}[language=Python, caption=Sample Validation Output]
Validating setup for problem: OrganOnChip Test Problem
Number of domains: 1
Total DOFs: 86
✓ Initial conditions created
✓ Global vector round-trip test passed
✓ Bulk solutions created
✓ Forcing terms computed
✓ Global residual and Jacobian assembled
  Residual norm: 1.234567e-02
  Jacobian condition number: 2.34e+03
\end{lstlisting}

\subsection{Factory Functions}
\label{subsec:factory_functions_detailed}

\paragraph{create\_solver\_setup()}
\begin{lstlisting}[language=Python, caption=Create Solver Setup Function]
def create_solver_setup(problem_module: str = "ooc1d.problems.test_problem2") -> SolverSetup
\end{lstlisting}

\textbf{Parameters:}
\begin{itemize}
    \item \texttt{problem\_module}: String path to problem module (default: "ooc1d.problems.test\_problem2")
\end{itemize}

\textbf{Returns:} \texttt{SolverSetup} - Initialized SolverSetup instance

\textbf{Algorithm:}
\begin{lstlisting}[language=Python, caption=Factory Function Implementation]
setup = SolverSetup(problem_module)
setup.initialize()
return setup
\end{lstlisting}

\textbf{Usage:}
\begin{lstlisting}[language=Python, caption=Factory Function Usage]
# Create and initialize setup
setup = create_solver_setup("ooc1d.problems.ooc_test_problem")

# Immediate access to components
assembler = setup.global_assembler
bulk_manager = setup.bulk_data_manager
\end{lstlisting}

\paragraph{quick\_setup()}
\begin{lstlisting}[language=Python, caption=Quick Setup Function]
def quick_setup(problem_module: str = "ooc1d.problems.test_problem2", 
               validate: bool = True) -> SolverSetup
\end{lstlisting}

\textbf{Parameters:}
\begin{itemize}
    \item \texttt{problem\_module}: String path to problem module (default: "ooc1d.problems.test\_problem2")
    \item \texttt{validate}: If True, run validation tests (default: True)
\end{itemize}

\textbf{Returns:} \texttt{SolverSetup} - Validated SolverSetup instance

\textbf{Raises:} \texttt{RuntimeError} if validation fails

\textbf{Algorithm:}
\begin{lstlisting}[language=Python, caption=Quick Setup Implementation]
setup = create_solver_setup(problem_module)
if validate:
    if not setup.validate_setup(verbose=True):
        raise RuntimeError("Setup validation failed")

return setup
\end{lstlisting}

\textbf{Usage:}
\begin{lstlisting}[language=Python, caption=Quick Setup Usage]
# Quick setup with validation
setup = quick_setup("ooc1d.problems.ooc_test_problem", validate=True)

# Quick setup without validation (faster)
setup = quick_setup("ooc1d.problems.test_problem", validate=False)
\end{lstlisting}

\subsection{Complete Usage Examples}
\label{subsec:complete_usage_examples_detailed}

\subsubsection{Complete Solver Setup and Newton Iteration}

\begin{lstlisting}[language=Python, caption=Complete Solver Setup Example]
from setup_solver import quick_setup, SolverSetup
import numpy as np

def complete_solver_example():
    """Complete example of solver setup and Newton iteration."""
    
    # Step 1: Initialize solver with validation
    print("Setting up solver...")
    setup = quick_setup("ooc1d.problems.ooc_test_problem", validate=True)
    
    # Step 2: Get problem information
    info = setup.get_problem_info()
    print(f"\nProblem: {info['problem_name']}")
    print(f"Domains: {info['num_domains']}")
    print(f"Total DOFs: {info['total_trace_dofs'] + info['num_constraints']}")
    
    # Step 3: Create initial conditions
    trace_solutions, multipliers = setup.create_initial_conditions()
    global_solution = setup.create_global_solution_vector(trace_solutions, multipliers)
    
    print(f"\nInitial conditions created:")
    print(f"  Global solution shape: {global_solution.shape}")
    print(f"  Initial residual norm: {np.linalg.norm(global_solution):.6e}")
    
    # Step 4: Setup Newton iteration
    assembler = setup.global_assembler
    bulk_manager = setup.bulk_data_manager
    static_condensations = setup.static_condensations
    
    # Step 5: Newton iteration loop
    tolerance = 1e-10
    max_iterations = 20
    current_time = 0.0
    dt = setup.global_discretization.dt
    
    print(f"\nStarting Newton iterations (tol={tolerance:.0e})...")
    
    for iteration in range(max_iterations):
        # Create bulk data for forcing terms
        bulk_data_list = bulk_manager.initialize_all_bulk_data(
            setup.problems,
            setup.global_discretization.spatial_discretizations,
            time=current_time
        )
        
        # Compute forcing terms
        forcing_terms = bulk_manager.compute_forcing_terms(
            bulk_data_list, setup.problems,
            setup.global_discretization.spatial_discretizations,
            current_time, dt
        )
        
        # Assemble system
        residual, jacobian = assembler.assemble_residual_and_jacobian(
            global_solution=global_solution,
            forcing_terms=forcing_terms,
            static_condensations=static_condensations,
            time=current_time
        )
        
        # Check convergence
        residual_norm = np.linalg.norm(residual)
        print(f"  Iteration {iteration}: ||R|| = {residual_norm:.6e}")
        
        if residual_norm < tolerance:
            print("  ✓ Newton solver converged")
            break
        
        # Newton update
        try:
            delta = np.linalg.solve(jacobian, -residual)
            global_solution += delta
        except np.linalg.LinAlgError:
            print("  ✗ Newton solver failed: singular Jacobian")
            break
    
    # Step 6: Extract final solutions
    final_traces, final_multipliers = setup.extract_domain_solutions(global_solution)
    
    print(f"\nSolver completed:")
    print(f"  Final residual norm: {residual_norm:.6e}")
    print(f"  Domain solutions: {len(final_traces)}")
    print(f"  Constraint multipliers: {len(final_multipliers)}")
    
    return setup, global_solution, final_traces, final_multipliers

# Usage
setup, solution, traces, multipliers = complete_solver_example()
\end{lstlisting}

\subsection{Method Summary Table}
\label{subsec:method_summary_detailed}

\begin{longtable}{|p{4cm}|p{2.5cm}|p{6.5cm}|}
\hline
\textbf{Method/Property} & \textbf{Returns} & \textbf{Purpose} \\
\hline
\endhead

\texttt{\_\_init\_\_} & \texttt{None} & Initialize setup with problem module path \\
\hline

\texttt{initialize} & \texttt{None} & Load and initialize problem configuration \\
\hline

\texttt{elementary\_matrices} & \texttt{ElementaryMatrices} & Get cached elementary matrices \\
\hline

\texttt{static\_condensations} & \texttt{List} & Get cached static condensation implementations \\
\hline

\texttt{global\_assembler} & \texttt{GlobalAssembler} & Get cached global assembler \\
\hline

\texttt{bulk\_data\_manager} & \texttt{BulkDataManager} & Get cached bulk data manager \\
\hline

\texttt{get\_problem\_info} & \texttt{Dict} & Get comprehensive problem information \\
\hline

\texttt{create\_initial\_conditions} & \texttt{Tuple} & Create initial trace solutions and multipliers \\
\hline

\texttt{create\_global\_solution\_vector} & \texttt{np.ndarray} & Assemble global solution from components \\
\hline

\texttt{extract\_domain\_solutions} & \texttt{Tuple} & Extract components from global solution \\
\hline

\texttt{validate\_setup} & \texttt{bool} & Run comprehensive validation tests \\
\hline

\texttt{create\_solver\_setup} & \texttt{SolverSetup} & Factory function for setup creation \\
\hline

\texttt{quick\_setup} & \texttt{SolverSetup} & Factory with optional validation \\
\hline

\end{longtable}

\subsection{Key Features and Design Principles}

\begin{itemize}
    \item \textbf{Lean Architecture}: Minimal data storage with on-demand component creation
    \item \textbf{Component Caching}: Expensive objects created once and cached
    \item \textbf{Dynamic Loading}: Problem modules loaded at runtime via importlib
    \item \textbf{Lazy Initialization}: Components created only when accessed
    \item \textbf{Comprehensive Validation}: Built-in testing of all components and round-trip consistency
    \item \textbf{Clean Interfaces}: Simple property-based access to complex components
    \item \textbf{Factory Patterns}: Convenience functions for common setup scenarios
    \item \textbf{Backward Compatibility}: Aliases and fallbacks for existing code
    \item \textbf{Error Handling}: Robust error reporting and fallback strategies
    \item \textbf{Flexible Architecture}: Support for arbitrary problem modules and configurations
\end{itemize}

This documentation provides an exact reference for the setup solver module, emphasizing its role as the central orchestrator for BioNetFlux solver initialization and its lean, efficient approach to component management.

% End of setup solver module detailed API documentation


\section{Setup Solver Module Detailed API Reference}
\label{sec:setup_solver_detailed_api}

This section provides an exact reference for the setup solver module (\texttt{setup\_solver.py}) based on detailed analysis of the actual implementation. The module provides lean solver orchestration for OOC1D problems on networks, minimizing data redundancy while providing clean interfaces for different problem types.

\subsection{Module Overview}

The setup solver module provides:
\begin{itemize}
    \item Lean solver setup orchestration with minimal data storage
    \item Dynamic problem module loading and initialization
    \item On-demand component creation with caching
    \item Clean interfaces for accessing all solver components
    \item Comprehensive validation and testing capabilities
    \item Factory functions for common setup patterns
\end{itemize}

\subsection{Module Imports and Dependencies}

\begin{lstlisting}[language=Python, caption=Module Dependencies]
import numpy as np
from typing import List, Tuple, Optional, Dict, Any
import importlib

# Core imports
from ooc1d.core.discretization import Discretization, GlobalDiscretization
from ooc1d.utils.elementary_matrices import ElementaryMatrices
from ooc1d.core.static_condensation_factory import StaticCondensationFactory
from ooc1d.core.constraints import ConstraintManager
from ooc1d.core.lean_global_assembly import GlobalAssembler
from ooc1d.core.lean_bulk_data_manager import BulkDataManager
\end{lstlisting}

\subsection{SolverSetup Class}
\label{subsec:solver_setup_class_detailed}

Main class that orchestrates initialization of all solver components with lean data storage.

\subsubsection{Constructor}

\paragraph{\_\_init\_\_()}
\begin{lstlisting}[language=Python, caption=SolverSetup Constructor]
def __init__(self, problem_module: str = "ooc1d.problems.test_problem2")
\end{lstlisting}

\textbf{Parameters:}
\begin{itemize}
    \item \texttt{problem\_module}: String path to problem module containing \texttt{create\_global\_framework} (default: "ooc1d.problems.test\_problem2")
\end{itemize}

\textbf{Side Effects:} 
\begin{itemize}
    \item Sets \texttt{self.problem\_module} to specified module path
    \item Initializes \texttt{self.\_initialized} to False
    \item Sets all framework object attributes to None
    \item Sets all cached component attributes to None
\end{itemize}

\textbf{Usage:}
\begin{lstlisting}[language=Python, caption=Constructor Usage Examples]
# Default problem module
setup = SolverSetup()

# Custom problem module
setup = SolverSetup("ooc1d.problems.ooc_test_problem")

# OrganOnChip problem
setup = SolverSetup("ooc1d.problems.organ_on_chip_example")
\end{lstlisting}

\subsubsection{Core Attributes}

\begin{longtable}{|p{3.5cm}|p{2.5cm}|p{7cm}|}
\hline
\textbf{Attribute} & \textbf{Type} & \textbf{Description} \\
\hline
\endhead

\texttt{problem\_module} & \texttt{str} & String path to problem module \\
\hline

\texttt{\_initialized} & \texttt{bool} & Flag tracking initialization state \\
\hline

\texttt{problems} & \texttt{Optional[List]} & List of Problem instances (loaded on initialization) \\
\hline

\texttt{global\_discretization} & \texttt{Optional[GlobalDiscretization]} & Global discretization instance \\
\hline

\texttt{constraint\_manager} & \texttt{Optional[ConstraintManager]} & Constraint manager for boundary/junction conditions \\
\hline

\texttt{problem\_name} & \texttt{Optional[str]} & Descriptive name of the problem \\
\hline

\texttt{constraints} & \texttt{Optional[ConstraintManager]} & Alias for \texttt{constraint\_manager} (backward compatibility) \\
\hline

\end{longtable}

\subsubsection{Cached Component Attributes (Private)}

\begin{longtable}{|p{3.5cm}|p{2.5cm}|p{7cm}|}
\hline
\textbf{Attribute} & \textbf{Type} & \textbf{Description} \\
\hline
\endhead

\texttt{\_elementary\_matrices} & \texttt{Optional[ElementaryMatrices]} & Cached elementary matrices instance \\
\hline

\texttt{\_static\_condensations} & \texttt{Optional[List]} & Cached static condensation implementations \\
\hline

\texttt{\_global\_assembler} & \texttt{Optional[GlobalAssembler]} & Cached global assembler instance \\
\hline

\texttt{\_bulk\_data\_manager} & \texttt{Optional[BulkDataManager]} & Cached bulk data manager instance \\
\hline

\texttt{\_domain\_data} & \texttt{Optional[List]} & Cached extracted domain data \\
\hline

\end{longtable}

\subsubsection{Initialization Methods}

\paragraph{initialize()}
\begin{lstlisting}[language=Python, caption=Initialize Method]
def initialize(self) -> None
\end{lstlisting}

\textbf{Returns:} \texttt{None}

\textbf{Algorithm:}
\begin{enumerate}
    \item Check if already initialized (early return if True)
    \item Import problem module using \texttt{importlib.import\_module()}
    \item Get \texttt{create\_global\_framework} function from module
    \item Call function to get: problems, global\_discretization, constraint\_manager, problem\_name
    \item Set \texttt{self.constraints} as alias to \texttt{constraint\_manager}
    \item Set \texttt{self.\_initialized} to True
\end{enumerate}

\textbf{Side Effects:}
\begin{itemize}
    \item Imports specified problem module
    \item Calls \texttt{create\_global\_framework()} from module
    \item Sets core problem attributes
    \item Sets \texttt{\_initialized} flag to True
\end{itemize}

\textbf{Raises:} \texttt{ImportError} if problem module or function not found

\textbf{Usage:}
\begin{lstlisting}[language=Python, caption=Initialize Usage]
setup = SolverSetup("ooc1d.problems.ooc_test_problem")
setup.initialize()  # Explicit initialization

# Or use properties that auto-initialize
print(setup.problem_name)  # Triggers initialization if needed
\end{lstlisting}

\paragraph{\_ensure\_initialized()}
\begin{lstlisting}[language=Python, caption=Ensure Initialized Method]
def _ensure_initialized(self) -> None
\end{lstlisting}

\textbf{Purpose:} Internal method ensuring initialization has occurred

\textbf{Algorithm:}
\begin{lstlisting}[language=Python, caption=Ensure Initialized Implementation]
if not self._initialized:
    self.initialize()
\end{lstlisting}

\textbf{Usage:} Called automatically by property accessors

\subsubsection{Component Properties (Cached)}

\paragraph{elementary\_matrices}
\begin{lstlisting}[language=Python, caption=Elementary Matrices Property]
@property
def elementary_matrices(self) -> ElementaryMatrices
\end{lstlisting}

\textbf{Returns:} \texttt{ElementaryMatrices} - Elementary matrices instance (created once, cached)

\textbf{Algorithm:}
\begin{lstlisting}[language=Python, caption=Elementary Matrices Property Implementation]
if self._elementary_matrices is None:
    self._elementary_matrices = ElementaryMatrices(orthonormal_basis=False)
return self._elementary_matrices
\end{lstlisting}

\textbf{Configuration:} Uses Lagrange basis (\texttt{orthonormal\_basis=False})

\textbf{Usage:}
\begin{lstlisting}[language=Python, caption=Elementary Matrices Usage]
setup = SolverSetup()
elem_matrices = setup.elementary_matrices

# Get specific matrices
M = elem_matrices.get_matrix('M')        # Mass matrix
T = elem_matrices.get_matrix('T')        # Trace matrix
D = elem_matrices.get_matrix('D')        # Derivative matrix
QUAD = elem_matrices.get_matrix('QUAD')  # Quadrature matrix
\end{lstlisting}

\paragraph{static\_condensations}
\begin{lstlisting}[language=Python, caption=Static Condensations Property]
@property
def static_condensations(self) -> List
\end{lstlisting}

\textbf{Returns:} \texttt{List} - Static condensation implementations for all domains (created once, cached)

\textbf{Algorithm:}
\begin{lstlisting}[language=Python, caption=Static Condensations Implementation]
if self._static_condensations is None:
    self._ensure_initialized()
    self._static_condensations = []
    
    for domain_idx in range(len(self.problems)):
        sc = StaticCondensationFactory.create(
            self.problems[domain_idx],
            self.global_discretization,
            self.elementary_matrices,
            domain_idx
        )
        # Build matrices immediately to cache them
        sc.build_matrices()
        self._static_condensations.append(sc)
        
return self._static_condensations
\end{lstlisting}

\textbf{Process:}
\begin{enumerate}
    \item Creates static condensation for each domain using factory
    \item Immediately builds matrices for caching
    \item Returns list of configured implementations
\end{enumerate}

\textbf{Usage:}
\begin{lstlisting}[language=Python, caption=Static Condensations Usage]
static_condensations = setup.static_condensations

# Access specific domain
sc_domain_0 = static_condensations[0]
matrices = sc_domain_0.build_matrices()  # Already built and cached

# Use in computations
for i, sc in enumerate(static_condensations):
    print(f"Domain {i}: {type(sc).__name__}")
\end{lstlisting}

\paragraph{global\_assembler}
\begin{lstlisting}[language=Python, caption=Global Assembler Property]
@property
def global_assembler(self) -> GlobalAssembler
\end{lstlisting}

\textbf{Returns:} \texttt{GlobalAssembler} - Global assembler instance (created once, cached)

\textbf{Algorithm:}
\begin{lstlisting}[language=Python, caption=Global Assembler Implementation]
if self._global_assembler is None:
    self._ensure_initialized()
    # Check the actual constructor signature and use correct parameters
    self._global_assembler = GlobalAssembler.from_framework_objects(
        self.problems,
        self.global_discretization,
        self.static_condensations,
        self.constraint_manager
    )
return self._global_assembler
\end{lstlisting}

\textbf{Creation:} Uses factory method \texttt{GlobalAssembler.from\_framework\_objects()}

\textbf{Usage:}
\begin{lstlisting}[language=Python, caption=Global Assembler Usage]
assembler = setup.global_assembler

print(f"Total DOFs: {assembler.total_dofs}")
print(f"Trace DOFs: {assembler.total_trace_dofs}")
print(f"Multipliers: {assembler.n_multipliers}")

# Use for system assembly
global_solution = np.random.rand(assembler.total_dofs)
residual, jacobian = assembler.assemble_residual_and_jacobian(
    global_solution, forcing_terms, setup.static_condensations, time=0.0
)
\end{lstlisting}

\paragraph{bulk\_data\_manager}
\begin{lstlisting}[language=Python, caption=Bulk Data Manager Property]
@property
def bulk_data_manager(self) -> BulkDataManager
\end{lstlisting}

\textbf{Returns:} \texttt{BulkDataManager} - Bulk data manager instance (created once, cached)

\textbf{Algorithm:}
\begin{lstlisting}[language=Python, caption=Bulk Data Manager Implementation]
if self._bulk_data_manager is None:
    self._ensure_initialized()
    discretizations = self.global_discretization.spatial_discretizations
    self._domain_data = BulkDataManager.extract_domain_data_list(
        self.problems, discretizations, self.static_condensations
    )
    self._bulk_data_manager = BulkDataManager(
        self._domain_data
    )
return self._bulk_data_manager
\end{lstlisting}

\textbf{Process:}
\begin{enumerate}
    \item Extracts domain data using \texttt{BulkDataManager.extract\_domain\_data\_list()}
    \item Creates lean BulkDataManager with extracted data
    \item Caches both domain data and manager
\end{enumerate}

\textbf{Usage:}
\begin{lstlisting}[language=Python, caption=Bulk Data Manager Usage]
bulk_manager = setup.bulk_data_manager

# Initialize bulk data
bulk_data_list = bulk_manager.initialize_all_bulk_data(
    setup.problems,
    setup.global_discretization.spatial_discretizations,
    time=0.0
)

# Compute forcing terms
forcing_terms = bulk_manager.compute_forcing_terms(
    bulk_data_list, setup.problems,
    setup.global_discretization.spatial_discretizations,
    time=0.1, dt=0.01
)
\end{lstlisting}

\subsubsection{Information and Analysis Methods}

\paragraph{get\_problem\_info()}
\begin{lstlisting}[language=Python, caption=Get Problem Info Method]
def get_problem_info(self) -> Dict[str, Any]
\end{lstlisting}

\textbf{Returns:} \texttt{Dict[str, Any]} - Comprehensive problem information dictionary

\textbf{Algorithm:}
\begin{lstlisting}[language=Python, caption=Problem Info Algorithm]
self._ensure_initialized()

info = {
    'problem_name': self.problem_name,
    'num_domains': len(self.problems),
    'total_elements': sum(disc.n_elements for disc in self.global_discretization.spatial_discretizations),
    'total_trace_dofs': self.global_assembler.total_trace_dofs,
    'num_constraints': self.constraint_manager.n_multipliers if self.constraint_manager else 0,
    'time_discretization': {
        'dt': self.global_discretization.dt,
        'T': self.global_discretization.T,
        'n_steps': self.global_discretization.n_time_steps
    },
    'domains': []
}

for i, (problem, discretization) in enumerate(zip(self.problems, self.global_discretization.spatial_discretizations)):
    domain_info = {
        'index': i,
        'type': problem.type,
        'domain': [problem.domain_start, problem.domain_end],
        'n_elements': discretization.n_elements,
        'n_equations': problem.neq,
        'trace_size': problem.neq * (discretization.n_elements + 1)
    }
    info['domains'].append(domain_info)

return info
\end{lstlisting}

\textbf{Information Structure:}
\begin{lstlisting}[language=Python, caption=Problem Info Structure]
{
    'problem_name': str,
    'num_domains': int,
    'total_elements': int,
    'total_trace_dofs': int,
    'num_constraints': int,
    'time_discretization': {
        'dt': float,
        'T': float,
        'n_steps': int
    },
    'domains': [
        {
            'index': int,
            'type': str,
            'domain': [float, float],  # [start, end]
            'n_elements': int,
            'n_equations': int,
            'trace_size': int
        },
        # ... for each domain
    ]
}
\end{lstlisting}

\textbf{Usage:}
\begin{lstlisting}[language=Python, caption=Problem Info Usage]
info = setup.get_problem_info()

print(f"Problem: {info['problem_name']}")
print(f"Domains: {info['num_domains']}")
print(f"Total DOFs: {info['total_trace_dofs']}")
print(f"Time stepping: dt={info['time_discretization']['dt']}, T={info['time_discretization']['T']}")

# Domain details
for domain_info in info['domains']:
    print(f"  Domain {domain_info['index']}: {domain_info['type']}")
    print(f"    Range: {domain_info['domain']}")
    print(f"    Elements: {domain_info['n_elements']}")
    print(f"    Equations: {domain_info['n_equations']}")
\end{lstlisting}

\subsubsection{Initial Condition and Solution Vector Management}

\paragraph{create\_initial\_conditions()}
\begin{lstlisting}[language=Python, caption=Create Initial Conditions Method]
def create_initial_conditions(self) -> Tuple[List[np.ndarray], np.ndarray]
\end{lstlisting}

\textbf{Returns:} \texttt{Tuple[List[np.ndarray], np.ndarray]} - (trace\_solutions, initial\_multipliers)

\textbf{Algorithm:}
\begin{lstlisting}[language=Python, caption=Initial Conditions Algorithm]
self._ensure_initialized()

trace_solutions = []

for i, (problem, discretization) in enumerate(zip(self.problems, self.global_discretization.spatial_discretizations)):
    n_nodes = discretization.n_elements + 1
    trace_size = problem.neq * n_nodes
    nodes = discretization.nodes
    
    trace_solution = np.zeros(trace_size)
    
    # Apply initial conditions if available
    for eq in range(problem.neq):
        for j in range(n_nodes):
            node_idx = eq * n_nodes + j
            if hasattr(problem, 'u0') and len(problem.u0) > eq and callable(problem.u0[eq]):
                trace_solution[node_idx] = problem.u0[eq](nodes[j])
            elif hasattr(problem, 'initial_conditions') and len(problem.initial_conditions) > eq:
                if callable(problem.initial_conditions[eq]):
                    trace_solution[node_idx] = problem.initial_conditions[eq](nodes[j])
    
    trace_solutions.append(trace_solution)

# Initialize multipliers to zero
n_multipliers = self.constraint_manager.n_multipliers if self.constraint_manager else 0
initial_multipliers = np.zeros(n_multipliers)

return trace_solutions, initial_multipliers
\end{lstlisting}

\textbf{Process:}
\begin{enumerate}
    \item For each domain: create trace solution array
    \item For each equation: evaluate initial condition at mesh nodes
    \item Initialize multipliers to zero
    \item Return domain traces and multiplier arrays
\end{enumerate}

\textbf{Initial Condition Access Patterns:}
\begin{itemize}
    \item \texttt{problem.u0[eq]} (primary pattern)
    \item \texttt{problem.initial\_conditions[eq]} (fallback)
    \item Zero initialization if no initial conditions found
\end{itemize}

\textbf{Usage:}
\begin{lstlisting}[language=Python, caption=Initial Conditions Usage]
trace_solutions, multipliers = setup.create_initial_conditions()

print(f"Created {len(trace_solutions)} domain trace solutions")
for i, trace_sol in enumerate(trace_solutions):
    print(f"  Domain {i}: shape {trace_sol.shape}")

print(f"Initial multipliers: {len(multipliers)} (all zero)")
\end{lstlisting}

\paragraph{create\_global\_solution\_vector()}
\begin{lstlisting}[language=Python, caption=Create Global Solution Vector Method]
def create_global_solution_vector(self, trace_solutions: List[np.ndarray], 
                                 multipliers: np.ndarray) -> np.ndarray
\end{lstlisting}

\textbf{Parameters:}
\begin{itemize}
    \item \texttt{trace\_solutions}: List of trace solution arrays for each domain
    \item \texttt{multipliers}: Array of Lagrange multiplier values
\end{itemize}

\textbf{Returns:} \texttt{np.ndarray} - Global solution vector

\textbf{Algorithm:}
\begin{lstlisting}[language=Python, caption=Global Vector Assembly Algorithm]
global_assembler = self.global_assembler

# Calculate total size
total_trace_size = sum(len(trace) for trace in trace_solutions)
total_size = total_trace_size + len(multipliers)

# Create global solution vector
global_solution = np.zeros(total_size)

# Fill trace solutions
offset = 0
for trace in trace_solutions:
    trace_flat = trace.flatten() if trace.ndim > 1 else trace
    global_solution[offset:offset+len(trace_flat)] = trace_flat
    offset += len(trace_flat)

# Fill multipliers
if len(multipliers) > 0:
    global_solution[offset:offset+len(multipliers)] = multipliers

return global_solution
\end{lstlisting}

\textbf{Assembly Structure:} \texttt{[trace\_domain\_0, trace\_domain\_1, ..., multipliers]}

\textbf{Usage:}
\begin{lstlisting}[language=Python, caption=Global Vector Assembly Usage]
# Create initial conditions
trace_solutions, multipliers = setup.create_initial_conditions()

# Assemble global vector
global_solution = setup.create_global_solution_vector(trace_solutions, multipliers)

print(f"Global solution vector shape: {global_solution.shape}")
print(f"Total DOFs: {len(global_solution)}")

# Verify size matches assembler
expected_size = setup.global_assembler.total_dofs
print(f"Matches assembler DOFs: {len(global_solution) == expected_size}")
\end{lstlisting}

\paragraph{extract\_domain\_solutions()}
\begin{lstlisting}[language=Python, caption=Extract Domain Solutions Method]
def extract_domain_solutions(self, global_solution: np.ndarray) -> Tuple[List[np.ndarray], np.ndarray]
\end{lstlisting}

\textbf{Parameters:}
\begin{itemize}
    \item \texttt{global\_solution}: Global solution vector
\end{itemize}

\textbf{Returns:} \texttt{Tuple[List[np.ndarray], np.ndarray]} - (trace\_solutions, multipliers)

\textbf{Algorithm:}
\begin{lstlisting}[language=Python, caption=Domain Solution Extraction Algorithm]
self._ensure_initialized()

trace_solutions = []
offset = 0

# Extract trace solutions for each domain
for i, (problem, discretization) in enumerate(zip(self.problems, self.global_discretization.spatial_discretizations)):
    n_nodes = discretization.n_elements + 1
    trace_size = problem.neq * n_nodes
    
    trace_solution = global_solution[offset:offset+trace_size]
    trace_solutions.append(trace_solution)
    offset += trace_size

# Extract multipliers
n_multipliers = self.constraint_manager.n_multipliers if self.constraint_manager else 0
if n_multipliers > 0:
    multipliers = global_solution[offset:offset+n_multipliers]
else:
    multipliers = np.array([])

return trace_solutions, multipliers
\end{lstlisting}

\textbf{Purpose:} Inverse operation of \texttt{create\_global\_solution\_vector()}

\textbf{Usage:}
\begin{lstlisting}[language=Python, caption=Domain Solution Extraction Usage]
# Extract from global solution
extracted_traces, extracted_multipliers = setup.extract_domain_solutions(global_solution)

# Verify round-trip consistency
for i, (orig, extracted) in enumerate(zip(trace_solutions, extracted_traces)):
    consistent = np.allclose(orig, extracted)
    print(f"Domain {i} round-trip consistent: {consistent}")

multiplier_consistent = np.allclose(multipliers, extracted_multipliers)
print(f"Multiplier round-trip consistent: {multiplier_consistent}")
\end{lstlisting}

\subsubsection{Validation and Testing}

\paragraph{validate\_setup()}
\begin{lstlisting}[language=Python, caption=Validate Setup Method]
def validate_setup(self, verbose: bool = False) -> bool
\end{lstlisting}

\textbf{Parameters:}
\begin{itemize}
    \item \texttt{verbose}: If True, print detailed validation information (default: False)
\end{itemize}

\textbf{Returns:} \texttt{bool} - True if all validation tests pass, False otherwise

\textbf{Validation Tests:}
\begin{enumerate}
    \item \textbf{Initial Condition Creation}: Tests creation of trace solutions and multipliers
    \item \textbf{Global Vector Round-Trip}: Tests assembly and extraction consistency
    \item \textbf{Multiplier Round-Trip}: Validates multiplier handling
    \item \textbf{Bulk Solution Creation}: Tests BulkData creation for all domains
    \item \textbf{Forcing Term Computation}: Validates forcing term calculation
    \item \textbf{Residual/Jacobian Assembly}: Tests global system assembly
\end{enumerate}

\textbf{Algorithm:}
\begin{lstlisting}[language=Python, caption=Validate Setup Algorithm]
try:
    self._ensure_initialized()
    
    if verbose:
        print(f"Validating setup for problem: {self.problem_name}")
        print(f"Number of domains: {len(self.problems)}")
        print(f"Total DOFs: {self.global_assembler.total_dofs}")
    
    # Test initial conditions
    trace_solutions, multipliers = self.create_initial_conditions()
    if verbose:
        print(f"✓ Initial conditions created")
    
    # Test global vector assembly/extraction
    global_solution = self.create_global_solution_vector(trace_solutions, multipliers)
    extracted_traces, extracted_multipliers = self.extract_domain_solutions(global_solution)
    
    # Verify round-trip consistency
    for i, (orig, extracted) in enumerate(zip(trace_solutions, extracted_traces)):
        if not np.allclose(orig, extracted):
            if verbose:
                print(f"✗ Round-trip test failed for domain {i}")
            return False
    
    if not np.allclose(multipliers, extracted_multipliers):
        if verbose:
            print(f"✗ Round-trip test failed for multipliers")
        return False
    
    if verbose:
        print(f"✓ Global vector round-trip test passed")
    
    # Test bulk data manager
    bulk_solutions = []
    for i in range(len(self.problems)):
        problem = self.problems[i]
        discretization = self.global_discretization.spatial_discretizations[i]
        bulk_sol = self.bulk_data_manager.create_bulk_data(i, problem, discretization)
        bulk_solutions.append(bulk_sol)
    
    if verbose:
        print(f"✓ Bulk solutions created")
    
    # Test forcing term computation
    forcing_terms = self.bulk_data_manager.compute_forcing_terms(bulk_solutions, 
                                                                 self.problems, 
                                                                 self.global_discretization.spatial_discretizations, 
                                                                 0.0, 
                                                                 self.global_discretization.dt
                                                                 )

    if verbose:
        print(f"✓ Forcing terms computed")
    
    # Test residual/jacobian computation
    global_residual, global_jacobian = self.global_assembler.assemble_residual_and_jacobian(
        global_solution=global_solution,
        forcing_terms=forcing_terms,
        static_condensations=self._static_condensations,
        time=0.0
    )
    
    if verbose:
        print(f"✓ Global residual and Jacobian assembled")
        print(f"  Residual norm: {np.linalg.norm(global_residual):.6e}")
        print(f"  Jacobian condition number: {np.linalg.cond(global_jacobian):.2e}")
    
    return True
    
except Exception as e:
    if verbose:
        print(f"✗ Validation failed: {e}")
        import traceback
        traceback.print_exc()
    return False
\end{lstlisting}

\textbf{Usage:}
\begin{lstlisting}[language=Python, caption=Validation Usage]
# Quick validation
is_valid = setup.validate_setup()
if is_valid:
    print("✓ Setup validation passed")
else:
    print("✗ Setup validation failed")

# Detailed validation with output
is_valid = setup.validate_setup(verbose=True)
\end{lstlisting}

\textbf{Sample Validation Output:}
\begin{lstlisting}[language=Python, caption=Sample Validation Output]
Validating setup for problem: OrganOnChip Test Problem
Number of domains: 1
Total DOFs: 86
✓ Initial conditions created
✓ Global vector round-trip test passed
✓ Bulk solutions created
✓ Forcing terms computed
✓ Global residual and Jacobian assembled
  Residual norm: 1.234567e-02
  Jacobian condition number: 2.34e+03
\end{lstlisting}

\subsection{Factory Functions}
\label{subsec:factory_functions_detailed}

\paragraph{create\_solver\_setup()}
\begin{lstlisting}[language=Python, caption=Create Solver Setup Function]
def create_solver_setup(problem_module: str = "ooc1d.problems.test_problem2") -> SolverSetup
\end{lstlisting}

\textbf{Parameters:}
\begin{itemize}
    \item \texttt{problem\_module}: String path to problem module (default: "ooc1d.problems.test\_problem2")
\end{itemize}

\textbf{Returns:} \texttt{SolverSetup} - Initialized SolverSetup instance

\textbf{Algorithm:}
\begin{lstlisting}[language=Python, caption=Factory Function Implementation]
setup = SolverSetup(problem_module)
setup.initialize()
return setup
\end{lstlisting}

\textbf{Usage:}
\begin{lstlisting}[language=Python, caption=Factory Function Usage]
# Create and initialize setup
setup = create_solver_setup("ooc1d.problems.ooc_test_problem")

# Immediate access to components
assembler = setup.global_assembler
bulk_manager = setup.bulk_data_manager
\end{lstlisting}

\paragraph{quick\_setup()}
\begin{lstlisting}[language=Python, caption=Quick Setup Function]
def quick_setup(problem_module: str = "ooc1d.problems.test_problem2", 
               validate: bool = True) -> SolverSetup
\end{lstlisting}

\textbf{Parameters:}
\begin{itemize}
    \item \texttt{problem\_module}: String path to problem module (default: "ooc1d.problems.test\_problem2")
    \item \texttt{validate}: If True, run validation tests (default: True)
\end{itemize}

\textbf{Returns:} \texttt{SolverSetup} - Validated SolverSetup instance

\textbf{Raises:} \texttt{RuntimeError} if validation fails

\textbf{Algorithm:}
\begin{lstlisting}[language=Python, caption=Quick Setup Implementation]
setup = create_solver_setup(problem_module)
if validate:
    if not setup.validate_setup(verbose=True):
        raise RuntimeError("Setup validation failed")

return setup
\end{lstlisting}

\textbf{Usage:}
\begin{lstlisting}[language=Python, caption=Quick Setup Usage]
# Quick setup with validation
setup = quick_setup("ooc1d.problems.ooc_test_problem", validate=True)

# Quick setup without validation (faster)
setup = quick_setup("ooc1d.problems.test_problem", validate=False)
\end{lstlisting}

\subsection{Complete Usage Examples}
\label{subsec:complete_usage_examples_detailed}

\subsubsection{Complete Solver Setup and Newton Iteration}

\begin{lstlisting}[language=Python, caption=Complete Solver Setup Example]
from setup_solver import quick_setup, SolverSetup
import numpy as np

def complete_solver_example():
    """Complete example of solver setup and Newton iteration."""
    
    # Step 1: Initialize solver with validation
    print("Setting up solver...")
    setup = quick_setup("ooc1d.problems.ooc_test_problem", validate=True)
    
    # Step 2: Get problem information
    info = setup.get_problem_info()
    print(f"\nProblem: {info['problem_name']}")
    print(f"Domains: {info['num_domains']}")
    print(f"Total DOFs: {info['total_trace_dofs'] + info['num_constraints']}")
    
    # Step 3: Create initial conditions
    trace_solutions, multipliers = setup.create_initial_conditions()
    global_solution = setup.create_global_solution_vector(trace_solutions, multipliers)
    
    print(f"\nInitial conditions created:")
    print(f"  Global solution shape: {global_solution.shape}")
    print(f"  Initial residual norm: {np.linalg.norm(global_solution):.6e}")
    
    # Step 4: Setup Newton iteration
    assembler = setup.global_assembler
    bulk_manager = setup.bulk_data_manager
    static_condensations = setup.static_condensations
    
    # Step 5: Newton iteration loop
    tolerance = 1e-10
    max_iterations = 20
    current_time = 0.0
    dt = setup.global_discretization.dt
    
    print(f"\nStarting Newton iterations (tol={tolerance:.0e})...")
    
    for iteration in range(max_iterations):
        # Create bulk data for forcing terms
        bulk_data_list = bulk_manager.initialize_all_bulk_data(
            setup.problems,
            setup.global_discretization.spatial_discretizations,
            time=current_time
        )
        
        # Compute forcing terms
        forcing_terms = bulk_manager.compute_forcing_terms(
            bulk_data_list, setup.problems,
            setup.global_discretization.spatial_discretizations,
            current_time, dt
        )
        
        # Assemble system
        residual, jacobian = assembler.assemble_residual_and_jacobian(
            global_solution=global_solution,
            forcing_terms=forcing_terms,
            static_condensations=static_condensations,
            time=current_time
        )
        
        # Check convergence
        residual_norm = np.linalg.norm(residual)
        print(f"  Iteration {iteration}: ||R|| = {residual_norm:.6e}")
        
        if residual_norm < tolerance:
            print("  ✓ Newton solver converged")
            break
        
        # Newton update
        try:
            delta = np.linalg.solve(jacobian, -residual)
            global_solution += delta
        except np.linalg.LinAlgError:
            print("  ✗ Newton solver failed: singular Jacobian")
            break
    
    # Step 6: Extract final solutions
    final_traces, final_multipliers = setup.extract_domain_solutions(global_solution)
    
    print(f"\nSolver completed:")
    print(f"  Final residual norm: {residual_norm:.6e}")
    print(f"  Domain solutions: {len(final_traces)}")
    print(f"  Constraint multipliers: {len(final_multipliers)}")
    
    return setup, global_solution, final_traces, final_multipliers

# Usage
setup, solution, traces, multipliers = complete_solver_example()
\end{lstlisting}

\subsection{Method Summary Table}
\label{subsec:method_summary_detailed}

\begin{longtable}{|p{4cm}|p{2.5cm}|p{6.5cm}|}
\hline
\textbf{Method/Property} & \textbf{Returns} & \textbf{Purpose} \\
\hline
\endhead

\texttt{\_\_init\_\_} & \texttt{None} & Initialize setup with problem module path \\
\hline

\texttt{initialize} & \texttt{None} & Load and initialize problem configuration \\
\hline

\texttt{elementary\_matrices} & \texttt{ElementaryMatrices} & Get cached elementary matrices \\
\hline

\texttt{static\_condensations} & \texttt{List} & Get cached static condensation implementations \\
\hline

\texttt{global\_assembler} & \texttt{GlobalAssembler} & Get cached global assembler \\
\hline

\texttt{bulk\_data\_manager} & \texttt{BulkDataManager} & Get cached bulk data manager \\
\hline

\texttt{get\_problem\_info} & \texttt{Dict} & Get comprehensive problem information \\
\hline

\texttt{create\_initial\_conditions} & \texttt{Tuple} & Create initial trace solutions and multipliers \\
\hline

\texttt{create\_global\_solution\_vector} & \texttt{np.ndarray} & Assemble global solution from components \\
\hline

\texttt{extract\_domain\_solutions} & \texttt{Tuple} & Extract components from global solution \\
\hline

\texttt{validate\_setup} & \texttt{bool} & Run comprehensive validation tests \\
\hline

\texttt{create\_solver\_setup} & \texttt{SolverSetup} & Factory function for setup creation \\
\hline

\texttt{quick\_setup} & \texttt{SolverSetup} & Factory with optional validation \\
\hline

\end{longtable}

\subsection{Key Features and Design Principles}

\begin{itemize}
    \item \textbf{Lean Architecture}: Minimal data storage with on-demand component creation
    \item \textbf{Component Caching}: Expensive objects created once and cached
    \item \textbf{Dynamic Loading}: Problem modules loaded at runtime via importlib
    \item \textbf{Lazy Initialization}: Components created only when accessed
    \item \textbf{Comprehensive Validation}: Built-in testing of all components and round-trip consistency
    \item \textbf{Clean Interfaces}: Simple property-based access to complex components
    \item \textbf{Factory Patterns}: Convenience functions for common setup scenarios
    \item \textbf{Backward Compatibility}: Aliases and fallbacks for existing code
    \item \textbf{Error Handling}: Robust error reporting and fallback strategies
    \item \textbf{Flexible Architecture}: Support for arbitrary problem modules and configurations
\end{itemize}

This documentation provides an exact reference for the setup solver module, emphasizing its role as the central orchestrator for BioNetFlux solver initialization and its lean, efficient approach to component management.

% End of setup solver module detailed API documentation


\section{Setup Solver Module Detailed API Reference}
\label{sec:setup_solver_detailed_api}

This section provides an exact reference for the setup solver module (\texttt{setup\_solver.py}) based on detailed analysis of the actual implementation. The module provides lean solver orchestration for OOC1D problems on networks, minimizing data redundancy while providing clean interfaces for different problem types.

\subsection{Module Overview}

The setup solver module provides:
\begin{itemize}
    \item Lean solver setup orchestration with minimal data storage
    \item Dynamic problem module loading and initialization
    \item On-demand component creation with caching
    \item Clean interfaces for accessing all solver components
    \item Comprehensive validation and testing capabilities
    \item Factory functions for common setup patterns
\end{itemize}

\subsection{Module Imports and Dependencies}

\begin{lstlisting}[language=Python, caption=Module Dependencies]
import numpy as np
from typing import List, Tuple, Optional, Dict, Any
import importlib

# Core imports
from ooc1d.core.discretization import Discretization, GlobalDiscretization
from ooc1d.utils.elementary_matrices import ElementaryMatrices
from ooc1d.core.static_condensation_factory import StaticCondensationFactory
from ooc1d.core.constraints import ConstraintManager
from ooc1d.core.lean_global_assembly import GlobalAssembler
from ooc1d.core.lean_bulk_data_manager import BulkDataManager
\end{lstlisting}

\subsection{SolverSetup Class}
\label{subsec:solver_setup_class_detailed}

Main class that orchestrates initialization of all solver components with lean data storage.

\subsubsection{Constructor}

\paragraph{\_\_init\_\_()}
\begin{lstlisting}[language=Python, caption=SolverSetup Constructor]
def __init__(self, problem_module: str = "ooc1d.problems.test_problem2")
\end{lstlisting}

\textbf{Parameters:}
\begin{itemize}
    \item \texttt{problem\_module}: String path to problem module containing \texttt{create\_global\_framework} (default: "ooc1d.problems.test\_problem2")
\end{itemize}

\textbf{Side Effects:} 
\begin{itemize}
    \item Sets \texttt{self.problem\_module} to specified module path
    \item Initializes \texttt{self.\_initialized} to False
    \item Sets all framework object attributes to None
    \item Sets all cached component attributes to None
\end{itemize}

\textbf{Usage:}
\begin{lstlisting}[language=Python, caption=Constructor Usage Examples]
# Default problem module
setup = SolverSetup()

# Custom problem module
setup = SolverSetup("ooc1d.problems.ooc_test_problem")

# OrganOnChip problem
setup = SolverSetup("ooc1d.problems.organ_on_chip_example")
\end{lstlisting}

\subsubsection{Core Attributes}

\begin{longtable}{|p{3.5cm}|p{2.5cm}|p{7cm}|}
\hline
\textbf{Attribute} & \textbf{Type} & \textbf{Description} \\
\hline
\endhead

\texttt{problem\_module} & \texttt{str} & String path to problem module \\
\hline

\texttt{\_initialized} & \texttt{bool} & Flag tracking initialization state \\
\hline

\texttt{problems} & \texttt{Optional[List]} & List of Problem instances (loaded on initialization) \\
\hline

\texttt{global\_discretization} & \texttt{Optional[GlobalDiscretization]} & Global discretization instance \\
\hline

\texttt{constraint\_manager} & \texttt{Optional[ConstraintManager]} & Constraint manager for boundary/junction conditions \\
\hline

\texttt{problem\_name} & \texttt{Optional[str]} & Descriptive name of the problem \\
\hline

\texttt{constraints} & \texttt{Optional[ConstraintManager]} & Alias for \texttt{constraint\_manager} (backward compatibility) \\
\hline

\end{longtable}

\subsubsection{Cached Component Attributes (Private)}

\begin{longtable}{|p{3.5cm}|p{2.5cm}|p{7cm}|}
\hline
\textbf{Attribute} & \textbf{Type} & \textbf{Description} \\
\hline
\endhead

\texttt{\_elementary\_matrices} & \texttt{Optional[ElementaryMatrices]} & Cached elementary matrices instance \\
\hline

\texttt{\_static\_condensations} & \texttt{Optional[List]} & Cached static condensation implementations \\
\hline

\texttt{\_global\_assembler} & \texttt{Optional[GlobalAssembler]} & Cached global assembler instance \\
\hline

\texttt{\_bulk\_data\_manager} & \texttt{Optional[BulkDataManager]} & Cached bulk data manager instance \\
\hline

\texttt{\_domain\_data} & \texttt{Optional[List]} & Cached extracted domain data \\
\hline

\end{longtable}

\subsubsection{Initialization Methods}

\paragraph{initialize()}
\begin{lstlisting}[language=Python, caption=Initialize Method]
def initialize(self) -> None
\end{lstlisting}

\textbf{Returns:} \texttt{None}

\textbf{Algorithm:}
\begin{enumerate}
    \item Check if already initialized (early return if True)
    \item Import problem module using \texttt{importlib.import\_module()}
    \item Get \texttt{create\_global\_framework} function from module
    \item Call function to get: problems, global\_discretization, constraint\_manager, problem\_name
    \item Set \texttt{self.constraints} as alias to \texttt{constraint\_manager}
    \item Set \texttt{self.\_initialized} to True
\end{enumerate}

\textbf{Side Effects:}
\begin{itemize}
    \item Imports specified problem module
    \item Calls \texttt{create\_global\_framework()} from module
    \item Sets core problem attributes
    \item Sets \texttt{\_initialized} flag to True
\end{itemize}

\textbf{Raises:} \texttt{ImportError} if problem module or function not found

\textbf{Usage:}
\begin{lstlisting}[language=Python, caption=Initialize Usage]
setup = SolverSetup("ooc1d.problems.ooc_test_problem")
setup.initialize()  # Explicit initialization

# Or use properties that auto-initialize
print(setup.problem_name)  # Triggers initialization if needed
\end{lstlisting}

\paragraph{\_ensure\_initialized()}
\begin{lstlisting}[language=Python, caption=Ensure Initialized Method]
def _ensure_initialized(self) -> None
\end{lstlisting}

\textbf{Purpose:} Internal method ensuring initialization has occurred

\textbf{Algorithm:}
\begin{lstlisting}[language=Python, caption=Ensure Initialized Implementation]
if not self._initialized:
    self.initialize()
\end{lstlisting}

\textbf{Usage:} Called automatically by property accessors

\subsubsection{Component Properties (Cached)}

\paragraph{elementary\_matrices}
\begin{lstlisting}[language=Python, caption=Elementary Matrices Property]
@property
def elementary_matrices(self) -> ElementaryMatrices
\end{lstlisting}

\textbf{Returns:} \texttt{ElementaryMatrices} - Elementary matrices instance (created once, cached)

\textbf{Algorithm:}
\begin{lstlisting}[language=Python, caption=Elementary Matrices Property Implementation]
if self._elementary_matrices is None:
    self._elementary_matrices = ElementaryMatrices(orthonormal_basis=False)
return self._elementary_matrices
\end{lstlisting}

\textbf{Configuration:} Uses Lagrange basis (\texttt{orthonormal\_basis=False})

\textbf{Usage:}
\begin{lstlisting}[language=Python, caption=Elementary Matrices Usage]
setup = SolverSetup()
elem_matrices = setup.elementary_matrices

# Get specific matrices
M = elem_matrices.get_matrix('M')        # Mass matrix
T = elem_matrices.get_matrix('T')        # Trace matrix
D = elem_matrices.get_matrix('D')        # Derivative matrix
QUAD = elem_matrices.get_matrix('QUAD')  # Quadrature matrix
\end{lstlisting}

\paragraph{static\_condensations}
\begin{lstlisting}[language=Python, caption=Static Condensations Property]
@property
def static_condensations(self) -> List
\end{lstlisting}

\textbf{Returns:} \texttt{List} - Static condensation implementations for all domains (created once, cached)

\textbf{Algorithm:}
\begin{lstlisting}[language=Python, caption=Static Condensations Implementation]
if self._static_condensations is None:
    self._ensure_initialized()
    self._static_condensations = []
    
    for domain_idx in range(len(self.problems)):
        sc = StaticCondensationFactory.create(
            self.problems[domain_idx],
            self.global_discretization,
            self.elementary_matrices,
            domain_idx
        )
        # Build matrices immediately to cache them
        sc.build_matrices()
        self._static_condensations.append(sc)
        
return self._static_condensations
\end{lstlisting}

\textbf{Process:}
\begin{enumerate}
    \item Creates static condensation for each domain using factory
    \item Immediately builds matrices for caching
    \item Returns list of configured implementations
\end{enumerate}

\textbf{Usage:}
\begin{lstlisting}[language=Python, caption=Static Condensations Usage]
static_condensations = setup.static_condensations

# Access specific domain
sc_domain_0 = static_condensations[0]
matrices = sc_domain_0.build_matrices()  # Already built and cached

# Use in computations
for i, sc in enumerate(static_condensations):
    print(f"Domain {i}: {type(sc).__name__}")
\end{lstlisting}

\paragraph{global\_assembler}
\begin{lstlisting}[language=Python, caption=Global Assembler Property]
@property
def global_assembler(self) -> GlobalAssembler
\end{lstlisting}

\textbf{Returns:} \texttt{GlobalAssembler} - Global assembler instance (created once, cached)

\textbf{Algorithm:}
\begin{lstlisting}[language=Python, caption=Global Assembler Implementation]
if self._global_assembler is None:
    self._ensure_initialized()
    # Check the actual constructor signature and use correct parameters
    self._global_assembler = GlobalAssembler.from_framework_objects(
        self.problems,
        self.global_discretization,
        self.static_condensations,
        self.constraint_manager
    )
return self._global_assembler
\end{lstlisting}

\textbf{Creation:} Uses factory method \texttt{GlobalAssembler.from\_framework\_objects()}

\textbf{Usage:}
\begin{lstlisting}[language=Python, caption=Global Assembler Usage]
assembler = setup.global_assembler

print(f"Total DOFs: {assembler.total_dofs}")
print(f"Trace DOFs: {assembler.total_trace_dofs}")
print(f"Multipliers: {assembler.n_multipliers}")

# Use for system assembly
global_solution = np.random.rand(assembler.total_dofs)
residual, jacobian = assembler.assemble_residual_and_jacobian(
    global_solution, forcing_terms, setup.static_condensations, time=0.0
)
\end{lstlisting}

\paragraph{bulk\_data\_manager}
\begin{lstlisting}[language=Python, caption=Bulk Data Manager Property]
@property
def bulk_data_manager(self) -> BulkDataManager
\end{lstlisting}

\textbf{Returns:} \texttt{BulkDataManager} - Bulk data manager instance (created once, cached)

\textbf{Algorithm:}
\begin{lstlisting}[language=Python, caption=Bulk Data Manager Implementation]
if self._bulk_data_manager is None:
    self._ensure_initialized()
    discretizations = self.global_discretization.spatial_discretizations
    self._domain_data = BulkDataManager.extract_domain_data_list(
        self.problems, discretizations, self.static_condensations
    )
    self._bulk_data_manager = BulkDataManager(
        self._domain_data
    )
return self._bulk_data_manager
\end{lstlisting}

\textbf{Process:}
\begin{enumerate}
    \item Extracts domain data using \texttt{BulkDataManager.extract\_domain\_data\_list()}
    \item Creates lean BulkDataManager with extracted data
    \item Caches both domain data and manager
\end{enumerate}

\textbf{Usage:}
\begin{lstlisting}[language=Python, caption=Bulk Data Manager Usage]
bulk_manager = setup.bulk_data_manager

# Initialize bulk data
bulk_data_list = bulk_manager.initialize_all_bulk_data(
    setup.problems,
    setup.global_discretization.spatial_discretizations,
    time=0.0
)

# Compute forcing terms
forcing_terms = bulk_manager.compute_forcing_terms(
    bulk_data_list, setup.problems,
    setup.global_discretization.spatial_discretizations,
    time=0.1, dt=0.01
)
\end{lstlisting}

\subsubsection{Information and Analysis Methods}

\paragraph{get\_problem\_info()}
\begin{lstlisting}[language=Python, caption=Get Problem Info Method]
def get_problem_info(self) -> Dict[str, Any]
\end{lstlisting}

\textbf{Returns:} \texttt{Dict[str, Any]} - Comprehensive problem information dictionary

\textbf{Algorithm:}
\begin{lstlisting}[language=Python, caption=Problem Info Algorithm]
self._ensure_initialized()

info = {
    'problem_name': self.problem_name,
    'num_domains': len(self.problems),
    'total_elements': sum(disc.n_elements for disc in self.global_discretization.spatial_discretizations),
    'total_trace_dofs': self.global_assembler.total_trace_dofs,
    'num_constraints': self.constraint_manager.n_multipliers if self.constraint_manager else 0,
    'time_discretization': {
        'dt': self.global_discretization.dt,
        'T': self.global_discretization.T,
        'n_steps': self.global_discretization.n_time_steps
    },
    'domains': []
}

for i, (problem, discretization) in enumerate(zip(self.problems, self.global_discretization.spatial_discretizations)):
    domain_info = {
        'index': i,
        'type': problem.type,
        'domain': [problem.domain_start, problem.domain_end],
        'n_elements': discretization.n_elements,
        'n_equations': problem.neq,
        'trace_size': problem.neq * (discretization.n_elements + 1)
    }
    info['domains'].append(domain_info)

return info
\end{lstlisting}

\textbf{Information Structure:}
\begin{lstlisting}[language=Python, caption=Problem Info Structure]
{
    'problem_name': str,
    'num_domains': int,
    'total_elements': int,
    'total_trace_dofs': int,
    'num_constraints': int,
    'time_discretization': {
        'dt': float,
        'T': float,
        'n_steps': int
    },
    'domains': [
        {
            'index': int,
            'type': str,
            'domain': [float, float],  # [start, end]
            'n_elements': int,
            'n_equations': int,
            'trace_size': int
        },
        # ... for each domain
    ]
}
\end{lstlisting}

\textbf{Usage:}
\begin{lstlisting}[language=Python, caption=Problem Info Usage]
info = setup.get_problem_info()

print(f"Problem: {info['problem_name']}")
print(f"Domains: {info['num_domains']}")
print(f"Total DOFs: {info['total_trace_dofs']}")
print(f"Time stepping: dt={info['time_discretization']['dt']}, T={info['time_discretization']['T']}")

# Domain details
for domain_info in info['domains']:
    print(f"  Domain {domain_info['index']}: {domain_info['type']}")
    print(f"    Range: {domain_info['domain']}")
    print(f"    Elements: {domain_info['n_elements']}")
    print(f"    Equations: {domain_info['n_equations']}")
\end{lstlisting}

\subsubsection{Initial Condition and Solution Vector Management}

\paragraph{create\_initial\_conditions()}
\begin{lstlisting}[language=Python, caption=Create Initial Conditions Method]
def create_initial_conditions(self) -> Tuple[List[np.ndarray], np.ndarray]
\end{lstlisting}

\textbf{Returns:} \texttt{Tuple[List[np.ndarray], np.ndarray]} - (trace\_solutions, initial\_multipliers)

\textbf{Algorithm:}
\begin{lstlisting}[language=Python, caption=Initial Conditions Algorithm]
self._ensure_initialized()

trace_solutions = []

for i, (problem, discretization) in enumerate(zip(self.problems, self.global_discretization.spatial_discretizations)):
    n_nodes = discretization.n_elements + 1
    trace_size = problem.neq * n_nodes
    nodes = discretization.nodes
    
    trace_solution = np.zeros(trace_size)
    
    # Apply initial conditions if available
    for eq in range(problem.neq):
        for j in range(n_nodes):
            node_idx = eq * n_nodes + j
            if hasattr(problem, 'u0') and len(problem.u0) > eq and callable(problem.u0[eq]):
                trace_solution[node_idx] = problem.u0[eq](nodes[j])
            elif hasattr(problem, 'initial_conditions') and len(problem.initial_conditions) > eq:
                if callable(problem.initial_conditions[eq]):
                    trace_solution[node_idx] = problem.initial_conditions[eq](nodes[j])
    
    trace_solutions.append(trace_solution)

# Initialize multipliers to zero
n_multipliers = self.constraint_manager.n_multipliers if self.constraint_manager else 0
initial_multipliers = np.zeros(n_multipliers)

return trace_solutions, initial_multipliers
\end{lstlisting}

\textbf{Process:}
\begin{enumerate}
    \item For each domain: create trace solution array
    \item For each equation: evaluate initial condition at mesh nodes
    \item Initialize multipliers to zero
    \item Return domain traces and multiplier arrays
\end{enumerate}

\textbf{Initial Condition Access Patterns:}
\begin{itemize}
    \item \texttt{problem.u0[eq]} (primary pattern)
    \item \texttt{problem.initial\_conditions[eq]} (fallback)
    \item Zero initialization if no initial conditions found
\end{itemize}

\textbf{Usage:}
\begin{lstlisting}[language=Python, caption=Initial Conditions Usage]
trace_solutions, multipliers = setup.create_initial_conditions()

print(f"Created {len(trace_solutions)} domain trace solutions")
for i, trace_sol in enumerate(trace_solutions):
    print(f"  Domain {i}: shape {trace_sol.shape}")

print(f"Initial multipliers: {len(multipliers)} (all zero)")
\end{lstlisting}

\paragraph{create\_global\_solution\_vector()}
\begin{lstlisting}[language=Python, caption=Create Global Solution Vector Method]
def create_global_solution_vector(self, trace_solutions: List[np.ndarray], 
                                 multipliers: np.ndarray) -> np.ndarray
\end{lstlisting}

\textbf{Parameters:}
\begin{itemize}
    \item \texttt{trace\_solutions}: List of trace solution arrays for each domain
    \item \texttt{multipliers}: Array of Lagrange multiplier values
\end{itemize}

\textbf{Returns:} \texttt{np.ndarray} - Global solution vector

\textbf{Algorithm:}
\begin{lstlisting}[language=Python, caption=Global Vector Assembly Algorithm]
global_assembler = self.global_assembler

# Calculate total size
total_trace_size = sum(len(trace) for trace in trace_solutions)
total_size = total_trace_size + len(multipliers)

# Create global solution vector
global_solution = np.zeros(total_size)

# Fill trace solutions
offset = 0
for trace in trace_solutions:
    trace_flat = trace.flatten() if trace.ndim > 1 else trace
    global_solution[offset:offset+len(trace_flat)] = trace_flat
    offset += len(trace_flat)

# Fill multipliers
if len(multipliers) > 0:
    global_solution[offset:offset+len(multipliers)] = multipliers

return global_solution
\end{lstlisting}

\textbf{Assembly Structure:} \texttt{[trace\_domain\_0, trace\_domain\_1, ..., multipliers]}

\textbf{Usage:}
\begin{lstlisting}[language=Python, caption=Global Vector Assembly Usage]
# Create initial conditions
trace_solutions, multipliers = setup.create_initial_conditions()

# Assemble global vector
global_solution = setup.create_global_solution_vector(trace_solutions, multipliers)

print(f"Global solution vector shape: {global_solution.shape}")
print(f"Total DOFs: {len(global_solution)}")

# Verify size matches assembler
expected_size = setup.global_assembler.total_dofs
print(f"Matches assembler DOFs: {len(global_solution) == expected_size}")
\end{lstlisting}

\paragraph{extract\_domain\_solutions()}
\begin{lstlisting}[language=Python, caption=Extract Domain Solutions Method]
def extract_domain_solutions(self, global_solution: np.ndarray) -> Tuple[List[np.ndarray], np.ndarray]
\end{lstlisting}

\textbf{Parameters:}
\begin{itemize}
    \item \texttt{global\_solution}: Global solution vector
\end{itemize}

\textbf{Returns:} \texttt{Tuple[List[np.ndarray], np.ndarray]} - (trace\_solutions, multipliers)

\textbf{Algorithm:}
\begin{lstlisting}[language=Python, caption=Domain Solution Extraction Algorithm]
self._ensure_initialized()

trace_solutions = []
offset = 0

# Extract trace solutions for each domain
for i, (problem, discretization) in enumerate(zip(self.problems, self.global_discretization.spatial_discretizations)):
    n_nodes = discretization.n_elements + 1
    trace_size = problem.neq * n_nodes
    
    trace_solution = global_solution[offset:offset+trace_size]
    trace_solutions.append(trace_solution)
    offset += trace_size

# Extract multipliers
n_multipliers = self.constraint_manager.n_multipliers if self.constraint_manager else 0
if n_multipliers > 0:
    multipliers = global_solution[offset:offset+n_multipliers]
else:
    multipliers = np.array([])

return trace_solutions, multipliers
\end{lstlisting}

\textbf{Purpose:} Inverse operation of \texttt{create\_global\_solution\_vector()}

\textbf{Usage:}
\begin{lstlisting}[language=Python, caption=Domain Solution Extraction Usage]
# Extract from global solution
extracted_traces, extracted_multipliers = setup.extract_domain_solutions(global_solution)

# Verify round-trip consistency
for i, (orig, extracted) in enumerate(zip(trace_solutions, extracted_traces)):
    consistent = np.allclose(orig, extracted)
    print(f"Domain {i} round-trip consistent: {consistent}")

multiplier_consistent = np.allclose(multipliers, extracted_multipliers)
print(f"Multiplier round-trip consistent: {multiplier_consistent}")
\end{lstlisting}

\subsubsection{Validation and Testing}

\paragraph{validate\_setup()}
\begin{lstlisting}[language=Python, caption=Validate Setup Method]
def validate_setup(self, verbose: bool = False) -> bool
\end{lstlisting}

\textbf{Parameters:}
\begin{itemize}
    \item \texttt{verbose}: If True, print detailed validation information (default: False)
\end{itemize}

\textbf{Returns:} \texttt{bool} - True if all validation tests pass, False otherwise

\textbf{Validation Tests:}
\begin{enumerate}
    \item \textbf{Initial Condition Creation}: Tests creation of trace solutions and multipliers
    \item \textbf{Global Vector Round-Trip}: Tests assembly and extraction consistency
    \item \textbf{Multiplier Round-Trip}: Validates multiplier handling
    \item \textbf{Bulk Solution Creation}: Tests BulkData creation for all domains
    \item \textbf{Forcing Term Computation}: Validates forcing term calculation
    \item \textbf{Residual/Jacobian Assembly}: Tests global system assembly
\end{enumerate}

\textbf{Algorithm:}
\begin{lstlisting}[language=Python, caption=Validate Setup Algorithm]
try:
    self._ensure_initialized()
    
    if verbose:
        print(f"Validating setup for problem: {self.problem_name}")
        print(f"Number of domains: {len(self.problems)}")
        print(f"Total DOFs: {self.global_assembler.total_dofs}")
    
    # Test initial conditions
    trace_solutions, multipliers = self.create_initial_conditions()
    if verbose:
        print(f"✓ Initial conditions created")
    
    # Test global vector assembly/extraction
    global_solution = self.create_global_solution_vector(trace_solutions, multipliers)
    extracted_traces, extracted_multipliers = self.extract_domain_solutions(global_solution)
    
    # Verify round-trip consistency
    for i, (orig, extracted) in enumerate(zip(trace_solutions, extracted_traces)):
        if not np.allclose(orig, extracted):
            if verbose:
                print(f"✗ Round-trip test failed for domain {i}")
            return False
    
    if not np.allclose(multipliers, extracted_multipliers):
        if verbose:
            print(f"✗ Round-trip test failed for multipliers")
        return False
    
    if verbose:
        print(f"✓ Global vector round-trip test passed")
    
    # Test bulk data manager
    bulk_solutions = []
    for i in range(len(self.problems)):
        problem = self.problems[i]
        discretization = self.global_discretization.spatial_discretizations[i]
        bulk_sol = self.bulk_data_manager.create_bulk_data(i, problem, discretization)
        bulk_solutions.append(bulk_sol)
    
    if verbose:
        print(f"✓ Bulk solutions created")
    
    # Test forcing term computation
    forcing_terms = self.bulk_data_manager.compute_forcing_terms(bulk_solutions, 
                                                                 self.problems, 
                                                                 self.global_discretization.spatial_discretizations, 
                                                                 0.0, 
                                                                 self.global_discretization.dt
                                                                 )

    if verbose:
        print(f"✓ Forcing terms computed")
    
    # Test residual/jacobian computation
    global_residual, global_jacobian = self.global_assembler.assemble_residual_and_jacobian(
        global_solution=global_solution,
        forcing_terms=forcing_terms,
        static_condensations=self._static_condensations,
        time=0.0
    )
    
    if verbose:
        print(f"✓ Global residual and Jacobian assembled")
        print(f"  Residual norm: {np.linalg.norm(global_residual):.6e}")
        print(f"  Jacobian condition number: {np.linalg.cond(global_jacobian):.2e}")
    
    return True
    
except Exception as e:
    if verbose:
        print(f"✗ Validation failed: {e}")
        import traceback
        traceback.print_exc()
    return False
\end{lstlisting}

\textbf{Usage:}
\begin{lstlisting}[language=Python, caption=Validation Usage]
# Quick validation
is_valid = setup.validate_setup()
if is_valid:
    print("✓ Setup validation passed")
else:
    print("✗ Setup validation failed")

# Detailed validation with output
is_valid = setup.validate_setup(verbose=True)
\end{lstlisting}

\textbf{Sample Validation Output:}
\begin{lstlisting}[language=Python, caption=Sample Validation Output]
Validating setup for problem: OrganOnChip Test Problem
Number of domains: 1
Total DOFs: 86
✓ Initial conditions created
✓ Global vector round-trip test passed
✓ Bulk solutions created
✓ Forcing terms computed
✓ Global residual and Jacobian assembled
  Residual norm: 1.234567e-02
  Jacobian condition number: 2.34e+03
\end{lstlisting}

\subsection{Factory Functions}
\label{subsec:factory_functions_detailed}

\paragraph{create\_solver\_setup()}
\begin{lstlisting}[language=Python, caption=Create Solver Setup Function]
def create_solver_setup(problem_module: str = "ooc1d.problems.test_problem2") -> SolverSetup
\end{lstlisting}

\textbf{Parameters:}
\begin{itemize}
    \item \texttt{problem\_module}: String path to problem module (default: "ooc1d.problems.test\_problem2")
\end{itemize}

\textbf{Returns:} \texttt{SolverSetup} - Initialized SolverSetup instance

\textbf{Algorithm:}
\begin{lstlisting}[language=Python, caption=Factory Function Implementation]
setup = SolverSetup(problem_module)
setup.initialize()
return setup
\end{lstlisting}

\textbf{Usage:}
\begin{lstlisting}[language=Python, caption=Factory Function Usage]
# Create and initialize setup
setup = create_solver_setup("ooc1d.problems.ooc_test_problem")

# Immediate access to components
assembler = setup.global_assembler
bulk_manager = setup.bulk_data_manager
\end{lstlisting}

\paragraph{quick\_setup()}
\begin{lstlisting}[language=Python, caption=Quick Setup Function]
def quick_setup(problem_module: str = "ooc1d.problems.test_problem2", 
               validate: bool = True) -> SolverSetup
\end{lstlisting}

\textbf{Parameters:}
\begin{itemize}
    \item \texttt{problem\_module}: String path to problem module (default: "ooc1d.problems.test\_problem2")
    \item \texttt{validate}: If True, run validation tests (default: True)
\end{itemize}

\textbf{Returns:} \texttt{SolverSetup} - Validated SolverSetup instance

\textbf{Raises:} \texttt{RuntimeError} if validation fails

\textbf{Algorithm:}
\begin{lstlisting}[language=Python, caption=Quick Setup Implementation]
setup = create_solver_setup(problem_module)
if validate:
    if not setup.validate_setup(verbose=True):
        raise RuntimeError("Setup validation failed")

return setup
\end{lstlisting}

\textbf{Usage:}
\begin{lstlisting}[language=Python, caption=Quick Setup Usage]
# Quick setup with validation
setup = quick_setup("ooc1d.problems.ooc_test_problem", validate=True)

# Quick setup without validation (faster)
setup = quick_setup("ooc1d.problems.test_problem", validate=False)
\end{lstlisting}

\subsection{Complete Usage Examples}
\label{subsec:complete_usage_examples_detailed}

\subsubsection{Complete Solver Setup and Newton Iteration}

\begin{lstlisting}[language=Python, caption=Complete Solver Setup Example]
from setup_solver import quick_setup, SolverSetup
import numpy as np

def complete_solver_example():
    """Complete example of solver setup and Newton iteration."""
    
    # Step 1: Initialize solver with validation
    print("Setting up solver...")
    setup = quick_setup("ooc1d.problems.ooc_test_problem", validate=True)
    
    # Step 2: Get problem information
    info = setup.get_problem_info()
    print(f"\nProblem: {info['problem_name']}")
    print(f"Domains: {info['num_domains']}")
    print(f"Total DOFs: {info['total_trace_dofs'] + info['num_constraints']}")
    
    # Step 3: Create initial conditions
    trace_solutions, multipliers = setup.create_initial_conditions()
    global_solution = setup.create_global_solution_vector(trace_solutions, multipliers)
    
    print(f"\nInitial conditions created:")
    print(f"  Global solution shape: {global_solution.shape}")
    print(f"  Initial residual norm: {np.linalg.norm(global_solution):.6e}")
    
    # Step 4: Setup Newton iteration
    assembler = setup.global_assembler
    bulk_manager = setup.bulk_data_manager
    static_condensations = setup.static_condensations
    
    # Step 5: Newton iteration loop
    tolerance = 1e-10
    max_iterations = 20
    current_time = 0.0
    dt = setup.global_discretization.dt
    
    print(f"\nStarting Newton iterations (tol={tolerance:.0e})...")
    
    for iteration in range(max_iterations):
        # Create bulk data for forcing terms
        bulk_data_list = bulk_manager.initialize_all_bulk_data(
            setup.problems,
            setup.global_discretization.spatial_discretizations,
            time=current_time
        )
        
        # Compute forcing terms
        forcing_terms = bulk_manager.compute_forcing_terms(
            bulk_data_list, setup.problems,
            setup.global_discretization.spatial_discretizations,
            current_time, dt
        )
        
        # Assemble system
        residual, jacobian = assembler.assemble_residual_and_jacobian(
            global_solution=global_solution,
            forcing_terms=forcing_terms,
            static_condensations=static_condensations,
            time=current_time
        )
        
        # Check convergence
        residual_norm = np.linalg.norm(residual)
        print(f"  Iteration {iteration}: ||R|| = {residual_norm:.6e}")
        
        if residual_norm < tolerance:
            print("  ✓ Newton solver converged")
            break
        
        # Newton update
        try:
            delta = np.linalg.solve(jacobian, -residual)
            global_solution += delta
        except np.linalg.LinAlgError:
            print("  ✗ Newton solver failed: singular Jacobian")
            break
    
    # Step 6: Extract final solutions
    final_traces, final_multipliers = setup.extract_domain_solutions(global_solution)
    
    print(f"\nSolver completed:")
    print(f"  Final residual norm: {residual_norm:.6e}")
    print(f"  Domain solutions: {len(final_traces)}")
    print(f"  Constraint multipliers: {len(final_multipliers)}")
    
    return setup, global_solution, final_traces, final_multipliers

# Usage
setup, solution, traces, multipliers = complete_solver_example()
\end{lstlisting}

\subsection{Method Summary Table}
\label{subsec:method_summary_detailed}

\begin{longtable}{|p{4cm}|p{2.5cm}|p{6.5cm}|}
\hline
\textbf{Method/Property} & \textbf{Returns} & \textbf{Purpose} \\
\hline
\endhead

\texttt{\_\_init\_\_} & \texttt{None} & Initialize setup with problem module path \\
\hline

\texttt{initialize} & \texttt{None} & Load and initialize problem configuration \\
\hline

\texttt{elementary\_matrices} & \texttt{ElementaryMatrices} & Get cached elementary matrices \\
\hline

\texttt{static\_condensations} & \texttt{List} & Get cached static condensation implementations \\
\hline

\texttt{global\_assembler} & \texttt{GlobalAssembler} & Get cached global assembler \\
\hline

\texttt{bulk\_data\_manager} & \texttt{BulkDataManager} & Get cached bulk data manager \\
\hline

\texttt{get\_problem\_info} & \texttt{Dict} & Get comprehensive problem information \\
\hline

\texttt{create\_initial\_conditions} & \texttt{Tuple} & Create initial trace solutions and multipliers \\
\hline

\texttt{create\_global\_solution\_vector} & \texttt{np.ndarray} & Assemble global solution from components \\
\hline

\texttt{extract\_domain\_solutions} & \texttt{Tuple} & Extract components from global solution \\
\hline

\texttt{validate\_setup} & \texttt{bool} & Run comprehensive validation tests \\
\hline

\texttt{create\_solver\_setup} & \texttt{SolverSetup} & Factory function for setup creation \\
\hline

\texttt{quick\_setup} & \texttt{SolverSetup} & Factory with optional validation \\
\hline

\end{longtable}

\subsection{Key Features and Design Principles}

\begin{itemize}
    \item \textbf{Lean Architecture}: Minimal data storage with on-demand component creation
    \item \textbf{Component Caching}: Expensive objects created once and cached
    \item \textbf{Dynamic Loading}: Problem modules loaded at runtime via importlib
    \item \textbf{Lazy Initialization}: Components created only when accessed
    \item \textbf{Comprehensive Validation}: Built-in testing of all components and round-trip consistency
    \item \textbf{Clean Interfaces}: Simple property-based access to complex components
    \item \textbf{Factory Patterns}: Convenience functions for common setup scenarios
    \item \textbf{Backward Compatibility}: Aliases and fallbacks for existing code
    \item \textbf{Error Handling}: Robust error reporting and fallback strategies
    \item \textbf{Flexible Architecture}: Support for arbitrary problem modules and configurations
\end{itemize}

This documentation provides an exact reference for the setup solver module, emphasizing its role as the central orchestrator for BioNetFlux solver initialization and its lean, efficient approach to component management.

% End of setup solver module detailed API documentation
