% Discretization Module API Documentation (Accurate Analysis)
% To be included in master LaTeX document
%
% Usage: % Discretization Module API Documentation (Accurate Analysis)
% To be included in master LaTeX document
%
% Usage: % Discretization Module API Documentation (Accurate Analysis)
% To be included in master LaTeX document
%
% Usage: % Discretization Module API Documentation (Accurate Analysis)
% To be included in master LaTeX document
%
% Usage: \input{docs/discretization_module_api}

\section{Discretization Module API Reference}
\label{sec:discretization_module_api}

This section provides a comprehensive reference for the Discretization classes (\texttt{ooc1d.core.discretization}) based on analysis of the BioNetFlux implementation patterns and MATLAB reference files. The module contains classes for spatial and temporal discretization management.

\subsection{Module Overview}

The discretization module contains two main classes:
\begin{itemize}
    \item \texttt{Discretization}: Single-domain spatial discretization with stabilization parameters
    \item \texttt{GlobalDiscretization}: Multi-domain coordinator with time stepping parameters
\end{itemize}

\subsection{Module Dependencies}

\begin{lstlisting}[language=Python, caption=Module Dependencies]
import numpy as np
from typing import List, Optional, Union, Tuple
\end{lstlisting}

\subsection{Discretization Class}
\label{subsec:discretization_class}

The main class for single-domain spatial discretization, handling mesh generation and stabilization parameters for HDG methods.

\subsubsection{Constructor}

\paragraph{\_\_init\_\_()}
\begin{lstlisting}[language=Python, caption=Discretization Constructor]
def __init__(self, 
             n_elements: int,
             domain_start: float = 0.0,
             domain_length: float = 1.0,
             stab_constant: float = 1.0)
\end{lstlisting}

\textbf{Parameters:}
\begin{itemize}
    \item \texttt{n\_elements}: Number of elements in the mesh
    \item \texttt{domain\_start}: Domain start coordinate (default: 0.0)
    \item \texttt{domain\_length}: Domain length (default: 1.0)
    \item \texttt{stab\_constant}: Stabilization constant for HDG method (default: 1.0)
\end{itemize}

\textbf{Usage Examples:}
\begin{lstlisting}[language=Python, caption=Discretization Constructor Usage]
# Basic discretization
disc1 = Discretization(n_elements=20)

# Custom domain discretization (matching MATLAB TestProblem.m)
disc2 = Discretization(
    n_elements=40,
    domain_start=0.0,  # MATLAB: A = 0
    domain_length=1.0, # MATLAB: L = 1
    stab_constant=1.0
)

# Fine mesh discretization
disc3 = Discretization(
    n_elements=100,
    domain_start=-1.0,
    domain_length=2.0,
    stab_constant=0.5
)
\end{lstlisting}

\subsubsection{Core Attributes}

\begin{longtable}{|p{3.5cm}|p{2.5cm}|p{7cm}|}
\hline
\textbf{Attribute} & \textbf{Type} & \textbf{Description} \\
\hline
\endhead

\texttt{n\_elements} & \texttt{int} & Number of elements in the mesh \\
\hline

\texttt{domain\_start} & \texttt{float} & Start coordinate of the domain \\
\hline

\texttt{domain\_length} & \texttt{float} & Length of the domain \\
\hline

\texttt{domain\_end} & \texttt{float} & End coordinate: \texttt{domain\_start + domain\_length} \\
\hline

\texttt{stab\_constant} & \texttt{float} & HDG stabilization constant \\
\hline

\texttt{n\_nodes} & \texttt{int} & Number of nodes: \texttt{n\_elements + 1} \\
\hline

\texttt{element\_length} & \texttt{float} & Uniform element size: \texttt{domain\_length / n\_elements} \\
\hline

\texttt{nodes} & \texttt{np.ndarray} & Node coordinates array (length \texttt{n\_nodes}) \\
\hline

\texttt{elements} & \texttt{List[Tuple]} & Element connectivity: \texttt{[(i, i+1) for i in range(n\_elements)]} \\
\hline

\texttt{tau} & \texttt{List[float]} & Stabilization parameters per equation \\
\hline

\texttt{dt} & \texttt{Optional[float]} & Time step size (set by GlobalDiscretization) \\
\hline

\end{longtable}

\subsubsection{Mesh Generation Methods}

\paragraph{generate\_nodes()}
\begin{lstlisting}[language=Python, caption=Generate Nodes Method]
def generate_nodes(self) -> np.ndarray
\end{lstlisting}

\textbf{Returns:} \texttt{np.ndarray} - Array of node coordinates

\textbf{Implementation:} \texttt{np.linspace(domain\_start, domain\_start + domain\_length, n\_nodes)}

\textbf{Usage:}
\begin{lstlisting}[language=Python, caption=Node Generation Usage]
disc = Discretization(n_elements=10, domain_start=0.0, domain_length=2.0)
nodes = disc.generate_nodes()
# nodes = [0.0, 0.2, 0.4, 0.6, 0.8, 1.0, 1.2, 1.4, 1.6, 1.8, 2.0]
print(f"Node coordinates: {nodes}")
print(f"Number of nodes: {len(nodes)}")
\end{lstlisting}

\paragraph{generate\_elements()}
\begin{lstlisting}[language=Python, caption=Generate Elements Method]
def generate_elements(self) -> List[Tuple[int, int]]
\end{lstlisting}

\textbf{Returns:} \texttt{List[Tuple[int, int]]} - Element connectivity list

\textbf{Implementation:} \texttt{[(i, i+1) for i in range(n\_elements)]}

\textbf{Usage:}
\begin{lstlisting}[language=Python, caption=Element Generation Usage]
disc = Discretization(n_elements=4)
elements = disc.generate_elements()
# elements = [(0, 1), (1, 2), (2, 3), (3, 4)]
for i, (node1, node2) in enumerate(elements):
    print(f"Element {i}: nodes {node1} -> {node2}")
\end{lstlisting}

\subsubsection{Stabilization Parameter Management}

\paragraph{set\_tau()}
\begin{lstlisting}[language=Python, caption=Set Tau Method]
def set_tau(self, tau_values: Union[float, List[float]])
\end{lstlisting}

\textbf{Parameters:}
\begin{itemize}
    \item \texttt{tau\_values}: Single value or list of stabilization parameters per equation
\end{itemize}

\textbf{Side Effects:} Sets \texttt{self.tau} attribute

\textbf{Usage (Based on MATLAB scBlocks.m):}
\begin{lstlisting}[language=Python, caption=Tau Parameter Usage]
# Keller-Segel problem (2 equations)
ks_disc = Discretization(n_elements=20)
ks_disc.set_tau([1.0, 1.0])  # [tau_u, tau_phi]

# OrganOnChip problem (4 equations) - from MATLAB TestProblem.m
ooc_disc = Discretization(n_elements=40)
ooc_disc.set_tau([1.0, 1.0, 1.0, 1.0])  # [tu, to, tv, tp]

# Single tau for all equations
simple_disc = Discretization(n_elements=10)
simple_disc.set_tau(0.5)  # Applied to all equations
\end{lstlisting}

\paragraph{get\_tau()}
\begin{lstlisting}[language=Python, caption=Get Tau Method]
def get_tau(self, equation_idx: Optional[int] = None) -> Union[float, List[float]]
\end{lstlisting}

\textbf{Parameters:}
\begin{itemize}
    \item \texttt{equation\_idx}: Equation index (optional, returns all if None)
\end{itemize}

\textbf{Returns:} \texttt{float} or \texttt{List[float]} - Stabilization parameter(s)

\textbf{Usage:}
\begin{lstlisting}[language=Python, caption=Get Tau Usage]
# Get all tau values
all_tau = disc.get_tau()
print(f"All tau values: {all_tau}")

# Get specific tau value  
tau_u = disc.get_tau(0)  # First equation
tau_phi = disc.get_tau(1)  # Second equation
\end{lstlisting}

\subsubsection{Geometric Query Methods}

\paragraph{get\_element\_center()}
\begin{lstlisting}[language=Python, caption=Get Element Center Method]
def get_element_center(self, element_idx: int) -> float
\end{lstlisting}

\textbf{Parameters:}
\begin{itemize}
    \item \texttt{element\_idx}: Element index (0 to \texttt{n\_elements-1})
\end{itemize}

\textbf{Returns:} \texttt{float} - Center coordinate of the element

\textbf{Usage:}
\begin{lstlisting}[language=Python, caption=Element Center Usage]
disc = Discretization(n_elements=10, domain_start=0.0, domain_length=1.0)
center_0 = disc.get_element_center(0)  # First element center
center_5 = disc.get_element_center(5)  # Sixth element center
print(f"Element 0 center: {center_0}")
print(f"Element 5 center: {center_5}")
\end{lstlisting}

\paragraph{get\_element\_bounds()}\leavevmode
\begin{lstlisting}[language=Python, caption=Get Element Bounds Method]
def get_element_bounds(self, element_idx: int) -> Tuple[float, float]
\end{lstlisting}

\textbf{Parameters:}
\begin{itemize}
    \item \texttt{element\_idx}: Element index
\end{itemize}

\textbf{Returns:} \texttt{Tuple[float, float]} - Element start and end coordinates

\textbf{Usage:}
\begin{lstlisting}[language=Python, caption=Element Bounds Usage]
start, end = disc.get_element_bounds(3)  # Fourth element
print(f"Element 3 bounds: [{start:.3f}, {end:.3f}]")
element_length = end - start
\end{lstlisting}

\paragraph{find\_element\_containing\_point()}\leavevmode
[Yet to be implemented]
\begin{lstlisting}[language=Python, caption=Find Element Method]
def find_element_containing_point(self, point: float) -> int
\end{lstlisting}

\textbf{Parameters:}
\begin{itemize}
    \item \texttt{point}: Coordinate to locate
\end{itemize}

\textbf{Returns:} \texttt{int} - Element index containing the point (-1 if outside domain)

\textbf{Usage:}
\begin{lstlisting}[language=Python, caption=Find Element Usage]
disc = Discretization(n_elements=10, domain_start=0.0, domain_length=1.0)
element_idx = disc.find_element_containing_point(0.35)
print(f"Point 0.35 is in element {element_idx}")
\end{lstlisting}

\subsubsection{Utility Methods}

\paragraph{refine\_mesh()}\leavevmode
[Yet to be implemented]
\begin{lstlisting}[language=Python, caption=Refine Mesh Method]
def refine_mesh(self, refinement_factor: int = 2) -> 'Discretization'
\end{lstlisting}

\textbf{Parameters:}
\begin{itemize}
    \item \texttt{refinement\_factor}: Factor by which to increase element count (default: 2)
\end{itemize}

\textbf{Returns:} \texttt{Discretization} - New discretization with refined mesh

\textbf{Usage:}
\begin{lstlisting}[language=Python, caption=Mesh Refinement Usage]
coarse_disc = Discretization(n_elements=10)
fine_disc = coarse_disc.refine_mesh(refinement_factor=2)
print(f"Original elements: {coarse_disc.n_elements}")
print(f"Refined elements: {fine_disc.n_elements}")  # Should be 20
\end{lstlisting}

\paragraph{summary()}\leavevmode
\begin{lstlisting}[language=Python, caption=Discretization Summary Method]
def summary(self) -> str
\end{lstlisting}

\textbf{Returns:} \texttt{str} - Multi-line summary of discretization properties

\textbf{Usage:}
\begin{lstlisting}[language=Python, caption=Summary Usage]
disc = Discretization(n_elements=20, domain_start=0.0, domain_length=2.0)
disc.set_tau([1.0, 0.5])
print(disc.summary())
# Output:
# Discretization Summary:
#   Domain: [0.000, 2.000] (length: 2.000)
#   Elements: 20, Nodes: 21
#   Element size: 0.100
#   Stabilization: tau = [1.0, 0.5]
#   Time step: dt = 0.010 (if set)
\end{lstlisting}

\subsection{GlobalDiscretization Class}
\label{subsec:globaldiscretization_class}

Coordinator class for managing multiple discretizations and time stepping parameters.

\subsubsection{Constructor}

\paragraph{\_\_init\_\_()}
\begin{lstlisting}[language=Python, caption=GlobalDiscretization Constructor]
def __init__(self, spatial_discretizations: List[Discretization])
\end{lstlisting}

\textbf{Parameters:}
\begin{itemize}
    \item \texttt{spatial\_discretizations}: List of Discretization instances for each domain
\end{itemize}

\textbf{Usage:}
\begin{lstlisting}[language=Python, caption=GlobalDiscretization Usage]
# Single domain
disc1 = Discretization(n_elements=20, domain_start=0.0, domain_length=1.0)
global_disc = GlobalDiscretization([disc1])

# Multi-domain (Y-junction example)
main_disc = Discretization(n_elements=30, domain_start=0.0, domain_length=1.0)
branch1_disc = Discretization(n_elements=20, domain_start=1.0, domain_length=0.8)
branch2_disc = Discretization(n_elements=20, domain_start=1.0, domain_length=0.8)

multi_global_disc = GlobalDiscretization([main_disc, branch1_disc, branch2_disc])
\end{lstlisting}

\subsubsection{Core Attributes}

\begin{longtable}{|p{3.5cm}|p{2.5cm}|p{7cm}|}
\hline
\textbf{Attribute} & \textbf{Type} & \textbf{Description} \\
\hline
\endhead

\texttt{spatial\_discretizations} & \texttt{List[Discretization]} & List of spatial discretizations \\
\hline

\texttt{n\_domains} & \texttt{int} & Number of domains \\
\hline

\texttt{dt} & \texttt{Optional[float]} & Global time step size \\
\hline

\texttt{T} & \texttt{Optional[float]} & Final time \\
\hline

\texttt{n\_time\_steps} & \texttt{Optional[int]} & Number of time steps: \texttt{int(T/dt)} \\
\hline

\texttt{total\_elements} & \texttt{int} & Sum of elements across all domains \\
\hline

\texttt{total\_nodes} & \texttt{int} & Sum of nodes across all domains \\
\hline

\end{longtable}

\subsubsection{Time Parameter Management}

\paragraph{set\_time\_parameters()}
\begin{lstlisting}[language=Python, caption=Set Time Parameters Method]
def set_time_parameters(self, dt: float, T: float)
\end{lstlisting}

\textbf{Parameters:}
\begin{itemize}
    \item \texttt{dt}: Time step size (corresponding to MATLAB \texttt{discretization.dt})
    \item \texttt{T}: Final time
\end{itemize}

\textbf{Side Effects:} 
\begin{itemize}
    \item Sets \texttt{self.dt}, \texttt{self.T}, and \texttt{self.n\_time\_steps}
    \item Propagates \texttt{dt} to all spatial discretizations
\end{itemize}

\textbf{Usage:}
\begin{lstlisting}[language=Python, caption=Time Parameters Usage]
# Set time parameters (matching MATLAB TestProblem.m patterns)
global_disc.set_time_parameters(dt=0.01, T=1.0)
print(f"Time step: {global_disc.dt}")
print(f"Final time: {global_disc.T}")
print(f"Number of time steps: {global_disc.n_time_steps}")

# Verify propagation to spatial discretizations
for i, disc in enumerate(global_disc.spatial_discretizations):
    print(f"Domain {i} dt: {disc.dt}")
\end{lstlisting}

\paragraph{get\_time\_points()}
\begin{lstlisting}[language=Python, caption=Get Time Points Method]
def get_time_points(self) -> np.ndarray
\end{lstlisting}

\textbf{Returns:} \texttt{np.ndarray} - Array of time points from 0 to T

\textbf{Usage:}
\begin{lstlisting}[language=Python, caption=Time Points Usage]
time_points = global_disc.get_time_points()
print(f"Time points: {time_points[:5]}...")  # First 5 time points
print(f"Total time points: {len(time_points)}")
\end{lstlisting}

\subsubsection{Domain Access Methods}

\paragraph{get\_discretization()}
\begin{lstlisting}[language=Python, caption=Get Discretization Method]
def get_discretization(self, domain_idx: int) -> Discretization
\end{lstlisting}

\textbf{Parameters:}
\begin{itemize}
    \item \texttt{domain\_idx}: Domain index (0 to \texttt{n\_domains-1})
\end{itemize}

\textbf{Returns:} \texttt{Discretization} - Spatial discretization for specified domain

\textbf{Usage:}
\begin{lstlisting}[language=Python, caption=Get Discretization Usage]
domain_0_disc = global_disc.get_discretization(0)
print(f"Domain 0 elements: {domain_0_disc.n_elements}")
print(f"Domain 0 nodes: {domain_0_disc.n_nodes}")
\end{lstlisting}

\paragraph{get\_all\_discretizations()}
\begin{lstlisting}[language=Python, caption=Get All Discretizations Method]
def get_all_discretizations(self) -> List[Discretization]
\end{lstlisting}

\textbf{Returns:} \texttt{List[Discretization]} - Copy of all spatial discretizations

\textbf{Usage:}
\begin{lstlisting}[language=Python, caption=Get All Discretizations Usage]
all_disc = global_disc.get_all_discretizations()
for i, disc in enumerate(all_disc):
    print(f"Domain {i}: {disc.n_elements} elements")
\end{lstlisting}

\subsubsection{Global Statistics Methods}

\paragraph{compute\_global\_statistics()}
\begin{lstlisting}[language=Python, caption=Compute Global Statistics Method]
def compute_global_statistics(self) -> dict
\end{lstlisting}

\textbf{Returns:} \texttt{dict} - Dictionary with global discretization statistics

\textbf{Usage:}
\begin{lstlisting}[language=Python, caption=Global Statistics Usage]
stats = global_disc.compute_global_statistics()
print(f"Total elements: {stats['total_elements']}")
print(f"Total nodes: {stats['total_nodes']}")
print(f"Average element size: {stats['avg_element_size']:.6f}")
print(f"Min element size: {stats['min_element_size']:.6f}")
print(f"Max element size: {stats['max_element_size']:.6f}")
\end{lstlisting}

\paragraph{get\_cfl\_number()}
\begin{lstlisting}[language=Python, caption=Get CFL Number Method]
def get_cfl_number(self, wave_speed: float = 1.0) -> float
\end{lstlisting}

\textbf{Parameters:}
\begin{itemize}
    \item \texttt{wave\_speed}: Characteristic wave speed for CFL calculation (default: 1.0)
\end{itemize}

\textbf{Returns:} \texttt{float} - CFL number: \texttt{wave\_speed * dt / min\_element\_size}

\textbf{Usage:}
\begin{lstlisting}[language=Python, caption=CFL Number Usage]
cfl = global_disc.get_cfl_number(wave_speed=1.0)
print(f"CFL number: {cfl:.6f}")

# Check stability condition
if cfl <= 1.0:
    print("✓ CFL condition satisfied")
else:
    print("⚠ CFL condition violated - consider smaller dt or larger elements")
\end{lstlisting}

\subsection{Integration with BioNetFlux Components}
\label{subsec:discretization_integration}

\subsubsection{Integration with Problem Class}

\begin{lstlisting}[language=Python, caption=Problem-Discretization Integration]
from ooc1d.core.problem import Problem

# Create problem and matching discretization
problem = Problem(
    neq=4,
    domain_start=0.0,
    domain_length=1.0,
    parameters=np.array([1.0, 2.0, 1.0, 1.0, 0.0, 1.0, 0.0, 1.0, 1.0]),
    problem_type="organ_on_chip"
)

# Create matching discretization
discretization = Discretization(
    n_elements=40,
    domain_start=problem.domain_start,
    domain_length=problem.domain_length,
    stab_constant=1.0
)

# Set stabilization parameters for 4-equation system
discretization.set_tau([1.0, 1.0, 1.0, 1.0])  # [tu, to, tv, tp]

# Create global discretization with time parameters
global_disc = GlobalDiscretization([discretization])
global_disc.set_time_parameters(dt=0.01, T=1.0)
\end{lstlisting}

\subsubsection{Integration with Static Condensation}

\begin{lstlisting}[language=Python, caption=Static Condensation Integration]
from ooc1d.core.static_condensation_ooc import StaticCondensationOOC
from ooc1d.utils.elementary_matrices import ElementaryMatrices

# Create static condensation using discretization
elementary_matrices = ElementaryMatrices()
static_condensation = StaticCondensationOOC(
    problem=problem,
    discretization=discretization,
    elementary_matrices=elementary_matrices
)

# Verify discretization parameters are accessible
print(f"Element length: {discretization.element_length}")
print(f"Time step: {discretization.dt}")
print(f"Stabilization parameters: {discretization.tau}")
\end{lstlisting}

\subsection{Complete Usage Examples}
\label{subsec:discretization_complete_examples}

\subsubsection{Single Domain OrganOnChip Setup}

\begin{lstlisting}[language=Python, caption=Complete Single Domain Setup]
def create_ooc_discretization(n_elements: int = 40,
                             domain_length: float = 1.0,
                             dt: float = 0.01,
                             T: float = 1.0) -> GlobalDiscretization:
    """Create discretization for OrganOnChip problem matching MATLAB TestProblem.m."""
    
    # Create spatial discretization
    spatial_disc = Discretization(
        n_elements=n_elements,
        domain_start=0.0,  # MATLAB: A = 0
        domain_length=domain_length,  # MATLAB: L = 1
        stab_constant=1.0
    )
    
    # Set stabilization parameters for 4-equation OrganOnChip system
    # Based on MATLAB scBlocks.m: tu, to, tv, tp
    spatial_disc.set_tau([1.0, 1.0, 1.0, 1.0])
    
    # Create global discretization
    global_disc = GlobalDiscretization([spatial_disc])
    
    # Set time parameters (matching MATLAB discretization.dt)
    global_disc.set_time_parameters(dt=dt, T=T)
    
    return global_disc

# Usage
ooc_global_disc = create_ooc_discretization()
print(ooc_global_disc.spatial_discretizations[0].summary())
\end{lstlisting}

\subsubsection{Multi-Domain Network Setup}

\begin{lstlisting}[language=Python, caption=Multi-Domain Network Discretization]
def create_network_discretization(domain_specs: List[dict],
                                 elements_per_domain: int = 20,
                                 dt: float = 0.01,
                                 T: float = 0.5) -> GlobalDiscretization:
    """Create discretization for multi-domain network."""
    
    spatial_discretizations = []
    
    for i, spec in enumerate(domain_specs):
        disc = Discretization(
            n_elements=elements_per_domain,
            domain_start=spec['start'],
            domain_length=spec['length'],
            stab_constant=1.0
        )
        
        # Set stabilization parameters based on problem type
        if 'problem_type' in spec and spec['problem_type'] == 'organ_on_chip':
            disc.set_tau([1.0, 1.0, 1.0, 1.0])  # 4 equations
        else:
            disc.set_tau([1.0, 1.0])  # 2 equations (Keller-Segel)
        
        spatial_discretizations.append(disc)
    
    # Create global discretization
    global_disc = GlobalDiscretization(spatial_discretizations)
    global_disc.set_time_parameters(dt=dt, T=T)
    
    return global_disc

# Usage: Y-junction network
domain_specs = [
    {'start': 0.0, 'length': 1.0, 'problem_type': 'organ_on_chip'},  # Main
    {'start': 1.0, 'length': 0.8, 'problem_type': 'organ_on_chip'},  # Branch 1
    {'start': 1.0, 'length': 0.8, 'problem_type': 'organ_on_chip'}   # Branch 2
]

network_global_disc = create_network_discretization(domain_specs)
print(f"Network has {network_global_disc.n_domains} domains")
print(f"Total elements: {network_global_disc.total_elements}")
\end{lstlisting}

\subsection{Method Summary Table}
\label{subsec:discretization_method_summary}

\subsubsection{Discretization Class Methods}

\begin{longtable}{|p{4cm}|p{2cm}|p{7cm}|}
\hline
\textbf{Method} & \textbf{Returns} & \textbf{Purpose} \\
\hline
\endhead

\texttt{generate\_nodes} & \texttt{np.ndarray} & Generate uniform node coordinates \\
\hline

\texttt{generate\_elements} & \texttt{List[Tuple]} & Generate element connectivity \\
\hline

\texttt{set\_tau} & \texttt{None} & Set stabilization parameters \\
\hline

\texttt{get\_tau} & \texttt{float/List} & Retrieve stabilization parameters \\
\hline

\texttt{get\_element\_center} & \texttt{float} & Get element center coordinate \\
\hline

\texttt{get\_element\_bounds} & \texttt{Tuple} & Get element start/end coordinates \\
\hline

\texttt{find\_element\_containing\_point} & \texttt{int} & Locate element containing given point \\
\hline

\texttt{refine\_mesh} & \texttt{Discretization} & Create refined discretization \\
\hline

\texttt{summary} & \texttt{str} & Generate discretization summary \\
\hline

\end{longtable}

\subsubsection{GlobalDiscretization Class Methods}

\begin{longtable}{|p{4cm}|p{2cm}|p{7cm}|}
\hline
\textbf{Method} & \textbf{Returns} & \textbf{Purpose} \\
\hline
\endhead

\texttt{set\_time\_parameters} & \texttt{None} & Set global time step and final time \\
\hline

\texttt{get\_time\_points} & \texttt{np.ndarray} & Generate array of time points \\
\hline

\texttt{get\_discretization} & \texttt{Discretization} & Access specific domain discretization \\
\hline

\texttt{get\_all\_discretizations} & \texttt{List} & Get all spatial discretizations \\
\hline

\texttt{compute\_global\_statistics} & \texttt{dict} & Calculate global mesh statistics \\
\hline

\texttt{get\_cfl\_number} & \texttt{float} & Compute CFL number for stability \\
\hline

\end{longtable}

This documentation provides an exact reference for the Discretization module based on the BioNetFlux implementation patterns and MATLAB reference files, with practical usage examples for both single and multi-domain scenarios.

% End of discretization module API documentation


\section{Discretization Module API Reference}
\label{sec:discretization_module_api}

This section provides a comprehensive reference for the Discretization classes (\texttt{ooc1d.core.discretization}) based on analysis of the BioNetFlux implementation patterns and MATLAB reference files. The module contains classes for spatial and temporal discretization management.

\subsection{Module Overview}

The discretization module contains two main classes:
\begin{itemize}
    \item \texttt{Discretization}: Single-domain spatial discretization with stabilization parameters
    \item \texttt{GlobalDiscretization}: Multi-domain coordinator with time stepping parameters
\end{itemize}

\subsection{Module Dependencies}

\begin{lstlisting}[language=Python, caption=Module Dependencies]
import numpy as np
from typing import List, Optional, Union, Tuple
\end{lstlisting}

\subsection{Discretization Class}
\label{subsec:discretization_class}

The main class for single-domain spatial discretization, handling mesh generation and stabilization parameters for HDG methods.

\subsubsection{Constructor}

\paragraph{\_\_init\_\_()}
\begin{lstlisting}[language=Python, caption=Discretization Constructor]
def __init__(self, 
             n_elements: int,
             domain_start: float = 0.0,
             domain_length: float = 1.0,
             stab_constant: float = 1.0)
\end{lstlisting}

\textbf{Parameters:}
\begin{itemize}
    \item \texttt{n\_elements}: Number of elements in the mesh
    \item \texttt{domain\_start}: Domain start coordinate (default: 0.0)
    \item \texttt{domain\_length}: Domain length (default: 1.0)
    \item \texttt{stab\_constant}: Stabilization constant for HDG method (default: 1.0)
\end{itemize}

\textbf{Usage Examples:}
\begin{lstlisting}[language=Python, caption=Discretization Constructor Usage]
# Basic discretization
disc1 = Discretization(n_elements=20)

# Custom domain discretization (matching MATLAB TestProblem.m)
disc2 = Discretization(
    n_elements=40,
    domain_start=0.0,  # MATLAB: A = 0
    domain_length=1.0, # MATLAB: L = 1
    stab_constant=1.0
)

# Fine mesh discretization
disc3 = Discretization(
    n_elements=100,
    domain_start=-1.0,
    domain_length=2.0,
    stab_constant=0.5
)
\end{lstlisting}

\subsubsection{Core Attributes}

\begin{longtable}{|p{3.5cm}|p{2.5cm}|p{7cm}|}
\hline
\textbf{Attribute} & \textbf{Type} & \textbf{Description} \\
\hline
\endhead

\texttt{n\_elements} & \texttt{int} & Number of elements in the mesh \\
\hline

\texttt{domain\_start} & \texttt{float} & Start coordinate of the domain \\
\hline

\texttt{domain\_length} & \texttt{float} & Length of the domain \\
\hline

\texttt{domain\_end} & \texttt{float} & End coordinate: \texttt{domain\_start + domain\_length} \\
\hline

\texttt{stab\_constant} & \texttt{float} & HDG stabilization constant \\
\hline

\texttt{n\_nodes} & \texttt{int} & Number of nodes: \texttt{n\_elements + 1} \\
\hline

\texttt{element\_length} & \texttt{float} & Uniform element size: \texttt{domain\_length / n\_elements} \\
\hline

\texttt{nodes} & \texttt{np.ndarray} & Node coordinates array (length \texttt{n\_nodes}) \\
\hline

\texttt{elements} & \texttt{List[Tuple]} & Element connectivity: \texttt{[(i, i+1) for i in range(n\_elements)]} \\
\hline

\texttt{tau} & \texttt{List[float]} & Stabilization parameters per equation \\
\hline

\texttt{dt} & \texttt{Optional[float]} & Time step size (set by GlobalDiscretization) \\
\hline

\end{longtable}

\subsubsection{Mesh Generation Methods}

\paragraph{generate\_nodes()}
\begin{lstlisting}[language=Python, caption=Generate Nodes Method]
def generate_nodes(self) -> np.ndarray
\end{lstlisting}

\textbf{Returns:} \texttt{np.ndarray} - Array of node coordinates

\textbf{Implementation:} \texttt{np.linspace(domain\_start, domain\_start + domain\_length, n\_nodes)}

\textbf{Usage:}
\begin{lstlisting}[language=Python, caption=Node Generation Usage]
disc = Discretization(n_elements=10, domain_start=0.0, domain_length=2.0)
nodes = disc.generate_nodes()
# nodes = [0.0, 0.2, 0.4, 0.6, 0.8, 1.0, 1.2, 1.4, 1.6, 1.8, 2.0]
print(f"Node coordinates: {nodes}")
print(f"Number of nodes: {len(nodes)}")
\end{lstlisting}

\paragraph{generate\_elements()}
\begin{lstlisting}[language=Python, caption=Generate Elements Method]
def generate_elements(self) -> List[Tuple[int, int]]
\end{lstlisting}

\textbf{Returns:} \texttt{List[Tuple[int, int]]} - Element connectivity list

\textbf{Implementation:} \texttt{[(i, i+1) for i in range(n\_elements)]}

\textbf{Usage:}
\begin{lstlisting}[language=Python, caption=Element Generation Usage]
disc = Discretization(n_elements=4)
elements = disc.generate_elements()
# elements = [(0, 1), (1, 2), (2, 3), (3, 4)]
for i, (node1, node2) in enumerate(elements):
    print(f"Element {i}: nodes {node1} -> {node2}")
\end{lstlisting}

\subsubsection{Stabilization Parameter Management}

\paragraph{set\_tau()}
\begin{lstlisting}[language=Python, caption=Set Tau Method]
def set_tau(self, tau_values: Union[float, List[float]])
\end{lstlisting}

\textbf{Parameters:}
\begin{itemize}
    \item \texttt{tau\_values}: Single value or list of stabilization parameters per equation
\end{itemize}

\textbf{Side Effects:} Sets \texttt{self.tau} attribute

\textbf{Usage (Based on MATLAB scBlocks.m):}
\begin{lstlisting}[language=Python, caption=Tau Parameter Usage]
# Keller-Segel problem (2 equations)
ks_disc = Discretization(n_elements=20)
ks_disc.set_tau([1.0, 1.0])  # [tau_u, tau_phi]

# OrganOnChip problem (4 equations) - from MATLAB TestProblem.m
ooc_disc = Discretization(n_elements=40)
ooc_disc.set_tau([1.0, 1.0, 1.0, 1.0])  # [tu, to, tv, tp]

# Single tau for all equations
simple_disc = Discretization(n_elements=10)
simple_disc.set_tau(0.5)  # Applied to all equations
\end{lstlisting}

\paragraph{get\_tau()}
\begin{lstlisting}[language=Python, caption=Get Tau Method]
def get_tau(self, equation_idx: Optional[int] = None) -> Union[float, List[float]]
\end{lstlisting}

\textbf{Parameters:}
\begin{itemize}
    \item \texttt{equation\_idx}: Equation index (optional, returns all if None)
\end{itemize}

\textbf{Returns:} \texttt{float} or \texttt{List[float]} - Stabilization parameter(s)

\textbf{Usage:}
\begin{lstlisting}[language=Python, caption=Get Tau Usage]
# Get all tau values
all_tau = disc.get_tau()
print(f"All tau values: {all_tau}")

# Get specific tau value  
tau_u = disc.get_tau(0)  # First equation
tau_phi = disc.get_tau(1)  # Second equation
\end{lstlisting}

\subsubsection{Geometric Query Methods}

\paragraph{get\_element\_center()}
\begin{lstlisting}[language=Python, caption=Get Element Center Method]
def get_element_center(self, element_idx: int) -> float
\end{lstlisting}

\textbf{Parameters:}
\begin{itemize}
    \item \texttt{element\_idx}: Element index (0 to \texttt{n\_elements-1})
\end{itemize}

\textbf{Returns:} \texttt{float} - Center coordinate of the element

\textbf{Usage:}
\begin{lstlisting}[language=Python, caption=Element Center Usage]
disc = Discretization(n_elements=10, domain_start=0.0, domain_length=1.0)
center_0 = disc.get_element_center(0)  # First element center
center_5 = disc.get_element_center(5)  # Sixth element center
print(f"Element 0 center: {center_0}")
print(f"Element 5 center: {center_5}")
\end{lstlisting}

\paragraph{get\_element\_bounds()}\leavevmode
\begin{lstlisting}[language=Python, caption=Get Element Bounds Method]
def get_element_bounds(self, element_idx: int) -> Tuple[float, float]
\end{lstlisting}

\textbf{Parameters:}
\begin{itemize}
    \item \texttt{element\_idx}: Element index
\end{itemize}

\textbf{Returns:} \texttt{Tuple[float, float]} - Element start and end coordinates

\textbf{Usage:}
\begin{lstlisting}[language=Python, caption=Element Bounds Usage]
start, end = disc.get_element_bounds(3)  # Fourth element
print(f"Element 3 bounds: [{start:.3f}, {end:.3f}]")
element_length = end - start
\end{lstlisting}

\paragraph{find\_element\_containing\_point()}\leavevmode
[Yet to be implemented]
\begin{lstlisting}[language=Python, caption=Find Element Method]
def find_element_containing_point(self, point: float) -> int
\end{lstlisting}

\textbf{Parameters:}
\begin{itemize}
    \item \texttt{point}: Coordinate to locate
\end{itemize}

\textbf{Returns:} \texttt{int} - Element index containing the point (-1 if outside domain)

\textbf{Usage:}
\begin{lstlisting}[language=Python, caption=Find Element Usage]
disc = Discretization(n_elements=10, domain_start=0.0, domain_length=1.0)
element_idx = disc.find_element_containing_point(0.35)
print(f"Point 0.35 is in element {element_idx}")
\end{lstlisting}

\subsubsection{Utility Methods}

\paragraph{refine\_mesh()}\leavevmode
[Yet to be implemented]
\begin{lstlisting}[language=Python, caption=Refine Mesh Method]
def refine_mesh(self, refinement_factor: int = 2) -> 'Discretization'
\end{lstlisting}

\textbf{Parameters:}
\begin{itemize}
    \item \texttt{refinement\_factor}: Factor by which to increase element count (default: 2)
\end{itemize}

\textbf{Returns:} \texttt{Discretization} - New discretization with refined mesh

\textbf{Usage:}
\begin{lstlisting}[language=Python, caption=Mesh Refinement Usage]
coarse_disc = Discretization(n_elements=10)
fine_disc = coarse_disc.refine_mesh(refinement_factor=2)
print(f"Original elements: {coarse_disc.n_elements}")
print(f"Refined elements: {fine_disc.n_elements}")  # Should be 20
\end{lstlisting}

\paragraph{summary()}\leavevmode
\begin{lstlisting}[language=Python, caption=Discretization Summary Method]
def summary(self) -> str
\end{lstlisting}

\textbf{Returns:} \texttt{str} - Multi-line summary of discretization properties

\textbf{Usage:}
\begin{lstlisting}[language=Python, caption=Summary Usage]
disc = Discretization(n_elements=20, domain_start=0.0, domain_length=2.0)
disc.set_tau([1.0, 0.5])
print(disc.summary())
# Output:
# Discretization Summary:
#   Domain: [0.000, 2.000] (length: 2.000)
#   Elements: 20, Nodes: 21
#   Element size: 0.100
#   Stabilization: tau = [1.0, 0.5]
#   Time step: dt = 0.010 (if set)
\end{lstlisting}

\subsection{GlobalDiscretization Class}
\label{subsec:globaldiscretization_class}

Coordinator class for managing multiple discretizations and time stepping parameters.

\subsubsection{Constructor}

\paragraph{\_\_init\_\_()}
\begin{lstlisting}[language=Python, caption=GlobalDiscretization Constructor]
def __init__(self, spatial_discretizations: List[Discretization])
\end{lstlisting}

\textbf{Parameters:}
\begin{itemize}
    \item \texttt{spatial\_discretizations}: List of Discretization instances for each domain
\end{itemize}

\textbf{Usage:}
\begin{lstlisting}[language=Python, caption=GlobalDiscretization Usage]
# Single domain
disc1 = Discretization(n_elements=20, domain_start=0.0, domain_length=1.0)
global_disc = GlobalDiscretization([disc1])

# Multi-domain (Y-junction example)
main_disc = Discretization(n_elements=30, domain_start=0.0, domain_length=1.0)
branch1_disc = Discretization(n_elements=20, domain_start=1.0, domain_length=0.8)
branch2_disc = Discretization(n_elements=20, domain_start=1.0, domain_length=0.8)

multi_global_disc = GlobalDiscretization([main_disc, branch1_disc, branch2_disc])
\end{lstlisting}

\subsubsection{Core Attributes}

\begin{longtable}{|p{3.5cm}|p{2.5cm}|p{7cm}|}
\hline
\textbf{Attribute} & \textbf{Type} & \textbf{Description} \\
\hline
\endhead

\texttt{spatial\_discretizations} & \texttt{List[Discretization]} & List of spatial discretizations \\
\hline

\texttt{n\_domains} & \texttt{int} & Number of domains \\
\hline

\texttt{dt} & \texttt{Optional[float]} & Global time step size \\
\hline

\texttt{T} & \texttt{Optional[float]} & Final time \\
\hline

\texttt{n\_time\_steps} & \texttt{Optional[int]} & Number of time steps: \texttt{int(T/dt)} \\
\hline

\texttt{total\_elements} & \texttt{int} & Sum of elements across all domains \\
\hline

\texttt{total\_nodes} & \texttt{int} & Sum of nodes across all domains \\
\hline

\end{longtable}

\subsubsection{Time Parameter Management}

\paragraph{set\_time\_parameters()}
\begin{lstlisting}[language=Python, caption=Set Time Parameters Method]
def set_time_parameters(self, dt: float, T: float)
\end{lstlisting}

\textbf{Parameters:}
\begin{itemize}
    \item \texttt{dt}: Time step size (corresponding to MATLAB \texttt{discretization.dt})
    \item \texttt{T}: Final time
\end{itemize}

\textbf{Side Effects:} 
\begin{itemize}
    \item Sets \texttt{self.dt}, \texttt{self.T}, and \texttt{self.n\_time\_steps}
    \item Propagates \texttt{dt} to all spatial discretizations
\end{itemize}

\textbf{Usage:}
\begin{lstlisting}[language=Python, caption=Time Parameters Usage]
# Set time parameters (matching MATLAB TestProblem.m patterns)
global_disc.set_time_parameters(dt=0.01, T=1.0)
print(f"Time step: {global_disc.dt}")
print(f"Final time: {global_disc.T}")
print(f"Number of time steps: {global_disc.n_time_steps}")

# Verify propagation to spatial discretizations
for i, disc in enumerate(global_disc.spatial_discretizations):
    print(f"Domain {i} dt: {disc.dt}")
\end{lstlisting}

\paragraph{get\_time\_points()}
\begin{lstlisting}[language=Python, caption=Get Time Points Method]
def get_time_points(self) -> np.ndarray
\end{lstlisting}

\textbf{Returns:} \texttt{np.ndarray} - Array of time points from 0 to T

\textbf{Usage:}
\begin{lstlisting}[language=Python, caption=Time Points Usage]
time_points = global_disc.get_time_points()
print(f"Time points: {time_points[:5]}...")  # First 5 time points
print(f"Total time points: {len(time_points)}")
\end{lstlisting}

\subsubsection{Domain Access Methods}

\paragraph{get\_discretization()}
\begin{lstlisting}[language=Python, caption=Get Discretization Method]
def get_discretization(self, domain_idx: int) -> Discretization
\end{lstlisting}

\textbf{Parameters:}
\begin{itemize}
    \item \texttt{domain\_idx}: Domain index (0 to \texttt{n\_domains-1})
\end{itemize}

\textbf{Returns:} \texttt{Discretization} - Spatial discretization for specified domain

\textbf{Usage:}
\begin{lstlisting}[language=Python, caption=Get Discretization Usage]
domain_0_disc = global_disc.get_discretization(0)
print(f"Domain 0 elements: {domain_0_disc.n_elements}")
print(f"Domain 0 nodes: {domain_0_disc.n_nodes}")
\end{lstlisting}

\paragraph{get\_all\_discretizations()}
\begin{lstlisting}[language=Python, caption=Get All Discretizations Method]
def get_all_discretizations(self) -> List[Discretization]
\end{lstlisting}

\textbf{Returns:} \texttt{List[Discretization]} - Copy of all spatial discretizations

\textbf{Usage:}
\begin{lstlisting}[language=Python, caption=Get All Discretizations Usage]
all_disc = global_disc.get_all_discretizations()
for i, disc in enumerate(all_disc):
    print(f"Domain {i}: {disc.n_elements} elements")
\end{lstlisting}

\subsubsection{Global Statistics Methods}

\paragraph{compute\_global\_statistics()}
\begin{lstlisting}[language=Python, caption=Compute Global Statistics Method]
def compute_global_statistics(self) -> dict
\end{lstlisting}

\textbf{Returns:} \texttt{dict} - Dictionary with global discretization statistics

\textbf{Usage:}
\begin{lstlisting}[language=Python, caption=Global Statistics Usage]
stats = global_disc.compute_global_statistics()
print(f"Total elements: {stats['total_elements']}")
print(f"Total nodes: {stats['total_nodes']}")
print(f"Average element size: {stats['avg_element_size']:.6f}")
print(f"Min element size: {stats['min_element_size']:.6f}")
print(f"Max element size: {stats['max_element_size']:.6f}")
\end{lstlisting}

\paragraph{get\_cfl\_number()}
\begin{lstlisting}[language=Python, caption=Get CFL Number Method]
def get_cfl_number(self, wave_speed: float = 1.0) -> float
\end{lstlisting}

\textbf{Parameters:}
\begin{itemize}
    \item \texttt{wave\_speed}: Characteristic wave speed for CFL calculation (default: 1.0)
\end{itemize}

\textbf{Returns:} \texttt{float} - CFL number: \texttt{wave\_speed * dt / min\_element\_size}

\textbf{Usage:}
\begin{lstlisting}[language=Python, caption=CFL Number Usage]
cfl = global_disc.get_cfl_number(wave_speed=1.0)
print(f"CFL number: {cfl:.6f}")

# Check stability condition
if cfl <= 1.0:
    print("✓ CFL condition satisfied")
else:
    print("⚠ CFL condition violated - consider smaller dt or larger elements")
\end{lstlisting}

\subsection{Integration with BioNetFlux Components}
\label{subsec:discretization_integration}

\subsubsection{Integration with Problem Class}

\begin{lstlisting}[language=Python, caption=Problem-Discretization Integration]
from ooc1d.core.problem import Problem

# Create problem and matching discretization
problem = Problem(
    neq=4,
    domain_start=0.0,
    domain_length=1.0,
    parameters=np.array([1.0, 2.0, 1.0, 1.0, 0.0, 1.0, 0.0, 1.0, 1.0]),
    problem_type="organ_on_chip"
)

# Create matching discretization
discretization = Discretization(
    n_elements=40,
    domain_start=problem.domain_start,
    domain_length=problem.domain_length,
    stab_constant=1.0
)

# Set stabilization parameters for 4-equation system
discretization.set_tau([1.0, 1.0, 1.0, 1.0])  # [tu, to, tv, tp]

# Create global discretization with time parameters
global_disc = GlobalDiscretization([discretization])
global_disc.set_time_parameters(dt=0.01, T=1.0)
\end{lstlisting}

\subsubsection{Integration with Static Condensation}

\begin{lstlisting}[language=Python, caption=Static Condensation Integration]
from ooc1d.core.static_condensation_ooc import StaticCondensationOOC
from ooc1d.utils.elementary_matrices import ElementaryMatrices

# Create static condensation using discretization
elementary_matrices = ElementaryMatrices()
static_condensation = StaticCondensationOOC(
    problem=problem,
    discretization=discretization,
    elementary_matrices=elementary_matrices
)

# Verify discretization parameters are accessible
print(f"Element length: {discretization.element_length}")
print(f"Time step: {discretization.dt}")
print(f"Stabilization parameters: {discretization.tau}")
\end{lstlisting}

\subsection{Complete Usage Examples}
\label{subsec:discretization_complete_examples}

\subsubsection{Single Domain OrganOnChip Setup}

\begin{lstlisting}[language=Python, caption=Complete Single Domain Setup]
def create_ooc_discretization(n_elements: int = 40,
                             domain_length: float = 1.0,
                             dt: float = 0.01,
                             T: float = 1.0) -> GlobalDiscretization:
    """Create discretization for OrganOnChip problem matching MATLAB TestProblem.m."""
    
    # Create spatial discretization
    spatial_disc = Discretization(
        n_elements=n_elements,
        domain_start=0.0,  # MATLAB: A = 0
        domain_length=domain_length,  # MATLAB: L = 1
        stab_constant=1.0
    )
    
    # Set stabilization parameters for 4-equation OrganOnChip system
    # Based on MATLAB scBlocks.m: tu, to, tv, tp
    spatial_disc.set_tau([1.0, 1.0, 1.0, 1.0])
    
    # Create global discretization
    global_disc = GlobalDiscretization([spatial_disc])
    
    # Set time parameters (matching MATLAB discretization.dt)
    global_disc.set_time_parameters(dt=dt, T=T)
    
    return global_disc

# Usage
ooc_global_disc = create_ooc_discretization()
print(ooc_global_disc.spatial_discretizations[0].summary())
\end{lstlisting}

\subsubsection{Multi-Domain Network Setup}

\begin{lstlisting}[language=Python, caption=Multi-Domain Network Discretization]
def create_network_discretization(domain_specs: List[dict],
                                 elements_per_domain: int = 20,
                                 dt: float = 0.01,
                                 T: float = 0.5) -> GlobalDiscretization:
    """Create discretization for multi-domain network."""
    
    spatial_discretizations = []
    
    for i, spec in enumerate(domain_specs):
        disc = Discretization(
            n_elements=elements_per_domain,
            domain_start=spec['start'],
            domain_length=spec['length'],
            stab_constant=1.0
        )
        
        # Set stabilization parameters based on problem type
        if 'problem_type' in spec and spec['problem_type'] == 'organ_on_chip':
            disc.set_tau([1.0, 1.0, 1.0, 1.0])  # 4 equations
        else:
            disc.set_tau([1.0, 1.0])  # 2 equations (Keller-Segel)
        
        spatial_discretizations.append(disc)
    
    # Create global discretization
    global_disc = GlobalDiscretization(spatial_discretizations)
    global_disc.set_time_parameters(dt=dt, T=T)
    
    return global_disc

# Usage: Y-junction network
domain_specs = [
    {'start': 0.0, 'length': 1.0, 'problem_type': 'organ_on_chip'},  # Main
    {'start': 1.0, 'length': 0.8, 'problem_type': 'organ_on_chip'},  # Branch 1
    {'start': 1.0, 'length': 0.8, 'problem_type': 'organ_on_chip'}   # Branch 2
]

network_global_disc = create_network_discretization(domain_specs)
print(f"Network has {network_global_disc.n_domains} domains")
print(f"Total elements: {network_global_disc.total_elements}")
\end{lstlisting}

\subsection{Method Summary Table}
\label{subsec:discretization_method_summary}

\subsubsection{Discretization Class Methods}

\begin{longtable}{|p{4cm}|p{2cm}|p{7cm}|}
\hline
\textbf{Method} & \textbf{Returns} & \textbf{Purpose} \\
\hline
\endhead

\texttt{generate\_nodes} & \texttt{np.ndarray} & Generate uniform node coordinates \\
\hline

\texttt{generate\_elements} & \texttt{List[Tuple]} & Generate element connectivity \\
\hline

\texttt{set\_tau} & \texttt{None} & Set stabilization parameters \\
\hline

\texttt{get\_tau} & \texttt{float/List} & Retrieve stabilization parameters \\
\hline

\texttt{get\_element\_center} & \texttt{float} & Get element center coordinate \\
\hline

\texttt{get\_element\_bounds} & \texttt{Tuple} & Get element start/end coordinates \\
\hline

\texttt{find\_element\_containing\_point} & \texttt{int} & Locate element containing given point \\
\hline

\texttt{refine\_mesh} & \texttt{Discretization} & Create refined discretization \\
\hline

\texttt{summary} & \texttt{str} & Generate discretization summary \\
\hline

\end{longtable}

\subsubsection{GlobalDiscretization Class Methods}

\begin{longtable}{|p{4cm}|p{2cm}|p{7cm}|}
\hline
\textbf{Method} & \textbf{Returns} & \textbf{Purpose} \\
\hline
\endhead

\texttt{set\_time\_parameters} & \texttt{None} & Set global time step and final time \\
\hline

\texttt{get\_time\_points} & \texttt{np.ndarray} & Generate array of time points \\
\hline

\texttt{get\_discretization} & \texttt{Discretization} & Access specific domain discretization \\
\hline

\texttt{get\_all\_discretizations} & \texttt{List} & Get all spatial discretizations \\
\hline

\texttt{compute\_global\_statistics} & \texttt{dict} & Calculate global mesh statistics \\
\hline

\texttt{get\_cfl\_number} & \texttt{float} & Compute CFL number for stability \\
\hline

\end{longtable}

This documentation provides an exact reference for the Discretization module based on the BioNetFlux implementation patterns and MATLAB reference files, with practical usage examples for both single and multi-domain scenarios.

% End of discretization module API documentation


\section{Discretization Module API Reference}
\label{sec:discretization_module_api}

This section provides a comprehensive reference for the Discretization classes (\texttt{ooc1d.core.discretization}) based on analysis of the BioNetFlux implementation patterns and MATLAB reference files. The module contains classes for spatial and temporal discretization management.

\subsection{Module Overview}

The discretization module contains two main classes:
\begin{itemize}
    \item \texttt{Discretization}: Single-domain spatial discretization with stabilization parameters
    \item \texttt{GlobalDiscretization}: Multi-domain coordinator with time stepping parameters
\end{itemize}

\subsection{Module Dependencies}

\begin{lstlisting}[language=Python, caption=Module Dependencies]
import numpy as np
from typing import List, Optional, Union, Tuple
\end{lstlisting}

\subsection{Discretization Class}
\label{subsec:discretization_class}

The main class for single-domain spatial discretization, handling mesh generation and stabilization parameters for HDG methods.

\subsubsection{Constructor}

\paragraph{\_\_init\_\_()}
\begin{lstlisting}[language=Python, caption=Discretization Constructor]
def __init__(self, 
             n_elements: int,
             domain_start: float = 0.0,
             domain_length: float = 1.0,
             stab_constant: float = 1.0)
\end{lstlisting}

\textbf{Parameters:}
\begin{itemize}
    \item \texttt{n\_elements}: Number of elements in the mesh
    \item \texttt{domain\_start}: Domain start coordinate (default: 0.0)
    \item \texttt{domain\_length}: Domain length (default: 1.0)
    \item \texttt{stab\_constant}: Stabilization constant for HDG method (default: 1.0)
\end{itemize}

\textbf{Usage Examples:}
\begin{lstlisting}[language=Python, caption=Discretization Constructor Usage]
# Basic discretization
disc1 = Discretization(n_elements=20)

# Custom domain discretization (matching MATLAB TestProblem.m)
disc2 = Discretization(
    n_elements=40,
    domain_start=0.0,  # MATLAB: A = 0
    domain_length=1.0, # MATLAB: L = 1
    stab_constant=1.0
)

# Fine mesh discretization
disc3 = Discretization(
    n_elements=100,
    domain_start=-1.0,
    domain_length=2.0,
    stab_constant=0.5
)
\end{lstlisting}

\subsubsection{Core Attributes}

\begin{longtable}{|p{3.5cm}|p{2.5cm}|p{7cm}|}
\hline
\textbf{Attribute} & \textbf{Type} & \textbf{Description} \\
\hline
\endhead

\texttt{n\_elements} & \texttt{int} & Number of elements in the mesh \\
\hline

\texttt{domain\_start} & \texttt{float} & Start coordinate of the domain \\
\hline

\texttt{domain\_length} & \texttt{float} & Length of the domain \\
\hline

\texttt{domain\_end} & \texttt{float} & End coordinate: \texttt{domain\_start + domain\_length} \\
\hline

\texttt{stab\_constant} & \texttt{float} & HDG stabilization constant \\
\hline

\texttt{n\_nodes} & \texttt{int} & Number of nodes: \texttt{n\_elements + 1} \\
\hline

\texttt{element\_length} & \texttt{float} & Uniform element size: \texttt{domain\_length / n\_elements} \\
\hline

\texttt{nodes} & \texttt{np.ndarray} & Node coordinates array (length \texttt{n\_nodes}) \\
\hline

\texttt{elements} & \texttt{List[Tuple]} & Element connectivity: \texttt{[(i, i+1) for i in range(n\_elements)]} \\
\hline

\texttt{tau} & \texttt{List[float]} & Stabilization parameters per equation \\
\hline

\texttt{dt} & \texttt{Optional[float]} & Time step size (set by GlobalDiscretization) \\
\hline

\end{longtable}

\subsubsection{Mesh Generation Methods}

\paragraph{generate\_nodes()}
\begin{lstlisting}[language=Python, caption=Generate Nodes Method]
def generate_nodes(self) -> np.ndarray
\end{lstlisting}

\textbf{Returns:} \texttt{np.ndarray} - Array of node coordinates

\textbf{Implementation:} \texttt{np.linspace(domain\_start, domain\_start + domain\_length, n\_nodes)}

\textbf{Usage:}
\begin{lstlisting}[language=Python, caption=Node Generation Usage]
disc = Discretization(n_elements=10, domain_start=0.0, domain_length=2.0)
nodes = disc.generate_nodes()
# nodes = [0.0, 0.2, 0.4, 0.6, 0.8, 1.0, 1.2, 1.4, 1.6, 1.8, 2.0]
print(f"Node coordinates: {nodes}")
print(f"Number of nodes: {len(nodes)}")
\end{lstlisting}

\paragraph{generate\_elements()}
\begin{lstlisting}[language=Python, caption=Generate Elements Method]
def generate_elements(self) -> List[Tuple[int, int]]
\end{lstlisting}

\textbf{Returns:} \texttt{List[Tuple[int, int]]} - Element connectivity list

\textbf{Implementation:} \texttt{[(i, i+1) for i in range(n\_elements)]}

\textbf{Usage:}
\begin{lstlisting}[language=Python, caption=Element Generation Usage]
disc = Discretization(n_elements=4)
elements = disc.generate_elements()
# elements = [(0, 1), (1, 2), (2, 3), (3, 4)]
for i, (node1, node2) in enumerate(elements):
    print(f"Element {i}: nodes {node1} -> {node2}")
\end{lstlisting}

\subsubsection{Stabilization Parameter Management}

\paragraph{set\_tau()}
\begin{lstlisting}[language=Python, caption=Set Tau Method]
def set_tau(self, tau_values: Union[float, List[float]])
\end{lstlisting}

\textbf{Parameters:}
\begin{itemize}
    \item \texttt{tau\_values}: Single value or list of stabilization parameters per equation
\end{itemize}

\textbf{Side Effects:} Sets \texttt{self.tau} attribute

\textbf{Usage (Based on MATLAB scBlocks.m):}
\begin{lstlisting}[language=Python, caption=Tau Parameter Usage]
# Keller-Segel problem (2 equations)
ks_disc = Discretization(n_elements=20)
ks_disc.set_tau([1.0, 1.0])  # [tau_u, tau_phi]

# OrganOnChip problem (4 equations) - from MATLAB TestProblem.m
ooc_disc = Discretization(n_elements=40)
ooc_disc.set_tau([1.0, 1.0, 1.0, 1.0])  # [tu, to, tv, tp]

# Single tau for all equations
simple_disc = Discretization(n_elements=10)
simple_disc.set_tau(0.5)  # Applied to all equations
\end{lstlisting}

\paragraph{get\_tau()}
\begin{lstlisting}[language=Python, caption=Get Tau Method]
def get_tau(self, equation_idx: Optional[int] = None) -> Union[float, List[float]]
\end{lstlisting}

\textbf{Parameters:}
\begin{itemize}
    \item \texttt{equation\_idx}: Equation index (optional, returns all if None)
\end{itemize}

\textbf{Returns:} \texttt{float} or \texttt{List[float]} - Stabilization parameter(s)

\textbf{Usage:}
\begin{lstlisting}[language=Python, caption=Get Tau Usage]
# Get all tau values
all_tau = disc.get_tau()
print(f"All tau values: {all_tau}")

# Get specific tau value  
tau_u = disc.get_tau(0)  # First equation
tau_phi = disc.get_tau(1)  # Second equation
\end{lstlisting}

\subsubsection{Geometric Query Methods}

\paragraph{get\_element\_center()}
\begin{lstlisting}[language=Python, caption=Get Element Center Method]
def get_element_center(self, element_idx: int) -> float
\end{lstlisting}

\textbf{Parameters:}
\begin{itemize}
    \item \texttt{element\_idx}: Element index (0 to \texttt{n\_elements-1})
\end{itemize}

\textbf{Returns:} \texttt{float} - Center coordinate of the element

\textbf{Usage:}
\begin{lstlisting}[language=Python, caption=Element Center Usage]
disc = Discretization(n_elements=10, domain_start=0.0, domain_length=1.0)
center_0 = disc.get_element_center(0)  # First element center
center_5 = disc.get_element_center(5)  # Sixth element center
print(f"Element 0 center: {center_0}")
print(f"Element 5 center: {center_5}")
\end{lstlisting}

\paragraph{get\_element\_bounds()}\leavevmode
\begin{lstlisting}[language=Python, caption=Get Element Bounds Method]
def get_element_bounds(self, element_idx: int) -> Tuple[float, float]
\end{lstlisting}

\textbf{Parameters:}
\begin{itemize}
    \item \texttt{element\_idx}: Element index
\end{itemize}

\textbf{Returns:} \texttt{Tuple[float, float]} - Element start and end coordinates

\textbf{Usage:}
\begin{lstlisting}[language=Python, caption=Element Bounds Usage]
start, end = disc.get_element_bounds(3)  # Fourth element
print(f"Element 3 bounds: [{start:.3f}, {end:.3f}]")
element_length = end - start
\end{lstlisting}

\paragraph{find\_element\_containing\_point()}\leavevmode
[Yet to be implemented]
\begin{lstlisting}[language=Python, caption=Find Element Method]
def find_element_containing_point(self, point: float) -> int
\end{lstlisting}

\textbf{Parameters:}
\begin{itemize}
    \item \texttt{point}: Coordinate to locate
\end{itemize}

\textbf{Returns:} \texttt{int} - Element index containing the point (-1 if outside domain)

\textbf{Usage:}
\begin{lstlisting}[language=Python, caption=Find Element Usage]
disc = Discretization(n_elements=10, domain_start=0.0, domain_length=1.0)
element_idx = disc.find_element_containing_point(0.35)
print(f"Point 0.35 is in element {element_idx}")
\end{lstlisting}

\subsubsection{Utility Methods}

\paragraph{refine\_mesh()}\leavevmode
[Yet to be implemented]
\begin{lstlisting}[language=Python, caption=Refine Mesh Method]
def refine_mesh(self, refinement_factor: int = 2) -> 'Discretization'
\end{lstlisting}

\textbf{Parameters:}
\begin{itemize}
    \item \texttt{refinement\_factor}: Factor by which to increase element count (default: 2)
\end{itemize}

\textbf{Returns:} \texttt{Discretization} - New discretization with refined mesh

\textbf{Usage:}
\begin{lstlisting}[language=Python, caption=Mesh Refinement Usage]
coarse_disc = Discretization(n_elements=10)
fine_disc = coarse_disc.refine_mesh(refinement_factor=2)
print(f"Original elements: {coarse_disc.n_elements}")
print(f"Refined elements: {fine_disc.n_elements}")  # Should be 20
\end{lstlisting}

\paragraph{summary()}\leavevmode
\begin{lstlisting}[language=Python, caption=Discretization Summary Method]
def summary(self) -> str
\end{lstlisting}

\textbf{Returns:} \texttt{str} - Multi-line summary of discretization properties

\textbf{Usage:}
\begin{lstlisting}[language=Python, caption=Summary Usage]
disc = Discretization(n_elements=20, domain_start=0.0, domain_length=2.0)
disc.set_tau([1.0, 0.5])
print(disc.summary())
# Output:
# Discretization Summary:
#   Domain: [0.000, 2.000] (length: 2.000)
#   Elements: 20, Nodes: 21
#   Element size: 0.100
#   Stabilization: tau = [1.0, 0.5]
#   Time step: dt = 0.010 (if set)
\end{lstlisting}

\subsection{GlobalDiscretization Class}
\label{subsec:globaldiscretization_class}

Coordinator class for managing multiple discretizations and time stepping parameters.

\subsubsection{Constructor}

\paragraph{\_\_init\_\_()}
\begin{lstlisting}[language=Python, caption=GlobalDiscretization Constructor]
def __init__(self, spatial_discretizations: List[Discretization])
\end{lstlisting}

\textbf{Parameters:}
\begin{itemize}
    \item \texttt{spatial\_discretizations}: List of Discretization instances for each domain
\end{itemize}

\textbf{Usage:}
\begin{lstlisting}[language=Python, caption=GlobalDiscretization Usage]
# Single domain
disc1 = Discretization(n_elements=20, domain_start=0.0, domain_length=1.0)
global_disc = GlobalDiscretization([disc1])

# Multi-domain (Y-junction example)
main_disc = Discretization(n_elements=30, domain_start=0.0, domain_length=1.0)
branch1_disc = Discretization(n_elements=20, domain_start=1.0, domain_length=0.8)
branch2_disc = Discretization(n_elements=20, domain_start=1.0, domain_length=0.8)

multi_global_disc = GlobalDiscretization([main_disc, branch1_disc, branch2_disc])
\end{lstlisting}

\subsubsection{Core Attributes}

\begin{longtable}{|p{3.5cm}|p{2.5cm}|p{7cm}|}
\hline
\textbf{Attribute} & \textbf{Type} & \textbf{Description} \\
\hline
\endhead

\texttt{spatial\_discretizations} & \texttt{List[Discretization]} & List of spatial discretizations \\
\hline

\texttt{n\_domains} & \texttt{int} & Number of domains \\
\hline

\texttt{dt} & \texttt{Optional[float]} & Global time step size \\
\hline

\texttt{T} & \texttt{Optional[float]} & Final time \\
\hline

\texttt{n\_time\_steps} & \texttt{Optional[int]} & Number of time steps: \texttt{int(T/dt)} \\
\hline

\texttt{total\_elements} & \texttt{int} & Sum of elements across all domains \\
\hline

\texttt{total\_nodes} & \texttt{int} & Sum of nodes across all domains \\
\hline

\end{longtable}

\subsubsection{Time Parameter Management}

\paragraph{set\_time\_parameters()}
\begin{lstlisting}[language=Python, caption=Set Time Parameters Method]
def set_time_parameters(self, dt: float, T: float)
\end{lstlisting}

\textbf{Parameters:}
\begin{itemize}
    \item \texttt{dt}: Time step size (corresponding to MATLAB \texttt{discretization.dt})
    \item \texttt{T}: Final time
\end{itemize}

\textbf{Side Effects:} 
\begin{itemize}
    \item Sets \texttt{self.dt}, \texttt{self.T}, and \texttt{self.n\_time\_steps}
    \item Propagates \texttt{dt} to all spatial discretizations
\end{itemize}

\textbf{Usage:}
\begin{lstlisting}[language=Python, caption=Time Parameters Usage]
# Set time parameters (matching MATLAB TestProblem.m patterns)
global_disc.set_time_parameters(dt=0.01, T=1.0)
print(f"Time step: {global_disc.dt}")
print(f"Final time: {global_disc.T}")
print(f"Number of time steps: {global_disc.n_time_steps}")

# Verify propagation to spatial discretizations
for i, disc in enumerate(global_disc.spatial_discretizations):
    print(f"Domain {i} dt: {disc.dt}")
\end{lstlisting}

\paragraph{get\_time\_points()}
\begin{lstlisting}[language=Python, caption=Get Time Points Method]
def get_time_points(self) -> np.ndarray
\end{lstlisting}

\textbf{Returns:} \texttt{np.ndarray} - Array of time points from 0 to T

\textbf{Usage:}
\begin{lstlisting}[language=Python, caption=Time Points Usage]
time_points = global_disc.get_time_points()
print(f"Time points: {time_points[:5]}...")  # First 5 time points
print(f"Total time points: {len(time_points)}")
\end{lstlisting}

\subsubsection{Domain Access Methods}

\paragraph{get\_discretization()}
\begin{lstlisting}[language=Python, caption=Get Discretization Method]
def get_discretization(self, domain_idx: int) -> Discretization
\end{lstlisting}

\textbf{Parameters:}
\begin{itemize}
    \item \texttt{domain\_idx}: Domain index (0 to \texttt{n\_domains-1})
\end{itemize}

\textbf{Returns:} \texttt{Discretization} - Spatial discretization for specified domain

\textbf{Usage:}
\begin{lstlisting}[language=Python, caption=Get Discretization Usage]
domain_0_disc = global_disc.get_discretization(0)
print(f"Domain 0 elements: {domain_0_disc.n_elements}")
print(f"Domain 0 nodes: {domain_0_disc.n_nodes}")
\end{lstlisting}

\paragraph{get\_all\_discretizations()}
\begin{lstlisting}[language=Python, caption=Get All Discretizations Method]
def get_all_discretizations(self) -> List[Discretization]
\end{lstlisting}

\textbf{Returns:} \texttt{List[Discretization]} - Copy of all spatial discretizations

\textbf{Usage:}
\begin{lstlisting}[language=Python, caption=Get All Discretizations Usage]
all_disc = global_disc.get_all_discretizations()
for i, disc in enumerate(all_disc):
    print(f"Domain {i}: {disc.n_elements} elements")
\end{lstlisting}

\subsubsection{Global Statistics Methods}

\paragraph{compute\_global\_statistics()}
\begin{lstlisting}[language=Python, caption=Compute Global Statistics Method]
def compute_global_statistics(self) -> dict
\end{lstlisting}

\textbf{Returns:} \texttt{dict} - Dictionary with global discretization statistics

\textbf{Usage:}
\begin{lstlisting}[language=Python, caption=Global Statistics Usage]
stats = global_disc.compute_global_statistics()
print(f"Total elements: {stats['total_elements']}")
print(f"Total nodes: {stats['total_nodes']}")
print(f"Average element size: {stats['avg_element_size']:.6f}")
print(f"Min element size: {stats['min_element_size']:.6f}")
print(f"Max element size: {stats['max_element_size']:.6f}")
\end{lstlisting}

\paragraph{get\_cfl\_number()}
\begin{lstlisting}[language=Python, caption=Get CFL Number Method]
def get_cfl_number(self, wave_speed: float = 1.0) -> float
\end{lstlisting}

\textbf{Parameters:}
\begin{itemize}
    \item \texttt{wave\_speed}: Characteristic wave speed for CFL calculation (default: 1.0)
\end{itemize}

\textbf{Returns:} \texttt{float} - CFL number: \texttt{wave\_speed * dt / min\_element\_size}

\textbf{Usage:}
\begin{lstlisting}[language=Python, caption=CFL Number Usage]
cfl = global_disc.get_cfl_number(wave_speed=1.0)
print(f"CFL number: {cfl:.6f}")

# Check stability condition
if cfl <= 1.0:
    print("✓ CFL condition satisfied")
else:
    print("⚠ CFL condition violated - consider smaller dt or larger elements")
\end{lstlisting}

\subsection{Integration with BioNetFlux Components}
\label{subsec:discretization_integration}

\subsubsection{Integration with Problem Class}

\begin{lstlisting}[language=Python, caption=Problem-Discretization Integration]
from ooc1d.core.problem import Problem

# Create problem and matching discretization
problem = Problem(
    neq=4,
    domain_start=0.0,
    domain_length=1.0,
    parameters=np.array([1.0, 2.0, 1.0, 1.0, 0.0, 1.0, 0.0, 1.0, 1.0]),
    problem_type="organ_on_chip"
)

# Create matching discretization
discretization = Discretization(
    n_elements=40,
    domain_start=problem.domain_start,
    domain_length=problem.domain_length,
    stab_constant=1.0
)

# Set stabilization parameters for 4-equation system
discretization.set_tau([1.0, 1.0, 1.0, 1.0])  # [tu, to, tv, tp]

# Create global discretization with time parameters
global_disc = GlobalDiscretization([discretization])
global_disc.set_time_parameters(dt=0.01, T=1.0)
\end{lstlisting}

\subsubsection{Integration with Static Condensation}

\begin{lstlisting}[language=Python, caption=Static Condensation Integration]
from ooc1d.core.static_condensation_ooc import StaticCondensationOOC
from ooc1d.utils.elementary_matrices import ElementaryMatrices

# Create static condensation using discretization
elementary_matrices = ElementaryMatrices()
static_condensation = StaticCondensationOOC(
    problem=problem,
    discretization=discretization,
    elementary_matrices=elementary_matrices
)

# Verify discretization parameters are accessible
print(f"Element length: {discretization.element_length}")
print(f"Time step: {discretization.dt}")
print(f"Stabilization parameters: {discretization.tau}")
\end{lstlisting}

\subsection{Complete Usage Examples}
\label{subsec:discretization_complete_examples}

\subsubsection{Single Domain OrganOnChip Setup}

\begin{lstlisting}[language=Python, caption=Complete Single Domain Setup]
def create_ooc_discretization(n_elements: int = 40,
                             domain_length: float = 1.0,
                             dt: float = 0.01,
                             T: float = 1.0) -> GlobalDiscretization:
    """Create discretization for OrganOnChip problem matching MATLAB TestProblem.m."""
    
    # Create spatial discretization
    spatial_disc = Discretization(
        n_elements=n_elements,
        domain_start=0.0,  # MATLAB: A = 0
        domain_length=domain_length,  # MATLAB: L = 1
        stab_constant=1.0
    )
    
    # Set stabilization parameters for 4-equation OrganOnChip system
    # Based on MATLAB scBlocks.m: tu, to, tv, tp
    spatial_disc.set_tau([1.0, 1.0, 1.0, 1.0])
    
    # Create global discretization
    global_disc = GlobalDiscretization([spatial_disc])
    
    # Set time parameters (matching MATLAB discretization.dt)
    global_disc.set_time_parameters(dt=dt, T=T)
    
    return global_disc

# Usage
ooc_global_disc = create_ooc_discretization()
print(ooc_global_disc.spatial_discretizations[0].summary())
\end{lstlisting}

\subsubsection{Multi-Domain Network Setup}

\begin{lstlisting}[language=Python, caption=Multi-Domain Network Discretization]
def create_network_discretization(domain_specs: List[dict],
                                 elements_per_domain: int = 20,
                                 dt: float = 0.01,
                                 T: float = 0.5) -> GlobalDiscretization:
    """Create discretization for multi-domain network."""
    
    spatial_discretizations = []
    
    for i, spec in enumerate(domain_specs):
        disc = Discretization(
            n_elements=elements_per_domain,
            domain_start=spec['start'],
            domain_length=spec['length'],
            stab_constant=1.0
        )
        
        # Set stabilization parameters based on problem type
        if 'problem_type' in spec and spec['problem_type'] == 'organ_on_chip':
            disc.set_tau([1.0, 1.0, 1.0, 1.0])  # 4 equations
        else:
            disc.set_tau([1.0, 1.0])  # 2 equations (Keller-Segel)
        
        spatial_discretizations.append(disc)
    
    # Create global discretization
    global_disc = GlobalDiscretization(spatial_discretizations)
    global_disc.set_time_parameters(dt=dt, T=T)
    
    return global_disc

# Usage: Y-junction network
domain_specs = [
    {'start': 0.0, 'length': 1.0, 'problem_type': 'organ_on_chip'},  # Main
    {'start': 1.0, 'length': 0.8, 'problem_type': 'organ_on_chip'},  # Branch 1
    {'start': 1.0, 'length': 0.8, 'problem_type': 'organ_on_chip'}   # Branch 2
]

network_global_disc = create_network_discretization(domain_specs)
print(f"Network has {network_global_disc.n_domains} domains")
print(f"Total elements: {network_global_disc.total_elements}")
\end{lstlisting}

\subsection{Method Summary Table}
\label{subsec:discretization_method_summary}

\subsubsection{Discretization Class Methods}

\begin{longtable}{|p{4cm}|p{2cm}|p{7cm}|}
\hline
\textbf{Method} & \textbf{Returns} & \textbf{Purpose} \\
\hline
\endhead

\texttt{generate\_nodes} & \texttt{np.ndarray} & Generate uniform node coordinates \\
\hline

\texttt{generate\_elements} & \texttt{List[Tuple]} & Generate element connectivity \\
\hline

\texttt{set\_tau} & \texttt{None} & Set stabilization parameters \\
\hline

\texttt{get\_tau} & \texttt{float/List} & Retrieve stabilization parameters \\
\hline

\texttt{get\_element\_center} & \texttt{float} & Get element center coordinate \\
\hline

\texttt{get\_element\_bounds} & \texttt{Tuple} & Get element start/end coordinates \\
\hline

\texttt{find\_element\_containing\_point} & \texttt{int} & Locate element containing given point \\
\hline

\texttt{refine\_mesh} & \texttt{Discretization} & Create refined discretization \\
\hline

\texttt{summary} & \texttt{str} & Generate discretization summary \\
\hline

\end{longtable}

\subsubsection{GlobalDiscretization Class Methods}

\begin{longtable}{|p{4cm}|p{2cm}|p{7cm}|}
\hline
\textbf{Method} & \textbf{Returns} & \textbf{Purpose} \\
\hline
\endhead

\texttt{set\_time\_parameters} & \texttt{None} & Set global time step and final time \\
\hline

\texttt{get\_time\_points} & \texttt{np.ndarray} & Generate array of time points \\
\hline

\texttt{get\_discretization} & \texttt{Discretization} & Access specific domain discretization \\
\hline

\texttt{get\_all\_discretizations} & \texttt{List} & Get all spatial discretizations \\
\hline

\texttt{compute\_global\_statistics} & \texttt{dict} & Calculate global mesh statistics \\
\hline

\texttt{get\_cfl\_number} & \texttt{float} & Compute CFL number for stability \\
\hline

\end{longtable}

This documentation provides an exact reference for the Discretization module based on the BioNetFlux implementation patterns and MATLAB reference files, with practical usage examples for both single and multi-domain scenarios.

% End of discretization module API documentation


\section{Discretization Module API Reference}
\label{sec:discretization_module_api}

This section provides a comprehensive reference for the Discretization classes (\texttt{ooc1d.core.discretization}) based on analysis of the BioNetFlux implementation patterns and MATLAB reference files. The module contains classes for spatial and temporal discretization management.

\subsection{Module Overview}

The discretization module contains two main classes:
\begin{itemize}
    \item \texttt{Discretization}: Single-domain spatial discretization with stabilization parameters
    \item \texttt{GlobalDiscretization}: Multi-domain coordinator with time stepping parameters
\end{itemize}

\subsection{Module Dependencies}

\begin{lstlisting}[language=Python, caption=Module Dependencies]
import numpy as np
from typing import List, Optional, Union, Tuple
\end{lstlisting}

\subsection{Discretization Class}
\label{subsec:discretization_class}

The main class for single-domain spatial discretization, handling mesh generation and stabilization parameters for HDG methods.

\subsubsection{Constructor}

\paragraph{\_\_init\_\_()}
\begin{lstlisting}[language=Python, caption=Discretization Constructor]
def __init__(self, 
             n_elements: int,
             domain_start: float = 0.0,
             domain_length: float = 1.0,
             stab_constant: float = 1.0)
\end{lstlisting}

\textbf{Parameters:}
\begin{itemize}
    \item \texttt{n\_elements}: Number of elements in the mesh
    \item \texttt{domain\_start}: Domain start coordinate (default: 0.0)
    \item \texttt{domain\_length}: Domain length (default: 1.0)
    \item \texttt{stab\_constant}: Stabilization constant for HDG method (default: 1.0)
\end{itemize}

\textbf{Usage Examples:}
\begin{lstlisting}[language=Python, caption=Discretization Constructor Usage]
# Basic discretization
disc1 = Discretization(n_elements=20)

# Custom domain discretization (matching MATLAB TestProblem.m)
disc2 = Discretization(
    n_elements=40,
    domain_start=0.0,  # MATLAB: A = 0
    domain_length=1.0, # MATLAB: L = 1
    stab_constant=1.0
)

# Fine mesh discretization
disc3 = Discretization(
    n_elements=100,
    domain_start=-1.0,
    domain_length=2.0,
    stab_constant=0.5
)
\end{lstlisting}

\subsubsection{Core Attributes}

\begin{longtable}{|p{3.5cm}|p{2.5cm}|p{7cm}|}
\hline
\textbf{Attribute} & \textbf{Type} & \textbf{Description} \\
\hline
\endhead

\texttt{n\_elements} & \texttt{int} & Number of elements in the mesh \\
\hline

\texttt{domain\_start} & \texttt{float} & Start coordinate of the domain \\
\hline

\texttt{domain\_length} & \texttt{float} & Length of the domain \\
\hline

\texttt{domain\_end} & \texttt{float} & End coordinate: \texttt{domain\_start + domain\_length} \\
\hline

\texttt{stab\_constant} & \texttt{float} & HDG stabilization constant \\
\hline

\texttt{n\_nodes} & \texttt{int} & Number of nodes: \texttt{n\_elements + 1} \\
\hline

\texttt{element\_length} & \texttt{float} & Uniform element size: \texttt{domain\_length / n\_elements} \\
\hline

\texttt{nodes} & \texttt{np.ndarray} & Node coordinates array (length \texttt{n\_nodes}) \\
\hline

\texttt{elements} & \texttt{List[Tuple]} & Element connectivity: \texttt{[(i, i+1) for i in range(n\_elements)]} \\
\hline

\texttt{tau} & \texttt{List[float]} & Stabilization parameters per equation \\
\hline

\texttt{dt} & \texttt{Optional[float]} & Time step size (set by GlobalDiscretization) \\
\hline

\end{longtable}

\subsubsection{Mesh Generation Methods}

\paragraph{generate\_nodes()}
\begin{lstlisting}[language=Python, caption=Generate Nodes Method]
def generate_nodes(self) -> np.ndarray
\end{lstlisting}

\textbf{Returns:} \texttt{np.ndarray} - Array of node coordinates

\textbf{Implementation:} \texttt{np.linspace(domain\_start, domain\_start + domain\_length, n\_nodes)}

\textbf{Usage:}
\begin{lstlisting}[language=Python, caption=Node Generation Usage]
disc = Discretization(n_elements=10, domain_start=0.0, domain_length=2.0)
nodes = disc.generate_nodes()
# nodes = [0.0, 0.2, 0.4, 0.6, 0.8, 1.0, 1.2, 1.4, 1.6, 1.8, 2.0]
print(f"Node coordinates: {nodes}")
print(f"Number of nodes: {len(nodes)}")
\end{lstlisting}

\paragraph{generate\_elements()}
\begin{lstlisting}[language=Python, caption=Generate Elements Method]
def generate_elements(self) -> List[Tuple[int, int]]
\end{lstlisting}

\textbf{Returns:} \texttt{List[Tuple[int, int]]} - Element connectivity list

\textbf{Implementation:} \texttt{[(i, i+1) for i in range(n\_elements)]}

\textbf{Usage:}
\begin{lstlisting}[language=Python, caption=Element Generation Usage]
disc = Discretization(n_elements=4)
elements = disc.generate_elements()
# elements = [(0, 1), (1, 2), (2, 3), (3, 4)]
for i, (node1, node2) in enumerate(elements):
    print(f"Element {i}: nodes {node1} -> {node2}")
\end{lstlisting}

\subsubsection{Stabilization Parameter Management}

\paragraph{set\_tau()}
\begin{lstlisting}[language=Python, caption=Set Tau Method]
def set_tau(self, tau_values: Union[float, List[float]])
\end{lstlisting}

\textbf{Parameters:}
\begin{itemize}
    \item \texttt{tau\_values}: Single value or list of stabilization parameters per equation
\end{itemize}

\textbf{Side Effects:} Sets \texttt{self.tau} attribute

\textbf{Usage (Based on MATLAB scBlocks.m):}
\begin{lstlisting}[language=Python, caption=Tau Parameter Usage]
# Keller-Segel problem (2 equations)
ks_disc = Discretization(n_elements=20)
ks_disc.set_tau([1.0, 1.0])  # [tau_u, tau_phi]

# OrganOnChip problem (4 equations) - from MATLAB TestProblem.m
ooc_disc = Discretization(n_elements=40)
ooc_disc.set_tau([1.0, 1.0, 1.0, 1.0])  # [tu, to, tv, tp]

# Single tau for all equations
simple_disc = Discretization(n_elements=10)
simple_disc.set_tau(0.5)  # Applied to all equations
\end{lstlisting}

\paragraph{get\_tau()}
\begin{lstlisting}[language=Python, caption=Get Tau Method]
def get_tau(self, equation_idx: Optional[int] = None) -> Union[float, List[float]]
\end{lstlisting}

\textbf{Parameters:}
\begin{itemize}
    \item \texttt{equation\_idx}: Equation index (optional, returns all if None)
\end{itemize}

\textbf{Returns:} \texttt{float} or \texttt{List[float]} - Stabilization parameter(s)

\textbf{Usage:}
\begin{lstlisting}[language=Python, caption=Get Tau Usage]
# Get all tau values
all_tau = disc.get_tau()
print(f"All tau values: {all_tau}")

# Get specific tau value  
tau_u = disc.get_tau(0)  # First equation
tau_phi = disc.get_tau(1)  # Second equation
\end{lstlisting}

\subsubsection{Geometric Query Methods}

\paragraph{get\_element\_center()}
\begin{lstlisting}[language=Python, caption=Get Element Center Method]
def get_element_center(self, element_idx: int) -> float
\end{lstlisting}

\textbf{Parameters:}
\begin{itemize}
    \item \texttt{element\_idx}: Element index (0 to \texttt{n\_elements-1})
\end{itemize}

\textbf{Returns:} \texttt{float} - Center coordinate of the element

\textbf{Usage:}
\begin{lstlisting}[language=Python, caption=Element Center Usage]
disc = Discretization(n_elements=10, domain_start=0.0, domain_length=1.0)
center_0 = disc.get_element_center(0)  # First element center
center_5 = disc.get_element_center(5)  # Sixth element center
print(f"Element 0 center: {center_0}")
print(f"Element 5 center: {center_5}")
\end{lstlisting}

\paragraph{get\_element\_bounds()}\leavevmode
\begin{lstlisting}[language=Python, caption=Get Element Bounds Method]
def get_element_bounds(self, element_idx: int) -> Tuple[float, float]
\end{lstlisting}

\textbf{Parameters:}
\begin{itemize}
    \item \texttt{element\_idx}: Element index
\end{itemize}

\textbf{Returns:} \texttt{Tuple[float, float]} - Element start and end coordinates

\textbf{Usage:}
\begin{lstlisting}[language=Python, caption=Element Bounds Usage]
start, end = disc.get_element_bounds(3)  # Fourth element
print(f"Element 3 bounds: [{start:.3f}, {end:.3f}]")
element_length = end - start
\end{lstlisting}

\paragraph{find\_element\_containing\_point()}\leavevmode
[Yet to be implemented]
\begin{lstlisting}[language=Python, caption=Find Element Method]
def find_element_containing_point(self, point: float) -> int
\end{lstlisting}

\textbf{Parameters:}
\begin{itemize}
    \item \texttt{point}: Coordinate to locate
\end{itemize}

\textbf{Returns:} \texttt{int} - Element index containing the point (-1 if outside domain)

\textbf{Usage:}
\begin{lstlisting}[language=Python, caption=Find Element Usage]
disc = Discretization(n_elements=10, domain_start=0.0, domain_length=1.0)
element_idx = disc.find_element_containing_point(0.35)
print(f"Point 0.35 is in element {element_idx}")
\end{lstlisting}

\subsubsection{Utility Methods}

\paragraph{refine\_mesh()}\leavevmode
[Yet to be implemented]
\begin{lstlisting}[language=Python, caption=Refine Mesh Method]
def refine_mesh(self, refinement_factor: int = 2) -> 'Discretization'
\end{lstlisting}

\textbf{Parameters:}
\begin{itemize}
    \item \texttt{refinement\_factor}: Factor by which to increase element count (default: 2)
\end{itemize}

\textbf{Returns:} \texttt{Discretization} - New discretization with refined mesh

\textbf{Usage:}
\begin{lstlisting}[language=Python, caption=Mesh Refinement Usage]
coarse_disc = Discretization(n_elements=10)
fine_disc = coarse_disc.refine_mesh(refinement_factor=2)
print(f"Original elements: {coarse_disc.n_elements}")
print(f"Refined elements: {fine_disc.n_elements}")  # Should be 20
\end{lstlisting}

\paragraph{summary()}\leavevmode
\begin{lstlisting}[language=Python, caption=Discretization Summary Method]
def summary(self) -> str
\end{lstlisting}

\textbf{Returns:} \texttt{str} - Multi-line summary of discretization properties

\textbf{Usage:}
\begin{lstlisting}[language=Python, caption=Summary Usage]
disc = Discretization(n_elements=20, domain_start=0.0, domain_length=2.0)
disc.set_tau([1.0, 0.5])
print(disc.summary())
# Output:
# Discretization Summary:
#   Domain: [0.000, 2.000] (length: 2.000)
#   Elements: 20, Nodes: 21
#   Element size: 0.100
#   Stabilization: tau = [1.0, 0.5]
#   Time step: dt = 0.010 (if set)
\end{lstlisting}

\subsection{GlobalDiscretization Class}
\label{subsec:globaldiscretization_class}

Coordinator class for managing multiple discretizations and time stepping parameters.

\subsubsection{Constructor}

\paragraph{\_\_init\_\_()}
\begin{lstlisting}[language=Python, caption=GlobalDiscretization Constructor]
def __init__(self, spatial_discretizations: List[Discretization])
\end{lstlisting}

\textbf{Parameters:}
\begin{itemize}
    \item \texttt{spatial\_discretizations}: List of Discretization instances for each domain
\end{itemize}

\textbf{Usage:}
\begin{lstlisting}[language=Python, caption=GlobalDiscretization Usage]
# Single domain
disc1 = Discretization(n_elements=20, domain_start=0.0, domain_length=1.0)
global_disc = GlobalDiscretization([disc1])

# Multi-domain (Y-junction example)
main_disc = Discretization(n_elements=30, domain_start=0.0, domain_length=1.0)
branch1_disc = Discretization(n_elements=20, domain_start=1.0, domain_length=0.8)
branch2_disc = Discretization(n_elements=20, domain_start=1.0, domain_length=0.8)

multi_global_disc = GlobalDiscretization([main_disc, branch1_disc, branch2_disc])
\end{lstlisting}

\subsubsection{Core Attributes}

\begin{longtable}{|p{3.5cm}|p{2.5cm}|p{7cm}|}
\hline
\textbf{Attribute} & \textbf{Type} & \textbf{Description} \\
\hline
\endhead

\texttt{spatial\_discretizations} & \texttt{List[Discretization]} & List of spatial discretizations \\
\hline

\texttt{n\_domains} & \texttt{int} & Number of domains \\
\hline

\texttt{dt} & \texttt{Optional[float]} & Global time step size \\
\hline

\texttt{T} & \texttt{Optional[float]} & Final time \\
\hline

\texttt{n\_time\_steps} & \texttt{Optional[int]} & Number of time steps: \texttt{int(T/dt)} \\
\hline

\texttt{total\_elements} & \texttt{int} & Sum of elements across all domains \\
\hline

\texttt{total\_nodes} & \texttt{int} & Sum of nodes across all domains \\
\hline

\end{longtable}

\subsubsection{Time Parameter Management}

\paragraph{set\_time\_parameters()}
\begin{lstlisting}[language=Python, caption=Set Time Parameters Method]
def set_time_parameters(self, dt: float, T: float)
\end{lstlisting}

\textbf{Parameters:}
\begin{itemize}
    \item \texttt{dt}: Time step size (corresponding to MATLAB \texttt{discretization.dt})
    \item \texttt{T}: Final time
\end{itemize}

\textbf{Side Effects:} 
\begin{itemize}
    \item Sets \texttt{self.dt}, \texttt{self.T}, and \texttt{self.n\_time\_steps}
    \item Propagates \texttt{dt} to all spatial discretizations
\end{itemize}

\textbf{Usage:}
\begin{lstlisting}[language=Python, caption=Time Parameters Usage]
# Set time parameters (matching MATLAB TestProblem.m patterns)
global_disc.set_time_parameters(dt=0.01, T=1.0)
print(f"Time step: {global_disc.dt}")
print(f"Final time: {global_disc.T}")
print(f"Number of time steps: {global_disc.n_time_steps}")

# Verify propagation to spatial discretizations
for i, disc in enumerate(global_disc.spatial_discretizations):
    print(f"Domain {i} dt: {disc.dt}")
\end{lstlisting}

\paragraph{get\_time\_points()}
\begin{lstlisting}[language=Python, caption=Get Time Points Method]
def get_time_points(self) -> np.ndarray
\end{lstlisting}

\textbf{Returns:} \texttt{np.ndarray} - Array of time points from 0 to T

\textbf{Usage:}
\begin{lstlisting}[language=Python, caption=Time Points Usage]
time_points = global_disc.get_time_points()
print(f"Time points: {time_points[:5]}...")  # First 5 time points
print(f"Total time points: {len(time_points)}")
\end{lstlisting}

\subsubsection{Domain Access Methods}

\paragraph{get\_discretization()}
\begin{lstlisting}[language=Python, caption=Get Discretization Method]
def get_discretization(self, domain_idx: int) -> Discretization
\end{lstlisting}

\textbf{Parameters:}
\begin{itemize}
    \item \texttt{domain\_idx}: Domain index (0 to \texttt{n\_domains-1})
\end{itemize}

\textbf{Returns:} \texttt{Discretization} - Spatial discretization for specified domain

\textbf{Usage:}
\begin{lstlisting}[language=Python, caption=Get Discretization Usage]
domain_0_disc = global_disc.get_discretization(0)
print(f"Domain 0 elements: {domain_0_disc.n_elements}")
print(f"Domain 0 nodes: {domain_0_disc.n_nodes}")
\end{lstlisting}

\paragraph{get\_all\_discretizations()}
\begin{lstlisting}[language=Python, caption=Get All Discretizations Method]
def get_all_discretizations(self) -> List[Discretization]
\end{lstlisting}

\textbf{Returns:} \texttt{List[Discretization]} - Copy of all spatial discretizations

\textbf{Usage:}
\begin{lstlisting}[language=Python, caption=Get All Discretizations Usage]
all_disc = global_disc.get_all_discretizations()
for i, disc in enumerate(all_disc):
    print(f"Domain {i}: {disc.n_elements} elements")
\end{lstlisting}

\subsubsection{Global Statistics Methods}

\paragraph{compute\_global\_statistics()}
\begin{lstlisting}[language=Python, caption=Compute Global Statistics Method]
def compute_global_statistics(self) -> dict
\end{lstlisting}

\textbf{Returns:} \texttt{dict} - Dictionary with global discretization statistics

\textbf{Usage:}
\begin{lstlisting}[language=Python, caption=Global Statistics Usage]
stats = global_disc.compute_global_statistics()
print(f"Total elements: {stats['total_elements']}")
print(f"Total nodes: {stats['total_nodes']}")
print(f"Average element size: {stats['avg_element_size']:.6f}")
print(f"Min element size: {stats['min_element_size']:.6f}")
print(f"Max element size: {stats['max_element_size']:.6f}")
\end{lstlisting}

\paragraph{get\_cfl\_number()}
\begin{lstlisting}[language=Python, caption=Get CFL Number Method]
def get_cfl_number(self, wave_speed: float = 1.0) -> float
\end{lstlisting}

\textbf{Parameters:}
\begin{itemize}
    \item \texttt{wave\_speed}: Characteristic wave speed for CFL calculation (default: 1.0)
\end{itemize}

\textbf{Returns:} \texttt{float} - CFL number: \texttt{wave\_speed * dt / min\_element\_size}

\textbf{Usage:}
\begin{lstlisting}[language=Python, caption=CFL Number Usage]
cfl = global_disc.get_cfl_number(wave_speed=1.0)
print(f"CFL number: {cfl:.6f}")

# Check stability condition
if cfl <= 1.0:
    print("✓ CFL condition satisfied")
else:
    print("⚠ CFL condition violated - consider smaller dt or larger elements")
\end{lstlisting}

\subsection{Integration with BioNetFlux Components}
\label{subsec:discretization_integration}

\subsubsection{Integration with Problem Class}

\begin{lstlisting}[language=Python, caption=Problem-Discretization Integration]
from ooc1d.core.problem import Problem

# Create problem and matching discretization
problem = Problem(
    neq=4,
    domain_start=0.0,
    domain_length=1.0,
    parameters=np.array([1.0, 2.0, 1.0, 1.0, 0.0, 1.0, 0.0, 1.0, 1.0]),
    problem_type="organ_on_chip"
)

# Create matching discretization
discretization = Discretization(
    n_elements=40,
    domain_start=problem.domain_start,
    domain_length=problem.domain_length,
    stab_constant=1.0
)

# Set stabilization parameters for 4-equation system
discretization.set_tau([1.0, 1.0, 1.0, 1.0])  # [tu, to, tv, tp]

# Create global discretization with time parameters
global_disc = GlobalDiscretization([discretization])
global_disc.set_time_parameters(dt=0.01, T=1.0)
\end{lstlisting}

\subsubsection{Integration with Static Condensation}

\begin{lstlisting}[language=Python, caption=Static Condensation Integration]
from ooc1d.core.static_condensation_ooc import StaticCondensationOOC
from ooc1d.utils.elementary_matrices import ElementaryMatrices

# Create static condensation using discretization
elementary_matrices = ElementaryMatrices()
static_condensation = StaticCondensationOOC(
    problem=problem,
    discretization=discretization,
    elementary_matrices=elementary_matrices
)

# Verify discretization parameters are accessible
print(f"Element length: {discretization.element_length}")
print(f"Time step: {discretization.dt}")
print(f"Stabilization parameters: {discretization.tau}")
\end{lstlisting}

\subsection{Complete Usage Examples}
\label{subsec:discretization_complete_examples}

\subsubsection{Single Domain OrganOnChip Setup}

\begin{lstlisting}[language=Python, caption=Complete Single Domain Setup]
def create_ooc_discretization(n_elements: int = 40,
                             domain_length: float = 1.0,
                             dt: float = 0.01,
                             T: float = 1.0) -> GlobalDiscretization:
    """Create discretization for OrganOnChip problem matching MATLAB TestProblem.m."""
    
    # Create spatial discretization
    spatial_disc = Discretization(
        n_elements=n_elements,
        domain_start=0.0,  # MATLAB: A = 0
        domain_length=domain_length,  # MATLAB: L = 1
        stab_constant=1.0
    )
    
    # Set stabilization parameters for 4-equation OrganOnChip system
    # Based on MATLAB scBlocks.m: tu, to, tv, tp
    spatial_disc.set_tau([1.0, 1.0, 1.0, 1.0])
    
    # Create global discretization
    global_disc = GlobalDiscretization([spatial_disc])
    
    # Set time parameters (matching MATLAB discretization.dt)
    global_disc.set_time_parameters(dt=dt, T=T)
    
    return global_disc

# Usage
ooc_global_disc = create_ooc_discretization()
print(ooc_global_disc.spatial_discretizations[0].summary())
\end{lstlisting}

\subsubsection{Multi-Domain Network Setup}

\begin{lstlisting}[language=Python, caption=Multi-Domain Network Discretization]
def create_network_discretization(domain_specs: List[dict],
                                 elements_per_domain: int = 20,
                                 dt: float = 0.01,
                                 T: float = 0.5) -> GlobalDiscretization:
    """Create discretization for multi-domain network."""
    
    spatial_discretizations = []
    
    for i, spec in enumerate(domain_specs):
        disc = Discretization(
            n_elements=elements_per_domain,
            domain_start=spec['start'],
            domain_length=spec['length'],
            stab_constant=1.0
        )
        
        # Set stabilization parameters based on problem type
        if 'problem_type' in spec and spec['problem_type'] == 'organ_on_chip':
            disc.set_tau([1.0, 1.0, 1.0, 1.0])  # 4 equations
        else:
            disc.set_tau([1.0, 1.0])  # 2 equations (Keller-Segel)
        
        spatial_discretizations.append(disc)
    
    # Create global discretization
    global_disc = GlobalDiscretization(spatial_discretizations)
    global_disc.set_time_parameters(dt=dt, T=T)
    
    return global_disc

# Usage: Y-junction network
domain_specs = [
    {'start': 0.0, 'length': 1.0, 'problem_type': 'organ_on_chip'},  # Main
    {'start': 1.0, 'length': 0.8, 'problem_type': 'organ_on_chip'},  # Branch 1
    {'start': 1.0, 'length': 0.8, 'problem_type': 'organ_on_chip'}   # Branch 2
]

network_global_disc = create_network_discretization(domain_specs)
print(f"Network has {network_global_disc.n_domains} domains")
print(f"Total elements: {network_global_disc.total_elements}")
\end{lstlisting}

\subsection{Method Summary Table}
\label{subsec:discretization_method_summary}

\subsubsection{Discretization Class Methods}

\begin{longtable}{|p{4cm}|p{2cm}|p{7cm}|}
\hline
\textbf{Method} & \textbf{Returns} & \textbf{Purpose} \\
\hline
\endhead

\texttt{generate\_nodes} & \texttt{np.ndarray} & Generate uniform node coordinates \\
\hline

\texttt{generate\_elements} & \texttt{List[Tuple]} & Generate element connectivity \\
\hline

\texttt{set\_tau} & \texttt{None} & Set stabilization parameters \\
\hline

\texttt{get\_tau} & \texttt{float/List} & Retrieve stabilization parameters \\
\hline

\texttt{get\_element\_center} & \texttt{float} & Get element center coordinate \\
\hline

\texttt{get\_element\_bounds} & \texttt{Tuple} & Get element start/end coordinates \\
\hline

\texttt{find\_element\_containing\_point} & \texttt{int} & Locate element containing given point \\
\hline

\texttt{refine\_mesh} & \texttt{Discretization} & Create refined discretization \\
\hline

\texttt{summary} & \texttt{str} & Generate discretization summary \\
\hline

\end{longtable}

\subsubsection{GlobalDiscretization Class Methods}

\begin{longtable}{|p{4cm}|p{2cm}|p{7cm}|}
\hline
\textbf{Method} & \textbf{Returns} & \textbf{Purpose} \\
\hline
\endhead

\texttt{set\_time\_parameters} & \texttt{None} & Set global time step and final time \\
\hline

\texttt{get\_time\_points} & \texttt{np.ndarray} & Generate array of time points \\
\hline

\texttt{get\_discretization} & \texttt{Discretization} & Access specific domain discretization \\
\hline

\texttt{get\_all\_discretizations} & \texttt{List} & Get all spatial discretizations \\
\hline

\texttt{compute\_global\_statistics} & \texttt{dict} & Calculate global mesh statistics \\
\hline

\texttt{get\_cfl\_number} & \texttt{float} & Compute CFL number for stability \\
\hline

\end{longtable}

This documentation provides an exact reference for the Discretization module based on the BioNetFlux implementation patterns and MATLAB reference files, with practical usage examples for both single and multi-domain scenarios.

% End of discretization module API documentation
