% Static Condensation Modules API Documentation (Based on MATLAB Reference)
% To be included in master LaTeX document
%
% Usage: % Static Condensation Modules API Documentation (Based on MATLAB Reference)
% To be included in master LaTeX document
%
% Usage: % Static Condensation Modules API Documentation (Based on MATLAB Reference)
% To be included in master LaTeX document
%
% Usage: % Static Condensation Modules API Documentation (Based on MATLAB Reference)
% To be included in master LaTeX document
%
% Usage: \input{docs/static_condensation_modules_api}

\section{Static Condensation Modules API Reference}
\label{sec:static_condensation_modules_api}

The static condensation module is where the  problem specific HDG discretization happens. This section provides a comprehensive reference for the static condensation module hierarchy based on analysis of the MATLAB reference files and BioNetFlux architecture patterns. The modules implement HDG static condensation for different problem types.

\subsection{Module Overview}

The static condensation system consists of:
\begin{itemize}
    \item \textbf{StaticCondensationBase}: Abstract base class defining the interface
    \item \textbf{Derived classes}
   \begin{itemize}
    \item \textbf{KellerSegelStaticCondensation}: Implementation for 2-equation chemotaxis systems
    \item \textbf{StaticCondensationOOC}: Implementation for 4-equation OrganOnChip systems
   \end{itemize}
    \item \textbf{StaticCondensationFactory}: Factory for creating appropriate implementations
\end{itemize}

%\subsection{MATLAB Reference Analysis}
%
%Based on the MATLAB files, the static condensation process follows this structure:
%
%\textbf{From StaticC.m:} The main static condensation function \texttt{[U,hJ,dhJ] = StaticC(hU,rhs,problem,discretization,scMatrices)} performs:
%\begin{enumerate}
%    \item Local trace-to-bulk reconstruction: $\hat{U} \rightarrow U$
%    \item Flux jump computation: $\tilde{J} = D_1 U - D_2 \hat{U}$
%    \item Normal flux computation: $j = \hat{B}_4 \hat{U} + \tilde{J}^T Q U$
%    \item Final assembly: $\hat{J} = [\hat{j}; \hat{J}]$
%\end{enumerate}
%
%\textbf{From scBlocks.m:} The matrix construction involves building operators for each step of the 4-equation OrganOnChip system.

\subsection{StaticCondensationBase Class}
\label{subsec:static_condensation_base}

Abstract base class defining the interface for all static condensation implementations.

\subsubsection{Constructor}

\paragraph{\_\_init\_\_()}\leavevmode
\begin{lstlisting}[language=Python, caption=StaticCondensationBase Constructor]
def __init__(self, 
             problem: Problem, 
             global_discretization, 
             elementary_matrices, 
             domain_index: int = 0)
\end{lstlisting}

\textbf{Parameters:}
\begin{itemize}
    \item \texttt{problem}: Problem instance with equations and parameters
    \item \texttt{global\_discretization}: GlobalDiscretization instance
    \item \texttt{elementary\_matrices}: ElementaryMatrices instance
    \item \texttt{domain\_index}: Domain index for multi-domain problems (default: 0)
\end{itemize}

\textbf{Side Effects:} Initializes base attributes and extracts domain-specific discretization

\subsubsection{Core Attributes}

\begin{longtable}{|p{3.8cm}|p{3.5cm}|p{7cm}|}
\hline
\textbf{Attribute} & \textbf{Type} & \textbf{Description} \\
\hline
\endhead

\texttt{problem} & \texttt{Problem} & Problem definition with parameters and equations \\
\hline

\texttt{discretization} & \texttt{Discretization} & Domain-specific discretization \\
\hline

\texttt{elementary\_matrices} & \texttt{ElementaryMatrices} & Pre-computed elementary matrices \\
\hline

\texttt{domain\_index} & \texttt{int} & Domain index for multi-domain systems \\
\hline

\texttt{neq} & \texttt{int} & Number of equations (extracted from problem) \\
\hline

\texttt{n\_elements} & \texttt{int} & Number of elements (extracted from discretization) \\
\hline

\texttt{element\_length} & \texttt{float} & Element size: \texttt{h = domain\_length / n\_elements} \\
\hline

\texttt{dt} & \texttt{Optional[float]} & Time step size (from discretization if available) \\
\hline

\end{longtable}

\subsubsection{Abstract Methods (Must be Implemented by Subclasses)}

\paragraph{build\_matrices()}
\begin{lstlisting}[language=Python, caption=Abstract Build Matrices Method]
@abstractmethod
def build_matrices(self) -> dict
\end{lstlisting}

\textbf{Returns:} \texttt{dict} - Dictionary of static condensation matrices

\textbf{Purpose:} Build all matrices required for static condensation (problem-specific)

\paragraph{static\_condensation()}\leavevmode
\begin{lstlisting}[language=Python, caption=Abstract Static Condensation Method]
@abstractmethod
def static_condensation(self, 
                       trace_values: np.ndarray, 
                       rhs: np.ndarray, 
                       time: float = 0.0) -> Tuple[np.ndarray, np.ndarray, np.ndarray]
\end{lstlisting}

\textbf{Parameters:}
\begin{itemize}
    \item \texttt{trace\_values}: Local trace values $\hat{U}$ (size: \texttt{neq*(n\_elements+1)})
    \item \texttt{rhs}: Right-hand side vector (size: \texttt{2*neq*n\_elements})
    \item \texttt{time}: Current time for time-dependent problems (default: 0.0)
\end{itemize}

\textbf{Returns:} \texttt{Tuple[np.ndarray, np.ndarray, np.ndarray]} - (bulk\_solution, flux\_jump, jacobian)

\textbf{Purpose:} Perform local static condensation step

\subsubsection{Concrete Methods (Provided by Base Class)}

\paragraph{get\_problem\_parameters()}\leavevmode
\begin{lstlisting}[language=Python, caption=Get Problem Parameters Method]
def get_problem_parameters(self) -> np.ndarray
\end{lstlisting}

\textbf{Returns:} \texttt{np.ndarray} - Problem parameters array

\textbf{Usage:}
\begin{lstlisting}[language=Python, caption=Get Parameters Usage]
params = static_condensation.get_problem_parameters()
# For OrganOnChip: [nu, mu, epsilon, sigma, a, b, c, d, chi]
# For Keller-Segel: [mu, nu, a, b]
\end{lstlisting}

\paragraph{get\_stabilization\_parameters()}\leavevmode
\begin{lstlisting}[language=Python, caption=Get Stabilization Parameters Method]
def get_stabilization_parameters(self) -> np.ndarray
\end{lstlisting}

\textbf{Returns:} \texttt{np.ndarray} - Stabilization parameters $\tau$ for each equation

\textbf{Usage:}
\begin{lstlisting}[language=Python, caption=Get Stabilization Usage]
tau_values = static_condensation.get_stabilization_parameters()
# For OrganOnChip: [tu, to, tv, tp] (4 equations)
# For Keller-Segel: [tu, tp] (2 equations)
\end{lstlisting}

\paragraph{validate\_input()}\leavevmode
\begin{lstlisting}[language=Python, caption=Validate Input Method]
def validate_input(self, 
                  trace_values: np.ndarray, 
                  rhs: np.ndarray) -> None
\end{lstlisting}

\textbf{Parameters:}
\begin{itemize}
    \item \texttt{trace\_values}: Trace values to validate
    \item \texttt{rhs}: Right-hand side to validate
\end{itemize}

\textbf{Returns:} \texttt{None}

\textbf{Raises:} \texttt{ValueError} for invalid input dimensions or content

\textbf{Validation Checks:}
\begin{itemize}
    \item Trace values size: \texttt{neq * (n\_elements + 1)}
    \item RHS size: \texttt{2 * neq * n\_elements}
    \item No NaN or infinite values
\end{itemize}

\subsection{KellerSegelStaticCondensation Class}
\label{subsec:keller_segel_static_condensation}

Implementation for 2-equation Keller-Segel chemotaxis systems (u, $\phi$).

\subsubsection{Constructor}

\paragraph{\_\_init\_\_()}\leavevmode
\begin{lstlisting}[language=Python, caption=KellerSegel Constructor]
def __init__(self, 
             problem: Problem, 
             global_discretization, 
             elementary_matrices, 
             domain_index: int = 0)
\end{lstlisting}

\textbf{Requirements:} \texttt{problem.neq == 2} and \texttt{problem.type == "keller\_segel"}

\textbf{Usage:}
\begin{lstlisting}[language=Python, caption=KellerSegel Constructor Usage]
# Create Keller-Segel problem
problem = Problem(
    neq=2,
    domain_start=0.0,
    domain_length=1.0,
    parameters=np.array([2.0, 1.0, 0.1, 1.0]),  # [mu, nu, a, b]
    problem_type="keller_segel"
)

# Create static condensation
ks_sc = KellerSegelStaticCondensation(
    problem=problem,
    global_discretization=global_disc,
    elementary_matrices=elementary_matrices,
    domain_index=0
)
\end{lstlisting}

\subsubsection{Specific Attributes}

\begin{longtable}{|p{3.2cm}|p{3.5cm}|p{7cm}|}
\hline
\textbf{Attribute} & \textbf{Type} & \textbf{Description} \\
\hline
\endhead

\texttt{mu} & \texttt{float} & Diffusion coefficient for $\phi$ equation (parameter 0) \\
\hline

\texttt{nu} & \texttt{float} & Diffusion coefficient for $u$ equation (parameter 1) \\
\hline

\texttt{a} & \texttt{float} & Reaction parameter (parameter 2) \\
\hline

\texttt{b} & \texttt{float} & Coupling parameter (parameter 3) \\
\hline

\texttt{chi\_function} & \texttt{Optional[Callable]} & Chemotactic sensitivity $\chi(\phi)$ \\
\hline

\texttt{dchi\_function} & \texttt{Optional[Callable]} & Derivative $\chi'(\phi)$ \\
\hline

\end{longtable}

\subsubsection{Matrix Construction}

\paragraph{build\_matrices()}\leavevmode
\begin{lstlisting}[language=Python, caption=KellerSegel Build Matrices]
def build_matrices(self) -> dict
\end{lstlisting}

\textbf{Returns:} \texttt{dict} - Dictionary with matrices for 2-equation system

\textbf{Matrix Dictionary Keys:}
\begin{itemize}
    \item \texttt{'M'}: Mass matrix
    \item \texttt{'T'}: Trace matrix  
    \item \texttt{'D'}: Differentiation matrix
    \item \texttt{'B1'}, \texttt{'L1'}: Matrices for $u$ equation reconstruction
    \item \texttt{'B2'}, \texttt{'L2'}: Matrices for $\phi$ equation reconstruction
    \item \texttt{'Q'}: Coupling matrix for chemotaxis
\end{itemize}

\textbf{Mathematical Formulation:}
For Keller-Segel system:
\begin{align}
\frac{\partial u}{\partial t} - \nu \nabla^2 u + \nabla \cdot (u \chi(\phi) \nabla \phi) &= f_u \\
\frac{\partial \phi}{\partial t} - \mu \nabla^2 \phi &= f_\phi + a u
\end{align}

\paragraph{static\_condensation()}\leavevmode
\begin{lstlisting}[language=Python, caption=KellerSegel Static Condensation]
def static_condensation(self, 
                       trace_values: np.ndarray, 
                       rhs: np.ndarray, 
                       time: float = 0.0) -> Tuple[np.ndarray, np.ndarray, np.ndarray]
\end{lstlisting}

\textbf{Algorithm:} 
\begin{enumerate}
    \item Extract trace values for $u$ and $\phi$
    \item Reconstruct bulk solutions: $u = B_1 \hat{u} + L_1 g_u$, $\phi = B_2 \hat{\phi} + L_2 g_\phi$
    \item Compute chemotaxis terms if $\chi$ functions available
    \item Compute flux jumps and Jacobian contributions
    \item Return bulk solutions, flux jumps, and Jacobian
\end{enumerate}

\textbf{Usage:}
\begin{lstlisting}[language=Python, caption=KellerSegel Static Condensation Usage]
# Prepare input
trace_vals = np.random.rand(2 * (n_elements + 1))  # u and phi traces
rhs = np.random.rand(2 * 2 * n_elements)  # RHS for both equations

# Perform static condensation
bulk_solution, flux_jump, jacobian = ks_sc.static_condensation(
    trace_values=trace_vals,
    rhs=rhs,
    time=0.5
)

print(f"Bulk solution shape: {bulk_solution.shape}")  # (4, n_elements)
print(f"Flux jump shape: {flux_jump.shape}")  # (2,)
print(f"Jacobian shape: {jacobian.shape}")  # (4, 2*(n_elements+1))
\end{lstlisting}


\subsection{StaticCondensationOOC Class}
\label{subsec:static_condensation_ooc}

Implementation for 4-equation OrganOnChip systems (u, $\omega$, v, $\phi$) based on MATLAB StaticC.m.

\subsubsection{Constructor}

\paragraph{\_\_init\_\_()}\leavevmode
\begin{lstlisting}[language=Python, caption=StaticCondensationOOC Constructor]
def __init__(self, 
             problem: Problem, 
             global_discretization, 
             elementary_matrices, 
             domain_index: int = 0)
\end{lstlisting}

\textbf{Requirements:} \texttt{problem.neq == 4} and \texttt{problem.type == "organ\_on\_chip"}

\textbf{Usage:}
\begin{lstlisting}[language=Python, caption=OrganOnChip Constructor Usage]
# Create OrganOnChip problem (from MATLAB TestProblem.m)
ooc_params = np.array([1.0, 2.0, 1.0, 1.0, 0.0, 1.0, 0.0, 1.0, 1.0])
problem = Problem(
    neq=4,
    domain_start=0.0,  # A = 0
    domain_length=1.0, # L = 1
    parameters=ooc_params,  # [nu, mu, epsilon, sigma, a, b, c, d, chi]
    problem_type="organ_on_chip"
)

# Create static condensation
ooc_sc = StaticCondensationOOC(
    problem=problem,
    global_discretization=global_disc,
    elementary_matrices=elementary_matrices,
    domain_index=0
)
\end{lstlisting}

%\subsubsection{Specific Attributes (From MATLAB TestProblem.m)}
%
%\begin{longtable}{|p{3.5cm}|p{3.5cm}|p{7cm}|}
%\hline
%\textbf{Attribute} & \textbf{Type} & \textbf{Description} \\
%\hline
%\endhead
%
%\texttt{nu} & \texttt{float} & Viscosity parameter (parameter 0) \\
%\hline
%
%\texttt{mu} & \texttt{float} & Viscosity parameter (parameter 1) \\
%\hline
%
%\texttt{epsilon} & \texttt{float} & Coupling parameter (parameter 2) \\
%\hline
%
%\texttt{sigma} & \texttt{float} & Coupling parameter (parameter 3) \\
%\hline
%
%\texttt{a} & \texttt{float} & Reaction parameter (parameter 4) \\
%\hline
%
%\texttt{b} & \texttt{float} & Inter-equation coupling (parameter 5) \\
%\hline
%
%\texttt{c} & \texttt{float} & Reaction parameter (parameter 6) \\
%\hline
%
%\texttt{d} & \texttt{float} & Inter-equation coupling (parameter 7) \\
%\hline
%
%\texttt{chi} & \texttt{float} & Coupling strength (parameter 8) \\
%\hline
%
%\texttt{lambda\_function} & \texttt{Optional[Callable]} & Nonlinear response $\lambda(\omega)$ \\
%\hline
%
%\texttt{dlambda\_function} & \texttt{Optional[Callable]} & Derivative $\lambda'(\omega)$ \\
%\hline
%
%\end{longtable}
%
%\subsubsection{Mathematical System (From MATLAB References)}
%
%The OrganOnChip system implements:
%\begin{align}
%\frac{\partial u}{\partial t} - \nu \nabla^2 u &= f_u \label{eq:ooc_u} \\
%\frac{\partial \omega}{\partial t} + \epsilon \nabla \cdot \theta + c\omega &= f_\omega + d u \label{eq:ooc_omega} \\
%\frac{\partial v}{\partial t} + \sigma \nabla \cdot q + \lambda(\bar{\omega}) v &= f_v \label{eq:ooc_v} \\
%\frac{\partial \phi}{\partial t} + \mu \nabla \cdot \psi + a\phi &= f_\phi + b v \label{eq:ooc_phi}
%\end{align}
%
%With auxiliary relations:
%\begin{align}
%\theta &= \epsilon(\nabla \omega - \hat{\omega}) \\
%q &= \sigma(\nabla v - \hat{v}) \\
%\psi &= \mu(\nabla \phi - \hat{\phi})
%\end{align}

\subsubsection{Matrix Construction (Based on MATLAB scBlocks.m)}

\paragraph{build\_matrices()}\leavevmode
\begin{lstlisting}[language=Python, caption=OrganOnChip Build Matrices]
def build_matrices(self) -> dict
\end{lstlisting}

\textbf{Returns:} Dictionary with OrganOnChip-specific matrices

\textbf{Matrix Dictionary Keys (From MATLAB scBlocks.m):}
\begin{itemize}
    \item \texttt{'M'}: Mass matrix
    \item \texttt{'T'}: Trace matrix
    \item \texttt{'B1'}, \texttt{'L1'}: Matrices for $u$ equation
    \item \texttt{'B2'}, \texttt{'C2'}, \texttt{'L2'}: Matrices for $\omega$ equation  
    \item \texttt{'A3'}, \texttt{'S3'}, \texttt{'H3'}: Matrices for $v$ equation
    \item \texttt{'B4'}, \texttt{'C4'}, \texttt{'L4'}: Matrices for $\phi$ equation
    \item \texttt{'D1'}, \texttt{'D2'}: Flux jump matrices
    \item \texttt{'Q'}: Coupling matrix
    \item \texttt{'hB4'}, \texttt{'B5'}, \texttt{'B6'}, \texttt{'B7'}: Assembly matrices
    \item \texttt{'hatB0'}, \texttt{'hatB1'}, \texttt{'hatB2'}: Final assembly matrices
    \item \texttt{'Av'}: Averaging matrix for $\bar{\omega}$ computation
\end{itemize}

\paragraph{static\_condensation()}\leavevmode
\begin{lstlisting}[language=Python, caption=OrganOnChip Static Condensation]
def static_condensation(self, 
                       trace_values: np.ndarray, 
                       rhs: np.ndarray, 
                       time: float = 0.0) -> Tuple[np.ndarray, np.ndarray, np.ndarray]
\end{lstlisting}

\textbf{Algorithm (Following MATLAB StaticC.m):}
\begin{enumerate}
    \item \textbf{Step 1}: $\hat{U} \rightarrow U$ reconstruction
    \begin{itemize}
        \item $u = B_1 \hat{u} + y_1$ where $y_1 = L_1 g_u$
        \item $\omega = C_2 \hat{u} + B_2 \hat{\omega} + y_2$ where $y_2 = L_2(g_\omega + dt \cdot d \cdot M y_1)$
    \end{itemize}
    
    \item \textbf{Step 2}: Nonlinear coupling computation
    \begin{itemize}
        \item $\bar{\omega} = A_{av} \omega$ (averaging)
        \item $\bar{\lambda}_\omega = \lambda(\bar{\omega})$ (nonlinear function evaluation)
        \item $J_\lambda = \lambda'(\bar{\omega}) A_{av}$ (Jacobian contribution)
    \end{itemize}
    
    \item \textbf{Step 3}: $v$ equation solution
    \begin{itemize}
        \item $L_3(\omega) = (A_3 + \bar{\lambda}_\omega S_3)^{-1}$ (nonlinear operator)
        \item $v = B_3(\omega) \hat{v} + y_3(\omega)$ where $B_3(\omega) = L_3(\omega) H_3$
    \end{itemize}
    
    \item \textbf{Step 4}: $\phi$ equation solution
    \begin{itemize}
        \item $\phi = B_4 \hat{\phi} + C_4 v + L_4 g_\phi$
    \end{itemize}
    
    \item \textbf{Step 5}: Flux jump computation
    \begin{itemize}
        \item $\tilde{J} = D_1 U - D_2 \hat{U}$
        \item $j = \hat{B}_4 \hat{U} + \tilde{J}^T Q U$
    \end{itemize}
    
    \item \textbf{Step 6}: Final assembly
    \begin{itemize}
        \item $\hat{j} = B_5 j + B_6 U + B_7 \hat{U}$
        \item $\hat{J} = \hat{B}_0 \tilde{J} + \hat{B}_1 U - \hat{B}_2 \hat{U}$
    \end{itemize}
\end{enumerate}

\textbf{Usage:}
\begin{lstlisting}[language=Python, caption=OrganOnChip Static Condensation Usage]
# Prepare input (4 equations, matching MATLAB TestProblem.m)
trace_vals = np.random.rand(4 * (n_elements + 1))  # u, omega, v, phi traces
rhs = np.random.rand(2 * 4 * n_elements)  # RHS for all 4 equations

# Set lambda function (from MATLAB TestProblem.m: constant function)
ooc_sc.set_lambda_functions(
    lambda_function=lambda omega: np.ones_like(omega),
    dlambda_function=lambda omega: np.zeros_like(omega)
)

# Perform static condensation
bulk_solution, flux_jump, jacobian = ooc_sc.static_condensation(
    trace_values=trace_vals,
    rhs=rhs,
    time=0.0
)

print(f"Bulk solution shape: {bulk_solution.shape}")  # (8, n_elements)
print(f"Flux jump shape: {flux_jump.shape}")  # (4,) 
print(f"Jacobian shape: {jacobian.shape}")  # (8, 4*(n_elements+1))
\end{lstlisting}


\subsection{StaticCondensationFactory Usage}
\label{subsec:factory_usage}

The factory class provides automatic selection of appropriate static condensation implementation.

\subsubsection{Factory Methods}

\paragraph{create()}\leavevmode
\begin{lstlisting}[language=Python, caption=Factory Create Method]
@classmethod
def create(cls, problem: Problem, global_disc, elementary_matrices,
           i: int = 0) -> StaticCondensationBase
\end{lstlisting}

\textbf{Parameters:}
\begin{itemize}
    \item \texttt{problem}: Problem instance with \texttt{type} attribute
    \item \texttt{global\_disc}: GlobalDiscretization instance
    \item \texttt{elementary\_matrices}: ElementaryMatrices instance
    \item \texttt{i}: Domain index (default: 0)
\end{itemize}

\textbf{Returns:} Appropriate StaticCondensation implementation

\textbf{Usage:}
\begin{lstlisting}[language=Python, caption=Factory Usage Examples]
from ooc1d.core.static_condensation_factory import StaticCondensationFactory

# Automatic selection for Keller-Segel
ks_problem = Problem(neq=2, ..., problem_type="keller_segel")
ks_sc = StaticCondensationFactory.create(
    problem=ks_problem,
    global_disc=global_discretization,
    elementary_matrices=elem_matrices,
    i=0
)
# Returns: KellerSegelStaticCondensation instance

# Automatic selection for OrganOnChip  
ooc_problem = Problem(neq=4, ..., problem_type="organ_on_chip")
ooc_sc = StaticCondensationFactory.create(
    problem=ooc_problem,
    global_disc=global_discretization,
    elementary_matrices=elem_matrices,
    i=0
)
# Returns: StaticCondensationOOC instance
\end{lstlisting}

\paragraph{register\_implementation()}\leavevmode
\begin{lstlisting}[language=Python, caption=Register Implementation Method]
@classmethod
def register_implementation(cls, problem_type: str, 
                           implementation_class: Type[StaticCondensationBase])
\end{lstlisting}

\textbf{Purpose:} Register new static condensation implementations

\textbf{Usage:}
\begin{lstlisting}[language=Python, caption=Register New Implementation]
# Register custom implementation
class CustomStaticCondensation(StaticCondensationBase):
    # ... implementation ...
    pass

StaticCondensationFactory.register_implementation(
    "custom_problem", CustomStaticCondensation
)

# Now available through factory
custom_problem = Problem(neq=3, ..., problem_type="custom_problem")
custom_sc = StaticCondensationFactory.create(custom_problem, ...)
\end{lstlisting}

\subsection{Complete Usage Examples}
\label{subsec:complete_usage_examples}

\subsubsection{Keller-Segel Complete Workflow}

\begin{lstlisting}[language=Python, caption=Complete KellerSegel Workflow]
import numpy as np
from ooc1d.core.problem import Problem
from ooc1d.core.discretization import Discretization, GlobalDiscretization
from ooc1d.utils.elementary_matrices import ElementaryMatrices
from ooc1d.core.static_condensation_factory import StaticCondensationFactory

# Step 1: Create Keller-Segel problem
problem = Problem(
    neq=2,
    domain_start=0.0,
    domain_length=1.0,
    parameters=np.array([2.0, 1.0, 0.1, 1.0]),  # [mu, nu, a, b]
    problem_type="keller_segel",
    name="chemotaxis_problem"
)

# Set chemotaxis functions
problem.set_chemotaxis(
    chi=lambda phi: np.ones_like(phi),
    dchi=lambda phi: np.zeros_like(phi)
)

# Step 2: Create discretization
discretization = Discretization(n_elements=20)
discretization.set_tau([1.0, 1.0])  # [tau_u, tau_phi]
global_disc = GlobalDiscretization([discretization])

# Step 3: Create elementary matrices
elementary_matrices = ElementaryMatrices()

# Step 4: Create static condensation via factory
static_condensation = StaticCondensationFactory.create(
    problem=problem,
    global_disc=global_disc,
    elementary_matrices=elementary_matrices,
    i=0
)

print(f"Created: {type(static_condensation).__name__}")  # KellerSegelStaticCondensation

# Step 5: Build matrices
matrices = static_condensation.build_matrices()
print(f"Available matrices: {list(matrices.keys())}")

# Step 6: Perform static condensation
trace_values = np.random.rand(2 * 21)  # 2 equations, 21 nodes
rhs = np.random.rand(2 * 2 * 20)  # 2 equations, 2 DOFs per element, 20 elements

bulk_solution, flux_jump, jacobian = static_condensation.static_condensation(
    trace_values=trace_values,
    rhs=rhs,
    time=0.0
)

print(f"Static condensation completed:")
print(f"  Bulk solution: {bulk_solution.shape}")
print(f"  Flux jump: {flux_jump.shape}")
print(f"  Jacobian: {jacobian.shape}")
\end{lstlisting}

\subsubsection{OrganOnChip Complete Workflow (MATLAB Compatible)}

\begin{lstlisting}[language=Python, caption=Complete OrganOnChip Workflow]
# Step 1: Create OrganOnChip problem (matching MATLAB TestProblem.m)
ooc_params = np.array([1.0, 2.0, 1.0, 1.0, 0.0, 1.0, 0.0, 1.0, 1.0])
# [nu, mu, epsilon, sigma, a, b, c, d, chi]

problem = Problem(
    neq=4,
    domain_start=0.0,  # A = 0 (MATLAB)
    domain_length=1.0, # L = 1 (MATLAB)
    parameters=ooc_params,
    problem_type="organ_on_chip",
    name="microfluidic_device"
)

# Set initial conditions (matching MATLAB TestProblem.m)
problem.set_initial_condition(0, lambda x, t: np.sin(2*np.pi*x))  # u
for eq_idx in [1, 2, 3]:  # omega, v, phi
    problem.set_initial_condition(eq_idx, lambda x, t: np.zeros_like(x))

# Set lambda function (matching MATLAB: constant_function)
problem.set_function('lambda_function', lambda x: np.ones_like(x))
problem.set_function('dlambda_function', lambda x: np.zeros_like(x))

# Step 2: Create discretization
discretization = Discretization(n_elements=40)  # Matching MATLAB
discretization.set_tau([1.0, 1.0, 1.0, 1.0])  # [tu, to, tv, tp]
global_disc = GlobalDiscretization([discretization])
global_disc.set_time_parameters(dt=0.01, T=0.5)  # Matching MATLAB

# Step 3: Create static condensation
elementary_matrices = ElementaryMatrices()
static_condensation = StaticCondensationFactory.create(
    problem=problem,
    global_disc=global_disc,
    elementary_matrices=elementary_matrices,
    i=0
)

print(f"Created: {type(static_condensation).__name__}")  # StaticCondensationOOC

# Set lambda functions for nonlinear coupling
static_condensation.set_lambda_functions(
    lambda_function=lambda omega: np.ones_like(omega),  # Constant (MATLAB)
    dlambda_function=lambda omega: np.zeros_like(omega)
)

# Step 4: Build matrices (following MATLAB scBlocks.m structure)
matrices = static_condensation.build_matrices()
required_matrices = ['M', 'T', 'B1', 'L1', 'B2', 'C2', 'L2', 'A3', 'S3', 'H3', 
                    'B4', 'C4', 'L4', 'D1', 'D2', 'Q', 'Av']
print(f"Required matrices available: {all(m in matrices for m in required_matrices)}")

# Step 5: Perform static condensation (following MATLAB StaticC.m)
trace_values = np.random.rand(4 * 41)  # 4 equations, 41 nodes
rhs = np.random.rand(2 * 4 * 40)  # 4 equations, 2 DOFs per element, 40 elements

bulk_solution, flux_jump, jacobian = static_condensation.static_condensation(
    trace_values=trace_values,
    rhs=rhs,
    time=0.0
)

# Verify MATLAB compatibility
print(f"OrganOnChip static condensation results:")
print(f"  Bulk solution (U): {bulk_solution.shape}")  # Should be (8, 40)
print(f"  Flux jump (hJ): {flux_jump.shape}")  # Should be (4,) or (8,)
print(f"  Jacobian (dhJ): {jacobian.shape}")  # Should be (8, 4*41)

# Check parameter extraction matches MATLAB
params = static_condensation.get_problem_parameters()
matlab_params = {'nu': params[0], 'mu': params[1], 'epsilon': params[2], 
                'sigma': params[3], 'a': params[4], 'b': params[5], 
                'c': params[6], 'd': params[7], 'chi': params[8]}
print(f"MATLAB-compatible parameters: {matlab_params}")
\end{lstlisting}

\subsection{Method Summary Table}
\label{subsec:static_condensation_method_summary}

\subsubsection{StaticCondensationBase Methods}

\begin{longtable}{|p{5.5cm}|p{2cm}|p{7cm}|}
\hline
\textbf{Method} & \textbf{Returns} & \textbf{Purpose} \\
\hline
\endhead

\texttt{\_\_init\_\_} & \texttt{None} & Initialize base class with common attributes \\
\hline

\texttt{build\_matrices} & \texttt{dict} & Abstract: build problem-specific matrices \\
\hline

\texttt{static\_condensation} & \texttt{Tuple} & Abstract: perform local static condensation \\
\hline

\texttt{get\_problem\_parameters} & \texttt{np.ndarray} & Extract problem parameter array \\
\hline

\texttt{get\_stabilization\_parameters} & \texttt{np.ndarray} & Extract stabilization parameters \\
\hline

\texttt{validate\_input} & \texttt{None} & Validate trace values and RHS dimensions \\
\hline

\end{longtable}

\subsubsection{KellerSegelStaticCondensation Methods}

\begin{longtable}{|p{4.7cm}|p{2cm}|p{7cm}|}
\hline
\textbf{Method} & \textbf{Returns} & \textbf{Purpose} \\
\hline
\endhead

\texttt{build\_matrices} & \texttt{dict} & Build 2-equation Keller-Segel matrices \\
\hline

\texttt{static\_condensation} & \texttt{Tuple} & Perform Keller-Segel static condensation \\
\hline

\texttt{set\_chemotaxis\_functions} & \texttt{None} & Set nonlinear chemotaxis functions \\
\hline

\end{longtable}

\subsubsection{StaticCondensationOOC Methods}

\begin{longtable}{|p{4cm}|p{2cm}|p{7cm}|}
\hline
\textbf{Method} & \textbf{Returns} & \textbf{Purpose} \\
\hline
\endhead

\texttt{build\_matrices} & \texttt{dict} & Build 4-equation OrganOnChip matrices (MATLAB compatible) \\
\hline

\texttt{static\_condensation} & \texttt{Tuple} & Perform OrganOnChip static condensation (MATLAB StaticC.m) \\
\hline

\texttt{set\_lambda\_functions} & \texttt{None} & Set nonlinear response functions $\lambda(\omega)$ \\
\hline

\end{longtable}

This documentation provides a comprehensive reference for the static condensation module hierarchy, with implementations that follow the MATLAB reference files while integrating with the BioNetFlux Python architecture.

% End of static condensation modules API documentation


\section{Static Condensation Modules API Reference}
\label{sec:static_condensation_modules_api}

The static condensation module is where the  problem specific HDG discretization happens. This section provides a comprehensive reference for the static condensation module hierarchy based on analysis of the MATLAB reference files and BioNetFlux architecture patterns. The modules implement HDG static condensation for different problem types.

\subsection{Module Overview}

The static condensation system consists of:
\begin{itemize}
    \item \textbf{StaticCondensationBase}: Abstract base class defining the interface
    \item \textbf{Derived classes}
   \begin{itemize}
    \item \textbf{KellerSegelStaticCondensation}: Implementation for 2-equation chemotaxis systems
    \item \textbf{StaticCondensationOOC}: Implementation for 4-equation OrganOnChip systems
   \end{itemize}
    \item \textbf{StaticCondensationFactory}: Factory for creating appropriate implementations
\end{itemize}

%\subsection{MATLAB Reference Analysis}
%
%Based on the MATLAB files, the static condensation process follows this structure:
%
%\textbf{From StaticC.m:} The main static condensation function \texttt{[U,hJ,dhJ] = StaticC(hU,rhs,problem,discretization,scMatrices)} performs:
%\begin{enumerate}
%    \item Local trace-to-bulk reconstruction: $\hat{U} \rightarrow U$
%    \item Flux jump computation: $\tilde{J} = D_1 U - D_2 \hat{U}$
%    \item Normal flux computation: $j = \hat{B}_4 \hat{U} + \tilde{J}^T Q U$
%    \item Final assembly: $\hat{J} = [\hat{j}; \hat{J}]$
%\end{enumerate}
%
%\textbf{From scBlocks.m:} The matrix construction involves building operators for each step of the 4-equation OrganOnChip system.

\subsection{StaticCondensationBase Class}
\label{subsec:static_condensation_base}

Abstract base class defining the interface for all static condensation implementations.

\subsubsection{Constructor}

\paragraph{\_\_init\_\_()}\leavevmode
\begin{lstlisting}[language=Python, caption=StaticCondensationBase Constructor]
def __init__(self, 
             problem: Problem, 
             global_discretization, 
             elementary_matrices, 
             domain_index: int = 0)
\end{lstlisting}

\textbf{Parameters:}
\begin{itemize}
    \item \texttt{problem}: Problem instance with equations and parameters
    \item \texttt{global\_discretization}: GlobalDiscretization instance
    \item \texttt{elementary\_matrices}: ElementaryMatrices instance
    \item \texttt{domain\_index}: Domain index for multi-domain problems (default: 0)
\end{itemize}

\textbf{Side Effects:} Initializes base attributes and extracts domain-specific discretization

\subsubsection{Core Attributes}

\begin{longtable}{|p{3.8cm}|p{3.5cm}|p{7cm}|}
\hline
\textbf{Attribute} & \textbf{Type} & \textbf{Description} \\
\hline
\endhead

\texttt{problem} & \texttt{Problem} & Problem definition with parameters and equations \\
\hline

\texttt{discretization} & \texttt{Discretization} & Domain-specific discretization \\
\hline

\texttt{elementary\_matrices} & \texttt{ElementaryMatrices} & Pre-computed elementary matrices \\
\hline

\texttt{domain\_index} & \texttt{int} & Domain index for multi-domain systems \\
\hline

\texttt{neq} & \texttt{int} & Number of equations (extracted from problem) \\
\hline

\texttt{n\_elements} & \texttt{int} & Number of elements (extracted from discretization) \\
\hline

\texttt{element\_length} & \texttt{float} & Element size: \texttt{h = domain\_length / n\_elements} \\
\hline

\texttt{dt} & \texttt{Optional[float]} & Time step size (from discretization if available) \\
\hline

\end{longtable}

\subsubsection{Abstract Methods (Must be Implemented by Subclasses)}

\paragraph{build\_matrices()}
\begin{lstlisting}[language=Python, caption=Abstract Build Matrices Method]
@abstractmethod
def build_matrices(self) -> dict
\end{lstlisting}

\textbf{Returns:} \texttt{dict} - Dictionary of static condensation matrices

\textbf{Purpose:} Build all matrices required for static condensation (problem-specific)

\paragraph{static\_condensation()}\leavevmode
\begin{lstlisting}[language=Python, caption=Abstract Static Condensation Method]
@abstractmethod
def static_condensation(self, 
                       trace_values: np.ndarray, 
                       rhs: np.ndarray, 
                       time: float = 0.0) -> Tuple[np.ndarray, np.ndarray, np.ndarray]
\end{lstlisting}

\textbf{Parameters:}
\begin{itemize}
    \item \texttt{trace\_values}: Local trace values $\hat{U}$ (size: \texttt{neq*(n\_elements+1)})
    \item \texttt{rhs}: Right-hand side vector (size: \texttt{2*neq*n\_elements})
    \item \texttt{time}: Current time for time-dependent problems (default: 0.0)
\end{itemize}

\textbf{Returns:} \texttt{Tuple[np.ndarray, np.ndarray, np.ndarray]} - (bulk\_solution, flux\_jump, jacobian)

\textbf{Purpose:} Perform local static condensation step

\subsubsection{Concrete Methods (Provided by Base Class)}

\paragraph{get\_problem\_parameters()}\leavevmode
\begin{lstlisting}[language=Python, caption=Get Problem Parameters Method]
def get_problem_parameters(self) -> np.ndarray
\end{lstlisting}

\textbf{Returns:} \texttt{np.ndarray} - Problem parameters array

\textbf{Usage:}
\begin{lstlisting}[language=Python, caption=Get Parameters Usage]
params = static_condensation.get_problem_parameters()
# For OrganOnChip: [nu, mu, epsilon, sigma, a, b, c, d, chi]
# For Keller-Segel: [mu, nu, a, b]
\end{lstlisting}

\paragraph{get\_stabilization\_parameters()}\leavevmode
\begin{lstlisting}[language=Python, caption=Get Stabilization Parameters Method]
def get_stabilization_parameters(self) -> np.ndarray
\end{lstlisting}

\textbf{Returns:} \texttt{np.ndarray} - Stabilization parameters $\tau$ for each equation

\textbf{Usage:}
\begin{lstlisting}[language=Python, caption=Get Stabilization Usage]
tau_values = static_condensation.get_stabilization_parameters()
# For OrganOnChip: [tu, to, tv, tp] (4 equations)
# For Keller-Segel: [tu, tp] (2 equations)
\end{lstlisting}

\paragraph{validate\_input()}\leavevmode
\begin{lstlisting}[language=Python, caption=Validate Input Method]
def validate_input(self, 
                  trace_values: np.ndarray, 
                  rhs: np.ndarray) -> None
\end{lstlisting}

\textbf{Parameters:}
\begin{itemize}
    \item \texttt{trace\_values}: Trace values to validate
    \item \texttt{rhs}: Right-hand side to validate
\end{itemize}

\textbf{Returns:} \texttt{None}

\textbf{Raises:} \texttt{ValueError} for invalid input dimensions or content

\textbf{Validation Checks:}
\begin{itemize}
    \item Trace values size: \texttt{neq * (n\_elements + 1)}
    \item RHS size: \texttt{2 * neq * n\_elements}
    \item No NaN or infinite values
\end{itemize}

\subsection{KellerSegelStaticCondensation Class}
\label{subsec:keller_segel_static_condensation}

Implementation for 2-equation Keller-Segel chemotaxis systems (u, $\phi$).

\subsubsection{Constructor}

\paragraph{\_\_init\_\_()}\leavevmode
\begin{lstlisting}[language=Python, caption=KellerSegel Constructor]
def __init__(self, 
             problem: Problem, 
             global_discretization, 
             elementary_matrices, 
             domain_index: int = 0)
\end{lstlisting}

\textbf{Requirements:} \texttt{problem.neq == 2} and \texttt{problem.type == "keller\_segel"}

\textbf{Usage:}
\begin{lstlisting}[language=Python, caption=KellerSegel Constructor Usage]
# Create Keller-Segel problem
problem = Problem(
    neq=2,
    domain_start=0.0,
    domain_length=1.0,
    parameters=np.array([2.0, 1.0, 0.1, 1.0]),  # [mu, nu, a, b]
    problem_type="keller_segel"
)

# Create static condensation
ks_sc = KellerSegelStaticCondensation(
    problem=problem,
    global_discretization=global_disc,
    elementary_matrices=elementary_matrices,
    domain_index=0
)
\end{lstlisting}

\subsubsection{Specific Attributes}

\begin{longtable}{|p{3.2cm}|p{3.5cm}|p{7cm}|}
\hline
\textbf{Attribute} & \textbf{Type} & \textbf{Description} \\
\hline
\endhead

\texttt{mu} & \texttt{float} & Diffusion coefficient for $\phi$ equation (parameter 0) \\
\hline

\texttt{nu} & \texttt{float} & Diffusion coefficient for $u$ equation (parameter 1) \\
\hline

\texttt{a} & \texttt{float} & Reaction parameter (parameter 2) \\
\hline

\texttt{b} & \texttt{float} & Coupling parameter (parameter 3) \\
\hline

\texttt{chi\_function} & \texttt{Optional[Callable]} & Chemotactic sensitivity $\chi(\phi)$ \\
\hline

\texttt{dchi\_function} & \texttt{Optional[Callable]} & Derivative $\chi'(\phi)$ \\
\hline

\end{longtable}

\subsubsection{Matrix Construction}

\paragraph{build\_matrices()}\leavevmode
\begin{lstlisting}[language=Python, caption=KellerSegel Build Matrices]
def build_matrices(self) -> dict
\end{lstlisting}

\textbf{Returns:} \texttt{dict} - Dictionary with matrices for 2-equation system

\textbf{Matrix Dictionary Keys:}
\begin{itemize}
    \item \texttt{'M'}: Mass matrix
    \item \texttt{'T'}: Trace matrix  
    \item \texttt{'D'}: Differentiation matrix
    \item \texttt{'B1'}, \texttt{'L1'}: Matrices for $u$ equation reconstruction
    \item \texttt{'B2'}, \texttt{'L2'}: Matrices for $\phi$ equation reconstruction
    \item \texttt{'Q'}: Coupling matrix for chemotaxis
\end{itemize}

\textbf{Mathematical Formulation:}
For Keller-Segel system:
\begin{align}
\frac{\partial u}{\partial t} - \nu \nabla^2 u + \nabla \cdot (u \chi(\phi) \nabla \phi) &= f_u \\
\frac{\partial \phi}{\partial t} - \mu \nabla^2 \phi &= f_\phi + a u
\end{align}

\paragraph{static\_condensation()}\leavevmode
\begin{lstlisting}[language=Python, caption=KellerSegel Static Condensation]
def static_condensation(self, 
                       trace_values: np.ndarray, 
                       rhs: np.ndarray, 
                       time: float = 0.0) -> Tuple[np.ndarray, np.ndarray, np.ndarray]
\end{lstlisting}

\textbf{Algorithm:} 
\begin{enumerate}
    \item Extract trace values for $u$ and $\phi$
    \item Reconstruct bulk solutions: $u = B_1 \hat{u} + L_1 g_u$, $\phi = B_2 \hat{\phi} + L_2 g_\phi$
    \item Compute chemotaxis terms if $\chi$ functions available
    \item Compute flux jumps and Jacobian contributions
    \item Return bulk solutions, flux jumps, and Jacobian
\end{enumerate}

\textbf{Usage:}
\begin{lstlisting}[language=Python, caption=KellerSegel Static Condensation Usage]
# Prepare input
trace_vals = np.random.rand(2 * (n_elements + 1))  # u and phi traces
rhs = np.random.rand(2 * 2 * n_elements)  # RHS for both equations

# Perform static condensation
bulk_solution, flux_jump, jacobian = ks_sc.static_condensation(
    trace_values=trace_vals,
    rhs=rhs,
    time=0.5
)

print(f"Bulk solution shape: {bulk_solution.shape}")  # (4, n_elements)
print(f"Flux jump shape: {flux_jump.shape}")  # (2,)
print(f"Jacobian shape: {jacobian.shape}")  # (4, 2*(n_elements+1))
\end{lstlisting}


\subsection{StaticCondensationOOC Class}
\label{subsec:static_condensation_ooc}

Implementation for 4-equation OrganOnChip systems (u, $\omega$, v, $\phi$) based on MATLAB StaticC.m.

\subsubsection{Constructor}

\paragraph{\_\_init\_\_()}\leavevmode
\begin{lstlisting}[language=Python, caption=StaticCondensationOOC Constructor]
def __init__(self, 
             problem: Problem, 
             global_discretization, 
             elementary_matrices, 
             domain_index: int = 0)
\end{lstlisting}

\textbf{Requirements:} \texttt{problem.neq == 4} and \texttt{problem.type == "organ\_on\_chip"}

\textbf{Usage:}
\begin{lstlisting}[language=Python, caption=OrganOnChip Constructor Usage]
# Create OrganOnChip problem (from MATLAB TestProblem.m)
ooc_params = np.array([1.0, 2.0, 1.0, 1.0, 0.0, 1.0, 0.0, 1.0, 1.0])
problem = Problem(
    neq=4,
    domain_start=0.0,  # A = 0
    domain_length=1.0, # L = 1
    parameters=ooc_params,  # [nu, mu, epsilon, sigma, a, b, c, d, chi]
    problem_type="organ_on_chip"
)

# Create static condensation
ooc_sc = StaticCondensationOOC(
    problem=problem,
    global_discretization=global_disc,
    elementary_matrices=elementary_matrices,
    domain_index=0
)
\end{lstlisting}

%\subsubsection{Specific Attributes (From MATLAB TestProblem.m)}
%
%\begin{longtable}{|p{3.5cm}|p{3.5cm}|p{7cm}|}
%\hline
%\textbf{Attribute} & \textbf{Type} & \textbf{Description} \\
%\hline
%\endhead
%
%\texttt{nu} & \texttt{float} & Viscosity parameter (parameter 0) \\
%\hline
%
%\texttt{mu} & \texttt{float} & Viscosity parameter (parameter 1) \\
%\hline
%
%\texttt{epsilon} & \texttt{float} & Coupling parameter (parameter 2) \\
%\hline
%
%\texttt{sigma} & \texttt{float} & Coupling parameter (parameter 3) \\
%\hline
%
%\texttt{a} & \texttt{float} & Reaction parameter (parameter 4) \\
%\hline
%
%\texttt{b} & \texttt{float} & Inter-equation coupling (parameter 5) \\
%\hline
%
%\texttt{c} & \texttt{float} & Reaction parameter (parameter 6) \\
%\hline
%
%\texttt{d} & \texttt{float} & Inter-equation coupling (parameter 7) \\
%\hline
%
%\texttt{chi} & \texttt{float} & Coupling strength (parameter 8) \\
%\hline
%
%\texttt{lambda\_function} & \texttt{Optional[Callable]} & Nonlinear response $\lambda(\omega)$ \\
%\hline
%
%\texttt{dlambda\_function} & \texttt{Optional[Callable]} & Derivative $\lambda'(\omega)$ \\
%\hline
%
%\end{longtable}
%
%\subsubsection{Mathematical System (From MATLAB References)}
%
%The OrganOnChip system implements:
%\begin{align}
%\frac{\partial u}{\partial t} - \nu \nabla^2 u &= f_u \label{eq:ooc_u} \\
%\frac{\partial \omega}{\partial t} + \epsilon \nabla \cdot \theta + c\omega &= f_\omega + d u \label{eq:ooc_omega} \\
%\frac{\partial v}{\partial t} + \sigma \nabla \cdot q + \lambda(\bar{\omega}) v &= f_v \label{eq:ooc_v} \\
%\frac{\partial \phi}{\partial t} + \mu \nabla \cdot \psi + a\phi &= f_\phi + b v \label{eq:ooc_phi}
%\end{align}
%
%With auxiliary relations:
%\begin{align}
%\theta &= \epsilon(\nabla \omega - \hat{\omega}) \\
%q &= \sigma(\nabla v - \hat{v}) \\
%\psi &= \mu(\nabla \phi - \hat{\phi})
%\end{align}

\subsubsection{Matrix Construction (Based on MATLAB scBlocks.m)}

\paragraph{build\_matrices()}\leavevmode
\begin{lstlisting}[language=Python, caption=OrganOnChip Build Matrices]
def build_matrices(self) -> dict
\end{lstlisting}

\textbf{Returns:} Dictionary with OrganOnChip-specific matrices

\textbf{Matrix Dictionary Keys (From MATLAB scBlocks.m):}
\begin{itemize}
    \item \texttt{'M'}: Mass matrix
    \item \texttt{'T'}: Trace matrix
    \item \texttt{'B1'}, \texttt{'L1'}: Matrices for $u$ equation
    \item \texttt{'B2'}, \texttt{'C2'}, \texttt{'L2'}: Matrices for $\omega$ equation  
    \item \texttt{'A3'}, \texttt{'S3'}, \texttt{'H3'}: Matrices for $v$ equation
    \item \texttt{'B4'}, \texttt{'C4'}, \texttt{'L4'}: Matrices for $\phi$ equation
    \item \texttt{'D1'}, \texttt{'D2'}: Flux jump matrices
    \item \texttt{'Q'}: Coupling matrix
    \item \texttt{'hB4'}, \texttt{'B5'}, \texttt{'B6'}, \texttt{'B7'}: Assembly matrices
    \item \texttt{'hatB0'}, \texttt{'hatB1'}, \texttt{'hatB2'}: Final assembly matrices
    \item \texttt{'Av'}: Averaging matrix for $\bar{\omega}$ computation
\end{itemize}

\paragraph{static\_condensation()}\leavevmode
\begin{lstlisting}[language=Python, caption=OrganOnChip Static Condensation]
def static_condensation(self, 
                       trace_values: np.ndarray, 
                       rhs: np.ndarray, 
                       time: float = 0.0) -> Tuple[np.ndarray, np.ndarray, np.ndarray]
\end{lstlisting}

\textbf{Algorithm (Following MATLAB StaticC.m):}
\begin{enumerate}
    \item \textbf{Step 1}: $\hat{U} \rightarrow U$ reconstruction
    \begin{itemize}
        \item $u = B_1 \hat{u} + y_1$ where $y_1 = L_1 g_u$
        \item $\omega = C_2 \hat{u} + B_2 \hat{\omega} + y_2$ where $y_2 = L_2(g_\omega + dt \cdot d \cdot M y_1)$
    \end{itemize}
    
    \item \textbf{Step 2}: Nonlinear coupling computation
    \begin{itemize}
        \item $\bar{\omega} = A_{av} \omega$ (averaging)
        \item $\bar{\lambda}_\omega = \lambda(\bar{\omega})$ (nonlinear function evaluation)
        \item $J_\lambda = \lambda'(\bar{\omega}) A_{av}$ (Jacobian contribution)
    \end{itemize}
    
    \item \textbf{Step 3}: $v$ equation solution
    \begin{itemize}
        \item $L_3(\omega) = (A_3 + \bar{\lambda}_\omega S_3)^{-1}$ (nonlinear operator)
        \item $v = B_3(\omega) \hat{v} + y_3(\omega)$ where $B_3(\omega) = L_3(\omega) H_3$
    \end{itemize}
    
    \item \textbf{Step 4}: $\phi$ equation solution
    \begin{itemize}
        \item $\phi = B_4 \hat{\phi} + C_4 v + L_4 g_\phi$
    \end{itemize}
    
    \item \textbf{Step 5}: Flux jump computation
    \begin{itemize}
        \item $\tilde{J} = D_1 U - D_2 \hat{U}$
        \item $j = \hat{B}_4 \hat{U} + \tilde{J}^T Q U$
    \end{itemize}
    
    \item \textbf{Step 6}: Final assembly
    \begin{itemize}
        \item $\hat{j} = B_5 j + B_6 U + B_7 \hat{U}$
        \item $\hat{J} = \hat{B}_0 \tilde{J} + \hat{B}_1 U - \hat{B}_2 \hat{U}$
    \end{itemize}
\end{enumerate}

\textbf{Usage:}
\begin{lstlisting}[language=Python, caption=OrganOnChip Static Condensation Usage]
# Prepare input (4 equations, matching MATLAB TestProblem.m)
trace_vals = np.random.rand(4 * (n_elements + 1))  # u, omega, v, phi traces
rhs = np.random.rand(2 * 4 * n_elements)  # RHS for all 4 equations

# Set lambda function (from MATLAB TestProblem.m: constant function)
ooc_sc.set_lambda_functions(
    lambda_function=lambda omega: np.ones_like(omega),
    dlambda_function=lambda omega: np.zeros_like(omega)
)

# Perform static condensation
bulk_solution, flux_jump, jacobian = ooc_sc.static_condensation(
    trace_values=trace_vals,
    rhs=rhs,
    time=0.0
)

print(f"Bulk solution shape: {bulk_solution.shape}")  # (8, n_elements)
print(f"Flux jump shape: {flux_jump.shape}")  # (4,) 
print(f"Jacobian shape: {jacobian.shape}")  # (8, 4*(n_elements+1))
\end{lstlisting}


\subsection{StaticCondensationFactory Usage}
\label{subsec:factory_usage}

The factory class provides automatic selection of appropriate static condensation implementation.

\subsubsection{Factory Methods}

\paragraph{create()}\leavevmode
\begin{lstlisting}[language=Python, caption=Factory Create Method]
@classmethod
def create(cls, problem: Problem, global_disc, elementary_matrices,
           i: int = 0) -> StaticCondensationBase
\end{lstlisting}

\textbf{Parameters:}
\begin{itemize}
    \item \texttt{problem}: Problem instance with \texttt{type} attribute
    \item \texttt{global\_disc}: GlobalDiscretization instance
    \item \texttt{elementary\_matrices}: ElementaryMatrices instance
    \item \texttt{i}: Domain index (default: 0)
\end{itemize}

\textbf{Returns:} Appropriate StaticCondensation implementation

\textbf{Usage:}
\begin{lstlisting}[language=Python, caption=Factory Usage Examples]
from ooc1d.core.static_condensation_factory import StaticCondensationFactory

# Automatic selection for Keller-Segel
ks_problem = Problem(neq=2, ..., problem_type="keller_segel")
ks_sc = StaticCondensationFactory.create(
    problem=ks_problem,
    global_disc=global_discretization,
    elementary_matrices=elem_matrices,
    i=0
)
# Returns: KellerSegelStaticCondensation instance

# Automatic selection for OrganOnChip  
ooc_problem = Problem(neq=4, ..., problem_type="organ_on_chip")
ooc_sc = StaticCondensationFactory.create(
    problem=ooc_problem,
    global_disc=global_discretization,
    elementary_matrices=elem_matrices,
    i=0
)
# Returns: StaticCondensationOOC instance
\end{lstlisting}

\paragraph{register\_implementation()}\leavevmode
\begin{lstlisting}[language=Python, caption=Register Implementation Method]
@classmethod
def register_implementation(cls, problem_type: str, 
                           implementation_class: Type[StaticCondensationBase])
\end{lstlisting}

\textbf{Purpose:} Register new static condensation implementations

\textbf{Usage:}
\begin{lstlisting}[language=Python, caption=Register New Implementation]
# Register custom implementation
class CustomStaticCondensation(StaticCondensationBase):
    # ... implementation ...
    pass

StaticCondensationFactory.register_implementation(
    "custom_problem", CustomStaticCondensation
)

# Now available through factory
custom_problem = Problem(neq=3, ..., problem_type="custom_problem")
custom_sc = StaticCondensationFactory.create(custom_problem, ...)
\end{lstlisting}

\subsection{Complete Usage Examples}
\label{subsec:complete_usage_examples}

\subsubsection{Keller-Segel Complete Workflow}

\begin{lstlisting}[language=Python, caption=Complete KellerSegel Workflow]
import numpy as np
from ooc1d.core.problem import Problem
from ooc1d.core.discretization import Discretization, GlobalDiscretization
from ooc1d.utils.elementary_matrices import ElementaryMatrices
from ooc1d.core.static_condensation_factory import StaticCondensationFactory

# Step 1: Create Keller-Segel problem
problem = Problem(
    neq=2,
    domain_start=0.0,
    domain_length=1.0,
    parameters=np.array([2.0, 1.0, 0.1, 1.0]),  # [mu, nu, a, b]
    problem_type="keller_segel",
    name="chemotaxis_problem"
)

# Set chemotaxis functions
problem.set_chemotaxis(
    chi=lambda phi: np.ones_like(phi),
    dchi=lambda phi: np.zeros_like(phi)
)

# Step 2: Create discretization
discretization = Discretization(n_elements=20)
discretization.set_tau([1.0, 1.0])  # [tau_u, tau_phi]
global_disc = GlobalDiscretization([discretization])

# Step 3: Create elementary matrices
elementary_matrices = ElementaryMatrices()

# Step 4: Create static condensation via factory
static_condensation = StaticCondensationFactory.create(
    problem=problem,
    global_disc=global_disc,
    elementary_matrices=elementary_matrices,
    i=0
)

print(f"Created: {type(static_condensation).__name__}")  # KellerSegelStaticCondensation

# Step 5: Build matrices
matrices = static_condensation.build_matrices()
print(f"Available matrices: {list(matrices.keys())}")

# Step 6: Perform static condensation
trace_values = np.random.rand(2 * 21)  # 2 equations, 21 nodes
rhs = np.random.rand(2 * 2 * 20)  # 2 equations, 2 DOFs per element, 20 elements

bulk_solution, flux_jump, jacobian = static_condensation.static_condensation(
    trace_values=trace_values,
    rhs=rhs,
    time=0.0
)

print(f"Static condensation completed:")
print(f"  Bulk solution: {bulk_solution.shape}")
print(f"  Flux jump: {flux_jump.shape}")
print(f"  Jacobian: {jacobian.shape}")
\end{lstlisting}

\subsubsection{OrganOnChip Complete Workflow (MATLAB Compatible)}

\begin{lstlisting}[language=Python, caption=Complete OrganOnChip Workflow]
# Step 1: Create OrganOnChip problem (matching MATLAB TestProblem.m)
ooc_params = np.array([1.0, 2.0, 1.0, 1.0, 0.0, 1.0, 0.0, 1.0, 1.0])
# [nu, mu, epsilon, sigma, a, b, c, d, chi]

problem = Problem(
    neq=4,
    domain_start=0.0,  # A = 0 (MATLAB)
    domain_length=1.0, # L = 1 (MATLAB)
    parameters=ooc_params,
    problem_type="organ_on_chip",
    name="microfluidic_device"
)

# Set initial conditions (matching MATLAB TestProblem.m)
problem.set_initial_condition(0, lambda x, t: np.sin(2*np.pi*x))  # u
for eq_idx in [1, 2, 3]:  # omega, v, phi
    problem.set_initial_condition(eq_idx, lambda x, t: np.zeros_like(x))

# Set lambda function (matching MATLAB: constant_function)
problem.set_function('lambda_function', lambda x: np.ones_like(x))
problem.set_function('dlambda_function', lambda x: np.zeros_like(x))

# Step 2: Create discretization
discretization = Discretization(n_elements=40)  # Matching MATLAB
discretization.set_tau([1.0, 1.0, 1.0, 1.0])  # [tu, to, tv, tp]
global_disc = GlobalDiscretization([discretization])
global_disc.set_time_parameters(dt=0.01, T=0.5)  # Matching MATLAB

# Step 3: Create static condensation
elementary_matrices = ElementaryMatrices()
static_condensation = StaticCondensationFactory.create(
    problem=problem,
    global_disc=global_disc,
    elementary_matrices=elementary_matrices,
    i=0
)

print(f"Created: {type(static_condensation).__name__}")  # StaticCondensationOOC

# Set lambda functions for nonlinear coupling
static_condensation.set_lambda_functions(
    lambda_function=lambda omega: np.ones_like(omega),  # Constant (MATLAB)
    dlambda_function=lambda omega: np.zeros_like(omega)
)

# Step 4: Build matrices (following MATLAB scBlocks.m structure)
matrices = static_condensation.build_matrices()
required_matrices = ['M', 'T', 'B1', 'L1', 'B2', 'C2', 'L2', 'A3', 'S3', 'H3', 
                    'B4', 'C4', 'L4', 'D1', 'D2', 'Q', 'Av']
print(f"Required matrices available: {all(m in matrices for m in required_matrices)}")

# Step 5: Perform static condensation (following MATLAB StaticC.m)
trace_values = np.random.rand(4 * 41)  # 4 equations, 41 nodes
rhs = np.random.rand(2 * 4 * 40)  # 4 equations, 2 DOFs per element, 40 elements

bulk_solution, flux_jump, jacobian = static_condensation.static_condensation(
    trace_values=trace_values,
    rhs=rhs,
    time=0.0
)

# Verify MATLAB compatibility
print(f"OrganOnChip static condensation results:")
print(f"  Bulk solution (U): {bulk_solution.shape}")  # Should be (8, 40)
print(f"  Flux jump (hJ): {flux_jump.shape}")  # Should be (4,) or (8,)
print(f"  Jacobian (dhJ): {jacobian.shape}")  # Should be (8, 4*41)

# Check parameter extraction matches MATLAB
params = static_condensation.get_problem_parameters()
matlab_params = {'nu': params[0], 'mu': params[1], 'epsilon': params[2], 
                'sigma': params[3], 'a': params[4], 'b': params[5], 
                'c': params[6], 'd': params[7], 'chi': params[8]}
print(f"MATLAB-compatible parameters: {matlab_params}")
\end{lstlisting}

\subsection{Method Summary Table}
\label{subsec:static_condensation_method_summary}

\subsubsection{StaticCondensationBase Methods}

\begin{longtable}{|p{5.5cm}|p{2cm}|p{7cm}|}
\hline
\textbf{Method} & \textbf{Returns} & \textbf{Purpose} \\
\hline
\endhead

\texttt{\_\_init\_\_} & \texttt{None} & Initialize base class with common attributes \\
\hline

\texttt{build\_matrices} & \texttt{dict} & Abstract: build problem-specific matrices \\
\hline

\texttt{static\_condensation} & \texttt{Tuple} & Abstract: perform local static condensation \\
\hline

\texttt{get\_problem\_parameters} & \texttt{np.ndarray} & Extract problem parameter array \\
\hline

\texttt{get\_stabilization\_parameters} & \texttt{np.ndarray} & Extract stabilization parameters \\
\hline

\texttt{validate\_input} & \texttt{None} & Validate trace values and RHS dimensions \\
\hline

\end{longtable}

\subsubsection{KellerSegelStaticCondensation Methods}

\begin{longtable}{|p{4.7cm}|p{2cm}|p{7cm}|}
\hline
\textbf{Method} & \textbf{Returns} & \textbf{Purpose} \\
\hline
\endhead

\texttt{build\_matrices} & \texttt{dict} & Build 2-equation Keller-Segel matrices \\
\hline

\texttt{static\_condensation} & \texttt{Tuple} & Perform Keller-Segel static condensation \\
\hline

\texttt{set\_chemotaxis\_functions} & \texttt{None} & Set nonlinear chemotaxis functions \\
\hline

\end{longtable}

\subsubsection{StaticCondensationOOC Methods}

\begin{longtable}{|p{4cm}|p{2cm}|p{7cm}|}
\hline
\textbf{Method} & \textbf{Returns} & \textbf{Purpose} \\
\hline
\endhead

\texttt{build\_matrices} & \texttt{dict} & Build 4-equation OrganOnChip matrices (MATLAB compatible) \\
\hline

\texttt{static\_condensation} & \texttt{Tuple} & Perform OrganOnChip static condensation (MATLAB StaticC.m) \\
\hline

\texttt{set\_lambda\_functions} & \texttt{None} & Set nonlinear response functions $\lambda(\omega)$ \\
\hline

\end{longtable}

This documentation provides a comprehensive reference for the static condensation module hierarchy, with implementations that follow the MATLAB reference files while integrating with the BioNetFlux Python architecture.

% End of static condensation modules API documentation


\section{Static Condensation Modules API Reference}
\label{sec:static_condensation_modules_api}

The static condensation module is where the  problem specific HDG discretization happens. This section provides a comprehensive reference for the static condensation module hierarchy based on analysis of the MATLAB reference files and BioNetFlux architecture patterns. The modules implement HDG static condensation for different problem types.

\subsection{Module Overview}

The static condensation system consists of:
\begin{itemize}
    \item \textbf{StaticCondensationBase}: Abstract base class defining the interface
    \item \textbf{Derived classes}
   \begin{itemize}
    \item \textbf{KellerSegelStaticCondensation}: Implementation for 2-equation chemotaxis systems
    \item \textbf{StaticCondensationOOC}: Implementation for 4-equation OrganOnChip systems
   \end{itemize}
    \item \textbf{StaticCondensationFactory}: Factory for creating appropriate implementations
\end{itemize}

%\subsection{MATLAB Reference Analysis}
%
%Based on the MATLAB files, the static condensation process follows this structure:
%
%\textbf{From StaticC.m:} The main static condensation function \texttt{[U,hJ,dhJ] = StaticC(hU,rhs,problem,discretization,scMatrices)} performs:
%\begin{enumerate}
%    \item Local trace-to-bulk reconstruction: $\hat{U} \rightarrow U$
%    \item Flux jump computation: $\tilde{J} = D_1 U - D_2 \hat{U}$
%    \item Normal flux computation: $j = \hat{B}_4 \hat{U} + \tilde{J}^T Q U$
%    \item Final assembly: $\hat{J} = [\hat{j}; \hat{J}]$
%\end{enumerate}
%
%\textbf{From scBlocks.m:} The matrix construction involves building operators for each step of the 4-equation OrganOnChip system.

\subsection{StaticCondensationBase Class}
\label{subsec:static_condensation_base}

Abstract base class defining the interface for all static condensation implementations.

\subsubsection{Constructor}

\paragraph{\_\_init\_\_()}\leavevmode
\begin{lstlisting}[language=Python, caption=StaticCondensationBase Constructor]
def __init__(self, 
             problem: Problem, 
             global_discretization, 
             elementary_matrices, 
             domain_index: int = 0)
\end{lstlisting}

\textbf{Parameters:}
\begin{itemize}
    \item \texttt{problem}: Problem instance with equations and parameters
    \item \texttt{global\_discretization}: GlobalDiscretization instance
    \item \texttt{elementary\_matrices}: ElementaryMatrices instance
    \item \texttt{domain\_index}: Domain index for multi-domain problems (default: 0)
\end{itemize}

\textbf{Side Effects:} Initializes base attributes and extracts domain-specific discretization

\subsubsection{Core Attributes}

\begin{longtable}{|p{3.8cm}|p{3.5cm}|p{7cm}|}
\hline
\textbf{Attribute} & \textbf{Type} & \textbf{Description} \\
\hline
\endhead

\texttt{problem} & \texttt{Problem} & Problem definition with parameters and equations \\
\hline

\texttt{discretization} & \texttt{Discretization} & Domain-specific discretization \\
\hline

\texttt{elementary\_matrices} & \texttt{ElementaryMatrices} & Pre-computed elementary matrices \\
\hline

\texttt{domain\_index} & \texttt{int} & Domain index for multi-domain systems \\
\hline

\texttt{neq} & \texttt{int} & Number of equations (extracted from problem) \\
\hline

\texttt{n\_elements} & \texttt{int} & Number of elements (extracted from discretization) \\
\hline

\texttt{element\_length} & \texttt{float} & Element size: \texttt{h = domain\_length / n\_elements} \\
\hline

\texttt{dt} & \texttt{Optional[float]} & Time step size (from discretization if available) \\
\hline

\end{longtable}

\subsubsection{Abstract Methods (Must be Implemented by Subclasses)}

\paragraph{build\_matrices()}
\begin{lstlisting}[language=Python, caption=Abstract Build Matrices Method]
@abstractmethod
def build_matrices(self) -> dict
\end{lstlisting}

\textbf{Returns:} \texttt{dict} - Dictionary of static condensation matrices

\textbf{Purpose:} Build all matrices required for static condensation (problem-specific)

\paragraph{static\_condensation()}\leavevmode
\begin{lstlisting}[language=Python, caption=Abstract Static Condensation Method]
@abstractmethod
def static_condensation(self, 
                       trace_values: np.ndarray, 
                       rhs: np.ndarray, 
                       time: float = 0.0) -> Tuple[np.ndarray, np.ndarray, np.ndarray]
\end{lstlisting}

\textbf{Parameters:}
\begin{itemize}
    \item \texttt{trace\_values}: Local trace values $\hat{U}$ (size: \texttt{neq*(n\_elements+1)})
    \item \texttt{rhs}: Right-hand side vector (size: \texttt{2*neq*n\_elements})
    \item \texttt{time}: Current time for time-dependent problems (default: 0.0)
\end{itemize}

\textbf{Returns:} \texttt{Tuple[np.ndarray, np.ndarray, np.ndarray]} - (bulk\_solution, flux\_jump, jacobian)

\textbf{Purpose:} Perform local static condensation step

\subsubsection{Concrete Methods (Provided by Base Class)}

\paragraph{get\_problem\_parameters()}\leavevmode
\begin{lstlisting}[language=Python, caption=Get Problem Parameters Method]
def get_problem_parameters(self) -> np.ndarray
\end{lstlisting}

\textbf{Returns:} \texttt{np.ndarray} - Problem parameters array

\textbf{Usage:}
\begin{lstlisting}[language=Python, caption=Get Parameters Usage]
params = static_condensation.get_problem_parameters()
# For OrganOnChip: [nu, mu, epsilon, sigma, a, b, c, d, chi]
# For Keller-Segel: [mu, nu, a, b]
\end{lstlisting}

\paragraph{get\_stabilization\_parameters()}\leavevmode
\begin{lstlisting}[language=Python, caption=Get Stabilization Parameters Method]
def get_stabilization_parameters(self) -> np.ndarray
\end{lstlisting}

\textbf{Returns:} \texttt{np.ndarray} - Stabilization parameters $\tau$ for each equation

\textbf{Usage:}
\begin{lstlisting}[language=Python, caption=Get Stabilization Usage]
tau_values = static_condensation.get_stabilization_parameters()
# For OrganOnChip: [tu, to, tv, tp] (4 equations)
# For Keller-Segel: [tu, tp] (2 equations)
\end{lstlisting}

\paragraph{validate\_input()}\leavevmode
\begin{lstlisting}[language=Python, caption=Validate Input Method]
def validate_input(self, 
                  trace_values: np.ndarray, 
                  rhs: np.ndarray) -> None
\end{lstlisting}

\textbf{Parameters:}
\begin{itemize}
    \item \texttt{trace\_values}: Trace values to validate
    \item \texttt{rhs}: Right-hand side to validate
\end{itemize}

\textbf{Returns:} \texttt{None}

\textbf{Raises:} \texttt{ValueError} for invalid input dimensions or content

\textbf{Validation Checks:}
\begin{itemize}
    \item Trace values size: \texttt{neq * (n\_elements + 1)}
    \item RHS size: \texttt{2 * neq * n\_elements}
    \item No NaN or infinite values
\end{itemize}

\subsection{KellerSegelStaticCondensation Class}
\label{subsec:keller_segel_static_condensation}

Implementation for 2-equation Keller-Segel chemotaxis systems (u, $\phi$).

\subsubsection{Constructor}

\paragraph{\_\_init\_\_()}\leavevmode
\begin{lstlisting}[language=Python, caption=KellerSegel Constructor]
def __init__(self, 
             problem: Problem, 
             global_discretization, 
             elementary_matrices, 
             domain_index: int = 0)
\end{lstlisting}

\textbf{Requirements:} \texttt{problem.neq == 2} and \texttt{problem.type == "keller\_segel"}

\textbf{Usage:}
\begin{lstlisting}[language=Python, caption=KellerSegel Constructor Usage]
# Create Keller-Segel problem
problem = Problem(
    neq=2,
    domain_start=0.0,
    domain_length=1.0,
    parameters=np.array([2.0, 1.0, 0.1, 1.0]),  # [mu, nu, a, b]
    problem_type="keller_segel"
)

# Create static condensation
ks_sc = KellerSegelStaticCondensation(
    problem=problem,
    global_discretization=global_disc,
    elementary_matrices=elementary_matrices,
    domain_index=0
)
\end{lstlisting}

\subsubsection{Specific Attributes}

\begin{longtable}{|p{3.2cm}|p{3.5cm}|p{7cm}|}
\hline
\textbf{Attribute} & \textbf{Type} & \textbf{Description} \\
\hline
\endhead

\texttt{mu} & \texttt{float} & Diffusion coefficient for $\phi$ equation (parameter 0) \\
\hline

\texttt{nu} & \texttt{float} & Diffusion coefficient for $u$ equation (parameter 1) \\
\hline

\texttt{a} & \texttt{float} & Reaction parameter (parameter 2) \\
\hline

\texttt{b} & \texttt{float} & Coupling parameter (parameter 3) \\
\hline

\texttt{chi\_function} & \texttt{Optional[Callable]} & Chemotactic sensitivity $\chi(\phi)$ \\
\hline

\texttt{dchi\_function} & \texttt{Optional[Callable]} & Derivative $\chi'(\phi)$ \\
\hline

\end{longtable}

\subsubsection{Matrix Construction}

\paragraph{build\_matrices()}\leavevmode
\begin{lstlisting}[language=Python, caption=KellerSegel Build Matrices]
def build_matrices(self) -> dict
\end{lstlisting}

\textbf{Returns:} \texttt{dict} - Dictionary with matrices for 2-equation system

\textbf{Matrix Dictionary Keys:}
\begin{itemize}
    \item \texttt{'M'}: Mass matrix
    \item \texttt{'T'}: Trace matrix  
    \item \texttt{'D'}: Differentiation matrix
    \item \texttt{'B1'}, \texttt{'L1'}: Matrices for $u$ equation reconstruction
    \item \texttt{'B2'}, \texttt{'L2'}: Matrices for $\phi$ equation reconstruction
    \item \texttt{'Q'}: Coupling matrix for chemotaxis
\end{itemize}

\textbf{Mathematical Formulation:}
For Keller-Segel system:
\begin{align}
\frac{\partial u}{\partial t} - \nu \nabla^2 u + \nabla \cdot (u \chi(\phi) \nabla \phi) &= f_u \\
\frac{\partial \phi}{\partial t} - \mu \nabla^2 \phi &= f_\phi + a u
\end{align}

\paragraph{static\_condensation()}\leavevmode
\begin{lstlisting}[language=Python, caption=KellerSegel Static Condensation]
def static_condensation(self, 
                       trace_values: np.ndarray, 
                       rhs: np.ndarray, 
                       time: float = 0.0) -> Tuple[np.ndarray, np.ndarray, np.ndarray]
\end{lstlisting}

\textbf{Algorithm:} 
\begin{enumerate}
    \item Extract trace values for $u$ and $\phi$
    \item Reconstruct bulk solutions: $u = B_1 \hat{u} + L_1 g_u$, $\phi = B_2 \hat{\phi} + L_2 g_\phi$
    \item Compute chemotaxis terms if $\chi$ functions available
    \item Compute flux jumps and Jacobian contributions
    \item Return bulk solutions, flux jumps, and Jacobian
\end{enumerate}

\textbf{Usage:}
\begin{lstlisting}[language=Python, caption=KellerSegel Static Condensation Usage]
# Prepare input
trace_vals = np.random.rand(2 * (n_elements + 1))  # u and phi traces
rhs = np.random.rand(2 * 2 * n_elements)  # RHS for both equations

# Perform static condensation
bulk_solution, flux_jump, jacobian = ks_sc.static_condensation(
    trace_values=trace_vals,
    rhs=rhs,
    time=0.5
)

print(f"Bulk solution shape: {bulk_solution.shape}")  # (4, n_elements)
print(f"Flux jump shape: {flux_jump.shape}")  # (2,)
print(f"Jacobian shape: {jacobian.shape}")  # (4, 2*(n_elements+1))
\end{lstlisting}


\subsection{StaticCondensationOOC Class}
\label{subsec:static_condensation_ooc}

Implementation for 4-equation OrganOnChip systems (u, $\omega$, v, $\phi$) based on MATLAB StaticC.m.

\subsubsection{Constructor}

\paragraph{\_\_init\_\_()}\leavevmode
\begin{lstlisting}[language=Python, caption=StaticCondensationOOC Constructor]
def __init__(self, 
             problem: Problem, 
             global_discretization, 
             elementary_matrices, 
             domain_index: int = 0)
\end{lstlisting}

\textbf{Requirements:} \texttt{problem.neq == 4} and \texttt{problem.type == "organ\_on\_chip"}

\textbf{Usage:}
\begin{lstlisting}[language=Python, caption=OrganOnChip Constructor Usage]
# Create OrganOnChip problem (from MATLAB TestProblem.m)
ooc_params = np.array([1.0, 2.0, 1.0, 1.0, 0.0, 1.0, 0.0, 1.0, 1.0])
problem = Problem(
    neq=4,
    domain_start=0.0,  # A = 0
    domain_length=1.0, # L = 1
    parameters=ooc_params,  # [nu, mu, epsilon, sigma, a, b, c, d, chi]
    problem_type="organ_on_chip"
)

# Create static condensation
ooc_sc = StaticCondensationOOC(
    problem=problem,
    global_discretization=global_disc,
    elementary_matrices=elementary_matrices,
    domain_index=0
)
\end{lstlisting}

%\subsubsection{Specific Attributes (From MATLAB TestProblem.m)}
%
%\begin{longtable}{|p{3.5cm}|p{3.5cm}|p{7cm}|}
%\hline
%\textbf{Attribute} & \textbf{Type} & \textbf{Description} \\
%\hline
%\endhead
%
%\texttt{nu} & \texttt{float} & Viscosity parameter (parameter 0) \\
%\hline
%
%\texttt{mu} & \texttt{float} & Viscosity parameter (parameter 1) \\
%\hline
%
%\texttt{epsilon} & \texttt{float} & Coupling parameter (parameter 2) \\
%\hline
%
%\texttt{sigma} & \texttt{float} & Coupling parameter (parameter 3) \\
%\hline
%
%\texttt{a} & \texttt{float} & Reaction parameter (parameter 4) \\
%\hline
%
%\texttt{b} & \texttt{float} & Inter-equation coupling (parameter 5) \\
%\hline
%
%\texttt{c} & \texttt{float} & Reaction parameter (parameter 6) \\
%\hline
%
%\texttt{d} & \texttt{float} & Inter-equation coupling (parameter 7) \\
%\hline
%
%\texttt{chi} & \texttt{float} & Coupling strength (parameter 8) \\
%\hline
%
%\texttt{lambda\_function} & \texttt{Optional[Callable]} & Nonlinear response $\lambda(\omega)$ \\
%\hline
%
%\texttt{dlambda\_function} & \texttt{Optional[Callable]} & Derivative $\lambda'(\omega)$ \\
%\hline
%
%\end{longtable}
%
%\subsubsection{Mathematical System (From MATLAB References)}
%
%The OrganOnChip system implements:
%\begin{align}
%\frac{\partial u}{\partial t} - \nu \nabla^2 u &= f_u \label{eq:ooc_u} \\
%\frac{\partial \omega}{\partial t} + \epsilon \nabla \cdot \theta + c\omega &= f_\omega + d u \label{eq:ooc_omega} \\
%\frac{\partial v}{\partial t} + \sigma \nabla \cdot q + \lambda(\bar{\omega}) v &= f_v \label{eq:ooc_v} \\
%\frac{\partial \phi}{\partial t} + \mu \nabla \cdot \psi + a\phi &= f_\phi + b v \label{eq:ooc_phi}
%\end{align}
%
%With auxiliary relations:
%\begin{align}
%\theta &= \epsilon(\nabla \omega - \hat{\omega}) \\
%q &= \sigma(\nabla v - \hat{v}) \\
%\psi &= \mu(\nabla \phi - \hat{\phi})
%\end{align}

\subsubsection{Matrix Construction (Based on MATLAB scBlocks.m)}

\paragraph{build\_matrices()}\leavevmode
\begin{lstlisting}[language=Python, caption=OrganOnChip Build Matrices]
def build_matrices(self) -> dict
\end{lstlisting}

\textbf{Returns:} Dictionary with OrganOnChip-specific matrices

\textbf{Matrix Dictionary Keys (From MATLAB scBlocks.m):}
\begin{itemize}
    \item \texttt{'M'}: Mass matrix
    \item \texttt{'T'}: Trace matrix
    \item \texttt{'B1'}, \texttt{'L1'}: Matrices for $u$ equation
    \item \texttt{'B2'}, \texttt{'C2'}, \texttt{'L2'}: Matrices for $\omega$ equation  
    \item \texttt{'A3'}, \texttt{'S3'}, \texttt{'H3'}: Matrices for $v$ equation
    \item \texttt{'B4'}, \texttt{'C4'}, \texttt{'L4'}: Matrices for $\phi$ equation
    \item \texttt{'D1'}, \texttt{'D2'}: Flux jump matrices
    \item \texttt{'Q'}: Coupling matrix
    \item \texttt{'hB4'}, \texttt{'B5'}, \texttt{'B6'}, \texttt{'B7'}: Assembly matrices
    \item \texttt{'hatB0'}, \texttt{'hatB1'}, \texttt{'hatB2'}: Final assembly matrices
    \item \texttt{'Av'}: Averaging matrix for $\bar{\omega}$ computation
\end{itemize}

\paragraph{static\_condensation()}\leavevmode
\begin{lstlisting}[language=Python, caption=OrganOnChip Static Condensation]
def static_condensation(self, 
                       trace_values: np.ndarray, 
                       rhs: np.ndarray, 
                       time: float = 0.0) -> Tuple[np.ndarray, np.ndarray, np.ndarray]
\end{lstlisting}

\textbf{Algorithm (Following MATLAB StaticC.m):}
\begin{enumerate}
    \item \textbf{Step 1}: $\hat{U} \rightarrow U$ reconstruction
    \begin{itemize}
        \item $u = B_1 \hat{u} + y_1$ where $y_1 = L_1 g_u$
        \item $\omega = C_2 \hat{u} + B_2 \hat{\omega} + y_2$ where $y_2 = L_2(g_\omega + dt \cdot d \cdot M y_1)$
    \end{itemize}
    
    \item \textbf{Step 2}: Nonlinear coupling computation
    \begin{itemize}
        \item $\bar{\omega} = A_{av} \omega$ (averaging)
        \item $\bar{\lambda}_\omega = \lambda(\bar{\omega})$ (nonlinear function evaluation)
        \item $J_\lambda = \lambda'(\bar{\omega}) A_{av}$ (Jacobian contribution)
    \end{itemize}
    
    \item \textbf{Step 3}: $v$ equation solution
    \begin{itemize}
        \item $L_3(\omega) = (A_3 + \bar{\lambda}_\omega S_3)^{-1}$ (nonlinear operator)
        \item $v = B_3(\omega) \hat{v} + y_3(\omega)$ where $B_3(\omega) = L_3(\omega) H_3$
    \end{itemize}
    
    \item \textbf{Step 4}: $\phi$ equation solution
    \begin{itemize}
        \item $\phi = B_4 \hat{\phi} + C_4 v + L_4 g_\phi$
    \end{itemize}
    
    \item \textbf{Step 5}: Flux jump computation
    \begin{itemize}
        \item $\tilde{J} = D_1 U - D_2 \hat{U}$
        \item $j = \hat{B}_4 \hat{U} + \tilde{J}^T Q U$
    \end{itemize}
    
    \item \textbf{Step 6}: Final assembly
    \begin{itemize}
        \item $\hat{j} = B_5 j + B_6 U + B_7 \hat{U}$
        \item $\hat{J} = \hat{B}_0 \tilde{J} + \hat{B}_1 U - \hat{B}_2 \hat{U}$
    \end{itemize}
\end{enumerate}

\textbf{Usage:}
\begin{lstlisting}[language=Python, caption=OrganOnChip Static Condensation Usage]
# Prepare input (4 equations, matching MATLAB TestProblem.m)
trace_vals = np.random.rand(4 * (n_elements + 1))  # u, omega, v, phi traces
rhs = np.random.rand(2 * 4 * n_elements)  # RHS for all 4 equations

# Set lambda function (from MATLAB TestProblem.m: constant function)
ooc_sc.set_lambda_functions(
    lambda_function=lambda omega: np.ones_like(omega),
    dlambda_function=lambda omega: np.zeros_like(omega)
)

# Perform static condensation
bulk_solution, flux_jump, jacobian = ooc_sc.static_condensation(
    trace_values=trace_vals,
    rhs=rhs,
    time=0.0
)

print(f"Bulk solution shape: {bulk_solution.shape}")  # (8, n_elements)
print(f"Flux jump shape: {flux_jump.shape}")  # (4,) 
print(f"Jacobian shape: {jacobian.shape}")  # (8, 4*(n_elements+1))
\end{lstlisting}


\subsection{StaticCondensationFactory Usage}
\label{subsec:factory_usage}

The factory class provides automatic selection of appropriate static condensation implementation.

\subsubsection{Factory Methods}

\paragraph{create()}\leavevmode
\begin{lstlisting}[language=Python, caption=Factory Create Method]
@classmethod
def create(cls, problem: Problem, global_disc, elementary_matrices,
           i: int = 0) -> StaticCondensationBase
\end{lstlisting}

\textbf{Parameters:}
\begin{itemize}
    \item \texttt{problem}: Problem instance with \texttt{type} attribute
    \item \texttt{global\_disc}: GlobalDiscretization instance
    \item \texttt{elementary\_matrices}: ElementaryMatrices instance
    \item \texttt{i}: Domain index (default: 0)
\end{itemize}

\textbf{Returns:} Appropriate StaticCondensation implementation

\textbf{Usage:}
\begin{lstlisting}[language=Python, caption=Factory Usage Examples]
from ooc1d.core.static_condensation_factory import StaticCondensationFactory

# Automatic selection for Keller-Segel
ks_problem = Problem(neq=2, ..., problem_type="keller_segel")
ks_sc = StaticCondensationFactory.create(
    problem=ks_problem,
    global_disc=global_discretization,
    elementary_matrices=elem_matrices,
    i=0
)
# Returns: KellerSegelStaticCondensation instance

# Automatic selection for OrganOnChip  
ooc_problem = Problem(neq=4, ..., problem_type="organ_on_chip")
ooc_sc = StaticCondensationFactory.create(
    problem=ooc_problem,
    global_disc=global_discretization,
    elementary_matrices=elem_matrices,
    i=0
)
# Returns: StaticCondensationOOC instance
\end{lstlisting}

\paragraph{register\_implementation()}\leavevmode
\begin{lstlisting}[language=Python, caption=Register Implementation Method]
@classmethod
def register_implementation(cls, problem_type: str, 
                           implementation_class: Type[StaticCondensationBase])
\end{lstlisting}

\textbf{Purpose:} Register new static condensation implementations

\textbf{Usage:}
\begin{lstlisting}[language=Python, caption=Register New Implementation]
# Register custom implementation
class CustomStaticCondensation(StaticCondensationBase):
    # ... implementation ...
    pass

StaticCondensationFactory.register_implementation(
    "custom_problem", CustomStaticCondensation
)

# Now available through factory
custom_problem = Problem(neq=3, ..., problem_type="custom_problem")
custom_sc = StaticCondensationFactory.create(custom_problem, ...)
\end{lstlisting}

\subsection{Complete Usage Examples}
\label{subsec:complete_usage_examples}

\subsubsection{Keller-Segel Complete Workflow}

\begin{lstlisting}[language=Python, caption=Complete KellerSegel Workflow]
import numpy as np
from ooc1d.core.problem import Problem
from ooc1d.core.discretization import Discretization, GlobalDiscretization
from ooc1d.utils.elementary_matrices import ElementaryMatrices
from ooc1d.core.static_condensation_factory import StaticCondensationFactory

# Step 1: Create Keller-Segel problem
problem = Problem(
    neq=2,
    domain_start=0.0,
    domain_length=1.0,
    parameters=np.array([2.0, 1.0, 0.1, 1.0]),  # [mu, nu, a, b]
    problem_type="keller_segel",
    name="chemotaxis_problem"
)

# Set chemotaxis functions
problem.set_chemotaxis(
    chi=lambda phi: np.ones_like(phi),
    dchi=lambda phi: np.zeros_like(phi)
)

# Step 2: Create discretization
discretization = Discretization(n_elements=20)
discretization.set_tau([1.0, 1.0])  # [tau_u, tau_phi]
global_disc = GlobalDiscretization([discretization])

# Step 3: Create elementary matrices
elementary_matrices = ElementaryMatrices()

# Step 4: Create static condensation via factory
static_condensation = StaticCondensationFactory.create(
    problem=problem,
    global_disc=global_disc,
    elementary_matrices=elementary_matrices,
    i=0
)

print(f"Created: {type(static_condensation).__name__}")  # KellerSegelStaticCondensation

# Step 5: Build matrices
matrices = static_condensation.build_matrices()
print(f"Available matrices: {list(matrices.keys())}")

# Step 6: Perform static condensation
trace_values = np.random.rand(2 * 21)  # 2 equations, 21 nodes
rhs = np.random.rand(2 * 2 * 20)  # 2 equations, 2 DOFs per element, 20 elements

bulk_solution, flux_jump, jacobian = static_condensation.static_condensation(
    trace_values=trace_values,
    rhs=rhs,
    time=0.0
)

print(f"Static condensation completed:")
print(f"  Bulk solution: {bulk_solution.shape}")
print(f"  Flux jump: {flux_jump.shape}")
print(f"  Jacobian: {jacobian.shape}")
\end{lstlisting}

\subsubsection{OrganOnChip Complete Workflow (MATLAB Compatible)}

\begin{lstlisting}[language=Python, caption=Complete OrganOnChip Workflow]
# Step 1: Create OrganOnChip problem (matching MATLAB TestProblem.m)
ooc_params = np.array([1.0, 2.0, 1.0, 1.0, 0.0, 1.0, 0.0, 1.0, 1.0])
# [nu, mu, epsilon, sigma, a, b, c, d, chi]

problem = Problem(
    neq=4,
    domain_start=0.0,  # A = 0 (MATLAB)
    domain_length=1.0, # L = 1 (MATLAB)
    parameters=ooc_params,
    problem_type="organ_on_chip",
    name="microfluidic_device"
)

# Set initial conditions (matching MATLAB TestProblem.m)
problem.set_initial_condition(0, lambda x, t: np.sin(2*np.pi*x))  # u
for eq_idx in [1, 2, 3]:  # omega, v, phi
    problem.set_initial_condition(eq_idx, lambda x, t: np.zeros_like(x))

# Set lambda function (matching MATLAB: constant_function)
problem.set_function('lambda_function', lambda x: np.ones_like(x))
problem.set_function('dlambda_function', lambda x: np.zeros_like(x))

# Step 2: Create discretization
discretization = Discretization(n_elements=40)  # Matching MATLAB
discretization.set_tau([1.0, 1.0, 1.0, 1.0])  # [tu, to, tv, tp]
global_disc = GlobalDiscretization([discretization])
global_disc.set_time_parameters(dt=0.01, T=0.5)  # Matching MATLAB

# Step 3: Create static condensation
elementary_matrices = ElementaryMatrices()
static_condensation = StaticCondensationFactory.create(
    problem=problem,
    global_disc=global_disc,
    elementary_matrices=elementary_matrices,
    i=0
)

print(f"Created: {type(static_condensation).__name__}")  # StaticCondensationOOC

# Set lambda functions for nonlinear coupling
static_condensation.set_lambda_functions(
    lambda_function=lambda omega: np.ones_like(omega),  # Constant (MATLAB)
    dlambda_function=lambda omega: np.zeros_like(omega)
)

# Step 4: Build matrices (following MATLAB scBlocks.m structure)
matrices = static_condensation.build_matrices()
required_matrices = ['M', 'T', 'B1', 'L1', 'B2', 'C2', 'L2', 'A3', 'S3', 'H3', 
                    'B4', 'C4', 'L4', 'D1', 'D2', 'Q', 'Av']
print(f"Required matrices available: {all(m in matrices for m in required_matrices)}")

# Step 5: Perform static condensation (following MATLAB StaticC.m)
trace_values = np.random.rand(4 * 41)  # 4 equations, 41 nodes
rhs = np.random.rand(2 * 4 * 40)  # 4 equations, 2 DOFs per element, 40 elements

bulk_solution, flux_jump, jacobian = static_condensation.static_condensation(
    trace_values=trace_values,
    rhs=rhs,
    time=0.0
)

# Verify MATLAB compatibility
print(f"OrganOnChip static condensation results:")
print(f"  Bulk solution (U): {bulk_solution.shape}")  # Should be (8, 40)
print(f"  Flux jump (hJ): {flux_jump.shape}")  # Should be (4,) or (8,)
print(f"  Jacobian (dhJ): {jacobian.shape}")  # Should be (8, 4*41)

# Check parameter extraction matches MATLAB
params = static_condensation.get_problem_parameters()
matlab_params = {'nu': params[0], 'mu': params[1], 'epsilon': params[2], 
                'sigma': params[3], 'a': params[4], 'b': params[5], 
                'c': params[6], 'd': params[7], 'chi': params[8]}
print(f"MATLAB-compatible parameters: {matlab_params}")
\end{lstlisting}

\subsection{Method Summary Table}
\label{subsec:static_condensation_method_summary}

\subsubsection{StaticCondensationBase Methods}

\begin{longtable}{|p{5.5cm}|p{2cm}|p{7cm}|}
\hline
\textbf{Method} & \textbf{Returns} & \textbf{Purpose} \\
\hline
\endhead

\texttt{\_\_init\_\_} & \texttt{None} & Initialize base class with common attributes \\
\hline

\texttt{build\_matrices} & \texttt{dict} & Abstract: build problem-specific matrices \\
\hline

\texttt{static\_condensation} & \texttt{Tuple} & Abstract: perform local static condensation \\
\hline

\texttt{get\_problem\_parameters} & \texttt{np.ndarray} & Extract problem parameter array \\
\hline

\texttt{get\_stabilization\_parameters} & \texttt{np.ndarray} & Extract stabilization parameters \\
\hline

\texttt{validate\_input} & \texttt{None} & Validate trace values and RHS dimensions \\
\hline

\end{longtable}

\subsubsection{KellerSegelStaticCondensation Methods}

\begin{longtable}{|p{4.7cm}|p{2cm}|p{7cm}|}
\hline
\textbf{Method} & \textbf{Returns} & \textbf{Purpose} \\
\hline
\endhead

\texttt{build\_matrices} & \texttt{dict} & Build 2-equation Keller-Segel matrices \\
\hline

\texttt{static\_condensation} & \texttt{Tuple} & Perform Keller-Segel static condensation \\
\hline

\texttt{set\_chemotaxis\_functions} & \texttt{None} & Set nonlinear chemotaxis functions \\
\hline

\end{longtable}

\subsubsection{StaticCondensationOOC Methods}

\begin{longtable}{|p{4cm}|p{2cm}|p{7cm}|}
\hline
\textbf{Method} & \textbf{Returns} & \textbf{Purpose} \\
\hline
\endhead

\texttt{build\_matrices} & \texttt{dict} & Build 4-equation OrganOnChip matrices (MATLAB compatible) \\
\hline

\texttt{static\_condensation} & \texttt{Tuple} & Perform OrganOnChip static condensation (MATLAB StaticC.m) \\
\hline

\texttt{set\_lambda\_functions} & \texttt{None} & Set nonlinear response functions $\lambda(\omega)$ \\
\hline

\end{longtable}

This documentation provides a comprehensive reference for the static condensation module hierarchy, with implementations that follow the MATLAB reference files while integrating with the BioNetFlux Python architecture.

% End of static condensation modules API documentation


\section{Static Condensation Modules API Reference}
\label{sec:static_condensation_modules_api}

The static condensation module is where the  problem specific HDG discretization happens. This section provides a comprehensive reference for the static condensation module hierarchy based on analysis of the MATLAB reference files and BioNetFlux architecture patterns. The modules implement HDG static condensation for different problem types.

\subsection{Module Overview}

The static condensation system consists of:
\begin{itemize}
    \item \textbf{StaticCondensationBase}: Abstract base class defining the interface
    \item \textbf{Derived classes}
   \begin{itemize}
    \item \textbf{KellerSegelStaticCondensation}: Implementation for 2-equation chemotaxis systems
    \item \textbf{StaticCondensationOOC}: Implementation for 4-equation OrganOnChip systems
   \end{itemize}
    \item \textbf{StaticCondensationFactory}: Factory for creating appropriate implementations
\end{itemize}

%\subsection{MATLAB Reference Analysis}
%
%Based on the MATLAB files, the static condensation process follows this structure:
%
%\textbf{From StaticC.m:} The main static condensation function \texttt{[U,hJ,dhJ] = StaticC(hU,rhs,problem,discretization,scMatrices)} performs:
%\begin{enumerate}
%    \item Local trace-to-bulk reconstruction: $\hat{U} \rightarrow U$
%    \item Flux jump computation: $\tilde{J} = D_1 U - D_2 \hat{U}$
%    \item Normal flux computation: $j = \hat{B}_4 \hat{U} + \tilde{J}^T Q U$
%    \item Final assembly: $\hat{J} = [\hat{j}; \hat{J}]$
%\end{enumerate}
%
%\textbf{From scBlocks.m:} The matrix construction involves building operators for each step of the 4-equation OrganOnChip system.

\subsection{StaticCondensationBase Class}
\label{subsec:static_condensation_base}

Abstract base class defining the interface for all static condensation implementations.

\subsubsection{Constructor}

\paragraph{\_\_init\_\_()}\leavevmode
\begin{lstlisting}[language=Python, caption=StaticCondensationBase Constructor]
def __init__(self, 
             problem: Problem, 
             global_discretization, 
             elementary_matrices, 
             domain_index: int = 0)
\end{lstlisting}

\textbf{Parameters:}
\begin{itemize}
    \item \texttt{problem}: Problem instance with equations and parameters
    \item \texttt{global\_discretization}: GlobalDiscretization instance
    \item \texttt{elementary\_matrices}: ElementaryMatrices instance
    \item \texttt{domain\_index}: Domain index for multi-domain problems (default: 0)
\end{itemize}

\textbf{Side Effects:} Initializes base attributes and extracts domain-specific discretization

\subsubsection{Core Attributes}

\begin{longtable}{|p{3.8cm}|p{3.5cm}|p{7cm}|}
\hline
\textbf{Attribute} & \textbf{Type} & \textbf{Description} \\
\hline
\endhead

\texttt{problem} & \texttt{Problem} & Problem definition with parameters and equations \\
\hline

\texttt{discretization} & \texttt{Discretization} & Domain-specific discretization \\
\hline

\texttt{elementary\_matrices} & \texttt{ElementaryMatrices} & Pre-computed elementary matrices \\
\hline

\texttt{domain\_index} & \texttt{int} & Domain index for multi-domain systems \\
\hline

\texttt{neq} & \texttt{int} & Number of equations (extracted from problem) \\
\hline

\texttt{n\_elements} & \texttt{int} & Number of elements (extracted from discretization) \\
\hline

\texttt{element\_length} & \texttt{float} & Element size: \texttt{h = domain\_length / n\_elements} \\
\hline

\texttt{dt} & \texttt{Optional[float]} & Time step size (from discretization if available) \\
\hline

\end{longtable}

\subsubsection{Abstract Methods (Must be Implemented by Subclasses)}

\paragraph{build\_matrices()}
\begin{lstlisting}[language=Python, caption=Abstract Build Matrices Method]
@abstractmethod
def build_matrices(self) -> dict
\end{lstlisting}

\textbf{Returns:} \texttt{dict} - Dictionary of static condensation matrices

\textbf{Purpose:} Build all matrices required for static condensation (problem-specific)

\paragraph{static\_condensation()}\leavevmode
\begin{lstlisting}[language=Python, caption=Abstract Static Condensation Method]
@abstractmethod
def static_condensation(self, 
                       trace_values: np.ndarray, 
                       rhs: np.ndarray, 
                       time: float = 0.0) -> Tuple[np.ndarray, np.ndarray, np.ndarray]
\end{lstlisting}

\textbf{Parameters:}
\begin{itemize}
    \item \texttt{trace\_values}: Local trace values $\hat{U}$ (size: \texttt{neq*(n\_elements+1)})
    \item \texttt{rhs}: Right-hand side vector (size: \texttt{2*neq*n\_elements})
    \item \texttt{time}: Current time for time-dependent problems (default: 0.0)
\end{itemize}

\textbf{Returns:} \texttt{Tuple[np.ndarray, np.ndarray, np.ndarray]} - (bulk\_solution, flux\_jump, jacobian)

\textbf{Purpose:} Perform local static condensation step

\subsubsection{Concrete Methods (Provided by Base Class)}

\paragraph{get\_problem\_parameters()}\leavevmode
\begin{lstlisting}[language=Python, caption=Get Problem Parameters Method]
def get_problem_parameters(self) -> np.ndarray
\end{lstlisting}

\textbf{Returns:} \texttt{np.ndarray} - Problem parameters array

\textbf{Usage:}
\begin{lstlisting}[language=Python, caption=Get Parameters Usage]
params = static_condensation.get_problem_parameters()
# For OrganOnChip: [nu, mu, epsilon, sigma, a, b, c, d, chi]
# For Keller-Segel: [mu, nu, a, b]
\end{lstlisting}

\paragraph{get\_stabilization\_parameters()}\leavevmode
\begin{lstlisting}[language=Python, caption=Get Stabilization Parameters Method]
def get_stabilization_parameters(self) -> np.ndarray
\end{lstlisting}

\textbf{Returns:} \texttt{np.ndarray} - Stabilization parameters $\tau$ for each equation

\textbf{Usage:}
\begin{lstlisting}[language=Python, caption=Get Stabilization Usage]
tau_values = static_condensation.get_stabilization_parameters()
# For OrganOnChip: [tu, to, tv, tp] (4 equations)
# For Keller-Segel: [tu, tp] (2 equations)
\end{lstlisting}

\paragraph{validate\_input()}\leavevmode
\begin{lstlisting}[language=Python, caption=Validate Input Method]
def validate_input(self, 
                  trace_values: np.ndarray, 
                  rhs: np.ndarray) -> None
\end{lstlisting}

\textbf{Parameters:}
\begin{itemize}
    \item \texttt{trace\_values}: Trace values to validate
    \item \texttt{rhs}: Right-hand side to validate
\end{itemize}

\textbf{Returns:} \texttt{None}

\textbf{Raises:} \texttt{ValueError} for invalid input dimensions or content

\textbf{Validation Checks:}
\begin{itemize}
    \item Trace values size: \texttt{neq * (n\_elements + 1)}
    \item RHS size: \texttt{2 * neq * n\_elements}
    \item No NaN or infinite values
\end{itemize}

\subsection{KellerSegelStaticCondensation Class}
\label{subsec:keller_segel_static_condensation}

Implementation for 2-equation Keller-Segel chemotaxis systems (u, $\phi$).

\subsubsection{Constructor}

\paragraph{\_\_init\_\_()}\leavevmode
\begin{lstlisting}[language=Python, caption=KellerSegel Constructor]
def __init__(self, 
             problem: Problem, 
             global_discretization, 
             elementary_matrices, 
             domain_index: int = 0)
\end{lstlisting}

\textbf{Requirements:} \texttt{problem.neq == 2} and \texttt{problem.type == "keller\_segel"}

\textbf{Usage:}
\begin{lstlisting}[language=Python, caption=KellerSegel Constructor Usage]
# Create Keller-Segel problem
problem = Problem(
    neq=2,
    domain_start=0.0,
    domain_length=1.0,
    parameters=np.array([2.0, 1.0, 0.1, 1.0]),  # [mu, nu, a, b]
    problem_type="keller_segel"
)

# Create static condensation
ks_sc = KellerSegelStaticCondensation(
    problem=problem,
    global_discretization=global_disc,
    elementary_matrices=elementary_matrices,
    domain_index=0
)
\end{lstlisting}

\subsubsection{Specific Attributes}

\begin{longtable}{|p{3.2cm}|p{3.5cm}|p{7cm}|}
\hline
\textbf{Attribute} & \textbf{Type} & \textbf{Description} \\
\hline
\endhead

\texttt{mu} & \texttt{float} & Diffusion coefficient for $\phi$ equation (parameter 0) \\
\hline

\texttt{nu} & \texttt{float} & Diffusion coefficient for $u$ equation (parameter 1) \\
\hline

\texttt{a} & \texttt{float} & Reaction parameter (parameter 2) \\
\hline

\texttt{b} & \texttt{float} & Coupling parameter (parameter 3) \\
\hline

\texttt{chi\_function} & \texttt{Optional[Callable]} & Chemotactic sensitivity $\chi(\phi)$ \\
\hline

\texttt{dchi\_function} & \texttt{Optional[Callable]} & Derivative $\chi'(\phi)$ \\
\hline

\end{longtable}

\subsubsection{Matrix Construction}

\paragraph{build\_matrices()}\leavevmode
\begin{lstlisting}[language=Python, caption=KellerSegel Build Matrices]
def build_matrices(self) -> dict
\end{lstlisting}

\textbf{Returns:} \texttt{dict} - Dictionary with matrices for 2-equation system

\textbf{Matrix Dictionary Keys:}
\begin{itemize}
    \item \texttt{'M'}: Mass matrix
    \item \texttt{'T'}: Trace matrix  
    \item \texttt{'D'}: Differentiation matrix
    \item \texttt{'B1'}, \texttt{'L1'}: Matrices for $u$ equation reconstruction
    \item \texttt{'B2'}, \texttt{'L2'}: Matrices for $\phi$ equation reconstruction
    \item \texttt{'Q'}: Coupling matrix for chemotaxis
\end{itemize}

\textbf{Mathematical Formulation:}
For Keller-Segel system:
\begin{align}
\frac{\partial u}{\partial t} - \nu \nabla^2 u + \nabla \cdot (u \chi(\phi) \nabla \phi) &= f_u \\
\frac{\partial \phi}{\partial t} - \mu \nabla^2 \phi &= f_\phi + a u
\end{align}

\paragraph{static\_condensation()}\leavevmode
\begin{lstlisting}[language=Python, caption=KellerSegel Static Condensation]
def static_condensation(self, 
                       trace_values: np.ndarray, 
                       rhs: np.ndarray, 
                       time: float = 0.0) -> Tuple[np.ndarray, np.ndarray, np.ndarray]
\end{lstlisting}

\textbf{Algorithm:} 
\begin{enumerate}
    \item Extract trace values for $u$ and $\phi$
    \item Reconstruct bulk solutions: $u = B_1 \hat{u} + L_1 g_u$, $\phi = B_2 \hat{\phi} + L_2 g_\phi$
    \item Compute chemotaxis terms if $\chi$ functions available
    \item Compute flux jumps and Jacobian contributions
    \item Return bulk solutions, flux jumps, and Jacobian
\end{enumerate}

\textbf{Usage:}
\begin{lstlisting}[language=Python, caption=KellerSegel Static Condensation Usage]
# Prepare input
trace_vals = np.random.rand(2 * (n_elements + 1))  # u and phi traces
rhs = np.random.rand(2 * 2 * n_elements)  # RHS for both equations

# Perform static condensation
bulk_solution, flux_jump, jacobian = ks_sc.static_condensation(
    trace_values=trace_vals,
    rhs=rhs,
    time=0.5
)

print(f"Bulk solution shape: {bulk_solution.shape}")  # (4, n_elements)
print(f"Flux jump shape: {flux_jump.shape}")  # (2,)
print(f"Jacobian shape: {jacobian.shape}")  # (4, 2*(n_elements+1))
\end{lstlisting}


\subsection{StaticCondensationOOC Class}
\label{subsec:static_condensation_ooc}

Implementation for 4-equation OrganOnChip systems (u, $\omega$, v, $\phi$) based on MATLAB StaticC.m.

\subsubsection{Constructor}

\paragraph{\_\_init\_\_()}\leavevmode
\begin{lstlisting}[language=Python, caption=StaticCondensationOOC Constructor]
def __init__(self, 
             problem: Problem, 
             global_discretization, 
             elementary_matrices, 
             domain_index: int = 0)
\end{lstlisting}

\textbf{Requirements:} \texttt{problem.neq == 4} and \texttt{problem.type == "organ\_on\_chip"}

\textbf{Usage:}
\begin{lstlisting}[language=Python, caption=OrganOnChip Constructor Usage]
# Create OrganOnChip problem (from MATLAB TestProblem.m)
ooc_params = np.array([1.0, 2.0, 1.0, 1.0, 0.0, 1.0, 0.0, 1.0, 1.0])
problem = Problem(
    neq=4,
    domain_start=0.0,  # A = 0
    domain_length=1.0, # L = 1
    parameters=ooc_params,  # [nu, mu, epsilon, sigma, a, b, c, d, chi]
    problem_type="organ_on_chip"
)

# Create static condensation
ooc_sc = StaticCondensationOOC(
    problem=problem,
    global_discretization=global_disc,
    elementary_matrices=elementary_matrices,
    domain_index=0
)
\end{lstlisting}

%\subsubsection{Specific Attributes (From MATLAB TestProblem.m)}
%
%\begin{longtable}{|p{3.5cm}|p{3.5cm}|p{7cm}|}
%\hline
%\textbf{Attribute} & \textbf{Type} & \textbf{Description} \\
%\hline
%\endhead
%
%\texttt{nu} & \texttt{float} & Viscosity parameter (parameter 0) \\
%\hline
%
%\texttt{mu} & \texttt{float} & Viscosity parameter (parameter 1) \\
%\hline
%
%\texttt{epsilon} & \texttt{float} & Coupling parameter (parameter 2) \\
%\hline
%
%\texttt{sigma} & \texttt{float} & Coupling parameter (parameter 3) \\
%\hline
%
%\texttt{a} & \texttt{float} & Reaction parameter (parameter 4) \\
%\hline
%
%\texttt{b} & \texttt{float} & Inter-equation coupling (parameter 5) \\
%\hline
%
%\texttt{c} & \texttt{float} & Reaction parameter (parameter 6) \\
%\hline
%
%\texttt{d} & \texttt{float} & Inter-equation coupling (parameter 7) \\
%\hline
%
%\texttt{chi} & \texttt{float} & Coupling strength (parameter 8) \\
%\hline
%
%\texttt{lambda\_function} & \texttt{Optional[Callable]} & Nonlinear response $\lambda(\omega)$ \\
%\hline
%
%\texttt{dlambda\_function} & \texttt{Optional[Callable]} & Derivative $\lambda'(\omega)$ \\
%\hline
%
%\end{longtable}
%
%\subsubsection{Mathematical System (From MATLAB References)}
%
%The OrganOnChip system implements:
%\begin{align}
%\frac{\partial u}{\partial t} - \nu \nabla^2 u &= f_u \label{eq:ooc_u} \\
%\frac{\partial \omega}{\partial t} + \epsilon \nabla \cdot \theta + c\omega &= f_\omega + d u \label{eq:ooc_omega} \\
%\frac{\partial v}{\partial t} + \sigma \nabla \cdot q + \lambda(\bar{\omega}) v &= f_v \label{eq:ooc_v} \\
%\frac{\partial \phi}{\partial t} + \mu \nabla \cdot \psi + a\phi &= f_\phi + b v \label{eq:ooc_phi}
%\end{align}
%
%With auxiliary relations:
%\begin{align}
%\theta &= \epsilon(\nabla \omega - \hat{\omega}) \\
%q &= \sigma(\nabla v - \hat{v}) \\
%\psi &= \mu(\nabla \phi - \hat{\phi})
%\end{align}

\subsubsection{Matrix Construction (Based on MATLAB scBlocks.m)}

\paragraph{build\_matrices()}\leavevmode
\begin{lstlisting}[language=Python, caption=OrganOnChip Build Matrices]
def build_matrices(self) -> dict
\end{lstlisting}

\textbf{Returns:} Dictionary with OrganOnChip-specific matrices

\textbf{Matrix Dictionary Keys (From MATLAB scBlocks.m):}
\begin{itemize}
    \item \texttt{'M'}: Mass matrix
    \item \texttt{'T'}: Trace matrix
    \item \texttt{'B1'}, \texttt{'L1'}: Matrices for $u$ equation
    \item \texttt{'B2'}, \texttt{'C2'}, \texttt{'L2'}: Matrices for $\omega$ equation  
    \item \texttt{'A3'}, \texttt{'S3'}, \texttt{'H3'}: Matrices for $v$ equation
    \item \texttt{'B4'}, \texttt{'C4'}, \texttt{'L4'}: Matrices for $\phi$ equation
    \item \texttt{'D1'}, \texttt{'D2'}: Flux jump matrices
    \item \texttt{'Q'}: Coupling matrix
    \item \texttt{'hB4'}, \texttt{'B5'}, \texttt{'B6'}, \texttt{'B7'}: Assembly matrices
    \item \texttt{'hatB0'}, \texttt{'hatB1'}, \texttt{'hatB2'}: Final assembly matrices
    \item \texttt{'Av'}: Averaging matrix for $\bar{\omega}$ computation
\end{itemize}

\paragraph{static\_condensation()}\leavevmode
\begin{lstlisting}[language=Python, caption=OrganOnChip Static Condensation]
def static_condensation(self, 
                       trace_values: np.ndarray, 
                       rhs: np.ndarray, 
                       time: float = 0.0) -> Tuple[np.ndarray, np.ndarray, np.ndarray]
\end{lstlisting}

\textbf{Algorithm (Following MATLAB StaticC.m):}
\begin{enumerate}
    \item \textbf{Step 1}: $\hat{U} \rightarrow U$ reconstruction
    \begin{itemize}
        \item $u = B_1 \hat{u} + y_1$ where $y_1 = L_1 g_u$
        \item $\omega = C_2 \hat{u} + B_2 \hat{\omega} + y_2$ where $y_2 = L_2(g_\omega + dt \cdot d \cdot M y_1)$
    \end{itemize}
    
    \item \textbf{Step 2}: Nonlinear coupling computation
    \begin{itemize}
        \item $\bar{\omega} = A_{av} \omega$ (averaging)
        \item $\bar{\lambda}_\omega = \lambda(\bar{\omega})$ (nonlinear function evaluation)
        \item $J_\lambda = \lambda'(\bar{\omega}) A_{av}$ (Jacobian contribution)
    \end{itemize}
    
    \item \textbf{Step 3}: $v$ equation solution
    \begin{itemize}
        \item $L_3(\omega) = (A_3 + \bar{\lambda}_\omega S_3)^{-1}$ (nonlinear operator)
        \item $v = B_3(\omega) \hat{v} + y_3(\omega)$ where $B_3(\omega) = L_3(\omega) H_3$
    \end{itemize}
    
    \item \textbf{Step 4}: $\phi$ equation solution
    \begin{itemize}
        \item $\phi = B_4 \hat{\phi} + C_4 v + L_4 g_\phi$
    \end{itemize}
    
    \item \textbf{Step 5}: Flux jump computation
    \begin{itemize}
        \item $\tilde{J} = D_1 U - D_2 \hat{U}$
        \item $j = \hat{B}_4 \hat{U} + \tilde{J}^T Q U$
    \end{itemize}
    
    \item \textbf{Step 6}: Final assembly
    \begin{itemize}
        \item $\hat{j} = B_5 j + B_6 U + B_7 \hat{U}$
        \item $\hat{J} = \hat{B}_0 \tilde{J} + \hat{B}_1 U - \hat{B}_2 \hat{U}$
    \end{itemize}
\end{enumerate}

\textbf{Usage:}
\begin{lstlisting}[language=Python, caption=OrganOnChip Static Condensation Usage]
# Prepare input (4 equations, matching MATLAB TestProblem.m)
trace_vals = np.random.rand(4 * (n_elements + 1))  # u, omega, v, phi traces
rhs = np.random.rand(2 * 4 * n_elements)  # RHS for all 4 equations

# Set lambda function (from MATLAB TestProblem.m: constant function)
ooc_sc.set_lambda_functions(
    lambda_function=lambda omega: np.ones_like(omega),
    dlambda_function=lambda omega: np.zeros_like(omega)
)

# Perform static condensation
bulk_solution, flux_jump, jacobian = ooc_sc.static_condensation(
    trace_values=trace_vals,
    rhs=rhs,
    time=0.0
)

print(f"Bulk solution shape: {bulk_solution.shape}")  # (8, n_elements)
print(f"Flux jump shape: {flux_jump.shape}")  # (4,) 
print(f"Jacobian shape: {jacobian.shape}")  # (8, 4*(n_elements+1))
\end{lstlisting}


\subsection{StaticCondensationFactory Usage}
\label{subsec:factory_usage}

The factory class provides automatic selection of appropriate static condensation implementation.

\subsubsection{Factory Methods}

\paragraph{create()}\leavevmode
\begin{lstlisting}[language=Python, caption=Factory Create Method]
@classmethod
def create(cls, problem: Problem, global_disc, elementary_matrices,
           i: int = 0) -> StaticCondensationBase
\end{lstlisting}

\textbf{Parameters:}
\begin{itemize}
    \item \texttt{problem}: Problem instance with \texttt{type} attribute
    \item \texttt{global\_disc}: GlobalDiscretization instance
    \item \texttt{elementary\_matrices}: ElementaryMatrices instance
    \item \texttt{i}: Domain index (default: 0)
\end{itemize}

\textbf{Returns:} Appropriate StaticCondensation implementation

\textbf{Usage:}
\begin{lstlisting}[language=Python, caption=Factory Usage Examples]
from ooc1d.core.static_condensation_factory import StaticCondensationFactory

# Automatic selection for Keller-Segel
ks_problem = Problem(neq=2, ..., problem_type="keller_segel")
ks_sc = StaticCondensationFactory.create(
    problem=ks_problem,
    global_disc=global_discretization,
    elementary_matrices=elem_matrices,
    i=0
)
# Returns: KellerSegelStaticCondensation instance

# Automatic selection for OrganOnChip  
ooc_problem = Problem(neq=4, ..., problem_type="organ_on_chip")
ooc_sc = StaticCondensationFactory.create(
    problem=ooc_problem,
    global_disc=global_discretization,
    elementary_matrices=elem_matrices,
    i=0
)
# Returns: StaticCondensationOOC instance
\end{lstlisting}

\paragraph{register\_implementation()}\leavevmode
\begin{lstlisting}[language=Python, caption=Register Implementation Method]
@classmethod
def register_implementation(cls, problem_type: str, 
                           implementation_class: Type[StaticCondensationBase])
\end{lstlisting}

\textbf{Purpose:} Register new static condensation implementations

\textbf{Usage:}
\begin{lstlisting}[language=Python, caption=Register New Implementation]
# Register custom implementation
class CustomStaticCondensation(StaticCondensationBase):
    # ... implementation ...
    pass

StaticCondensationFactory.register_implementation(
    "custom_problem", CustomStaticCondensation
)

# Now available through factory
custom_problem = Problem(neq=3, ..., problem_type="custom_problem")
custom_sc = StaticCondensationFactory.create(custom_problem, ...)
\end{lstlisting}

\subsection{Complete Usage Examples}
\label{subsec:complete_usage_examples}

\subsubsection{Keller-Segel Complete Workflow}

\begin{lstlisting}[language=Python, caption=Complete KellerSegel Workflow]
import numpy as np
from ooc1d.core.problem import Problem
from ooc1d.core.discretization import Discretization, GlobalDiscretization
from ooc1d.utils.elementary_matrices import ElementaryMatrices
from ooc1d.core.static_condensation_factory import StaticCondensationFactory

# Step 1: Create Keller-Segel problem
problem = Problem(
    neq=2,
    domain_start=0.0,
    domain_length=1.0,
    parameters=np.array([2.0, 1.0, 0.1, 1.0]),  # [mu, nu, a, b]
    problem_type="keller_segel",
    name="chemotaxis_problem"
)

# Set chemotaxis functions
problem.set_chemotaxis(
    chi=lambda phi: np.ones_like(phi),
    dchi=lambda phi: np.zeros_like(phi)
)

# Step 2: Create discretization
discretization = Discretization(n_elements=20)
discretization.set_tau([1.0, 1.0])  # [tau_u, tau_phi]
global_disc = GlobalDiscretization([discretization])

# Step 3: Create elementary matrices
elementary_matrices = ElementaryMatrices()

# Step 4: Create static condensation via factory
static_condensation = StaticCondensationFactory.create(
    problem=problem,
    global_disc=global_disc,
    elementary_matrices=elementary_matrices,
    i=0
)

print(f"Created: {type(static_condensation).__name__}")  # KellerSegelStaticCondensation

# Step 5: Build matrices
matrices = static_condensation.build_matrices()
print(f"Available matrices: {list(matrices.keys())}")

# Step 6: Perform static condensation
trace_values = np.random.rand(2 * 21)  # 2 equations, 21 nodes
rhs = np.random.rand(2 * 2 * 20)  # 2 equations, 2 DOFs per element, 20 elements

bulk_solution, flux_jump, jacobian = static_condensation.static_condensation(
    trace_values=trace_values,
    rhs=rhs,
    time=0.0
)

print(f"Static condensation completed:")
print(f"  Bulk solution: {bulk_solution.shape}")
print(f"  Flux jump: {flux_jump.shape}")
print(f"  Jacobian: {jacobian.shape}")
\end{lstlisting}

\subsubsection{OrganOnChip Complete Workflow (MATLAB Compatible)}

\begin{lstlisting}[language=Python, caption=Complete OrganOnChip Workflow]
# Step 1: Create OrganOnChip problem (matching MATLAB TestProblem.m)
ooc_params = np.array([1.0, 2.0, 1.0, 1.0, 0.0, 1.0, 0.0, 1.0, 1.0])
# [nu, mu, epsilon, sigma, a, b, c, d, chi]

problem = Problem(
    neq=4,
    domain_start=0.0,  # A = 0 (MATLAB)
    domain_length=1.0, # L = 1 (MATLAB)
    parameters=ooc_params,
    problem_type="organ_on_chip",
    name="microfluidic_device"
)

# Set initial conditions (matching MATLAB TestProblem.m)
problem.set_initial_condition(0, lambda x, t: np.sin(2*np.pi*x))  # u
for eq_idx in [1, 2, 3]:  # omega, v, phi
    problem.set_initial_condition(eq_idx, lambda x, t: np.zeros_like(x))

# Set lambda function (matching MATLAB: constant_function)
problem.set_function('lambda_function', lambda x: np.ones_like(x))
problem.set_function('dlambda_function', lambda x: np.zeros_like(x))

# Step 2: Create discretization
discretization = Discretization(n_elements=40)  # Matching MATLAB
discretization.set_tau([1.0, 1.0, 1.0, 1.0])  # [tu, to, tv, tp]
global_disc = GlobalDiscretization([discretization])
global_disc.set_time_parameters(dt=0.01, T=0.5)  # Matching MATLAB

# Step 3: Create static condensation
elementary_matrices = ElementaryMatrices()
static_condensation = StaticCondensationFactory.create(
    problem=problem,
    global_disc=global_disc,
    elementary_matrices=elementary_matrices,
    i=0
)

print(f"Created: {type(static_condensation).__name__}")  # StaticCondensationOOC

# Set lambda functions for nonlinear coupling
static_condensation.set_lambda_functions(
    lambda_function=lambda omega: np.ones_like(omega),  # Constant (MATLAB)
    dlambda_function=lambda omega: np.zeros_like(omega)
)

# Step 4: Build matrices (following MATLAB scBlocks.m structure)
matrices = static_condensation.build_matrices()
required_matrices = ['M', 'T', 'B1', 'L1', 'B2', 'C2', 'L2', 'A3', 'S3', 'H3', 
                    'B4', 'C4', 'L4', 'D1', 'D2', 'Q', 'Av']
print(f"Required matrices available: {all(m in matrices for m in required_matrices)}")

# Step 5: Perform static condensation (following MATLAB StaticC.m)
trace_values = np.random.rand(4 * 41)  # 4 equations, 41 nodes
rhs = np.random.rand(2 * 4 * 40)  # 4 equations, 2 DOFs per element, 40 elements

bulk_solution, flux_jump, jacobian = static_condensation.static_condensation(
    trace_values=trace_values,
    rhs=rhs,
    time=0.0
)

# Verify MATLAB compatibility
print(f"OrganOnChip static condensation results:")
print(f"  Bulk solution (U): {bulk_solution.shape}")  # Should be (8, 40)
print(f"  Flux jump (hJ): {flux_jump.shape}")  # Should be (4,) or (8,)
print(f"  Jacobian (dhJ): {jacobian.shape}")  # Should be (8, 4*41)

# Check parameter extraction matches MATLAB
params = static_condensation.get_problem_parameters()
matlab_params = {'nu': params[0], 'mu': params[1], 'epsilon': params[2], 
                'sigma': params[3], 'a': params[4], 'b': params[5], 
                'c': params[6], 'd': params[7], 'chi': params[8]}
print(f"MATLAB-compatible parameters: {matlab_params}")
\end{lstlisting}

\subsection{Method Summary Table}
\label{subsec:static_condensation_method_summary}

\subsubsection{StaticCondensationBase Methods}

\begin{longtable}{|p{5.5cm}|p{2cm}|p{7cm}|}
\hline
\textbf{Method} & \textbf{Returns} & \textbf{Purpose} \\
\hline
\endhead

\texttt{\_\_init\_\_} & \texttt{None} & Initialize base class with common attributes \\
\hline

\texttt{build\_matrices} & \texttt{dict} & Abstract: build problem-specific matrices \\
\hline

\texttt{static\_condensation} & \texttt{Tuple} & Abstract: perform local static condensation \\
\hline

\texttt{get\_problem\_parameters} & \texttt{np.ndarray} & Extract problem parameter array \\
\hline

\texttt{get\_stabilization\_parameters} & \texttt{np.ndarray} & Extract stabilization parameters \\
\hline

\texttt{validate\_input} & \texttt{None} & Validate trace values and RHS dimensions \\
\hline

\end{longtable}

\subsubsection{KellerSegelStaticCondensation Methods}

\begin{longtable}{|p{4.7cm}|p{2cm}|p{7cm}|}
\hline
\textbf{Method} & \textbf{Returns} & \textbf{Purpose} \\
\hline
\endhead

\texttt{build\_matrices} & \texttt{dict} & Build 2-equation Keller-Segel matrices \\
\hline

\texttt{static\_condensation} & \texttt{Tuple} & Perform Keller-Segel static condensation \\
\hline

\texttt{set\_chemotaxis\_functions} & \texttt{None} & Set nonlinear chemotaxis functions \\
\hline

\end{longtable}

\subsubsection{StaticCondensationOOC Methods}

\begin{longtable}{|p{4cm}|p{2cm}|p{7cm}|}
\hline
\textbf{Method} & \textbf{Returns} & \textbf{Purpose} \\
\hline
\endhead

\texttt{build\_matrices} & \texttt{dict} & Build 4-equation OrganOnChip matrices (MATLAB compatible) \\
\hline

\texttt{static\_condensation} & \texttt{Tuple} & Perform OrganOnChip static condensation (MATLAB StaticC.m) \\
\hline

\texttt{set\_lambda\_functions} & \texttt{None} & Set nonlinear response functions $\lambda(\omega)$ \\
\hline

\end{longtable}

This documentation provides a comprehensive reference for the static condensation module hierarchy, with implementations that follow the MATLAB reference files while integrating with the BioNetFlux Python architecture.

% End of static condensation modules API documentation
