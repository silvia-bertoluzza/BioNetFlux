% Elementary Matrices Module API Documentation (Accurate Analysis)
% To be included in master LaTeX document
%
% Usage: % Elementary Matrices Module API Documentation (Accurate Analysis)
% To be included in master LaTeX document
%
% Usage: % Elementary Matrices Module API Documentation (Accurate Analysis)
% To be included in master LaTeX document
%
% Usage: % Elementary Matrices Module API Documentation (Accurate Analysis)
% To be included in master LaTeX document
%
% Usage: \input{docs/elementary_matrices_module_api}

\section{Elementary Matrices Module API Reference}
\label{sec:elementary_matrices_module_api}

This section provides reference  based on detailed analysis of the actual implementation for the  module \texttt{ooc1d.utils.elementary\_matrices.ElementaryMatrices}. The module computes elementary matrices for HDG methods on the reference element, serving as the Python equivalent of MATLAB \texttt{build\_eMatrices.m}.

\subsection{Module Overview}

The elementary matrices module provides:
\begin{itemize}
    \item Elementary matrix computation for HDG methods on reference element [0,1]
    \item Support for both Lagrange and orthonormal basis functions
    \item Symbolic computation using SymPy for exact mathematical operations
    \item Quadrature integration using Legendre-Gauss-Lobatto nodes
    \item Comprehensive matrix validation and testing
    \item MATLAB compatibility for existing HDG implementations
\end{itemize}

\subsection{Module Imports and Dependencies}

\begin{lstlisting}[language=Python, caption=Module Dependencies]
import numpy as np
import sympy as sp
from typing import Dict, Any
\end{lstlisting}

\subsection{ElementaryMatrices Class}
\label{subsec:elementary_matrices_class}

Main class for computing and managing elementary matrices on the reference element.

\subsubsection{Constructor}

\paragraph{\_\_init\_\_()}\leavevmode
\begin{lstlisting}[language=Python, caption=ElementaryMatrices Constructor]
def __init__(self, orthonormal_basis: bool = False)
\end{lstlisting}

\textbf{Parameters:}
\begin{itemize}
    \item \texttt{orthonormal\_basis}: Use orthonormal basis if True, Lagrange basis if False (default: False)
\end{itemize}

\textbf{Side Effects:}
\begin{itemize}
    \item Initializes all elementary matrices
    \item Stores matrices in internal dictionary
    \item Performs matrix validation tests
    \item Sets up quadrature rules
\end{itemize}

\textbf{Usage:}
\begin{lstlisting}[language=Python, caption=Constructor Usage Examples]
# Standard Lagrange basis (default)
elem_matrices = ElementaryMatrices()

# Orthonormal basis
elem_matrices_ortho = ElementaryMatrices(orthonormal_basis=True)

# Access matrices immediately after construction
mass_matrix = elem_matrices.get_matrix('M')
trace_matrix = elem_matrices.get_matrix('T')
\end{lstlisting}

\subsubsection{Core Attributes}

\begin{longtable}{|p{5.5cm}|p{2.5cm}|p{6cm}|}
\hline
\textbf{Attribute} & \textbf{Type} & \textbf{Description} \\
\hline
\endhead

\texttt{orthonormal\_basis} & \texttt{bool} & Flag indicating basis type used \\
\hline

\texttt{matrices} & \texttt{Dict[str, np.ndarray]} & Dictionary of all computed elementary matrices \\
\hline

\texttt{base} & \texttt{sp.Matrix} & SymPy symbolic basis functions \\
\hline

\texttt{hbase} & \texttt{sp.Matrix} & SymPy symbolic hat basis functions \\
\hline

\texttt{normali} & \texttt{sp.Matrix} & SymPy normal vectors at boundaries \\
\hline

\texttt{test\_integration\_by\_parts} & \texttt{np.ndarray} & Test matrix: $D + D^T - \tilde{N}$ (should be zero) \\
\hline

\texttt{test\_dxx\_zero} & \texttt{np.ndarray} & Test matrix for $\frac{d^2}{dx^2} = 0$ for linear functions \\
\hline

\texttt{test\_mb\_gb} & \texttt{np.ndarray} & Test matrix: $M^{\partial} - G^b T$ (should be zero) \\
\hline

\end{longtable}

\subsubsection{Matrix Construction Methods}

\paragraph{\_build\_matrices()}\leavevmode
\begin{lstlisting}[language=Python, caption=Build Matrices Method]
def _build_matrices(self)
\end{lstlisting}

\textbf{Purpose:} Build all elementary matrices on reference element [0,1]

\textbf{Process:}
\begin{enumerate}
    \item Define symbolic variable and reference interval
    \item Construct basis functions (Lagrange or orthonormal)
    \item Build basic matrices using symbolic integration
    \item Perform validation tests
    \item Build quadrature matrices
\end{enumerate}

\textbf{Basis Functions:}

\textbf{Lagrange Basis (default):}
\begin{align}
e_0(y) &= 1 - y \\
e_1(y) &= y
\end{align}

\textbf{Orthonormal Basis:} Constructed by solving orthogonality conditions:
\begin{align}
e_0 &= 1 \\
e_1 &= c \cdot y + d
\end{align}
Subject to: $\int_0^1 e_0 e_1 \, dy = 0$ and $\int_0^1 e_1^2 \, dy = 1$

\paragraph{\_build\_basic\_matrices()}\leavevmode
\begin{lstlisting}[language=Python, caption=Build Basic Matrices Method]
def _build_basic_matrices(self, y, base, hbase, normali, x0, x1)
\end{lstlisting}

\textbf{Parameters:}
\begin{itemize}
    \item \texttt{y}: SymPy symbolic variable
    \item \texttt{base}: Basis functions matrix
    \item \texttt{hbase}: Hat basis functions matrix
    \item \texttt{normali}: Normal vectors matrix
    \item \texttt{x0, x1}: Reference interval endpoints (0, 1)
\end{itemize}

\textbf{Matrices Computed:}

\textbf{Mass Matrix:} $M_{ij} = \int_0^1 e_j e_i \, dy$
\begin{lstlisting}[language=Python, caption=Mass Matrix Computation]
M = sp.integrate(base * base.T, (y, 0, 1))
M_numeric = np.array(M).astype(float)
# For Lagrange basis: M = [[1/3, 1/6], [1/6, 1/3]]
\end{lstlisting}

\textbf{Trace Matrix:} $T_{ij} = e_j(x_i)$ (transpose of Gramian)
\begin{lstlisting}[language=Python, caption=Trace Matrix Computation]
Gb = base.subs(y, x0) * hbase.T.subs(y, x0) + base.subs(y, x1) * hbase.T.subs(y, x1)
Trace = Gb.T
# For Lagrange basis: T = [[1, 0], [0, 1]]
\end{lstlisting}

\textbf{Derivative Matrix:} $D_{ij} = \int_0^1 e_j \frac{\partial e_i}{\partial x} \, dy$
\begin{lstlisting}[language=Python, caption=Derivative Matrix Computation]
D = sp.integrate(sp.diff(base, y) * base.T, (y, x0, x1))
# For Lagrange basis: D = [[1, -1], [1, -1]]
\end{lstlisting}

\paragraph{\_build\_quadrature()}\leavevmode
\begin{lstlisting}[language=Python, caption=Build Quadrature Method]
def _build_quadrature(self)
\end{lstlisting}

\textbf{Purpose:} Build quadrature matrices using Legendre-Gauss-Lobatto nodes

\textbf{Quadrature Rule:} 4-point Legendre-Gauss-Lobatto on [-1,1], transformed to [0,1]
\begin{lstlisting}[language=Python, caption=Quadrature Setup]
# LGL nodes on [-1,1]
qnodes = np.array([-1.0, -0.4472136, 0.4472136, 1.0])
qweights = np.array([1/6, 5/6, 5/6, 1/6])

# Transform to [0,1]
xi_01 = (qnodes + 1) / 2

# Build quadrature matrix: QUAD[i,j] = e_i(ξ_j) * w_j / 2
\end{lstlisting}

\subsubsection{Matrix Access Methods}

\paragraph{get\_matrix()}\leavevmode
\begin{lstlisting}[language=Python, caption=Get Matrix Method]
def get_matrix(self, name: str) -> np.ndarray
\end{lstlisting}

\textbf{Parameters:}
\begin{itemize}
    \item \texttt{name}: Matrix name (see available matrices table below)
\end{itemize}

\textbf{Returns:} \texttt{np.ndarray} - Requested elementary matrix

\textbf{Raises:} \texttt{KeyError} if matrix name not found

\textbf{Usage:}
\begin{lstlisting}[language=Python, caption=Get Matrix Usage]
elem_matrices = ElementaryMatrices()

# Get individual matrices
M = elem_matrices.get_matrix('M')        # Mass matrix
T = elem_matrices.get_matrix('T')        # Trace matrix  
D = elem_matrices.get_matrix('D')        # Derivative matrix
QUAD = elem_matrices.get_matrix('QUAD')  # Quadrature matrix

# Error handling
try:
    invalid = elem_matrices.get_matrix('INVALID')
except KeyError as e:
    print(f"Matrix not found: {e}")
\end{lstlisting}

\paragraph{get\_all\_matrices()}\leavevmode
\begin{lstlisting}[language=Python, caption=Get All Matrices Method]
def get_all_matrices(self) -> Dict[str, np.ndarray]
\end{lstlisting}

\textbf{Returns:} \texttt{Dict[str, np.ndarray]} - Copy of all elementary matrices

\textbf{Usage:}
\begin{lstlisting}[language=Python, caption=Get All Matrices Usage]
all_matrices = elem_matrices.get_all_matrices()

print("Available matrices:")
for name, matrix in all_matrices.items():
    print(f"  {name}: {matrix.shape}")

# Safe to modify returned dictionary
all_matrices['M'] *= 2.0  # Doesn't affect internal matrices
\end{lstlisting}

\subsubsection{Available Elementary Matrices}

\begin{longtable}{|p{2cm}|p{2.5cm}|p{8.5cm}|}
\hline
\textbf{Name} & \textbf{Shape} & \textbf{Description and Formula} \\
\hline
\endhead

\texttt{Z} & $(2, 2)$ & Zero matrix \\
\hline

\texttt{M} & $(2, 2)$ & Mass matrix: $M_{ij} = \int_0^1 e_j e_i \, dy$ \\
\hline

\texttt{IM} & $(2, 2)$ & Inverse mass matrix: $M^{-1}$ \\
\hline

\texttt{Gb} & $(2, 2)$ & Gramian matrix: $G^b_{ij} = e_i(0)\hat{e}_j(0) + e_i(1)\hat{e}_j(1)$ \\
\hline

\texttt{Mb} & $(2, 2)$ & Boundary mass matrix: $M^{\partial}_{ij} = e_i(0)e_j(0) + e_i(1)e_j(1)$ \\
\hline

\texttt{T} & $(2, 2)$ & Trace matrix: $T_{ij} = e_j(x_i)$ (transpose of Gramian) \\
\hline

\texttt{Av} & $(1, 2)$ & Averaging matrix: $(T^{-1} \mathbf{1})^T M$ where $\mathbf{1} = [1, 1]^T$ \\
\hline

\texttt{Ntil} & $(2, 2)$ & Normal matrix: $\tilde{N}_{ij} = e_j(1)e_i(1)n_1 + e_j(0)e_i(0)n_0$ \\
\hline

\texttt{Nhat} & $(2, 2)$ & Hat normal matrix: boundary basis times normal matrix \\
\hline

\texttt{D} & $(2, 2)$ & Derivative matrix: $D_{ij} = \int_0^1 e_j \frac{\partial e_i}{\partial x} \, dy$ \\
\hline

\texttt{qnodes} & $(4,)$ & Quadrature nodes on [-1,1]: $[-1, -0.4472136, 0.4472136, 1]$ \\
\hline

\texttt{QUAD} & $(2, 4)$ & Quadrature matrix: $QUAD_{ij} = e_i(\xi_j) w_j / 2$ \\
\hline

\end{longtable}

\subsubsection{Matrix Values for Lagrange Basis}

For the standard Lagrange basis $e_0(y) = 1-y$, $e_1(y) = y$:

\begin{lstlisting}[language=Python, caption=Lagrange Basis Matrix Values]
# Mass matrix
M = [[1/3, 1/6],
     [1/6, 1/3]]

# Trace matrix  
T = [[1, 0],
     [0, 1]]

# Derivative matrix
D = [[-1, -1],
     [ 1,  1]]

# Normal matrix (with n0=-1, n1=1)
Ntil = [[ 1,  0],
        [ 0, -1]]

# Averaging matrix
Av = [[1/2, 1/2]]

# Quadrature matrix (for 4 LGL points)
QUAD = [[0.91666667, 0.62360680, 0.37639320, 0.08333333],
        [0.08333333, 0.37639320, 0.62360680, 0.91666667]]
\end{lstlisting}

\subsubsection{Validation and Testing}

\paragraph{\_test\_matrices()}\leavevmode
\begin{lstlisting}[language=Python, caption=Test Matrices Method]
def _test_matrices(self, D, Ntil, M, Trace, Nhat, Gb)
\end{lstlisting}

\textbf{Purpose:} Perform mathematical validation tests (same as MATLAB code)

\textbf{Tests Performed:}

\textbf{Test 1: Integration by Parts}
\begin{align}
D + D^T - \tilde{N} = 0
\end{align}
This verifies the integration by parts identity for the derivative matrix.

\textbf{Test 2: Second Derivative Zero}
\begin{align}
(\tilde{N} - D) M^{-1} (D - \hat{N} T) = 0
\end{align}
This verifies that $\frac{d^2}{dx^2} = 0$ for linear functions.

\textbf{Test 3: Boundary Mass Consistency}
\begin{align}
M^{\partial} - G^b T = 0
\end{align}

\paragraph{print\_tests()}\leavevmode
\begin{lstlisting}[language=Python, caption=Print Tests Method]
def print_tests(self)
\end{lstlisting}

\textbf{Purpose:} Print validation test results

\textbf{Usage:}
\begin{lstlisting}[language=Python, caption=Print Tests Usage]
elem_matrices = ElementaryMatrices()
elem_matrices.print_tests()

# Sample output:
# Test: Integration by parts (should be zero):
# [[ 0.00000000e+00  0.00000000e+00]
#  [ 0.00000000e+00  0.00000000e+00]]
# 
# Test: d²x/dx² = 0 (should be zero):
# [[ 0.00000000e+00  0.00000000e+00]
#  [ 0.00000000e+00  0.00000000e+00]]
#
# Test: Mb - Gb*T (should be zero):
# [[ 0.00000000e+00  0.00000000e+00]
#  [ 0.00000000e+00  0.00000000e+00]]
\end{lstlisting}

\subsection{Complete Usage Examples}
\label{subsec:elementary_matrices_complete_examples}

\subsubsection{Basic Usage with HDG Method}

\begin{lstlisting}[language=Python, caption=Basic HDG Usage Example]
from ooc1d.utils.elementary_matrices import ElementaryMatrices
import numpy as np

# Create elementary matrices
elem_matrices = ElementaryMatrices()

# Validate matrices
elem_matrices.print_tests()

# Get matrices for HDG implementation
M = elem_matrices.get_matrix('M')        # Mass matrix
T = elem_matrices.get_matrix('T')        # Trace matrix
D = elem_matrices.get_matrix('D')        # Derivative matrix
QUAD = elem_matrices.get_matrix('QUAD')  # Quadrature matrix
qnodes = elem_matrices.get_matrix('qnodes')  # Quadrature nodes

print("Elementary matrices for HDG:")
print(f"Mass matrix M:\n{M}")
print(f"Trace matrix T:\n{T}")
print(f"Derivative matrix D:\n{D}")
print(f"Quadrature nodes: {qnodes}")
print(f"Quadrature matrix shape: {QUAD.shape}")

# Element length scaling (for physical elements)
h = 0.1  # Element length
M_scaled = h * M           # Scaled mass matrix
D_scaled = D               # Derivative matrix unchanged
QUAD_scaled = h * QUAD     # Scaled quadrature matrix

print(f"\nScaled matrices (h={h}):")
print(f"Scaled mass matrix:\n{M_scaled}")
print(f"Scaled quadrature matrix shape: {QUAD_scaled.shape}")
\end{lstlisting}

\subsubsection{Integration with BulkData}

\begin{lstlisting}[language=Python, caption=BulkData Integration Example]
from ooc1d.core.bulk_data import BulkData
from ooc1d.core.problem import Problem
from ooc1d.core.discretization import Discretization
from ooc1d.utils.elementary_matrices import ElementaryMatrices

# Setup problem and discretization
problem = Problem(neq=2, domain_start=0.0, domain_length=1.0)
discretization = Discretization(n_elements=10)

# Create elementary matrices
elem_matrices = ElementaryMatrices()

# Create BulkData with elementary matrices
bulk_data = BulkData(problem, discretization, dual=False)

# The BulkData constructor automatically uses ElementaryMatrices
print("BulkData created with elementary matrices:")
print(f"  Trace matrix from BulkData: {bulk_data.trace_matrix}")
print(f"  Mass matrix from BulkData: {bulk_data.mass_matrix}")

# Verify matrices match
T_elem = elem_matrices.get_matrix('T')
M_elem = elem_matrices.get_matrix('M')

print("Matrix consistency check:")
print(f"  Trace matrices match: {np.allclose(bulk_data.trace_matrix, T_elem)}")
print(f"  Mass matrices match: {np.allclose(bulk_data.mass_matrix / discretization.element_length, M_elem)}")
\end{lstlisting}

\subsubsection{Quadrature Integration Example}

\begin{lstlisting}[language=Python, caption=Quadrature Integration Example]
from ooc1d.utils.elementary_matrices import ElementaryMatrices
import numpy as np

# Create elementary matrices
elem_matrices = ElementaryMatrices()

# Get quadrature data
QUAD = elem_matrices.get_matrix('QUAD')
qnodes = elem_matrices.get_matrix('qnodes')

print("Quadrature integration example:")
print(f"Quadrature nodes on [-1,1]: {qnodes}")
print(f"Quadrature matrix shape: {QUAD.shape}")

# Define a test function to integrate
def test_function(x):
    """Test function: f(x) = x^2"""
    return x**2

# Transform quadrature nodes to [0,1] for integration
xi_01 = (qnodes + 1) / 2
f_values = test_function(xi_01)

print(f"Function values at quadrature points: {f_values}")

# Integrate using quadrature matrix
# For each basis function: ∫₀¹ eᵢ(x) f(x) dx ≈ Σⱼ eᵢ(ξⱼ) f(ξⱼ) wⱼ
integrated_coeffs = QUAD @ f_values

print(f"Integrated coefficients: {integrated_coeffs}")

# Analytical result for ∫₀¹ f(x) dx where f(x) = x²
analytical = 1/3
numerical = np.sum(integrated_coeffs)  # Sum of coefficients gives total integral
print(f"Analytical integral: {analytical:.6f}")
print(f"Numerical integral: {numerical:.6f}")
print(f"Error: {abs(numerical - analytical):.6e}")
\end{lstlisting}

\subsubsection{Orthonormal vs Lagrange Basis Comparison}

\begin{lstlisting}[language=Python, caption=Basis Comparison Example]
from ooc1d.utils.elementary_matrices import ElementaryMatrices
import numpy as np

# Create both types of elementary matrices
lagrange_matrices = ElementaryMatrices(orthonormal_basis=False)
ortho_matrices = ElementaryMatrices(orthonormal_basis=True)

print("Basis function comparison:")
print("="*50)

# Compare mass matrices
M_lagrange = lagrange_matrices.get_matrix('M')
M_ortho = ortho_matrices.get_matrix('M')

print("Lagrange mass matrix:")
print(M_lagrange)
print("\nOrthonormal mass matrix:")
print(M_ortho)

# Compare trace matrices
T_lagrange = lagrange_matrices.get_matrix('T')
T_ortho = ortho_matrices.get_matrix('T')

print("\nLagrange trace matrix:")
print(T_lagrange)
print("\nOrthonormal trace matrix:")
print(T_ortho)

# Test orthonormality
print("\nOrthonormality test for orthonormal basis:")
if np.allclose(M_ortho, np.eye(2), atol=1e-10):
    print("✓ Orthonormal mass matrix is identity")
else:
    print("✗ Orthonormal mass matrix is not identity")
    print(f"Difference from identity:\n{M_ortho - np.eye(2)}")

# Run tests for both
print("\nLagrange basis tests:")
lagrange_matrices.print_tests()

print("\nOrthonormal basis tests:")
ortho_matrices.print_tests()
\end{lstlisting}

\subsubsection{Static Condensation Integration}

\begin{lstlisting}[language=Python, caption=Static Condensation Integration Example]
# This example shows how ElementaryMatrices integrates with static condensation
from ooc1d.utils.elementary_matrices import ElementaryMatrices

def create_static_condensation_matrices(h, tau_values):
    """
    Create matrices for static condensation using elementary matrices.
    
    Args:
        h: Element length
        tau_values: List of stabilization parameters [tau_u, tau_phi, ...]
    
    Returns:
        Dictionary of static condensation matrices
    """
    # Get elementary matrices
    elem_matrices = ElementaryMatrices()
    
    # Basic matrices
    M = elem_matrices.get_matrix('M')
    T = elem_matrices.get_matrix('T')
    D = elem_matrices.get_matrix('D')
    Ntil = elem_matrices.get_matrix('Ntil')
    Mb = elem_matrices.get_matrix('Mb')
    
    # Scale by element length
    M_scaled = h * M
    Mb_scaled = h * Mb
    # D and T are scale-invariant for this element type
    
    # Build static condensation matrices
    sc_matrices = {}
    
    # Mass matrix with stabilization
    for i, tau in enumerate(tau_values):
        sc_matrices[f'A{i+1}'] = M_scaled + tau * Mb_scaled
        sc_matrices[f'L{i+1}'] = np.linalg.inv(sc_matrices[f'A{i+1}'])
        sc_matrices[f'B{i+1}'] = sc_matrices[f'L{i+1}'] @ T
    
    # Store basic matrices
    sc_matrices['M'] = M_scaled
    sc_matrices['T'] = T
    sc_matrices['D'] = D
    sc_matrices['Ntil'] = Ntil
    
    return sc_matrices

# Usage example
h = 0.1  # Element length
tau_values = [1.0, 1.0]  # Stabilization parameters for 2-equation system

sc_matrices = create_static_condensation_matrices(h, tau_values)

print("Static condensation matrices created:")
for name, matrix in sc_matrices.items():
    print(f"  {name}: {matrix.shape}")

# Example: compute bulk solution from trace values
trace_values = np.array([1.0, 0.5])  # Trace values at element boundaries
bulk_coeffs = sc_matrices['B1'] @ trace_values

print(f"\nTrace values: {trace_values}")
print(f"Bulk coefficients: {bulk_coeffs}")
\end{lstlisting}

\subsection{Method Summary Table}
\label{subsec:elementary_matrices_method_summary}

\begin{longtable}{|p{4.5cm}|p{2cm}|p{7cm}|}
\hline
\textbf{Method} & \textbf{Returns} & \textbf{Purpose} \\
\hline
\endhead

\texttt{\_\_init\_\_} & \texttt{None} & Initialize and compute all elementary matrices \\
\hline

\texttt{\_build\_matrices} & \texttt{None} & Build matrices using symbolic computation \\
\hline

\texttt{\_build\_basic\_matrices} & \texttt{None} & Compute mass, trace, derivative, and other basic matrices \\
\hline

\texttt{\_build\_quadrature} & \texttt{None} & Setup Legendre-Gauss-Lobatto quadrature \\
\hline

\texttt{\_test\_matrices} & \texttt{None} & Perform mathematical validation tests \\
\hline

\texttt{get\_matrix} & \texttt{np.ndarray} & Retrieve specific elementary matrix by name \\
\hline

\texttt{get\_all\_matrices} & \texttt{Dict} & Get all computed matrices as dictionary \\
\hline

\texttt{print\_tests} & \texttt{None} & Print validation test results \\
\hline

\end{longtable}

\subsection{Key Features and Mathematical Properties}

\begin{itemize}
    \item \textbf{MATLAB Compatibility}: Direct Python equivalent of MATLAB \texttt{build\_eMatrices.m}
    \item \textbf{Symbolic Computation}: Uses SymPy for exact mathematical operations
    \item \textbf{Multiple Basis Support}: Both Lagrange and orthonormal basis functions
    \item \textbf{Automatic Validation}: Built-in mathematical consistency tests
    \item \textbf{Quadrature Integration}: 4-point Legendre-Gauss-Lobatto rule
    \item \textbf{HDG Method Support}: All matrices needed for HDG static condensation
    \item \textbf{Element Scaling}: Proper scaling factors for physical element lengths
    \item \textbf{Reference Element}: All computations on standard interval [0,1]
\end{itemize}

This documentation provides an exact reference for the elementary matrices module, emphasizing its role as the foundation for HDG method computations and its seamless integration with other BioNetFlux components.

% End of elementary matrices module API documentation


\section{Elementary Matrices Module API Reference}
\label{sec:elementary_matrices_module_api}

This section provides reference  based on detailed analysis of the actual implementation for the  module \texttt{ooc1d.utils.elementary\_matrices.ElementaryMatrices}. The module computes elementary matrices for HDG methods on the reference element, serving as the Python equivalent of MATLAB \texttt{build\_eMatrices.m}.

\subsection{Module Overview}

The elementary matrices module provides:
\begin{itemize}
    \item Elementary matrix computation for HDG methods on reference element [0,1]
    \item Support for both Lagrange and orthonormal basis functions
    \item Symbolic computation using SymPy for exact mathematical operations
    \item Quadrature integration using Legendre-Gauss-Lobatto nodes
    \item Comprehensive matrix validation and testing
    \item MATLAB compatibility for existing HDG implementations
\end{itemize}

\subsection{Module Imports and Dependencies}

\begin{lstlisting}[language=Python, caption=Module Dependencies]
import numpy as np
import sympy as sp
from typing import Dict, Any
\end{lstlisting}

\subsection{ElementaryMatrices Class}
\label{subsec:elementary_matrices_class}

Main class for computing and managing elementary matrices on the reference element.

\subsubsection{Constructor}

\paragraph{\_\_init\_\_()}\leavevmode
\begin{lstlisting}[language=Python, caption=ElementaryMatrices Constructor]
def __init__(self, orthonormal_basis: bool = False)
\end{lstlisting}

\textbf{Parameters:}
\begin{itemize}
    \item \texttt{orthonormal\_basis}: Use orthonormal basis if True, Lagrange basis if False (default: False)
\end{itemize}

\textbf{Side Effects:}
\begin{itemize}
    \item Initializes all elementary matrices
    \item Stores matrices in internal dictionary
    \item Performs matrix validation tests
    \item Sets up quadrature rules
\end{itemize}

\textbf{Usage:}
\begin{lstlisting}[language=Python, caption=Constructor Usage Examples]
# Standard Lagrange basis (default)
elem_matrices = ElementaryMatrices()

# Orthonormal basis
elem_matrices_ortho = ElementaryMatrices(orthonormal_basis=True)

# Access matrices immediately after construction
mass_matrix = elem_matrices.get_matrix('M')
trace_matrix = elem_matrices.get_matrix('T')
\end{lstlisting}

\subsubsection{Core Attributes}

\begin{longtable}{|p{5.5cm}|p{2.5cm}|p{6cm}|}
\hline
\textbf{Attribute} & \textbf{Type} & \textbf{Description} \\
\hline
\endhead

\texttt{orthonormal\_basis} & \texttt{bool} & Flag indicating basis type used \\
\hline

\texttt{matrices} & \texttt{Dict[str, np.ndarray]} & Dictionary of all computed elementary matrices \\
\hline

\texttt{base} & \texttt{sp.Matrix} & SymPy symbolic basis functions \\
\hline

\texttt{hbase} & \texttt{sp.Matrix} & SymPy symbolic hat basis functions \\
\hline

\texttt{normali} & \texttt{sp.Matrix} & SymPy normal vectors at boundaries \\
\hline

\texttt{test\_integration\_by\_parts} & \texttt{np.ndarray} & Test matrix: $D + D^T - \tilde{N}$ (should be zero) \\
\hline

\texttt{test\_dxx\_zero} & \texttt{np.ndarray} & Test matrix for $\frac{d^2}{dx^2} = 0$ for linear functions \\
\hline

\texttt{test\_mb\_gb} & \texttt{np.ndarray} & Test matrix: $M^{\partial} - G^b T$ (should be zero) \\
\hline

\end{longtable}

\subsubsection{Matrix Construction Methods}

\paragraph{\_build\_matrices()}\leavevmode
\begin{lstlisting}[language=Python, caption=Build Matrices Method]
def _build_matrices(self)
\end{lstlisting}

\textbf{Purpose:} Build all elementary matrices on reference element [0,1]

\textbf{Process:}
\begin{enumerate}
    \item Define symbolic variable and reference interval
    \item Construct basis functions (Lagrange or orthonormal)
    \item Build basic matrices using symbolic integration
    \item Perform validation tests
    \item Build quadrature matrices
\end{enumerate}

\textbf{Basis Functions:}

\textbf{Lagrange Basis (default):}
\begin{align}
e_0(y) &= 1 - y \\
e_1(y) &= y
\end{align}

\textbf{Orthonormal Basis:} Constructed by solving orthogonality conditions:
\begin{align}
e_0 &= 1 \\
e_1 &= c \cdot y + d
\end{align}
Subject to: $\int_0^1 e_0 e_1 \, dy = 0$ and $\int_0^1 e_1^2 \, dy = 1$

\paragraph{\_build\_basic\_matrices()}\leavevmode
\begin{lstlisting}[language=Python, caption=Build Basic Matrices Method]
def _build_basic_matrices(self, y, base, hbase, normali, x0, x1)
\end{lstlisting}

\textbf{Parameters:}
\begin{itemize}
    \item \texttt{y}: SymPy symbolic variable
    \item \texttt{base}: Basis functions matrix
    \item \texttt{hbase}: Hat basis functions matrix
    \item \texttt{normali}: Normal vectors matrix
    \item \texttt{x0, x1}: Reference interval endpoints (0, 1)
\end{itemize}

\textbf{Matrices Computed:}

\textbf{Mass Matrix:} $M_{ij} = \int_0^1 e_j e_i \, dy$
\begin{lstlisting}[language=Python, caption=Mass Matrix Computation]
M = sp.integrate(base * base.T, (y, 0, 1))
M_numeric = np.array(M).astype(float)
# For Lagrange basis: M = [[1/3, 1/6], [1/6, 1/3]]
\end{lstlisting}

\textbf{Trace Matrix:} $T_{ij} = e_j(x_i)$ (transpose of Gramian)
\begin{lstlisting}[language=Python, caption=Trace Matrix Computation]
Gb = base.subs(y, x0) * hbase.T.subs(y, x0) + base.subs(y, x1) * hbase.T.subs(y, x1)
Trace = Gb.T
# For Lagrange basis: T = [[1, 0], [0, 1]]
\end{lstlisting}

\textbf{Derivative Matrix:} $D_{ij} = \int_0^1 e_j \frac{\partial e_i}{\partial x} \, dy$
\begin{lstlisting}[language=Python, caption=Derivative Matrix Computation]
D = sp.integrate(sp.diff(base, y) * base.T, (y, x0, x1))
# For Lagrange basis: D = [[1, -1], [1, -1]]
\end{lstlisting}

\paragraph{\_build\_quadrature()}\leavevmode
\begin{lstlisting}[language=Python, caption=Build Quadrature Method]
def _build_quadrature(self)
\end{lstlisting}

\textbf{Purpose:} Build quadrature matrices using Legendre-Gauss-Lobatto nodes

\textbf{Quadrature Rule:} 4-point Legendre-Gauss-Lobatto on [-1,1], transformed to [0,1]
\begin{lstlisting}[language=Python, caption=Quadrature Setup]
# LGL nodes on [-1,1]
qnodes = np.array([-1.0, -0.4472136, 0.4472136, 1.0])
qweights = np.array([1/6, 5/6, 5/6, 1/6])

# Transform to [0,1]
xi_01 = (qnodes + 1) / 2

# Build quadrature matrix: QUAD[i,j] = e_i(ξ_j) * w_j / 2
\end{lstlisting}

\subsubsection{Matrix Access Methods}

\paragraph{get\_matrix()}\leavevmode
\begin{lstlisting}[language=Python, caption=Get Matrix Method]
def get_matrix(self, name: str) -> np.ndarray
\end{lstlisting}

\textbf{Parameters:}
\begin{itemize}
    \item \texttt{name}: Matrix name (see available matrices table below)
\end{itemize}

\textbf{Returns:} \texttt{np.ndarray} - Requested elementary matrix

\textbf{Raises:} \texttt{KeyError} if matrix name not found

\textbf{Usage:}
\begin{lstlisting}[language=Python, caption=Get Matrix Usage]
elem_matrices = ElementaryMatrices()

# Get individual matrices
M = elem_matrices.get_matrix('M')        # Mass matrix
T = elem_matrices.get_matrix('T')        # Trace matrix  
D = elem_matrices.get_matrix('D')        # Derivative matrix
QUAD = elem_matrices.get_matrix('QUAD')  # Quadrature matrix

# Error handling
try:
    invalid = elem_matrices.get_matrix('INVALID')
except KeyError as e:
    print(f"Matrix not found: {e}")
\end{lstlisting}

\paragraph{get\_all\_matrices()}\leavevmode
\begin{lstlisting}[language=Python, caption=Get All Matrices Method]
def get_all_matrices(self) -> Dict[str, np.ndarray]
\end{lstlisting}

\textbf{Returns:} \texttt{Dict[str, np.ndarray]} - Copy of all elementary matrices

\textbf{Usage:}
\begin{lstlisting}[language=Python, caption=Get All Matrices Usage]
all_matrices = elem_matrices.get_all_matrices()

print("Available matrices:")
for name, matrix in all_matrices.items():
    print(f"  {name}: {matrix.shape}")

# Safe to modify returned dictionary
all_matrices['M'] *= 2.0  # Doesn't affect internal matrices
\end{lstlisting}

\subsubsection{Available Elementary Matrices}

\begin{longtable}{|p{2cm}|p{2.5cm}|p{8.5cm}|}
\hline
\textbf{Name} & \textbf{Shape} & \textbf{Description and Formula} \\
\hline
\endhead

\texttt{Z} & $(2, 2)$ & Zero matrix \\
\hline

\texttt{M} & $(2, 2)$ & Mass matrix: $M_{ij} = \int_0^1 e_j e_i \, dy$ \\
\hline

\texttt{IM} & $(2, 2)$ & Inverse mass matrix: $M^{-1}$ \\
\hline

\texttt{Gb} & $(2, 2)$ & Gramian matrix: $G^b_{ij} = e_i(0)\hat{e}_j(0) + e_i(1)\hat{e}_j(1)$ \\
\hline

\texttt{Mb} & $(2, 2)$ & Boundary mass matrix: $M^{\partial}_{ij} = e_i(0)e_j(0) + e_i(1)e_j(1)$ \\
\hline

\texttt{T} & $(2, 2)$ & Trace matrix: $T_{ij} = e_j(x_i)$ (transpose of Gramian) \\
\hline

\texttt{Av} & $(1, 2)$ & Averaging matrix: $(T^{-1} \mathbf{1})^T M$ where $\mathbf{1} = [1, 1]^T$ \\
\hline

\texttt{Ntil} & $(2, 2)$ & Normal matrix: $\tilde{N}_{ij} = e_j(1)e_i(1)n_1 + e_j(0)e_i(0)n_0$ \\
\hline

\texttt{Nhat} & $(2, 2)$ & Hat normal matrix: boundary basis times normal matrix \\
\hline

\texttt{D} & $(2, 2)$ & Derivative matrix: $D_{ij} = \int_0^1 e_j \frac{\partial e_i}{\partial x} \, dy$ \\
\hline

\texttt{qnodes} & $(4,)$ & Quadrature nodes on [-1,1]: $[-1, -0.4472136, 0.4472136, 1]$ \\
\hline

\texttt{QUAD} & $(2, 4)$ & Quadrature matrix: $QUAD_{ij} = e_i(\xi_j) w_j / 2$ \\
\hline

\end{longtable}

\subsubsection{Matrix Values for Lagrange Basis}

For the standard Lagrange basis $e_0(y) = 1-y$, $e_1(y) = y$:

\begin{lstlisting}[language=Python, caption=Lagrange Basis Matrix Values]
# Mass matrix
M = [[1/3, 1/6],
     [1/6, 1/3]]

# Trace matrix  
T = [[1, 0],
     [0, 1]]

# Derivative matrix
D = [[-1, -1],
     [ 1,  1]]

# Normal matrix (with n0=-1, n1=1)
Ntil = [[ 1,  0],
        [ 0, -1]]

# Averaging matrix
Av = [[1/2, 1/2]]

# Quadrature matrix (for 4 LGL points)
QUAD = [[0.91666667, 0.62360680, 0.37639320, 0.08333333],
        [0.08333333, 0.37639320, 0.62360680, 0.91666667]]
\end{lstlisting}

\subsubsection{Validation and Testing}

\paragraph{\_test\_matrices()}\leavevmode
\begin{lstlisting}[language=Python, caption=Test Matrices Method]
def _test_matrices(self, D, Ntil, M, Trace, Nhat, Gb)
\end{lstlisting}

\textbf{Purpose:} Perform mathematical validation tests (same as MATLAB code)

\textbf{Tests Performed:}

\textbf{Test 1: Integration by Parts}
\begin{align}
D + D^T - \tilde{N} = 0
\end{align}
This verifies the integration by parts identity for the derivative matrix.

\textbf{Test 2: Second Derivative Zero}
\begin{align}
(\tilde{N} - D) M^{-1} (D - \hat{N} T) = 0
\end{align}
This verifies that $\frac{d^2}{dx^2} = 0$ for linear functions.

\textbf{Test 3: Boundary Mass Consistency}
\begin{align}
M^{\partial} - G^b T = 0
\end{align}

\paragraph{print\_tests()}\leavevmode
\begin{lstlisting}[language=Python, caption=Print Tests Method]
def print_tests(self)
\end{lstlisting}

\textbf{Purpose:} Print validation test results

\textbf{Usage:}
\begin{lstlisting}[language=Python, caption=Print Tests Usage]
elem_matrices = ElementaryMatrices()
elem_matrices.print_tests()

# Sample output:
# Test: Integration by parts (should be zero):
# [[ 0.00000000e+00  0.00000000e+00]
#  [ 0.00000000e+00  0.00000000e+00]]
# 
# Test: d²x/dx² = 0 (should be zero):
# [[ 0.00000000e+00  0.00000000e+00]
#  [ 0.00000000e+00  0.00000000e+00]]
#
# Test: Mb - Gb*T (should be zero):
# [[ 0.00000000e+00  0.00000000e+00]
#  [ 0.00000000e+00  0.00000000e+00]]
\end{lstlisting}

\subsection{Complete Usage Examples}
\label{subsec:elementary_matrices_complete_examples}

\subsubsection{Basic Usage with HDG Method}

\begin{lstlisting}[language=Python, caption=Basic HDG Usage Example]
from ooc1d.utils.elementary_matrices import ElementaryMatrices
import numpy as np

# Create elementary matrices
elem_matrices = ElementaryMatrices()

# Validate matrices
elem_matrices.print_tests()

# Get matrices for HDG implementation
M = elem_matrices.get_matrix('M')        # Mass matrix
T = elem_matrices.get_matrix('T')        # Trace matrix
D = elem_matrices.get_matrix('D')        # Derivative matrix
QUAD = elem_matrices.get_matrix('QUAD')  # Quadrature matrix
qnodes = elem_matrices.get_matrix('qnodes')  # Quadrature nodes

print("Elementary matrices for HDG:")
print(f"Mass matrix M:\n{M}")
print(f"Trace matrix T:\n{T}")
print(f"Derivative matrix D:\n{D}")
print(f"Quadrature nodes: {qnodes}")
print(f"Quadrature matrix shape: {QUAD.shape}")

# Element length scaling (for physical elements)
h = 0.1  # Element length
M_scaled = h * M           # Scaled mass matrix
D_scaled = D               # Derivative matrix unchanged
QUAD_scaled = h * QUAD     # Scaled quadrature matrix

print(f"\nScaled matrices (h={h}):")
print(f"Scaled mass matrix:\n{M_scaled}")
print(f"Scaled quadrature matrix shape: {QUAD_scaled.shape}")
\end{lstlisting}

\subsubsection{Integration with BulkData}

\begin{lstlisting}[language=Python, caption=BulkData Integration Example]
from ooc1d.core.bulk_data import BulkData
from ooc1d.core.problem import Problem
from ooc1d.core.discretization import Discretization
from ooc1d.utils.elementary_matrices import ElementaryMatrices

# Setup problem and discretization
problem = Problem(neq=2, domain_start=0.0, domain_length=1.0)
discretization = Discretization(n_elements=10)

# Create elementary matrices
elem_matrices = ElementaryMatrices()

# Create BulkData with elementary matrices
bulk_data = BulkData(problem, discretization, dual=False)

# The BulkData constructor automatically uses ElementaryMatrices
print("BulkData created with elementary matrices:")
print(f"  Trace matrix from BulkData: {bulk_data.trace_matrix}")
print(f"  Mass matrix from BulkData: {bulk_data.mass_matrix}")

# Verify matrices match
T_elem = elem_matrices.get_matrix('T')
M_elem = elem_matrices.get_matrix('M')

print("Matrix consistency check:")
print(f"  Trace matrices match: {np.allclose(bulk_data.trace_matrix, T_elem)}")
print(f"  Mass matrices match: {np.allclose(bulk_data.mass_matrix / discretization.element_length, M_elem)}")
\end{lstlisting}

\subsubsection{Quadrature Integration Example}

\begin{lstlisting}[language=Python, caption=Quadrature Integration Example]
from ooc1d.utils.elementary_matrices import ElementaryMatrices
import numpy as np

# Create elementary matrices
elem_matrices = ElementaryMatrices()

# Get quadrature data
QUAD = elem_matrices.get_matrix('QUAD')
qnodes = elem_matrices.get_matrix('qnodes')

print("Quadrature integration example:")
print(f"Quadrature nodes on [-1,1]: {qnodes}")
print(f"Quadrature matrix shape: {QUAD.shape}")

# Define a test function to integrate
def test_function(x):
    """Test function: f(x) = x^2"""
    return x**2

# Transform quadrature nodes to [0,1] for integration
xi_01 = (qnodes + 1) / 2
f_values = test_function(xi_01)

print(f"Function values at quadrature points: {f_values}")

# Integrate using quadrature matrix
# For each basis function: ∫₀¹ eᵢ(x) f(x) dx ≈ Σⱼ eᵢ(ξⱼ) f(ξⱼ) wⱼ
integrated_coeffs = QUAD @ f_values

print(f"Integrated coefficients: {integrated_coeffs}")

# Analytical result for ∫₀¹ f(x) dx where f(x) = x²
analytical = 1/3
numerical = np.sum(integrated_coeffs)  # Sum of coefficients gives total integral
print(f"Analytical integral: {analytical:.6f}")
print(f"Numerical integral: {numerical:.6f}")
print(f"Error: {abs(numerical - analytical):.6e}")
\end{lstlisting}

\subsubsection{Orthonormal vs Lagrange Basis Comparison}

\begin{lstlisting}[language=Python, caption=Basis Comparison Example]
from ooc1d.utils.elementary_matrices import ElementaryMatrices
import numpy as np

# Create both types of elementary matrices
lagrange_matrices = ElementaryMatrices(orthonormal_basis=False)
ortho_matrices = ElementaryMatrices(orthonormal_basis=True)

print("Basis function comparison:")
print("="*50)

# Compare mass matrices
M_lagrange = lagrange_matrices.get_matrix('M')
M_ortho = ortho_matrices.get_matrix('M')

print("Lagrange mass matrix:")
print(M_lagrange)
print("\nOrthonormal mass matrix:")
print(M_ortho)

# Compare trace matrices
T_lagrange = lagrange_matrices.get_matrix('T')
T_ortho = ortho_matrices.get_matrix('T')

print("\nLagrange trace matrix:")
print(T_lagrange)
print("\nOrthonormal trace matrix:")
print(T_ortho)

# Test orthonormality
print("\nOrthonormality test for orthonormal basis:")
if np.allclose(M_ortho, np.eye(2), atol=1e-10):
    print("✓ Orthonormal mass matrix is identity")
else:
    print("✗ Orthonormal mass matrix is not identity")
    print(f"Difference from identity:\n{M_ortho - np.eye(2)}")

# Run tests for both
print("\nLagrange basis tests:")
lagrange_matrices.print_tests()

print("\nOrthonormal basis tests:")
ortho_matrices.print_tests()
\end{lstlisting}

\subsubsection{Static Condensation Integration}

\begin{lstlisting}[language=Python, caption=Static Condensation Integration Example]
# This example shows how ElementaryMatrices integrates with static condensation
from ooc1d.utils.elementary_matrices import ElementaryMatrices

def create_static_condensation_matrices(h, tau_values):
    """
    Create matrices for static condensation using elementary matrices.
    
    Args:
        h: Element length
        tau_values: List of stabilization parameters [tau_u, tau_phi, ...]
    
    Returns:
        Dictionary of static condensation matrices
    """
    # Get elementary matrices
    elem_matrices = ElementaryMatrices()
    
    # Basic matrices
    M = elem_matrices.get_matrix('M')
    T = elem_matrices.get_matrix('T')
    D = elem_matrices.get_matrix('D')
    Ntil = elem_matrices.get_matrix('Ntil')
    Mb = elem_matrices.get_matrix('Mb')
    
    # Scale by element length
    M_scaled = h * M
    Mb_scaled = h * Mb
    # D and T are scale-invariant for this element type
    
    # Build static condensation matrices
    sc_matrices = {}
    
    # Mass matrix with stabilization
    for i, tau in enumerate(tau_values):
        sc_matrices[f'A{i+1}'] = M_scaled + tau * Mb_scaled
        sc_matrices[f'L{i+1}'] = np.linalg.inv(sc_matrices[f'A{i+1}'])
        sc_matrices[f'B{i+1}'] = sc_matrices[f'L{i+1}'] @ T
    
    # Store basic matrices
    sc_matrices['M'] = M_scaled
    sc_matrices['T'] = T
    sc_matrices['D'] = D
    sc_matrices['Ntil'] = Ntil
    
    return sc_matrices

# Usage example
h = 0.1  # Element length
tau_values = [1.0, 1.0]  # Stabilization parameters for 2-equation system

sc_matrices = create_static_condensation_matrices(h, tau_values)

print("Static condensation matrices created:")
for name, matrix in sc_matrices.items():
    print(f"  {name}: {matrix.shape}")

# Example: compute bulk solution from trace values
trace_values = np.array([1.0, 0.5])  # Trace values at element boundaries
bulk_coeffs = sc_matrices['B1'] @ trace_values

print(f"\nTrace values: {trace_values}")
print(f"Bulk coefficients: {bulk_coeffs}")
\end{lstlisting}

\subsection{Method Summary Table}
\label{subsec:elementary_matrices_method_summary}

\begin{longtable}{|p{4.5cm}|p{2cm}|p{7cm}|}
\hline
\textbf{Method} & \textbf{Returns} & \textbf{Purpose} \\
\hline
\endhead

\texttt{\_\_init\_\_} & \texttt{None} & Initialize and compute all elementary matrices \\
\hline

\texttt{\_build\_matrices} & \texttt{None} & Build matrices using symbolic computation \\
\hline

\texttt{\_build\_basic\_matrices} & \texttt{None} & Compute mass, trace, derivative, and other basic matrices \\
\hline

\texttt{\_build\_quadrature} & \texttt{None} & Setup Legendre-Gauss-Lobatto quadrature \\
\hline

\texttt{\_test\_matrices} & \texttt{None} & Perform mathematical validation tests \\
\hline

\texttt{get\_matrix} & \texttt{np.ndarray} & Retrieve specific elementary matrix by name \\
\hline

\texttt{get\_all\_matrices} & \texttt{Dict} & Get all computed matrices as dictionary \\
\hline

\texttt{print\_tests} & \texttt{None} & Print validation test results \\
\hline

\end{longtable}

\subsection{Key Features and Mathematical Properties}

\begin{itemize}
    \item \textbf{MATLAB Compatibility}: Direct Python equivalent of MATLAB \texttt{build\_eMatrices.m}
    \item \textbf{Symbolic Computation}: Uses SymPy for exact mathematical operations
    \item \textbf{Multiple Basis Support}: Both Lagrange and orthonormal basis functions
    \item \textbf{Automatic Validation}: Built-in mathematical consistency tests
    \item \textbf{Quadrature Integration}: 4-point Legendre-Gauss-Lobatto rule
    \item \textbf{HDG Method Support}: All matrices needed for HDG static condensation
    \item \textbf{Element Scaling}: Proper scaling factors for physical element lengths
    \item \textbf{Reference Element}: All computations on standard interval [0,1]
\end{itemize}

This documentation provides an exact reference for the elementary matrices module, emphasizing its role as the foundation for HDG method computations and its seamless integration with other BioNetFlux components.

% End of elementary matrices module API documentation


\section{Elementary Matrices Module API Reference}
\label{sec:elementary_matrices_module_api}

This section provides reference  based on detailed analysis of the actual implementation for the  module \texttt{ooc1d.utils.elementary\_matrices.ElementaryMatrices}. The module computes elementary matrices for HDG methods on the reference element, serving as the Python equivalent of MATLAB \texttt{build\_eMatrices.m}.

\subsection{Module Overview}

The elementary matrices module provides:
\begin{itemize}
    \item Elementary matrix computation for HDG methods on reference element [0,1]
    \item Support for both Lagrange and orthonormal basis functions
    \item Symbolic computation using SymPy for exact mathematical operations
    \item Quadrature integration using Legendre-Gauss-Lobatto nodes
    \item Comprehensive matrix validation and testing
    \item MATLAB compatibility for existing HDG implementations
\end{itemize}

\subsection{Module Imports and Dependencies}

\begin{lstlisting}[language=Python, caption=Module Dependencies]
import numpy as np
import sympy as sp
from typing import Dict, Any
\end{lstlisting}

\subsection{ElementaryMatrices Class}
\label{subsec:elementary_matrices_class}

Main class for computing and managing elementary matrices on the reference element.

\subsubsection{Constructor}

\paragraph{\_\_init\_\_()}\leavevmode
\begin{lstlisting}[language=Python, caption=ElementaryMatrices Constructor]
def __init__(self, orthonormal_basis: bool = False)
\end{lstlisting}

\textbf{Parameters:}
\begin{itemize}
    \item \texttt{orthonormal\_basis}: Use orthonormal basis if True, Lagrange basis if False (default: False)
\end{itemize}

\textbf{Side Effects:}
\begin{itemize}
    \item Initializes all elementary matrices
    \item Stores matrices in internal dictionary
    \item Performs matrix validation tests
    \item Sets up quadrature rules
\end{itemize}

\textbf{Usage:}
\begin{lstlisting}[language=Python, caption=Constructor Usage Examples]
# Standard Lagrange basis (default)
elem_matrices = ElementaryMatrices()

# Orthonormal basis
elem_matrices_ortho = ElementaryMatrices(orthonormal_basis=True)

# Access matrices immediately after construction
mass_matrix = elem_matrices.get_matrix('M')
trace_matrix = elem_matrices.get_matrix('T')
\end{lstlisting}

\subsubsection{Core Attributes}

\begin{longtable}{|p{5.5cm}|p{2.5cm}|p{6cm}|}
\hline
\textbf{Attribute} & \textbf{Type} & \textbf{Description} \\
\hline
\endhead

\texttt{orthonormal\_basis} & \texttt{bool} & Flag indicating basis type used \\
\hline

\texttt{matrices} & \texttt{Dict[str, np.ndarray]} & Dictionary of all computed elementary matrices \\
\hline

\texttt{base} & \texttt{sp.Matrix} & SymPy symbolic basis functions \\
\hline

\texttt{hbase} & \texttt{sp.Matrix} & SymPy symbolic hat basis functions \\
\hline

\texttt{normali} & \texttt{sp.Matrix} & SymPy normal vectors at boundaries \\
\hline

\texttt{test\_integration\_by\_parts} & \texttt{np.ndarray} & Test matrix: $D + D^T - \tilde{N}$ (should be zero) \\
\hline

\texttt{test\_dxx\_zero} & \texttt{np.ndarray} & Test matrix for $\frac{d^2}{dx^2} = 0$ for linear functions \\
\hline

\texttt{test\_mb\_gb} & \texttt{np.ndarray} & Test matrix: $M^{\partial} - G^b T$ (should be zero) \\
\hline

\end{longtable}

\subsubsection{Matrix Construction Methods}

\paragraph{\_build\_matrices()}\leavevmode
\begin{lstlisting}[language=Python, caption=Build Matrices Method]
def _build_matrices(self)
\end{lstlisting}

\textbf{Purpose:} Build all elementary matrices on reference element [0,1]

\textbf{Process:}
\begin{enumerate}
    \item Define symbolic variable and reference interval
    \item Construct basis functions (Lagrange or orthonormal)
    \item Build basic matrices using symbolic integration
    \item Perform validation tests
    \item Build quadrature matrices
\end{enumerate}

\textbf{Basis Functions:}

\textbf{Lagrange Basis (default):}
\begin{align}
e_0(y) &= 1 - y \\
e_1(y) &= y
\end{align}

\textbf{Orthonormal Basis:} Constructed by solving orthogonality conditions:
\begin{align}
e_0 &= 1 \\
e_1 &= c \cdot y + d
\end{align}
Subject to: $\int_0^1 e_0 e_1 \, dy = 0$ and $\int_0^1 e_1^2 \, dy = 1$

\paragraph{\_build\_basic\_matrices()}\leavevmode
\begin{lstlisting}[language=Python, caption=Build Basic Matrices Method]
def _build_basic_matrices(self, y, base, hbase, normali, x0, x1)
\end{lstlisting}

\textbf{Parameters:}
\begin{itemize}
    \item \texttt{y}: SymPy symbolic variable
    \item \texttt{base}: Basis functions matrix
    \item \texttt{hbase}: Hat basis functions matrix
    \item \texttt{normali}: Normal vectors matrix
    \item \texttt{x0, x1}: Reference interval endpoints (0, 1)
\end{itemize}

\textbf{Matrices Computed:}

\textbf{Mass Matrix:} $M_{ij} = \int_0^1 e_j e_i \, dy$
\begin{lstlisting}[language=Python, caption=Mass Matrix Computation]
M = sp.integrate(base * base.T, (y, 0, 1))
M_numeric = np.array(M).astype(float)
# For Lagrange basis: M = [[1/3, 1/6], [1/6, 1/3]]
\end{lstlisting}

\textbf{Trace Matrix:} $T_{ij} = e_j(x_i)$ (transpose of Gramian)
\begin{lstlisting}[language=Python, caption=Trace Matrix Computation]
Gb = base.subs(y, x0) * hbase.T.subs(y, x0) + base.subs(y, x1) * hbase.T.subs(y, x1)
Trace = Gb.T
# For Lagrange basis: T = [[1, 0], [0, 1]]
\end{lstlisting}

\textbf{Derivative Matrix:} $D_{ij} = \int_0^1 e_j \frac{\partial e_i}{\partial x} \, dy$
\begin{lstlisting}[language=Python, caption=Derivative Matrix Computation]
D = sp.integrate(sp.diff(base, y) * base.T, (y, x0, x1))
# For Lagrange basis: D = [[1, -1], [1, -1]]
\end{lstlisting}

\paragraph{\_build\_quadrature()}\leavevmode
\begin{lstlisting}[language=Python, caption=Build Quadrature Method]
def _build_quadrature(self)
\end{lstlisting}

\textbf{Purpose:} Build quadrature matrices using Legendre-Gauss-Lobatto nodes

\textbf{Quadrature Rule:} 4-point Legendre-Gauss-Lobatto on [-1,1], transformed to [0,1]
\begin{lstlisting}[language=Python, caption=Quadrature Setup]
# LGL nodes on [-1,1]
qnodes = np.array([-1.0, -0.4472136, 0.4472136, 1.0])
qweights = np.array([1/6, 5/6, 5/6, 1/6])

# Transform to [0,1]
xi_01 = (qnodes + 1) / 2

# Build quadrature matrix: QUAD[i,j] = e_i(ξ_j) * w_j / 2
\end{lstlisting}

\subsubsection{Matrix Access Methods}

\paragraph{get\_matrix()}\leavevmode
\begin{lstlisting}[language=Python, caption=Get Matrix Method]
def get_matrix(self, name: str) -> np.ndarray
\end{lstlisting}

\textbf{Parameters:}
\begin{itemize}
    \item \texttt{name}: Matrix name (see available matrices table below)
\end{itemize}

\textbf{Returns:} \texttt{np.ndarray} - Requested elementary matrix

\textbf{Raises:} \texttt{KeyError} if matrix name not found

\textbf{Usage:}
\begin{lstlisting}[language=Python, caption=Get Matrix Usage]
elem_matrices = ElementaryMatrices()

# Get individual matrices
M = elem_matrices.get_matrix('M')        # Mass matrix
T = elem_matrices.get_matrix('T')        # Trace matrix  
D = elem_matrices.get_matrix('D')        # Derivative matrix
QUAD = elem_matrices.get_matrix('QUAD')  # Quadrature matrix

# Error handling
try:
    invalid = elem_matrices.get_matrix('INVALID')
except KeyError as e:
    print(f"Matrix not found: {e}")
\end{lstlisting}

\paragraph{get\_all\_matrices()}\leavevmode
\begin{lstlisting}[language=Python, caption=Get All Matrices Method]
def get_all_matrices(self) -> Dict[str, np.ndarray]
\end{lstlisting}

\textbf{Returns:} \texttt{Dict[str, np.ndarray]} - Copy of all elementary matrices

\textbf{Usage:}
\begin{lstlisting}[language=Python, caption=Get All Matrices Usage]
all_matrices = elem_matrices.get_all_matrices()

print("Available matrices:")
for name, matrix in all_matrices.items():
    print(f"  {name}: {matrix.shape}")

# Safe to modify returned dictionary
all_matrices['M'] *= 2.0  # Doesn't affect internal matrices
\end{lstlisting}

\subsubsection{Available Elementary Matrices}

\begin{longtable}{|p{2cm}|p{2.5cm}|p{8.5cm}|}
\hline
\textbf{Name} & \textbf{Shape} & \textbf{Description and Formula} \\
\hline
\endhead

\texttt{Z} & $(2, 2)$ & Zero matrix \\
\hline

\texttt{M} & $(2, 2)$ & Mass matrix: $M_{ij} = \int_0^1 e_j e_i \, dy$ \\
\hline

\texttt{IM} & $(2, 2)$ & Inverse mass matrix: $M^{-1}$ \\
\hline

\texttt{Gb} & $(2, 2)$ & Gramian matrix: $G^b_{ij} = e_i(0)\hat{e}_j(0) + e_i(1)\hat{e}_j(1)$ \\
\hline

\texttt{Mb} & $(2, 2)$ & Boundary mass matrix: $M^{\partial}_{ij} = e_i(0)e_j(0) + e_i(1)e_j(1)$ \\
\hline

\texttt{T} & $(2, 2)$ & Trace matrix: $T_{ij} = e_j(x_i)$ (transpose of Gramian) \\
\hline

\texttt{Av} & $(1, 2)$ & Averaging matrix: $(T^{-1} \mathbf{1})^T M$ where $\mathbf{1} = [1, 1]^T$ \\
\hline

\texttt{Ntil} & $(2, 2)$ & Normal matrix: $\tilde{N}_{ij} = e_j(1)e_i(1)n_1 + e_j(0)e_i(0)n_0$ \\
\hline

\texttt{Nhat} & $(2, 2)$ & Hat normal matrix: boundary basis times normal matrix \\
\hline

\texttt{D} & $(2, 2)$ & Derivative matrix: $D_{ij} = \int_0^1 e_j \frac{\partial e_i}{\partial x} \, dy$ \\
\hline

\texttt{qnodes} & $(4,)$ & Quadrature nodes on [-1,1]: $[-1, -0.4472136, 0.4472136, 1]$ \\
\hline

\texttt{QUAD} & $(2, 4)$ & Quadrature matrix: $QUAD_{ij} = e_i(\xi_j) w_j / 2$ \\
\hline

\end{longtable}

\subsubsection{Matrix Values for Lagrange Basis}

For the standard Lagrange basis $e_0(y) = 1-y$, $e_1(y) = y$:

\begin{lstlisting}[language=Python, caption=Lagrange Basis Matrix Values]
# Mass matrix
M = [[1/3, 1/6],
     [1/6, 1/3]]

# Trace matrix  
T = [[1, 0],
     [0, 1]]

# Derivative matrix
D = [[-1, -1],
     [ 1,  1]]

# Normal matrix (with n0=-1, n1=1)
Ntil = [[ 1,  0],
        [ 0, -1]]

# Averaging matrix
Av = [[1/2, 1/2]]

# Quadrature matrix (for 4 LGL points)
QUAD = [[0.91666667, 0.62360680, 0.37639320, 0.08333333],
        [0.08333333, 0.37639320, 0.62360680, 0.91666667]]
\end{lstlisting}

\subsubsection{Validation and Testing}

\paragraph{\_test\_matrices()}\leavevmode
\begin{lstlisting}[language=Python, caption=Test Matrices Method]
def _test_matrices(self, D, Ntil, M, Trace, Nhat, Gb)
\end{lstlisting}

\textbf{Purpose:} Perform mathematical validation tests (same as MATLAB code)

\textbf{Tests Performed:}

\textbf{Test 1: Integration by Parts}
\begin{align}
D + D^T - \tilde{N} = 0
\end{align}
This verifies the integration by parts identity for the derivative matrix.

\textbf{Test 2: Second Derivative Zero}
\begin{align}
(\tilde{N} - D) M^{-1} (D - \hat{N} T) = 0
\end{align}
This verifies that $\frac{d^2}{dx^2} = 0$ for linear functions.

\textbf{Test 3: Boundary Mass Consistency}
\begin{align}
M^{\partial} - G^b T = 0
\end{align}

\paragraph{print\_tests()}\leavevmode
\begin{lstlisting}[language=Python, caption=Print Tests Method]
def print_tests(self)
\end{lstlisting}

\textbf{Purpose:} Print validation test results

\textbf{Usage:}
\begin{lstlisting}[language=Python, caption=Print Tests Usage]
elem_matrices = ElementaryMatrices()
elem_matrices.print_tests()

# Sample output:
# Test: Integration by parts (should be zero):
# [[ 0.00000000e+00  0.00000000e+00]
#  [ 0.00000000e+00  0.00000000e+00]]
# 
# Test: d²x/dx² = 0 (should be zero):
# [[ 0.00000000e+00  0.00000000e+00]
#  [ 0.00000000e+00  0.00000000e+00]]
#
# Test: Mb - Gb*T (should be zero):
# [[ 0.00000000e+00  0.00000000e+00]
#  [ 0.00000000e+00  0.00000000e+00]]
\end{lstlisting}

\subsection{Complete Usage Examples}
\label{subsec:elementary_matrices_complete_examples}

\subsubsection{Basic Usage with HDG Method}

\begin{lstlisting}[language=Python, caption=Basic HDG Usage Example]
from ooc1d.utils.elementary_matrices import ElementaryMatrices
import numpy as np

# Create elementary matrices
elem_matrices = ElementaryMatrices()

# Validate matrices
elem_matrices.print_tests()

# Get matrices for HDG implementation
M = elem_matrices.get_matrix('M')        # Mass matrix
T = elem_matrices.get_matrix('T')        # Trace matrix
D = elem_matrices.get_matrix('D')        # Derivative matrix
QUAD = elem_matrices.get_matrix('QUAD')  # Quadrature matrix
qnodes = elem_matrices.get_matrix('qnodes')  # Quadrature nodes

print("Elementary matrices for HDG:")
print(f"Mass matrix M:\n{M}")
print(f"Trace matrix T:\n{T}")
print(f"Derivative matrix D:\n{D}")
print(f"Quadrature nodes: {qnodes}")
print(f"Quadrature matrix shape: {QUAD.shape}")

# Element length scaling (for physical elements)
h = 0.1  # Element length
M_scaled = h * M           # Scaled mass matrix
D_scaled = D               # Derivative matrix unchanged
QUAD_scaled = h * QUAD     # Scaled quadrature matrix

print(f"\nScaled matrices (h={h}):")
print(f"Scaled mass matrix:\n{M_scaled}")
print(f"Scaled quadrature matrix shape: {QUAD_scaled.shape}")
\end{lstlisting}

\subsubsection{Integration with BulkData}

\begin{lstlisting}[language=Python, caption=BulkData Integration Example]
from ooc1d.core.bulk_data import BulkData
from ooc1d.core.problem import Problem
from ooc1d.core.discretization import Discretization
from ooc1d.utils.elementary_matrices import ElementaryMatrices

# Setup problem and discretization
problem = Problem(neq=2, domain_start=0.0, domain_length=1.0)
discretization = Discretization(n_elements=10)

# Create elementary matrices
elem_matrices = ElementaryMatrices()

# Create BulkData with elementary matrices
bulk_data = BulkData(problem, discretization, dual=False)

# The BulkData constructor automatically uses ElementaryMatrices
print("BulkData created with elementary matrices:")
print(f"  Trace matrix from BulkData: {bulk_data.trace_matrix}")
print(f"  Mass matrix from BulkData: {bulk_data.mass_matrix}")

# Verify matrices match
T_elem = elem_matrices.get_matrix('T')
M_elem = elem_matrices.get_matrix('M')

print("Matrix consistency check:")
print(f"  Trace matrices match: {np.allclose(bulk_data.trace_matrix, T_elem)}")
print(f"  Mass matrices match: {np.allclose(bulk_data.mass_matrix / discretization.element_length, M_elem)}")
\end{lstlisting}

\subsubsection{Quadrature Integration Example}

\begin{lstlisting}[language=Python, caption=Quadrature Integration Example]
from ooc1d.utils.elementary_matrices import ElementaryMatrices
import numpy as np

# Create elementary matrices
elem_matrices = ElementaryMatrices()

# Get quadrature data
QUAD = elem_matrices.get_matrix('QUAD')
qnodes = elem_matrices.get_matrix('qnodes')

print("Quadrature integration example:")
print(f"Quadrature nodes on [-1,1]: {qnodes}")
print(f"Quadrature matrix shape: {QUAD.shape}")

# Define a test function to integrate
def test_function(x):
    """Test function: f(x) = x^2"""
    return x**2

# Transform quadrature nodes to [0,1] for integration
xi_01 = (qnodes + 1) / 2
f_values = test_function(xi_01)

print(f"Function values at quadrature points: {f_values}")

# Integrate using quadrature matrix
# For each basis function: ∫₀¹ eᵢ(x) f(x) dx ≈ Σⱼ eᵢ(ξⱼ) f(ξⱼ) wⱼ
integrated_coeffs = QUAD @ f_values

print(f"Integrated coefficients: {integrated_coeffs}")

# Analytical result for ∫₀¹ f(x) dx where f(x) = x²
analytical = 1/3
numerical = np.sum(integrated_coeffs)  # Sum of coefficients gives total integral
print(f"Analytical integral: {analytical:.6f}")
print(f"Numerical integral: {numerical:.6f}")
print(f"Error: {abs(numerical - analytical):.6e}")
\end{lstlisting}

\subsubsection{Orthonormal vs Lagrange Basis Comparison}

\begin{lstlisting}[language=Python, caption=Basis Comparison Example]
from ooc1d.utils.elementary_matrices import ElementaryMatrices
import numpy as np

# Create both types of elementary matrices
lagrange_matrices = ElementaryMatrices(orthonormal_basis=False)
ortho_matrices = ElementaryMatrices(orthonormal_basis=True)

print("Basis function comparison:")
print("="*50)

# Compare mass matrices
M_lagrange = lagrange_matrices.get_matrix('M')
M_ortho = ortho_matrices.get_matrix('M')

print("Lagrange mass matrix:")
print(M_lagrange)
print("\nOrthonormal mass matrix:")
print(M_ortho)

# Compare trace matrices
T_lagrange = lagrange_matrices.get_matrix('T')
T_ortho = ortho_matrices.get_matrix('T')

print("\nLagrange trace matrix:")
print(T_lagrange)
print("\nOrthonormal trace matrix:")
print(T_ortho)

# Test orthonormality
print("\nOrthonormality test for orthonormal basis:")
if np.allclose(M_ortho, np.eye(2), atol=1e-10):
    print("✓ Orthonormal mass matrix is identity")
else:
    print("✗ Orthonormal mass matrix is not identity")
    print(f"Difference from identity:\n{M_ortho - np.eye(2)}")

# Run tests for both
print("\nLagrange basis tests:")
lagrange_matrices.print_tests()

print("\nOrthonormal basis tests:")
ortho_matrices.print_tests()
\end{lstlisting}

\subsubsection{Static Condensation Integration}

\begin{lstlisting}[language=Python, caption=Static Condensation Integration Example]
# This example shows how ElementaryMatrices integrates with static condensation
from ooc1d.utils.elementary_matrices import ElementaryMatrices

def create_static_condensation_matrices(h, tau_values):
    """
    Create matrices for static condensation using elementary matrices.
    
    Args:
        h: Element length
        tau_values: List of stabilization parameters [tau_u, tau_phi, ...]
    
    Returns:
        Dictionary of static condensation matrices
    """
    # Get elementary matrices
    elem_matrices = ElementaryMatrices()
    
    # Basic matrices
    M = elem_matrices.get_matrix('M')
    T = elem_matrices.get_matrix('T')
    D = elem_matrices.get_matrix('D')
    Ntil = elem_matrices.get_matrix('Ntil')
    Mb = elem_matrices.get_matrix('Mb')
    
    # Scale by element length
    M_scaled = h * M
    Mb_scaled = h * Mb
    # D and T are scale-invariant for this element type
    
    # Build static condensation matrices
    sc_matrices = {}
    
    # Mass matrix with stabilization
    for i, tau in enumerate(tau_values):
        sc_matrices[f'A{i+1}'] = M_scaled + tau * Mb_scaled
        sc_matrices[f'L{i+1}'] = np.linalg.inv(sc_matrices[f'A{i+1}'])
        sc_matrices[f'B{i+1}'] = sc_matrices[f'L{i+1}'] @ T
    
    # Store basic matrices
    sc_matrices['M'] = M_scaled
    sc_matrices['T'] = T
    sc_matrices['D'] = D
    sc_matrices['Ntil'] = Ntil
    
    return sc_matrices

# Usage example
h = 0.1  # Element length
tau_values = [1.0, 1.0]  # Stabilization parameters for 2-equation system

sc_matrices = create_static_condensation_matrices(h, tau_values)

print("Static condensation matrices created:")
for name, matrix in sc_matrices.items():
    print(f"  {name}: {matrix.shape}")

# Example: compute bulk solution from trace values
trace_values = np.array([1.0, 0.5])  # Trace values at element boundaries
bulk_coeffs = sc_matrices['B1'] @ trace_values

print(f"\nTrace values: {trace_values}")
print(f"Bulk coefficients: {bulk_coeffs}")
\end{lstlisting}

\subsection{Method Summary Table}
\label{subsec:elementary_matrices_method_summary}

\begin{longtable}{|p{4.5cm}|p{2cm}|p{7cm}|}
\hline
\textbf{Method} & \textbf{Returns} & \textbf{Purpose} \\
\hline
\endhead

\texttt{\_\_init\_\_} & \texttt{None} & Initialize and compute all elementary matrices \\
\hline

\texttt{\_build\_matrices} & \texttt{None} & Build matrices using symbolic computation \\
\hline

\texttt{\_build\_basic\_matrices} & \texttt{None} & Compute mass, trace, derivative, and other basic matrices \\
\hline

\texttt{\_build\_quadrature} & \texttt{None} & Setup Legendre-Gauss-Lobatto quadrature \\
\hline

\texttt{\_test\_matrices} & \texttt{None} & Perform mathematical validation tests \\
\hline

\texttt{get\_matrix} & \texttt{np.ndarray} & Retrieve specific elementary matrix by name \\
\hline

\texttt{get\_all\_matrices} & \texttt{Dict} & Get all computed matrices as dictionary \\
\hline

\texttt{print\_tests} & \texttt{None} & Print validation test results \\
\hline

\end{longtable}

\subsection{Key Features and Mathematical Properties}

\begin{itemize}
    \item \textbf{MATLAB Compatibility}: Direct Python equivalent of MATLAB \texttt{build\_eMatrices.m}
    \item \textbf{Symbolic Computation}: Uses SymPy for exact mathematical operations
    \item \textbf{Multiple Basis Support}: Both Lagrange and orthonormal basis functions
    \item \textbf{Automatic Validation}: Built-in mathematical consistency tests
    \item \textbf{Quadrature Integration}: 4-point Legendre-Gauss-Lobatto rule
    \item \textbf{HDG Method Support}: All matrices needed for HDG static condensation
    \item \textbf{Element Scaling}: Proper scaling factors for physical element lengths
    \item \textbf{Reference Element}: All computations on standard interval [0,1]
\end{itemize}

This documentation provides an exact reference for the elementary matrices module, emphasizing its role as the foundation for HDG method computations and its seamless integration with other BioNetFlux components.

% End of elementary matrices module API documentation


\section{Elementary Matrices Module API Reference}
\label{sec:elementary_matrices_module_api}

This section provides reference  based on detailed analysis of the actual implementation for the  module \texttt{ooc1d.utils.elementary\_matrices.ElementaryMatrices}. The module computes elementary matrices for HDG methods on the reference element, serving as the Python equivalent of MATLAB \texttt{build\_eMatrices.m}.

\subsection{Module Overview}

The elementary matrices module provides:
\begin{itemize}
    \item Elementary matrix computation for HDG methods on reference element [0,1]
    \item Support for both Lagrange and orthonormal basis functions
    \item Symbolic computation using SymPy for exact mathematical operations
    \item Quadrature integration using Legendre-Gauss-Lobatto nodes
    \item Comprehensive matrix validation and testing
    \item MATLAB compatibility for existing HDG implementations
\end{itemize}

\subsection{Module Imports and Dependencies}

\begin{lstlisting}[language=Python, caption=Module Dependencies]
import numpy as np
import sympy as sp
from typing import Dict, Any
\end{lstlisting}

\subsection{ElementaryMatrices Class}
\label{subsec:elementary_matrices_class}

Main class for computing and managing elementary matrices on the reference element.

\subsubsection{Constructor}

\paragraph{\_\_init\_\_()}\leavevmode
\begin{lstlisting}[language=Python, caption=ElementaryMatrices Constructor]
def __init__(self, orthonormal_basis: bool = False)
\end{lstlisting}

\textbf{Parameters:}
\begin{itemize}
    \item \texttt{orthonormal\_basis}: Use orthonormal basis if True, Lagrange basis if False (default: False)
\end{itemize}

\textbf{Side Effects:}
\begin{itemize}
    \item Initializes all elementary matrices
    \item Stores matrices in internal dictionary
    \item Performs matrix validation tests
    \item Sets up quadrature rules
\end{itemize}

\textbf{Usage:}
\begin{lstlisting}[language=Python, caption=Constructor Usage Examples]
# Standard Lagrange basis (default)
elem_matrices = ElementaryMatrices()

# Orthonormal basis
elem_matrices_ortho = ElementaryMatrices(orthonormal_basis=True)

# Access matrices immediately after construction
mass_matrix = elem_matrices.get_matrix('M')
trace_matrix = elem_matrices.get_matrix('T')
\end{lstlisting}

\subsubsection{Core Attributes}

\begin{longtable}{|p{5.5cm}|p{2.5cm}|p{6cm}|}
\hline
\textbf{Attribute} & \textbf{Type} & \textbf{Description} \\
\hline
\endhead

\texttt{orthonormal\_basis} & \texttt{bool} & Flag indicating basis type used \\
\hline

\texttt{matrices} & \texttt{Dict[str, np.ndarray]} & Dictionary of all computed elementary matrices \\
\hline

\texttt{base} & \texttt{sp.Matrix} & SymPy symbolic basis functions \\
\hline

\texttt{hbase} & \texttt{sp.Matrix} & SymPy symbolic hat basis functions \\
\hline

\texttt{normali} & \texttt{sp.Matrix} & SymPy normal vectors at boundaries \\
\hline

\texttt{test\_integration\_by\_parts} & \texttt{np.ndarray} & Test matrix: $D + D^T - \tilde{N}$ (should be zero) \\
\hline

\texttt{test\_dxx\_zero} & \texttt{np.ndarray} & Test matrix for $\frac{d^2}{dx^2} = 0$ for linear functions \\
\hline

\texttt{test\_mb\_gb} & \texttt{np.ndarray} & Test matrix: $M^{\partial} - G^b T$ (should be zero) \\
\hline

\end{longtable}

\subsubsection{Matrix Construction Methods}

\paragraph{\_build\_matrices()}\leavevmode
\begin{lstlisting}[language=Python, caption=Build Matrices Method]
def _build_matrices(self)
\end{lstlisting}

\textbf{Purpose:} Build all elementary matrices on reference element [0,1]

\textbf{Process:}
\begin{enumerate}
    \item Define symbolic variable and reference interval
    \item Construct basis functions (Lagrange or orthonormal)
    \item Build basic matrices using symbolic integration
    \item Perform validation tests
    \item Build quadrature matrices
\end{enumerate}

\textbf{Basis Functions:}

\textbf{Lagrange Basis (default):}
\begin{align}
e_0(y) &= 1 - y \\
e_1(y) &= y
\end{align}

\textbf{Orthonormal Basis:} Constructed by solving orthogonality conditions:
\begin{align}
e_0 &= 1 \\
e_1 &= c \cdot y + d
\end{align}
Subject to: $\int_0^1 e_0 e_1 \, dy = 0$ and $\int_0^1 e_1^2 \, dy = 1$

\paragraph{\_build\_basic\_matrices()}\leavevmode
\begin{lstlisting}[language=Python, caption=Build Basic Matrices Method]
def _build_basic_matrices(self, y, base, hbase, normali, x0, x1)
\end{lstlisting}

\textbf{Parameters:}
\begin{itemize}
    \item \texttt{y}: SymPy symbolic variable
    \item \texttt{base}: Basis functions matrix
    \item \texttt{hbase}: Hat basis functions matrix
    \item \texttt{normali}: Normal vectors matrix
    \item \texttt{x0, x1}: Reference interval endpoints (0, 1)
\end{itemize}

\textbf{Matrices Computed:}

\textbf{Mass Matrix:} $M_{ij} = \int_0^1 e_j e_i \, dy$
\begin{lstlisting}[language=Python, caption=Mass Matrix Computation]
M = sp.integrate(base * base.T, (y, 0, 1))
M_numeric = np.array(M).astype(float)
# For Lagrange basis: M = [[1/3, 1/6], [1/6, 1/3]]
\end{lstlisting}

\textbf{Trace Matrix:} $T_{ij} = e_j(x_i)$ (transpose of Gramian)
\begin{lstlisting}[language=Python, caption=Trace Matrix Computation]
Gb = base.subs(y, x0) * hbase.T.subs(y, x0) + base.subs(y, x1) * hbase.T.subs(y, x1)
Trace = Gb.T
# For Lagrange basis: T = [[1, 0], [0, 1]]
\end{lstlisting}

\textbf{Derivative Matrix:} $D_{ij} = \int_0^1 e_j \frac{\partial e_i}{\partial x} \, dy$
\begin{lstlisting}[language=Python, caption=Derivative Matrix Computation]
D = sp.integrate(sp.diff(base, y) * base.T, (y, x0, x1))
# For Lagrange basis: D = [[1, -1], [1, -1]]
\end{lstlisting}

\paragraph{\_build\_quadrature()}\leavevmode
\begin{lstlisting}[language=Python, caption=Build Quadrature Method]
def _build_quadrature(self)
\end{lstlisting}

\textbf{Purpose:} Build quadrature matrices using Legendre-Gauss-Lobatto nodes

\textbf{Quadrature Rule:} 4-point Legendre-Gauss-Lobatto on [-1,1], transformed to [0,1]
\begin{lstlisting}[language=Python, caption=Quadrature Setup]
# LGL nodes on [-1,1]
qnodes = np.array([-1.0, -0.4472136, 0.4472136, 1.0])
qweights = np.array([1/6, 5/6, 5/6, 1/6])

# Transform to [0,1]
xi_01 = (qnodes + 1) / 2

# Build quadrature matrix: QUAD[i,j] = e_i(ξ_j) * w_j / 2
\end{lstlisting}

\subsubsection{Matrix Access Methods}

\paragraph{get\_matrix()}\leavevmode
\begin{lstlisting}[language=Python, caption=Get Matrix Method]
def get_matrix(self, name: str) -> np.ndarray
\end{lstlisting}

\textbf{Parameters:}
\begin{itemize}
    \item \texttt{name}: Matrix name (see available matrices table below)
\end{itemize}

\textbf{Returns:} \texttt{np.ndarray} - Requested elementary matrix

\textbf{Raises:} \texttt{KeyError} if matrix name not found

\textbf{Usage:}
\begin{lstlisting}[language=Python, caption=Get Matrix Usage]
elem_matrices = ElementaryMatrices()

# Get individual matrices
M = elem_matrices.get_matrix('M')        # Mass matrix
T = elem_matrices.get_matrix('T')        # Trace matrix  
D = elem_matrices.get_matrix('D')        # Derivative matrix
QUAD = elem_matrices.get_matrix('QUAD')  # Quadrature matrix

# Error handling
try:
    invalid = elem_matrices.get_matrix('INVALID')
except KeyError as e:
    print(f"Matrix not found: {e}")
\end{lstlisting}

\paragraph{get\_all\_matrices()}\leavevmode
\begin{lstlisting}[language=Python, caption=Get All Matrices Method]
def get_all_matrices(self) -> Dict[str, np.ndarray]
\end{lstlisting}

\textbf{Returns:} \texttt{Dict[str, np.ndarray]} - Copy of all elementary matrices

\textbf{Usage:}
\begin{lstlisting}[language=Python, caption=Get All Matrices Usage]
all_matrices = elem_matrices.get_all_matrices()

print("Available matrices:")
for name, matrix in all_matrices.items():
    print(f"  {name}: {matrix.shape}")

# Safe to modify returned dictionary
all_matrices['M'] *= 2.0  # Doesn't affect internal matrices
\end{lstlisting}

\subsubsection{Available Elementary Matrices}

\begin{longtable}{|p{2cm}|p{2.5cm}|p{8.5cm}|}
\hline
\textbf{Name} & \textbf{Shape} & \textbf{Description and Formula} \\
\hline
\endhead

\texttt{Z} & $(2, 2)$ & Zero matrix \\
\hline

\texttt{M} & $(2, 2)$ & Mass matrix: $M_{ij} = \int_0^1 e_j e_i \, dy$ \\
\hline

\texttt{IM} & $(2, 2)$ & Inverse mass matrix: $M^{-1}$ \\
\hline

\texttt{Gb} & $(2, 2)$ & Gramian matrix: $G^b_{ij} = e_i(0)\hat{e}_j(0) + e_i(1)\hat{e}_j(1)$ \\
\hline

\texttt{Mb} & $(2, 2)$ & Boundary mass matrix: $M^{\partial}_{ij} = e_i(0)e_j(0) + e_i(1)e_j(1)$ \\
\hline

\texttt{T} & $(2, 2)$ & Trace matrix: $T_{ij} = e_j(x_i)$ (transpose of Gramian) \\
\hline

\texttt{Av} & $(1, 2)$ & Averaging matrix: $(T^{-1} \mathbf{1})^T M$ where $\mathbf{1} = [1, 1]^T$ \\
\hline

\texttt{Ntil} & $(2, 2)$ & Normal matrix: $\tilde{N}_{ij} = e_j(1)e_i(1)n_1 + e_j(0)e_i(0)n_0$ \\
\hline

\texttt{Nhat} & $(2, 2)$ & Hat normal matrix: boundary basis times normal matrix \\
\hline

\texttt{D} & $(2, 2)$ & Derivative matrix: $D_{ij} = \int_0^1 e_j \frac{\partial e_i}{\partial x} \, dy$ \\
\hline

\texttt{qnodes} & $(4,)$ & Quadrature nodes on [-1,1]: $[-1, -0.4472136, 0.4472136, 1]$ \\
\hline

\texttt{QUAD} & $(2, 4)$ & Quadrature matrix: $QUAD_{ij} = e_i(\xi_j) w_j / 2$ \\
\hline

\end{longtable}

\subsubsection{Matrix Values for Lagrange Basis}

For the standard Lagrange basis $e_0(y) = 1-y$, $e_1(y) = y$:

\begin{lstlisting}[language=Python, caption=Lagrange Basis Matrix Values]
# Mass matrix
M = [[1/3, 1/6],
     [1/6, 1/3]]

# Trace matrix  
T = [[1, 0],
     [0, 1]]

# Derivative matrix
D = [[-1, -1],
     [ 1,  1]]

# Normal matrix (with n0=-1, n1=1)
Ntil = [[ 1,  0],
        [ 0, -1]]

# Averaging matrix
Av = [[1/2, 1/2]]

# Quadrature matrix (for 4 LGL points)
QUAD = [[0.91666667, 0.62360680, 0.37639320, 0.08333333],
        [0.08333333, 0.37639320, 0.62360680, 0.91666667]]
\end{lstlisting}

\subsubsection{Validation and Testing}

\paragraph{\_test\_matrices()}\leavevmode
\begin{lstlisting}[language=Python, caption=Test Matrices Method]
def _test_matrices(self, D, Ntil, M, Trace, Nhat, Gb)
\end{lstlisting}

\textbf{Purpose:} Perform mathematical validation tests (same as MATLAB code)

\textbf{Tests Performed:}

\textbf{Test 1: Integration by Parts}
\begin{align}
D + D^T - \tilde{N} = 0
\end{align}
This verifies the integration by parts identity for the derivative matrix.

\textbf{Test 2: Second Derivative Zero}
\begin{align}
(\tilde{N} - D) M^{-1} (D - \hat{N} T) = 0
\end{align}
This verifies that $\frac{d^2}{dx^2} = 0$ for linear functions.

\textbf{Test 3: Boundary Mass Consistency}
\begin{align}
M^{\partial} - G^b T = 0
\end{align}

\paragraph{print\_tests()}\leavevmode
\begin{lstlisting}[language=Python, caption=Print Tests Method]
def print_tests(self)
\end{lstlisting}

\textbf{Purpose:} Print validation test results

\textbf{Usage:}
\begin{lstlisting}[language=Python, caption=Print Tests Usage]
elem_matrices = ElementaryMatrices()
elem_matrices.print_tests()

# Sample output:
# Test: Integration by parts (should be zero):
# [[ 0.00000000e+00  0.00000000e+00]
#  [ 0.00000000e+00  0.00000000e+00]]
# 
# Test: d²x/dx² = 0 (should be zero):
# [[ 0.00000000e+00  0.00000000e+00]
#  [ 0.00000000e+00  0.00000000e+00]]
#
# Test: Mb - Gb*T (should be zero):
# [[ 0.00000000e+00  0.00000000e+00]
#  [ 0.00000000e+00  0.00000000e+00]]
\end{lstlisting}

\subsection{Complete Usage Examples}
\label{subsec:elementary_matrices_complete_examples}

\subsubsection{Basic Usage with HDG Method}

\begin{lstlisting}[language=Python, caption=Basic HDG Usage Example]
from ooc1d.utils.elementary_matrices import ElementaryMatrices
import numpy as np

# Create elementary matrices
elem_matrices = ElementaryMatrices()

# Validate matrices
elem_matrices.print_tests()

# Get matrices for HDG implementation
M = elem_matrices.get_matrix('M')        # Mass matrix
T = elem_matrices.get_matrix('T')        # Trace matrix
D = elem_matrices.get_matrix('D')        # Derivative matrix
QUAD = elem_matrices.get_matrix('QUAD')  # Quadrature matrix
qnodes = elem_matrices.get_matrix('qnodes')  # Quadrature nodes

print("Elementary matrices for HDG:")
print(f"Mass matrix M:\n{M}")
print(f"Trace matrix T:\n{T}")
print(f"Derivative matrix D:\n{D}")
print(f"Quadrature nodes: {qnodes}")
print(f"Quadrature matrix shape: {QUAD.shape}")

# Element length scaling (for physical elements)
h = 0.1  # Element length
M_scaled = h * M           # Scaled mass matrix
D_scaled = D               # Derivative matrix unchanged
QUAD_scaled = h * QUAD     # Scaled quadrature matrix

print(f"\nScaled matrices (h={h}):")
print(f"Scaled mass matrix:\n{M_scaled}")
print(f"Scaled quadrature matrix shape: {QUAD_scaled.shape}")
\end{lstlisting}

\subsubsection{Integration with BulkData}

\begin{lstlisting}[language=Python, caption=BulkData Integration Example]
from ooc1d.core.bulk_data import BulkData
from ooc1d.core.problem import Problem
from ooc1d.core.discretization import Discretization
from ooc1d.utils.elementary_matrices import ElementaryMatrices

# Setup problem and discretization
problem = Problem(neq=2, domain_start=0.0, domain_length=1.0)
discretization = Discretization(n_elements=10)

# Create elementary matrices
elem_matrices = ElementaryMatrices()

# Create BulkData with elementary matrices
bulk_data = BulkData(problem, discretization, dual=False)

# The BulkData constructor automatically uses ElementaryMatrices
print("BulkData created with elementary matrices:")
print(f"  Trace matrix from BulkData: {bulk_data.trace_matrix}")
print(f"  Mass matrix from BulkData: {bulk_data.mass_matrix}")

# Verify matrices match
T_elem = elem_matrices.get_matrix('T')
M_elem = elem_matrices.get_matrix('M')

print("Matrix consistency check:")
print(f"  Trace matrices match: {np.allclose(bulk_data.trace_matrix, T_elem)}")
print(f"  Mass matrices match: {np.allclose(bulk_data.mass_matrix / discretization.element_length, M_elem)}")
\end{lstlisting}

\subsubsection{Quadrature Integration Example}

\begin{lstlisting}[language=Python, caption=Quadrature Integration Example]
from ooc1d.utils.elementary_matrices import ElementaryMatrices
import numpy as np

# Create elementary matrices
elem_matrices = ElementaryMatrices()

# Get quadrature data
QUAD = elem_matrices.get_matrix('QUAD')
qnodes = elem_matrices.get_matrix('qnodes')

print("Quadrature integration example:")
print(f"Quadrature nodes on [-1,1]: {qnodes}")
print(f"Quadrature matrix shape: {QUAD.shape}")

# Define a test function to integrate
def test_function(x):
    """Test function: f(x) = x^2"""
    return x**2

# Transform quadrature nodes to [0,1] for integration
xi_01 = (qnodes + 1) / 2
f_values = test_function(xi_01)

print(f"Function values at quadrature points: {f_values}")

# Integrate using quadrature matrix
# For each basis function: ∫₀¹ eᵢ(x) f(x) dx ≈ Σⱼ eᵢ(ξⱼ) f(ξⱼ) wⱼ
integrated_coeffs = QUAD @ f_values

print(f"Integrated coefficients: {integrated_coeffs}")

# Analytical result for ∫₀¹ f(x) dx where f(x) = x²
analytical = 1/3
numerical = np.sum(integrated_coeffs)  # Sum of coefficients gives total integral
print(f"Analytical integral: {analytical:.6f}")
print(f"Numerical integral: {numerical:.6f}")
print(f"Error: {abs(numerical - analytical):.6e}")
\end{lstlisting}

\subsubsection{Orthonormal vs Lagrange Basis Comparison}

\begin{lstlisting}[language=Python, caption=Basis Comparison Example]
from ooc1d.utils.elementary_matrices import ElementaryMatrices
import numpy as np

# Create both types of elementary matrices
lagrange_matrices = ElementaryMatrices(orthonormal_basis=False)
ortho_matrices = ElementaryMatrices(orthonormal_basis=True)

print("Basis function comparison:")
print("="*50)

# Compare mass matrices
M_lagrange = lagrange_matrices.get_matrix('M')
M_ortho = ortho_matrices.get_matrix('M')

print("Lagrange mass matrix:")
print(M_lagrange)
print("\nOrthonormal mass matrix:")
print(M_ortho)

# Compare trace matrices
T_lagrange = lagrange_matrices.get_matrix('T')
T_ortho = ortho_matrices.get_matrix('T')

print("\nLagrange trace matrix:")
print(T_lagrange)
print("\nOrthonormal trace matrix:")
print(T_ortho)

# Test orthonormality
print("\nOrthonormality test for orthonormal basis:")
if np.allclose(M_ortho, np.eye(2), atol=1e-10):
    print("✓ Orthonormal mass matrix is identity")
else:
    print("✗ Orthonormal mass matrix is not identity")
    print(f"Difference from identity:\n{M_ortho - np.eye(2)}")

# Run tests for both
print("\nLagrange basis tests:")
lagrange_matrices.print_tests()

print("\nOrthonormal basis tests:")
ortho_matrices.print_tests()
\end{lstlisting}

\subsubsection{Static Condensation Integration}

\begin{lstlisting}[language=Python, caption=Static Condensation Integration Example]
# This example shows how ElementaryMatrices integrates with static condensation
from ooc1d.utils.elementary_matrices import ElementaryMatrices

def create_static_condensation_matrices(h, tau_values):
    """
    Create matrices for static condensation using elementary matrices.
    
    Args:
        h: Element length
        tau_values: List of stabilization parameters [tau_u, tau_phi, ...]
    
    Returns:
        Dictionary of static condensation matrices
    """
    # Get elementary matrices
    elem_matrices = ElementaryMatrices()
    
    # Basic matrices
    M = elem_matrices.get_matrix('M')
    T = elem_matrices.get_matrix('T')
    D = elem_matrices.get_matrix('D')
    Ntil = elem_matrices.get_matrix('Ntil')
    Mb = elem_matrices.get_matrix('Mb')
    
    # Scale by element length
    M_scaled = h * M
    Mb_scaled = h * Mb
    # D and T are scale-invariant for this element type
    
    # Build static condensation matrices
    sc_matrices = {}
    
    # Mass matrix with stabilization
    for i, tau in enumerate(tau_values):
        sc_matrices[f'A{i+1}'] = M_scaled + tau * Mb_scaled
        sc_matrices[f'L{i+1}'] = np.linalg.inv(sc_matrices[f'A{i+1}'])
        sc_matrices[f'B{i+1}'] = sc_matrices[f'L{i+1}'] @ T
    
    # Store basic matrices
    sc_matrices['M'] = M_scaled
    sc_matrices['T'] = T
    sc_matrices['D'] = D
    sc_matrices['Ntil'] = Ntil
    
    return sc_matrices

# Usage example
h = 0.1  # Element length
tau_values = [1.0, 1.0]  # Stabilization parameters for 2-equation system

sc_matrices = create_static_condensation_matrices(h, tau_values)

print("Static condensation matrices created:")
for name, matrix in sc_matrices.items():
    print(f"  {name}: {matrix.shape}")

# Example: compute bulk solution from trace values
trace_values = np.array([1.0, 0.5])  # Trace values at element boundaries
bulk_coeffs = sc_matrices['B1'] @ trace_values

print(f"\nTrace values: {trace_values}")
print(f"Bulk coefficients: {bulk_coeffs}")
\end{lstlisting}

\subsection{Method Summary Table}
\label{subsec:elementary_matrices_method_summary}

\begin{longtable}{|p{4.5cm}|p{2cm}|p{7cm}|}
\hline
\textbf{Method} & \textbf{Returns} & \textbf{Purpose} \\
\hline
\endhead

\texttt{\_\_init\_\_} & \texttt{None} & Initialize and compute all elementary matrices \\
\hline

\texttt{\_build\_matrices} & \texttt{None} & Build matrices using symbolic computation \\
\hline

\texttt{\_build\_basic\_matrices} & \texttt{None} & Compute mass, trace, derivative, and other basic matrices \\
\hline

\texttt{\_build\_quadrature} & \texttt{None} & Setup Legendre-Gauss-Lobatto quadrature \\
\hline

\texttt{\_test\_matrices} & \texttt{None} & Perform mathematical validation tests \\
\hline

\texttt{get\_matrix} & \texttt{np.ndarray} & Retrieve specific elementary matrix by name \\
\hline

\texttt{get\_all\_matrices} & \texttt{Dict} & Get all computed matrices as dictionary \\
\hline

\texttt{print\_tests} & \texttt{None} & Print validation test results \\
\hline

\end{longtable}

\subsection{Key Features and Mathematical Properties}

\begin{itemize}
    \item \textbf{MATLAB Compatibility}: Direct Python equivalent of MATLAB \texttt{build\_eMatrices.m}
    \item \textbf{Symbolic Computation}: Uses SymPy for exact mathematical operations
    \item \textbf{Multiple Basis Support}: Both Lagrange and orthonormal basis functions
    \item \textbf{Automatic Validation}: Built-in mathematical consistency tests
    \item \textbf{Quadrature Integration}: 4-point Legendre-Gauss-Lobatto rule
    \item \textbf{HDG Method Support}: All matrices needed for HDG static condensation
    \item \textbf{Element Scaling}: Proper scaling factors for physical element lengths
    \item \textbf{Reference Element}: All computations on standard interval [0,1]
\end{itemize}

This documentation provides an exact reference for the elementary matrices module, emphasizing its role as the foundation for HDG method computations and its seamless integration with other BioNetFlux components.

% End of elementary matrices module API documentation
