% Lean Global Assembly Module API Documentation (Accurate Analysis)
% To be included in master LaTeX document
%
% Usage: % Lean Global Assembly Module API Documentation (Accurate Analysis)
% To be included in master LaTeX document
%
% Usage: % Lean Global Assembly Module API Documentation (Accurate Analysis)
% To be included in master LaTeX document
%
% Usage: % Lean Global Assembly Module API Documentation (Accurate Analysis)
% To be included in master LaTeX document
%
% Usage: \input{docs/lean_global_assembly_api}

\section{Lean Global Assembly Module API Reference}
\label{sec:lean_global_assembly_api}

This section provides an exact reference for the lean global assembly module (\texttt{ooc1d.core.lean\_global\_assembly.GlobalAssembler}) based on detailed analysis of the actual implementation. This class provides global assembly functionality using the lean BulkDataManager approach where framework objects are passed as parameters.

\subsection{Module Overview}

The lean global assembly module provides:
\begin{itemize}
    \item Global residual and Jacobian assembly from domain flux jumps
    \item Constraint integration for boundary and junction conditions
    \item Memory-efficient operation using lean BulkDataManager
    \item DOF management for multi-domain systems
    \item Comprehensive testing infrastructure
\end{itemize}

\subsection{Module Imports and Dependencies}

\begin{lstlisting}[language=Python, caption=Module Dependencies]
import numpy as np
from typing import List, Tuple, Optional

from .discretization import GlobalDiscretization
from .flux_jump import domain_flux_jump
from .constraints import ConstraintManager
from .lean_bulk_data_manager import BulkDataManager
from .bulk_data import BulkData
\end{lstlisting}

\subsection{GlobalAssembler Class}
\label{subsec:lean_global_assembler_class}

Main class for lean global assembly that uses external framework objects as parameters.

\subsubsection{Constructor}

\paragraph{\_\_init\_\_()}\leavevmode
\begin{lstlisting}[language=Python, caption=GlobalAssembler Constructor]
def __init__(self, 
             domain_data_list: List,
             constraint_manager: Optional[ConstraintManager] = None)
\end{lstlisting}

\textbf{Parameters:}
\begin{itemize}
    \item \texttt{domain\_data\_list}: List of DomainData objects with essential information
    \item \texttt{constraint\_manager}: Optional constraint manager for boundary/junction conditions
\end{itemize}

\textbf{Side Effects:}
\begin{itemize}
    \item Creates internal BulkDataManager with domain data
    \item Computes DOF structure and domain offsets
    \item Stores constraint manager reference
\end{itemize}

\textbf{Usage:}
\begin{lstlisting}[language=Python, caption=Constructor Usage]
# Extract domain data first
domain_data_list = BulkDataManager.extract_domain_data_list(
    problems, discretizations, static_condensations
)

# Create constraint manager
constraint_manager = ConstraintManager()
constraint_manager.add_dirichlet(0, 0, 0.0)
constraint_manager.map_to_discretizations(discretizations)

# Create lean assembler
assembler = GlobalAssembler(domain_data_list, constraint_manager)
\end{lstlisting}

\subsubsection{Core Attributes}

\begin{longtable}{|p{3.5cm}|p{2.5cm}|p{7cm}|}
\hline
\textbf{Attribute} & \textbf{Type} & \textbf{Description} \\
\hline
\endhead

\texttt{bulk\_manager} & \texttt{BulkDataManager} & Lean bulk data manager instance \\
\hline

\texttt{constraint\_manager} & \texttt{Optional[ConstraintManager]} & Constraint manager for boundary/junction conditions \\
\hline

\texttt{n\_domains} & \texttt{int} & Number of domains in the system \\
\hline

\texttt{domain\_trace\_sizes} & \texttt{List[int]} & Trace DOF count for each domain \\
\hline

\texttt{domain\_trace\_offsets} & \texttt{List[int]} & Starting indices for domain traces in global vector \\
\hline

\texttt{total\_trace\_dofs} & \texttt{int} & Total number of trace DOFs across all domains \\
\hline

\texttt{n\_multipliers} & \texttt{int} & Number of constraint multipliers \\
\hline

\texttt{total\_dofs} & \texttt{int} & Total DOFs: \texttt{total\_trace\_dofs + n\_multipliers} \\
\hline

\end{longtable}

\subsubsection{Factory Method}

\paragraph{from\_framework\_objects()}\leavevmode
\begin{lstlisting}[language=Python, caption=Factory Method]
@classmethod
def from_framework_objects(cls,
                          problems: List,
                          global_discretization,
                          static_condensations: List,
                          constraint_manager: Optional[ConstraintManager] = None)
\end{lstlisting}

\textbf{Parameters:}
\begin{itemize}
    \item \texttt{problems}: List of Problem instances for each domain
    \item \texttt{global\_discretization}: GlobalDiscretization instance
    \item \texttt{static\_condensations}: List of static condensation implementations
    \item \texttt{constraint\_manager}: Optional constraint manager
\end{itemize}

\textbf{Returns:} \texttt{GlobalAssembler} - Configured assembler instance

\textbf{Purpose:} Factory method to create assembler from framework objects

\textbf{Usage:}
\begin{lstlisting}[language=Python, caption=Factory Method Usage]
# Create assembler from framework objects
assembler = GlobalAssembler.from_framework_objects(
    problems=problems,
    global_discretization=global_disc,
    static_condensations=static_condensations,
    constraint_manager=constraint_manager
)

print(f"Created assembler: {assembler}")
\end{lstlisting}

\subsubsection{Internal Initialization Methods}

\paragraph{\_compute\_dof\_structure()}\leavevmode
\begin{lstlisting}[language=Python, caption=Compute DOF Structure Method]
def _compute_dof_structure(self)
\end{lstlisting}

\textbf{Purpose:} Compute global DOF structure from domain data

\textbf{Side Effects:} Sets all DOF-related attributes

\textbf{Algorithm:}
\begin{enumerate}
    \item For each domain: compute \texttt{trace\_size = neq * (n\_elements + 1)}
    \item Compute cumulative offsets for global vector assembly
    \item Calculate total trace DOFs and multipliers
    \item Set total DOF count
\end{enumerate}

\subsection{Primary Assembly Methods}
\label{subsec:primary_assembly_methods}

\paragraph{assemble\_residual\_and\_jacobian()}\leavevmode
\begin{lstlisting}[language=Python, caption=Assemble Residual and Jacobian Method]
def assemble_residual_and_jacobian(self, 
                                 global_solution: np.ndarray,
                                 forcing_terms: List[np.ndarray],
                                 static_condensations: List,
                                 time: float) -> Tuple[np.ndarray, np.ndarray]
\end{lstlisting}

\textbf{Parameters:}
\begin{itemize}
    \item \texttt{global\_solution}: Global solution vector \texttt{[trace\_solutions; multipliers]}
    \item \texttt{forcing\_terms}: List of forcing term arrays for each domain (pre-computed)
    \item \texttt{static\_condensations}: List of static condensation objects for flux jump computation
    \item \texttt{time}: Current time for constraint evaluation
\end{itemize}

\textbf{Returns:} \texttt{Tuple[np.ndarray, np.ndarray]} - (residual, jacobian) global arrays

\textbf{Purpose:} Assemble global residual and Jacobian from domain flux jumps and constraints

\textbf{Mathematical Formulation:} Solves the nonlinear system $F(U; F_{ext}) = 0$ where:
\begin{itemize}
    \item $U$ is the trace solution
    \item $F_{ext}$ are the forcing terms (pre-computed as $dt \cdot f + M \cdot u_{old}$)
\end{itemize}

\textbf{Algorithm:}
\begin{enumerate}
    \item Extract trace solutions and multipliers from global solution
    \item Initialize global residual and Jacobian arrays
    \item For each domain: compute flux jump using \texttt{domain\_flux\_jump()}
    \item Add domain contributions to global arrays
    \item Add constraint contributions (multiplier coupling and constraint residuals)
    \item Add constraint Jacobian contributions
\end{enumerate}

\textbf{Usage:}
\begin{lstlisting}[language=Python, caption=Residual and Jacobian Assembly Usage]
# Prepare inputs
global_solution = np.random.rand(assembler.total_dofs)
forcing_terms = [np.random.rand(2*neq, n_elements) for _ in range(n_domains)]

# Assemble system
residual, jacobian = assembler.assemble_residual_and_jacobian(
    global_solution=global_solution,
    forcing_terms=forcing_terms,
    static_condensations=static_condensations,
    time=0.5
)

print(f"System assembled:")
print(f"  Residual shape: {residual.shape}")
print(f"  Jacobian shape: {jacobian.shape}")
print(f"  Residual norm: {np.linalg.norm(residual):.6e}")
print(f"  Jacobian condition: {np.linalg.cond(jacobian):.2e}")
\end{lstlisting}

\paragraph{bulk\_by\_static\_condensation()}
\leavevmode
\begin{lstlisting}[language=Python, caption=Bulk by Static Condensation Method]
def bulk_by_static_condensation(self, 
                               global_solution: np.ndarray,
                               forcing_terms: List[np.ndarray],
                               static_condensations: List,
                               time: float) -> Tuple[np.ndarray, np.ndarray]
\end{lstlisting}

\textbf{Parameters:} Same as \texttt{assemble\_residual\_and\_jacobian()}

\textbf{Returns:} \texttt{List[np.ndarray]} - List of bulk solution arrays for each domain

\textbf{Purpose:} Compute bulk solutions from trace solutions using static condensation

\textbf{Usage:}
\begin{lstlisting}[language=Python, caption=Bulk Solution Computation Usage]
# Compute bulk solutions only
bulk_solutions = assembler.bulk_by_static_condensation(
    global_solution=global_solution,
    forcing_terms=forcing_terms,
    static_condensations=static_condensations,
    time=0.5
)

for i, bulk_sol in enumerate(bulk_solutions):
    print(f"Domain {i} bulk solution shape: {bulk_sol.shape}")
\end{lstlisting}

\subsubsection{Forcing Term Computation}

\paragraph{compute\_forcing\_terms()}\leavevmode
\begin{lstlisting}[language=Python, caption=Compute Forcing Terms Method]
def compute_forcing_terms(self,
                        bulk_data_list: List[BulkData],
                        problems: List,
                        discretizations: List,
                        time: float,
                        dt: float) -> List[np.ndarray]
\end{lstlisting}

\textbf{Parameters:}
\begin{itemize}
    \item \texttt{bulk\_data\_list}: List of BulkData objects from previous time step
    \item \texttt{problems}: List of Problem objects
    \item \texttt{discretizations}: List of discretization objects
    \item \texttt{time}: Current time
    \item \texttt{dt}: Time step size
\end{itemize}

\textbf{Returns:} \texttt{List[np.ndarray]} - Forcing term arrays for implicit Euler

\textbf{Computation:} For each domain: $\text{forcing\_term} = dt \cdot f + M \cdot u_{old}$

\textbf{Usage:}
\begin{lstlisting}[language=Python, caption=Forcing Terms Usage]
# Previous time step solutions
bulk_data_list = assembler.initialize_bulk_data(problems, discretizations, time=0.0)

# Compute forcing terms for next time step
forcing_terms = assembler.compute_forcing_terms(
    bulk_data_list=bulk_data_list,
    problems=problems,
    discretizations=discretizations,
    time=0.1,
    dt=0.01
)

print(f"Computed {len(forcing_terms)} forcing term arrays")
for i, ft in enumerate(forcing_terms):
    print(f"  Domain {i}: {ft.shape}")
\end{lstlisting}

\subsection{Solution Vector Management}
\label{subsec:solution_vector_management}

\paragraph{\_extract\_trace\_solutions()}\leavevmode
\begin{lstlisting}[language=Python, caption=Extract Trace Solutions Method]
def _extract_trace_solutions(self, global_solution: np.ndarray) -> List[np.ndarray]
\end{lstlisting}

\textbf{Parameters:}
\begin{itemize}
    \item \texttt{global\_solution}: Global solution vector
\end{itemize}

\textbf{Returns:} \texttt{List[np.ndarray]} - Individual domain trace solutions

\textbf{Purpose:} Extract domain-specific trace solutions from global vector

\paragraph{get\_domain\_solutions()}\leavevmode
\begin{lstlisting}[language=Python, caption=Get Domain Solutions Method]
def get_domain_solutions(self, global_solution: np.ndarray) -> List[np.ndarray]
\end{lstlisting}

\textbf{Parameters:}
\begin{itemize}
    \item \texttt{global\_solution}: Global solution vector
\end{itemize}

\textbf{Returns:} \texttt{List[np.ndarray]} - Domain trace solutions (public interface)

\textbf{Usage:}
\begin{lstlisting}[language=Python, caption=Solution Extraction Usage]
# Extract individual domain solutions
domain_solutions = assembler.get_domain_solutions(global_solution)

for i, domain_sol in enumerate(domain_solutions):
    print(f"Domain {i} solution shape: {domain_sol.shape}")
    print(f"  Solution range: [{np.min(domain_sol):.6e}, {np.max(domain_sol):.6e}]")
\end{lstlisting}

\paragraph{get\_multipliers()}\leavevmode
\begin{lstlisting}[language=Python, caption=Get Multipliers Method]
def get_multipliers(self, global_solution: np.ndarray) -> np.ndarray
\end{lstlisting}

\textbf{Parameters:}
\begin{itemize}
    \item \texttt{global\_solution}: Global solution vector
\end{itemize}

\textbf{Returns:} \texttt{np.ndarray} - Constraint multipliers (empty if no constraints)

\textbf{Usage:}
\begin{lstlisting}[language=Python, caption=Multiplier Extraction Usage]
multipliers = assembler.get_multipliers(global_solution)
if len(multipliers) > 0:
    print(f"Constraint multipliers: {multipliers}")
else:
    print("No constraints defined")
\end{lstlisting}

\subsection{Initial Guess Creation}
\label{subsec:initial_guess_creation}

\paragraph{create\_initial\_guess\_from\_bulk\_data()}\leavevmode
\begin{lstlisting}[language=Python, caption=Initial Guess from BulkData Method]
def create_initial_guess_from_bulk_data(self, 
                                       bulk_data_list: List[BulkData]) -> np.ndarray
\end{lstlisting}

\textbf{Parameters:}
\begin{itemize}
    \item \texttt{bulk\_data\_list}: List of BulkData objects
\end{itemize}

\textbf{Returns:} \texttt{np.ndarray} - Initial guess for global solution vector

\textbf{Purpose:} Create initial guess from existing BulkData trace values

\textbf{Usage:}
\begin{lstlisting}[language=Python, caption=BulkData Initial Guess Usage]
# Initialize BulkData from problems
bulk_data_list = assembler.initialize_bulk_data(problems, discretizations, time=0.0)

# Create initial guess
initial_guess = assembler.create_initial_guess_from_bulk_data(bulk_data_list)
print(f"Initial guess shape: {initial_guess.shape}")
\end{lstlisting}

\paragraph{create\_initial\_guess\_from\_problems()}\leavevmode
\begin{lstlisting}[language=Python, caption=Initial Guess from Problems Method]
def create_initial_guess_from_problems(self, 
                                     problems: List,
                                     discretizations: List,
                                     time: float = 0.0) -> np.ndarray
\end{lstlisting}

\textbf{Parameters:}
\begin{itemize}
    \item \texttt{problems}: List of Problem objects
    \item \texttt{discretizations}: List of discretization objects
    \item \texttt{time}: Time for initial condition evaluation (default: 0.0)
\end{itemize}

\textbf{Returns:} \texttt{np.ndarray} - Initial guess for global solution vector

\textbf{Purpose:} Create initial guess directly from problem initial conditions

\textbf{Algorithm:}
\begin{enumerate}
    \item Initialize global solution vector with zeros
    \item For each domain and equation: evaluate initial condition at mesh nodes
    \item Fill corresponding entries in global vector
    \item Set multipliers to zero
\end{enumerate}

\textbf{Usage:}
\begin{lstlisting}[language=Python, caption=Problem Initial Guess Usage]
# Create initial guess directly from problem definitions
initial_guess = assembler.create_initial_guess_from_problems(
    problems=problems,
    discretizations=discretizations,
    time=0.0
)

print(f"Initial guess from problems: {initial_guess.shape}")
print(f"Non-zero entries: {np.count_nonzero(initial_guess)}")
\end{lstlisting}

\subsection{Constraint Integration}
\label{subsec:constraint_integration}

\paragraph{\_add\_constraint\_jacobian\_contributions()}\leavevmode
\begin{lstlisting}[language=Python, caption=Add Constraint Jacobian Contributions Method]
def _add_constraint_jacobian_contributions(self, 
                                         jacobian: np.ndarray,
                                         trace_solutions: List[np.ndarray],
                                         multipliers: np.ndarray,
                                         time: float)
\end{lstlisting}

\textbf{Parameters:}
\begin{itemize}
    \item \texttt{jacobian}: Global Jacobian matrix to modify
    \item \texttt{trace\_solutions}: List of trace solution arrays
    \item \texttt{multipliers}: Array of constraint multipliers
    \item \texttt{time}: Current time for constraint evaluation
\end{itemize}

\textbf{Returns:} \texttt{None}

\textbf{Side Effects:} Modifies the global Jacobian matrix in place

\textbf{Purpose:} Add constraint Jacobian contributions for different constraint types

\textbf{Constraint Types Handled:}
\begin{itemize}
    \item \textbf{Dirichlet}: $\frac{\partial}{\partial u}(u - g) = 1$
    \item \textbf{Neumann}: $\frac{\partial}{\partial \text{flux}}(\text{flux} - g) = 1$
    \item \textbf{Trace Continuity}: $\frac{\partial}{\partial u_1}(u_1 - u_2) = 1$, $\frac{\partial}{\partial u_2}(u_1 - u_2) = -1$
    \item \textbf{Kedem-Katchalsky}: $\frac{\partial}{\partial u_1}(\text{flux} + P(u_1 - u_2)) = P$
\end{itemize}

\subsection{BulkData Integration}
\label{subsec_bulk_data_integration}

\paragraph{initialize\_bulk\_data()}\leavevmode
\begin{lstlisting}[language=Python, caption=Initialize BulkData Method]
def initialize_bulk_data(self, 
                       problems: List,
                       discretizations: List,
                       time: float = 0.0) -> List[BulkData]
\end{lstlisting}

\textbf{Parameters:}
\begin{itemize}
    \item \texttt{problems}: List of Problem objects
    \item \texttt{discretizations}: List of discretization objects
    \item \texttt{time}: Initial time (default: 0.0)
\end{itemize}

\textbf{Returns:} \texttt{List[BulkData]} - Initialized BulkData objects for all domains

\textbf{Usage:}
\begin{lstlisting}[language=Python, caption=BulkData Initialization Usage]
# Initialize BulkData objects
bulk_data_list = assembler.initialize_bulk_data(
    problems=problems,
    discretizations=discretizations,
    time=0.0
)

print(f"Initialized {len(bulk_data_list)} BulkData objects")
for i, bd in enumerate(bulk_data_list):
    print(f"  Domain {i}: {bd}")
\end{lstlisting}

\paragraph{compute\_mass\_conservation()}
\leavevmode
\begin{lstlisting}[language=Python, caption=Compute Mass Conservation Method]
def compute_mass_conservation(self, bulk_data_list: List[BulkData]) -> float
\end{lstlisting}

\textbf{Parameters:}
\begin{itemize}
    \item \texttt{bulk\_data\_list}: List of BulkData instances
\end{itemize}

\textbf{Returns:} \texttt{float} - Total mass across all domains

\textbf{Usage:}
\begin{lstlisting}[language=Python, caption=Mass Conservation Usage]
# Monitor mass conservation
initial_bulk_data = assembler.initialize_bulk_data(problems, discretizations, time=0.0)
initial_mass = assembler.compute_mass_conservation(initial_bulk_data)

# After time evolution...
current_mass = assembler.compute_mass_conservation(current_bulk_data)
mass_change = abs(current_mass - initial_mass) / initial_mass

print(f"Mass conservation check:")
print(f"  Initial mass: {initial_mass:.6e}")
print(f"  Current mass: {current_mass:.6e}")
print(f"  Relative change: {mass_change:.6e}")
\end{lstlisting}

\subsection{Utility Methods}
\label{subsec:utility_methods}

\paragraph{get\_num\_domains()}\leavevmode
\begin{lstlisting}[language=Python, caption=Get Number of Domains Method]
def get_num_domains(self) -> int
\end{lstlisting}

\textbf{Returns:} \texttt{int} - Number of domains in the system

\paragraph{get\_domain\_info()}\leavevmode
\begin{lstlisting}[language=Python, caption=Get Domain Info Method]
def get_domain_info(self, domain_idx: int)
\end{lstlisting}

\textbf{Parameters:}
\begin{itemize}
    \item \texttt{domain\_idx}: Domain index
\end{itemize}

\textbf{Returns:} \texttt{DomainData} - Domain data object for inspection

\textbf{Usage:}
\begin{lstlisting}[language=Python, caption=Domain Info Usage]
for i in range(assembler.get_num_domains()):
    domain_info = assembler.get_domain_info(i)
    print(f"Domain {i}:")
    print(f"  Equations: {domain_info.neq}")
    print(f"  Elements: {domain_info.n_elements}")
    print(f"  Element length: {domain_info.element_length}")
\end{lstlisting}

\subsection{Testing Infrastructure}
\label{subsec:testing_infrastructure}

\paragraph{test()}\leavevmode
\begin{lstlisting}[language=Python, caption=Test Method]
def test(self, 
         problems: List = None,
         discretizations: List = None,
         static_condensations: List = None) -> bool
\end{lstlisting}

\textbf{Parameters:}
\begin{itemize}
    \item \texttt{problems}: List of Problem objects for testing (optional)
    \item \texttt{discretizations}: List of discretization objects for testing (optional)
    \item \texttt{static\_condensations}: List of static condensation objects for testing (optional)
\end{itemize}

\textbf{Returns:} \texttt{bool} - True if all tests pass, False otherwise

\textbf{Test Suite:}
\begin{enumerate}
    \item \textbf{Parameter Validation}: Check list lengths match and equal \texttt{n\_domains}
    \item \textbf{BulkDataManager Test}: Validate underlying BulkDataManager
    \item \textbf{DOF Structure Validation}: Check trace DOF counts and offsets
    \item \textbf{Initial Guess Tests}: Test BulkData-based and problem-based creation
    \item \textbf{Forcing Term Tests}: Validate forcing term computation
    \item \textbf{Assembly Tests}: Test residual and Jacobian assembly
    \item \textbf{Solution Extraction Tests}: Validate domain solution and multiplier extraction
    \item \textbf{Mass Conservation Tests}: Check mass computation
    \item \textbf{Constraint Tests}: Validate constraint handling (if constraints exist)
    \item \textbf{Factory Method Tests}: Test creation from framework objects
\end{enumerate}

\textbf{Sample Test Output:}
\begin{lstlisting}[language=Python, caption=Sample Test Output]
Testing LeanGlobalAssembler with 3 domains
PASS: Parameter list lengths validated (3 domains)
PASS: BulkDataManager test passed
PASS: DOF structure validated (trace=84, multipliers=2)
PASS: Domain offsets and sizes validated
PASS: Initial guess creation tests passed
PASS: Forcing term computation test passed
PASS: Residual and Jacobian assembly test passed
  Residual range: [-1.234567e-02, 2.345678e-02]
  Jacobian range: [-5.432109e-01, 6.543210e-01]
PASS: Zero forcing terms test passed
  Zero residual range: [-9.876543e-03, 1.234567e-02]
PASS: Solution extraction tests passed
PASS: Mass conservation test passed (total_mass=2.345678e+00)
PASS: Constraint handling test passed
PASS: Factory method test passed
✓ All LeanGlobalAssembler tests passed!
\end{lstlisting}

\textbf{Usage:}
\begin{lstlisting}[language=Python, caption=Test Usage]
# Comprehensive testing
if assembler.test(problems, discretizations, static_condensations):
    print("✓ LeanGlobalAssembler fully validated")
else:
    print("✗ LeanGlobalAssembler validation failed")

# Basic testing without framework objects
if assembler.test():
    print("✓ Basic structure validation passed")
\end{lstlisting}

\subsection{Special Methods}
\label{subsec:special_methods}

\paragraph{\_\_str\_\_()}\leavevmode
\begin{lstlisting}[language=Python, caption=String Representation Method]
def __str__(self) -> str
\end{lstlisting}

\textbf{Returns:} \texttt{str} - Human-readable representation

\textbf{Format:} \texttt{"LeanGlobalAssembler(domains=N, trace\_dofs=M, multipliers=K, total\_dofs=T)"}

\paragraph{\_\_repr\_\_()}\leavevmode
\begin{lstlisting}[language=Python, caption=Repr Method]
def __repr__(self) -> str
\end{lstlisting}

\textbf{Returns:} \texttt{str} - Developer-oriented representation

\textbf{Format:} \texttt{"LeanGlobalAssembler(n\_domains=N, domain\_trace\_sizes=[...], n\_multipliers=K)"}

\textbf{Usage:}
\begin{lstlisting}[language=Python, caption=String Methods Usage]
print(str(assembler))
# Output: LeanGlobalAssembler(domains=3, trace_dofs=84, multipliers=2, total_dofs=86)

print(repr(assembler))
# Output: LeanGlobalAssembler(n_domains=3, domain_trace_sizes=[28, 28, 28], n_multipliers=2)
\end{lstlisting}

\subsection{Complete Usage Examples}
\label{subsec:complete_usage_examples}

\subsubsection{Complete Newton Solver Integration}

\begin{lstlisting}[language=Python, caption=Complete Newton Solver Integration]
from ooc1d.core.lean_global_assembly import GlobalAssembler
from ooc1d.core.problem import Problem
from ooc1d.core.discretization import Discretization, GlobalDiscretization
from ooc1d.core.constraints import ConstraintManager
from ooc1d.core.lean_bulk_data_manager import BulkDataManager
import numpy as np

def newton_solve_with_lean_assembler(problems, discretizations, static_condensations):
    """Complete Newton solver using lean global assembler."""
    
    # Step 1: Create constraints
    constraint_manager = ConstraintManager()
    constraint_manager.add_dirichlet(0, 0, 0.0, lambda t: np.sin(t))
    constraint_manager.add_neumann(0, -1, problems[-1].domain_end, lambda t: 0.0)
    constraint_manager.map_to_discretizations(discretizations)
    
    # Step 2: Create lean assembler using factory method
    global_disc = GlobalDiscretization(discretizations)
    assembler = GlobalAssembler.from_framework_objects(
        problems=problems,
        global_discretization=global_disc,
        static_condensations=static_condensations,
        constraint_manager=constraint_manager
    )
    
    print(f"Created assembler: {assembler}")
    
    # Step 3: Initialize solution
    initial_guess = assembler.create_initial_guess_from_problems(
        problems, discretizations, time=0.0
    )
    
    # Step 4: Newton iteration
    solution = initial_guess.copy()
    tolerance = 1e-10
    max_iterations = 20
    
    print("Starting Newton iterations...")
    
    for iteration in range(max_iterations):
        # Initialize bulk data for forcing terms
        bulk_data_list = assembler.initialize_bulk_data(problems, discretizations, time=0.0)
        
        # Compute forcing terms (for time-dependent case)
        dt = 0.01
        current_time = iteration * dt
        forcing_terms = assembler.compute_forcing_terms(
            bulk_data_list, problems, discretizations, current_time, dt
        )
        
        # Assemble system
        residual, jacobian = assembler.assemble_residual_and_jacobian(
            global_solution=solution,
            forcing_terms=forcing_terms,
            static_condensations=static_condensations,
            time=current_time
        )
        
        # Check convergence
        residual_norm = np.linalg.norm(residual)
        print(f"  Iteration {iteration}: ||R|| = {residual_norm:.6e}")
        
        if residual_norm < tolerance:
            print("✓ Newton solver converged")
            break
        
        # Newton update
        try:
            delta = np.linalg.solve(jacobian, -residual)
            solution += delta
        except np.linalg.LinAlgError:
            print("✗ Newton solver failed: singular Jacobian")
            break
    
    # Step 5: Extract solutions
    domain_solutions = assembler.get_domain_solutions(solution)
    multipliers = assembler.get_multipliers(solution)
    
    print(f"\nSolver completed:")
    print(f"  Final residual norm: {residual_norm:.6e}")
    print(f"  Domain solutions: {len(domain_solutions)}")
    print(f"  Constraint multipliers: {len(multipliers)}")
    
    return solution, domain_solutions, multipliers

# Usage
problems = [create_problem(i) for i in range(3)]  # User-defined
discretizations = [create_discretization(i) for i in range(3)]  # User-defined
static_condensations = [create_static_condensation(i) for i in range(3)]  # User-defined

solution, domain_sols, multipliers = newton_solve_with_lean_assembler(
    problems, discretizations, static_condensations
)
\end{lstlisting}

\subsubsection{Time Evolution with Lean Assembler}

\begin{lstlisting}[language=Python, caption=Time Evolution with Lean Assembler]
def time_evolution_with_lean_assembler(problems, discretizations, static_condensations):
    """Time evolution loop using lean global assembler."""
    
    # Setup
    constraint_manager = ConstraintManager()
    # Add constraints as needed...
    
    global_disc = GlobalDiscretization(discretizations)
    global_disc.set_time_parameters(dt=0.01, T=1.0)
    
    assembler = GlobalAssembler.from_framework_objects(
        problems, global_disc, static_condensations, constraint_manager
    )
    
    # Initial conditions
    current_solution = assembler.create_initial_guess_from_problems(
        problems, discretizations, time=0.0
    )
    
    # Time evolution
    dt = global_disc.dt
    time_points = global_disc.get_time_points()
    
    solution_history = [current_solution.copy()]
    mass_history = []
    
    print(f"Starting time evolution: {len(time_points)} time steps")
    
    for step, current_time in enumerate(time_points[1:], 1):
        print(f"Time step {step}: t = {current_time:.3f}")
        
        # Get current bulk data for forcing terms
        domain_solutions = assembler.get_domain_solutions(current_solution)
        bulk_data_list = []
        
        for i, domain_sol in enumerate(domain_solutions):
            bulk_data = assembler.bulk_manager.create_bulk_data(
                i, problems[i], discretizations[i], dual=False
            )
            bulk_data.set_data(domain_sol.reshape(-1, 1))
            bulk_data_list.append(bulk_data)
        
        # Compute forcing terms
        forcing_terms = assembler.compute_forcing_terms(
            bulk_data_list, problems, discretizations, current_time, dt
        )
        
        # Newton solve for this time step
        tolerance = 1e-12
        max_newton_iterations = 10
        
        for newton_iter in range(max_newton_iterations):
            residual, jacobian = assembler.assemble_residual_and_jacobian(
                current_solution, forcing_terms, static_condensations, current_time
            )
            
            residual_norm = np.linalg.norm(residual)
            if residual_norm < tolerance:
                break
            
            delta = np.linalg.solve(jacobian, -residual)
            current_solution += delta
        
        # Monitor mass conservation
        current_mass = assembler.compute_mass_conservation(bulk_data_list)
        mass_history.append(current_mass)
        
        # Store solution
        solution_history.append(current_solution.copy())
        
        if step % 10 == 0:
            print(f"  Mass: {current_mass:.6e}, Newton iterations: {newton_iter + 1}")
    
    print("✓ Time evolution completed")
    
    # Mass conservation check
    initial_mass = mass_history[0] if mass_history else 0.0
    final_mass = mass_history[-1] if mass_history else 0.0
    mass_change = abs(final_mass - initial_mass) / initial_mass if initial_mass != 0 else 0.0
    
    print(f"Mass conservation: {mass_change:.6e} relative change")
    
    return solution_history, mass_history

# Usage
solution_history, mass_history = time_evolution_with_lean_assembler(
    problems, discretizations, static_condensations
)
\end{lstlisting}

\subsection{Method Summary Table}
\label{subsec:lean_assembler_method_summary}

\begin{longtable}{|p{7cm}|p{3cm}|p{4.8cm}|}
\hline
\textbf{Method} & \textbf{Returns} & \textbf{Purpose} \\
\hline
\endhead

\texttt{\_\_init\_\_} & \texttt{None} & Initialize lean assembler with domain data \\
\hline

\texttt{from\_framework\_objects} & \texttt{GlobalAssembler} & Factory method for creation from framework objects \\
\hline

\texttt{assemble\_residual\_and\_jacobian} & \texttt{Tuple} & Global system assembly for Newton solver \\
\hline

\texttt{bulk\_by\_static\_condensation} & \texttt{List} & Bulk solution computation from traces \\
\hline

\texttt{compute\_forcing\_terms} & \texttt{List} & Forcing term computation for implicit Euler \\
\hline

\texttt{create\_initial\_guess\_from\_bulk\_data} & \texttt{np.ndarray} & Initial guess from BulkData objects \\
\hline

\texttt{create\_initial\_guess\_from\_problems} & \texttt{np.ndarray} & Initial guess from problem definitions \\
\hline

\texttt{get\_domain\_solutions} & \texttt{List} & Extract domain solutions from global vector \\
\hline

\texttt{get\_multipliers} & \texttt{np.ndarray} & Extract constraint multipliers \\
\hline

\texttt{initialize\_bulk\_data} & \texttt{List} & Initialize BulkData objects from problems \\
\hline

\texttt{compute\_mass\_conservation} & \texttt{float} & Calculate total mass for conservation check \\
\hline

\texttt{get\_num\_domains} & \texttt{int} & Get number of domains in system \\
\hline

\texttt{get\_domain\_info} & \texttt{DomainData} & Access domain data for inspection \\
\hline

\texttt{test} & \texttt{bool} & Comprehensive validation and testing \\
\hline

\end{longtable}

This documentation provides an exact reference for the lean global assembly module, emphasizing its memory-efficient design, integration with the lean BulkDataManager, and comprehensive constraint handling capabilities for multi-domain HDG systems.

% End of lean global assembly module API documentation


\section{Lean Global Assembly Module API Reference}
\label{sec:lean_global_assembly_api}

This section provides an exact reference for the lean global assembly module (\texttt{ooc1d.core.lean\_global\_assembly.GlobalAssembler}) based on detailed analysis of the actual implementation. This class provides global assembly functionality using the lean BulkDataManager approach where framework objects are passed as parameters.

\subsection{Module Overview}

The lean global assembly module provides:
\begin{itemize}
    \item Global residual and Jacobian assembly from domain flux jumps
    \item Constraint integration for boundary and junction conditions
    \item Memory-efficient operation using lean BulkDataManager
    \item DOF management for multi-domain systems
    \item Comprehensive testing infrastructure
\end{itemize}

\subsection{Module Imports and Dependencies}

\begin{lstlisting}[language=Python, caption=Module Dependencies]
import numpy as np
from typing import List, Tuple, Optional

from .discretization import GlobalDiscretization
from .flux_jump import domain_flux_jump
from .constraints import ConstraintManager
from .lean_bulk_data_manager import BulkDataManager
from .bulk_data import BulkData
\end{lstlisting}

\subsection{GlobalAssembler Class}
\label{subsec:lean_global_assembler_class}

Main class for lean global assembly that uses external framework objects as parameters.

\subsubsection{Constructor}

\paragraph{\_\_init\_\_()}\leavevmode
\begin{lstlisting}[language=Python, caption=GlobalAssembler Constructor]
def __init__(self, 
             domain_data_list: List,
             constraint_manager: Optional[ConstraintManager] = None)
\end{lstlisting}

\textbf{Parameters:}
\begin{itemize}
    \item \texttt{domain\_data\_list}: List of DomainData objects with essential information
    \item \texttt{constraint\_manager}: Optional constraint manager for boundary/junction conditions
\end{itemize}

\textbf{Side Effects:}
\begin{itemize}
    \item Creates internal BulkDataManager with domain data
    \item Computes DOF structure and domain offsets
    \item Stores constraint manager reference
\end{itemize}

\textbf{Usage:}
\begin{lstlisting}[language=Python, caption=Constructor Usage]
# Extract domain data first
domain_data_list = BulkDataManager.extract_domain_data_list(
    problems, discretizations, static_condensations
)

# Create constraint manager
constraint_manager = ConstraintManager()
constraint_manager.add_dirichlet(0, 0, 0.0)
constraint_manager.map_to_discretizations(discretizations)

# Create lean assembler
assembler = GlobalAssembler(domain_data_list, constraint_manager)
\end{lstlisting}

\subsubsection{Core Attributes}

\begin{longtable}{|p{3.5cm}|p{2.5cm}|p{7cm}|}
\hline
\textbf{Attribute} & \textbf{Type} & \textbf{Description} \\
\hline
\endhead

\texttt{bulk\_manager} & \texttt{BulkDataManager} & Lean bulk data manager instance \\
\hline

\texttt{constraint\_manager} & \texttt{Optional[ConstraintManager]} & Constraint manager for boundary/junction conditions \\
\hline

\texttt{n\_domains} & \texttt{int} & Number of domains in the system \\
\hline

\texttt{domain\_trace\_sizes} & \texttt{List[int]} & Trace DOF count for each domain \\
\hline

\texttt{domain\_trace\_offsets} & \texttt{List[int]} & Starting indices for domain traces in global vector \\
\hline

\texttt{total\_trace\_dofs} & \texttt{int} & Total number of trace DOFs across all domains \\
\hline

\texttt{n\_multipliers} & \texttt{int} & Number of constraint multipliers \\
\hline

\texttt{total\_dofs} & \texttt{int} & Total DOFs: \texttt{total\_trace\_dofs + n\_multipliers} \\
\hline

\end{longtable}

\subsubsection{Factory Method}

\paragraph{from\_framework\_objects()}\leavevmode
\begin{lstlisting}[language=Python, caption=Factory Method]
@classmethod
def from_framework_objects(cls,
                          problems: List,
                          global_discretization,
                          static_condensations: List,
                          constraint_manager: Optional[ConstraintManager] = None)
\end{lstlisting}

\textbf{Parameters:}
\begin{itemize}
    \item \texttt{problems}: List of Problem instances for each domain
    \item \texttt{global\_discretization}: GlobalDiscretization instance
    \item \texttt{static\_condensations}: List of static condensation implementations
    \item \texttt{constraint\_manager}: Optional constraint manager
\end{itemize}

\textbf{Returns:} \texttt{GlobalAssembler} - Configured assembler instance

\textbf{Purpose:} Factory method to create assembler from framework objects

\textbf{Usage:}
\begin{lstlisting}[language=Python, caption=Factory Method Usage]
# Create assembler from framework objects
assembler = GlobalAssembler.from_framework_objects(
    problems=problems,
    global_discretization=global_disc,
    static_condensations=static_condensations,
    constraint_manager=constraint_manager
)

print(f"Created assembler: {assembler}")
\end{lstlisting}

\subsubsection{Internal Initialization Methods}

\paragraph{\_compute\_dof\_structure()}\leavevmode
\begin{lstlisting}[language=Python, caption=Compute DOF Structure Method]
def _compute_dof_structure(self)
\end{lstlisting}

\textbf{Purpose:} Compute global DOF structure from domain data

\textbf{Side Effects:} Sets all DOF-related attributes

\textbf{Algorithm:}
\begin{enumerate}
    \item For each domain: compute \texttt{trace\_size = neq * (n\_elements + 1)}
    \item Compute cumulative offsets for global vector assembly
    \item Calculate total trace DOFs and multipliers
    \item Set total DOF count
\end{enumerate}

\subsection{Primary Assembly Methods}
\label{subsec:primary_assembly_methods}

\paragraph{assemble\_residual\_and\_jacobian()}\leavevmode
\begin{lstlisting}[language=Python, caption=Assemble Residual and Jacobian Method]
def assemble_residual_and_jacobian(self, 
                                 global_solution: np.ndarray,
                                 forcing_terms: List[np.ndarray],
                                 static_condensations: List,
                                 time: float) -> Tuple[np.ndarray, np.ndarray]
\end{lstlisting}

\textbf{Parameters:}
\begin{itemize}
    \item \texttt{global\_solution}: Global solution vector \texttt{[trace\_solutions; multipliers]}
    \item \texttt{forcing\_terms}: List of forcing term arrays for each domain (pre-computed)
    \item \texttt{static\_condensations}: List of static condensation objects for flux jump computation
    \item \texttt{time}: Current time for constraint evaluation
\end{itemize}

\textbf{Returns:} \texttt{Tuple[np.ndarray, np.ndarray]} - (residual, jacobian) global arrays

\textbf{Purpose:} Assemble global residual and Jacobian from domain flux jumps and constraints

\textbf{Mathematical Formulation:} Solves the nonlinear system $F(U; F_{ext}) = 0$ where:
\begin{itemize}
    \item $U$ is the trace solution
    \item $F_{ext}$ are the forcing terms (pre-computed as $dt \cdot f + M \cdot u_{old}$)
\end{itemize}

\textbf{Algorithm:}
\begin{enumerate}
    \item Extract trace solutions and multipliers from global solution
    \item Initialize global residual and Jacobian arrays
    \item For each domain: compute flux jump using \texttt{domain\_flux\_jump()}
    \item Add domain contributions to global arrays
    \item Add constraint contributions (multiplier coupling and constraint residuals)
    \item Add constraint Jacobian contributions
\end{enumerate}

\textbf{Usage:}
\begin{lstlisting}[language=Python, caption=Residual and Jacobian Assembly Usage]
# Prepare inputs
global_solution = np.random.rand(assembler.total_dofs)
forcing_terms = [np.random.rand(2*neq, n_elements) for _ in range(n_domains)]

# Assemble system
residual, jacobian = assembler.assemble_residual_and_jacobian(
    global_solution=global_solution,
    forcing_terms=forcing_terms,
    static_condensations=static_condensations,
    time=0.5
)

print(f"System assembled:")
print(f"  Residual shape: {residual.shape}")
print(f"  Jacobian shape: {jacobian.shape}")
print(f"  Residual norm: {np.linalg.norm(residual):.6e}")
print(f"  Jacobian condition: {np.linalg.cond(jacobian):.2e}")
\end{lstlisting}

\paragraph{bulk\_by\_static\_condensation()}
\leavevmode
\begin{lstlisting}[language=Python, caption=Bulk by Static Condensation Method]
def bulk_by_static_condensation(self, 
                               global_solution: np.ndarray,
                               forcing_terms: List[np.ndarray],
                               static_condensations: List,
                               time: float) -> Tuple[np.ndarray, np.ndarray]
\end{lstlisting}

\textbf{Parameters:} Same as \texttt{assemble\_residual\_and\_jacobian()}

\textbf{Returns:} \texttt{List[np.ndarray]} - List of bulk solution arrays for each domain

\textbf{Purpose:} Compute bulk solutions from trace solutions using static condensation

\textbf{Usage:}
\begin{lstlisting}[language=Python, caption=Bulk Solution Computation Usage]
# Compute bulk solutions only
bulk_solutions = assembler.bulk_by_static_condensation(
    global_solution=global_solution,
    forcing_terms=forcing_terms,
    static_condensations=static_condensations,
    time=0.5
)

for i, bulk_sol in enumerate(bulk_solutions):
    print(f"Domain {i} bulk solution shape: {bulk_sol.shape}")
\end{lstlisting}

\subsubsection{Forcing Term Computation}

\paragraph{compute\_forcing\_terms()}\leavevmode
\begin{lstlisting}[language=Python, caption=Compute Forcing Terms Method]
def compute_forcing_terms(self,
                        bulk_data_list: List[BulkData],
                        problems: List,
                        discretizations: List,
                        time: float,
                        dt: float) -> List[np.ndarray]
\end{lstlisting}

\textbf{Parameters:}
\begin{itemize}
    \item \texttt{bulk\_data\_list}: List of BulkData objects from previous time step
    \item \texttt{problems}: List of Problem objects
    \item \texttt{discretizations}: List of discretization objects
    \item \texttt{time}: Current time
    \item \texttt{dt}: Time step size
\end{itemize}

\textbf{Returns:} \texttt{List[np.ndarray]} - Forcing term arrays for implicit Euler

\textbf{Computation:} For each domain: $\text{forcing\_term} = dt \cdot f + M \cdot u_{old}$

\textbf{Usage:}
\begin{lstlisting}[language=Python, caption=Forcing Terms Usage]
# Previous time step solutions
bulk_data_list = assembler.initialize_bulk_data(problems, discretizations, time=0.0)

# Compute forcing terms for next time step
forcing_terms = assembler.compute_forcing_terms(
    bulk_data_list=bulk_data_list,
    problems=problems,
    discretizations=discretizations,
    time=0.1,
    dt=0.01
)

print(f"Computed {len(forcing_terms)} forcing term arrays")
for i, ft in enumerate(forcing_terms):
    print(f"  Domain {i}: {ft.shape}")
\end{lstlisting}

\subsection{Solution Vector Management}
\label{subsec:solution_vector_management}

\paragraph{\_extract\_trace\_solutions()}\leavevmode
\begin{lstlisting}[language=Python, caption=Extract Trace Solutions Method]
def _extract_trace_solutions(self, global_solution: np.ndarray) -> List[np.ndarray]
\end{lstlisting}

\textbf{Parameters:}
\begin{itemize}
    \item \texttt{global\_solution}: Global solution vector
\end{itemize}

\textbf{Returns:} \texttt{List[np.ndarray]} - Individual domain trace solutions

\textbf{Purpose:} Extract domain-specific trace solutions from global vector

\paragraph{get\_domain\_solutions()}\leavevmode
\begin{lstlisting}[language=Python, caption=Get Domain Solutions Method]
def get_domain_solutions(self, global_solution: np.ndarray) -> List[np.ndarray]
\end{lstlisting}

\textbf{Parameters:}
\begin{itemize}
    \item \texttt{global\_solution}: Global solution vector
\end{itemize}

\textbf{Returns:} \texttt{List[np.ndarray]} - Domain trace solutions (public interface)

\textbf{Usage:}
\begin{lstlisting}[language=Python, caption=Solution Extraction Usage]
# Extract individual domain solutions
domain_solutions = assembler.get_domain_solutions(global_solution)

for i, domain_sol in enumerate(domain_solutions):
    print(f"Domain {i} solution shape: {domain_sol.shape}")
    print(f"  Solution range: [{np.min(domain_sol):.6e}, {np.max(domain_sol):.6e}]")
\end{lstlisting}

\paragraph{get\_multipliers()}\leavevmode
\begin{lstlisting}[language=Python, caption=Get Multipliers Method]
def get_multipliers(self, global_solution: np.ndarray) -> np.ndarray
\end{lstlisting}

\textbf{Parameters:}
\begin{itemize}
    \item \texttt{global\_solution}: Global solution vector
\end{itemize}

\textbf{Returns:} \texttt{np.ndarray} - Constraint multipliers (empty if no constraints)

\textbf{Usage:}
\begin{lstlisting}[language=Python, caption=Multiplier Extraction Usage]
multipliers = assembler.get_multipliers(global_solution)
if len(multipliers) > 0:
    print(f"Constraint multipliers: {multipliers}")
else:
    print("No constraints defined")
\end{lstlisting}

\subsection{Initial Guess Creation}
\label{subsec:initial_guess_creation}

\paragraph{create\_initial\_guess\_from\_bulk\_data()}\leavevmode
\begin{lstlisting}[language=Python, caption=Initial Guess from BulkData Method]
def create_initial_guess_from_bulk_data(self, 
                                       bulk_data_list: List[BulkData]) -> np.ndarray
\end{lstlisting}

\textbf{Parameters:}
\begin{itemize}
    \item \texttt{bulk\_data\_list}: List of BulkData objects
\end{itemize}

\textbf{Returns:} \texttt{np.ndarray} - Initial guess for global solution vector

\textbf{Purpose:} Create initial guess from existing BulkData trace values

\textbf{Usage:}
\begin{lstlisting}[language=Python, caption=BulkData Initial Guess Usage]
# Initialize BulkData from problems
bulk_data_list = assembler.initialize_bulk_data(problems, discretizations, time=0.0)

# Create initial guess
initial_guess = assembler.create_initial_guess_from_bulk_data(bulk_data_list)
print(f"Initial guess shape: {initial_guess.shape}")
\end{lstlisting}

\paragraph{create\_initial\_guess\_from\_problems()}\leavevmode
\begin{lstlisting}[language=Python, caption=Initial Guess from Problems Method]
def create_initial_guess_from_problems(self, 
                                     problems: List,
                                     discretizations: List,
                                     time: float = 0.0) -> np.ndarray
\end{lstlisting}

\textbf{Parameters:}
\begin{itemize}
    \item \texttt{problems}: List of Problem objects
    \item \texttt{discretizations}: List of discretization objects
    \item \texttt{time}: Time for initial condition evaluation (default: 0.0)
\end{itemize}

\textbf{Returns:} \texttt{np.ndarray} - Initial guess for global solution vector

\textbf{Purpose:} Create initial guess directly from problem initial conditions

\textbf{Algorithm:}
\begin{enumerate}
    \item Initialize global solution vector with zeros
    \item For each domain and equation: evaluate initial condition at mesh nodes
    \item Fill corresponding entries in global vector
    \item Set multipliers to zero
\end{enumerate}

\textbf{Usage:}
\begin{lstlisting}[language=Python, caption=Problem Initial Guess Usage]
# Create initial guess directly from problem definitions
initial_guess = assembler.create_initial_guess_from_problems(
    problems=problems,
    discretizations=discretizations,
    time=0.0
)

print(f"Initial guess from problems: {initial_guess.shape}")
print(f"Non-zero entries: {np.count_nonzero(initial_guess)}")
\end{lstlisting}

\subsection{Constraint Integration}
\label{subsec:constraint_integration}

\paragraph{\_add\_constraint\_jacobian\_contributions()}\leavevmode
\begin{lstlisting}[language=Python, caption=Add Constraint Jacobian Contributions Method]
def _add_constraint_jacobian_contributions(self, 
                                         jacobian: np.ndarray,
                                         trace_solutions: List[np.ndarray],
                                         multipliers: np.ndarray,
                                         time: float)
\end{lstlisting}

\textbf{Parameters:}
\begin{itemize}
    \item \texttt{jacobian}: Global Jacobian matrix to modify
    \item \texttt{trace\_solutions}: List of trace solution arrays
    \item \texttt{multipliers}: Array of constraint multipliers
    \item \texttt{time}: Current time for constraint evaluation
\end{itemize}

\textbf{Returns:} \texttt{None}

\textbf{Side Effects:} Modifies the global Jacobian matrix in place

\textbf{Purpose:} Add constraint Jacobian contributions for different constraint types

\textbf{Constraint Types Handled:}
\begin{itemize}
    \item \textbf{Dirichlet}: $\frac{\partial}{\partial u}(u - g) = 1$
    \item \textbf{Neumann}: $\frac{\partial}{\partial \text{flux}}(\text{flux} - g) = 1$
    \item \textbf{Trace Continuity}: $\frac{\partial}{\partial u_1}(u_1 - u_2) = 1$, $\frac{\partial}{\partial u_2}(u_1 - u_2) = -1$
    \item \textbf{Kedem-Katchalsky}: $\frac{\partial}{\partial u_1}(\text{flux} + P(u_1 - u_2)) = P$
\end{itemize}

\subsection{BulkData Integration}
\label{subsec_bulk_data_integration}

\paragraph{initialize\_bulk\_data()}\leavevmode
\begin{lstlisting}[language=Python, caption=Initialize BulkData Method]
def initialize_bulk_data(self, 
                       problems: List,
                       discretizations: List,
                       time: float = 0.0) -> List[BulkData]
\end{lstlisting}

\textbf{Parameters:}
\begin{itemize}
    \item \texttt{problems}: List of Problem objects
    \item \texttt{discretizations}: List of discretization objects
    \item \texttt{time}: Initial time (default: 0.0)
\end{itemize}

\textbf{Returns:} \texttt{List[BulkData]} - Initialized BulkData objects for all domains

\textbf{Usage:}
\begin{lstlisting}[language=Python, caption=BulkData Initialization Usage]
# Initialize BulkData objects
bulk_data_list = assembler.initialize_bulk_data(
    problems=problems,
    discretizations=discretizations,
    time=0.0
)

print(f"Initialized {len(bulk_data_list)} BulkData objects")
for i, bd in enumerate(bulk_data_list):
    print(f"  Domain {i}: {bd}")
\end{lstlisting}

\paragraph{compute\_mass\_conservation()}
\leavevmode
\begin{lstlisting}[language=Python, caption=Compute Mass Conservation Method]
def compute_mass_conservation(self, bulk_data_list: List[BulkData]) -> float
\end{lstlisting}

\textbf{Parameters:}
\begin{itemize}
    \item \texttt{bulk\_data\_list}: List of BulkData instances
\end{itemize}

\textbf{Returns:} \texttt{float} - Total mass across all domains

\textbf{Usage:}
\begin{lstlisting}[language=Python, caption=Mass Conservation Usage]
# Monitor mass conservation
initial_bulk_data = assembler.initialize_bulk_data(problems, discretizations, time=0.0)
initial_mass = assembler.compute_mass_conservation(initial_bulk_data)

# After time evolution...
current_mass = assembler.compute_mass_conservation(current_bulk_data)
mass_change = abs(current_mass - initial_mass) / initial_mass

print(f"Mass conservation check:")
print(f"  Initial mass: {initial_mass:.6e}")
print(f"  Current mass: {current_mass:.6e}")
print(f"  Relative change: {mass_change:.6e}")
\end{lstlisting}

\subsection{Utility Methods}
\label{subsec:utility_methods}

\paragraph{get\_num\_domains()}\leavevmode
\begin{lstlisting}[language=Python, caption=Get Number of Domains Method]
def get_num_domains(self) -> int
\end{lstlisting}

\textbf{Returns:} \texttt{int} - Number of domains in the system

\paragraph{get\_domain\_info()}\leavevmode
\begin{lstlisting}[language=Python, caption=Get Domain Info Method]
def get_domain_info(self, domain_idx: int)
\end{lstlisting}

\textbf{Parameters:}
\begin{itemize}
    \item \texttt{domain\_idx}: Domain index
\end{itemize}

\textbf{Returns:} \texttt{DomainData} - Domain data object for inspection

\textbf{Usage:}
\begin{lstlisting}[language=Python, caption=Domain Info Usage]
for i in range(assembler.get_num_domains()):
    domain_info = assembler.get_domain_info(i)
    print(f"Domain {i}:")
    print(f"  Equations: {domain_info.neq}")
    print(f"  Elements: {domain_info.n_elements}")
    print(f"  Element length: {domain_info.element_length}")
\end{lstlisting}

\subsection{Testing Infrastructure}
\label{subsec:testing_infrastructure}

\paragraph{test()}\leavevmode
\begin{lstlisting}[language=Python, caption=Test Method]
def test(self, 
         problems: List = None,
         discretizations: List = None,
         static_condensations: List = None) -> bool
\end{lstlisting}

\textbf{Parameters:}
\begin{itemize}
    \item \texttt{problems}: List of Problem objects for testing (optional)
    \item \texttt{discretizations}: List of discretization objects for testing (optional)
    \item \texttt{static\_condensations}: List of static condensation objects for testing (optional)
\end{itemize}

\textbf{Returns:} \texttt{bool} - True if all tests pass, False otherwise

\textbf{Test Suite:}
\begin{enumerate}
    \item \textbf{Parameter Validation}: Check list lengths match and equal \texttt{n\_domains}
    \item \textbf{BulkDataManager Test}: Validate underlying BulkDataManager
    \item \textbf{DOF Structure Validation}: Check trace DOF counts and offsets
    \item \textbf{Initial Guess Tests}: Test BulkData-based and problem-based creation
    \item \textbf{Forcing Term Tests}: Validate forcing term computation
    \item \textbf{Assembly Tests}: Test residual and Jacobian assembly
    \item \textbf{Solution Extraction Tests}: Validate domain solution and multiplier extraction
    \item \textbf{Mass Conservation Tests}: Check mass computation
    \item \textbf{Constraint Tests}: Validate constraint handling (if constraints exist)
    \item \textbf{Factory Method Tests}: Test creation from framework objects
\end{enumerate}

\textbf{Sample Test Output:}
\begin{lstlisting}[language=Python, caption=Sample Test Output]
Testing LeanGlobalAssembler with 3 domains
PASS: Parameter list lengths validated (3 domains)
PASS: BulkDataManager test passed
PASS: DOF structure validated (trace=84, multipliers=2)
PASS: Domain offsets and sizes validated
PASS: Initial guess creation tests passed
PASS: Forcing term computation test passed
PASS: Residual and Jacobian assembly test passed
  Residual range: [-1.234567e-02, 2.345678e-02]
  Jacobian range: [-5.432109e-01, 6.543210e-01]
PASS: Zero forcing terms test passed
  Zero residual range: [-9.876543e-03, 1.234567e-02]
PASS: Solution extraction tests passed
PASS: Mass conservation test passed (total_mass=2.345678e+00)
PASS: Constraint handling test passed
PASS: Factory method test passed
✓ All LeanGlobalAssembler tests passed!
\end{lstlisting}

\textbf{Usage:}
\begin{lstlisting}[language=Python, caption=Test Usage]
# Comprehensive testing
if assembler.test(problems, discretizations, static_condensations):
    print("✓ LeanGlobalAssembler fully validated")
else:
    print("✗ LeanGlobalAssembler validation failed")

# Basic testing without framework objects
if assembler.test():
    print("✓ Basic structure validation passed")
\end{lstlisting}

\subsection{Special Methods}
\label{subsec:special_methods}

\paragraph{\_\_str\_\_()}\leavevmode
\begin{lstlisting}[language=Python, caption=String Representation Method]
def __str__(self) -> str
\end{lstlisting}

\textbf{Returns:} \texttt{str} - Human-readable representation

\textbf{Format:} \texttt{"LeanGlobalAssembler(domains=N, trace\_dofs=M, multipliers=K, total\_dofs=T)"}

\paragraph{\_\_repr\_\_()}\leavevmode
\begin{lstlisting}[language=Python, caption=Repr Method]
def __repr__(self) -> str
\end{lstlisting}

\textbf{Returns:} \texttt{str} - Developer-oriented representation

\textbf{Format:} \texttt{"LeanGlobalAssembler(n\_domains=N, domain\_trace\_sizes=[...], n\_multipliers=K)"}

\textbf{Usage:}
\begin{lstlisting}[language=Python, caption=String Methods Usage]
print(str(assembler))
# Output: LeanGlobalAssembler(domains=3, trace_dofs=84, multipliers=2, total_dofs=86)

print(repr(assembler))
# Output: LeanGlobalAssembler(n_domains=3, domain_trace_sizes=[28, 28, 28], n_multipliers=2)
\end{lstlisting}

\subsection{Complete Usage Examples}
\label{subsec:complete_usage_examples}

\subsubsection{Complete Newton Solver Integration}

\begin{lstlisting}[language=Python, caption=Complete Newton Solver Integration]
from ooc1d.core.lean_global_assembly import GlobalAssembler
from ooc1d.core.problem import Problem
from ooc1d.core.discretization import Discretization, GlobalDiscretization
from ooc1d.core.constraints import ConstraintManager
from ooc1d.core.lean_bulk_data_manager import BulkDataManager
import numpy as np

def newton_solve_with_lean_assembler(problems, discretizations, static_condensations):
    """Complete Newton solver using lean global assembler."""
    
    # Step 1: Create constraints
    constraint_manager = ConstraintManager()
    constraint_manager.add_dirichlet(0, 0, 0.0, lambda t: np.sin(t))
    constraint_manager.add_neumann(0, -1, problems[-1].domain_end, lambda t: 0.0)
    constraint_manager.map_to_discretizations(discretizations)
    
    # Step 2: Create lean assembler using factory method
    global_disc = GlobalDiscretization(discretizations)
    assembler = GlobalAssembler.from_framework_objects(
        problems=problems,
        global_discretization=global_disc,
        static_condensations=static_condensations,
        constraint_manager=constraint_manager
    )
    
    print(f"Created assembler: {assembler}")
    
    # Step 3: Initialize solution
    initial_guess = assembler.create_initial_guess_from_problems(
        problems, discretizations, time=0.0
    )
    
    # Step 4: Newton iteration
    solution = initial_guess.copy()
    tolerance = 1e-10
    max_iterations = 20
    
    print("Starting Newton iterations...")
    
    for iteration in range(max_iterations):
        # Initialize bulk data for forcing terms
        bulk_data_list = assembler.initialize_bulk_data(problems, discretizations, time=0.0)
        
        # Compute forcing terms (for time-dependent case)
        dt = 0.01
        current_time = iteration * dt
        forcing_terms = assembler.compute_forcing_terms(
            bulk_data_list, problems, discretizations, current_time, dt
        )
        
        # Assemble system
        residual, jacobian = assembler.assemble_residual_and_jacobian(
            global_solution=solution,
            forcing_terms=forcing_terms,
            static_condensations=static_condensations,
            time=current_time
        )
        
        # Check convergence
        residual_norm = np.linalg.norm(residual)
        print(f"  Iteration {iteration}: ||R|| = {residual_norm:.6e}")
        
        if residual_norm < tolerance:
            print("✓ Newton solver converged")
            break
        
        # Newton update
        try:
            delta = np.linalg.solve(jacobian, -residual)
            solution += delta
        except np.linalg.LinAlgError:
            print("✗ Newton solver failed: singular Jacobian")
            break
    
    # Step 5: Extract solutions
    domain_solutions = assembler.get_domain_solutions(solution)
    multipliers = assembler.get_multipliers(solution)
    
    print(f"\nSolver completed:")
    print(f"  Final residual norm: {residual_norm:.6e}")
    print(f"  Domain solutions: {len(domain_solutions)}")
    print(f"  Constraint multipliers: {len(multipliers)}")
    
    return solution, domain_solutions, multipliers

# Usage
problems = [create_problem(i) for i in range(3)]  # User-defined
discretizations = [create_discretization(i) for i in range(3)]  # User-defined
static_condensations = [create_static_condensation(i) for i in range(3)]  # User-defined

solution, domain_sols, multipliers = newton_solve_with_lean_assembler(
    problems, discretizations, static_condensations
)
\end{lstlisting}

\subsubsection{Time Evolution with Lean Assembler}

\begin{lstlisting}[language=Python, caption=Time Evolution with Lean Assembler]
def time_evolution_with_lean_assembler(problems, discretizations, static_condensations):
    """Time evolution loop using lean global assembler."""
    
    # Setup
    constraint_manager = ConstraintManager()
    # Add constraints as needed...
    
    global_disc = GlobalDiscretization(discretizations)
    global_disc.set_time_parameters(dt=0.01, T=1.0)
    
    assembler = GlobalAssembler.from_framework_objects(
        problems, global_disc, static_condensations, constraint_manager
    )
    
    # Initial conditions
    current_solution = assembler.create_initial_guess_from_problems(
        problems, discretizations, time=0.0
    )
    
    # Time evolution
    dt = global_disc.dt
    time_points = global_disc.get_time_points()
    
    solution_history = [current_solution.copy()]
    mass_history = []
    
    print(f"Starting time evolution: {len(time_points)} time steps")
    
    for step, current_time in enumerate(time_points[1:], 1):
        print(f"Time step {step}: t = {current_time:.3f}")
        
        # Get current bulk data for forcing terms
        domain_solutions = assembler.get_domain_solutions(current_solution)
        bulk_data_list = []
        
        for i, domain_sol in enumerate(domain_solutions):
            bulk_data = assembler.bulk_manager.create_bulk_data(
                i, problems[i], discretizations[i], dual=False
            )
            bulk_data.set_data(domain_sol.reshape(-1, 1))
            bulk_data_list.append(bulk_data)
        
        # Compute forcing terms
        forcing_terms = assembler.compute_forcing_terms(
            bulk_data_list, problems, discretizations, current_time, dt
        )
        
        # Newton solve for this time step
        tolerance = 1e-12
        max_newton_iterations = 10
        
        for newton_iter in range(max_newton_iterations):
            residual, jacobian = assembler.assemble_residual_and_jacobian(
                current_solution, forcing_terms, static_condensations, current_time
            )
            
            residual_norm = np.linalg.norm(residual)
            if residual_norm < tolerance:
                break
            
            delta = np.linalg.solve(jacobian, -residual)
            current_solution += delta
        
        # Monitor mass conservation
        current_mass = assembler.compute_mass_conservation(bulk_data_list)
        mass_history.append(current_mass)
        
        # Store solution
        solution_history.append(current_solution.copy())
        
        if step % 10 == 0:
            print(f"  Mass: {current_mass:.6e}, Newton iterations: {newton_iter + 1}")
    
    print("✓ Time evolution completed")
    
    # Mass conservation check
    initial_mass = mass_history[0] if mass_history else 0.0
    final_mass = mass_history[-1] if mass_history else 0.0
    mass_change = abs(final_mass - initial_mass) / initial_mass if initial_mass != 0 else 0.0
    
    print(f"Mass conservation: {mass_change:.6e} relative change")
    
    return solution_history, mass_history

# Usage
solution_history, mass_history = time_evolution_with_lean_assembler(
    problems, discretizations, static_condensations
)
\end{lstlisting}

\subsection{Method Summary Table}
\label{subsec:lean_assembler_method_summary}

\begin{longtable}{|p{7cm}|p{3cm}|p{4.8cm}|}
\hline
\textbf{Method} & \textbf{Returns} & \textbf{Purpose} \\
\hline
\endhead

\texttt{\_\_init\_\_} & \texttt{None} & Initialize lean assembler with domain data \\
\hline

\texttt{from\_framework\_objects} & \texttt{GlobalAssembler} & Factory method for creation from framework objects \\
\hline

\texttt{assemble\_residual\_and\_jacobian} & \texttt{Tuple} & Global system assembly for Newton solver \\
\hline

\texttt{bulk\_by\_static\_condensation} & \texttt{List} & Bulk solution computation from traces \\
\hline

\texttt{compute\_forcing\_terms} & \texttt{List} & Forcing term computation for implicit Euler \\
\hline

\texttt{create\_initial\_guess\_from\_bulk\_data} & \texttt{np.ndarray} & Initial guess from BulkData objects \\
\hline

\texttt{create\_initial\_guess\_from\_problems} & \texttt{np.ndarray} & Initial guess from problem definitions \\
\hline

\texttt{get\_domain\_solutions} & \texttt{List} & Extract domain solutions from global vector \\
\hline

\texttt{get\_multipliers} & \texttt{np.ndarray} & Extract constraint multipliers \\
\hline

\texttt{initialize\_bulk\_data} & \texttt{List} & Initialize BulkData objects from problems \\
\hline

\texttt{compute\_mass\_conservation} & \texttt{float} & Calculate total mass for conservation check \\
\hline

\texttt{get\_num\_domains} & \texttt{int} & Get number of domains in system \\
\hline

\texttt{get\_domain\_info} & \texttt{DomainData} & Access domain data for inspection \\
\hline

\texttt{test} & \texttt{bool} & Comprehensive validation and testing \\
\hline

\end{longtable}

This documentation provides an exact reference for the lean global assembly module, emphasizing its memory-efficient design, integration with the lean BulkDataManager, and comprehensive constraint handling capabilities for multi-domain HDG systems.

% End of lean global assembly module API documentation


\section{Lean Global Assembly Module API Reference}
\label{sec:lean_global_assembly_api}

This section provides an exact reference for the lean global assembly module (\texttt{ooc1d.core.lean\_global\_assembly.GlobalAssembler}) based on detailed analysis of the actual implementation. This class provides global assembly functionality using the lean BulkDataManager approach where framework objects are passed as parameters.

\subsection{Module Overview}

The lean global assembly module provides:
\begin{itemize}
    \item Global residual and Jacobian assembly from domain flux jumps
    \item Constraint integration for boundary and junction conditions
    \item Memory-efficient operation using lean BulkDataManager
    \item DOF management for multi-domain systems
    \item Comprehensive testing infrastructure
\end{itemize}

\subsection{Module Imports and Dependencies}

\begin{lstlisting}[language=Python, caption=Module Dependencies]
import numpy as np
from typing import List, Tuple, Optional

from .discretization import GlobalDiscretization
from .flux_jump import domain_flux_jump
from .constraints import ConstraintManager
from .lean_bulk_data_manager import BulkDataManager
from .bulk_data import BulkData
\end{lstlisting}

\subsection{GlobalAssembler Class}
\label{subsec:lean_global_assembler_class}

Main class for lean global assembly that uses external framework objects as parameters.

\subsubsection{Constructor}

\paragraph{\_\_init\_\_()}\leavevmode
\begin{lstlisting}[language=Python, caption=GlobalAssembler Constructor]
def __init__(self, 
             domain_data_list: List,
             constraint_manager: Optional[ConstraintManager] = None)
\end{lstlisting}

\textbf{Parameters:}
\begin{itemize}
    \item \texttt{domain\_data\_list}: List of DomainData objects with essential information
    \item \texttt{constraint\_manager}: Optional constraint manager for boundary/junction conditions
\end{itemize}

\textbf{Side Effects:}
\begin{itemize}
    \item Creates internal BulkDataManager with domain data
    \item Computes DOF structure and domain offsets
    \item Stores constraint manager reference
\end{itemize}

\textbf{Usage:}
\begin{lstlisting}[language=Python, caption=Constructor Usage]
# Extract domain data first
domain_data_list = BulkDataManager.extract_domain_data_list(
    problems, discretizations, static_condensations
)

# Create constraint manager
constraint_manager = ConstraintManager()
constraint_manager.add_dirichlet(0, 0, 0.0)
constraint_manager.map_to_discretizations(discretizations)

# Create lean assembler
assembler = GlobalAssembler(domain_data_list, constraint_manager)
\end{lstlisting}

\subsubsection{Core Attributes}

\begin{longtable}{|p{3.5cm}|p{2.5cm}|p{7cm}|}
\hline
\textbf{Attribute} & \textbf{Type} & \textbf{Description} \\
\hline
\endhead

\texttt{bulk\_manager} & \texttt{BulkDataManager} & Lean bulk data manager instance \\
\hline

\texttt{constraint\_manager} & \texttt{Optional[ConstraintManager]} & Constraint manager for boundary/junction conditions \\
\hline

\texttt{n\_domains} & \texttt{int} & Number of domains in the system \\
\hline

\texttt{domain\_trace\_sizes} & \texttt{List[int]} & Trace DOF count for each domain \\
\hline

\texttt{domain\_trace\_offsets} & \texttt{List[int]} & Starting indices for domain traces in global vector \\
\hline

\texttt{total\_trace\_dofs} & \texttt{int} & Total number of trace DOFs across all domains \\
\hline

\texttt{n\_multipliers} & \texttt{int} & Number of constraint multipliers \\
\hline

\texttt{total\_dofs} & \texttt{int} & Total DOFs: \texttt{total\_trace\_dofs + n\_multipliers} \\
\hline

\end{longtable}

\subsubsection{Factory Method}

\paragraph{from\_framework\_objects()}\leavevmode
\begin{lstlisting}[language=Python, caption=Factory Method]
@classmethod
def from_framework_objects(cls,
                          problems: List,
                          global_discretization,
                          static_condensations: List,
                          constraint_manager: Optional[ConstraintManager] = None)
\end{lstlisting}

\textbf{Parameters:}
\begin{itemize}
    \item \texttt{problems}: List of Problem instances for each domain
    \item \texttt{global\_discretization}: GlobalDiscretization instance
    \item \texttt{static\_condensations}: List of static condensation implementations
    \item \texttt{constraint\_manager}: Optional constraint manager
\end{itemize}

\textbf{Returns:} \texttt{GlobalAssembler} - Configured assembler instance

\textbf{Purpose:} Factory method to create assembler from framework objects

\textbf{Usage:}
\begin{lstlisting}[language=Python, caption=Factory Method Usage]
# Create assembler from framework objects
assembler = GlobalAssembler.from_framework_objects(
    problems=problems,
    global_discretization=global_disc,
    static_condensations=static_condensations,
    constraint_manager=constraint_manager
)

print(f"Created assembler: {assembler}")
\end{lstlisting}

\subsubsection{Internal Initialization Methods}

\paragraph{\_compute\_dof\_structure()}\leavevmode
\begin{lstlisting}[language=Python, caption=Compute DOF Structure Method]
def _compute_dof_structure(self)
\end{lstlisting}

\textbf{Purpose:} Compute global DOF structure from domain data

\textbf{Side Effects:} Sets all DOF-related attributes

\textbf{Algorithm:}
\begin{enumerate}
    \item For each domain: compute \texttt{trace\_size = neq * (n\_elements + 1)}
    \item Compute cumulative offsets for global vector assembly
    \item Calculate total trace DOFs and multipliers
    \item Set total DOF count
\end{enumerate}

\subsection{Primary Assembly Methods}
\label{subsec:primary_assembly_methods}

\paragraph{assemble\_residual\_and\_jacobian()}\leavevmode
\begin{lstlisting}[language=Python, caption=Assemble Residual and Jacobian Method]
def assemble_residual_and_jacobian(self, 
                                 global_solution: np.ndarray,
                                 forcing_terms: List[np.ndarray],
                                 static_condensations: List,
                                 time: float) -> Tuple[np.ndarray, np.ndarray]
\end{lstlisting}

\textbf{Parameters:}
\begin{itemize}
    \item \texttt{global\_solution}: Global solution vector \texttt{[trace\_solutions; multipliers]}
    \item \texttt{forcing\_terms}: List of forcing term arrays for each domain (pre-computed)
    \item \texttt{static\_condensations}: List of static condensation objects for flux jump computation
    \item \texttt{time}: Current time for constraint evaluation
\end{itemize}

\textbf{Returns:} \texttt{Tuple[np.ndarray, np.ndarray]} - (residual, jacobian) global arrays

\textbf{Purpose:} Assemble global residual and Jacobian from domain flux jumps and constraints

\textbf{Mathematical Formulation:} Solves the nonlinear system $F(U; F_{ext}) = 0$ where:
\begin{itemize}
    \item $U$ is the trace solution
    \item $F_{ext}$ are the forcing terms (pre-computed as $dt \cdot f + M \cdot u_{old}$)
\end{itemize}

\textbf{Algorithm:}
\begin{enumerate}
    \item Extract trace solutions and multipliers from global solution
    \item Initialize global residual and Jacobian arrays
    \item For each domain: compute flux jump using \texttt{domain\_flux\_jump()}
    \item Add domain contributions to global arrays
    \item Add constraint contributions (multiplier coupling and constraint residuals)
    \item Add constraint Jacobian contributions
\end{enumerate}

\textbf{Usage:}
\begin{lstlisting}[language=Python, caption=Residual and Jacobian Assembly Usage]
# Prepare inputs
global_solution = np.random.rand(assembler.total_dofs)
forcing_terms = [np.random.rand(2*neq, n_elements) for _ in range(n_domains)]

# Assemble system
residual, jacobian = assembler.assemble_residual_and_jacobian(
    global_solution=global_solution,
    forcing_terms=forcing_terms,
    static_condensations=static_condensations,
    time=0.5
)

print(f"System assembled:")
print(f"  Residual shape: {residual.shape}")
print(f"  Jacobian shape: {jacobian.shape}")
print(f"  Residual norm: {np.linalg.norm(residual):.6e}")
print(f"  Jacobian condition: {np.linalg.cond(jacobian):.2e}")
\end{lstlisting}

\paragraph{bulk\_by\_static\_condensation()}
\leavevmode
\begin{lstlisting}[language=Python, caption=Bulk by Static Condensation Method]
def bulk_by_static_condensation(self, 
                               global_solution: np.ndarray,
                               forcing_terms: List[np.ndarray],
                               static_condensations: List,
                               time: float) -> Tuple[np.ndarray, np.ndarray]
\end{lstlisting}

\textbf{Parameters:} Same as \texttt{assemble\_residual\_and\_jacobian()}

\textbf{Returns:} \texttt{List[np.ndarray]} - List of bulk solution arrays for each domain

\textbf{Purpose:} Compute bulk solutions from trace solutions using static condensation

\textbf{Usage:}
\begin{lstlisting}[language=Python, caption=Bulk Solution Computation Usage]
# Compute bulk solutions only
bulk_solutions = assembler.bulk_by_static_condensation(
    global_solution=global_solution,
    forcing_terms=forcing_terms,
    static_condensations=static_condensations,
    time=0.5
)

for i, bulk_sol in enumerate(bulk_solutions):
    print(f"Domain {i} bulk solution shape: {bulk_sol.shape}")
\end{lstlisting}

\subsubsection{Forcing Term Computation}

\paragraph{compute\_forcing\_terms()}\leavevmode
\begin{lstlisting}[language=Python, caption=Compute Forcing Terms Method]
def compute_forcing_terms(self,
                        bulk_data_list: List[BulkData],
                        problems: List,
                        discretizations: List,
                        time: float,
                        dt: float) -> List[np.ndarray]
\end{lstlisting}

\textbf{Parameters:}
\begin{itemize}
    \item \texttt{bulk\_data\_list}: List of BulkData objects from previous time step
    \item \texttt{problems}: List of Problem objects
    \item \texttt{discretizations}: List of discretization objects
    \item \texttt{time}: Current time
    \item \texttt{dt}: Time step size
\end{itemize}

\textbf{Returns:} \texttt{List[np.ndarray]} - Forcing term arrays for implicit Euler

\textbf{Computation:} For each domain: $\text{forcing\_term} = dt \cdot f + M \cdot u_{old}$

\textbf{Usage:}
\begin{lstlisting}[language=Python, caption=Forcing Terms Usage]
# Previous time step solutions
bulk_data_list = assembler.initialize_bulk_data(problems, discretizations, time=0.0)

# Compute forcing terms for next time step
forcing_terms = assembler.compute_forcing_terms(
    bulk_data_list=bulk_data_list,
    problems=problems,
    discretizations=discretizations,
    time=0.1,
    dt=0.01
)

print(f"Computed {len(forcing_terms)} forcing term arrays")
for i, ft in enumerate(forcing_terms):
    print(f"  Domain {i}: {ft.shape}")
\end{lstlisting}

\subsection{Solution Vector Management}
\label{subsec:solution_vector_management}

\paragraph{\_extract\_trace\_solutions()}\leavevmode
\begin{lstlisting}[language=Python, caption=Extract Trace Solutions Method]
def _extract_trace_solutions(self, global_solution: np.ndarray) -> List[np.ndarray]
\end{lstlisting}

\textbf{Parameters:}
\begin{itemize}
    \item \texttt{global\_solution}: Global solution vector
\end{itemize}

\textbf{Returns:} \texttt{List[np.ndarray]} - Individual domain trace solutions

\textbf{Purpose:} Extract domain-specific trace solutions from global vector

\paragraph{get\_domain\_solutions()}\leavevmode
\begin{lstlisting}[language=Python, caption=Get Domain Solutions Method]
def get_domain_solutions(self, global_solution: np.ndarray) -> List[np.ndarray]
\end{lstlisting}

\textbf{Parameters:}
\begin{itemize}
    \item \texttt{global\_solution}: Global solution vector
\end{itemize}

\textbf{Returns:} \texttt{List[np.ndarray]} - Domain trace solutions (public interface)

\textbf{Usage:}
\begin{lstlisting}[language=Python, caption=Solution Extraction Usage]
# Extract individual domain solutions
domain_solutions = assembler.get_domain_solutions(global_solution)

for i, domain_sol in enumerate(domain_solutions):
    print(f"Domain {i} solution shape: {domain_sol.shape}")
    print(f"  Solution range: [{np.min(domain_sol):.6e}, {np.max(domain_sol):.6e}]")
\end{lstlisting}

\paragraph{get\_multipliers()}\leavevmode
\begin{lstlisting}[language=Python, caption=Get Multipliers Method]
def get_multipliers(self, global_solution: np.ndarray) -> np.ndarray
\end{lstlisting}

\textbf{Parameters:}
\begin{itemize}
    \item \texttt{global\_solution}: Global solution vector
\end{itemize}

\textbf{Returns:} \texttt{np.ndarray} - Constraint multipliers (empty if no constraints)

\textbf{Usage:}
\begin{lstlisting}[language=Python, caption=Multiplier Extraction Usage]
multipliers = assembler.get_multipliers(global_solution)
if len(multipliers) > 0:
    print(f"Constraint multipliers: {multipliers}")
else:
    print("No constraints defined")
\end{lstlisting}

\subsection{Initial Guess Creation}
\label{subsec:initial_guess_creation}

\paragraph{create\_initial\_guess\_from\_bulk\_data()}\leavevmode
\begin{lstlisting}[language=Python, caption=Initial Guess from BulkData Method]
def create_initial_guess_from_bulk_data(self, 
                                       bulk_data_list: List[BulkData]) -> np.ndarray
\end{lstlisting}

\textbf{Parameters:}
\begin{itemize}
    \item \texttt{bulk\_data\_list}: List of BulkData objects
\end{itemize}

\textbf{Returns:} \texttt{np.ndarray} - Initial guess for global solution vector

\textbf{Purpose:} Create initial guess from existing BulkData trace values

\textbf{Usage:}
\begin{lstlisting}[language=Python, caption=BulkData Initial Guess Usage]
# Initialize BulkData from problems
bulk_data_list = assembler.initialize_bulk_data(problems, discretizations, time=0.0)

# Create initial guess
initial_guess = assembler.create_initial_guess_from_bulk_data(bulk_data_list)
print(f"Initial guess shape: {initial_guess.shape}")
\end{lstlisting}

\paragraph{create\_initial\_guess\_from\_problems()}\leavevmode
\begin{lstlisting}[language=Python, caption=Initial Guess from Problems Method]
def create_initial_guess_from_problems(self, 
                                     problems: List,
                                     discretizations: List,
                                     time: float = 0.0) -> np.ndarray
\end{lstlisting}

\textbf{Parameters:}
\begin{itemize}
    \item \texttt{problems}: List of Problem objects
    \item \texttt{discretizations}: List of discretization objects
    \item \texttt{time}: Time for initial condition evaluation (default: 0.0)
\end{itemize}

\textbf{Returns:} \texttt{np.ndarray} - Initial guess for global solution vector

\textbf{Purpose:} Create initial guess directly from problem initial conditions

\textbf{Algorithm:}
\begin{enumerate}
    \item Initialize global solution vector with zeros
    \item For each domain and equation: evaluate initial condition at mesh nodes
    \item Fill corresponding entries in global vector
    \item Set multipliers to zero
\end{enumerate}

\textbf{Usage:}
\begin{lstlisting}[language=Python, caption=Problem Initial Guess Usage]
# Create initial guess directly from problem definitions
initial_guess = assembler.create_initial_guess_from_problems(
    problems=problems,
    discretizations=discretizations,
    time=0.0
)

print(f"Initial guess from problems: {initial_guess.shape}")
print(f"Non-zero entries: {np.count_nonzero(initial_guess)}")
\end{lstlisting}

\subsection{Constraint Integration}
\label{subsec:constraint_integration}

\paragraph{\_add\_constraint\_jacobian\_contributions()}\leavevmode
\begin{lstlisting}[language=Python, caption=Add Constraint Jacobian Contributions Method]
def _add_constraint_jacobian_contributions(self, 
                                         jacobian: np.ndarray,
                                         trace_solutions: List[np.ndarray],
                                         multipliers: np.ndarray,
                                         time: float)
\end{lstlisting}

\textbf{Parameters:}
\begin{itemize}
    \item \texttt{jacobian}: Global Jacobian matrix to modify
    \item \texttt{trace\_solutions}: List of trace solution arrays
    \item \texttt{multipliers}: Array of constraint multipliers
    \item \texttt{time}: Current time for constraint evaluation
\end{itemize}

\textbf{Returns:} \texttt{None}

\textbf{Side Effects:} Modifies the global Jacobian matrix in place

\textbf{Purpose:} Add constraint Jacobian contributions for different constraint types

\textbf{Constraint Types Handled:}
\begin{itemize}
    \item \textbf{Dirichlet}: $\frac{\partial}{\partial u}(u - g) = 1$
    \item \textbf{Neumann}: $\frac{\partial}{\partial \text{flux}}(\text{flux} - g) = 1$
    \item \textbf{Trace Continuity}: $\frac{\partial}{\partial u_1}(u_1 - u_2) = 1$, $\frac{\partial}{\partial u_2}(u_1 - u_2) = -1$
    \item \textbf{Kedem-Katchalsky}: $\frac{\partial}{\partial u_1}(\text{flux} + P(u_1 - u_2)) = P$
\end{itemize}

\subsection{BulkData Integration}
\label{subsec_bulk_data_integration}

\paragraph{initialize\_bulk\_data()}\leavevmode
\begin{lstlisting}[language=Python, caption=Initialize BulkData Method]
def initialize_bulk_data(self, 
                       problems: List,
                       discretizations: List,
                       time: float = 0.0) -> List[BulkData]
\end{lstlisting}

\textbf{Parameters:}
\begin{itemize}
    \item \texttt{problems}: List of Problem objects
    \item \texttt{discretizations}: List of discretization objects
    \item \texttt{time}: Initial time (default: 0.0)
\end{itemize}

\textbf{Returns:} \texttt{List[BulkData]} - Initialized BulkData objects for all domains

\textbf{Usage:}
\begin{lstlisting}[language=Python, caption=BulkData Initialization Usage]
# Initialize BulkData objects
bulk_data_list = assembler.initialize_bulk_data(
    problems=problems,
    discretizations=discretizations,
    time=0.0
)

print(f"Initialized {len(bulk_data_list)} BulkData objects")
for i, bd in enumerate(bulk_data_list):
    print(f"  Domain {i}: {bd}")
\end{lstlisting}

\paragraph{compute\_mass\_conservation()}
\leavevmode
\begin{lstlisting}[language=Python, caption=Compute Mass Conservation Method]
def compute_mass_conservation(self, bulk_data_list: List[BulkData]) -> float
\end{lstlisting}

\textbf{Parameters:}
\begin{itemize}
    \item \texttt{bulk\_data\_list}: List of BulkData instances
\end{itemize}

\textbf{Returns:} \texttt{float} - Total mass across all domains

\textbf{Usage:}
\begin{lstlisting}[language=Python, caption=Mass Conservation Usage]
# Monitor mass conservation
initial_bulk_data = assembler.initialize_bulk_data(problems, discretizations, time=0.0)
initial_mass = assembler.compute_mass_conservation(initial_bulk_data)

# After time evolution...
current_mass = assembler.compute_mass_conservation(current_bulk_data)
mass_change = abs(current_mass - initial_mass) / initial_mass

print(f"Mass conservation check:")
print(f"  Initial mass: {initial_mass:.6e}")
print(f"  Current mass: {current_mass:.6e}")
print(f"  Relative change: {mass_change:.6e}")
\end{lstlisting}

\subsection{Utility Methods}
\label{subsec:utility_methods}

\paragraph{get\_num\_domains()}\leavevmode
\begin{lstlisting}[language=Python, caption=Get Number of Domains Method]
def get_num_domains(self) -> int
\end{lstlisting}

\textbf{Returns:} \texttt{int} - Number of domains in the system

\paragraph{get\_domain\_info()}\leavevmode
\begin{lstlisting}[language=Python, caption=Get Domain Info Method]
def get_domain_info(self, domain_idx: int)
\end{lstlisting}

\textbf{Parameters:}
\begin{itemize}
    \item \texttt{domain\_idx}: Domain index
\end{itemize}

\textbf{Returns:} \texttt{DomainData} - Domain data object for inspection

\textbf{Usage:}
\begin{lstlisting}[language=Python, caption=Domain Info Usage]
for i in range(assembler.get_num_domains()):
    domain_info = assembler.get_domain_info(i)
    print(f"Domain {i}:")
    print(f"  Equations: {domain_info.neq}")
    print(f"  Elements: {domain_info.n_elements}")
    print(f"  Element length: {domain_info.element_length}")
\end{lstlisting}

\subsection{Testing Infrastructure}
\label{subsec:testing_infrastructure}

\paragraph{test()}\leavevmode
\begin{lstlisting}[language=Python, caption=Test Method]
def test(self, 
         problems: List = None,
         discretizations: List = None,
         static_condensations: List = None) -> bool
\end{lstlisting}

\textbf{Parameters:}
\begin{itemize}
    \item \texttt{problems}: List of Problem objects for testing (optional)
    \item \texttt{discretizations}: List of discretization objects for testing (optional)
    \item \texttt{static\_condensations}: List of static condensation objects for testing (optional)
\end{itemize}

\textbf{Returns:} \texttt{bool} - True if all tests pass, False otherwise

\textbf{Test Suite:}
\begin{enumerate}
    \item \textbf{Parameter Validation}: Check list lengths match and equal \texttt{n\_domains}
    \item \textbf{BulkDataManager Test}: Validate underlying BulkDataManager
    \item \textbf{DOF Structure Validation}: Check trace DOF counts and offsets
    \item \textbf{Initial Guess Tests}: Test BulkData-based and problem-based creation
    \item \textbf{Forcing Term Tests}: Validate forcing term computation
    \item \textbf{Assembly Tests}: Test residual and Jacobian assembly
    \item \textbf{Solution Extraction Tests}: Validate domain solution and multiplier extraction
    \item \textbf{Mass Conservation Tests}: Check mass computation
    \item \textbf{Constraint Tests}: Validate constraint handling (if constraints exist)
    \item \textbf{Factory Method Tests}: Test creation from framework objects
\end{enumerate}

\textbf{Sample Test Output:}
\begin{lstlisting}[language=Python, caption=Sample Test Output]
Testing LeanGlobalAssembler with 3 domains
PASS: Parameter list lengths validated (3 domains)
PASS: BulkDataManager test passed
PASS: DOF structure validated (trace=84, multipliers=2)
PASS: Domain offsets and sizes validated
PASS: Initial guess creation tests passed
PASS: Forcing term computation test passed
PASS: Residual and Jacobian assembly test passed
  Residual range: [-1.234567e-02, 2.345678e-02]
  Jacobian range: [-5.432109e-01, 6.543210e-01]
PASS: Zero forcing terms test passed
  Zero residual range: [-9.876543e-03, 1.234567e-02]
PASS: Solution extraction tests passed
PASS: Mass conservation test passed (total_mass=2.345678e+00)
PASS: Constraint handling test passed
PASS: Factory method test passed
✓ All LeanGlobalAssembler tests passed!
\end{lstlisting}

\textbf{Usage:}
\begin{lstlisting}[language=Python, caption=Test Usage]
# Comprehensive testing
if assembler.test(problems, discretizations, static_condensations):
    print("✓ LeanGlobalAssembler fully validated")
else:
    print("✗ LeanGlobalAssembler validation failed")

# Basic testing without framework objects
if assembler.test():
    print("✓ Basic structure validation passed")
\end{lstlisting}

\subsection{Special Methods}
\label{subsec:special_methods}

\paragraph{\_\_str\_\_()}\leavevmode
\begin{lstlisting}[language=Python, caption=String Representation Method]
def __str__(self) -> str
\end{lstlisting}

\textbf{Returns:} \texttt{str} - Human-readable representation

\textbf{Format:} \texttt{"LeanGlobalAssembler(domains=N, trace\_dofs=M, multipliers=K, total\_dofs=T)"}

\paragraph{\_\_repr\_\_()}\leavevmode
\begin{lstlisting}[language=Python, caption=Repr Method]
def __repr__(self) -> str
\end{lstlisting}

\textbf{Returns:} \texttt{str} - Developer-oriented representation

\textbf{Format:} \texttt{"LeanGlobalAssembler(n\_domains=N, domain\_trace\_sizes=[...], n\_multipliers=K)"}

\textbf{Usage:}
\begin{lstlisting}[language=Python, caption=String Methods Usage]
print(str(assembler))
# Output: LeanGlobalAssembler(domains=3, trace_dofs=84, multipliers=2, total_dofs=86)

print(repr(assembler))
# Output: LeanGlobalAssembler(n_domains=3, domain_trace_sizes=[28, 28, 28], n_multipliers=2)
\end{lstlisting}

\subsection{Complete Usage Examples}
\label{subsec:complete_usage_examples}

\subsubsection{Complete Newton Solver Integration}

\begin{lstlisting}[language=Python, caption=Complete Newton Solver Integration]
from ooc1d.core.lean_global_assembly import GlobalAssembler
from ooc1d.core.problem import Problem
from ooc1d.core.discretization import Discretization, GlobalDiscretization
from ooc1d.core.constraints import ConstraintManager
from ooc1d.core.lean_bulk_data_manager import BulkDataManager
import numpy as np

def newton_solve_with_lean_assembler(problems, discretizations, static_condensations):
    """Complete Newton solver using lean global assembler."""
    
    # Step 1: Create constraints
    constraint_manager = ConstraintManager()
    constraint_manager.add_dirichlet(0, 0, 0.0, lambda t: np.sin(t))
    constraint_manager.add_neumann(0, -1, problems[-1].domain_end, lambda t: 0.0)
    constraint_manager.map_to_discretizations(discretizations)
    
    # Step 2: Create lean assembler using factory method
    global_disc = GlobalDiscretization(discretizations)
    assembler = GlobalAssembler.from_framework_objects(
        problems=problems,
        global_discretization=global_disc,
        static_condensations=static_condensations,
        constraint_manager=constraint_manager
    )
    
    print(f"Created assembler: {assembler}")
    
    # Step 3: Initialize solution
    initial_guess = assembler.create_initial_guess_from_problems(
        problems, discretizations, time=0.0
    )
    
    # Step 4: Newton iteration
    solution = initial_guess.copy()
    tolerance = 1e-10
    max_iterations = 20
    
    print("Starting Newton iterations...")
    
    for iteration in range(max_iterations):
        # Initialize bulk data for forcing terms
        bulk_data_list = assembler.initialize_bulk_data(problems, discretizations, time=0.0)
        
        # Compute forcing terms (for time-dependent case)
        dt = 0.01
        current_time = iteration * dt
        forcing_terms = assembler.compute_forcing_terms(
            bulk_data_list, problems, discretizations, current_time, dt
        )
        
        # Assemble system
        residual, jacobian = assembler.assemble_residual_and_jacobian(
            global_solution=solution,
            forcing_terms=forcing_terms,
            static_condensations=static_condensations,
            time=current_time
        )
        
        # Check convergence
        residual_norm = np.linalg.norm(residual)
        print(f"  Iteration {iteration}: ||R|| = {residual_norm:.6e}")
        
        if residual_norm < tolerance:
            print("✓ Newton solver converged")
            break
        
        # Newton update
        try:
            delta = np.linalg.solve(jacobian, -residual)
            solution += delta
        except np.linalg.LinAlgError:
            print("✗ Newton solver failed: singular Jacobian")
            break
    
    # Step 5: Extract solutions
    domain_solutions = assembler.get_domain_solutions(solution)
    multipliers = assembler.get_multipliers(solution)
    
    print(f"\nSolver completed:")
    print(f"  Final residual norm: {residual_norm:.6e}")
    print(f"  Domain solutions: {len(domain_solutions)}")
    print(f"  Constraint multipliers: {len(multipliers)}")
    
    return solution, domain_solutions, multipliers

# Usage
problems = [create_problem(i) for i in range(3)]  # User-defined
discretizations = [create_discretization(i) for i in range(3)]  # User-defined
static_condensations = [create_static_condensation(i) for i in range(3)]  # User-defined

solution, domain_sols, multipliers = newton_solve_with_lean_assembler(
    problems, discretizations, static_condensations
)
\end{lstlisting}

\subsubsection{Time Evolution with Lean Assembler}

\begin{lstlisting}[language=Python, caption=Time Evolution with Lean Assembler]
def time_evolution_with_lean_assembler(problems, discretizations, static_condensations):
    """Time evolution loop using lean global assembler."""
    
    # Setup
    constraint_manager = ConstraintManager()
    # Add constraints as needed...
    
    global_disc = GlobalDiscretization(discretizations)
    global_disc.set_time_parameters(dt=0.01, T=1.0)
    
    assembler = GlobalAssembler.from_framework_objects(
        problems, global_disc, static_condensations, constraint_manager
    )
    
    # Initial conditions
    current_solution = assembler.create_initial_guess_from_problems(
        problems, discretizations, time=0.0
    )
    
    # Time evolution
    dt = global_disc.dt
    time_points = global_disc.get_time_points()
    
    solution_history = [current_solution.copy()]
    mass_history = []
    
    print(f"Starting time evolution: {len(time_points)} time steps")
    
    for step, current_time in enumerate(time_points[1:], 1):
        print(f"Time step {step}: t = {current_time:.3f}")
        
        # Get current bulk data for forcing terms
        domain_solutions = assembler.get_domain_solutions(current_solution)
        bulk_data_list = []
        
        for i, domain_sol in enumerate(domain_solutions):
            bulk_data = assembler.bulk_manager.create_bulk_data(
                i, problems[i], discretizations[i], dual=False
            )
            bulk_data.set_data(domain_sol.reshape(-1, 1))
            bulk_data_list.append(bulk_data)
        
        # Compute forcing terms
        forcing_terms = assembler.compute_forcing_terms(
            bulk_data_list, problems, discretizations, current_time, dt
        )
        
        # Newton solve for this time step
        tolerance = 1e-12
        max_newton_iterations = 10
        
        for newton_iter in range(max_newton_iterations):
            residual, jacobian = assembler.assemble_residual_and_jacobian(
                current_solution, forcing_terms, static_condensations, current_time
            )
            
            residual_norm = np.linalg.norm(residual)
            if residual_norm < tolerance:
                break
            
            delta = np.linalg.solve(jacobian, -residual)
            current_solution += delta
        
        # Monitor mass conservation
        current_mass = assembler.compute_mass_conservation(bulk_data_list)
        mass_history.append(current_mass)
        
        # Store solution
        solution_history.append(current_solution.copy())
        
        if step % 10 == 0:
            print(f"  Mass: {current_mass:.6e}, Newton iterations: {newton_iter + 1}")
    
    print("✓ Time evolution completed")
    
    # Mass conservation check
    initial_mass = mass_history[0] if mass_history else 0.0
    final_mass = mass_history[-1] if mass_history else 0.0
    mass_change = abs(final_mass - initial_mass) / initial_mass if initial_mass != 0 else 0.0
    
    print(f"Mass conservation: {mass_change:.6e} relative change")
    
    return solution_history, mass_history

# Usage
solution_history, mass_history = time_evolution_with_lean_assembler(
    problems, discretizations, static_condensations
)
\end{lstlisting}

\subsection{Method Summary Table}
\label{subsec:lean_assembler_method_summary}

\begin{longtable}{|p{7cm}|p{3cm}|p{4.8cm}|}
\hline
\textbf{Method} & \textbf{Returns} & \textbf{Purpose} \\
\hline
\endhead

\texttt{\_\_init\_\_} & \texttt{None} & Initialize lean assembler with domain data \\
\hline

\texttt{from\_framework\_objects} & \texttt{GlobalAssembler} & Factory method for creation from framework objects \\
\hline

\texttt{assemble\_residual\_and\_jacobian} & \texttt{Tuple} & Global system assembly for Newton solver \\
\hline

\texttt{bulk\_by\_static\_condensation} & \texttt{List} & Bulk solution computation from traces \\
\hline

\texttt{compute\_forcing\_terms} & \texttt{List} & Forcing term computation for implicit Euler \\
\hline

\texttt{create\_initial\_guess\_from\_bulk\_data} & \texttt{np.ndarray} & Initial guess from BulkData objects \\
\hline

\texttt{create\_initial\_guess\_from\_problems} & \texttt{np.ndarray} & Initial guess from problem definitions \\
\hline

\texttt{get\_domain\_solutions} & \texttt{List} & Extract domain solutions from global vector \\
\hline

\texttt{get\_multipliers} & \texttt{np.ndarray} & Extract constraint multipliers \\
\hline

\texttt{initialize\_bulk\_data} & \texttt{List} & Initialize BulkData objects from problems \\
\hline

\texttt{compute\_mass\_conservation} & \texttt{float} & Calculate total mass for conservation check \\
\hline

\texttt{get\_num\_domains} & \texttt{int} & Get number of domains in system \\
\hline

\texttt{get\_domain\_info} & \texttt{DomainData} & Access domain data for inspection \\
\hline

\texttt{test} & \texttt{bool} & Comprehensive validation and testing \\
\hline

\end{longtable}

This documentation provides an exact reference for the lean global assembly module, emphasizing its memory-efficient design, integration with the lean BulkDataManager, and comprehensive constraint handling capabilities for multi-domain HDG systems.

% End of lean global assembly module API documentation


\section{Lean Global Assembly Module API Reference}
\label{sec:lean_global_assembly_api}

This section provides an exact reference for the lean global assembly module (\texttt{ooc1d.core.lean\_global\_assembly.GlobalAssembler}) based on detailed analysis of the actual implementation. This class provides global assembly functionality using the lean BulkDataManager approach where framework objects are passed as parameters.

\subsection{Module Overview}

The lean global assembly module provides:
\begin{itemize}
    \item Global residual and Jacobian assembly from domain flux jumps
    \item Constraint integration for boundary and junction conditions
    \item Memory-efficient operation using lean BulkDataManager
    \item DOF management for multi-domain systems
    \item Comprehensive testing infrastructure
\end{itemize}

\subsection{Module Imports and Dependencies}

\begin{lstlisting}[language=Python, caption=Module Dependencies]
import numpy as np
from typing import List, Tuple, Optional

from .discretization import GlobalDiscretization
from .flux_jump import domain_flux_jump
from .constraints import ConstraintManager
from .lean_bulk_data_manager import BulkDataManager
from .bulk_data import BulkData
\end{lstlisting}

\subsection{GlobalAssembler Class}
\label{subsec:lean_global_assembler_class}

Main class for lean global assembly that uses external framework objects as parameters.

\subsubsection{Constructor}

\paragraph{\_\_init\_\_()}\leavevmode
\begin{lstlisting}[language=Python, caption=GlobalAssembler Constructor]
def __init__(self, 
             domain_data_list: List,
             constraint_manager: Optional[ConstraintManager] = None)
\end{lstlisting}

\textbf{Parameters:}
\begin{itemize}
    \item \texttt{domain\_data\_list}: List of DomainData objects with essential information
    \item \texttt{constraint\_manager}: Optional constraint manager for boundary/junction conditions
\end{itemize}

\textbf{Side Effects:}
\begin{itemize}
    \item Creates internal BulkDataManager with domain data
    \item Computes DOF structure and domain offsets
    \item Stores constraint manager reference
\end{itemize}

\textbf{Usage:}
\begin{lstlisting}[language=Python, caption=Constructor Usage]
# Extract domain data first
domain_data_list = BulkDataManager.extract_domain_data_list(
    problems, discretizations, static_condensations
)

# Create constraint manager
constraint_manager = ConstraintManager()
constraint_manager.add_dirichlet(0, 0, 0.0)
constraint_manager.map_to_discretizations(discretizations)

# Create lean assembler
assembler = GlobalAssembler(domain_data_list, constraint_manager)
\end{lstlisting}

\subsubsection{Core Attributes}

\begin{longtable}{|p{3.5cm}|p{2.5cm}|p{7cm}|}
\hline
\textbf{Attribute} & \textbf{Type} & \textbf{Description} \\
\hline
\endhead

\texttt{bulk\_manager} & \texttt{BulkDataManager} & Lean bulk data manager instance \\
\hline

\texttt{constraint\_manager} & \texttt{Optional[ConstraintManager]} & Constraint manager for boundary/junction conditions \\
\hline

\texttt{n\_domains} & \texttt{int} & Number of domains in the system \\
\hline

\texttt{domain\_trace\_sizes} & \texttt{List[int]} & Trace DOF count for each domain \\
\hline

\texttt{domain\_trace\_offsets} & \texttt{List[int]} & Starting indices for domain traces in global vector \\
\hline

\texttt{total\_trace\_dofs} & \texttt{int} & Total number of trace DOFs across all domains \\
\hline

\texttt{n\_multipliers} & \texttt{int} & Number of constraint multipliers \\
\hline

\texttt{total\_dofs} & \texttt{int} & Total DOFs: \texttt{total\_trace\_dofs + n\_multipliers} \\
\hline

\end{longtable}

\subsubsection{Factory Method}

\paragraph{from\_framework\_objects()}\leavevmode
\begin{lstlisting}[language=Python, caption=Factory Method]
@classmethod
def from_framework_objects(cls,
                          problems: List,
                          global_discretization,
                          static_condensations: List,
                          constraint_manager: Optional[ConstraintManager] = None)
\end{lstlisting}

\textbf{Parameters:}
\begin{itemize}
    \item \texttt{problems}: List of Problem instances for each domain
    \item \texttt{global\_discretization}: GlobalDiscretization instance
    \item \texttt{static\_condensations}: List of static condensation implementations
    \item \texttt{constraint\_manager}: Optional constraint manager
\end{itemize}

\textbf{Returns:} \texttt{GlobalAssembler} - Configured assembler instance

\textbf{Purpose:} Factory method to create assembler from framework objects

\textbf{Usage:}
\begin{lstlisting}[language=Python, caption=Factory Method Usage]
# Create assembler from framework objects
assembler = GlobalAssembler.from_framework_objects(
    problems=problems,
    global_discretization=global_disc,
    static_condensations=static_condensations,
    constraint_manager=constraint_manager
)

print(f"Created assembler: {assembler}")
\end{lstlisting}

\subsubsection{Internal Initialization Methods}

\paragraph{\_compute\_dof\_structure()}\leavevmode
\begin{lstlisting}[language=Python, caption=Compute DOF Structure Method]
def _compute_dof_structure(self)
\end{lstlisting}

\textbf{Purpose:} Compute global DOF structure from domain data

\textbf{Side Effects:} Sets all DOF-related attributes

\textbf{Algorithm:}
\begin{enumerate}
    \item For each domain: compute \texttt{trace\_size = neq * (n\_elements + 1)}
    \item Compute cumulative offsets for global vector assembly
    \item Calculate total trace DOFs and multipliers
    \item Set total DOF count
\end{enumerate}

\subsection{Primary Assembly Methods}
\label{subsec:primary_assembly_methods}

\paragraph{assemble\_residual\_and\_jacobian()}\leavevmode
\begin{lstlisting}[language=Python, caption=Assemble Residual and Jacobian Method]
def assemble_residual_and_jacobian(self, 
                                 global_solution: np.ndarray,
                                 forcing_terms: List[np.ndarray],
                                 static_condensations: List,
                                 time: float) -> Tuple[np.ndarray, np.ndarray]
\end{lstlisting}

\textbf{Parameters:}
\begin{itemize}
    \item \texttt{global\_solution}: Global solution vector \texttt{[trace\_solutions; multipliers]}
    \item \texttt{forcing\_terms}: List of forcing term arrays for each domain (pre-computed)
    \item \texttt{static\_condensations}: List of static condensation objects for flux jump computation
    \item \texttt{time}: Current time for constraint evaluation
\end{itemize}

\textbf{Returns:} \texttt{Tuple[np.ndarray, np.ndarray]} - (residual, jacobian) global arrays

\textbf{Purpose:} Assemble global residual and Jacobian from domain flux jumps and constraints

\textbf{Mathematical Formulation:} Solves the nonlinear system $F(U; F_{ext}) = 0$ where:
\begin{itemize}
    \item $U$ is the trace solution
    \item $F_{ext}$ are the forcing terms (pre-computed as $dt \cdot f + M \cdot u_{old}$)
\end{itemize}

\textbf{Algorithm:}
\begin{enumerate}
    \item Extract trace solutions and multipliers from global solution
    \item Initialize global residual and Jacobian arrays
    \item For each domain: compute flux jump using \texttt{domain\_flux\_jump()}
    \item Add domain contributions to global arrays
    \item Add constraint contributions (multiplier coupling and constraint residuals)
    \item Add constraint Jacobian contributions
\end{enumerate}

\textbf{Usage:}
\begin{lstlisting}[language=Python, caption=Residual and Jacobian Assembly Usage]
# Prepare inputs
global_solution = np.random.rand(assembler.total_dofs)
forcing_terms = [np.random.rand(2*neq, n_elements) for _ in range(n_domains)]

# Assemble system
residual, jacobian = assembler.assemble_residual_and_jacobian(
    global_solution=global_solution,
    forcing_terms=forcing_terms,
    static_condensations=static_condensations,
    time=0.5
)

print(f"System assembled:")
print(f"  Residual shape: {residual.shape}")
print(f"  Jacobian shape: {jacobian.shape}")
print(f"  Residual norm: {np.linalg.norm(residual):.6e}")
print(f"  Jacobian condition: {np.linalg.cond(jacobian):.2e}")
\end{lstlisting}

\paragraph{bulk\_by\_static\_condensation()}
\leavevmode
\begin{lstlisting}[language=Python, caption=Bulk by Static Condensation Method]
def bulk_by_static_condensation(self, 
                               global_solution: np.ndarray,
                               forcing_terms: List[np.ndarray],
                               static_condensations: List,
                               time: float) -> Tuple[np.ndarray, np.ndarray]
\end{lstlisting}

\textbf{Parameters:} Same as \texttt{assemble\_residual\_and\_jacobian()}

\textbf{Returns:} \texttt{List[np.ndarray]} - List of bulk solution arrays for each domain

\textbf{Purpose:} Compute bulk solutions from trace solutions using static condensation

\textbf{Usage:}
\begin{lstlisting}[language=Python, caption=Bulk Solution Computation Usage]
# Compute bulk solutions only
bulk_solutions = assembler.bulk_by_static_condensation(
    global_solution=global_solution,
    forcing_terms=forcing_terms,
    static_condensations=static_condensations,
    time=0.5
)

for i, bulk_sol in enumerate(bulk_solutions):
    print(f"Domain {i} bulk solution shape: {bulk_sol.shape}")
\end{lstlisting}

\subsubsection{Forcing Term Computation}

\paragraph{compute\_forcing\_terms()}\leavevmode
\begin{lstlisting}[language=Python, caption=Compute Forcing Terms Method]
def compute_forcing_terms(self,
                        bulk_data_list: List[BulkData],
                        problems: List,
                        discretizations: List,
                        time: float,
                        dt: float) -> List[np.ndarray]
\end{lstlisting}

\textbf{Parameters:}
\begin{itemize}
    \item \texttt{bulk\_data\_list}: List of BulkData objects from previous time step
    \item \texttt{problems}: List of Problem objects
    \item \texttt{discretizations}: List of discretization objects
    \item \texttt{time}: Current time
    \item \texttt{dt}: Time step size
\end{itemize}

\textbf{Returns:} \texttt{List[np.ndarray]} - Forcing term arrays for implicit Euler

\textbf{Computation:} For each domain: $\text{forcing\_term} = dt \cdot f + M \cdot u_{old}$

\textbf{Usage:}
\begin{lstlisting}[language=Python, caption=Forcing Terms Usage]
# Previous time step solutions
bulk_data_list = assembler.initialize_bulk_data(problems, discretizations, time=0.0)

# Compute forcing terms for next time step
forcing_terms = assembler.compute_forcing_terms(
    bulk_data_list=bulk_data_list,
    problems=problems,
    discretizations=discretizations,
    time=0.1,
    dt=0.01
)

print(f"Computed {len(forcing_terms)} forcing term arrays")
for i, ft in enumerate(forcing_terms):
    print(f"  Domain {i}: {ft.shape}")
\end{lstlisting}

\subsection{Solution Vector Management}
\label{subsec:solution_vector_management}

\paragraph{\_extract\_trace\_solutions()}\leavevmode
\begin{lstlisting}[language=Python, caption=Extract Trace Solutions Method]
def _extract_trace_solutions(self, global_solution: np.ndarray) -> List[np.ndarray]
\end{lstlisting}

\textbf{Parameters:}
\begin{itemize}
    \item \texttt{global\_solution}: Global solution vector
\end{itemize}

\textbf{Returns:} \texttt{List[np.ndarray]} - Individual domain trace solutions

\textbf{Purpose:} Extract domain-specific trace solutions from global vector

\paragraph{get\_domain\_solutions()}\leavevmode
\begin{lstlisting}[language=Python, caption=Get Domain Solutions Method]
def get_domain_solutions(self, global_solution: np.ndarray) -> List[np.ndarray]
\end{lstlisting}

\textbf{Parameters:}
\begin{itemize}
    \item \texttt{global\_solution}: Global solution vector
\end{itemize}

\textbf{Returns:} \texttt{List[np.ndarray]} - Domain trace solutions (public interface)

\textbf{Usage:}
\begin{lstlisting}[language=Python, caption=Solution Extraction Usage]
# Extract individual domain solutions
domain_solutions = assembler.get_domain_solutions(global_solution)

for i, domain_sol in enumerate(domain_solutions):
    print(f"Domain {i} solution shape: {domain_sol.shape}")
    print(f"  Solution range: [{np.min(domain_sol):.6e}, {np.max(domain_sol):.6e}]")
\end{lstlisting}

\paragraph{get\_multipliers()}\leavevmode
\begin{lstlisting}[language=Python, caption=Get Multipliers Method]
def get_multipliers(self, global_solution: np.ndarray) -> np.ndarray
\end{lstlisting}

\textbf{Parameters:}
\begin{itemize}
    \item \texttt{global\_solution}: Global solution vector
\end{itemize}

\textbf{Returns:} \texttt{np.ndarray} - Constraint multipliers (empty if no constraints)

\textbf{Usage:}
\begin{lstlisting}[language=Python, caption=Multiplier Extraction Usage]
multipliers = assembler.get_multipliers(global_solution)
if len(multipliers) > 0:
    print(f"Constraint multipliers: {multipliers}")
else:
    print("No constraints defined")
\end{lstlisting}

\subsection{Initial Guess Creation}
\label{subsec:initial_guess_creation}

\paragraph{create\_initial\_guess\_from\_bulk\_data()}\leavevmode
\begin{lstlisting}[language=Python, caption=Initial Guess from BulkData Method]
def create_initial_guess_from_bulk_data(self, 
                                       bulk_data_list: List[BulkData]) -> np.ndarray
\end{lstlisting}

\textbf{Parameters:}
\begin{itemize}
    \item \texttt{bulk\_data\_list}: List of BulkData objects
\end{itemize}

\textbf{Returns:} \texttt{np.ndarray} - Initial guess for global solution vector

\textbf{Purpose:} Create initial guess from existing BulkData trace values

\textbf{Usage:}
\begin{lstlisting}[language=Python, caption=BulkData Initial Guess Usage]
# Initialize BulkData from problems
bulk_data_list = assembler.initialize_bulk_data(problems, discretizations, time=0.0)

# Create initial guess
initial_guess = assembler.create_initial_guess_from_bulk_data(bulk_data_list)
print(f"Initial guess shape: {initial_guess.shape}")
\end{lstlisting}

\paragraph{create\_initial\_guess\_from\_problems()}\leavevmode
\begin{lstlisting}[language=Python, caption=Initial Guess from Problems Method]
def create_initial_guess_from_problems(self, 
                                     problems: List,
                                     discretizations: List,
                                     time: float = 0.0) -> np.ndarray
\end{lstlisting}

\textbf{Parameters:}
\begin{itemize}
    \item \texttt{problems}: List of Problem objects
    \item \texttt{discretizations}: List of discretization objects
    \item \texttt{time}: Time for initial condition evaluation (default: 0.0)
\end{itemize}

\textbf{Returns:} \texttt{np.ndarray} - Initial guess for global solution vector

\textbf{Purpose:} Create initial guess directly from problem initial conditions

\textbf{Algorithm:}
\begin{enumerate}
    \item Initialize global solution vector with zeros
    \item For each domain and equation: evaluate initial condition at mesh nodes
    \item Fill corresponding entries in global vector
    \item Set multipliers to zero
\end{enumerate}

\textbf{Usage:}
\begin{lstlisting}[language=Python, caption=Problem Initial Guess Usage]
# Create initial guess directly from problem definitions
initial_guess = assembler.create_initial_guess_from_problems(
    problems=problems,
    discretizations=discretizations,
    time=0.0
)

print(f"Initial guess from problems: {initial_guess.shape}")
print(f"Non-zero entries: {np.count_nonzero(initial_guess)}")
\end{lstlisting}

\subsection{Constraint Integration}
\label{subsec:constraint_integration}

\paragraph{\_add\_constraint\_jacobian\_contributions()}\leavevmode
\begin{lstlisting}[language=Python, caption=Add Constraint Jacobian Contributions Method]
def _add_constraint_jacobian_contributions(self, 
                                         jacobian: np.ndarray,
                                         trace_solutions: List[np.ndarray],
                                         multipliers: np.ndarray,
                                         time: float)
\end{lstlisting}

\textbf{Parameters:}
\begin{itemize}
    \item \texttt{jacobian}: Global Jacobian matrix to modify
    \item \texttt{trace\_solutions}: List of trace solution arrays
    \item \texttt{multipliers}: Array of constraint multipliers
    \item \texttt{time}: Current time for constraint evaluation
\end{itemize}

\textbf{Returns:} \texttt{None}

\textbf{Side Effects:} Modifies the global Jacobian matrix in place

\textbf{Purpose:} Add constraint Jacobian contributions for different constraint types

\textbf{Constraint Types Handled:}
\begin{itemize}
    \item \textbf{Dirichlet}: $\frac{\partial}{\partial u}(u - g) = 1$
    \item \textbf{Neumann}: $\frac{\partial}{\partial \text{flux}}(\text{flux} - g) = 1$
    \item \textbf{Trace Continuity}: $\frac{\partial}{\partial u_1}(u_1 - u_2) = 1$, $\frac{\partial}{\partial u_2}(u_1 - u_2) = -1$
    \item \textbf{Kedem-Katchalsky}: $\frac{\partial}{\partial u_1}(\text{flux} + P(u_1 - u_2)) = P$
\end{itemize}

\subsection{BulkData Integration}
\label{subsec_bulk_data_integration}

\paragraph{initialize\_bulk\_data()}\leavevmode
\begin{lstlisting}[language=Python, caption=Initialize BulkData Method]
def initialize_bulk_data(self, 
                       problems: List,
                       discretizations: List,
                       time: float = 0.0) -> List[BulkData]
\end{lstlisting}

\textbf{Parameters:}
\begin{itemize}
    \item \texttt{problems}: List of Problem objects
    \item \texttt{discretizations}: List of discretization objects
    \item \texttt{time}: Initial time (default: 0.0)
\end{itemize}

\textbf{Returns:} \texttt{List[BulkData]} - Initialized BulkData objects for all domains

\textbf{Usage:}
\begin{lstlisting}[language=Python, caption=BulkData Initialization Usage]
# Initialize BulkData objects
bulk_data_list = assembler.initialize_bulk_data(
    problems=problems,
    discretizations=discretizations,
    time=0.0
)

print(f"Initialized {len(bulk_data_list)} BulkData objects")
for i, bd in enumerate(bulk_data_list):
    print(f"  Domain {i}: {bd}")
\end{lstlisting}

\paragraph{compute\_mass\_conservation()}
\leavevmode
\begin{lstlisting}[language=Python, caption=Compute Mass Conservation Method]
def compute_mass_conservation(self, bulk_data_list: List[BulkData]) -> float
\end{lstlisting}

\textbf{Parameters:}
\begin{itemize}
    \item \texttt{bulk\_data\_list}: List of BulkData instances
\end{itemize}

\textbf{Returns:} \texttt{float} - Total mass across all domains

\textbf{Usage:}
\begin{lstlisting}[language=Python, caption=Mass Conservation Usage]
# Monitor mass conservation
initial_bulk_data = assembler.initialize_bulk_data(problems, discretizations, time=0.0)
initial_mass = assembler.compute_mass_conservation(initial_bulk_data)

# After time evolution...
current_mass = assembler.compute_mass_conservation(current_bulk_data)
mass_change = abs(current_mass - initial_mass) / initial_mass

print(f"Mass conservation check:")
print(f"  Initial mass: {initial_mass:.6e}")
print(f"  Current mass: {current_mass:.6e}")
print(f"  Relative change: {mass_change:.6e}")
\end{lstlisting}

\subsection{Utility Methods}
\label{subsec:utility_methods}

\paragraph{get\_num\_domains()}\leavevmode
\begin{lstlisting}[language=Python, caption=Get Number of Domains Method]
def get_num_domains(self) -> int
\end{lstlisting}

\textbf{Returns:} \texttt{int} - Number of domains in the system

\paragraph{get\_domain\_info()}\leavevmode
\begin{lstlisting}[language=Python, caption=Get Domain Info Method]
def get_domain_info(self, domain_idx: int)
\end{lstlisting}

\textbf{Parameters:}
\begin{itemize}
    \item \texttt{domain\_idx}: Domain index
\end{itemize}

\textbf{Returns:} \texttt{DomainData} - Domain data object for inspection

\textbf{Usage:}
\begin{lstlisting}[language=Python, caption=Domain Info Usage]
for i in range(assembler.get_num_domains()):
    domain_info = assembler.get_domain_info(i)
    print(f"Domain {i}:")
    print(f"  Equations: {domain_info.neq}")
    print(f"  Elements: {domain_info.n_elements}")
    print(f"  Element length: {domain_info.element_length}")
\end{lstlisting}

\subsection{Testing Infrastructure}
\label{subsec:testing_infrastructure}

\paragraph{test()}\leavevmode
\begin{lstlisting}[language=Python, caption=Test Method]
def test(self, 
         problems: List = None,
         discretizations: List = None,
         static_condensations: List = None) -> bool
\end{lstlisting}

\textbf{Parameters:}
\begin{itemize}
    \item \texttt{problems}: List of Problem objects for testing (optional)
    \item \texttt{discretizations}: List of discretization objects for testing (optional)
    \item \texttt{static\_condensations}: List of static condensation objects for testing (optional)
\end{itemize}

\textbf{Returns:} \texttt{bool} - True if all tests pass, False otherwise

\textbf{Test Suite:}
\begin{enumerate}
    \item \textbf{Parameter Validation}: Check list lengths match and equal \texttt{n\_domains}
    \item \textbf{BulkDataManager Test}: Validate underlying BulkDataManager
    \item \textbf{DOF Structure Validation}: Check trace DOF counts and offsets
    \item \textbf{Initial Guess Tests}: Test BulkData-based and problem-based creation
    \item \textbf{Forcing Term Tests}: Validate forcing term computation
    \item \textbf{Assembly Tests}: Test residual and Jacobian assembly
    \item \textbf{Solution Extraction Tests}: Validate domain solution and multiplier extraction
    \item \textbf{Mass Conservation Tests}: Check mass computation
    \item \textbf{Constraint Tests}: Validate constraint handling (if constraints exist)
    \item \textbf{Factory Method Tests}: Test creation from framework objects
\end{enumerate}

\textbf{Sample Test Output:}
\begin{lstlisting}[language=Python, caption=Sample Test Output]
Testing LeanGlobalAssembler with 3 domains
PASS: Parameter list lengths validated (3 domains)
PASS: BulkDataManager test passed
PASS: DOF structure validated (trace=84, multipliers=2)
PASS: Domain offsets and sizes validated
PASS: Initial guess creation tests passed
PASS: Forcing term computation test passed
PASS: Residual and Jacobian assembly test passed
  Residual range: [-1.234567e-02, 2.345678e-02]
  Jacobian range: [-5.432109e-01, 6.543210e-01]
PASS: Zero forcing terms test passed
  Zero residual range: [-9.876543e-03, 1.234567e-02]
PASS: Solution extraction tests passed
PASS: Mass conservation test passed (total_mass=2.345678e+00)
PASS: Constraint handling test passed
PASS: Factory method test passed
✓ All LeanGlobalAssembler tests passed!
\end{lstlisting}

\textbf{Usage:}
\begin{lstlisting}[language=Python, caption=Test Usage]
# Comprehensive testing
if assembler.test(problems, discretizations, static_condensations):
    print("✓ LeanGlobalAssembler fully validated")
else:
    print("✗ LeanGlobalAssembler validation failed")

# Basic testing without framework objects
if assembler.test():
    print("✓ Basic structure validation passed")
\end{lstlisting}

\subsection{Special Methods}
\label{subsec:special_methods}

\paragraph{\_\_str\_\_()}\leavevmode
\begin{lstlisting}[language=Python, caption=String Representation Method]
def __str__(self) -> str
\end{lstlisting}

\textbf{Returns:} \texttt{str} - Human-readable representation

\textbf{Format:} \texttt{"LeanGlobalAssembler(domains=N, trace\_dofs=M, multipliers=K, total\_dofs=T)"}

\paragraph{\_\_repr\_\_()}\leavevmode
\begin{lstlisting}[language=Python, caption=Repr Method]
def __repr__(self) -> str
\end{lstlisting}

\textbf{Returns:} \texttt{str} - Developer-oriented representation

\textbf{Format:} \texttt{"LeanGlobalAssembler(n\_domains=N, domain\_trace\_sizes=[...], n\_multipliers=K)"}

\textbf{Usage:}
\begin{lstlisting}[language=Python, caption=String Methods Usage]
print(str(assembler))
# Output: LeanGlobalAssembler(domains=3, trace_dofs=84, multipliers=2, total_dofs=86)

print(repr(assembler))
# Output: LeanGlobalAssembler(n_domains=3, domain_trace_sizes=[28, 28, 28], n_multipliers=2)
\end{lstlisting}

\subsection{Complete Usage Examples}
\label{subsec:complete_usage_examples}

\subsubsection{Complete Newton Solver Integration}

\begin{lstlisting}[language=Python, caption=Complete Newton Solver Integration]
from ooc1d.core.lean_global_assembly import GlobalAssembler
from ooc1d.core.problem import Problem
from ooc1d.core.discretization import Discretization, GlobalDiscretization
from ooc1d.core.constraints import ConstraintManager
from ooc1d.core.lean_bulk_data_manager import BulkDataManager
import numpy as np

def newton_solve_with_lean_assembler(problems, discretizations, static_condensations):
    """Complete Newton solver using lean global assembler."""
    
    # Step 1: Create constraints
    constraint_manager = ConstraintManager()
    constraint_manager.add_dirichlet(0, 0, 0.0, lambda t: np.sin(t))
    constraint_manager.add_neumann(0, -1, problems[-1].domain_end, lambda t: 0.0)
    constraint_manager.map_to_discretizations(discretizations)
    
    # Step 2: Create lean assembler using factory method
    global_disc = GlobalDiscretization(discretizations)
    assembler = GlobalAssembler.from_framework_objects(
        problems=problems,
        global_discretization=global_disc,
        static_condensations=static_condensations,
        constraint_manager=constraint_manager
    )
    
    print(f"Created assembler: {assembler}")
    
    # Step 3: Initialize solution
    initial_guess = assembler.create_initial_guess_from_problems(
        problems, discretizations, time=0.0
    )
    
    # Step 4: Newton iteration
    solution = initial_guess.copy()
    tolerance = 1e-10
    max_iterations = 20
    
    print("Starting Newton iterations...")
    
    for iteration in range(max_iterations):
        # Initialize bulk data for forcing terms
        bulk_data_list = assembler.initialize_bulk_data(problems, discretizations, time=0.0)
        
        # Compute forcing terms (for time-dependent case)
        dt = 0.01
        current_time = iteration * dt
        forcing_terms = assembler.compute_forcing_terms(
            bulk_data_list, problems, discretizations, current_time, dt
        )
        
        # Assemble system
        residual, jacobian = assembler.assemble_residual_and_jacobian(
            global_solution=solution,
            forcing_terms=forcing_terms,
            static_condensations=static_condensations,
            time=current_time
        )
        
        # Check convergence
        residual_norm = np.linalg.norm(residual)
        print(f"  Iteration {iteration}: ||R|| = {residual_norm:.6e}")
        
        if residual_norm < tolerance:
            print("✓ Newton solver converged")
            break
        
        # Newton update
        try:
            delta = np.linalg.solve(jacobian, -residual)
            solution += delta
        except np.linalg.LinAlgError:
            print("✗ Newton solver failed: singular Jacobian")
            break
    
    # Step 5: Extract solutions
    domain_solutions = assembler.get_domain_solutions(solution)
    multipliers = assembler.get_multipliers(solution)
    
    print(f"\nSolver completed:")
    print(f"  Final residual norm: {residual_norm:.6e}")
    print(f"  Domain solutions: {len(domain_solutions)}")
    print(f"  Constraint multipliers: {len(multipliers)}")
    
    return solution, domain_solutions, multipliers

# Usage
problems = [create_problem(i) for i in range(3)]  # User-defined
discretizations = [create_discretization(i) for i in range(3)]  # User-defined
static_condensations = [create_static_condensation(i) for i in range(3)]  # User-defined

solution, domain_sols, multipliers = newton_solve_with_lean_assembler(
    problems, discretizations, static_condensations
)
\end{lstlisting}

\subsubsection{Time Evolution with Lean Assembler}

\begin{lstlisting}[language=Python, caption=Time Evolution with Lean Assembler]
def time_evolution_with_lean_assembler(problems, discretizations, static_condensations):
    """Time evolution loop using lean global assembler."""
    
    # Setup
    constraint_manager = ConstraintManager()
    # Add constraints as needed...
    
    global_disc = GlobalDiscretization(discretizations)
    global_disc.set_time_parameters(dt=0.01, T=1.0)
    
    assembler = GlobalAssembler.from_framework_objects(
        problems, global_disc, static_condensations, constraint_manager
    )
    
    # Initial conditions
    current_solution = assembler.create_initial_guess_from_problems(
        problems, discretizations, time=0.0
    )
    
    # Time evolution
    dt = global_disc.dt
    time_points = global_disc.get_time_points()
    
    solution_history = [current_solution.copy()]
    mass_history = []
    
    print(f"Starting time evolution: {len(time_points)} time steps")
    
    for step, current_time in enumerate(time_points[1:], 1):
        print(f"Time step {step}: t = {current_time:.3f}")
        
        # Get current bulk data for forcing terms
        domain_solutions = assembler.get_domain_solutions(current_solution)
        bulk_data_list = []
        
        for i, domain_sol in enumerate(domain_solutions):
            bulk_data = assembler.bulk_manager.create_bulk_data(
                i, problems[i], discretizations[i], dual=False
            )
            bulk_data.set_data(domain_sol.reshape(-1, 1))
            bulk_data_list.append(bulk_data)
        
        # Compute forcing terms
        forcing_terms = assembler.compute_forcing_terms(
            bulk_data_list, problems, discretizations, current_time, dt
        )
        
        # Newton solve for this time step
        tolerance = 1e-12
        max_newton_iterations = 10
        
        for newton_iter in range(max_newton_iterations):
            residual, jacobian = assembler.assemble_residual_and_jacobian(
                current_solution, forcing_terms, static_condensations, current_time
            )
            
            residual_norm = np.linalg.norm(residual)
            if residual_norm < tolerance:
                break
            
            delta = np.linalg.solve(jacobian, -residual)
            current_solution += delta
        
        # Monitor mass conservation
        current_mass = assembler.compute_mass_conservation(bulk_data_list)
        mass_history.append(current_mass)
        
        # Store solution
        solution_history.append(current_solution.copy())
        
        if step % 10 == 0:
            print(f"  Mass: {current_mass:.6e}, Newton iterations: {newton_iter + 1}")
    
    print("✓ Time evolution completed")
    
    # Mass conservation check
    initial_mass = mass_history[0] if mass_history else 0.0
    final_mass = mass_history[-1] if mass_history else 0.0
    mass_change = abs(final_mass - initial_mass) / initial_mass if initial_mass != 0 else 0.0
    
    print(f"Mass conservation: {mass_change:.6e} relative change")
    
    return solution_history, mass_history

# Usage
solution_history, mass_history = time_evolution_with_lean_assembler(
    problems, discretizations, static_condensations
)
\end{lstlisting}

\subsection{Method Summary Table}
\label{subsec:lean_assembler_method_summary}

\begin{longtable}{|p{7cm}|p{3cm}|p{4.8cm}|}
\hline
\textbf{Method} & \textbf{Returns} & \textbf{Purpose} \\
\hline
\endhead

\texttt{\_\_init\_\_} & \texttt{None} & Initialize lean assembler with domain data \\
\hline

\texttt{from\_framework\_objects} & \texttt{GlobalAssembler} & Factory method for creation from framework objects \\
\hline

\texttt{assemble\_residual\_and\_jacobian} & \texttt{Tuple} & Global system assembly for Newton solver \\
\hline

\texttt{bulk\_by\_static\_condensation} & \texttt{List} & Bulk solution computation from traces \\
\hline

\texttt{compute\_forcing\_terms} & \texttt{List} & Forcing term computation for implicit Euler \\
\hline

\texttt{create\_initial\_guess\_from\_bulk\_data} & \texttt{np.ndarray} & Initial guess from BulkData objects \\
\hline

\texttt{create\_initial\_guess\_from\_problems} & \texttt{np.ndarray} & Initial guess from problem definitions \\
\hline

\texttt{get\_domain\_solutions} & \texttt{List} & Extract domain solutions from global vector \\
\hline

\texttt{get\_multipliers} & \texttt{np.ndarray} & Extract constraint multipliers \\
\hline

\texttt{initialize\_bulk\_data} & \texttt{List} & Initialize BulkData objects from problems \\
\hline

\texttt{compute\_mass\_conservation} & \texttt{float} & Calculate total mass for conservation check \\
\hline

\texttt{get\_num\_domains} & \texttt{int} & Get number of domains in system \\
\hline

\texttt{get\_domain\_info} & \texttt{DomainData} & Access domain data for inspection \\
\hline

\texttt{test} & \texttt{bool} & Comprehensive validation and testing \\
\hline

\end{longtable}

This documentation provides an exact reference for the lean global assembly module, emphasizing its memory-efficient design, integration with the lean BulkDataManager, and comprehensive constraint handling capabilities for multi-domain HDG systems.

% End of lean global assembly module API documentation
