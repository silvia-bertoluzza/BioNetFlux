\documentclass[11pt,a4paper]{article}
\usepackage[utf8]{inputenc}
\usepackage[english]{babel}
\usepackage{amsmath,amsfonts,amssymb}
\usepackage{graphicx}
\usepackage{geometry}
\usepackage{fancyhdr}
\usepackage{listings}
\usepackage{xcolor}
\usepackage{hyperref}
\usepackage{tocloft}
\usepackage{titlesec}
\usepackage{float}
\usepackage{booktabs}
\usepackage{array}
\usepackage{longtable}

% Page setup
\geometry{margin=2.5cm}
\pagestyle{fancy}
\fancyhf{}
\fancyhead[L]{\textsc{BioNetFlux Documentation}}
\fancyhead[R]{\thepage}
\fancyfoot[C]{\textit{Multi-Domain Biological Network Flow Simulation}}

% Hyperlink setup
\hypersetup{
    colorlinks=true,
    linkcolor=blue,
    filecolor=magenta,      
    urlcolor=cyan,
    pdftitle={BioNetFlux Documentation},
    pdfauthor={BioNetFlux Development Team},
}

% Code listing setup
\definecolor{codegreen}{rgb}{0,0.6,0}
\definecolor{codegray}{rgb}{0.5,0.5,0.5}
\definecolor{codepurple}{rgb}{0.58,0,0.82}
\definecolor{backcolour}{rgb}{0.95,0.95,0.92}

\lstdefinestyle{mystyle}{
    backgroundcolor=\color{backcolour},   
    commentstyle=\color{codegreen},
    keywordstyle=\color{magenta},
    numberstyle=\tiny\color{codegray},
    stringstyle=\color{codepurple},
    basicstyle=\ttfamily\footnotesize,
    breakatwhitespace=false,         
    breaklines=true,                 
    captionpos=b,                    
    keepspaces=true,                 
    numbers=left,                    
    numbersep=5pt,                  
    showspaces=false,                
    showstringspaces=false,
    showtabs=false,                  
    tabsize=2
}

\lstset{style=mystyle}

% Custom commands
\newcommand{\code}[1]{\texttt{#1}}
\newcommand{\bionetflux}{\textsc{BioNetFlux}}

% Title page customization
\title{\Huge {\textbf{\bionetflux{} Documentation}} \\[0.5cm]
       \Large Multi-Domain Biological Network Flow Simulation}
\author{BioNetFlux Development Team}
\date{\today}

\begin{document}

% Title page
\begin{titlepage}
    \centering
    
    % BioNetFlux Logo
    \includegraphics[width=0.6\textwidth]{BioNetFlux_Logo.png}\\[1cm]
    
    {\Huge \textbf{\bionetflux{}} \\[0.5cm]}
    {\Large \textbf{Documentation} \\[1cm]}
    
    {\large Multi-Domain Biological Network Flow Simulation \\[0.5cm]}
    {\large A Python Framework for Complex Network Geometries \\[2cm]}

    
    {\Large BioNetFlux Development Team \\[0.5cm]}
    {\large \today}
    
    \vfill
    
    {\footnotesize 
    \textit{Comprehensive guide to multi-domain biological transport simulations} \\
    \textit{including Keller-Segel chemotaxis and organ-on-chip modeling}
    }
        
        \vskip3cm
        
    % Barra bar
    \includegraphics[width=\textwidth]{Barra_D34Health.png}\\[2cm]
\end{titlepage}

% Table of contents
\tableofcontents
\clearpage

\section{Introduction}

\bionetflux{} is a computational framework designed for simulating biological transport phenomena on complex network geometries. The framework specializes in solving coupled partial differential equations (PDEs) on multi-domain networks, with particular focus on:

\begin{itemize}
    \item \textbf{Keller-Segel chemotaxis models}: Cell migration driven by chemical gradients
    \item \textbf{Organ-on-Chip systems}: Microfluidic device simulations with multiple compartments
    \item \textbf{Multi-domain networks}: Complex geometries with junction conditions and interface constraints
\end{itemize}

\subsection{Key Features}

\begin{itemize}
    \item \textbf{Multi-Domain Support}: Handle complex network topologies with arbitrary domain connections
    \item \textbf{Geometry Management}: Intuitive geometry definition using the \code{DomainGeometry} class
    \item \textbf{Flexible Constraints}: Support for Neumann, Dirichlet, and Kedem-Katchalsky junction conditions
    \item \textbf{Advanced Visualization}: 2D curve plots, 3D flat views, and bird's eye network visualization
    \item \textbf{Time Evolution}: Implicit time stepping with Newton-Raphson nonlinear solver
    \item \textbf{Static Condensation}: Efficient element-level solution elimination
\end{itemize}

\section{Architecture Overview}

The \bionetflux{} framework is organized into several interconnected modules:

\begin{lstlisting}[language=bash, caption={BioNetFlux Directory Structure}]
BioNetFlux/
├── code/
│   ├── ooc1d/
│   │   ├── core/           # Core mathematical components
│   │   ├── geometry/       # Geometry management
│   │   ├── problems/       # Problem definitions
│   │   ├── solver/         # Numerical solvers
│   │   └── visualization/  # Plotting and visualization
│   ├── setup_solver.py    # Main setup interface
│   └── test_*.py          # Example test files
└── docs/                  # Documentation
\end{lstlisting}

\subsection{Core Components}

\begin{enumerate}
    \item \textbf{Problem Definition}: Physical parameters, equations, and boundary conditions
    \item \textbf{Geometry Management}: Domain layout and network topology
    \item \textbf{Discretization}: Finite element spatial discretization
    \item \textbf{Constraint System}: Interface conditions and boundary constraints
    \item \textbf{Time Evolution}: Implicit time stepping with Newton solver
    \item \textbf{Visualization}: Multi-mode plotting system
\end{enumerate}

\section{Module Documentation}

\subsection{Core Module (\code{ooc1d.core})}

\subsubsection{Problem Class (\code{problem.py})}

The \code{Problem} class encapsulates the physics of a single domain:

\begin{lstlisting}[language=Python, caption={Problem Class Structure}]
class Problem:
    def __init__(self, neq, domain_start, domain_length, 
                 parameters, problem_type, name):
        # Physical domain definition
        # Equation parameters
        # Problem identification
\end{lstlisting}

\textbf{Key Methods:}
\begin{itemize}
    \item \code{set\_chemotaxis(chi, dchi)}: Define chemotaxis functions
    \item \code{set\_force(eq\_idx, force\_func)}: Set source terms
    \item \code{set\_solution(eq\_idx, sol\_func)}: Set analytical solutions
    \item \code{set\_initial\_condition(eq\_idx, ic\_func)}: Define initial conditions
    \item \code{set\_extrema(start\_point, end\_point)}: Set 2D spatial coordinates
\end{itemize}


\subsubsection{Discretization Classes (\code{discretization.py})}

\begin{lstlisting}[language=Python, caption={Discretization Classes}]
class Discretization:
    # Single domain spatial discretization
    # Finite element nodes and connectivity
    
class GlobalDiscretization:
    # Multi-domain discretization management
    # Time stepping parameters
\end{lstlisting}

\subsubsection{Constraint Management (\code{constraints.py})}

\begin{lstlisting}[language=Python, caption={Constraint Manager Methods}]
class ConstraintManager:
    # Interface and boundary condition management
    def add_neumann(eq_idx, domain_idx, coordinate, flux_func)
    def add_trace_continuity(eq_idx, dom1_idx, dom2_idx, coord1, coord2)
    def add_kedem_katchalsky(eq_idx, dom1_idx, dom2_idx, 
                            coord1, coord2, perm)
\end{lstlisting}

\subsection{Geometry Module (\code{ooc1d.geometry})}

\subsubsection{DomainGeometry Class (\code{domain\_geometry.py})}

The geometry module provides intuitive tools for defining complex network topologies:

\begin{lstlisting}[language=Python, caption={DomainGeometry Class}]
class DomainGeometry:
    def __init__(self, name="unnamed_geometry"):
        # Initialize empty geometry
    
    def add_domain(self, extrema_start, extrema_end, 
                   domain_start=None, domain_length=None, 
                   name=None, **metadata):
        # Add a domain segment to the network
        
    def get_domain(self, domain_id):
        # Retrieve domain information
        
    def get_bounding_box(self):
        # Calculate network bounding box
\end{lstlisting}

\textbf{Domain Information Structure:}
\begin{lstlisting}[language=Python, caption={Domain Information Dataclass}]
@dataclass
class DomainInfo:
    domain_id: int
    extrema_start: Tuple[float, float]  # Physical coordinates
    extrema_end: Tuple[float, float]
    domain_start: float                 # Parameter space
    domain_length: float
    name: str
    metadata: Dict[str, Any]
\end{lstlisting}

\subsection{Solver Module (\code{ooc1d.solver})}

\subsubsection{Setup Interface (\code{setup\_solver.py})}

\begin{lstlisting}[language=Python, caption={Setup Interface}]
def quick_setup(problem_module, validate=True):
    # Automatic problem setup from module
    # Returns configured solver setup
    
class SolverSetup:
    # Complete solver configuration
    def create_initial_conditions()
    def create_global_solution_vector()
    def extract_domain_solutions()
\end{lstlisting}

\subsection{Visualization Module (\code{ooc1d.visualization})}

\subsubsection{LeanMatplotlibPlotter (\code{lean\_matplotlib\_plotter.py})}

Three complementary visualization modes:

\begin{enumerate}
    \item \textbf{2D Curve Plots}: Traditional solution vs. position plots (separate subplot per domain)
    \item \textbf{Flat 3D View}: Network segments with solution-colored scatter points above
    \item \textbf{Bird's Eye View}: Top-down network view with color-coded segments
\end{enumerate}

\begin{lstlisting}[language=Python, caption={Visualization Methods}]
class LeanMatplotlibPlotter:
    def plot_2d_curves(trace_solutions, title, 
                       show_mesh_points, save_filename)
    def plot_flat_3d(trace_solutions, equation_idx, 
                     view_angle, save_filename)
    def plot_birdview(trace_solutions, equation_idx, 
                      time, save_filename)
\end{lstlisting}

\section{Getting Started}

\subsection{Installation}

\begin{enumerate}
    \item Clone the repository:
    \begin{lstlisting}[language=bash]
git clone <repository-url>
cd BioNetFlux
    \end{lstlisting}
    
    \item Set up Python path:
    \begin{lstlisting}[language=Python]
import sys
sys.path.insert(0, '/path/to/BioNetFlux/code')
    \end{lstlisting}
\end{enumerate}

\subsection{Basic Usage}

\begin{lstlisting}[language=Python, caption={Basic Usage Example}]
from setup_solver import quick_setup
from ooc1d.visualization.lean_matplotlib_plotter import LeanMatplotlibPlotter

# Load a problem
setup = quick_setup("ooc1d.problems.my_problem", validate=True)

# Create initial conditions
trace_solutions, multipliers = setup.create_initial_conditions()

# Initialize visualization
plotter = LeanMatplotlibPlotter(
    problems=setup.problems,
    discretizations=setup.global_discretization.spatial_discretizations
)

# Plot initial conditions
plotter.plot_2d_curves(trace_solutions, title="Initial Conditions")
plotter.plot_birdview(trace_solutions, equation_idx=0, time=0.0)
\end{lstlisting}

\section{Creating New Problems}

\subsection{Problem Structure Template}

Create a new file in \code{ooc1d/problems/} following this structure:

\begin{lstlisting}[language=Python, caption={Problem Template Structure}]
# File: ooc1d/problems/my_new_problem.py
import numpy as np
from ..core.problem import Problem
from ..core.discretization import Discretization, GlobalDiscretization
from ..core.constraints import ConstraintManager
from ..geometry import DomainGeometry

def create_global_framework():
    """
    Create a new multi-domain problem.
    Returns: problems, global_discretization, 
             constraint_manager, problem_name
    """
    # 1. Global parameters
    neq = 2  # Number of equations
    T = 1.0  # Final time
    dt = 0.1  # Time step
    problem_name = "My New Problem"
    
    # 2. Physical parameters
    parameters = np.array([param1, param2, param3, param4])
    
    # 3. Define functions (chemotaxis, sources, solutions, etc.)
    def chi(x): return np.ones_like(x)
    def dchi(x): return np.zeros_like(x)
    def source_u(s, t): return 0.0 * s
    def source_phi(s, t): return 0.0 * s
    def initial_u(s, t=0.0): return np.ones_like(s)
    def initial_phi(s, t=0.0): return np.zeros_like(s)
    
    # 4. Create geometry
    geometry = DomainGeometry("my_geometry")
    # Add domains using geometry.add_domain(...)
    
    # 5. Create problems from geometry
    problems = []
    discretizations = []
    for domain_id in range(geometry.num_domains()):
        domain_info = geometry.get_domain(domain_id)
        # Create Problem and Discretization objects
    
    # 6. Set up constraints
    constraint_manager = ConstraintManager()
    # Add boundary and interface constraints
    
    # 7. Return framework components
    return problems, global_discretization, constraint_manager, problem_name
\end{lstlisting}

\subsection{Keller-Segel Problems}

For chemotaxis problems, include:

\begin{lstlisting}[language=Python, caption={Keller-Segel Problem Setup}]
# Chemotaxis sensitivity function
def chi(x):
    k1, k2 = 3.9e-9, 5.e-6
    return k1 / (k2 + x)**2

def dchi(x):
    k1, k2 = 3.9e-9, 5.e-6
    return -k1 * 2 / (k2 + x)**3

# Set chemotaxis for all problems
for problem in problems:
    problem.set_chemotaxis(chi, dchi)
    problem.set_force(0, source_u)      # Cell equation source
    problem.set_force(1, source_phi)    # Chemical equation source
\end{lstlisting}

\subsection{Organ-on-Chip Problems}

For microfluidic systems, focus on:

\begin{lstlisting}[language=Python, caption={Organ-on-Chip Problem Setup}]
# Multi-compartment setup
compartments = ["inlet", "cell_chamber", "outlet", "waste"]

# Different parameters per compartment
parameters_list = [
    np.array([D1, v1, k1, 0.0]),     # Inlet: high flow
    np.array([D2, v2, k2, k_cell]),  # Cell chamber: cell interaction
    np.array([D3, v3, k3, 0.0]),     # Outlet: medium flow
    np.array([D4, v4, k4, 0.0])      # Waste: low flow
]

# Junction conditions with permeabilities
permeabilities = [0.8, 1.0, 0.9]  # Between compartments
\end{lstlisting}

\section{Geometry Module Guide}

\subsection{Simple Linear Network}

\begin{lstlisting}[language=Python, caption={Linear Network Geometry}]
geometry = DomainGeometry("linear_chain")

# Add sequential domains
geometry.add_domain(
    extrema_start=(0.0, 0.0),
    extrema_end=(1.0, 0.0),
    name="segment1"
)

geometry.add_domain(
    extrema_start=(1.0, 0.0),
    extrema_end=(2.0, 0.0),
    name="segment2"
)
\end{lstlisting}

\subsection{T-Junction Network}

\begin{lstlisting}[language=Python, caption={T-Junction Geometry}]
geometry = DomainGeometry("t_junction")

# Main channel
geometry.add_domain(
    extrema_start=(0.0, -1.0),
    extrema_end=(0.0, 1.0),
    name="main_channel"
)

# Side branch
geometry.add_domain(
    extrema_start=(0.0, 0.0),
    extrema_end=(1.0, 0.0),
    name="side_branch"
)
\end{lstlisting}

\subsection{Grid Network}

\begin{lstlisting}[language=Python, caption={Grid Network Geometry}]
geometry = DomainGeometry("grid_network")

# Vertical segments
for i, x_pos in enumerate([-0.5, 0.5]):
    geometry.add_domain(
        extrema_start=(x_pos, 0.0),
        extrema_end=(x_pos, 1.0),
        name=f"vertical_{i}"
    )

# Horizontal connectors
for i, y_pos in enumerate([0.2, 0.4, 0.6, 0.8]):
    geometry.add_domain(
        extrema_start=(-0.5, y_pos),
        extrema_end=(0.5, y_pos),
        name=f"horizontal_{i}"
    )
\end{lstlisting}

\subsection{Complex Branching Network}

\begin{lstlisting}[language=Python, caption={Branching Network Geometry}]
geometry = DomainGeometry("branching_network")

# Main trunk
geometry.add_domain(
    extrema_start=(0.0, 0.0),
    extrema_end=(0.0, 2.0),
    name="trunk"
)

# Branches at different levels
branch_angles = [30, 60, 120, 150]  # degrees
for i, angle in enumerate(branch_angles):
    angle_rad = np.radians(angle)
    length = 1.0
    end_x = length * np.cos(angle_rad)
    end_y = 1.0 + length * np.sin(angle_rad)
    
    geometry.add_domain(
        extrema_start=(0.0, 1.0),
        extrema_end=(end_x, end_y),
        name=f"branch_{i}"
    )
\end{lstlisting}

\section{Visualization System}

\subsection{2D Curve Plots}

Best for analyzing solution profiles along individual domains:

\begin{lstlisting}[language=Python, caption={2D Curve Plotting}]
plotter.plot_2d_curves(
    trace_solutions=solutions,
    title="Solution Profiles",
    show_mesh_points=True,
    save_filename="solution_curves.png"
)
\end{lstlisting}

\textbf{Features:}
\begin{itemize}
    \item Separate subplot per domain
    \item All equations shown in each domain
    \item Mesh point markers
    \item Domain boundary indicators
\end{itemize}

\subsection{Flat 3D View}

Ideal for understanding network topology with solution values:

\begin{lstlisting}[language=Python, caption={Flat 3D Visualization}]
plotter.plot_flat_3d(
    trace_solutions=solutions,
    equation_idx=0,
    view_angle=(30, 45),
    save_filename="network_3d.png"
)
\end{lstlisting}

\textbf{Features:}
\begin{itemize}
    \item Network segments in xy-plane
    \item Solution values as colored scatter points above
    \item Connecting lines from segments to solution points
    \item Rotatable 3D view
\end{itemize}

\subsection{Bird's Eye View}

Perfect for network-level solution analysis:

\begin{lstlisting}[language=Python, caption={Bird's Eye View Plotting}]
plotter.plot_birdview(
    trace_solutions=solutions,
    equation_idx=0,
    time=current_time,
    save_filename="network_overview.png"
)
\end{lstlisting}

\textbf{Features:}
\begin{itemize}
    \item Top-down network view
    \item Color-coded segment thickness
    \item Solution point markers
    \item Clean network overview
\end{itemize}


\section{Geometry Module}
\label{sec:geometry_module}

The geometry module provides network topology support for BioNetFlux, enabling multi-domain biological transport problems. It integrates with the existing Problem, Discretization, and ConstraintManager classes to support complex network geometries while maintaining compatibility with the HDG solver framework.

\subsection{Overview}
\label{subsec:geometry_overview}

The geometry module extends BioNetFlux to support:

\begin{itemize}
	\item Multi-domain network topologies for biological applications
	\item Integration with existing Problem and Discretization classes
	\item Junction constraint generation for ConstraintManager
	\item Compatibility with OrganOnChip 4-equation systems
	\item Support for Keller-Segel chemotaxis networks
	\item Seamless integration with static condensation methods
\end{itemize}

\subsection{Integration with BioNetFlux Architecture}
\label{subsec:bionetflux_integration}

Unlike standalone geometry systems, the BioNetFlux geometry module is designed to work within the existing framework:

\subsubsection{Problem Class Integration}

\begin{lstlisting}[language=Python, caption=Problem Class Extension for Networks]
from ooc1d.core.problem import Problem
from ooc1d.core.discretization import Discretization, GlobalDiscretization
from ooc1d.core.constraints import ConstraintManager

class NetworkProblem:
    """Network extension for existing Problem class."""
    
    def __init__(self, problem_type="organ_on_chip"):
        self.problem_type = problem_type
        self.domains = []
        self.domain_problems = []
        
    def add_domain(self, domain_id, domain_start, domain_length, parameters):
        """Add domain using existing Problem class structure."""
        problem = Problem(
            neq=4,  # OrganOnChip: u, omega, v, phi
            domain_start=domain_start,
            domain_length=domain_length, 
            parameters=parameters,
            problem_type=self.problem_type,
            name=f"domain_{domain_id}"
        )
        self.domain_problems.append(problem)
        
        # Store domain specification
        domain_spec = {
            'id': domain_id,
            'start': domain_start,
            'length': domain_length,
            'end': domain_start + domain_length
        }
        self.domains.append(domain_spec)
        
        return problem
\end{lstlisting}

\subsubsection{Discretization Integration}

\begin{lstlisting}[language=Python, caption=Network Discretization]
def create_network_discretization(network_problem, n_elements_per_domain=10):
    """Create GlobalDiscretization from network specification."""
    discretizations = []
    
    for domain_spec in network_problem.domains:
        discretization = Discretization(
            n_elements=n_elements_per_domain,
            domain_start=domain_spec['start'],
            domain_length=domain_spec['length'],
            stab_constant=1.0
        )
        
        # Set stabilization parameters for 4-equation OrganOnChip system
        discretization.set_tau([1.0, 1.0, 1.0, 1.0])  # tu, to, tv, tp
        
        discretizations.append(discretization)
    
    return GlobalDiscretization(discretizations)
\end{lstlisting}

\subsection{Junction Management Compatible with ConstraintManager}
\label{subsec:junction_constraints}

Rather than creating a separate junction system, the geometry module generates constraints for the existing ConstraintManager:

\subsubsection{Junction Constraint Generation}

\begin{lstlisting}[language=Python, caption=Junction Constraint Integration]
class NetworkConstraintGenerator:
    """Generate constraints for network junctions using existing ConstraintManager."""
    
    def __init__(self, network_problem):
        self.network = network_problem
        
    def setup_junction_constraints(self, constraint_manager):
        """Setup junction constraints using existing ConstraintManager API."""
        
        # Find connected domains (adjacent domain boundaries)
        for i in range(len(self.network.domains) - 1):
            domain1 = self.network.domains[i]
            domain2 = self.network.domains[i + 1]
            
            # Check if domains are connected (end of domain1 = start of domain2)
            if abs(domain1['end'] - domain2['start']) < 1e-12:
                junction_position = domain1['end']
                
                # Add continuity constraints for all equations
                neq = self.network.domain_problems[0].neq
                for eq_idx in range(neq):
                    constraint_manager.add_trace_continuity(
                        eq_idx, i, i+1, 
                        domain1['end'], domain2['start']
                    )
        
        return constraint_manager
        
    def setup_bifurcation_constraints(self, constraint_manager, 
                                    parent_domain, child_domains, junction_pos):
        """Setup bifurcation constraints for Y-junctions."""
        
        neq = self.network.domain_problems[0].neq
        
        # Continuity of primary variables (u, phi)
        primary_equations = [0, 3]  # u and phi in OrganOnChip system
        
        for eq_idx in primary_equations:
            for child_idx in child_domains:
                constraint_manager.add_trace_continuity(
                    eq_idx, parent_domain, child_idx,
                    junction_pos, junction_pos
                )
        
        # Mass conservation at junction
        constraint_manager.add_mass_conservation_constraint(
            domains=[parent_domain] + child_domains,
            position=junction_pos,
            equation=0  # Conservation for u equation
        )
        
        return constraint_manager
\end{lstlisting}

\subsection{OrganOnChip Network Support}
\label{subsec:ooc_network}

Based on the MATLAB reference files (TestProblem.m, EmptyProblem.m), the geometry module supports OrganOnChip networks:

\subsubsection{OrganOnChip Parameters Integration}

\begin{lstlisting}[language=Python, caption=OrganOnChip Network Parameters]
def create_ooc_network_problem(domain_specs):
    """Create OrganOnChip network problem from MATLAB TestProblem.m."""
    
    # Physical parameters from MATLAB reference
    nu = 1.0      # viscosity
    mu = 2.0      # viscosity
    epsilon = 1.0 # coupling parameter
    sigma = 1.0   # coupling parameter
    
    # Reaction parameters
    a = 0.0       # reaction parameter
    c = 0.0       # reaction parameter
    
    # Coupling parameters  
    b = 1.0       # coupling parameter
    d = 1.0       # coupling parameter
    chi = 1.0     # coupling parameter
    
    # Parameter vector matching MATLAB OoC_pbParameters
    parameters = np.array([nu, mu, epsilon, sigma, a, b, c, d, chi])
    
    network = NetworkProblem(problem_type="organ_on_chip")
    
    for i, domain_spec in enumerate(domain_specs):
        domain_problem = network.add_domain(
            domain_id=i,
            domain_start=domain_spec['start'],
            domain_length=domain_spec['length'],
            parameters=parameters
        )
        
        # Set initial conditions from MATLAB reference
        if i == 0:  # First domain gets sin initial condition
            domain_problem.set_initial_condition(0, lambda s, t=0.0: np.sin(2*np.pi*s))
        else:
            domain_problem.set_initial_condition(0, lambda s, t=0.0: np.zeros_like(s))
            
        # All other equations start at zero (matching EmptyProblem.m)
        for eq_idx in [1, 2, 3]:  # omega, v, phi
            domain_problem.set_initial_condition(eq_idx, lambda s, t=0.0: np.zeros_like(s))
        
        # Set zero forcing terms (matching MATLAB)
        for eq_idx in range(4):
            domain_problem.set_force(eq_idx, lambda s, t: np.zeros_like(s))
    
    return network
\end{lstlisting}

\subsubsection{Boundary Condition Integration}

\begin{lstlisting}[language=Python, caption=Network Boundary Conditions]
def setup_network_boundary_conditions(network_problem, constraint_manager):
    """Setup boundary conditions matching MATLAB fluxu0/fluxu1 specification."""
    
    n_domains = len(network_problem.domains)
    
    # External boundaries (network inlet and outlet)
    for eq_idx in range(4):  # All four OrganOnChip equations
        
        # Inlet boundary (start of first domain) - zero flux from MATLAB
        constraint_manager.add_neumann(
            eq_idx, 0, network_problem.domains[0]['start'],
            lambda t: 0.0  # fluxu0 = 0 from MATLAB
        )
        
        # Outlet boundary (end of last domain) - zero flux from MATLAB  
        constraint_manager.add_neumann(
            eq_idx, n_domains-1, network_problem.domains[-1]['end'],
            lambda t: 0.0  # fluxu1 = 0 from MATLAB
        )
    
    return constraint_manager
\end{lstlisting}

\subsection{Simple Network Topologies}
\label{subsec:simple_topologies}

The geometry module provides factory methods for common biological network patterns:

\subsubsection{Linear Network Chain}

\begin{lstlisting}[language=Python, caption=Linear Domain Chain]
def create_linear_network(n_domains=3, total_length=3.0):
    """Create linear chain of domains (matching double_arc examples)."""
    
    domain_length = total_length / n_domains
    domain_specs = []
    
    for i in range(n_domains):
        domain_specs.append({
            'start': i * domain_length,
            'length': domain_length
        })
    
    network = create_ooc_network_problem(domain_specs)
    
    # Setup connectivity constraints
    constraint_manager = ConstraintManager()
    constraint_generator = NetworkConstraintGenerator(network)
    constraint_generator.setup_junction_constraints(constraint_manager)
    
    return network, constraint_manager
\end{lstlisting}

\subsubsection{Y-Junction Bifurcation}

\begin{lstlisting}[language=Python, caption=Y-Junction Network]
def create_y_junction_network(main_length=1.0, branch_length=1.0):
    """Create Y-junction bifurcation network."""
    
    # Three domains: main vessel + two branches
    domain_specs = [
        {'start': 0.0, 'length': main_length},           # Main vessel
        {'start': main_length, 'length': branch_length}, # Branch 1  
        {'start': main_length, 'length': branch_length}  # Branch 2
    ]
    
    network = create_ooc_network_problem(domain_specs)
    
    # Setup bifurcation constraints
    constraint_manager = ConstraintManager()
    constraint_generator = NetworkConstraintGenerator(network)
    
    # Setup bifurcation at junction between domain 0 and domains 1,2
    constraint_generator.setup_bifurcation_constraints(
        constraint_manager,
        parent_domain=0,
        child_domains=[1, 2], 
        junction_pos=main_length
    )
    
    return network, constraint_manager
\end{lstlisting}

\subsection{Integration with Static Condensation}
\label{subsec:static_condensation_integration}

The geometry module works seamlessly with existing static condensation classes:

\subsubsection{StaticCondensationOOC Compatibility}

\begin{lstlisting}[language=Python, caption=Static Condensation Integration]
def setup_network_static_condensation(network_problem, discretizations, 
                                    elementary_matrices):
    """Setup static condensation for network domains."""
    
    from ooc1d.core.static_condensation_ooc import StaticCondensationOOC
    
    static_condensations = []
    
    for i, (problem, discretization) in enumerate(
            zip(network_problem.domain_problems, discretizations)):
        
        # Use existing StaticCondensationOOC for each domain
        static_cond = StaticCondensationOOC(
            problem=problem,
            discretization=discretization,
            elementary_matrices=elementary_matrices
        )
        
        static_condensations.append(static_cond)
    
    return static_condensations
\end{lstlisting}

\subsection{Visualization Integration}
\label{subsec:visualization_integration}

The geometry module integrates with the existing MultiDomainPlotter:

\subsubsection{Network Plotting}

\begin{lstlisting}[language=Python, caption=Network Visualization]
def plot_network_solution(network_problem, trace_solutions, time=0.0):
    """Plot network solution using existing MultiDomainPlotter."""
    
    from ooc1d.visualization.multi_domain_plotter import MultiDomainPlotter
    
    # Create discretizations for plotting
    discretizations = []
    for domain_spec in network_problem.domains:
        disc = Discretization(
            n_elements=10,
            domain_start=domain_spec['start'],
            domain_length=domain_spec['length'],
            stab_constant=1.0
        )
        discretizations.append(disc)
    
    # Use existing MultiDomainPlotter
    plotter = MultiDomainPlotter(
        problems=network_problem.domain_problems,
        discretizations=discretizations,
        equation_names=['u', 'ω', 'v', 'φ']  # OrganOnChip equations
    )
    
    # Create continuous plot across network
    fig = plotter.plot_continuous_solution(
        trace_solutions=trace_solutions,
        time=time,
        title_prefix="Network Solution",
        show_domain_boundaries=True,
        show_domain_labels=True
    )
    
    return fig
\end{lstlisting}

\subsection{Usage Examples}
\label{subsec:usage_examples}

Complete examples showing integration with BioNetFlux workflow:

\subsubsection{Complete Network Setup}

\begin{lstlisting}[language=Python, caption=Complete Network Problem Setup]
def create_complete_network_problem():
    """Complete example matching BioNetFlux test_time_evolution structure."""
    
    # Create linear 3-domain network
    network, constraint_manager = create_linear_network(n_domains=3)
    
    # Create discretizations
    discretizations = []
    for domain_spec in network.domains:
        disc = Discretization(
            n_elements=10,
            domain_start=domain_spec['start'], 
            domain_length=domain_spec['length'],
            stab_constant=1.0
        )
        disc.set_tau([1.0, 1.0, 1.0, 1.0])  # OrganOnChip stabilization
        discretizations.append(disc)
    
    global_disc = GlobalDiscretization(discretizations)
    global_disc.set_time_parameters(dt=0.01, T=0.5)
    
    # Setup network boundary conditions
    constraint_manager = setup_network_boundary_conditions(
        network, constraint_manager
    )
    
    # Map constraints to discretizations
    constraint_manager.map_to_discretizations(discretizations)
    
    return network.domain_problems, global_disc, constraint_manager
\end{lstlisting}

\subsubsection{Integration with Existing Solver}

\begin{lstlisting}[language=Python, caption=Solver Integration]
# This integrates directly with existing BioNetFlux solver setup
problems, global_disc, constraint_manager = create_complete_network_problem()

# Use existing setup_solver.py infrastructure
from setup_solver import SolverSetup

setup = SolverSetup(
    problems=problems,
    global_discretization=global_disc, 
    constraint_manager=constraint_manager
)

# Continue with normal BioNetFlux workflow
trace_solutions, multipliers = setup.create_initial_conditions()
global_solution = setup.create_global_solution_vector(trace_solutions, multipliers)

# Time evolution using existing infrastructure
# ... (standard BioNetFlux time evolution loop)
\end{lstlisting}

\subsection{Limitations and Future Extensions}
\label{subsec:limitations}

The current geometry module is designed for compatibility with existing BioNetFlux architecture:

\subsubsection{Current Limitations}

\begin{itemize}
	\item 1D domains only (consistent with current BioNetFlux scope)
	\item Linear domain arrangements (extensions needed for complex branching)
	\item Manual junction specification (automatic detection future work)
	\item Limited to sequential domain connectivity
\end{itemize}

\subsubsection{Planned Extensions}

\begin{itemize}
	\item Automatic junction detection from domain connectivity
	\item Complex branching pattern support
	\item Integration with adaptive mesh refinement
	\item Support for time-dependent network geometries
	\item Optimization tools for network design
\end{itemize}

The geometry module provides essential network topology support while maintaining full compatibility with BioNetFlux's existing HDG solver framework, enabling multi-domain biological transport modeling without disrupting the established architecture.

% filepath: /Users/silviabertoluzza/GIT/BioNetFlux/docs/problem_module.tex

\section{Problem Module}
\label{sec:problem_module}

The problem module defines the mathematical formulations and physical models that BioNetFlux can solve. This module provides a flexible framework for specifying partial differential equations, initial conditions, boundary conditions, and source terms on complex network geometries.

\subsection{Overview}
\label{subsec:problem_overview}

The problem module is designed around the concept of \emph{domain problems}, where each domain in a network geometry can have its own set of equations, parameters, and physical properties. The module supports:

\begin{itemize}
	\item Multi-equation systems (arbitrary number of coupled PDEs)
	\item Time-dependent and steady-state problems
	\item Nonlinear reaction-diffusion systems
	\item Coupled transport phenomena
	\item Flexible parameter specification
	\item Custom initial and boundary conditions
\end{itemize}

\subsection{Problem Class Hierarchy}
\label{subsec:problem_hierarchy}

The problem module follows an object-oriented design with a base \texttt{Problem} class and specialized implementations:

\subsubsection{Base Problem Class}

The base \texttt{Problem} class defines the interface that all problem implementations must follow:

\begin{lstlisting}[language=Python, caption=Base Problem Class Structure]
	class Problem:
	def __init__(self, neq: int, parameters: dict):
	self.neq = neq  # Number of equations
	self.parameters = parameters
	
	def flux_function(self, u, x, t):
	"""Compute flux vector F(u,x,t)"""
	raise NotImplementedError
	
	def source_function(self, u, x, t):
	"""Compute source term S(u,x,t)"""
	raise NotImplementedError
	
	def initial_condition(self, x, equation_idx):
	"""Compute initial condition u_0(x)"""
	raise NotImplementedError
\end{lstlisting}

Key attributes:
\begin{itemize}
	\item \texttt{neq}: Number of equations in the system
	\item \texttt{parameters}: Dictionary of problem-specific parameters
	\item \texttt{domain\_length}: Length of the spatial domain
	\item \texttt{boundary\_conditions}: Specification of boundary conditions
\end{itemize}

\subsubsection{Keller-Segel Problems}

The Keller-Segel chemotaxis model is implemented for studying cell migration in response to chemical gradients:

\begin{align}
	\frac{\partial u}{\partial t} &= D_u \frac{\partial^2 u}{\partial x^2} - \chi \frac{\partial}{\partial x}\left(u \frac{\partial v}{\partial x}\right) + f_u(u,v,x,t) \label{eq:ks_u} \\
	\frac{\partial v}{\partial t} &= D_v \frac{\partial^2 v}{\partial x^2} + \alpha u - \beta v + f_v(u,v,x,t) \label{eq:ks_v}
\end{align}

where:
\begin{itemize}
	\item $u(x,t)$ is the cell density
	\item $v(x,t)$ is the chemoattractant concentration  
	\item $D_u, D_v$ are diffusion coefficients
	\item $\chi$ is the chemotaxis sensitivity
	\item $\alpha, \beta$ are production and degradation rates
	\item $f_u, f_v$ are source/sink terms
\end{itemize}

\subsubsection{Organ-on-Chip Problems}

The organ-on-chip (OoC) model describes drug transport and cellular uptake in microfluidic devices:

\begin{align}
	\frac{\partial u}{\partial t} &= D_u \frac{\partial^2 u}{\partial x^2} - k_{\text{on}} u + k_{\text{off}} v + f_u(u,v,w,\omega,x,t) \label{eq:ooc_u} \\
	\frac{\partial v}{\partial t} &= k_{\text{on}} u - k_{\text{off}} v - k_{\text{int}} v + f_v(u,v,w,\omega,x,t) \label{eq:ooc_v} \\
	\frac{\partial w}{\partial t} &= k_{\text{int}} v - k_{\text{deg}} w + f_w(u,v,w,\omega,x,t) \label{eq:ooc_w} \\
	\frac{\partial \omega}{\partial t} &= D_\omega \frac{\partial^2 \omega}{\partial x^2} + A \sin(\omega_0 t + \phi) + f_\omega(u,v,w,\omega,x,t) \label{eq:ooc_omega}
\end{align}

where:
\begin{itemize}
	\item $u(x,t)$ is the free drug concentration
	\item $v(x,t)$ is the bound drug concentration
	\item $w(x,t)$ is the cellular drug concentration
	\item $\omega(x,t)$ represents tissue deformation or other dynamic effects
	\item $k_{\text{on}}, k_{\text{off}}, k_{\text{int}}, k_{\text{deg}}$ are reaction rates
	\item $D_u, D_\omega$ are diffusion coefficients
	\item $A, \omega_0, \phi$ are oscillation parameters
\end{itemize}

\subsection{Problem Configuration}
\label{subsec:problem_config}

Problems are configured through parameter dictionaries and factory methods:

\begin{lstlisting}[language=Python, caption=Problem Configuration Example]
	# Keller-Segel configuration
	ks_params = {
		'D_u': 1.0,          # Cell diffusion coefficient
		'D_v': 10.0,         # Chemical diffusion coefficient  
		'chi': 5.0,          # Chemotaxis sensitivity
		'alpha': 1.0,        # Chemical production rate
		'beta': 1.0,         # Chemical degradation rate
		'domain_length': 1.0 # Spatial domain length
	}
	
	# Create problem instance
	problem = KellerSegelProblem(parameters=ks_params)
	
	# Organ-on-Chip configuration  
	ooc_params = {
		'D_u': 1.0,          # Drug diffusion
		'D_omega': 0.1,      # Tissue diffusion
		'k_on': 2.0,         # Binding rate
		'k_off': 0.5,        # Unbinding rate
		'k_int': 1.0,        # Internalization rate
		'k_deg': 0.1,        # Degradation rate
		'A': 0.5,            # Oscillation amplitude
		'omega0': 2*np.pi,   # Oscillation frequency
		'phi': 0.0,          # Phase shift
		'domain_length': 2.0
	}
	
	problem = OrganOnChipProblem(parameters=ooc_params)
\end{lstlisting}

\subsection{Initial Conditions}
\label{subsec:initial_conditions}

The problem module supports various types of initial conditions:

\subsubsection{Analytical Functions}

Initial conditions can be specified as mathematical functions:

\begin{lstlisting}[language=Python, caption=Analytical Initial Conditions]
	def gaussian_ic(self, x, equation_idx):
	"""Gaussian initial condition centered at domain midpoint"""
	if equation_idx == 0:  # Cell density
	center = self.parameters['domain_length'] / 2
	width = 0.1
	return np.exp(-((x - center) / width)**2)
	elif equation_idx == 1:  # Chemical concentration
	return np.ones_like(x) * 0.1
	else:
	return np.zeros_like(x)
\end{lstlisting}

\subsubsection{Piecewise Functions}

Complex initial distributions can be constructed piecewise:

\begin{lstlisting}[language=Python, caption=Piecewise Initial Conditions]
	def piecewise_ic(self, x, equation_idx):
	"""Piecewise initial condition"""
	result = np.zeros_like(x)
	L = self.parameters['domain_length']
	
	if equation_idx == 0:  # Cell density
	# High density in left third
	mask1 = x < L/3
	result[mask1] = 2.0
	
	# Medium density in middle third  
	mask2 = (x >= L/3) & (x < 2*L/3)
	result[mask2] = 1.0
	
	# Low density in right third
	mask3 = x >= 2*L/3
	result[mask3] = 0.5
	
	return result
\end{lstlisting}

\subsubsection{Random Initial Conditions}

Stochastic initial conditions for sensitivity analysis:

\begin{lstlisting}[language=Python, caption=Random Initial Conditions]
	def random_ic(self, x, equation_idx):
	"""Random initial condition with specified statistics"""
	np.random.seed(self.parameters.get('random_seed', 42))
	
	if equation_idx == 0:
	# Uniform random with specified bounds
	low = self.parameters.get('u_ic_low', 0.0)
	high = self.parameters.get('u_ic_high', 1.0)
	return np.random.uniform(low, high, x.shape)
	else:
	# Gaussian random
	mean = self.parameters.get('v_ic_mean', 0.5)
	std = self.parameters.get('v_ic_std', 0.1)
	return np.random.normal(mean, std, x.shape)
\end{lstlisting}

\subsection{Source Terms and Forcing}
\label{subsec:source_terms}

The problem module supports time-dependent source terms and external forcing:

\subsubsection{Constant Sources}

\begin{lstlisting}[language=Python, caption=Constant Source Terms]
	def constant_source(self, u, x, t):
	"""Constant source terms"""
	source = np.zeros((self.neq, len(x)))
	
	# Constant production for equation 0
	source[0, :] = self.parameters.get('source_u', 0.0)
	
	# Spatially varying source for equation 1
	center = self.parameters['domain_length'] / 2
	width = 0.2
	source[1, :] = self.parameters.get('source_v', 0.0) * \
	np.exp(-((x - center) / width)**2)
	
	return source
\end{lstlisting}

\subsubsection{Time-Dependent Forcing}

\begin{lstlisting}[language=Python, caption=Time-Dependent Source Terms]
	def oscillating_source(self, u, x, t):
	"""Oscillating source terms"""
	source = np.zeros((self.neq, len(x)))
	
	# Sinusoidal forcing
	freq = self.parameters.get('forcing_frequency', 1.0)
	amplitude = self.parameters.get('forcing_amplitude', 0.1)
	
	source[0, :] = amplitude * np.sin(2 * np.pi * freq * t)
	
	return source
\end{lstlisting}

\subsubsection{Nonlinear Reactions}

\begin{lstlisting}[language=Python, caption=Nonlinear Reaction Terms]
	def nonlinear_reactions(self, u, x, t):
	"""Nonlinear reaction terms"""
	source = np.zeros((self.neq, len(x)))
	
	# Logistic growth for cells
	K = self.parameters.get('carrying_capacity', 10.0)
	r = self.parameters.get('growth_rate', 0.5)
	source[0, :] = r * u[0, :] * (1 - u[0, :] / K)
	
	# Michaelis-Menten kinetics for chemical
	Km = self.parameters.get('Km', 1.0)
	Vmax = self.parameters.get('Vmax', 2.0)
	source[1, :] = -Vmax * u[1, :] / (Km + u[1, :])
	
	return source
\end{lstlisting}

\subsection{Boundary Conditions}
\label{subsec:boundary_conditions}

The problem module interfaces with the constraint system to specify boundary conditions:

\subsubsection{Dirichlet Conditions}

Fixed values at domain boundaries:

\begin{lstlisting}[language=Python, caption=Dirichlet Boundary Conditions]
	def setup_dirichlet_bc(self, constraint_manager, domain_idx):
	"""Setup Dirichlet boundary conditions"""
	# Fixed left boundary for equation 0
	constraint_manager.add_dirichlet(
	equation_index=0,
	domain_index=domain_idx, 
	position=0.0,  # Left boundary
	data_function=lambda t: self.parameters.get('u_left', 1.0)
	)
	
	# Fixed right boundary for equation 1
	constraint_manager.add_dirichlet(
	equation_index=1,
	domain_index=domain_idx,
	position=self.parameters['domain_length'],  # Right boundary
	data_function=lambda t: self.parameters.get('v_right', 0.0)
	)
\end{lstlisting}

\subsubsection{Neumann Conditions}

Specified flux at boundaries:

\begin{lstlisting}[language=Python, caption=Neumann Boundary Conditions]
	def setup_neumann_bc(self, constraint_manager, domain_idx):
	"""Setup Neumann boundary conditions"""
	# Zero flux (no-flux) at left boundary
	constraint_manager.add_neumann(
	equation_index=0,
	domain_index=domain_idx,
	position=0.0,
	data_function=lambda t: 0.0
	)
	
	# Time-dependent flux at right boundary
	def time_dependent_flux(t):
	return self.parameters.get('flux_amplitude', 0.1) * np.sin(t)
	
	constraint_manager.add_neumann(
	equation_index=1,
	domain_index=domain_idx,
	position=self.parameters['domain_length'],
	data_function=time_dependent_flux
	)
\end{lstlisting}

\subsubsection{Robin Conditions}

Mixed boundary conditions combining value and flux:

\begin{lstlisting}[language=Python, caption=Robin Boundary Conditions]
	def setup_robin_bc(self, constraint_manager, domain_idx):
	"""Setup Robin boundary conditions: alpha*u + beta*flux = data"""
	alpha = self.parameters.get('robin_alpha', 1.0)
	beta = self.parameters.get('robin_beta', 0.1)
	
	constraint_manager.add_robin(
	equation_index=0,
	domain_index=domain_idx,
	position=0.0,
	alpha=alpha,
	beta=beta,
	data_function=lambda t: self.parameters.get('robin_data', 0.0)
	)
\end{lstlisting}

\subsection{Junction Conditions}
\label{subsec:junction_conditions}

For network problems, the module supports junction conditions between domains:

\subsubsection{Trace Continuity}

Ensuring solution continuity at junctions:

\begin{lstlisting}[language=Python, caption=Trace Continuity Conditions]
	def setup_trace_continuity(self, constraint_manager, 
	domain1_idx, domain2_idx,
	junction_pos1, junction_pos2):
	"""Setup trace continuity between domains"""
	for eq_idx in range(self.neq):
	constraint_manager.add_trace_continuity(
	equation_index=eq_idx,
	domain1_index=domain1_idx,
	domain2_index=domain2_idx,
	position1=junction_pos1,
	position2=junction_pos2
	)
\end{lstlisting}

\subsubsection{Kedem-Katchalsky Conditions}

Membrane transport with permeability:

\begin{lstlisting}[language=Python, caption=Kedem-Katchalsky Conditions]
	def setup_kedem_katchalsky(self, constraint_manager,
	domain1_idx, domain2_idx, 
	junction_pos1, junction_pos2):
	"""Setup Kedem-Katchalsky membrane transport"""
	# Different permeabilities for different species
	permeabilities = [
	self.parameters.get('P_u', 1.0),  # Cell permeability
	self.parameters.get('P_v', 5.0),  # Chemical permeability
	self.parameters.get('P_w', 0.1),  # Drug permeability
	self.parameters.get('P_omega', 0.01)  # Tissue permeability
	]
	
	for eq_idx in range(min(self.neq, len(permeabilities))):
	constraint_manager.add_kedem_katchalsky(
	equation_index=eq_idx,
	domain1_index=domain1_idx,
	domain2_index=domain2_idx,
	position1=junction_pos1,
	position2=junction_pos2,
	permeability=permeabilities[eq_idx]
	)
\end{lstlisting}

\subsection{Problem Validation}
\label{subsec:problem_validation}

The problem module includes validation methods to ensure mathematical consistency:

\begin{lstlisting}[language=Python, caption=Problem Validation]
	def validate(self):
	"""Validate problem definition"""
	errors = []
	
	# Check parameter consistency
	if self.neq <= 0:
	errors.append("Number of equations must be positive")
	
	if self.parameters.get('domain_length', 0) <= 0:
	errors.append("Domain length must be positive")
	
	# Check diffusion coefficients are non-negative
	for param in ['D_u', 'D_v', 'D_omega']:
	if param in self.parameters and self.parameters[param] < 0:
	errors.append(f"Diffusion coefficient {param} must be non-negative")
	
	# Check reaction rates
	for param in ['k_on', 'k_off', 'k_int', 'k_deg']:
	if param in self.parameters and self.parameters[param] < 0:
	errors.append(f"Reaction rate {param} must be non-negative")
	
	# Validate initial conditions
	try:
	test_x = np.linspace(0, self.parameters['domain_length'], 10)
	for eq_idx in range(self.neq):
	ic_vals = self.initial_condition(test_x, eq_idx)
	if np.any(np.isnan(ic_vals)) or np.any(np.isinf(ic_vals)):
	errors.append(f"Initial condition for equation {eq_idx} contains invalid values")
	except Exception as e:
	errors.append(f"Error evaluating initial conditions: {e}")
	
	if errors:
	raise ValueError("Problem validation failed:\n" + "\n".join(f"  - {err}" for err in errors))
	
	return True
\end{lstlisting}

\subsection{Problem Factory}
\label{subsec:problem_factory}

A factory system simplifies problem creation:

\begin{lstlisting}[language=Python, caption=Problem Factory]
	class ProblemFactory:
	"""Factory for creating problem instances"""
	
	@staticmethod
	def create_keller_segel(D_u=1.0, D_v=10.0, chi=5.0, 
	alpha=1.0, beta=1.0, domain_length=1.0):
	"""Create standard Keller-Segel problem"""
	params = {
		'D_u': D_u, 'D_v': D_v, 'chi': chi,
		'alpha': alpha, 'beta': beta,
		'domain_length': domain_length
	}
	return KellerSegelProblem(parameters=params)
	
	@staticmethod  
	def create_organ_on_chip(D_u=1.0, D_omega=0.1, k_on=2.0, k_off=0.5,
	k_int=1.0, k_deg=0.1, A=0.5, omega0=2*np.pi,
	phi=0.0, domain_length=2.0):
	"""Create standard organ-on-chip problem"""
	params = {
		'D_u': D_u, 'D_omega': D_omega,
		'k_on': k_on, 'k_off': k_off, 'k_int': k_int, 'k_deg': k_deg,
		'A': A, 'omega0': omega0, 'phi': phi,
		'domain_length': domain_length
	}
	return OrganOnChipProblem(parameters=params)
	
	# Usage
	ks_problem = ProblemFactory.create_keller_segel(chi=10.0, domain_length=2.0)
	ooc_problem = ProblemFactory.create_organ_on_chip(k_on=5.0, A=1.0)
\end{lstlisting}

\subsection{Integration with Framework}
\label{subsec:problem_integration}

Problems integrate seamlessly with other framework components:

\begin{itemize}
	\item \textbf{Geometry Module}: Problems are assigned to domains in network geometries
	\item \textbf{Discretization}: Spatial operators are constructed based on problem flux functions
	\item \textbf{Time Integration}: Temporal schemes use problem source terms and parameters
	\item \textbf{Constraint System}: Boundary and junction conditions are automatically enforced
	\item \textbf{Visualization}: Plot labels and physical interpretations are derived from problem types
\end{itemize}

The problem module provides the mathematical foundation that drives all numerical computations in BioNetFlux, ensuring physical consistency and mathematical rigor throughout the solution process.
% filepath: /Users/silviabertoluzza/GIT/BioNetFlux/docs/discretization_module.tex

\section{Discretization Module}
\label{sec:discretization_module}

The discretization module in BioNetFlux handles the spatial and temporal discretization of network domains. Based on the framework architecture, discretization is integrated into the global assembly process rather than being a standalone comprehensive finite element library.

\subsection{Overview}
\label{subsec:discretization_overview}

The discretization approach in BioNetFlux is designed for networks of 1D segments where:

\begin{itemize}
	\item Each domain represents a single 1D segment (vessel, channel, etc.)
	\item Domains are discretized independently using simple 1D meshes
	\item Inter-domain coupling occurs through trace variables at boundaries
	\item Global assembly coordinates multiple domains through the \texttt{GlobalAssembler}
	\item Time discretization is handled through implicit/explicit schemes in the Newton solver
\end{itemize}

\subsection{Domain Discretization Structure}
\label{subsec:domain_discretization}

From the \texttt{lean\_global\_assembly.py} implementation, we can see that discretization information is managed through domain data structures:

\subsubsection{Domain Data Management}

\begin{lstlisting}[language=Python, caption=Domain Discretization from Global Assembly]
	class GlobalAssembler:
	def __init__(self, domain_data_list: List, constraint_manager=None):
	self.bulk_manager = BulkDataManager(domain_data_list)
	self.n_domains = len(domain_data_list)
	
	# Each domain_data contains discretization information:
	# - n_elements: Number of finite elements
	# - neq: Number of equations per domain
	# - nodes: Mesh node information
	
	self._compute_dof_structure()
	
	def _compute_dof_structure(self):
	"""Compute DOF structure from domain data."""
	self.domain_trace_sizes = []
	self.domain_trace_offsets = []
	
	total_trace_dofs = 0
	for domain_data in self.bulk_manager.domain_data_list:
	# Nodes = elements + 1 for 1D linear elements
	n_nodes = domain_data.n_elements + 1
	# Trace size = equations × nodes
	trace_size = domain_data.neq * n_nodes
	
	self.domain_trace_sizes.append(trace_size)
	self.domain_trace_offsets.append(total_trace_dofs)
	total_trace_dofs += trace_size
\end{lstlisting}

Key observations:
\begin{itemize}
	\item \textbf{1D Linear Elements}: The framework uses $n\_nodes = n\_elements + 1$, indicating linear finite elements
	\item \textbf{Multi-equation Systems}: Each domain supports \texttt{neq} equations
	\item \textbf{Trace-based Coupling}: Primary DOFs are trace variables on domain boundaries
	\item \textbf{Global DOF Management}: Automatic assignment of global DOF indices
\end{itemize}

\subsubsection{Global Discretization Management}

The framework manages discretization across multiple domains:

\begin{lstlisting}[language=Python, caption=Global Discretization Coordination]
	# From the factory method in GlobalAssembler
	@classmethod
	def from_framework_objects(cls, problems: List, global_discretization, 
	static_condensations: List, constraint_manager=None):
	"""Create assembler from framework objects."""
	
	# Extract domain data from framework objects
	domain_data_list = BulkDataManager.extract_domain_data_list(
	problems, 
	global_discretization.spatial_discretizations, 
	static_condensations
	)
	
	return cls(domain_data_list, constraint_manager)
\end{lstlisting}

This indicates:
\begin{itemize}
	\item \textbf{GlobalDiscretization class}: Exists and contains \texttt{spatial\_discretizations}
	\item \textbf{Framework integration}: Problems, discretizations, and static condensations work together
	\item \textbf{Data extraction}: Domain data is extracted from discretization objects
\end{itemize}

\subsection{Spatial Discretization Features}
\label{subsec:spatial_features}

Based on the code structure, the spatial discretization supports:

\subsubsection{1D Finite Element Discretization}

\begin{lstlisting}[language=Python, caption=1D Element Structure (inferred)]
	# Each spatial discretization likely contains:
	class SpatialDiscretization:
	def __init__(self, domain_start, domain_end, n_elements):
	self.n_elements = n_elements
	self.nodes = self._generate_nodes(domain_start, domain_end, n_elements)
	# Linear elements: n_nodes = n_elements + 1
	
	def _generate_nodes(self, start, end, n_elements):
	"""Generate uniform node distribution."""
	return np.linspace(start, end, n_elements + 1)
\end{lstlisting}

\subsubsection{Multi-Equation Support}

From the DOF structure computation, each domain supports multiple equations:

\begin{lstlisting}[language=Python, caption=Multi-Equation DOF Layout]
	# For a domain with neq equations and n_nodes:
	# DOF layout: [eq0_node0, eq0_node1, ..., eq0_nodeN, 
	#              eq1_node0, eq1_node1, ..., eq1_nodeN,
	#              ...]
	# Total DOFs per domain = neq * n_nodes
	
	def extract_equation_dofs(trace_solution, equation_idx, n_nodes):
	"""Extract DOFs for specific equation."""
	eq_start = equation_idx * n_nodes
	eq_end = eq_start + n_nodes
	return trace_solution[eq_start:eq_end]
\end{lstlisting}

\subsection{Constraint-Based Inter-Domain Coupling}
\label{subsec:interdomain_coupling}

The discretization integrates with the constraint system for domain coupling:

\subsubsection{Trace Space Coupling}

\begin{lstlisting}[language=Python, caption=Constraint-Based Coupling]
	# From constraint Jacobian contributions in GlobalAssembler
	def _add_constraint_jacobian_contributions(self, jacobian, trace_solutions, 
	multipliers, time):
	"""Add constraint coupling terms."""
	
	for constraint in self.constraint_manager.constraints:
	if constraint.is_boundary_condition:
	# Single domain constraint
	domain_idx = constraint.domains[0]
	# Calculate global trace index
	domain_data = self.bulk_manager.domain_data_list[domain_idx]
	n_nodes = domain_data.n_elements + 1
	trace_idx = (domain_offset + 
	constraint.equation_index * n_nodes + 
	node_idx)
	else:
	# Junction constraint - couples two domains
	domain1_idx, domain2_idx = constraint.domains
	# Calculate trace indices for both domains
	# Add coupling terms to Jacobian
\end{lstlisting}

Key coupling features:
\begin{itemize}
	\item \textbf{Boundary Conditions}: Dirichlet, Neumann, Robin conditions on domain boundaries
	\item \textbf{Junction Conditions}: Trace continuity and Kedem-Katchalsky conditions between domains
	\item \textbf{Lagrange Multipliers}: Used for constraint enforcement
	\item \textbf{Automatic Indexing}: Global DOF indices computed automatically
\end{itemize}

\subsection{Temporal Discretization}
\label{subsec:temporal_discretization}

While not explicitly shown in the assembly code, temporal discretization is handled through:

\subsubsection{Implicit Time Integration}

\begin{lstlisting}[language=Python, caption=Time Integration Framework]
	# From the Newton solver approach in examples
	def advance_solution(current_solution, dt, time):
	"""Advance solution by one time step."""
	
	# Newton iteration for implicit time step
	newton_solution = current_solution.copy()
	
	for newton_iter in range(max_newton_iterations):
	# Assemble residual and Jacobian at current time
	residual, jacobian = global_assembler.assemble_residual_and_jacobian(
	global_solution=newton_solution,
	forcing_terms=forcing_terms,
	static_condensations=static_condensations,
	time=time
	)
	
	# Check convergence
	if np.linalg.norm(residual) < tolerance:
	break
	
	# Newton update
	delta = np.linalg.solve(jacobian, -residual)
	newton_solution += delta
	
	return newton_solution
\end{lstlisting}

\subsubsection{Static Condensation Integration}

The framework uses static condensation to eliminate bulk DOFs:

\begin{lstlisting}[language=Python, caption=Static Condensation for Time Integration]
	# From bulk_by_static_condensation method
	def bulk_by_static_condensation(self, global_solution, forcing_terms, 
	static_condensations, time):
	"""Recover bulk solution from trace solution."""
	
	trace_solutions = self._extract_trace_solutions(global_solution)
	bulk_solution = []
	
	for i in range(self.n_domains):
	# Use static condensation to compute bulk from trace
	U, F, JF = domain_flux_jump(
	trace_solutions[i].reshape(-1, 1),
	forcing_terms[i],
	None, None,
	static_condensations[i]
	)
	bulk_solution.append(U)
	
	return bulk_solution
\end{lstlisting}

\subsection{Integration with Framework Components}
\label{subsec:framework_integration}

The discretization integrates with other framework components:

\subsubsection{Problem Module Integration}

\begin{lstlisting}[language=Python, caption=Problem-Discretization Interface]
	# Problems provide equation count and parameters
	def create_initial_guess_from_problems(self, problems, discretizations, time=0.0):
	"""Create initial conditions from problem definitions."""
	
	for i in range(self.n_domains):
	problem = problems[i]
	discretization = discretizations[i]
	
	# Use problem.neq to determine equation count
	for eq in range(problem.neq):
	# Use problem initial conditions if available
	if hasattr(domain_data, 'initial_conditions'):
	initial_values = domain_data.initial_conditions[eq](nodes, time)
\end{lstlisting}

\subsubsection{Bulk Data Manager Integration}

\begin{lstlisting}[language=Python, caption=Bulk Data Integration]
	# BulkDataManager handles discretization data
	def initialize_bulk_data(self, problems, discretizations, time=0.0):
	"""Initialize bulk data using discretization information."""
	return self.bulk_manager.initialize_all_bulk_data(
	problems, discretizations, time
	)
	
	def compute_forcing_terms(self, bulk_data_list, problems, 
	discretizations, time, dt):
	"""Compute forcing terms using discretization."""
	return self.bulk_manager.compute_forcing_terms(
	bulk_data_list, problems, discretizations, time, dt
	)
\end{lstlisting}

\subsection{Limitations and Extensions}
\label{subsec:limitations}

Based on the current implementation:

\subsubsection{Current Limitations}

\begin{itemize}
	\item \textbf{Linear Elements Only}: The $n\_nodes = n\_elements + 1$ relationship suggests only linear elements
	\item \textbf{Uniform Meshes}: No evidence of adaptive mesh refinement
	\item \textbf{Simple 1D}: Limited to one-dimensional segments
	\item \textbf{No Higher-Order}: No quadratic or higher-order elements visible
\end{itemize}

\subsubsection{Extension Points}

The framework architecture allows for extensions:

\begin{itemize}
	\item \textbf{Domain Data Structure}: Can be extended with more discretization information
	\item \textbf{Constraint System}: Already supports complex coupling conditions
	\item \textbf{Static Condensation}: Framework for eliminating internal DOFs
	\item \textbf{Modular Assembly}: Components can be replaced or extended
\end{itemize}

The discretization module in BioNetFlux provides a focused, efficient approach to handling 1D network problems with multi-equation systems and complex inter-domain coupling through constraints and static condensation.


\section{Example Applications}

\subsection{Example 1: Simple Keller-Segel Chain}

\begin{lstlisting}[language=Python, caption={Simple Keller-Segel Example}]
# File: examples/simple_keller_segel.py
import sys
sys.path.insert(0, '../code')

from setup_solver import quick_setup
from ooc1d.visualization.lean_matplotlib_plotter import LeanMatplotlibPlotter

def main():
    # Setup problem
    setup = quick_setup("ooc1d.problems.KS_with_geometry", validate=True)
    
    # Get initial conditions
    trace_solutions, multipliers = setup.create_initial_conditions()
    
    # Initialize plotter
    plotter = LeanMatplotlibPlotter(
        problems=setup.problems,
        discretizations=setup.global_discretization.spatial_discretizations
    )
    
    # Plot initial state
    plotter.plot_2d_curves(trace_solutions, title="Initial State")
    plotter.plot_birdview(trace_solutions, equation_idx=0, time=0.0)
    
    # Time evolution
    dt = setup.global_discretization.dt
    T = 0.5
    current_time = 0.0
    global_solution = setup.create_global_solution_vector(
        trace_solutions, multipliers)
    
    while current_time < T:
        # Newton iteration (simplified)
        current_time += dt
        # ... solver steps ...
        
        # Extract solutions
        final_traces, _ = setup.extract_domain_solutions(global_solution)
        
        # Visualize
        plotter.plot_birdview(final_traces, equation_idx=0, 
                             time=current_time)
    
    plotter.show_all()

if __name__ == "__main__":
    main()
\end{lstlisting}

\subsection{Example 2: Complex Grid Network}

\begin{lstlisting}[language=Python, caption={Grid Network Example}]
# File: examples/grid_network_example.py
import sys
sys.path.insert(0, '../code')

from setup_solver import quick_setup
from ooc1d.visualization.lean_matplotlib_plotter import LeanMatplotlibPlotter

def main():
    # Load complex grid problem
    setup = quick_setup("ooc1d.problems.KS_grid_geometry", validate=True)
    
    print(f"Problem: {setup.get_problem_info()['problem_name']}")
    print(f"Domains: {setup.get_problem_info()['num_domains']}")
    
    # Initial conditions
    trace_solutions, multipliers = setup.create_initial_conditions()
    
    # Visualization
    plotter = LeanMatplotlibPlotter(
        problems=setup.problems,
        discretizations=setup.global_discretization.spatial_discretizations,
        figsize=(15, 10)
    )
    
    # Multiple views of initial state
    plotter.plot_2d_curves(
        trace_solutions, 
        title="Grid Network - Domain Profiles",
        save_filename="grid_profiles.png"
    )
    
    for eq_idx in range(2):  # Both equations
        plotter.plot_flat_3d(
            trace_solutions,
            equation_idx=eq_idx,
            title=f"Grid Network - {plotter.equation_names[eq_idx]} (3D)",
            save_filename=f"grid_3d_eq{eq_idx}.png"
        )
        
        plotter.plot_birdview(
            trace_solutions,
            equation_idx=eq_idx,
            time=0.0,
            save_filename=f"grid_birdview_eq{eq_idx}.png"
        )
    
    plotter.show_all()

if __name__ == "__main__":
    main()
\end{lstlisting}

\section{API Reference}

\subsection{Quick Setup Function}

\begin{lstlisting}[language=Python, caption={Quick Setup API}]
setup_solver.quick_setup(problem_module: str, 
                         validate: bool = True) -> SolverSetup
\end{lstlisting}

\textbf{Parameters:}
\begin{itemize}
    \item \code{problem\_module}: Import path to problem definition (e.g., "ooc1d.problems.my\_problem")
    \item \code{validate}: Whether to validate setup after creation
\end{itemize}

\textbf{Returns:} Configured \code{SolverSetup} object

\subsection{SolverSetup Class}

\begin{lstlisting}[language=Python, caption={SolverSetup API}]
class SolverSetup:
    def get_problem_info() -> Dict[str, Any]
    def create_initial_conditions() -> Tuple[List[np.ndarray], np.ndarray]
    def create_global_solution_vector(traces, multipliers) -> np.ndarray
    def extract_domain_solutions(global_solution) -> Tuple[List[np.ndarray], 
                                                           np.ndarray]
\end{lstlisting}

\subsection{DomainGeometry Class}

\begin{lstlisting}[language=Python, caption={DomainGeometry API}]
class DomainGeometry:
    def add_domain(extrema_start: Tuple[float, float],
                   extrema_end: Tuple[float, float],
                   domain_start: float = None,
                   domain_length: float = None,
                   name: str = None,
                   **metadata) -> int
    
    def get_domain(domain_id: int) -> DomainInfo
    def get_bounding_box() -> Dict[str, float]
    def num_domains() -> int
    def summary() -> str
\end{lstlisting}

\subsection{LeanMatplotlibPlotter Class}

\begin{lstlisting}[language=Python, caption={Plotter API}]
class LeanMatplotlibPlotter:
    def __init__(problems, discretizations, 
                 equation_names=None, figsize=(12,8))
    
    def plot_2d_curves(trace_solutions, title, 
                       show_mesh_points=True,
                       save_filename=None) -> plt.Figure
    
    def plot_flat_3d(trace_solutions, equation_idx=0, 
                     view_angle=(30,45),
                     save_filename=None) -> plt.Figure
    
    def plot_birdview(trace_solutions, equation_idx=0, 
                      time=0.0,
                      save_filename=None) -> plt.Figure
    
    def plot_comparison(initial_traces, final_traces, 
                        initial_time=0.0,
                        final_time=1.0, 
                        save_filename=None) -> plt.Figure
\end{lstlisting}

\section{Troubleshooting}

\subsection{Common Issues}

\subsubsection{Import Errors}
\begin{lstlisting}[language=Python, caption={Path Setup}]
# Ensure correct path setup
import sys
sys.path.insert(0, '/path/to/BioNetFlux/code')
\end{lstlisting}

\subsubsection{Geometry Validation}
\begin{lstlisting}[language=Python, caption={Geometry Debugging}]
# Check geometry before problem creation
geometry = DomainGeometry("test")
# ... add domains ...
print(geometry.summary())  # Verify domain layout
print(geometry.get_bounding_box())  # Check coordinates
\end{lstlisting}

\subsubsection{Constraint Setup}
\begin{lstlisting}[language=Python, caption={Constraint Verification}]
# Verify constraint mapping
constraint_manager.map_to_discretizations(discretizations)
print(f"Total constraints: {constraint_manager.n_multipliers}")
\end{lstlisting}

\subsubsection{Solution Convergence}
\begin{lstlisting}[language=Python, caption={Convergence Monitoring}]
# Monitor Newton iteration
newton_tolerance = 1e-10
max_newton_iterations = 20

# Check residual norms during iteration
if residual_norm > newton_tolerance:
    print(f"Convergence issue: residual = {residual_norm:.2e}")
\end{lstlisting}

\subsection{Performance Optimization}

\begin{enumerate}
    \item \textbf{Mesh Resolution}: Balance accuracy vs. computational cost
    \item \textbf{Time Step Size}: Use adaptive time stepping for stability
    \item \textbf{Newton Tolerance}: Adjust based on problem requirements
    \item \textbf{Domain Decomposition}: Optimize domain sizes for load balancing
\end{enumerate}

\subsection{Debugging Tips}

\begin{enumerate}
    \item \textbf{Visualization}: Use all three plot types to understand solution behavior
    \item \textbf{Parameter Validation}: Check physical parameter ranges
    \item \textbf{Constraint Verification}: Ensure proper interface connectivity
    \item \textbf{Solution Monitoring}: Track solution norms and residuals
\end{enumerate}

\section{Contact and Support}

For questions, issues, or contributions:

\begin{itemize}
    \item \textbf{Repository}: [\bionetflux{} GitHub]
    \item \textbf{Documentation}: See \code{docs/} directory
    \item \textbf{Examples}: See \code{examples/} directory
    \item \textbf{Issues}: Submit via GitHub Issues
\end{itemize}

\vspace{2cm}

\begin{center}
\textbf{\bionetflux{} Development Team} \\
\textit{Multi-Domain Biological Network Flow Simulation Framework}
\end{center}

\end{document}
