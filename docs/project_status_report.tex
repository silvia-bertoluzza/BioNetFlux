% BioNetFlux Project Status Report
% To be included in master LaTeX document
%
% Usage: % BioNetFlux Project Status Report
% To be included in master LaTeX document
%
% Usage: % BioNetFlux Project Status Report
% To be included in master LaTeX document
%
% Usage: % BioNetFlux Project Status Report
% To be included in master LaTeX document
%
% Usage: \input{docs/project_status_report}

\section{BioNetFlux: Current Project Status}

\subsection{Project Overview}

\textbf{BioNetFlux} is a comprehensive Python framework for solving biological transport phenomena on network topologies using Hybridizable Discontinuous Galerkin (HDG) methods. The project targets multi-domain, multi-equation problems with particular emphasis on biological applications including chemotaxis, vascular networks, and organ-on-chip systems.

\subsubsection{Core Framework Features}
\begin{itemize}
    \item \textbf{Multi-Domain Support}: Native handling of connected 1D domain segments
    \item \textbf{Multi-Equation Systems}: Simultaneous solution of coupled PDEs (up to 4 equations demonstrated)
    \item \textbf{HDG Discretization}: Advanced discontinuous Galerkin methods with static condensation
    \item \textbf{Constraint Management}: Comprehensive boundary condition and junction constraint handling
    \item \textbf{Time Evolution}: Implicit Euler time stepping with Newton iteration
\end{itemize}

\subsection{Target Applications}

BioNetFlux specifically addresses the following biological and biomedical applications:

\begin{description}
    \item[Keller-Segel Chemotaxis] Cell migration and pattern formation with nonlinear chemotactic response
    \item[Organ-on-Chip Systems] Multi-component transport in microfluidic biological models
    \item[Vascular Networks] Blood flow and nutrient transport in circulatory systems
    \item[Neural Networks] Signal propagation and neurotransmitter diffusion
    \item[Root Systems] Water and nutrient uptake in plant vascular networks
    \item[Microbial Networks] Biofilm formation and quorum sensing mechanisms
\end{description}

\subsection{Mathematical Framework}

\subsubsection{Organ-on-Chip Model Implementation}

The current implementation focuses on a 4-equation organ-on-chip system derived from the MATLAB reference:

\begin{align}
\frac{\partial u}{\partial t} - \nu \nabla^2 u &= f_u \label{eq:ooc_u} \\
\frac{\partial \omega}{\partial t} + \epsilon \nabla \cdot \theta + c\omega &= f_\omega + d u \label{eq:ooc_omega} \\
\frac{\partial v}{\partial t} + \sigma \nabla \cdot q + \lambda(\bar{\omega}) v &= f_v \label{eq:ooc_v} \\
\frac{\partial \phi}{\partial t} + \mu \nabla \cdot \psi + a\phi &= f_\phi + b v \label{eq:ooc_phi}
\end{align}

with auxiliary relations:
\begin{align}
\theta &= \epsilon(\nabla \omega - \hat{\omega}) \\
q &= \sigma(\nabla v - \hat{v}) \\
\psi &= \mu(\nabla \phi - \hat{\phi})
\end{align}

\textbf{Physical Parameters:}
\begin{itemize}
    \item $\nu, \mu$: Viscosity parameters
    \item $\epsilon, \sigma$: Coupling coefficients  
    \item $a, c$: Reaction rates
    \item $b, d$: Inter-equation coupling strengths
    \item $\chi$: Chemotactic sensitivity
    \item $\lambda(\cdot)$: Nonlinear response function
\end{itemize}

\subsubsection{Static Condensation Implementation}

The HDG method employs static condensation following the MATLAB \texttt{StaticC.m} reference:

\begin{enumerate}
    \item \textbf{Trace-to-Solution Mapping}: $\hat{U} \rightarrow U$
    \begin{align}
        u &= B_1 \hat{u} + y_1 \\
        \omega &= C_2 \hat{u} + B_2 \hat{\omega} + y_2 \\
        v &= B_3(\lambda(\bar{\omega})) \hat{v} + y_3(\lambda(\bar{\omega})) \\
        \phi &= B_4 \hat{\phi} + C_4 v + L_4 g_\phi
    \end{align}
    
    \item \textbf{Flux Jump Construction}: $\tilde{J} = D_1 U - D_2 \hat{U}$
    
    \item \textbf{Normal Flux Computation}: 
    \begin{equation}
        j = \hat{B}_4 \hat{U} + \tilde{J}^T Q U
    \end{equation}
    
    \item \textbf{Final Assembly}:
    \begin{align}
        \hat{j} &= B_5 j + B_6 U + B_7 \hat{U} \\
        \hat{J} &= \hat{B}_0 \tilde{J} + \hat{B}_1 U - \hat{B}_2 \hat{U}
    \end{align}
\end{enumerate}

\subsection{Software Architecture}

\subsubsection{Core Components}

\begin{description}
    \item[\texttt{Problem}] Encapsulates PDE specifications, parameters, and boundary conditions
    \item[\texttt{Discretization}] Manages spatial mesh, time stepping, and stabilization parameters
    \item[\texttt{StaticCondensationOOC}] Implements organ-on-chip static condensation following MATLAB reference
    \item[\texttt{GlobalAssembler}] Coordinates multi-domain assembly and Newton iteration
    \item[\texttt{ConstraintManager}] Handles boundary conditions and junction constraints
    \item[\texttt{BulkDataManager}] Manages bulk solution data and forcing term computation
\end{description}

\subsubsection{Advanced Features}

\paragraph{Multi-Domain Visualization}
The \texttt{MultiDomainPlotter} class provides comprehensive visualization capabilities:
\begin{itemize}
    \item Continuous solution plots across domain boundaries
    \item Domain-wise comparison views
    \item Solution evolution tracking
    \item Automatic figure size optimization for screen fitting
    \item Statistical analysis and interface highlighting
\end{itemize}

\paragraph{Constraint Management}
Supports multiple constraint types:
\begin{itemize}
    \item Dirichlet boundary conditions
    \item Neumann flux specifications
    \item Junction continuity constraints
    \item Kirchhoff-Kedem conditions for network flows
\end{itemize}

\subsection{Current Implementation Status}

\subsubsection{Completed Components (\textcolor{green}{\checkmark})}

\begin{itemize}
    \item \textbf{Core Framework}: Full HDG implementation with static condensation
    \item \textbf{Organ-on-Chip Model}: 4-equation system with MATLAB compatibility
    \item \textbf{Time Evolution}: Implicit Euler with Newton solver
    \item \textbf{Multi-Domain Support}: Connected domain handling with junction constraints
    \item \textbf{Visualization System}: Comprehensive plotting with automatic sizing
    \item \textbf{Bulk Data Management}: Efficient data handling and forcing term computation
    \item \textbf{Elementary Matrices}: Pre-computed basis function integrals
    \item \textbf{Constraint System}: Boundary conditions and multi-domain coupling
\end{itemize}

\subsubsection{Active Development Areas}

\paragraph{High Priority Items}
\begin{enumerate}
    \item \textbf{Picard Iteration Methods}: Implementation of fixed-point iteration for arbitrary nonlinearities
    \begin{itemize}
        \item Design framework for general nonlinear problems
        \item Convergence acceleration techniques
        \item Hybrid Picard-Newton strategies
    \end{itemize}
    
    \item \textbf{Adaptive Time Stepping}: Dynamic time step control
    \begin{itemize}
        \item Embedded Runge-Kutta error estimation
        \item Automatic step size adjustment
        \item Stability-based limiters
    \end{itemize}
    
    \item \textbf{Non-Uniform Mesh Support}: Variable element sizing
    \begin{itemize}
        \item Adaptive mesh refinement (AMR)
        \item Mesh grading and clustering
        \item Load balancing for multi-domain systems
    \end{itemize}
\end{enumerate}

\paragraph{Medium Priority Development}
\begin{itemize}
    \item \textbf{Non-Homogeneous Boundary Conditions}: Time-dependent Dirichlet/Neumann data
    \item \textbf{Multi-Domain Testing}: Validation with complex junction networks
    \item \textbf{Performance Optimization}: Profiling and computational efficiency improvements
    \item \textbf{Mass Conservation Tracking}: Global conservation monitoring during time evolution
\end{itemize}

\subsubsection{Testing and Validation}

\paragraph{Verification Status}
\begin{itemize}
    \item \textbf{Single Domain}: Fully validated against analytical solutions
    \item \textbf{Static Condensation}: Verified against MATLAB reference implementation
    \item \textbf{Time Evolution}: Working for linear and mildly nonlinear problems
    \item \textbf{Visualization}: Comprehensive testing with multiple equation systems
    \item \textbf{Elementary Matrices}: Validated through extensive unit testing
\end{itemize}

\paragraph{Known Issues}
\begin{itemize}
    \item Junction conditions in T-junction examples require further investigation
    \item Multi-domain time evolution needs additional validation
    \item Large-scale performance optimization pending
\end{itemize}

\subsection{Technical Specifications}

\subsubsection{Dependencies and Requirements}
\begin{itemize}
    \item \textbf{Python}: 3.8+ with NumPy, SciPy, Matplotlib
    \item \textbf{Numerical Libraries}: Optimized linear algebra (BLAS/LAPACK)
    \item \textbf{Visualization}: Matplotlib with optional animation support
    \item \textbf{Development}: Git version control, pytest testing framework
\end{itemize}

\subsubsection{Performance Characteristics}
\begin{itemize}
    \item \textbf{Problem Size}: Successfully tested up to 4 equations, 40 elements per domain
    \item \textbf{Time Stepping}: Implicit Euler with Newton convergence in 1-5 iterations
    \item \textbf{Memory Usage}: Lean architecture with minimal memory overhead
    \item \textbf{Computational Complexity}: $O(N^3)$ for Newton linear solves, $O(N)$ for assembly
\end{itemize}

\subsection{Future Roadmap}

\subsubsection{Short-term Goals (3-6 months)}
\begin{enumerate}
    \item Complete Picard iteration implementation for general nonlinearities
    \item Implement adaptive time stepping with error control
    \item Add non-uniform mesh support with basic AMR
    \item Validate multi-domain problems with complex junction networks
    \item Performance optimization and large-scale testing
\end{enumerate}

\subsubsection{Medium-term Objectives (6-12 months)}
\begin{enumerate}
    \item Extend to 2D/3D network topologies
    \item Add support for moving boundaries and dynamic meshes
    \item Implement parallel computing capabilities
    \item Develop biological application-specific modules
    \item Create comprehensive documentation and user guides
\end{enumerate}

\subsubsection{Long-term Vision (1-2 years)}
\begin{enumerate}
    \item Integration with experimental data and validation
    \item Machine learning-enhanced parameter estimation
    \item Real-time simulation capabilities for organ-on-chip devices
    \item Commercial deployment for biomedical applications
    \item Open-source community development and contributions
\end{enumerate}

\subsection{Project Impact and Applications}

BioNetFlux represents a significant advancement in computational biology by providing:

\begin{itemize}
    \item \textbf{Unified Framework}: Single platform for diverse biological transport problems
    \item \textbf{High-Order Accuracy}: HDG methods provide superior accuracy compared to traditional FEM
    \item \textbf{Scalability}: Multi-domain architecture supports complex network topologies
    \item \textbf{Flexibility}: Modular design allows easy extension to new biological models
    \item \textbf{Validation}: MATLAB reference compatibility ensures numerical reliability
\end{itemize}

The framework is positioned to become a valuable tool for researchers in computational biology, biomedical engineering, and pharmaceutical development, particularly for organ-on-chip technology and biological network modeling.

% End of project status report


\section{BioNetFlux: Current Project Status}

\subsection{Project Overview}

\textbf{BioNetFlux} is a comprehensive Python framework for solving biological transport phenomena on network topologies using Hybridizable Discontinuous Galerkin (HDG) methods. The project targets multi-domain, multi-equation problems with particular emphasis on biological applications including chemotaxis, vascular networks, and organ-on-chip systems.

\subsubsection{Core Framework Features}
\begin{itemize}
    \item \textbf{Multi-Domain Support}: Native handling of connected 1D domain segments
    \item \textbf{Multi-Equation Systems}: Simultaneous solution of coupled PDEs (up to 4 equations demonstrated)
    \item \textbf{HDG Discretization}: Advanced discontinuous Galerkin methods with static condensation
    \item \textbf{Constraint Management}: Comprehensive boundary condition and junction constraint handling
    \item \textbf{Time Evolution}: Implicit Euler time stepping with Newton iteration
\end{itemize}

\subsection{Target Applications}

BioNetFlux specifically addresses the following biological and biomedical applications:

\begin{description}
    \item[Keller-Segel Chemotaxis] Cell migration and pattern formation with nonlinear chemotactic response
    \item[Organ-on-Chip Systems] Multi-component transport in microfluidic biological models
    \item[Vascular Networks] Blood flow and nutrient transport in circulatory systems
    \item[Neural Networks] Signal propagation and neurotransmitter diffusion
    \item[Root Systems] Water and nutrient uptake in plant vascular networks
    \item[Microbial Networks] Biofilm formation and quorum sensing mechanisms
\end{description}

\subsection{Mathematical Framework}

\subsubsection{Organ-on-Chip Model Implementation}

The current implementation focuses on a 4-equation organ-on-chip system derived from the MATLAB reference:

\begin{align}
\frac{\partial u}{\partial t} - \nu \nabla^2 u &= f_u \label{eq:ooc_u} \\
\frac{\partial \omega}{\partial t} + \epsilon \nabla \cdot \theta + c\omega &= f_\omega + d u \label{eq:ooc_omega} \\
\frac{\partial v}{\partial t} + \sigma \nabla \cdot q + \lambda(\bar{\omega}) v &= f_v \label{eq:ooc_v} \\
\frac{\partial \phi}{\partial t} + \mu \nabla \cdot \psi + a\phi &= f_\phi + b v \label{eq:ooc_phi}
\end{align}

with auxiliary relations:
\begin{align}
\theta &= \epsilon(\nabla \omega - \hat{\omega}) \\
q &= \sigma(\nabla v - \hat{v}) \\
\psi &= \mu(\nabla \phi - \hat{\phi})
\end{align}

\textbf{Physical Parameters:}
\begin{itemize}
    \item $\nu, \mu$: Viscosity parameters
    \item $\epsilon, \sigma$: Coupling coefficients  
    \item $a, c$: Reaction rates
    \item $b, d$: Inter-equation coupling strengths
    \item $\chi$: Chemotactic sensitivity
    \item $\lambda(\cdot)$: Nonlinear response function
\end{itemize}

\subsubsection{Static Condensation Implementation}

The HDG method employs static condensation following the MATLAB \texttt{StaticC.m} reference:

\begin{enumerate}
    \item \textbf{Trace-to-Solution Mapping}: $\hat{U} \rightarrow U$
    \begin{align}
        u &= B_1 \hat{u} + y_1 \\
        \omega &= C_2 \hat{u} + B_2 \hat{\omega} + y_2 \\
        v &= B_3(\lambda(\bar{\omega})) \hat{v} + y_3(\lambda(\bar{\omega})) \\
        \phi &= B_4 \hat{\phi} + C_4 v + L_4 g_\phi
    \end{align}
    
    \item \textbf{Flux Jump Construction}: $\tilde{J} = D_1 U - D_2 \hat{U}$
    
    \item \textbf{Normal Flux Computation}: 
    \begin{equation}
        j = \hat{B}_4 \hat{U} + \tilde{J}^T Q U
    \end{equation}
    
    \item \textbf{Final Assembly}:
    \begin{align}
        \hat{j} &= B_5 j + B_6 U + B_7 \hat{U} \\
        \hat{J} &= \hat{B}_0 \tilde{J} + \hat{B}_1 U - \hat{B}_2 \hat{U}
    \end{align}
\end{enumerate}

\subsection{Software Architecture}

\subsubsection{Core Components}

\begin{description}
    \item[\texttt{Problem}] Encapsulates PDE specifications, parameters, and boundary conditions
    \item[\texttt{Discretization}] Manages spatial mesh, time stepping, and stabilization parameters
    \item[\texttt{StaticCondensationOOC}] Implements organ-on-chip static condensation following MATLAB reference
    \item[\texttt{GlobalAssembler}] Coordinates multi-domain assembly and Newton iteration
    \item[\texttt{ConstraintManager}] Handles boundary conditions and junction constraints
    \item[\texttt{BulkDataManager}] Manages bulk solution data and forcing term computation
\end{description}

\subsubsection{Advanced Features}

\paragraph{Multi-Domain Visualization}
The \texttt{MultiDomainPlotter} class provides comprehensive visualization capabilities:
\begin{itemize}
    \item Continuous solution plots across domain boundaries
    \item Domain-wise comparison views
    \item Solution evolution tracking
    \item Automatic figure size optimization for screen fitting
    \item Statistical analysis and interface highlighting
\end{itemize}

\paragraph{Constraint Management}
Supports multiple constraint types:
\begin{itemize}
    \item Dirichlet boundary conditions
    \item Neumann flux specifications
    \item Junction continuity constraints
    \item Kirchhoff-Kedem conditions for network flows
\end{itemize}

\subsection{Current Implementation Status}

\subsubsection{Completed Components (\textcolor{green}{\checkmark})}

\begin{itemize}
    \item \textbf{Core Framework}: Full HDG implementation with static condensation
    \item \textbf{Organ-on-Chip Model}: 4-equation system with MATLAB compatibility
    \item \textbf{Time Evolution}: Implicit Euler with Newton solver
    \item \textbf{Multi-Domain Support}: Connected domain handling with junction constraints
    \item \textbf{Visualization System}: Comprehensive plotting with automatic sizing
    \item \textbf{Bulk Data Management}: Efficient data handling and forcing term computation
    \item \textbf{Elementary Matrices}: Pre-computed basis function integrals
    \item \textbf{Constraint System}: Boundary conditions and multi-domain coupling
\end{itemize}

\subsubsection{Active Development Areas}

\paragraph{High Priority Items}
\begin{enumerate}
    \item \textbf{Picard Iteration Methods}: Implementation of fixed-point iteration for arbitrary nonlinearities
    \begin{itemize}
        \item Design framework for general nonlinear problems
        \item Convergence acceleration techniques
        \item Hybrid Picard-Newton strategies
    \end{itemize}
    
    \item \textbf{Adaptive Time Stepping}: Dynamic time step control
    \begin{itemize}
        \item Embedded Runge-Kutta error estimation
        \item Automatic step size adjustment
        \item Stability-based limiters
    \end{itemize}
    
    \item \textbf{Non-Uniform Mesh Support}: Variable element sizing
    \begin{itemize}
        \item Adaptive mesh refinement (AMR)
        \item Mesh grading and clustering
        \item Load balancing for multi-domain systems
    \end{itemize}
\end{enumerate}

\paragraph{Medium Priority Development}
\begin{itemize}
    \item \textbf{Non-Homogeneous Boundary Conditions}: Time-dependent Dirichlet/Neumann data
    \item \textbf{Multi-Domain Testing}: Validation with complex junction networks
    \item \textbf{Performance Optimization}: Profiling and computational efficiency improvements
    \item \textbf{Mass Conservation Tracking}: Global conservation monitoring during time evolution
\end{itemize}

\subsubsection{Testing and Validation}

\paragraph{Verification Status}
\begin{itemize}
    \item \textbf{Single Domain}: Fully validated against analytical solutions
    \item \textbf{Static Condensation}: Verified against MATLAB reference implementation
    \item \textbf{Time Evolution}: Working for linear and mildly nonlinear problems
    \item \textbf{Visualization}: Comprehensive testing with multiple equation systems
    \item \textbf{Elementary Matrices}: Validated through extensive unit testing
\end{itemize}

\paragraph{Known Issues}
\begin{itemize}
    \item Junction conditions in T-junction examples require further investigation
    \item Multi-domain time evolution needs additional validation
    \item Large-scale performance optimization pending
\end{itemize}

\subsection{Technical Specifications}

\subsubsection{Dependencies and Requirements}
\begin{itemize}
    \item \textbf{Python}: 3.8+ with NumPy, SciPy, Matplotlib
    \item \textbf{Numerical Libraries}: Optimized linear algebra (BLAS/LAPACK)
    \item \textbf{Visualization}: Matplotlib with optional animation support
    \item \textbf{Development}: Git version control, pytest testing framework
\end{itemize}

\subsubsection{Performance Characteristics}
\begin{itemize}
    \item \textbf{Problem Size}: Successfully tested up to 4 equations, 40 elements per domain
    \item \textbf{Time Stepping}: Implicit Euler with Newton convergence in 1-5 iterations
    \item \textbf{Memory Usage}: Lean architecture with minimal memory overhead
    \item \textbf{Computational Complexity}: $O(N^3)$ for Newton linear solves, $O(N)$ for assembly
\end{itemize}

\subsection{Future Roadmap}

\subsubsection{Short-term Goals (3-6 months)}
\begin{enumerate}
    \item Complete Picard iteration implementation for general nonlinearities
    \item Implement adaptive time stepping with error control
    \item Add non-uniform mesh support with basic AMR
    \item Validate multi-domain problems with complex junction networks
    \item Performance optimization and large-scale testing
\end{enumerate}

\subsubsection{Medium-term Objectives (6-12 months)}
\begin{enumerate}
    \item Extend to 2D/3D network topologies
    \item Add support for moving boundaries and dynamic meshes
    \item Implement parallel computing capabilities
    \item Develop biological application-specific modules
    \item Create comprehensive documentation and user guides
\end{enumerate}

\subsubsection{Long-term Vision (1-2 years)}
\begin{enumerate}
    \item Integration with experimental data and validation
    \item Machine learning-enhanced parameter estimation
    \item Real-time simulation capabilities for organ-on-chip devices
    \item Commercial deployment for biomedical applications
    \item Open-source community development and contributions
\end{enumerate}

\subsection{Project Impact and Applications}

BioNetFlux represents a significant advancement in computational biology by providing:

\begin{itemize}
    \item \textbf{Unified Framework}: Single platform for diverse biological transport problems
    \item \textbf{High-Order Accuracy}: HDG methods provide superior accuracy compared to traditional FEM
    \item \textbf{Scalability}: Multi-domain architecture supports complex network topologies
    \item \textbf{Flexibility}: Modular design allows easy extension to new biological models
    \item \textbf{Validation}: MATLAB reference compatibility ensures numerical reliability
\end{itemize}

The framework is positioned to become a valuable tool for researchers in computational biology, biomedical engineering, and pharmaceutical development, particularly for organ-on-chip technology and biological network modeling.

% End of project status report


\section{BioNetFlux: Current Project Status}

\subsection{Project Overview}

\textbf{BioNetFlux} is a comprehensive Python framework for solving biological transport phenomena on network topologies using Hybridizable Discontinuous Galerkin (HDG) methods. The project targets multi-domain, multi-equation problems with particular emphasis on biological applications including chemotaxis, vascular networks, and organ-on-chip systems.

\subsubsection{Core Framework Features}
\begin{itemize}
    \item \textbf{Multi-Domain Support}: Native handling of connected 1D domain segments
    \item \textbf{Multi-Equation Systems}: Simultaneous solution of coupled PDEs (up to 4 equations demonstrated)
    \item \textbf{HDG Discretization}: Advanced discontinuous Galerkin methods with static condensation
    \item \textbf{Constraint Management}: Comprehensive boundary condition and junction constraint handling
    \item \textbf{Time Evolution}: Implicit Euler time stepping with Newton iteration
\end{itemize}

\subsection{Target Applications}

BioNetFlux specifically addresses the following biological and biomedical applications:

\begin{description}
    \item[Keller-Segel Chemotaxis] Cell migration and pattern formation with nonlinear chemotactic response
    \item[Organ-on-Chip Systems] Multi-component transport in microfluidic biological models
    \item[Vascular Networks] Blood flow and nutrient transport in circulatory systems
    \item[Neural Networks] Signal propagation and neurotransmitter diffusion
    \item[Root Systems] Water and nutrient uptake in plant vascular networks
    \item[Microbial Networks] Biofilm formation and quorum sensing mechanisms
\end{description}

\subsection{Mathematical Framework}

\subsubsection{Organ-on-Chip Model Implementation}

The current implementation focuses on a 4-equation organ-on-chip system derived from the MATLAB reference:

\begin{align}
\frac{\partial u}{\partial t} - \nu \nabla^2 u &= f_u \label{eq:ooc_u} \\
\frac{\partial \omega}{\partial t} + \epsilon \nabla \cdot \theta + c\omega &= f_\omega + d u \label{eq:ooc_omega} \\
\frac{\partial v}{\partial t} + \sigma \nabla \cdot q + \lambda(\bar{\omega}) v &= f_v \label{eq:ooc_v} \\
\frac{\partial \phi}{\partial t} + \mu \nabla \cdot \psi + a\phi &= f_\phi + b v \label{eq:ooc_phi}
\end{align}

with auxiliary relations:
\begin{align}
\theta &= \epsilon(\nabla \omega - \hat{\omega}) \\
q &= \sigma(\nabla v - \hat{v}) \\
\psi &= \mu(\nabla \phi - \hat{\phi})
\end{align}

\textbf{Physical Parameters:}
\begin{itemize}
    \item $\nu, \mu$: Viscosity parameters
    \item $\epsilon, \sigma$: Coupling coefficients  
    \item $a, c$: Reaction rates
    \item $b, d$: Inter-equation coupling strengths
    \item $\chi$: Chemotactic sensitivity
    \item $\lambda(\cdot)$: Nonlinear response function
\end{itemize}

\subsubsection{Static Condensation Implementation}

The HDG method employs static condensation following the MATLAB \texttt{StaticC.m} reference:

\begin{enumerate}
    \item \textbf{Trace-to-Solution Mapping}: $\hat{U} \rightarrow U$
    \begin{align}
        u &= B_1 \hat{u} + y_1 \\
        \omega &= C_2 \hat{u} + B_2 \hat{\omega} + y_2 \\
        v &= B_3(\lambda(\bar{\omega})) \hat{v} + y_3(\lambda(\bar{\omega})) \\
        \phi &= B_4 \hat{\phi} + C_4 v + L_4 g_\phi
    \end{align}
    
    \item \textbf{Flux Jump Construction}: $\tilde{J} = D_1 U - D_2 \hat{U}$
    
    \item \textbf{Normal Flux Computation}: 
    \begin{equation}
        j = \hat{B}_4 \hat{U} + \tilde{J}^T Q U
    \end{equation}
    
    \item \textbf{Final Assembly}:
    \begin{align}
        \hat{j} &= B_5 j + B_6 U + B_7 \hat{U} \\
        \hat{J} &= \hat{B}_0 \tilde{J} + \hat{B}_1 U - \hat{B}_2 \hat{U}
    \end{align}
\end{enumerate}

\subsection{Software Architecture}

\subsubsection{Core Components}

\begin{description}
    \item[\texttt{Problem}] Encapsulates PDE specifications, parameters, and boundary conditions
    \item[\texttt{Discretization}] Manages spatial mesh, time stepping, and stabilization parameters
    \item[\texttt{StaticCondensationOOC}] Implements organ-on-chip static condensation following MATLAB reference
    \item[\texttt{GlobalAssembler}] Coordinates multi-domain assembly and Newton iteration
    \item[\texttt{ConstraintManager}] Handles boundary conditions and junction constraints
    \item[\texttt{BulkDataManager}] Manages bulk solution data and forcing term computation
\end{description}

\subsubsection{Advanced Features}

\paragraph{Multi-Domain Visualization}
The \texttt{MultiDomainPlotter} class provides comprehensive visualization capabilities:
\begin{itemize}
    \item Continuous solution plots across domain boundaries
    \item Domain-wise comparison views
    \item Solution evolution tracking
    \item Automatic figure size optimization for screen fitting
    \item Statistical analysis and interface highlighting
\end{itemize}

\paragraph{Constraint Management}
Supports multiple constraint types:
\begin{itemize}
    \item Dirichlet boundary conditions
    \item Neumann flux specifications
    \item Junction continuity constraints
    \item Kirchhoff-Kedem conditions for network flows
\end{itemize}

\subsection{Current Implementation Status}

\subsubsection{Completed Components (\textcolor{green}{\checkmark})}

\begin{itemize}
    \item \textbf{Core Framework}: Full HDG implementation with static condensation
    \item \textbf{Organ-on-Chip Model}: 4-equation system with MATLAB compatibility
    \item \textbf{Time Evolution}: Implicit Euler with Newton solver
    \item \textbf{Multi-Domain Support}: Connected domain handling with junction constraints
    \item \textbf{Visualization System}: Comprehensive plotting with automatic sizing
    \item \textbf{Bulk Data Management}: Efficient data handling and forcing term computation
    \item \textbf{Elementary Matrices}: Pre-computed basis function integrals
    \item \textbf{Constraint System}: Boundary conditions and multi-domain coupling
\end{itemize}

\subsubsection{Active Development Areas}

\paragraph{High Priority Items}
\begin{enumerate}
    \item \textbf{Picard Iteration Methods}: Implementation of fixed-point iteration for arbitrary nonlinearities
    \begin{itemize}
        \item Design framework for general nonlinear problems
        \item Convergence acceleration techniques
        \item Hybrid Picard-Newton strategies
    \end{itemize}
    
    \item \textbf{Adaptive Time Stepping}: Dynamic time step control
    \begin{itemize}
        \item Embedded Runge-Kutta error estimation
        \item Automatic step size adjustment
        \item Stability-based limiters
    \end{itemize}
    
    \item \textbf{Non-Uniform Mesh Support}: Variable element sizing
    \begin{itemize}
        \item Adaptive mesh refinement (AMR)
        \item Mesh grading and clustering
        \item Load balancing for multi-domain systems
    \end{itemize}
\end{enumerate}

\paragraph{Medium Priority Development}
\begin{itemize}
    \item \textbf{Non-Homogeneous Boundary Conditions}: Time-dependent Dirichlet/Neumann data
    \item \textbf{Multi-Domain Testing}: Validation with complex junction networks
    \item \textbf{Performance Optimization}: Profiling and computational efficiency improvements
    \item \textbf{Mass Conservation Tracking}: Global conservation monitoring during time evolution
\end{itemize}

\subsubsection{Testing and Validation}

\paragraph{Verification Status}
\begin{itemize}
    \item \textbf{Single Domain}: Fully validated against analytical solutions
    \item \textbf{Static Condensation}: Verified against MATLAB reference implementation
    \item \textbf{Time Evolution}: Working for linear and mildly nonlinear problems
    \item \textbf{Visualization}: Comprehensive testing with multiple equation systems
    \item \textbf{Elementary Matrices}: Validated through extensive unit testing
\end{itemize}

\paragraph{Known Issues}
\begin{itemize}
    \item Junction conditions in T-junction examples require further investigation
    \item Multi-domain time evolution needs additional validation
    \item Large-scale performance optimization pending
\end{itemize}

\subsection{Technical Specifications}

\subsubsection{Dependencies and Requirements}
\begin{itemize}
    \item \textbf{Python}: 3.8+ with NumPy, SciPy, Matplotlib
    \item \textbf{Numerical Libraries}: Optimized linear algebra (BLAS/LAPACK)
    \item \textbf{Visualization}: Matplotlib with optional animation support
    \item \textbf{Development}: Git version control, pytest testing framework
\end{itemize}

\subsubsection{Performance Characteristics}
\begin{itemize}
    \item \textbf{Problem Size}: Successfully tested up to 4 equations, 40 elements per domain
    \item \textbf{Time Stepping}: Implicit Euler with Newton convergence in 1-5 iterations
    \item \textbf{Memory Usage}: Lean architecture with minimal memory overhead
    \item \textbf{Computational Complexity}: $O(N^3)$ for Newton linear solves, $O(N)$ for assembly
\end{itemize}

\subsection{Future Roadmap}

\subsubsection{Short-term Goals (3-6 months)}
\begin{enumerate}
    \item Complete Picard iteration implementation for general nonlinearities
    \item Implement adaptive time stepping with error control
    \item Add non-uniform mesh support with basic AMR
    \item Validate multi-domain problems with complex junction networks
    \item Performance optimization and large-scale testing
\end{enumerate}

\subsubsection{Medium-term Objectives (6-12 months)}
\begin{enumerate}
    \item Extend to 2D/3D network topologies
    \item Add support for moving boundaries and dynamic meshes
    \item Implement parallel computing capabilities
    \item Develop biological application-specific modules
    \item Create comprehensive documentation and user guides
\end{enumerate}

\subsubsection{Long-term Vision (1-2 years)}
\begin{enumerate}
    \item Integration with experimental data and validation
    \item Machine learning-enhanced parameter estimation
    \item Real-time simulation capabilities for organ-on-chip devices
    \item Commercial deployment for biomedical applications
    \item Open-source community development and contributions
\end{enumerate}

\subsection{Project Impact and Applications}

BioNetFlux represents a significant advancement in computational biology by providing:

\begin{itemize}
    \item \textbf{Unified Framework}: Single platform for diverse biological transport problems
    \item \textbf{High-Order Accuracy}: HDG methods provide superior accuracy compared to traditional FEM
    \item \textbf{Scalability}: Multi-domain architecture supports complex network topologies
    \item \textbf{Flexibility}: Modular design allows easy extension to new biological models
    \item \textbf{Validation}: MATLAB reference compatibility ensures numerical reliability
\end{itemize}

The framework is positioned to become a valuable tool for researchers in computational biology, biomedical engineering, and pharmaceutical development, particularly for organ-on-chip technology and biological network modeling.

% End of project status report


\section{BioNetFlux: Current Project Status}

\subsection{Project Overview}

\textbf{BioNetFlux} is a comprehensive Python framework for solving biological transport phenomena on network topologies using Hybridizable Discontinuous Galerkin (HDG) methods. The project targets multi-domain, multi-equation problems with particular emphasis on biological applications including chemotaxis, vascular networks, and organ-on-chip systems.

\subsubsection{Core Framework Features}
\begin{itemize}
    \item \textbf{Multi-Domain Support}: Native handling of connected 1D domain segments
    \item \textbf{Multi-Equation Systems}: Simultaneous solution of coupled PDEs (up to 4 equations demonstrated)
    \item \textbf{HDG Discretization}: Advanced discontinuous Galerkin methods with static condensation
    \item \textbf{Constraint Management}: Comprehensive boundary condition and junction constraint handling
    \item \textbf{Time Evolution}: Implicit Euler time stepping with Newton iteration
\end{itemize}

\subsection{Target Applications}

BioNetFlux specifically addresses the following biological and biomedical applications:

\begin{description}
    \item[Keller-Segel Chemotaxis] Cell migration and pattern formation with nonlinear chemotactic response
    \item[Organ-on-Chip Systems] Multi-component transport in microfluidic biological models
    \item[Vascular Networks] Blood flow and nutrient transport in circulatory systems
    \item[Neural Networks] Signal propagation and neurotransmitter diffusion
    \item[Root Systems] Water and nutrient uptake in plant vascular networks
    \item[Microbial Networks] Biofilm formation and quorum sensing mechanisms
\end{description}

\subsection{Mathematical Framework}

\subsubsection{Organ-on-Chip Model Implementation}

The current implementation focuses on a 4-equation organ-on-chip system derived from the MATLAB reference:

\begin{align}
\frac{\partial u}{\partial t} - \nu \nabla^2 u &= f_u \label{eq:ooc_u} \\
\frac{\partial \omega}{\partial t} + \epsilon \nabla \cdot \theta + c\omega &= f_\omega + d u \label{eq:ooc_omega} \\
\frac{\partial v}{\partial t} + \sigma \nabla \cdot q + \lambda(\bar{\omega}) v &= f_v \label{eq:ooc_v} \\
\frac{\partial \phi}{\partial t} + \mu \nabla \cdot \psi + a\phi &= f_\phi + b v \label{eq:ooc_phi}
\end{align}

with auxiliary relations:
\begin{align}
\theta &= \epsilon(\nabla \omega - \hat{\omega}) \\
q &= \sigma(\nabla v - \hat{v}) \\
\psi &= \mu(\nabla \phi - \hat{\phi})
\end{align}

\textbf{Physical Parameters:}
\begin{itemize}
    \item $\nu, \mu$: Viscosity parameters
    \item $\epsilon, \sigma$: Coupling coefficients  
    \item $a, c$: Reaction rates
    \item $b, d$: Inter-equation coupling strengths
    \item $\chi$: Chemotactic sensitivity
    \item $\lambda(\cdot)$: Nonlinear response function
\end{itemize}

\subsubsection{Static Condensation Implementation}

The HDG method employs static condensation following the MATLAB \texttt{StaticC.m} reference:

\begin{enumerate}
    \item \textbf{Trace-to-Solution Mapping}: $\hat{U} \rightarrow U$
    \begin{align}
        u &= B_1 \hat{u} + y_1 \\
        \omega &= C_2 \hat{u} + B_2 \hat{\omega} + y_2 \\
        v &= B_3(\lambda(\bar{\omega})) \hat{v} + y_3(\lambda(\bar{\omega})) \\
        \phi &= B_4 \hat{\phi} + C_4 v + L_4 g_\phi
    \end{align}
    
    \item \textbf{Flux Jump Construction}: $\tilde{J} = D_1 U - D_2 \hat{U}$
    
    \item \textbf{Normal Flux Computation}: 
    \begin{equation}
        j = \hat{B}_4 \hat{U} + \tilde{J}^T Q U
    \end{equation}
    
    \item \textbf{Final Assembly}:
    \begin{align}
        \hat{j} &= B_5 j + B_6 U + B_7 \hat{U} \\
        \hat{J} &= \hat{B}_0 \tilde{J} + \hat{B}_1 U - \hat{B}_2 \hat{U}
    \end{align}
\end{enumerate}

\subsection{Software Architecture}

\subsubsection{Core Components}

\begin{description}
    \item[\texttt{Problem}] Encapsulates PDE specifications, parameters, and boundary conditions
    \item[\texttt{Discretization}] Manages spatial mesh, time stepping, and stabilization parameters
    \item[\texttt{StaticCondensationOOC}] Implements organ-on-chip static condensation following MATLAB reference
    \item[\texttt{GlobalAssembler}] Coordinates multi-domain assembly and Newton iteration
    \item[\texttt{ConstraintManager}] Handles boundary conditions and junction constraints
    \item[\texttt{BulkDataManager}] Manages bulk solution data and forcing term computation
\end{description}

\subsubsection{Advanced Features}

\paragraph{Multi-Domain Visualization}
The \texttt{MultiDomainPlotter} class provides comprehensive visualization capabilities:
\begin{itemize}
    \item Continuous solution plots across domain boundaries
    \item Domain-wise comparison views
    \item Solution evolution tracking
    \item Automatic figure size optimization for screen fitting
    \item Statistical analysis and interface highlighting
\end{itemize}

\paragraph{Constraint Management}
Supports multiple constraint types:
\begin{itemize}
    \item Dirichlet boundary conditions
    \item Neumann flux specifications
    \item Junction continuity constraints
    \item Kirchhoff-Kedem conditions for network flows
\end{itemize}

\subsection{Current Implementation Status}

\subsubsection{Completed Components (\textcolor{green}{\checkmark})}

\begin{itemize}
    \item \textbf{Core Framework}: Full HDG implementation with static condensation
    \item \textbf{Organ-on-Chip Model}: 4-equation system with MATLAB compatibility
    \item \textbf{Time Evolution}: Implicit Euler with Newton solver
    \item \textbf{Multi-Domain Support}: Connected domain handling with junction constraints
    \item \textbf{Visualization System}: Comprehensive plotting with automatic sizing
    \item \textbf{Bulk Data Management}: Efficient data handling and forcing term computation
    \item \textbf{Elementary Matrices}: Pre-computed basis function integrals
    \item \textbf{Constraint System}: Boundary conditions and multi-domain coupling
\end{itemize}

\subsubsection{Active Development Areas}

\paragraph{High Priority Items}
\begin{enumerate}
    \item \textbf{Picard Iteration Methods}: Implementation of fixed-point iteration for arbitrary nonlinearities
    \begin{itemize}
        \item Design framework for general nonlinear problems
        \item Convergence acceleration techniques
        \item Hybrid Picard-Newton strategies
    \end{itemize}
    
    \item \textbf{Adaptive Time Stepping}: Dynamic time step control
    \begin{itemize}
        \item Embedded Runge-Kutta error estimation
        \item Automatic step size adjustment
        \item Stability-based limiters
    \end{itemize}
    
    \item \textbf{Non-Uniform Mesh Support}: Variable element sizing
    \begin{itemize}
        \item Adaptive mesh refinement (AMR)
        \item Mesh grading and clustering
        \item Load balancing for multi-domain systems
    \end{itemize}
\end{enumerate}

\paragraph{Medium Priority Development}
\begin{itemize}
    \item \textbf{Non-Homogeneous Boundary Conditions}: Time-dependent Dirichlet/Neumann data
    \item \textbf{Multi-Domain Testing}: Validation with complex junction networks
    \item \textbf{Performance Optimization}: Profiling and computational efficiency improvements
    \item \textbf{Mass Conservation Tracking}: Global conservation monitoring during time evolution
\end{itemize}

\subsubsection{Testing and Validation}

\paragraph{Verification Status}
\begin{itemize}
    \item \textbf{Single Domain}: Fully validated against analytical solutions
    \item \textbf{Static Condensation}: Verified against MATLAB reference implementation
    \item \textbf{Time Evolution}: Working for linear and mildly nonlinear problems
    \item \textbf{Visualization}: Comprehensive testing with multiple equation systems
    \item \textbf{Elementary Matrices}: Validated through extensive unit testing
\end{itemize}

\paragraph{Known Issues}
\begin{itemize}
    \item Junction conditions in T-junction examples require further investigation
    \item Multi-domain time evolution needs additional validation
    \item Large-scale performance optimization pending
\end{itemize}

\subsection{Technical Specifications}

\subsubsection{Dependencies and Requirements}
\begin{itemize}
    \item \textbf{Python}: 3.8+ with NumPy, SciPy, Matplotlib
    \item \textbf{Numerical Libraries}: Optimized linear algebra (BLAS/LAPACK)
    \item \textbf{Visualization}: Matplotlib with optional animation support
    \item \textbf{Development}: Git version control, pytest testing framework
\end{itemize}

\subsubsection{Performance Characteristics}
\begin{itemize}
    \item \textbf{Problem Size}: Successfully tested up to 4 equations, 40 elements per domain
    \item \textbf{Time Stepping}: Implicit Euler with Newton convergence in 1-5 iterations
    \item \textbf{Memory Usage}: Lean architecture with minimal memory overhead
    \item \textbf{Computational Complexity}: $O(N^3)$ for Newton linear solves, $O(N)$ for assembly
\end{itemize}

\subsection{Future Roadmap}

\subsubsection{Short-term Goals (3-6 months)}
\begin{enumerate}
    \item Complete Picard iteration implementation for general nonlinearities
    \item Implement adaptive time stepping with error control
    \item Add non-uniform mesh support with basic AMR
    \item Validate multi-domain problems with complex junction networks
    \item Performance optimization and large-scale testing
\end{enumerate}

\subsubsection{Medium-term Objectives (6-12 months)}
\begin{enumerate}
    \item Extend to 2D/3D network topologies
    \item Add support for moving boundaries and dynamic meshes
    \item Implement parallel computing capabilities
    \item Develop biological application-specific modules
    \item Create comprehensive documentation and user guides
\end{enumerate}

\subsubsection{Long-term Vision (1-2 years)}
\begin{enumerate}
    \item Integration with experimental data and validation
    \item Machine learning-enhanced parameter estimation
    \item Real-time simulation capabilities for organ-on-chip devices
    \item Commercial deployment for biomedical applications
    \item Open-source community development and contributions
\end{enumerate}

\subsection{Project Impact and Applications}

BioNetFlux represents a significant advancement in computational biology by providing:

\begin{itemize}
    \item \textbf{Unified Framework}: Single platform for diverse biological transport problems
    \item \textbf{High-Order Accuracy}: HDG methods provide superior accuracy compared to traditional FEM
    \item \textbf{Scalability}: Multi-domain architecture supports complex network topologies
    \item \textbf{Flexibility}: Modular design allows easy extension to new biological models
    \item \textbf{Validation}: MATLAB reference compatibility ensures numerical reliability
\end{itemize}

The framework is positioned to become a valuable tool for researchers in computational biology, biomedical engineering, and pharmaceutical development, particularly for organ-on-chip technology and biological network modeling.

% End of project status report
