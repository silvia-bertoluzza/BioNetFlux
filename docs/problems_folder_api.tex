% Problems Folder API Documentation (Accurate Analysis)
% To be included in master LaTeX document
%
% Usage: % Problems Folder API Documentation (Accurate Analysis)
% To be included in master LaTeX document
%
% Usage: % Problems Folder API Documentation (Accurate Analysis)
% To be included in master LaTeX document
%
% Usage: % Problems Folder API Documentation (Accurate Analysis)
% To be included in master LaTeX document
%
% Usage: \input{docs/problems_folder_api}

\section{Problems Folder API Reference}
\label{sec:problems_folder_api}

This section provides detailed documentation for the problem modules in the BioNetFlux problems folder, including their mathematical formulations, parameter specifications, and implementation details based on MATLAB reference implementations.

\subsection{Problems Folder Overview}

The problems folder contains:
\begin{itemize}
    \item \textbf{Organ-on-Chip (OoC) Models}: Multi-equation coupled systems for biological transport
    \item \textbf{MATLAB Reference Implementations}: Original MATLAB problem definitions
    \item \textbf{Python Port Modules}: Corresponding Python implementations using BioNetFlux framework
    \item \textbf{Parameter Configuration}: Physical and numerical parameter specifications
    \item \textbf{Boundary Condition Setup}: Dirichlet, Neumann, and coupling conditions
\end{itemize}

\subsection{Mathematical Model: Organ-on-Chip System}
\label{subsec:ooc_mathematical_model}

The Organ-on-Chip system consists of four coupled partial differential equations describing biological transport phenomena:

\subsubsection{Governing Equations}

\textbf{Cell Density Equation (u):}
\begin{align}
\frac{\partial u}{\partial t} &= \nu \nabla^2 u - \chi \nabla \cdot (u \nabla \phi) + a u + f_u(x,t)
\end{align}

\textbf{Chemical Gradient Equation ($\omega$):}
\begin{align}
\frac{\partial \omega}{\partial t} &= \epsilon \nabla^2 \omega + c \omega + d u + f_\omega(x,t)
\end{align}

\textbf{Velocity Field Equation (v):}
\begin{align}
\frac{\partial v}{\partial t} &= \sigma \nabla^2 v + \lambda(\bar{\omega}) v + f_v(x,t)
\end{align}

\textbf{Potential Field Equation ($\phi$):}
\begin{align}
\frac{\partial \phi}{\partial t} &= \mu \nabla^2 \phi + a \phi + b v + f_\phi(x,t)
\end{align}

where $\bar{\omega} = \frac{1}{|K|} \int_K \omega \, dx$ is the cell-averaged value of $\omega$.

\subsubsection{Parameter Definitions}

\begin{longtable}{|p{2cm}|p{3cm}|p{8cm}|}
\hline
\textbf{Parameter} & \textbf{Physical Meaning} & \textbf{Mathematical Role} \\
\hline
\endhead

$\nu$ & Cell diffusion coefficient & Viscosity/diffusion in u equation \\
\hline

$\mu$ & Potential diffusion & Viscosity/diffusion in $\phi$ equation \\
\hline

$\epsilon$ & Chemical diffusion & Viscosity/diffusion in $\omega$ equation \\
\hline

$\sigma$ & Velocity diffusion & Viscosity/diffusion in v equation \\
\hline

$\chi$ & Chemotactic sensitivity & Coupling strength between u and $\phi$ \\
\hline

$a$ & Growth/decay rate & Reaction term in u and $\phi$ equations \\
\hline

$b$ & Velocity coupling & Coupling strength from v to $\phi$ \\
\hline

$c$ & Chemical reaction rate & Reaction term in $\omega$ equation \\
\hline

$d$ & Cell-chemical coupling & Coupling strength from u to $\omega$ \\
\hline

$\lambda(\cdot)$ & Nonlinear velocity function & Nonlinear reaction in v equation \\
\hline

\end{longtable}

\subsection{MATLAB Reference Problems}
\label{subsec:matlab_reference_problems}

\subsubsection{TestProblem.m}

\paragraph{Problem Configuration}
\begin{lstlisting}[language=Matlab, caption=MATLAB TestProblem.m Parameters]
% Physical parameters (viscosity)
nu = 1.0;
mu = 2.0;
epsilon = 1.0;
sigma = 1.0;

% Reaction parameters
a = 0.0;
c = 0.0;

% Domain
A = 0;      % Domain start
L = 1.0;    % Domain length

% Coupling parameters
b = 1.0;
d = 1.0;
chi = 1.0;
lambda = @(x) constant_function(x);  % Constant nonlinearity
\end{lstlisting}

\paragraph{Initial Conditions}
\begin{lstlisting}[language=Matlab, caption=TestProblem.m Initial Conditions]
% Initial data functions u0{i}(x,t)
problem.u0{1} = @(x,t) sin(2*pi*x);  % Non-trivial for cell density
problem.u0{2} = @(x,t) zeros(size(x)); % Zero chemical
problem.u0{3} = @(x,t) zeros(size(x)); % Zero velocity  
problem.u0{4} = @(x,t) zeros(size(x)); % Zero potential

% Gradient initial condition
problem.phix0 = @(x,t) zeros(size(x));
\end{lstlisting}

\paragraph{Source Terms}
\begin{lstlisting}[language=Matlab, caption=TestProblem.m Source Terms]
% All source terms are zero
force{1} = @(x,t) zeros(size(x));  % f_u = 0
force{2} = @(x,t) zeros(size(x));  % f_omega = 0
force{3} = @(x,t) zeros(size(x));  % f_v = 0
force{4} = @(x,t) zeros(size(x));  % f_phi = 0
\end{lstlisting}

\paragraph{Boundary Conditions}
\begin{lstlisting}[language=Matlab, caption=TestProblem.m Boundary Conditions]
% Neumann flux conditions at both boundaries
% fluxu0{i}(t) = flux at left boundary (x = A)
% fluxu1{i}(t) = flux at right boundary (x = A + L)

problem.fluxu0{1} = @(t) 0.;  % Zero flux for u at x=0
problem.fluxu1{1} = @(t) 0.;  % Zero flux for u at x=1
problem.fluxu0{2} = @(t) 0.;  % Zero flux for omega
problem.fluxu1{2} = @(t) 0.;
problem.fluxu0{3} = @(t) 0.;  % Zero flux for v
problem.fluxu1{3} = @(t) 0.;
problem.fluxu0{4} = @(t) 0.;  % Zero flux for phi
problem.fluxu1{4} = @(t) 0.;

problem.NeumannData = zeros(8,1);  % 2 boundaries × 4 equations
\end{lstlisting}

\subsubsection{EmptyProblem.m}

\paragraph{Differences from TestProblem.m}
\begin{lstlisting}[language=Matlab, caption=EmptyProblem.m Differences]
% Same parameter configuration as TestProblem.m
% Key difference: All initial conditions are zero
problem.u0{1} = @(x,t) zeros(size(x));  % Zero initial cell density
problem.u0{2} = @(x,t) zeros(size(x));
problem.u0{3} = @(x,t) zeros(size(x));
problem.u0{4} = @(x,t) zeros(size(x));

% Updated source terms (if any modifications needed)
force_u = @(x,t) zeros(size(x));
force_phi = @(x,t) zeros(size(x));
force_v = @(x,t) zeros(size(x));
force_omega = @(x,t) zeros(size(x));
\end{lstlisting}

\textbf{Purpose:} EmptyProblem.m serves as a baseline/template with zero initial conditions for testing the numerical framework without initial transients.

\subsection{Python Problem Implementation}
\label{subsec:python_problem_implementation}

\subsubsection{ooc\_test\_problem.py}

\paragraph{Module Structure}
\begin{lstlisting}[language=Python, caption=Python OoC Problem Structure]
# Core imports
from ..core.problem import Problem
from ..core.discretization import Discretization, GlobalDiscretization
from ..core.constraints import ConstraintManager

def create_global_framework():
    """
    OrganOnChip test problem - Python port from MATLAB TestProblem.m
    """
    # Returns: problems, global_discretization, constraint_manager, problem_name
\end{lstlisting}

\paragraph{Parameter Configuration}
\begin{lstlisting}[language=Python, caption=Python Parameter Setup]
# Physical parameters from MATLAB (viscosity)
nu = 1.0
mu = 1.0      # Note: Different from MATLAB (mu = 2.0)
epsilon = 1.0
sigma = 1.0

# Reaction parameters
a = 1.0       # Note: Different from MATLAB (a = 0.0)
c = 1.0       # Note: Different from MATLAB (c = 0.0)

# Coupling parameters
b = 1.0
d = 1.0
chi = 1.0

# Nonlinear lambda function
lambda_function = lambda x: 1.0/(1.0 + x**2)
dlambda_function = lambda x: -2.0*x/(1.0 + x**2)**2

# Parameter vector [nu, mu, epsilon, sigma, a, b, c, d, chi]
parameters = np.array([nu, mu, epsilon, sigma, a, b, c, d, chi])
\end{lstlisting}

\paragraph{Domain and Discretization}
\begin{lstlisting}[language=Python, caption=Python Domain Setup]
# Domain definition
domain_start = 1.0   # Note: Different from MATLAB (A = 0)
domain_length = 1.0  # Same as MATLAB (L = 1.0)

# Mesh parameters
n_elements = 20      # Spatial discretization
neq = 4             # Four equations

# Time parameters
T = 1.0             # Final time
dt = 0.1            # Time step
\end{lstlisting}

\paragraph{Initial Conditions}
\begin{lstlisting}[language=Python, caption=Python Initial Conditions]
# Initial conditions - more complex than MATLAB version
def initial_u(s, t=0.0):
    s = np.asarray(s)
    return 0.0 * s                    # Zero (like EmptyProblem.m)

def initial_omega(s, t=0.0):
    s = np.asarray(s)
    return np.sin(2 * np.pi * s + np.pi * t)  # Time-dependent

def initial_v(s, t=0.0):
    s = np.asarray(s)
    return t * s                      # Time-dependent

def initial_phi(s, t=0.0):
    return s ** 2                     # Quadratic profile
\end{lstlisting}

\paragraph{Source Terms}
\begin{lstlisting}[language=Python, caption=Python Source Terms]
# Non-zero source terms (unlike MATLAB versions)
def force_u(s, t):
    s = np.asarray(s)
    return np.zeros_like(s)           # Still zero for u

def force_omega(s, t):
    s = np.asarray(s)
    x = 2 * np.pi * s + np.pi * t
    return np.sin(x) + 4 * np.pi**2 * np.sin(x) + np.pi * np.cos(x)

def force_v(s, t):
    omega_val = initial_omega(s, t)
    lambda_val = lambda_function(omega_val)
    s = np.asarray(s)
    return s + lambda_val * t * s

def force_phi(s, t):
    s = np.asarray(s)
    return - mu * 2.0 * np.ones_like(s) + a * s**2 - b * t * s
\end{lstlisting}

\paragraph{Boundary Conditions}
\begin{lstlisting}[language=Python, caption=Python Boundary Conditions]
# Analytical flux functions for Neumann conditions
flux_u = lambda s, t: 0.0
flux_omega = lambda s, t: 2 * np.pi * np.cos(2 * np.pi * s + np.pi * t) 
flux_v = lambda s, t: t
flux_phi = lambda s, t: 2 * s

# Setup Neumann boundary conditions for all equations
constraint_manager.add_neumann(0, 0, domain_start, lambda t: -flux_u(domain_start, t))
constraint_manager.add_neumann(0, 0, domain_end, lambda t: flux_u(domain_end, t))

constraint_manager.add_neumann(1, 0, domain_start, lambda t: -flux_omega(domain_start, t))
constraint_manager.add_neumann(1, 0, domain_end, lambda t: flux_omega(domain_end, t))

constraint_manager.add_neumann(2, 0, domain_start, lambda t: -flux_v(domain_start, t))
constraint_manager.add_neumann(2, 0, domain_end, lambda t: flux_v(domain_end, t))

constraint_manager.add_neumann(3, 0, domain_start, lambda t: -flux_phi(domain_start, t))
constraint_manager.add_neumann(3, 0, domain_end, lambda t: flux_phi(domain_end, t))
\end{lstlisting}

\paragraph{Framework Integration}
\begin{lstlisting}[language=Python, caption=BioNetFlux Framework Integration]
# Create Problem instance
problem = Problem(
    neq=neq,
    domain_start=domain_start,
    domain_length=domain_length,
    parameters=parameters,
    problem_type="organ_on_chip",
    name="ooc_test"
)

# Set lambda functions using flexible method
problem.set_function('lambda_function', lambda_function)  
problem.set_function('dlambda_function', dlambda_function)

# Set source terms for all 4 equations
problem.set_force(0, lambda s, t: force_u(s, t))
problem.set_force(1, lambda s, t: force_omega(s, t))
problem.set_force(2, lambda s, t: force_v(s, t))
problem.set_force(3, lambda s, t: force_phi(s, t))

# Set initial conditions
problem.set_initial_condition(0, initial_u)
problem.set_initial_condition(1, initial_omega)
problem.set_initial_condition(2, initial_v) 
problem.set_initial_condition(3, initial_phi)

# Create discretization with stabilization parameters
discretization = Discretization(
    n_elements=n_elements,
    domain_start=domain_start,
    domain_length=domain_length,
    stab_constant=1.0
)

# Set stabilization parameters for all 4 equations
tau_u = 1.0 / discretization.element_length
tau_omega = 1.0
tau_v = 1.0  
tau_phi = 1.0
discretization.set_tau([tau_u, tau_omega, tau_v, tau_phi])
\end{lstlisting}

\subsection{Comparison: MATLAB vs Python Implementations}
\label{subsec:matlab_vs_python_comparison}

\subsubsection{Parameter Differences}

\begin{longtable}{|p{3cm}|p{2cm}|p{2cm}|p{6cm}|}
\hline
\textbf{Parameter} & \textbf{MATLAB} & \textbf{Python} & \textbf{Impact} \\
\hline
\endhead

$\mu$ & 2.0 & 1.0 & Reduced potential diffusion in Python \\
\hline

$a$ & 0.0 & 1.0 & Added growth term in Python u and $\phi$ equations \\
\hline

$c$ & 0.0 & 1.0 & Added reaction term in Python $\omega$ equation \\
\hline

Domain start & 0.0 & 1.0 & Shifted domain in Python implementation \\
\hline

$\lambda(x)$ & constant & $\frac{1}{1+x^2}$ & Nonlinear saturation in Python \\
\hline

\end{longtable}

\subsubsection{Initial Condition Differences}

\begin{longtable}{|p{2cm}|p{4cm}|p{4cm}|p{3cm}|}
\hline
\textbf{Variable} & \textbf{MATLAB TestProblem.m} & \textbf{Python ooc\_test\_problem.py} & \textbf{Complexity} \\
\hline
\endhead

$u$ & $\sin(2\pi x)$ & $0$ & MATLAB: Non-trivial \\
\hline

$\omega$ & $0$ & $\sin(2\pi x + \pi t)$ & Python: Time-dependent \\
\hline

$v$ & $0$ & $t \cdot x$ & Python: Space-time coupling \\
\hline

$\phi$ & $0$ & $x^2$ & Python: Quadratic profile \\
\hline

\end{longtable}

\subsubsection{Source Term Complexity}

\begin{longtable}{|p{2cm}|p{5.5cm}|p{5.5cm}|}
\hline
\textbf{Equation} & \textbf{MATLAB} & \textbf{Python} \\
\hline
\endhead

$f_u$ & $0$ & $0$ \\
\hline

$f_\omega$ & $0$ & $\sin(x) + 4\pi^2\sin(x) + \pi\cos(x)$ \\
\hline

$f_v$ & $0$ & $x + \lambda(\omega(x,t)) \cdot t \cdot x$ \\
\hline

$f_\phi$ & $0$ & $-\mu \cdot 2 + a x^2 - b t x$ \\
\hline

\end{longtable}

\subsection{Static Condensation Implementation}
\label{subsec:static_condensation_implementation}

\subsubsection{StaticC.m Overview}

The MATLAB static condensation implementation performs HDG local assembly:

\paragraph{Input/Output Structure}
\begin{lstlisting}[language=Matlab, caption=StaticC.m Interface]
function [U,hJ,dhJ] = StaticC(hU,rhs,problem,discretization,scMatrices)
% Inputs:
%   hU - Local trace values [hat_u; hat_omega; hat_v; hat_phi] (8×1)
%   rhs - Right-hand side [g_u; g_omega; g_v; g_phi] (8×1)
%   problem - Problem parameters structure
%   discretization - Time and stabilization parameters
%   scMatrices - Pre-computed static condensation matrices
%
% Outputs:
%   U - Local bulk solution [u; omega; v; phi] (8×1)
%   hJ - Local flux jump [hat_j; hat_psi] (8×1)
%   dhJ - Jacobian of flux jump (8×8)
\end{lstlisting}

\paragraph{Static Condensation Steps}
\begin{lstlisting}[language=Matlab, caption=StaticC.m Algorithm Steps]
% Step 1: Solve for u
% (M + dt*tau*M_partial) u = g_u + dt*tau*G_partial*hat_u
u{1} = scMatrices.B1 * hu{1} + scMatrices.L1 * g{1};

% Step 2: Solve for omega (coupled with u)
% Include coupling term dt*d*M*u
y{2} = scMatrices.L2*(g{2} + dt * d * scMatrices.M * y{1});
u{2} = scMatrices.C2 * hu{1} + scMatrices.B2 * hu{2} + y{2};

% Step 3: Compute nonlinear lambda(bar_omega)
bar_omega = scMatrices.Av * u{2};      % Cell average
bl = problem.lambda(bar_omega);        % lambda(bar_omega)
dbl = problem.dlambda(bar_omega);      % lambda'(bar_omega)

% Step 4: Solve for v (nonlinear)
% (A_3 + lambda(bar_omega)*S_3) v = H_3*hat_v + g_v
L3 = (scMatrices.A3 + bl * scMatrices.S3)\eye(2);
u{3} = L3 * scMatrices.H3 * hu{3} + L3 * g{3};

% Step 5: Solve for phi (coupled with v)
u{4} = scMatrices.B4 * hu{4} + scMatrices.C4 * u{3} + scMatrices.L4 * g{4};
\end{lstlisting}

\paragraph{Jacobian Computation}
\begin{lstlisting}[language=Matlab, caption=StaticC.m Jacobian Assembly]
% Build Jacobian dU/d(hat_U) through chain rule
JAC = zeros(8);

% Contribution from each step
JAC = JAC + R{1}' * scMatrices.B1 * R{1};                    % Step 1
JAC = JAC + R{2}' * (scMatrices.C2*R{1} + scMatrices.B2*R{2}); % Step 2

% Nonlinear contribution from lambda'(bar_omega)
J1 = dbl * L3 * scMatrices.S3 * (scMatrices.H3 * hu{3} + g{3}) * scMatrices.Av;
JAC = JAC + R{3}' * (J0 * R{3} + J1 * R{3} * JAC);          % Step 3

JAC = JAC + R{4}' * (scMatrices.B4 * R{4} + scMatrices.C4 * R{3} * JAC); % Step 4
\end{lstlisting}

\subsection{Problem Module Usage Patterns}
\label{subsec:problem_usage_patterns}

\subsubsection{Integration with BioNetFlux Framework}

\begin{lstlisting}[language=Python, caption=Problem Module Usage Example]
from setup_solver import quick_setup

# Load OrganOnChip problem
setup = quick_setup("ooc1d.problems.ooc_test_problem")

# Access problem configuration
info = setup.get_problem_info()
print(f"Problem: {info['problem_name']}")
print(f"Equations: {info['domains'][0]['n_equations']}")
print(f"Elements: {info['domains'][0]['n_elements']}")

# Create initial conditions
trace_solutions, multipliers = setup.create_initial_conditions()

# Time evolution simulation
dt = setup.global_discretization.dt
T = setup.global_discretization.T
n_steps = int(T / dt)

for step in range(n_steps):
    current_time = step * dt
    
    # Newton iteration for current time step
    # ... (using GlobalAssembler and BulkDataManager)
    
    print(f"Time step {step}: t = {current_time:.3f}")
\end{lstlisting}

\subsubsection{Parameter Sensitivity Analysis}

\begin{lstlisting}[language=Python, caption=Parameter Sensitivity Example]
# Modify problem parameters for sensitivity analysis
def create_parameter_sweep():
    """Create multiple problem instances with varying parameters."""
    
    base_setup = quick_setup("ooc1d.problems.ooc_test_problem")
    base_problem = base_setup.problems[0]
    
    # Parameter variations
    chi_values = [0.1, 0.5, 1.0, 2.0, 5.0]  # Chemotactic sensitivity
    
    results = []
    for chi in chi_values:
        # Modify parameters
        new_params = base_problem.parameters.copy()
        new_params[8] = chi  # chi is parameter index 8
        
        # Create new problem with modified parameters
        modified_problem = Problem(
            neq=base_problem.neq,
            domain_start=base_problem.domain_start,
            domain_length=base_problem.domain_length,
            parameters=new_params,
            problem_type=base_problem.type,
            name=f"ooc_chi_{chi}"
        )
        
        # Copy functions and conditions from base problem
        for i in range(base_problem.neq):
            if base_problem.initial_conditions[i] is not None:
                modified_problem.set_initial_condition(i, base_problem.initial_conditions[i])
            if base_problem.force_functions[i] is not None:
                modified_problem.set_force(i, base_problem.force_functions[i])
        
        results.append((chi, modified_problem))
    
    return results

# Usage
parameter_sweep = create_parameter_sweep()
for chi_val, problem in parameter_sweep:
    print(f"Chi = {chi_val}: Problem configured for sensitivity analysis")
\end{lstlisting}

\subsection{Expected Problem Extensions}
\label{subsec:expected_problem_extensions}

\subsubsection{Multi-Domain Problems}

\begin{lstlisting}[language=Python, caption=Expected Multi-Domain Problem Structure]
def create_multi_domain_ooc():
    """
    Expected structure for multi-domain OrganOnChip problems.
    """
    
    # Multiple domains with different parameter sets
    domains = [
        {"start": 0.0, "length": 1.0, "chi": 1.0, "name": "vessel"},
        {"start": 1.0, "length": 0.5, "chi": 2.0, "name": "tissue"},
        {"start": 1.5, "length": 1.0, "chi": 0.5, "name": "organ"}
    ]
    
    problems = []
    discretizations = []
    
    for i, domain_config in enumerate(domains):
        # Create domain-specific problem
        problem = create_domain_problem(domain_config, domain_idx=i)
        problems.append(problem)
        
        # Create domain-specific discretization
        disc = Discretization(
            n_elements=20,
            domain_start=domain_config["start"],
            domain_length=domain_config["length"]
        )
        discretizations.append(disc)
    
    # Setup junction constraints
    constraint_manager = ConstraintManager()
    
    # Continuity at domain interfaces
    constraint_manager.add_trace_continuity(0, 0, 1, -1, 1.0)  # u continuity
    constraint_manager.add_trace_continuity(1, 0, 2, -1, 1.5)  # omega continuity
    
    # Flux continuity with permeability
    constraint_manager.add_kedem_katchalsky(2, 0, 1, -1, 1.0, P=0.1)  # v coupling
    constraint_manager.add_kedem_katchalsky(3, 0, 2, -1, 1.5, P=0.2)  # phi coupling
    
    return problems, GlobalDiscretization(discretizations), constraint_manager, "Multi-Domain OoC"
\end{lstlisting}

\subsubsection{Alternative Nonlinear Functions}

\begin{lstlisting}[language=Python, caption=Expected Alternative Lambda Functions]
# Saturation kinetics (Michaelis-Menten)
def lambda_michaelis_menten(omega, K=1.0, V_max=2.0):
    return V_max * omega / (K + omega)

def dlambda_michaelis_menten(omega, K=1.0, V_max=2.0):
    return V_max * K / (K + omega)**2

# Hill function (cooperative binding)
def lambda_hill(omega, K=1.0, n=2.0):
    omega_n = omega**n
    return omega_n / (K**n + omega_n)

def dlambda_hill(omega, K=1.0, n=2.0):
    omega_n = omega**n
    K_n = K**n
    return n * omega**(n-1) * K_n / (K_n + omega_n)**2

# Exponential activation
def lambda_exponential(omega, alpha=1.0, beta=2.0):
    return alpha * (1.0 - np.exp(-beta * omega))

def dlambda_exponential(omega, alpha=1.0, beta=2.0):
    return alpha * beta * np.exp(-beta * omega)
\end{lstlisting}

\subsection{Problem Module Summary}
\label{subsec:problem_module_summary}

\begin{longtable}{|p{4cm}|p{3cm}|p{6cm}|}
\hline
\textbf{Module/File} & \textbf{Type} & \textbf{Key Features} \\
\hline
\endhead

\texttt{TestProblem.m} & MATLAB Reference & Non-trivial u initial condition, zero sources \\
\hline

\texttt{EmptyProblem.m} & MATLAB Reference & All zero initial conditions, template structure \\
\hline

\texttt{StaticC.m} & MATLAB Algorithm & HDG static condensation with nonlinear lambda \\
\hline

\texttt{scBlocks.m} & MATLAB Documentation & Matrix construction theory and formulas \\
\hline

\texttt{ooc\_test\_problem.py} & Python Implementation & Rich test case with analytical solutions \\
\hline

\end{longtable}

\subsection{Key Design Principles}

\begin{itemize}
    \item \textbf{MATLAB Compatibility}: Python problems maintain mathematical equivalence with MATLAB references
    \item \textbf{Framework Integration}: Seamless integration with BioNetFlux Problem, Discretization, and ConstraintManager classes
    \item \textbf{Flexibility}: Support for arbitrary nonlinear functions, time-dependent data, and multi-domain configurations
    \item \textbf{Validation Ready}: Rich initial conditions and source terms enable analytical solution comparison
    \item \textbf{Biological Relevance}: Parameters and equations model realistic biological transport phenomena
    \item \textbf{Extensibility}: Clear patterns for creating new problem instances and parameter variations
\end{itemize}

This documentation provides comprehensive coverage of the problems folder contents, enabling users to understand, modify, and extend the OrganOnChip model implementations within the BioNetFlux framework.

% End of problems folder API documentation


\section{Problems Folder API Reference}
\label{sec:problems_folder_api}

This section provides detailed documentation for the problem modules in the BioNetFlux problems folder, including their mathematical formulations, parameter specifications, and implementation details based on MATLAB reference implementations.

\subsection{Problems Folder Overview}

The problems folder contains:
\begin{itemize}
    \item \textbf{Organ-on-Chip (OoC) Models}: Multi-equation coupled systems for biological transport
    \item \textbf{MATLAB Reference Implementations}: Original MATLAB problem definitions
    \item \textbf{Python Port Modules}: Corresponding Python implementations using BioNetFlux framework
    \item \textbf{Parameter Configuration}: Physical and numerical parameter specifications
    \item \textbf{Boundary Condition Setup}: Dirichlet, Neumann, and coupling conditions
\end{itemize}

\subsection{Mathematical Model: Organ-on-Chip System}
\label{subsec:ooc_mathematical_model}

The Organ-on-Chip system consists of four coupled partial differential equations describing biological transport phenomena:

\subsubsection{Governing Equations}

\textbf{Cell Density Equation (u):}
\begin{align}
\frac{\partial u}{\partial t} &= \nu \nabla^2 u - \chi \nabla \cdot (u \nabla \phi) + a u + f_u(x,t)
\end{align}

\textbf{Chemical Gradient Equation ($\omega$):}
\begin{align}
\frac{\partial \omega}{\partial t} &= \epsilon \nabla^2 \omega + c \omega + d u + f_\omega(x,t)
\end{align}

\textbf{Velocity Field Equation (v):}
\begin{align}
\frac{\partial v}{\partial t} &= \sigma \nabla^2 v + \lambda(\bar{\omega}) v + f_v(x,t)
\end{align}

\textbf{Potential Field Equation ($\phi$):}
\begin{align}
\frac{\partial \phi}{\partial t} &= \mu \nabla^2 \phi + a \phi + b v + f_\phi(x,t)
\end{align}

where $\bar{\omega} = \frac{1}{|K|} \int_K \omega \, dx$ is the cell-averaged value of $\omega$.

\subsubsection{Parameter Definitions}

\begin{longtable}{|p{2cm}|p{3cm}|p{8cm}|}
\hline
\textbf{Parameter} & \textbf{Physical Meaning} & \textbf{Mathematical Role} \\
\hline
\endhead

$\nu$ & Cell diffusion coefficient & Viscosity/diffusion in u equation \\
\hline

$\mu$ & Potential diffusion & Viscosity/diffusion in $\phi$ equation \\
\hline

$\epsilon$ & Chemical diffusion & Viscosity/diffusion in $\omega$ equation \\
\hline

$\sigma$ & Velocity diffusion & Viscosity/diffusion in v equation \\
\hline

$\chi$ & Chemotactic sensitivity & Coupling strength between u and $\phi$ \\
\hline

$a$ & Growth/decay rate & Reaction term in u and $\phi$ equations \\
\hline

$b$ & Velocity coupling & Coupling strength from v to $\phi$ \\
\hline

$c$ & Chemical reaction rate & Reaction term in $\omega$ equation \\
\hline

$d$ & Cell-chemical coupling & Coupling strength from u to $\omega$ \\
\hline

$\lambda(\cdot)$ & Nonlinear velocity function & Nonlinear reaction in v equation \\
\hline

\end{longtable}

\subsection{MATLAB Reference Problems}
\label{subsec:matlab_reference_problems}

\subsubsection{TestProblem.m}

\paragraph{Problem Configuration}
\begin{lstlisting}[language=Matlab, caption=MATLAB TestProblem.m Parameters]
% Physical parameters (viscosity)
nu = 1.0;
mu = 2.0;
epsilon = 1.0;
sigma = 1.0;

% Reaction parameters
a = 0.0;
c = 0.0;

% Domain
A = 0;      % Domain start
L = 1.0;    % Domain length

% Coupling parameters
b = 1.0;
d = 1.0;
chi = 1.0;
lambda = @(x) constant_function(x);  % Constant nonlinearity
\end{lstlisting}

\paragraph{Initial Conditions}
\begin{lstlisting}[language=Matlab, caption=TestProblem.m Initial Conditions]
% Initial data functions u0{i}(x,t)
problem.u0{1} = @(x,t) sin(2*pi*x);  % Non-trivial for cell density
problem.u0{2} = @(x,t) zeros(size(x)); % Zero chemical
problem.u0{3} = @(x,t) zeros(size(x)); % Zero velocity  
problem.u0{4} = @(x,t) zeros(size(x)); % Zero potential

% Gradient initial condition
problem.phix0 = @(x,t) zeros(size(x));
\end{lstlisting}

\paragraph{Source Terms}
\begin{lstlisting}[language=Matlab, caption=TestProblem.m Source Terms]
% All source terms are zero
force{1} = @(x,t) zeros(size(x));  % f_u = 0
force{2} = @(x,t) zeros(size(x));  % f_omega = 0
force{3} = @(x,t) zeros(size(x));  % f_v = 0
force{4} = @(x,t) zeros(size(x));  % f_phi = 0
\end{lstlisting}

\paragraph{Boundary Conditions}
\begin{lstlisting}[language=Matlab, caption=TestProblem.m Boundary Conditions]
% Neumann flux conditions at both boundaries
% fluxu0{i}(t) = flux at left boundary (x = A)
% fluxu1{i}(t) = flux at right boundary (x = A + L)

problem.fluxu0{1} = @(t) 0.;  % Zero flux for u at x=0
problem.fluxu1{1} = @(t) 0.;  % Zero flux for u at x=1
problem.fluxu0{2} = @(t) 0.;  % Zero flux for omega
problem.fluxu1{2} = @(t) 0.;
problem.fluxu0{3} = @(t) 0.;  % Zero flux for v
problem.fluxu1{3} = @(t) 0.;
problem.fluxu0{4} = @(t) 0.;  % Zero flux for phi
problem.fluxu1{4} = @(t) 0.;

problem.NeumannData = zeros(8,1);  % 2 boundaries × 4 equations
\end{lstlisting}

\subsubsection{EmptyProblem.m}

\paragraph{Differences from TestProblem.m}
\begin{lstlisting}[language=Matlab, caption=EmptyProblem.m Differences]
% Same parameter configuration as TestProblem.m
% Key difference: All initial conditions are zero
problem.u0{1} = @(x,t) zeros(size(x));  % Zero initial cell density
problem.u0{2} = @(x,t) zeros(size(x));
problem.u0{3} = @(x,t) zeros(size(x));
problem.u0{4} = @(x,t) zeros(size(x));

% Updated source terms (if any modifications needed)
force_u = @(x,t) zeros(size(x));
force_phi = @(x,t) zeros(size(x));
force_v = @(x,t) zeros(size(x));
force_omega = @(x,t) zeros(size(x));
\end{lstlisting}

\textbf{Purpose:} EmptyProblem.m serves as a baseline/template with zero initial conditions for testing the numerical framework without initial transients.

\subsection{Python Problem Implementation}
\label{subsec:python_problem_implementation}

\subsubsection{ooc\_test\_problem.py}

\paragraph{Module Structure}
\begin{lstlisting}[language=Python, caption=Python OoC Problem Structure]
# Core imports
from ..core.problem import Problem
from ..core.discretization import Discretization, GlobalDiscretization
from ..core.constraints import ConstraintManager

def create_global_framework():
    """
    OrganOnChip test problem - Python port from MATLAB TestProblem.m
    """
    # Returns: problems, global_discretization, constraint_manager, problem_name
\end{lstlisting}

\paragraph{Parameter Configuration}
\begin{lstlisting}[language=Python, caption=Python Parameter Setup]
# Physical parameters from MATLAB (viscosity)
nu = 1.0
mu = 1.0      # Note: Different from MATLAB (mu = 2.0)
epsilon = 1.0
sigma = 1.0

# Reaction parameters
a = 1.0       # Note: Different from MATLAB (a = 0.0)
c = 1.0       # Note: Different from MATLAB (c = 0.0)

# Coupling parameters
b = 1.0
d = 1.0
chi = 1.0

# Nonlinear lambda function
lambda_function = lambda x: 1.0/(1.0 + x**2)
dlambda_function = lambda x: -2.0*x/(1.0 + x**2)**2

# Parameter vector [nu, mu, epsilon, sigma, a, b, c, d, chi]
parameters = np.array([nu, mu, epsilon, sigma, a, b, c, d, chi])
\end{lstlisting}

\paragraph{Domain and Discretization}
\begin{lstlisting}[language=Python, caption=Python Domain Setup]
# Domain definition
domain_start = 1.0   # Note: Different from MATLAB (A = 0)
domain_length = 1.0  # Same as MATLAB (L = 1.0)

# Mesh parameters
n_elements = 20      # Spatial discretization
neq = 4             # Four equations

# Time parameters
T = 1.0             # Final time
dt = 0.1            # Time step
\end{lstlisting}

\paragraph{Initial Conditions}
\begin{lstlisting}[language=Python, caption=Python Initial Conditions]
# Initial conditions - more complex than MATLAB version
def initial_u(s, t=0.0):
    s = np.asarray(s)
    return 0.0 * s                    # Zero (like EmptyProblem.m)

def initial_omega(s, t=0.0):
    s = np.asarray(s)
    return np.sin(2 * np.pi * s + np.pi * t)  # Time-dependent

def initial_v(s, t=0.0):
    s = np.asarray(s)
    return t * s                      # Time-dependent

def initial_phi(s, t=0.0):
    return s ** 2                     # Quadratic profile
\end{lstlisting}

\paragraph{Source Terms}
\begin{lstlisting}[language=Python, caption=Python Source Terms]
# Non-zero source terms (unlike MATLAB versions)
def force_u(s, t):
    s = np.asarray(s)
    return np.zeros_like(s)           # Still zero for u

def force_omega(s, t):
    s = np.asarray(s)
    x = 2 * np.pi * s + np.pi * t
    return np.sin(x) + 4 * np.pi**2 * np.sin(x) + np.pi * np.cos(x)

def force_v(s, t):
    omega_val = initial_omega(s, t)
    lambda_val = lambda_function(omega_val)
    s = np.asarray(s)
    return s + lambda_val * t * s

def force_phi(s, t):
    s = np.asarray(s)
    return - mu * 2.0 * np.ones_like(s) + a * s**2 - b * t * s
\end{lstlisting}

\paragraph{Boundary Conditions}
\begin{lstlisting}[language=Python, caption=Python Boundary Conditions]
# Analytical flux functions for Neumann conditions
flux_u = lambda s, t: 0.0
flux_omega = lambda s, t: 2 * np.pi * np.cos(2 * np.pi * s + np.pi * t) 
flux_v = lambda s, t: t
flux_phi = lambda s, t: 2 * s

# Setup Neumann boundary conditions for all equations
constraint_manager.add_neumann(0, 0, domain_start, lambda t: -flux_u(domain_start, t))
constraint_manager.add_neumann(0, 0, domain_end, lambda t: flux_u(domain_end, t))

constraint_manager.add_neumann(1, 0, domain_start, lambda t: -flux_omega(domain_start, t))
constraint_manager.add_neumann(1, 0, domain_end, lambda t: flux_omega(domain_end, t))

constraint_manager.add_neumann(2, 0, domain_start, lambda t: -flux_v(domain_start, t))
constraint_manager.add_neumann(2, 0, domain_end, lambda t: flux_v(domain_end, t))

constraint_manager.add_neumann(3, 0, domain_start, lambda t: -flux_phi(domain_start, t))
constraint_manager.add_neumann(3, 0, domain_end, lambda t: flux_phi(domain_end, t))
\end{lstlisting}

\paragraph{Framework Integration}
\begin{lstlisting}[language=Python, caption=BioNetFlux Framework Integration]
# Create Problem instance
problem = Problem(
    neq=neq,
    domain_start=domain_start,
    domain_length=domain_length,
    parameters=parameters,
    problem_type="organ_on_chip",
    name="ooc_test"
)

# Set lambda functions using flexible method
problem.set_function('lambda_function', lambda_function)  
problem.set_function('dlambda_function', dlambda_function)

# Set source terms for all 4 equations
problem.set_force(0, lambda s, t: force_u(s, t))
problem.set_force(1, lambda s, t: force_omega(s, t))
problem.set_force(2, lambda s, t: force_v(s, t))
problem.set_force(3, lambda s, t: force_phi(s, t))

# Set initial conditions
problem.set_initial_condition(0, initial_u)
problem.set_initial_condition(1, initial_omega)
problem.set_initial_condition(2, initial_v) 
problem.set_initial_condition(3, initial_phi)

# Create discretization with stabilization parameters
discretization = Discretization(
    n_elements=n_elements,
    domain_start=domain_start,
    domain_length=domain_length,
    stab_constant=1.0
)

# Set stabilization parameters for all 4 equations
tau_u = 1.0 / discretization.element_length
tau_omega = 1.0
tau_v = 1.0  
tau_phi = 1.0
discretization.set_tau([tau_u, tau_omega, tau_v, tau_phi])
\end{lstlisting}

\subsection{Comparison: MATLAB vs Python Implementations}
\label{subsec:matlab_vs_python_comparison}

\subsubsection{Parameter Differences}

\begin{longtable}{|p{3cm}|p{2cm}|p{2cm}|p{6cm}|}
\hline
\textbf{Parameter} & \textbf{MATLAB} & \textbf{Python} & \textbf{Impact} \\
\hline
\endhead

$\mu$ & 2.0 & 1.0 & Reduced potential diffusion in Python \\
\hline

$a$ & 0.0 & 1.0 & Added growth term in Python u and $\phi$ equations \\
\hline

$c$ & 0.0 & 1.0 & Added reaction term in Python $\omega$ equation \\
\hline

Domain start & 0.0 & 1.0 & Shifted domain in Python implementation \\
\hline

$\lambda(x)$ & constant & $\frac{1}{1+x^2}$ & Nonlinear saturation in Python \\
\hline

\end{longtable}

\subsubsection{Initial Condition Differences}

\begin{longtable}{|p{2cm}|p{4cm}|p{4cm}|p{3cm}|}
\hline
\textbf{Variable} & \textbf{MATLAB TestProblem.m} & \textbf{Python ooc\_test\_problem.py} & \textbf{Complexity} \\
\hline
\endhead

$u$ & $\sin(2\pi x)$ & $0$ & MATLAB: Non-trivial \\
\hline

$\omega$ & $0$ & $\sin(2\pi x + \pi t)$ & Python: Time-dependent \\
\hline

$v$ & $0$ & $t \cdot x$ & Python: Space-time coupling \\
\hline

$\phi$ & $0$ & $x^2$ & Python: Quadratic profile \\
\hline

\end{longtable}

\subsubsection{Source Term Complexity}

\begin{longtable}{|p{2cm}|p{5.5cm}|p{5.5cm}|}
\hline
\textbf{Equation} & \textbf{MATLAB} & \textbf{Python} \\
\hline
\endhead

$f_u$ & $0$ & $0$ \\
\hline

$f_\omega$ & $0$ & $\sin(x) + 4\pi^2\sin(x) + \pi\cos(x)$ \\
\hline

$f_v$ & $0$ & $x + \lambda(\omega(x,t)) \cdot t \cdot x$ \\
\hline

$f_\phi$ & $0$ & $-\mu \cdot 2 + a x^2 - b t x$ \\
\hline

\end{longtable}

\subsection{Static Condensation Implementation}
\label{subsec:static_condensation_implementation}

\subsubsection{StaticC.m Overview}

The MATLAB static condensation implementation performs HDG local assembly:

\paragraph{Input/Output Structure}
\begin{lstlisting}[language=Matlab, caption=StaticC.m Interface]
function [U,hJ,dhJ] = StaticC(hU,rhs,problem,discretization,scMatrices)
% Inputs:
%   hU - Local trace values [hat_u; hat_omega; hat_v; hat_phi] (8×1)
%   rhs - Right-hand side [g_u; g_omega; g_v; g_phi] (8×1)
%   problem - Problem parameters structure
%   discretization - Time and stabilization parameters
%   scMatrices - Pre-computed static condensation matrices
%
% Outputs:
%   U - Local bulk solution [u; omega; v; phi] (8×1)
%   hJ - Local flux jump [hat_j; hat_psi] (8×1)
%   dhJ - Jacobian of flux jump (8×8)
\end{lstlisting}

\paragraph{Static Condensation Steps}
\begin{lstlisting}[language=Matlab, caption=StaticC.m Algorithm Steps]
% Step 1: Solve for u
% (M + dt*tau*M_partial) u = g_u + dt*tau*G_partial*hat_u
u{1} = scMatrices.B1 * hu{1} + scMatrices.L1 * g{1};

% Step 2: Solve for omega (coupled with u)
% Include coupling term dt*d*M*u
y{2} = scMatrices.L2*(g{2} + dt * d * scMatrices.M * y{1});
u{2} = scMatrices.C2 * hu{1} + scMatrices.B2 * hu{2} + y{2};

% Step 3: Compute nonlinear lambda(bar_omega)
bar_omega = scMatrices.Av * u{2};      % Cell average
bl = problem.lambda(bar_omega);        % lambda(bar_omega)
dbl = problem.dlambda(bar_omega);      % lambda'(bar_omega)

% Step 4: Solve for v (nonlinear)
% (A_3 + lambda(bar_omega)*S_3) v = H_3*hat_v + g_v
L3 = (scMatrices.A3 + bl * scMatrices.S3)\eye(2);
u{3} = L3 * scMatrices.H3 * hu{3} + L3 * g{3};

% Step 5: Solve for phi (coupled with v)
u{4} = scMatrices.B4 * hu{4} + scMatrices.C4 * u{3} + scMatrices.L4 * g{4};
\end{lstlisting}

\paragraph{Jacobian Computation}
\begin{lstlisting}[language=Matlab, caption=StaticC.m Jacobian Assembly]
% Build Jacobian dU/d(hat_U) through chain rule
JAC = zeros(8);

% Contribution from each step
JAC = JAC + R{1}' * scMatrices.B1 * R{1};                    % Step 1
JAC = JAC + R{2}' * (scMatrices.C2*R{1} + scMatrices.B2*R{2}); % Step 2

% Nonlinear contribution from lambda'(bar_omega)
J1 = dbl * L3 * scMatrices.S3 * (scMatrices.H3 * hu{3} + g{3}) * scMatrices.Av;
JAC = JAC + R{3}' * (J0 * R{3} + J1 * R{3} * JAC);          % Step 3

JAC = JAC + R{4}' * (scMatrices.B4 * R{4} + scMatrices.C4 * R{3} * JAC); % Step 4
\end{lstlisting}

\subsection{Problem Module Usage Patterns}
\label{subsec:problem_usage_patterns}

\subsubsection{Integration with BioNetFlux Framework}

\begin{lstlisting}[language=Python, caption=Problem Module Usage Example]
from setup_solver import quick_setup

# Load OrganOnChip problem
setup = quick_setup("ooc1d.problems.ooc_test_problem")

# Access problem configuration
info = setup.get_problem_info()
print(f"Problem: {info['problem_name']}")
print(f"Equations: {info['domains'][0]['n_equations']}")
print(f"Elements: {info['domains'][0]['n_elements']}")

# Create initial conditions
trace_solutions, multipliers = setup.create_initial_conditions()

# Time evolution simulation
dt = setup.global_discretization.dt
T = setup.global_discretization.T
n_steps = int(T / dt)

for step in range(n_steps):
    current_time = step * dt
    
    # Newton iteration for current time step
    # ... (using GlobalAssembler and BulkDataManager)
    
    print(f"Time step {step}: t = {current_time:.3f}")
\end{lstlisting}

\subsubsection{Parameter Sensitivity Analysis}

\begin{lstlisting}[language=Python, caption=Parameter Sensitivity Example]
# Modify problem parameters for sensitivity analysis
def create_parameter_sweep():
    """Create multiple problem instances with varying parameters."""
    
    base_setup = quick_setup("ooc1d.problems.ooc_test_problem")
    base_problem = base_setup.problems[0]
    
    # Parameter variations
    chi_values = [0.1, 0.5, 1.0, 2.0, 5.0]  # Chemotactic sensitivity
    
    results = []
    for chi in chi_values:
        # Modify parameters
        new_params = base_problem.parameters.copy()
        new_params[8] = chi  # chi is parameter index 8
        
        # Create new problem with modified parameters
        modified_problem = Problem(
            neq=base_problem.neq,
            domain_start=base_problem.domain_start,
            domain_length=base_problem.domain_length,
            parameters=new_params,
            problem_type=base_problem.type,
            name=f"ooc_chi_{chi}"
        )
        
        # Copy functions and conditions from base problem
        for i in range(base_problem.neq):
            if base_problem.initial_conditions[i] is not None:
                modified_problem.set_initial_condition(i, base_problem.initial_conditions[i])
            if base_problem.force_functions[i] is not None:
                modified_problem.set_force(i, base_problem.force_functions[i])
        
        results.append((chi, modified_problem))
    
    return results

# Usage
parameter_sweep = create_parameter_sweep()
for chi_val, problem in parameter_sweep:
    print(f"Chi = {chi_val}: Problem configured for sensitivity analysis")
\end{lstlisting}

\subsection{Expected Problem Extensions}
\label{subsec:expected_problem_extensions}

\subsubsection{Multi-Domain Problems}

\begin{lstlisting}[language=Python, caption=Expected Multi-Domain Problem Structure]
def create_multi_domain_ooc():
    """
    Expected structure for multi-domain OrganOnChip problems.
    """
    
    # Multiple domains with different parameter sets
    domains = [
        {"start": 0.0, "length": 1.0, "chi": 1.0, "name": "vessel"},
        {"start": 1.0, "length": 0.5, "chi": 2.0, "name": "tissue"},
        {"start": 1.5, "length": 1.0, "chi": 0.5, "name": "organ"}
    ]
    
    problems = []
    discretizations = []
    
    for i, domain_config in enumerate(domains):
        # Create domain-specific problem
        problem = create_domain_problem(domain_config, domain_idx=i)
        problems.append(problem)
        
        # Create domain-specific discretization
        disc = Discretization(
            n_elements=20,
            domain_start=domain_config["start"],
            domain_length=domain_config["length"]
        )
        discretizations.append(disc)
    
    # Setup junction constraints
    constraint_manager = ConstraintManager()
    
    # Continuity at domain interfaces
    constraint_manager.add_trace_continuity(0, 0, 1, -1, 1.0)  # u continuity
    constraint_manager.add_trace_continuity(1, 0, 2, -1, 1.5)  # omega continuity
    
    # Flux continuity with permeability
    constraint_manager.add_kedem_katchalsky(2, 0, 1, -1, 1.0, P=0.1)  # v coupling
    constraint_manager.add_kedem_katchalsky(3, 0, 2, -1, 1.5, P=0.2)  # phi coupling
    
    return problems, GlobalDiscretization(discretizations), constraint_manager, "Multi-Domain OoC"
\end{lstlisting}

\subsubsection{Alternative Nonlinear Functions}

\begin{lstlisting}[language=Python, caption=Expected Alternative Lambda Functions]
# Saturation kinetics (Michaelis-Menten)
def lambda_michaelis_menten(omega, K=1.0, V_max=2.0):
    return V_max * omega / (K + omega)

def dlambda_michaelis_menten(omega, K=1.0, V_max=2.0):
    return V_max * K / (K + omega)**2

# Hill function (cooperative binding)
def lambda_hill(omega, K=1.0, n=2.0):
    omega_n = omega**n
    return omega_n / (K**n + omega_n)

def dlambda_hill(omega, K=1.0, n=2.0):
    omega_n = omega**n
    K_n = K**n
    return n * omega**(n-1) * K_n / (K_n + omega_n)**2

# Exponential activation
def lambda_exponential(omega, alpha=1.0, beta=2.0):
    return alpha * (1.0 - np.exp(-beta * omega))

def dlambda_exponential(omega, alpha=1.0, beta=2.0):
    return alpha * beta * np.exp(-beta * omega)
\end{lstlisting}

\subsection{Problem Module Summary}
\label{subsec:problem_module_summary}

\begin{longtable}{|p{4cm}|p{3cm}|p{6cm}|}
\hline
\textbf{Module/File} & \textbf{Type} & \textbf{Key Features} \\
\hline
\endhead

\texttt{TestProblem.m} & MATLAB Reference & Non-trivial u initial condition, zero sources \\
\hline

\texttt{EmptyProblem.m} & MATLAB Reference & All zero initial conditions, template structure \\
\hline

\texttt{StaticC.m} & MATLAB Algorithm & HDG static condensation with nonlinear lambda \\
\hline

\texttt{scBlocks.m} & MATLAB Documentation & Matrix construction theory and formulas \\
\hline

\texttt{ooc\_test\_problem.py} & Python Implementation & Rich test case with analytical solutions \\
\hline

\end{longtable}

\subsection{Key Design Principles}

\begin{itemize}
    \item \textbf{MATLAB Compatibility}: Python problems maintain mathematical equivalence with MATLAB references
    \item \textbf{Framework Integration}: Seamless integration with BioNetFlux Problem, Discretization, and ConstraintManager classes
    \item \textbf{Flexibility}: Support for arbitrary nonlinear functions, time-dependent data, and multi-domain configurations
    \item \textbf{Validation Ready}: Rich initial conditions and source terms enable analytical solution comparison
    \item \textbf{Biological Relevance}: Parameters and equations model realistic biological transport phenomena
    \item \textbf{Extensibility}: Clear patterns for creating new problem instances and parameter variations
\end{itemize}

This documentation provides comprehensive coverage of the problems folder contents, enabling users to understand, modify, and extend the OrganOnChip model implementations within the BioNetFlux framework.

% End of problems folder API documentation


\section{Problems Folder API Reference}
\label{sec:problems_folder_api}

This section provides detailed documentation for the problem modules in the BioNetFlux problems folder, including their mathematical formulations, parameter specifications, and implementation details based on MATLAB reference implementations.

\subsection{Problems Folder Overview}

The problems folder contains:
\begin{itemize}
    \item \textbf{Organ-on-Chip (OoC) Models}: Multi-equation coupled systems for biological transport
    \item \textbf{MATLAB Reference Implementations}: Original MATLAB problem definitions
    \item \textbf{Python Port Modules}: Corresponding Python implementations using BioNetFlux framework
    \item \textbf{Parameter Configuration}: Physical and numerical parameter specifications
    \item \textbf{Boundary Condition Setup}: Dirichlet, Neumann, and coupling conditions
\end{itemize}

\subsection{Mathematical Model: Organ-on-Chip System}
\label{subsec:ooc_mathematical_model}

The Organ-on-Chip system consists of four coupled partial differential equations describing biological transport phenomena:

\subsubsection{Governing Equations}

\textbf{Cell Density Equation (u):}
\begin{align}
\frac{\partial u}{\partial t} &= \nu \nabla^2 u - \chi \nabla \cdot (u \nabla \phi) + a u + f_u(x,t)
\end{align}

\textbf{Chemical Gradient Equation ($\omega$):}
\begin{align}
\frac{\partial \omega}{\partial t} &= \epsilon \nabla^2 \omega + c \omega + d u + f_\omega(x,t)
\end{align}

\textbf{Velocity Field Equation (v):}
\begin{align}
\frac{\partial v}{\partial t} &= \sigma \nabla^2 v + \lambda(\bar{\omega}) v + f_v(x,t)
\end{align}

\textbf{Potential Field Equation ($\phi$):}
\begin{align}
\frac{\partial \phi}{\partial t} &= \mu \nabla^2 \phi + a \phi + b v + f_\phi(x,t)
\end{align}

where $\bar{\omega} = \frac{1}{|K|} \int_K \omega \, dx$ is the cell-averaged value of $\omega$.

\subsubsection{Parameter Definitions}

\begin{longtable}{|p{2cm}|p{3cm}|p{8cm}|}
\hline
\textbf{Parameter} & \textbf{Physical Meaning} & \textbf{Mathematical Role} \\
\hline
\endhead

$\nu$ & Cell diffusion coefficient & Viscosity/diffusion in u equation \\
\hline

$\mu$ & Potential diffusion & Viscosity/diffusion in $\phi$ equation \\
\hline

$\epsilon$ & Chemical diffusion & Viscosity/diffusion in $\omega$ equation \\
\hline

$\sigma$ & Velocity diffusion & Viscosity/diffusion in v equation \\
\hline

$\chi$ & Chemotactic sensitivity & Coupling strength between u and $\phi$ \\
\hline

$a$ & Growth/decay rate & Reaction term in u and $\phi$ equations \\
\hline

$b$ & Velocity coupling & Coupling strength from v to $\phi$ \\
\hline

$c$ & Chemical reaction rate & Reaction term in $\omega$ equation \\
\hline

$d$ & Cell-chemical coupling & Coupling strength from u to $\omega$ \\
\hline

$\lambda(\cdot)$ & Nonlinear velocity function & Nonlinear reaction in v equation \\
\hline

\end{longtable}

\subsection{MATLAB Reference Problems}
\label{subsec:matlab_reference_problems}

\subsubsection{TestProblem.m}

\paragraph{Problem Configuration}
\begin{lstlisting}[language=Matlab, caption=MATLAB TestProblem.m Parameters]
% Physical parameters (viscosity)
nu = 1.0;
mu = 2.0;
epsilon = 1.0;
sigma = 1.0;

% Reaction parameters
a = 0.0;
c = 0.0;

% Domain
A = 0;      % Domain start
L = 1.0;    % Domain length

% Coupling parameters
b = 1.0;
d = 1.0;
chi = 1.0;
lambda = @(x) constant_function(x);  % Constant nonlinearity
\end{lstlisting}

\paragraph{Initial Conditions}
\begin{lstlisting}[language=Matlab, caption=TestProblem.m Initial Conditions]
% Initial data functions u0{i}(x,t)
problem.u0{1} = @(x,t) sin(2*pi*x);  % Non-trivial for cell density
problem.u0{2} = @(x,t) zeros(size(x)); % Zero chemical
problem.u0{3} = @(x,t) zeros(size(x)); % Zero velocity  
problem.u0{4} = @(x,t) zeros(size(x)); % Zero potential

% Gradient initial condition
problem.phix0 = @(x,t) zeros(size(x));
\end{lstlisting}

\paragraph{Source Terms}
\begin{lstlisting}[language=Matlab, caption=TestProblem.m Source Terms]
% All source terms are zero
force{1} = @(x,t) zeros(size(x));  % f_u = 0
force{2} = @(x,t) zeros(size(x));  % f_omega = 0
force{3} = @(x,t) zeros(size(x));  % f_v = 0
force{4} = @(x,t) zeros(size(x));  % f_phi = 0
\end{lstlisting}

\paragraph{Boundary Conditions}
\begin{lstlisting}[language=Matlab, caption=TestProblem.m Boundary Conditions]
% Neumann flux conditions at both boundaries
% fluxu0{i}(t) = flux at left boundary (x = A)
% fluxu1{i}(t) = flux at right boundary (x = A + L)

problem.fluxu0{1} = @(t) 0.;  % Zero flux for u at x=0
problem.fluxu1{1} = @(t) 0.;  % Zero flux for u at x=1
problem.fluxu0{2} = @(t) 0.;  % Zero flux for omega
problem.fluxu1{2} = @(t) 0.;
problem.fluxu0{3} = @(t) 0.;  % Zero flux for v
problem.fluxu1{3} = @(t) 0.;
problem.fluxu0{4} = @(t) 0.;  % Zero flux for phi
problem.fluxu1{4} = @(t) 0.;

problem.NeumannData = zeros(8,1);  % 2 boundaries × 4 equations
\end{lstlisting}

\subsubsection{EmptyProblem.m}

\paragraph{Differences from TestProblem.m}
\begin{lstlisting}[language=Matlab, caption=EmptyProblem.m Differences]
% Same parameter configuration as TestProblem.m
% Key difference: All initial conditions are zero
problem.u0{1} = @(x,t) zeros(size(x));  % Zero initial cell density
problem.u0{2} = @(x,t) zeros(size(x));
problem.u0{3} = @(x,t) zeros(size(x));
problem.u0{4} = @(x,t) zeros(size(x));

% Updated source terms (if any modifications needed)
force_u = @(x,t) zeros(size(x));
force_phi = @(x,t) zeros(size(x));
force_v = @(x,t) zeros(size(x));
force_omega = @(x,t) zeros(size(x));
\end{lstlisting}

\textbf{Purpose:} EmptyProblem.m serves as a baseline/template with zero initial conditions for testing the numerical framework without initial transients.

\subsection{Python Problem Implementation}
\label{subsec:python_problem_implementation}

\subsubsection{ooc\_test\_problem.py}

\paragraph{Module Structure}
\begin{lstlisting}[language=Python, caption=Python OoC Problem Structure]
# Core imports
from ..core.problem import Problem
from ..core.discretization import Discretization, GlobalDiscretization
from ..core.constraints import ConstraintManager

def create_global_framework():
    """
    OrganOnChip test problem - Python port from MATLAB TestProblem.m
    """
    # Returns: problems, global_discretization, constraint_manager, problem_name
\end{lstlisting}

\paragraph{Parameter Configuration}
\begin{lstlisting}[language=Python, caption=Python Parameter Setup]
# Physical parameters from MATLAB (viscosity)
nu = 1.0
mu = 1.0      # Note: Different from MATLAB (mu = 2.0)
epsilon = 1.0
sigma = 1.0

# Reaction parameters
a = 1.0       # Note: Different from MATLAB (a = 0.0)
c = 1.0       # Note: Different from MATLAB (c = 0.0)

# Coupling parameters
b = 1.0
d = 1.0
chi = 1.0

# Nonlinear lambda function
lambda_function = lambda x: 1.0/(1.0 + x**2)
dlambda_function = lambda x: -2.0*x/(1.0 + x**2)**2

# Parameter vector [nu, mu, epsilon, sigma, a, b, c, d, chi]
parameters = np.array([nu, mu, epsilon, sigma, a, b, c, d, chi])
\end{lstlisting}

\paragraph{Domain and Discretization}
\begin{lstlisting}[language=Python, caption=Python Domain Setup]
# Domain definition
domain_start = 1.0   # Note: Different from MATLAB (A = 0)
domain_length = 1.0  # Same as MATLAB (L = 1.0)

# Mesh parameters
n_elements = 20      # Spatial discretization
neq = 4             # Four equations

# Time parameters
T = 1.0             # Final time
dt = 0.1            # Time step
\end{lstlisting}

\paragraph{Initial Conditions}
\begin{lstlisting}[language=Python, caption=Python Initial Conditions]
# Initial conditions - more complex than MATLAB version
def initial_u(s, t=0.0):
    s = np.asarray(s)
    return 0.0 * s                    # Zero (like EmptyProblem.m)

def initial_omega(s, t=0.0):
    s = np.asarray(s)
    return np.sin(2 * np.pi * s + np.pi * t)  # Time-dependent

def initial_v(s, t=0.0):
    s = np.asarray(s)
    return t * s                      # Time-dependent

def initial_phi(s, t=0.0):
    return s ** 2                     # Quadratic profile
\end{lstlisting}

\paragraph{Source Terms}
\begin{lstlisting}[language=Python, caption=Python Source Terms]
# Non-zero source terms (unlike MATLAB versions)
def force_u(s, t):
    s = np.asarray(s)
    return np.zeros_like(s)           # Still zero for u

def force_omega(s, t):
    s = np.asarray(s)
    x = 2 * np.pi * s + np.pi * t
    return np.sin(x) + 4 * np.pi**2 * np.sin(x) + np.pi * np.cos(x)

def force_v(s, t):
    omega_val = initial_omega(s, t)
    lambda_val = lambda_function(omega_val)
    s = np.asarray(s)
    return s + lambda_val * t * s

def force_phi(s, t):
    s = np.asarray(s)
    return - mu * 2.0 * np.ones_like(s) + a * s**2 - b * t * s
\end{lstlisting}

\paragraph{Boundary Conditions}
\begin{lstlisting}[language=Python, caption=Python Boundary Conditions]
# Analytical flux functions for Neumann conditions
flux_u = lambda s, t: 0.0
flux_omega = lambda s, t: 2 * np.pi * np.cos(2 * np.pi * s + np.pi * t) 
flux_v = lambda s, t: t
flux_phi = lambda s, t: 2 * s

# Setup Neumann boundary conditions for all equations
constraint_manager.add_neumann(0, 0, domain_start, lambda t: -flux_u(domain_start, t))
constraint_manager.add_neumann(0, 0, domain_end, lambda t: flux_u(domain_end, t))

constraint_manager.add_neumann(1, 0, domain_start, lambda t: -flux_omega(domain_start, t))
constraint_manager.add_neumann(1, 0, domain_end, lambda t: flux_omega(domain_end, t))

constraint_manager.add_neumann(2, 0, domain_start, lambda t: -flux_v(domain_start, t))
constraint_manager.add_neumann(2, 0, domain_end, lambda t: flux_v(domain_end, t))

constraint_manager.add_neumann(3, 0, domain_start, lambda t: -flux_phi(domain_start, t))
constraint_manager.add_neumann(3, 0, domain_end, lambda t: flux_phi(domain_end, t))
\end{lstlisting}

\paragraph{Framework Integration}
\begin{lstlisting}[language=Python, caption=BioNetFlux Framework Integration]
# Create Problem instance
problem = Problem(
    neq=neq,
    domain_start=domain_start,
    domain_length=domain_length,
    parameters=parameters,
    problem_type="organ_on_chip",
    name="ooc_test"
)

# Set lambda functions using flexible method
problem.set_function('lambda_function', lambda_function)  
problem.set_function('dlambda_function', dlambda_function)

# Set source terms for all 4 equations
problem.set_force(0, lambda s, t: force_u(s, t))
problem.set_force(1, lambda s, t: force_omega(s, t))
problem.set_force(2, lambda s, t: force_v(s, t))
problem.set_force(3, lambda s, t: force_phi(s, t))

# Set initial conditions
problem.set_initial_condition(0, initial_u)
problem.set_initial_condition(1, initial_omega)
problem.set_initial_condition(2, initial_v) 
problem.set_initial_condition(3, initial_phi)

# Create discretization with stabilization parameters
discretization = Discretization(
    n_elements=n_elements,
    domain_start=domain_start,
    domain_length=domain_length,
    stab_constant=1.0
)

# Set stabilization parameters for all 4 equations
tau_u = 1.0 / discretization.element_length
tau_omega = 1.0
tau_v = 1.0  
tau_phi = 1.0
discretization.set_tau([tau_u, tau_omega, tau_v, tau_phi])
\end{lstlisting}

\subsection{Comparison: MATLAB vs Python Implementations}
\label{subsec:matlab_vs_python_comparison}

\subsubsection{Parameter Differences}

\begin{longtable}{|p{3cm}|p{2cm}|p{2cm}|p{6cm}|}
\hline
\textbf{Parameter} & \textbf{MATLAB} & \textbf{Python} & \textbf{Impact} \\
\hline
\endhead

$\mu$ & 2.0 & 1.0 & Reduced potential diffusion in Python \\
\hline

$a$ & 0.0 & 1.0 & Added growth term in Python u and $\phi$ equations \\
\hline

$c$ & 0.0 & 1.0 & Added reaction term in Python $\omega$ equation \\
\hline

Domain start & 0.0 & 1.0 & Shifted domain in Python implementation \\
\hline

$\lambda(x)$ & constant & $\frac{1}{1+x^2}$ & Nonlinear saturation in Python \\
\hline

\end{longtable}

\subsubsection{Initial Condition Differences}

\begin{longtable}{|p{2cm}|p{4cm}|p{4cm}|p{3cm}|}
\hline
\textbf{Variable} & \textbf{MATLAB TestProblem.m} & \textbf{Python ooc\_test\_problem.py} & \textbf{Complexity} \\
\hline
\endhead

$u$ & $\sin(2\pi x)$ & $0$ & MATLAB: Non-trivial \\
\hline

$\omega$ & $0$ & $\sin(2\pi x + \pi t)$ & Python: Time-dependent \\
\hline

$v$ & $0$ & $t \cdot x$ & Python: Space-time coupling \\
\hline

$\phi$ & $0$ & $x^2$ & Python: Quadratic profile \\
\hline

\end{longtable}

\subsubsection{Source Term Complexity}

\begin{longtable}{|p{2cm}|p{5.5cm}|p{5.5cm}|}
\hline
\textbf{Equation} & \textbf{MATLAB} & \textbf{Python} \\
\hline
\endhead

$f_u$ & $0$ & $0$ \\
\hline

$f_\omega$ & $0$ & $\sin(x) + 4\pi^2\sin(x) + \pi\cos(x)$ \\
\hline

$f_v$ & $0$ & $x + \lambda(\omega(x,t)) \cdot t \cdot x$ \\
\hline

$f_\phi$ & $0$ & $-\mu \cdot 2 + a x^2 - b t x$ \\
\hline

\end{longtable}

\subsection{Static Condensation Implementation}
\label{subsec:static_condensation_implementation}

\subsubsection{StaticC.m Overview}

The MATLAB static condensation implementation performs HDG local assembly:

\paragraph{Input/Output Structure}
\begin{lstlisting}[language=Matlab, caption=StaticC.m Interface]
function [U,hJ,dhJ] = StaticC(hU,rhs,problem,discretization,scMatrices)
% Inputs:
%   hU - Local trace values [hat_u; hat_omega; hat_v; hat_phi] (8×1)
%   rhs - Right-hand side [g_u; g_omega; g_v; g_phi] (8×1)
%   problem - Problem parameters structure
%   discretization - Time and stabilization parameters
%   scMatrices - Pre-computed static condensation matrices
%
% Outputs:
%   U - Local bulk solution [u; omega; v; phi] (8×1)
%   hJ - Local flux jump [hat_j; hat_psi] (8×1)
%   dhJ - Jacobian of flux jump (8×8)
\end{lstlisting}

\paragraph{Static Condensation Steps}
\begin{lstlisting}[language=Matlab, caption=StaticC.m Algorithm Steps]
% Step 1: Solve for u
% (M + dt*tau*M_partial) u = g_u + dt*tau*G_partial*hat_u
u{1} = scMatrices.B1 * hu{1} + scMatrices.L1 * g{1};

% Step 2: Solve for omega (coupled with u)
% Include coupling term dt*d*M*u
y{2} = scMatrices.L2*(g{2} + dt * d * scMatrices.M * y{1});
u{2} = scMatrices.C2 * hu{1} + scMatrices.B2 * hu{2} + y{2};

% Step 3: Compute nonlinear lambda(bar_omega)
bar_omega = scMatrices.Av * u{2};      % Cell average
bl = problem.lambda(bar_omega);        % lambda(bar_omega)
dbl = problem.dlambda(bar_omega);      % lambda'(bar_omega)

% Step 4: Solve for v (nonlinear)
% (A_3 + lambda(bar_omega)*S_3) v = H_3*hat_v + g_v
L3 = (scMatrices.A3 + bl * scMatrices.S3)\eye(2);
u{3} = L3 * scMatrices.H3 * hu{3} + L3 * g{3};

% Step 5: Solve for phi (coupled with v)
u{4} = scMatrices.B4 * hu{4} + scMatrices.C4 * u{3} + scMatrices.L4 * g{4};
\end{lstlisting}

\paragraph{Jacobian Computation}
\begin{lstlisting}[language=Matlab, caption=StaticC.m Jacobian Assembly]
% Build Jacobian dU/d(hat_U) through chain rule
JAC = zeros(8);

% Contribution from each step
JAC = JAC + R{1}' * scMatrices.B1 * R{1};                    % Step 1
JAC = JAC + R{2}' * (scMatrices.C2*R{1} + scMatrices.B2*R{2}); % Step 2

% Nonlinear contribution from lambda'(bar_omega)
J1 = dbl * L3 * scMatrices.S3 * (scMatrices.H3 * hu{3} + g{3}) * scMatrices.Av;
JAC = JAC + R{3}' * (J0 * R{3} + J1 * R{3} * JAC);          % Step 3

JAC = JAC + R{4}' * (scMatrices.B4 * R{4} + scMatrices.C4 * R{3} * JAC); % Step 4
\end{lstlisting}

\subsection{Problem Module Usage Patterns}
\label{subsec:problem_usage_patterns}

\subsubsection{Integration with BioNetFlux Framework}

\begin{lstlisting}[language=Python, caption=Problem Module Usage Example]
from setup_solver import quick_setup

# Load OrganOnChip problem
setup = quick_setup("ooc1d.problems.ooc_test_problem")

# Access problem configuration
info = setup.get_problem_info()
print(f"Problem: {info['problem_name']}")
print(f"Equations: {info['domains'][0]['n_equations']}")
print(f"Elements: {info['domains'][0]['n_elements']}")

# Create initial conditions
trace_solutions, multipliers = setup.create_initial_conditions()

# Time evolution simulation
dt = setup.global_discretization.dt
T = setup.global_discretization.T
n_steps = int(T / dt)

for step in range(n_steps):
    current_time = step * dt
    
    # Newton iteration for current time step
    # ... (using GlobalAssembler and BulkDataManager)
    
    print(f"Time step {step}: t = {current_time:.3f}")
\end{lstlisting}

\subsubsection{Parameter Sensitivity Analysis}

\begin{lstlisting}[language=Python, caption=Parameter Sensitivity Example]
# Modify problem parameters for sensitivity analysis
def create_parameter_sweep():
    """Create multiple problem instances with varying parameters."""
    
    base_setup = quick_setup("ooc1d.problems.ooc_test_problem")
    base_problem = base_setup.problems[0]
    
    # Parameter variations
    chi_values = [0.1, 0.5, 1.0, 2.0, 5.0]  # Chemotactic sensitivity
    
    results = []
    for chi in chi_values:
        # Modify parameters
        new_params = base_problem.parameters.copy()
        new_params[8] = chi  # chi is parameter index 8
        
        # Create new problem with modified parameters
        modified_problem = Problem(
            neq=base_problem.neq,
            domain_start=base_problem.domain_start,
            domain_length=base_problem.domain_length,
            parameters=new_params,
            problem_type=base_problem.type,
            name=f"ooc_chi_{chi}"
        )
        
        # Copy functions and conditions from base problem
        for i in range(base_problem.neq):
            if base_problem.initial_conditions[i] is not None:
                modified_problem.set_initial_condition(i, base_problem.initial_conditions[i])
            if base_problem.force_functions[i] is not None:
                modified_problem.set_force(i, base_problem.force_functions[i])
        
        results.append((chi, modified_problem))
    
    return results

# Usage
parameter_sweep = create_parameter_sweep()
for chi_val, problem in parameter_sweep:
    print(f"Chi = {chi_val}: Problem configured for sensitivity analysis")
\end{lstlisting}

\subsection{Expected Problem Extensions}
\label{subsec:expected_problem_extensions}

\subsubsection{Multi-Domain Problems}

\begin{lstlisting}[language=Python, caption=Expected Multi-Domain Problem Structure]
def create_multi_domain_ooc():
    """
    Expected structure for multi-domain OrganOnChip problems.
    """
    
    # Multiple domains with different parameter sets
    domains = [
        {"start": 0.0, "length": 1.0, "chi": 1.0, "name": "vessel"},
        {"start": 1.0, "length": 0.5, "chi": 2.0, "name": "tissue"},
        {"start": 1.5, "length": 1.0, "chi": 0.5, "name": "organ"}
    ]
    
    problems = []
    discretizations = []
    
    for i, domain_config in enumerate(domains):
        # Create domain-specific problem
        problem = create_domain_problem(domain_config, domain_idx=i)
        problems.append(problem)
        
        # Create domain-specific discretization
        disc = Discretization(
            n_elements=20,
            domain_start=domain_config["start"],
            domain_length=domain_config["length"]
        )
        discretizations.append(disc)
    
    # Setup junction constraints
    constraint_manager = ConstraintManager()
    
    # Continuity at domain interfaces
    constraint_manager.add_trace_continuity(0, 0, 1, -1, 1.0)  # u continuity
    constraint_manager.add_trace_continuity(1, 0, 2, -1, 1.5)  # omega continuity
    
    # Flux continuity with permeability
    constraint_manager.add_kedem_katchalsky(2, 0, 1, -1, 1.0, P=0.1)  # v coupling
    constraint_manager.add_kedem_katchalsky(3, 0, 2, -1, 1.5, P=0.2)  # phi coupling
    
    return problems, GlobalDiscretization(discretizations), constraint_manager, "Multi-Domain OoC"
\end{lstlisting}

\subsubsection{Alternative Nonlinear Functions}

\begin{lstlisting}[language=Python, caption=Expected Alternative Lambda Functions]
# Saturation kinetics (Michaelis-Menten)
def lambda_michaelis_menten(omega, K=1.0, V_max=2.0):
    return V_max * omega / (K + omega)

def dlambda_michaelis_menten(omega, K=1.0, V_max=2.0):
    return V_max * K / (K + omega)**2

# Hill function (cooperative binding)
def lambda_hill(omega, K=1.0, n=2.0):
    omega_n = omega**n
    return omega_n / (K**n + omega_n)

def dlambda_hill(omega, K=1.0, n=2.0):
    omega_n = omega**n
    K_n = K**n
    return n * omega**(n-1) * K_n / (K_n + omega_n)**2

# Exponential activation
def lambda_exponential(omega, alpha=1.0, beta=2.0):
    return alpha * (1.0 - np.exp(-beta * omega))

def dlambda_exponential(omega, alpha=1.0, beta=2.0):
    return alpha * beta * np.exp(-beta * omega)
\end{lstlisting}

\subsection{Problem Module Summary}
\label{subsec:problem_module_summary}

\begin{longtable}{|p{4cm}|p{3cm}|p{6cm}|}
\hline
\textbf{Module/File} & \textbf{Type} & \textbf{Key Features} \\
\hline
\endhead

\texttt{TestProblem.m} & MATLAB Reference & Non-trivial u initial condition, zero sources \\
\hline

\texttt{EmptyProblem.m} & MATLAB Reference & All zero initial conditions, template structure \\
\hline

\texttt{StaticC.m} & MATLAB Algorithm & HDG static condensation with nonlinear lambda \\
\hline

\texttt{scBlocks.m} & MATLAB Documentation & Matrix construction theory and formulas \\
\hline

\texttt{ooc\_test\_problem.py} & Python Implementation & Rich test case with analytical solutions \\
\hline

\end{longtable}

\subsection{Key Design Principles}

\begin{itemize}
    \item \textbf{MATLAB Compatibility}: Python problems maintain mathematical equivalence with MATLAB references
    \item \textbf{Framework Integration}: Seamless integration with BioNetFlux Problem, Discretization, and ConstraintManager classes
    \item \textbf{Flexibility}: Support for arbitrary nonlinear functions, time-dependent data, and multi-domain configurations
    \item \textbf{Validation Ready}: Rich initial conditions and source terms enable analytical solution comparison
    \item \textbf{Biological Relevance}: Parameters and equations model realistic biological transport phenomena
    \item \textbf{Extensibility}: Clear patterns for creating new problem instances and parameter variations
\end{itemize}

This documentation provides comprehensive coverage of the problems folder contents, enabling users to understand, modify, and extend the OrganOnChip model implementations within the BioNetFlux framework.

% End of problems folder API documentation


\section{Problems Folder API Reference}
\label{sec:problems_folder_api}

This section provides detailed documentation for the problem modules in the BioNetFlux problems folder, including their mathematical formulations, parameter specifications, and implementation details based on MATLAB reference implementations.

\subsection{Problems Folder Overview}

The problems folder contains:
\begin{itemize}
    \item \textbf{Organ-on-Chip (OoC) Models}: Multi-equation coupled systems for biological transport
    \item \textbf{MATLAB Reference Implementations}: Original MATLAB problem definitions
    \item \textbf{Python Port Modules}: Corresponding Python implementations using BioNetFlux framework
    \item \textbf{Parameter Configuration}: Physical and numerical parameter specifications
    \item \textbf{Boundary Condition Setup}: Dirichlet, Neumann, and coupling conditions
\end{itemize}

\subsection{Mathematical Model: Organ-on-Chip System}
\label{subsec:ooc_mathematical_model}

The Organ-on-Chip system consists of four coupled partial differential equations describing biological transport phenomena:

\subsubsection{Governing Equations}

\textbf{Cell Density Equation (u):}
\begin{align}
\frac{\partial u}{\partial t} &= \nu \nabla^2 u - \chi \nabla \cdot (u \nabla \phi) + a u + f_u(x,t)
\end{align}

\textbf{Chemical Gradient Equation ($\omega$):}
\begin{align}
\frac{\partial \omega}{\partial t} &= \epsilon \nabla^2 \omega + c \omega + d u + f_\omega(x,t)
\end{align}

\textbf{Velocity Field Equation (v):}
\begin{align}
\frac{\partial v}{\partial t} &= \sigma \nabla^2 v + \lambda(\bar{\omega}) v + f_v(x,t)
\end{align}

\textbf{Potential Field Equation ($\phi$):}
\begin{align}
\frac{\partial \phi}{\partial t} &= \mu \nabla^2 \phi + a \phi + b v + f_\phi(x,t)
\end{align}

where $\bar{\omega} = \frac{1}{|K|} \int_K \omega \, dx$ is the cell-averaged value of $\omega$.

\subsubsection{Parameter Definitions}

\begin{longtable}{|p{2cm}|p{3cm}|p{8cm}|}
\hline
\textbf{Parameter} & \textbf{Physical Meaning} & \textbf{Mathematical Role} \\
\hline
\endhead

$\nu$ & Cell diffusion coefficient & Viscosity/diffusion in u equation \\
\hline

$\mu$ & Potential diffusion & Viscosity/diffusion in $\phi$ equation \\
\hline

$\epsilon$ & Chemical diffusion & Viscosity/diffusion in $\omega$ equation \\
\hline

$\sigma$ & Velocity diffusion & Viscosity/diffusion in v equation \\
\hline

$\chi$ & Chemotactic sensitivity & Coupling strength between u and $\phi$ \\
\hline

$a$ & Growth/decay rate & Reaction term in u and $\phi$ equations \\
\hline

$b$ & Velocity coupling & Coupling strength from v to $\phi$ \\
\hline

$c$ & Chemical reaction rate & Reaction term in $\omega$ equation \\
\hline

$d$ & Cell-chemical coupling & Coupling strength from u to $\omega$ \\
\hline

$\lambda(\cdot)$ & Nonlinear velocity function & Nonlinear reaction in v equation \\
\hline

\end{longtable}

\subsection{MATLAB Reference Problems}
\label{subsec:matlab_reference_problems}

\subsubsection{TestProblem.m}

\paragraph{Problem Configuration}
\begin{lstlisting}[language=Matlab, caption=MATLAB TestProblem.m Parameters]
% Physical parameters (viscosity)
nu = 1.0;
mu = 2.0;
epsilon = 1.0;
sigma = 1.0;

% Reaction parameters
a = 0.0;
c = 0.0;

% Domain
A = 0;      % Domain start
L = 1.0;    % Domain length

% Coupling parameters
b = 1.0;
d = 1.0;
chi = 1.0;
lambda = @(x) constant_function(x);  % Constant nonlinearity
\end{lstlisting}

\paragraph{Initial Conditions}
\begin{lstlisting}[language=Matlab, caption=TestProblem.m Initial Conditions]
% Initial data functions u0{i}(x,t)
problem.u0{1} = @(x,t) sin(2*pi*x);  % Non-trivial for cell density
problem.u0{2} = @(x,t) zeros(size(x)); % Zero chemical
problem.u0{3} = @(x,t) zeros(size(x)); % Zero velocity  
problem.u0{4} = @(x,t) zeros(size(x)); % Zero potential

% Gradient initial condition
problem.phix0 = @(x,t) zeros(size(x));
\end{lstlisting}

\paragraph{Source Terms}
\begin{lstlisting}[language=Matlab, caption=TestProblem.m Source Terms]
% All source terms are zero
force{1} = @(x,t) zeros(size(x));  % f_u = 0
force{2} = @(x,t) zeros(size(x));  % f_omega = 0
force{3} = @(x,t) zeros(size(x));  % f_v = 0
force{4} = @(x,t) zeros(size(x));  % f_phi = 0
\end{lstlisting}

\paragraph{Boundary Conditions}
\begin{lstlisting}[language=Matlab, caption=TestProblem.m Boundary Conditions]
% Neumann flux conditions at both boundaries
% fluxu0{i}(t) = flux at left boundary (x = A)
% fluxu1{i}(t) = flux at right boundary (x = A + L)

problem.fluxu0{1} = @(t) 0.;  % Zero flux for u at x=0
problem.fluxu1{1} = @(t) 0.;  % Zero flux for u at x=1
problem.fluxu0{2} = @(t) 0.;  % Zero flux for omega
problem.fluxu1{2} = @(t) 0.;
problem.fluxu0{3} = @(t) 0.;  % Zero flux for v
problem.fluxu1{3} = @(t) 0.;
problem.fluxu0{4} = @(t) 0.;  % Zero flux for phi
problem.fluxu1{4} = @(t) 0.;

problem.NeumannData = zeros(8,1);  % 2 boundaries × 4 equations
\end{lstlisting}

\subsubsection{EmptyProblem.m}

\paragraph{Differences from TestProblem.m}
\begin{lstlisting}[language=Matlab, caption=EmptyProblem.m Differences]
% Same parameter configuration as TestProblem.m
% Key difference: All initial conditions are zero
problem.u0{1} = @(x,t) zeros(size(x));  % Zero initial cell density
problem.u0{2} = @(x,t) zeros(size(x));
problem.u0{3} = @(x,t) zeros(size(x));
problem.u0{4} = @(x,t) zeros(size(x));

% Updated source terms (if any modifications needed)
force_u = @(x,t) zeros(size(x));
force_phi = @(x,t) zeros(size(x));
force_v = @(x,t) zeros(size(x));
force_omega = @(x,t) zeros(size(x));
\end{lstlisting}

\textbf{Purpose:} EmptyProblem.m serves as a baseline/template with zero initial conditions for testing the numerical framework without initial transients.

\subsection{Python Problem Implementation}
\label{subsec:python_problem_implementation}

\subsubsection{ooc\_test\_problem.py}

\paragraph{Module Structure}
\begin{lstlisting}[language=Python, caption=Python OoC Problem Structure]
# Core imports
from ..core.problem import Problem
from ..core.discretization import Discretization, GlobalDiscretization
from ..core.constraints import ConstraintManager

def create_global_framework():
    """
    OrganOnChip test problem - Python port from MATLAB TestProblem.m
    """
    # Returns: problems, global_discretization, constraint_manager, problem_name
\end{lstlisting}

\paragraph{Parameter Configuration}
\begin{lstlisting}[language=Python, caption=Python Parameter Setup]
# Physical parameters from MATLAB (viscosity)
nu = 1.0
mu = 1.0      # Note: Different from MATLAB (mu = 2.0)
epsilon = 1.0
sigma = 1.0

# Reaction parameters
a = 1.0       # Note: Different from MATLAB (a = 0.0)
c = 1.0       # Note: Different from MATLAB (c = 0.0)

# Coupling parameters
b = 1.0
d = 1.0
chi = 1.0

# Nonlinear lambda function
lambda_function = lambda x: 1.0/(1.0 + x**2)
dlambda_function = lambda x: -2.0*x/(1.0 + x**2)**2

# Parameter vector [nu, mu, epsilon, sigma, a, b, c, d, chi]
parameters = np.array([nu, mu, epsilon, sigma, a, b, c, d, chi])
\end{lstlisting}

\paragraph{Domain and Discretization}
\begin{lstlisting}[language=Python, caption=Python Domain Setup]
# Domain definition
domain_start = 1.0   # Note: Different from MATLAB (A = 0)
domain_length = 1.0  # Same as MATLAB (L = 1.0)

# Mesh parameters
n_elements = 20      # Spatial discretization
neq = 4             # Four equations

# Time parameters
T = 1.0             # Final time
dt = 0.1            # Time step
\end{lstlisting}

\paragraph{Initial Conditions}
\begin{lstlisting}[language=Python, caption=Python Initial Conditions]
# Initial conditions - more complex than MATLAB version
def initial_u(s, t=0.0):
    s = np.asarray(s)
    return 0.0 * s                    # Zero (like EmptyProblem.m)

def initial_omega(s, t=0.0):
    s = np.asarray(s)
    return np.sin(2 * np.pi * s + np.pi * t)  # Time-dependent

def initial_v(s, t=0.0):
    s = np.asarray(s)
    return t * s                      # Time-dependent

def initial_phi(s, t=0.0):
    return s ** 2                     # Quadratic profile
\end{lstlisting}

\paragraph{Source Terms}
\begin{lstlisting}[language=Python, caption=Python Source Terms]
# Non-zero source terms (unlike MATLAB versions)
def force_u(s, t):
    s = np.asarray(s)
    return np.zeros_like(s)           # Still zero for u

def force_omega(s, t):
    s = np.asarray(s)
    x = 2 * np.pi * s + np.pi * t
    return np.sin(x) + 4 * np.pi**2 * np.sin(x) + np.pi * np.cos(x)

def force_v(s, t):
    omega_val = initial_omega(s, t)
    lambda_val = lambda_function(omega_val)
    s = np.asarray(s)
    return s + lambda_val * t * s

def force_phi(s, t):
    s = np.asarray(s)
    return - mu * 2.0 * np.ones_like(s) + a * s**2 - b * t * s
\end{lstlisting}

\paragraph{Boundary Conditions}
\begin{lstlisting}[language=Python, caption=Python Boundary Conditions]
# Analytical flux functions for Neumann conditions
flux_u = lambda s, t: 0.0
flux_omega = lambda s, t: 2 * np.pi * np.cos(2 * np.pi * s + np.pi * t) 
flux_v = lambda s, t: t
flux_phi = lambda s, t: 2 * s

# Setup Neumann boundary conditions for all equations
constraint_manager.add_neumann(0, 0, domain_start, lambda t: -flux_u(domain_start, t))
constraint_manager.add_neumann(0, 0, domain_end, lambda t: flux_u(domain_end, t))

constraint_manager.add_neumann(1, 0, domain_start, lambda t: -flux_omega(domain_start, t))
constraint_manager.add_neumann(1, 0, domain_end, lambda t: flux_omega(domain_end, t))

constraint_manager.add_neumann(2, 0, domain_start, lambda t: -flux_v(domain_start, t))
constraint_manager.add_neumann(2, 0, domain_end, lambda t: flux_v(domain_end, t))

constraint_manager.add_neumann(3, 0, domain_start, lambda t: -flux_phi(domain_start, t))
constraint_manager.add_neumann(3, 0, domain_end, lambda t: flux_phi(domain_end, t))
\end{lstlisting}

\paragraph{Framework Integration}
\begin{lstlisting}[language=Python, caption=BioNetFlux Framework Integration]
# Create Problem instance
problem = Problem(
    neq=neq,
    domain_start=domain_start,
    domain_length=domain_length,
    parameters=parameters,
    problem_type="organ_on_chip",
    name="ooc_test"
)

# Set lambda functions using flexible method
problem.set_function('lambda_function', lambda_function)  
problem.set_function('dlambda_function', dlambda_function)

# Set source terms for all 4 equations
problem.set_force(0, lambda s, t: force_u(s, t))
problem.set_force(1, lambda s, t: force_omega(s, t))
problem.set_force(2, lambda s, t: force_v(s, t))
problem.set_force(3, lambda s, t: force_phi(s, t))

# Set initial conditions
problem.set_initial_condition(0, initial_u)
problem.set_initial_condition(1, initial_omega)
problem.set_initial_condition(2, initial_v) 
problem.set_initial_condition(3, initial_phi)

# Create discretization with stabilization parameters
discretization = Discretization(
    n_elements=n_elements,
    domain_start=domain_start,
    domain_length=domain_length,
    stab_constant=1.0
)

# Set stabilization parameters for all 4 equations
tau_u = 1.0 / discretization.element_length
tau_omega = 1.0
tau_v = 1.0  
tau_phi = 1.0
discretization.set_tau([tau_u, tau_omega, tau_v, tau_phi])
\end{lstlisting}

\subsection{Comparison: MATLAB vs Python Implementations}
\label{subsec:matlab_vs_python_comparison}

\subsubsection{Parameter Differences}

\begin{longtable}{|p{3cm}|p{2cm}|p{2cm}|p{6cm}|}
\hline
\textbf{Parameter} & \textbf{MATLAB} & \textbf{Python} & \textbf{Impact} \\
\hline
\endhead

$\mu$ & 2.0 & 1.0 & Reduced potential diffusion in Python \\
\hline

$a$ & 0.0 & 1.0 & Added growth term in Python u and $\phi$ equations \\
\hline

$c$ & 0.0 & 1.0 & Added reaction term in Python $\omega$ equation \\
\hline

Domain start & 0.0 & 1.0 & Shifted domain in Python implementation \\
\hline

$\lambda(x)$ & constant & $\frac{1}{1+x^2}$ & Nonlinear saturation in Python \\
\hline

\end{longtable}

\subsubsection{Initial Condition Differences}

\begin{longtable}{|p{2cm}|p{4cm}|p{4cm}|p{3cm}|}
\hline
\textbf{Variable} & \textbf{MATLAB TestProblem.m} & \textbf{Python ooc\_test\_problem.py} & \textbf{Complexity} \\
\hline
\endhead

$u$ & $\sin(2\pi x)$ & $0$ & MATLAB: Non-trivial \\
\hline

$\omega$ & $0$ & $\sin(2\pi x + \pi t)$ & Python: Time-dependent \\
\hline

$v$ & $0$ & $t \cdot x$ & Python: Space-time coupling \\
\hline

$\phi$ & $0$ & $x^2$ & Python: Quadratic profile \\
\hline

\end{longtable}

\subsubsection{Source Term Complexity}

\begin{longtable}{|p{2cm}|p{5.5cm}|p{5.5cm}|}
\hline
\textbf{Equation} & \textbf{MATLAB} & \textbf{Python} \\
\hline
\endhead

$f_u$ & $0$ & $0$ \\
\hline

$f_\omega$ & $0$ & $\sin(x) + 4\pi^2\sin(x) + \pi\cos(x)$ \\
\hline

$f_v$ & $0$ & $x + \lambda(\omega(x,t)) \cdot t \cdot x$ \\
\hline

$f_\phi$ & $0$ & $-\mu \cdot 2 + a x^2 - b t x$ \\
\hline

\end{longtable}

\subsection{Static Condensation Implementation}
\label{subsec:static_condensation_implementation}

\subsubsection{StaticC.m Overview}

The MATLAB static condensation implementation performs HDG local assembly:

\paragraph{Input/Output Structure}
\begin{lstlisting}[language=Matlab, caption=StaticC.m Interface]
function [U,hJ,dhJ] = StaticC(hU,rhs,problem,discretization,scMatrices)
% Inputs:
%   hU - Local trace values [hat_u; hat_omega; hat_v; hat_phi] (8×1)
%   rhs - Right-hand side [g_u; g_omega; g_v; g_phi] (8×1)
%   problem - Problem parameters structure
%   discretization - Time and stabilization parameters
%   scMatrices - Pre-computed static condensation matrices
%
% Outputs:
%   U - Local bulk solution [u; omega; v; phi] (8×1)
%   hJ - Local flux jump [hat_j; hat_psi] (8×1)
%   dhJ - Jacobian of flux jump (8×8)
\end{lstlisting}

\paragraph{Static Condensation Steps}
\begin{lstlisting}[language=Matlab, caption=StaticC.m Algorithm Steps]
% Step 1: Solve for u
% (M + dt*tau*M_partial) u = g_u + dt*tau*G_partial*hat_u
u{1} = scMatrices.B1 * hu{1} + scMatrices.L1 * g{1};

% Step 2: Solve for omega (coupled with u)
% Include coupling term dt*d*M*u
y{2} = scMatrices.L2*(g{2} + dt * d * scMatrices.M * y{1});
u{2} = scMatrices.C2 * hu{1} + scMatrices.B2 * hu{2} + y{2};

% Step 3: Compute nonlinear lambda(bar_omega)
bar_omega = scMatrices.Av * u{2};      % Cell average
bl = problem.lambda(bar_omega);        % lambda(bar_omega)
dbl = problem.dlambda(bar_omega);      % lambda'(bar_omega)

% Step 4: Solve for v (nonlinear)
% (A_3 + lambda(bar_omega)*S_3) v = H_3*hat_v + g_v
L3 = (scMatrices.A3 + bl * scMatrices.S3)\eye(2);
u{3} = L3 * scMatrices.H3 * hu{3} + L3 * g{3};

% Step 5: Solve for phi (coupled with v)
u{4} = scMatrices.B4 * hu{4} + scMatrices.C4 * u{3} + scMatrices.L4 * g{4};
\end{lstlisting}

\paragraph{Jacobian Computation}
\begin{lstlisting}[language=Matlab, caption=StaticC.m Jacobian Assembly]
% Build Jacobian dU/d(hat_U) through chain rule
JAC = zeros(8);

% Contribution from each step
JAC = JAC + R{1}' * scMatrices.B1 * R{1};                    % Step 1
JAC = JAC + R{2}' * (scMatrices.C2*R{1} + scMatrices.B2*R{2}); % Step 2

% Nonlinear contribution from lambda'(bar_omega)
J1 = dbl * L3 * scMatrices.S3 * (scMatrices.H3 * hu{3} + g{3}) * scMatrices.Av;
JAC = JAC + R{3}' * (J0 * R{3} + J1 * R{3} * JAC);          % Step 3

JAC = JAC + R{4}' * (scMatrices.B4 * R{4} + scMatrices.C4 * R{3} * JAC); % Step 4
\end{lstlisting}

\subsection{Problem Module Usage Patterns}
\label{subsec:problem_usage_patterns}

\subsubsection{Integration with BioNetFlux Framework}

\begin{lstlisting}[language=Python, caption=Problem Module Usage Example]
from setup_solver import quick_setup

# Load OrganOnChip problem
setup = quick_setup("ooc1d.problems.ooc_test_problem")

# Access problem configuration
info = setup.get_problem_info()
print(f"Problem: {info['problem_name']}")
print(f"Equations: {info['domains'][0]['n_equations']}")
print(f"Elements: {info['domains'][0]['n_elements']}")

# Create initial conditions
trace_solutions, multipliers = setup.create_initial_conditions()

# Time evolution simulation
dt = setup.global_discretization.dt
T = setup.global_discretization.T
n_steps = int(T / dt)

for step in range(n_steps):
    current_time = step * dt
    
    # Newton iteration for current time step
    # ... (using GlobalAssembler and BulkDataManager)
    
    print(f"Time step {step}: t = {current_time:.3f}")
\end{lstlisting}

\subsubsection{Parameter Sensitivity Analysis}

\begin{lstlisting}[language=Python, caption=Parameter Sensitivity Example]
# Modify problem parameters for sensitivity analysis
def create_parameter_sweep():
    """Create multiple problem instances with varying parameters."""
    
    base_setup = quick_setup("ooc1d.problems.ooc_test_problem")
    base_problem = base_setup.problems[0]
    
    # Parameter variations
    chi_values = [0.1, 0.5, 1.0, 2.0, 5.0]  # Chemotactic sensitivity
    
    results = []
    for chi in chi_values:
        # Modify parameters
        new_params = base_problem.parameters.copy()
        new_params[8] = chi  # chi is parameter index 8
        
        # Create new problem with modified parameters
        modified_problem = Problem(
            neq=base_problem.neq,
            domain_start=base_problem.domain_start,
            domain_length=base_problem.domain_length,
            parameters=new_params,
            problem_type=base_problem.type,
            name=f"ooc_chi_{chi}"
        )
        
        # Copy functions and conditions from base problem
        for i in range(base_problem.neq):
            if base_problem.initial_conditions[i] is not None:
                modified_problem.set_initial_condition(i, base_problem.initial_conditions[i])
            if base_problem.force_functions[i] is not None:
                modified_problem.set_force(i, base_problem.force_functions[i])
        
        results.append((chi, modified_problem))
    
    return results

# Usage
parameter_sweep = create_parameter_sweep()
for chi_val, problem in parameter_sweep:
    print(f"Chi = {chi_val}: Problem configured for sensitivity analysis")
\end{lstlisting}

\subsection{Expected Problem Extensions}
\label{subsec:expected_problem_extensions}

\subsubsection{Multi-Domain Problems}

\begin{lstlisting}[language=Python, caption=Expected Multi-Domain Problem Structure]
def create_multi_domain_ooc():
    """
    Expected structure for multi-domain OrganOnChip problems.
    """
    
    # Multiple domains with different parameter sets
    domains = [
        {"start": 0.0, "length": 1.0, "chi": 1.0, "name": "vessel"},
        {"start": 1.0, "length": 0.5, "chi": 2.0, "name": "tissue"},
        {"start": 1.5, "length": 1.0, "chi": 0.5, "name": "organ"}
    ]
    
    problems = []
    discretizations = []
    
    for i, domain_config in enumerate(domains):
        # Create domain-specific problem
        problem = create_domain_problem(domain_config, domain_idx=i)
        problems.append(problem)
        
        # Create domain-specific discretization
        disc = Discretization(
            n_elements=20,
            domain_start=domain_config["start"],
            domain_length=domain_config["length"]
        )
        discretizations.append(disc)
    
    # Setup junction constraints
    constraint_manager = ConstraintManager()
    
    # Continuity at domain interfaces
    constraint_manager.add_trace_continuity(0, 0, 1, -1, 1.0)  # u continuity
    constraint_manager.add_trace_continuity(1, 0, 2, -1, 1.5)  # omega continuity
    
    # Flux continuity with permeability
    constraint_manager.add_kedem_katchalsky(2, 0, 1, -1, 1.0, P=0.1)  # v coupling
    constraint_manager.add_kedem_katchalsky(3, 0, 2, -1, 1.5, P=0.2)  # phi coupling
    
    return problems, GlobalDiscretization(discretizations), constraint_manager, "Multi-Domain OoC"
\end{lstlisting}

\subsubsection{Alternative Nonlinear Functions}

\begin{lstlisting}[language=Python, caption=Expected Alternative Lambda Functions]
# Saturation kinetics (Michaelis-Menten)
def lambda_michaelis_menten(omega, K=1.0, V_max=2.0):
    return V_max * omega / (K + omega)

def dlambda_michaelis_menten(omega, K=1.0, V_max=2.0):
    return V_max * K / (K + omega)**2

# Hill function (cooperative binding)
def lambda_hill(omega, K=1.0, n=2.0):
    omega_n = omega**n
    return omega_n / (K**n + omega_n)

def dlambda_hill(omega, K=1.0, n=2.0):
    omega_n = omega**n
    K_n = K**n
    return n * omega**(n-1) * K_n / (K_n + omega_n)**2

# Exponential activation
def lambda_exponential(omega, alpha=1.0, beta=2.0):
    return alpha * (1.0 - np.exp(-beta * omega))

def dlambda_exponential(omega, alpha=1.0, beta=2.0):
    return alpha * beta * np.exp(-beta * omega)
\end{lstlisting}

\subsection{Problem Module Summary}
\label{subsec:problem_module_summary}

\begin{longtable}{|p{4cm}|p{3cm}|p{6cm}|}
\hline
\textbf{Module/File} & \textbf{Type} & \textbf{Key Features} \\
\hline
\endhead

\texttt{TestProblem.m} & MATLAB Reference & Non-trivial u initial condition, zero sources \\
\hline

\texttt{EmptyProblem.m} & MATLAB Reference & All zero initial conditions, template structure \\
\hline

\texttt{StaticC.m} & MATLAB Algorithm & HDG static condensation with nonlinear lambda \\
\hline

\texttt{scBlocks.m} & MATLAB Documentation & Matrix construction theory and formulas \\
\hline

\texttt{ooc\_test\_problem.py} & Python Implementation & Rich test case with analytical solutions \\
\hline

\end{longtable}

\subsection{Key Design Principles}

\begin{itemize}
    \item \textbf{MATLAB Compatibility}: Python problems maintain mathematical equivalence with MATLAB references
    \item \textbf{Framework Integration}: Seamless integration with BioNetFlux Problem, Discretization, and ConstraintManager classes
    \item \textbf{Flexibility}: Support for arbitrary nonlinear functions, time-dependent data, and multi-domain configurations
    \item \textbf{Validation Ready}: Rich initial conditions and source terms enable analytical solution comparison
    \item \textbf{Biological Relevance}: Parameters and equations model realistic biological transport phenomena
    \item \textbf{Extensibility}: Clear patterns for creating new problem instances and parameter variations
\end{itemize}

This documentation provides comprehensive coverage of the problems folder contents, enabling users to understand, modify, and extend the OrganOnChip model implementations within the BioNetFlux framework.

% End of problems folder API documentation
