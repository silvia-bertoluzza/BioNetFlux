% Flux Jump Module API Documentation (Accurate Analysis)
% To be included in master LaTeX document
%
% Usage: % Flux Jump Module API Documentation (Accurate Analysis)
% To be included in master LaTeX document
%
% Usage: % Flux Jump Module API Documentation (Accurate Analysis)
% To be included in master LaTeX document
%
% Usage: % Flux Jump Module API Documentation (Accurate Analysis)
% To be included in master LaTeX document
%
% Usage: \input{docs/flux_jump_module_api}

\section{Flux Jump Module API Reference (Accurate Analysis)}
\label{sec:flux_jump_module_api}

This section provides an exact reference for the flux jump module (\texttt{ooc1d.core.flux\_jump}) based on detailed analysis of the actual implementation. The module computes local flux balance contributions for HDG methods, serving as the Python equivalent of MATLAB \texttt{fluxJump.m}.

\subsection{Module Overview}

The flux jump module provides:
\begin{itemize}
    \item Local flux balance computation for HDG static condensation
    \item Element-wise bulk solution reconstruction
    \item Flux jump assembly across mesh points
    \item Jacobian computation for Newton solvers
    \item Comprehensive testing infrastructure
\end{itemize}

\subsection{Module Imports and Dependencies}

\begin{lstlisting}[language=Python, caption=Module Dependencies]
import numpy as np
from typing import Tuple
\end{lstlisting}

\subsection{Main Function}
\label{subsec:domain_flux_jump_function}

The primary function for computing local flux balance contributions.

\paragraph{domain\_flux\_jump()}
\begin{lstlisting}[language=Python, caption=Domain Flux Jump Function]
def domain_flux_jump(
    trace_solution: np.ndarray,
    forcing_term: np.ndarray,
    problem, # dummy placeholder for backwards compatibility
    discretization, # dummy placeholder for backwards compatibility
    static_condensation
) -> Tuple[np.ndarray, np.ndarray, np.ndarray]
\end{lstlisting}

\textbf{Purpose:} Python equivalent of MATLAB \texttt{fluxJump.m} function for local flux balance computation

\textbf{Parameters:}
\begin{itemize}
    \item \texttt{trace\_solution}: Trial trace solution vector of length \texttt{neq*(N+1)} where N is number of elements
    \item \texttt{forcing\_term}: \texttt{(2*neq)×N} matrix accounting for previous timestep and forcing terms (stacked for all equations)
    \item \texttt{problem}: Problem instance containing boundary conditions (placeholder for backwards compatibility)
    \item \texttt{discretization}: Discretization parameters (placeholder for backwards compatibility)
    \item \texttt{static\_condensation}: Static condensation implementation with \texttt{static\_condensation()} method
\end{itemize}

\textbf{Returns:} \texttt{Tuple[np.ndarray, np.ndarray, np.ndarray]} - (U, F, JF) where:
\begin{itemize}
    \item \texttt{U}: \texttt{(2*neq)×N} matrix of bulk solutions for each element
    \item \texttt{F}: \texttt{neq*(N+1)} vector of flux jumps at mesh points
    \item \texttt{JF}: \texttt{neq*(N+1)×neq*(N+1)} Jacobian matrix
\end{itemize}

\textbf{Algorithm Description:}
\begin{enumerate}
    \item \textbf{Input Deduction}: Extract N (elements) and neq (equations) from \texttt{forcing\_term.shape}
    \item \textbf{Initialization}: Create output arrays U, F, JF with appropriate dimensions
    \item \textbf{Element Loop}: For each element k = 0, ..., N-1:
    \begin{itemize}
        \item Extract local indices for element nodes
        \item Extract local forcing term and trace values
        \item Apply static condensation to get local solution and flux contributions
        \item Update global bulk solution matrix U
        \item Accumulate flux jump contributions in F
        \item Accumulate Jacobian contributions in JF
    \end{itemize}
\end{enumerate}

\textbf{Input Shape Requirements:}
\begin{itemize}
    \item \texttt{trace\_solution}: Shape \texttt{(neq*(N+1), 1)} or \texttt{(neq*(N+1),)}
    \item \texttt{forcing\_term}: Shape \texttt{(2*neq, N)}
    \item Static condensation must provide method: \texttt{static\_condensation(local\_trace, local\_forcing)}
\end{itemize}

\textbf{Output Shape Guarantees:}
\begin{itemize}
    \item \texttt{U}: Shape \texttt{(2*neq, N)} - bulk solutions for all elements
    \item \texttt{F}: Shape \texttt{(neq*(N+1), 1)} - flux jumps at all nodes
    \item \texttt{JF}: Shape \texttt{(neq*(N+1), neq*(N+1))} - full Jacobian matrix
\end{itemize}

\textbf{Usage Examples:}
\begin{lstlisting}[language=Python, caption=Domain Flux Jump Usage Examples]
import numpy as np
from ooc1d.core.flux_jump import domain_flux_jump

# Example 1: Single equation (neq=1), 4 elements
N = 4
neq = 1
n_nodes = N + 1  # 5 nodes

# Create input data
trace_solution = np.random.rand(neq * n_nodes, 1)  # Shape: (5, 1)
forcing_term = np.random.rand(2 * neq, N)          # Shape: (2, 4)

# Assume static_condensation is available
U, F, JF = domain_flux_jump(
    trace_solution=trace_solution,
    forcing_term=forcing_term,
    problem=None,  # Placeholder
    discretization=None,  # Placeholder
    static_condensation=static_condensation_instance
)

print(f"Bulk solutions U: {U.shape}")    # (2, 4)
print(f"Flux jumps F: {F.shape}")       # (5, 1)
print(f"Jacobian JF: {JF.shape}")       # (5, 5)

# Example 2: Two equations (neq=2), 3 elements
N = 3
neq = 2
n_nodes = N + 1  # 4 nodes

trace_solution = np.random.rand(neq * n_nodes, 1)  # Shape: (8, 1)
forcing_term = np.random.rand(2 * neq, N)          # Shape: (4, 3)

U, F, JF = domain_flux_jump(
    trace_solution, forcing_term, None, None, static_condensation_instance
)

print(f"Bulk solutions U: {U.shape}")    # (4, 3)
print(f"Flux jumps F: {F.shape}")       # (8, 1)  
print(f"Jacobian JF: {JF.shape}")       # (8, 8)
\end{lstlisting}

\textbf{Local Index Construction:} For element k, the function constructs local indices as:
\begin{lstlisting}[language=Python, caption=Local Index Construction Algorithm]
# Initialize local_indices as vector of length 2*neq
local_indices = np.zeros(2 * neq, dtype=int)

# Fill indices for each equation
for ieq in range(neq):
    # For equation ieq: indices k and k+1 in the corresponding block
    local_indices[ieq * 2] = ieq * n_nodes + k      # Left node for equation ieq
    local_indices[ieq * 2 + 1] = ieq * n_nodes + (k + 1)  # Right node for equation ieq

# Example for neq=2, k=1, n_nodes=5:
# local_indices = [1, 2, 6, 7]  # [u_left, u_right, phi_left, phi_right]
\end{lstlisting}

\textbf{Static Condensation Interface:} The function expects the static condensation object to provide:
\begin{lstlisting}[language=Python, caption=Required Static Condensation Interface]
def static_condensation(self, local_trace, local_source):
    """
    Required interface for static condensation objects.
    
    Args:
        local_trace: Local trace values, shape (2*neq, 1) or (2*neq,)
        local_source: Local forcing term, shape (2*neq, 1) or (2*neq,)
        
    Returns:
        Tuple[np.ndarray, np.ndarray, np.ndarray, np.ndarray]:
            - local_solution: Bulk coefficients, shape (2*neq,)
            - flux: Scalar flux value
            - flux_trace: Flux contributions, shape (2*neq,)
            - jacobian: Local Jacobian, shape (2*neq, 2*neq)
    """
    pass
\end{lstlisting}

\textbf{Error Handling:}
\begin{itemize}
    \item \texttt{RuntimeError}: Raised if static condensation fails for any element
    \item Includes element number in error message for debugging
\end{itemize}

\textbf{Integration with BioNetFlux:}
\begin{lstlisting}[language=Python, caption=BioNetFlux Integration Example]
# Integration with GlobalAssembler
class GlobalAssembler:
    def compute_element_contributions(self, trace_solutions, forcing_terms):
        """Example integration in GlobalAssembler."""
        all_bulk_solutions = []
        all_flux_jumps = []
        all_jacobians = []
        
        for domain_idx in range(self.n_domains):
            # Get domain-specific data
            trace_sol = trace_solutions[domain_idx]
            forcing = forcing_terms[domain_idx]
            static_cond = self.static_condensations[domain_idx]
            
            # Compute domain flux jump
            U, F, JF = domain_flux_jump(
                trace_solution=trace_sol,
                forcing_term=forcing,
                problem=self.problems[domain_idx],
                discretization=self.discretizations[domain_idx],
                static_condensation=static_cond
            )
            
            all_bulk_solutions.append(U)
            all_flux_jumps.append(F)
            all_jacobians.append(JF)
        
        return all_bulk_solutions, all_flux_jumps, all_jacobians
\end{lstlisting}

\subsection{Testing Infrastructure}
\label{subsec:flux_jump_testing}

The module provides comprehensive testing infrastructure for validation.

\paragraph{test\_domain\_flux\_jump()}
\begin{lstlisting}[language=Python, caption=Test Function]
def test_domain_flux_jump(verbose=True) -> bool
\end{lstlisting}

\textbf{Parameters:}
\begin{itemize}
    \item \texttt{verbose}: If True, print detailed test information (default: True)
\end{itemize}

\textbf{Returns:} \texttt{bool} - True if all tests pass, False otherwise

\textbf{Test Suite Components:}

\textbf{1. Mock Object Classes:}
\begin{lstlisting}[language=Python, caption=Mock Object Definitions]
class MockProblem:
    def __init__(self, neq=1):
        self.neq = neq
        self.domain_start = 0.0
        self.domain_end = 1.0

class MockDiscretization:
    def __init__(self, n_elements=4):
        self.n_elements = n_elements
        self.nodes = np.linspace(0, 1, n_elements + 1)
        self.element_length = 1.0 / n_elements

class MockStaticCondensation:
    def __init__(self, neq=1):
        self.neq = neq
        
    def static_condensation(self, local_trace, local_source=None):
        """Mock static condensation with predictable results."""
        # Returns deterministic outputs for testing
        pass
\end{lstlisting}

\textbf{2. Test Cases:}
\begin{itemize}
    \item \textbf{Case 1}: Single equation, 3 elements
    \item \textbf{Case 2}: Two equations, 4 elements  
    \item \textbf{Case 3}: Single equation, 5 elements
\end{itemize}

\textbf{3. Validation Tests:}
\begin{enumerate}
    \item \textbf{Shape Validation}: Verify output array shapes match expected dimensions
    \item \textbf{Finite Value Check}: Ensure no NaN or infinite values in outputs
    \item \textbf{Jacobian Properties}: Check Jacobian matrix properties
    \item \textbf{Edge Case Testing}: Test with zero trace and zero forcing
\end{enumerate}

\textbf{Test Output Format:}
\begin{lstlisting}[language=Python, caption=Sample Test Output]
============================================================
TESTING domain_flux_jump FUNCTION
============================================================

Test Case 1: Single equation, 3 elements
----------------------------------------
  Input shapes:
    trace_solution: (4, 1)
    forcing_term: (2, 3)
    ✓ U shape: (2, 3)
    ✓ F shape: (4, 1)
    ✓ JF shape: (4, 4)
    ✓ U contains only finite values
    ✓ F contains only finite values
    ✓ JF contains only finite values
    ✓ JF has non-zero entries
    ✓ Test case passed
  Results summary:
    |U|_max = 1.234567e+00
    |F|_norm = 2.345678e+00
    JF condition = 1.23e+01

Edge Case Tests:
----------------------------------------
  ✓ Zero trace test passed: |F| = 1.234567e-02
  ✓ Zero forcing test passed: |F| = 2.345678e-01

============================================================
✅ ALL TESTS PASSED
============================================================
\end{lstlisting}

\textbf{Usage:}
\begin{lstlisting}[language=Python, caption=Test Function Usage]
# Run tests with detailed output
success = test_domain_flux_jump(verbose=True)
if success:
    print("All tests passed")
else:
    print("Some tests failed")

# Run tests silently
success = test_domain_flux_jump(verbose=False)

# Command line execution
if __name__ == "__main__":
    success = test_domain_flux_jump(verbose=True)
    exit(0 if success else 1)
\end{lstlisting}

\textbf{Mock Static Condensation Algorithm:}
\begin{lstlisting}[language=Python, caption=Mock Static Condensation Details]
def static_condensation(self, local_trace, local_source=None):
    """Mock implementation with predictable behavior."""
    trace_length = len(local_trace.flatten())
    neq = trace_length // 2
    
    if local_source is None:
        local_source = np.zeros(2 * neq)
    
    local_trace_flat = local_trace.flatten()
    local_source_flat = local_source.flatten()
    
    # Mock local solution: 2*neq coefficients
    coeffs_per_element = 2 * neq
    local_solution = np.zeros(coeffs_per_element)
    
    # Fill with simple pattern for testing
    for i in range(min(coeffs_per_element, len(local_trace_flat))):
        local_solution[i] = 0.8 * local_trace_flat[i] + 0.1 * (i + 1)
    
    # Mock flux: scalar
    flux = np.sum(local_trace_flat) * 0.1
    
    # Mock flux_trace: same length as local_trace
    flux_trace = local_trace_flat * 0.9 + local_source_flat * 0.1
    
    # Mock jacobian
    jacobian = np.eye(len(local_trace_flat)) * 1.1 + 0.1
    
    return local_solution, flux, flux_trace, jacobian
\end{lstlisting}

\subsection{Complete Usage Examples}
\label{subsec:flux_jump_complete_examples}

\subsubsection{Integration with Real Static Condensation}

\begin{lstlisting}[language=Python, caption=Real Static Condensation Integration]
from ooc1d.core.flux_jump import domain_flux_jump
from ooc1d.core.static_condensation_ooc import StaticCondensationOOC
from ooc1d.core.problem import Problem
from ooc1d.core.discretization import Discretization, GlobalDiscretization
from ooc1d.utils.elementary_matrices import ElementaryMatrices
import numpy as np

# Setup OrganOnChip problem (4 equations)
ooc_params = np.array([1.0, 2.0, 1.0, 1.0, 0.0, 1.0, 0.0, 1.0, 1.0])
problem = Problem(
    neq=4,
    domain_start=0.0,
    domain_length=1.0,
    parameters=ooc_params,
    problem_type="organ_on_chip"
)

# Setup discretization
discretization = Discretization(n_elements=10)
global_disc = GlobalDiscretization([discretization])

# Create static condensation
elementary_matrices = ElementaryMatrices()
static_condensation = StaticCondensation(
    problem=problem,
    global_discretization=global_disc,
    elementary_matrices=elementary_matrices,
    domain_index=0
)

# Prepare realistic input data
N = 10  # elements
neq = 4  # equations
n_nodes = N + 1  # 11 nodes

# Create trace solution (4 equations × 11 nodes = 44 values)
trace_solution = np.random.rand(neq * n_nodes, 1)

# Create forcing term (8 coefficients × 10 elements)
forcing_term = np.random.rand(2 * neq, N)

# Compute flux jump
U, F, JF = domain_flux_jump(
    trace_solution=trace_solution,
    forcing_term=forcing_term,
    problem=problem,
    discretization=discretization,
    static_condensation=static_condensation
)

print(f"OrganOnChip flux jump computation:")
print(f"  Input: {neq} equations, {N} elements, {n_nodes} nodes")
print(f"  Output shapes:")
print(f"    Bulk solutions U: {U.shape}")     # (8, 10)
print(f"    Flux jumps F: {F.shape}")        # (44, 1)
print(f"    Jacobian JF: {JF.shape}")        # (44, 44)
print(f"  Solution statistics:")
print(f"    |U|_max = {np.max(np.abs(U)):.6e}")
print(f"    |F|_norm = {np.linalg.norm(F):.6e}")
print(f"    JF condition = {np.linalg.cond(JF):.2e}")
\end{lstlisting}



\subsubsection{Error Handling and Debugging}

\begin{lstlisting}[language=Python, caption=Error Handling Example]
def robust_flux_jump_computation(trace_solution, forcing_term, 
                                problem, discretization, static_condensation):
    """
    Robust wrapper for domain_flux_jump with comprehensive error handling.
    """
    try:
        # Validate inputs
        if trace_solution.size == 0:
            raise ValueError("Empty trace solution")
        
        if forcing_term.size == 0:
            raise ValueError("Empty forcing term")
        
        # Check for NaN or infinite values in inputs
        if np.any(np.isnan(trace_solution)) or np.any(np.isinf(trace_solution)):
            raise ValueError("Trace solution contains NaN or infinite values")
        
        if np.any(np.isnan(forcing_term)) or np.any(np.isinf(forcing_term)):
            raise ValueError("Forcing term contains NaN or infinite values")
        
        # Compute flux jump
        U, F, JF = domain_flux_jump(
            trace_solution, forcing_term, problem, discretization, static_condensation
        )
        
        # Validate outputs
        if np.any(np.isnan(U)) or np.any(np.isinf(U)):
            raise RuntimeError("Bulk solution contains NaN or infinite values")
        
        if np.any(np.isnan(F)) or np.any(np.isinf(F)):
            raise RuntimeError("Flux jump contains NaN or infinite values")
        
        if np.any(np.isnan(JF)) or np.any(np.isinf(JF)):
            raise RuntimeError("Jacobian contains NaN or infinite values")
        
        # Check Jacobian conditioning
        jf_condition = np.linalg.cond(JF)
        if jf_condition > 1e12:
            print(f"Warning: Jacobian is poorly conditioned (cond = {jf_condition:.2e})")
        
        return U, F, JF, {"status": "success", "condition": jf_condition}
        
    except Exception as e:
        error_info = {
            "status": "error",
            "error_type": type(e).__name__,
            "error_message": str(e),
            "trace_shape": trace_solution.shape if hasattr(trace_solution, 'shape') else None,
            "forcing_shape": forcing_term.shape if hasattr(forcing_term, 'shape') else None
        }
        
        print(f"Flux jump computation failed: {e}")
        print(f"Error details: {error_info}")
        
        # Return zeros with error info
        neq = forcing_term.shape[0] // 2 if hasattr(forcing_term, 'shape') and forcing_term.size > 0 else 1
        N = forcing_term.shape[1] if hasattr(forcing_term, 'shape') and len(forcing_term.shape) > 1 else 1
        n_nodes = N + 1
        
        U_zero = np.zeros((2 * neq, N))
        F_zero = np.zeros((neq * n_nodes, 1))
        JF_zero = np.eye(neq * n_nodes)  # Identity to avoid singular matrix
        
        return U_zero, F_zero, JF_zero, error_info

# Usage with error handling
U, F, JF, status = robust_flux_jump_computation(
    trace_solution, forcing_term, problem, discretization, static_condensation
)

if status["status"] == "success":
    print(f"✓ Computation successful (condition = {status['condition']:.2e})")
else:
    print(f"✗ Computation failed: {status['error_message']}")
\end{lstlisting}

\subsection{Method and Function Summary}
\label{subsec:flux_jump_summary}

\begin{longtable}{|p{4.3cm}|p{3.5cm}|p{6cm}|}
\hline
\textbf{Function} & \textbf{Returns} & \textbf{Purpose} \\
\hline
\endhead

\texttt{domain\_flux\_jump} & \texttt{Tuple[3×np.ndarray]} & Compute local flux balance for HDG method \\
\hline

\texttt{test\_domain\_flux\_jump} & \texttt{bool} & Comprehensive testing with mock objects \\
\hline

\end{longtable}

\subsection{Key Features and Capabilities}

\begin{itemize}
    \item \textbf{MATLAB Compatibility}: Direct Python equivalent of MATLAB \texttt{fluxJump.m}
    \item \textbf{Multi-Equation Support}: Handles arbitrary number of equations (\texttt{neq})
    \item \textbf{Flexible Element Count}: Works with any number of elements (N)
    \item \textbf{Static Condensation Integration}: Compatible with all static condensation implementations
    \item \textbf{Comprehensive Testing}: Built-in test suite with mock objects and edge cases
    \item \textbf{Error Handling}: Robust error reporting with element-specific debugging
    \item \textbf{Shape Validation}: Automatic input/output shape verification
    \item \textbf{Jacobian Assembly}: Full Jacobian matrix construction for Newton solvers
\end{itemize}

This documentation provides an exact reference for the flux jump module, emphasizing its role in HDG static condensation and integration with the broader BioNetFlux framework.

% End of flux jump module API documentation


\section{Flux Jump Module API Reference (Accurate Analysis)}
\label{sec:flux_jump_module_api}

This section provides an exact reference for the flux jump module (\texttt{ooc1d.core.flux\_jump}) based on detailed analysis of the actual implementation. The module computes local flux balance contributions for HDG methods, serving as the Python equivalent of MATLAB \texttt{fluxJump.m}.

\subsection{Module Overview}

The flux jump module provides:
\begin{itemize}
    \item Local flux balance computation for HDG static condensation
    \item Element-wise bulk solution reconstruction
    \item Flux jump assembly across mesh points
    \item Jacobian computation for Newton solvers
    \item Comprehensive testing infrastructure
\end{itemize}

\subsection{Module Imports and Dependencies}

\begin{lstlisting}[language=Python, caption=Module Dependencies]
import numpy as np
from typing import Tuple
\end{lstlisting}

\subsection{Main Function}
\label{subsec:domain_flux_jump_function}

The primary function for computing local flux balance contributions.

\paragraph{domain\_flux\_jump()}
\begin{lstlisting}[language=Python, caption=Domain Flux Jump Function]
def domain_flux_jump(
    trace_solution: np.ndarray,
    forcing_term: np.ndarray,
    problem, # dummy placeholder for backwards compatibility
    discretization, # dummy placeholder for backwards compatibility
    static_condensation
) -> Tuple[np.ndarray, np.ndarray, np.ndarray]
\end{lstlisting}

\textbf{Purpose:} Python equivalent of MATLAB \texttt{fluxJump.m} function for local flux balance computation

\textbf{Parameters:}
\begin{itemize}
    \item \texttt{trace\_solution}: Trial trace solution vector of length \texttt{neq*(N+1)} where N is number of elements
    \item \texttt{forcing\_term}: \texttt{(2*neq)×N} matrix accounting for previous timestep and forcing terms (stacked for all equations)
    \item \texttt{problem}: Problem instance containing boundary conditions (placeholder for backwards compatibility)
    \item \texttt{discretization}: Discretization parameters (placeholder for backwards compatibility)
    \item \texttt{static\_condensation}: Static condensation implementation with \texttt{static\_condensation()} method
\end{itemize}

\textbf{Returns:} \texttt{Tuple[np.ndarray, np.ndarray, np.ndarray]} - (U, F, JF) where:
\begin{itemize}
    \item \texttt{U}: \texttt{(2*neq)×N} matrix of bulk solutions for each element
    \item \texttt{F}: \texttt{neq*(N+1)} vector of flux jumps at mesh points
    \item \texttt{JF}: \texttt{neq*(N+1)×neq*(N+1)} Jacobian matrix
\end{itemize}

\textbf{Algorithm Description:}
\begin{enumerate}
    \item \textbf{Input Deduction}: Extract N (elements) and neq (equations) from \texttt{forcing\_term.shape}
    \item \textbf{Initialization}: Create output arrays U, F, JF with appropriate dimensions
    \item \textbf{Element Loop}: For each element k = 0, ..., N-1:
    \begin{itemize}
        \item Extract local indices for element nodes
        \item Extract local forcing term and trace values
        \item Apply static condensation to get local solution and flux contributions
        \item Update global bulk solution matrix U
        \item Accumulate flux jump contributions in F
        \item Accumulate Jacobian contributions in JF
    \end{itemize}
\end{enumerate}

\textbf{Input Shape Requirements:}
\begin{itemize}
    \item \texttt{trace\_solution}: Shape \texttt{(neq*(N+1), 1)} or \texttt{(neq*(N+1),)}
    \item \texttt{forcing\_term}: Shape \texttt{(2*neq, N)}
    \item Static condensation must provide method: \texttt{static\_condensation(local\_trace, local\_forcing)}
\end{itemize}

\textbf{Output Shape Guarantees:}
\begin{itemize}
    \item \texttt{U}: Shape \texttt{(2*neq, N)} - bulk solutions for all elements
    \item \texttt{F}: Shape \texttt{(neq*(N+1), 1)} - flux jumps at all nodes
    \item \texttt{JF}: Shape \texttt{(neq*(N+1), neq*(N+1))} - full Jacobian matrix
\end{itemize}

\textbf{Usage Examples:}
\begin{lstlisting}[language=Python, caption=Domain Flux Jump Usage Examples]
import numpy as np
from ooc1d.core.flux_jump import domain_flux_jump

# Example 1: Single equation (neq=1), 4 elements
N = 4
neq = 1
n_nodes = N + 1  # 5 nodes

# Create input data
trace_solution = np.random.rand(neq * n_nodes, 1)  # Shape: (5, 1)
forcing_term = np.random.rand(2 * neq, N)          # Shape: (2, 4)

# Assume static_condensation is available
U, F, JF = domain_flux_jump(
    trace_solution=trace_solution,
    forcing_term=forcing_term,
    problem=None,  # Placeholder
    discretization=None,  # Placeholder
    static_condensation=static_condensation_instance
)

print(f"Bulk solutions U: {U.shape}")    # (2, 4)
print(f"Flux jumps F: {F.shape}")       # (5, 1)
print(f"Jacobian JF: {JF.shape}")       # (5, 5)

# Example 2: Two equations (neq=2), 3 elements
N = 3
neq = 2
n_nodes = N + 1  # 4 nodes

trace_solution = np.random.rand(neq * n_nodes, 1)  # Shape: (8, 1)
forcing_term = np.random.rand(2 * neq, N)          # Shape: (4, 3)

U, F, JF = domain_flux_jump(
    trace_solution, forcing_term, None, None, static_condensation_instance
)

print(f"Bulk solutions U: {U.shape}")    # (4, 3)
print(f"Flux jumps F: {F.shape}")       # (8, 1)  
print(f"Jacobian JF: {JF.shape}")       # (8, 8)
\end{lstlisting}

\textbf{Local Index Construction:} For element k, the function constructs local indices as:
\begin{lstlisting}[language=Python, caption=Local Index Construction Algorithm]
# Initialize local_indices as vector of length 2*neq
local_indices = np.zeros(2 * neq, dtype=int)

# Fill indices for each equation
for ieq in range(neq):
    # For equation ieq: indices k and k+1 in the corresponding block
    local_indices[ieq * 2] = ieq * n_nodes + k      # Left node for equation ieq
    local_indices[ieq * 2 + 1] = ieq * n_nodes + (k + 1)  # Right node for equation ieq

# Example for neq=2, k=1, n_nodes=5:
# local_indices = [1, 2, 6, 7]  # [u_left, u_right, phi_left, phi_right]
\end{lstlisting}

\textbf{Static Condensation Interface:} The function expects the static condensation object to provide:
\begin{lstlisting}[language=Python, caption=Required Static Condensation Interface]
def static_condensation(self, local_trace, local_source):
    """
    Required interface for static condensation objects.
    
    Args:
        local_trace: Local trace values, shape (2*neq, 1) or (2*neq,)
        local_source: Local forcing term, shape (2*neq, 1) or (2*neq,)
        
    Returns:
        Tuple[np.ndarray, np.ndarray, np.ndarray, np.ndarray]:
            - local_solution: Bulk coefficients, shape (2*neq,)
            - flux: Scalar flux value
            - flux_trace: Flux contributions, shape (2*neq,)
            - jacobian: Local Jacobian, shape (2*neq, 2*neq)
    """
    pass
\end{lstlisting}

\textbf{Error Handling:}
\begin{itemize}
    \item \texttt{RuntimeError}: Raised if static condensation fails for any element
    \item Includes element number in error message for debugging
\end{itemize}

\textbf{Integration with BioNetFlux:}
\begin{lstlisting}[language=Python, caption=BioNetFlux Integration Example]
# Integration with GlobalAssembler
class GlobalAssembler:
    def compute_element_contributions(self, trace_solutions, forcing_terms):
        """Example integration in GlobalAssembler."""
        all_bulk_solutions = []
        all_flux_jumps = []
        all_jacobians = []
        
        for domain_idx in range(self.n_domains):
            # Get domain-specific data
            trace_sol = trace_solutions[domain_idx]
            forcing = forcing_terms[domain_idx]
            static_cond = self.static_condensations[domain_idx]
            
            # Compute domain flux jump
            U, F, JF = domain_flux_jump(
                trace_solution=trace_sol,
                forcing_term=forcing,
                problem=self.problems[domain_idx],
                discretization=self.discretizations[domain_idx],
                static_condensation=static_cond
            )
            
            all_bulk_solutions.append(U)
            all_flux_jumps.append(F)
            all_jacobians.append(JF)
        
        return all_bulk_solutions, all_flux_jumps, all_jacobians
\end{lstlisting}

\subsection{Testing Infrastructure}
\label{subsec:flux_jump_testing}

The module provides comprehensive testing infrastructure for validation.

\paragraph{test\_domain\_flux\_jump()}
\begin{lstlisting}[language=Python, caption=Test Function]
def test_domain_flux_jump(verbose=True) -> bool
\end{lstlisting}

\textbf{Parameters:}
\begin{itemize}
    \item \texttt{verbose}: If True, print detailed test information (default: True)
\end{itemize}

\textbf{Returns:} \texttt{bool} - True if all tests pass, False otherwise

\textbf{Test Suite Components:}

\textbf{1. Mock Object Classes:}
\begin{lstlisting}[language=Python, caption=Mock Object Definitions]
class MockProblem:
    def __init__(self, neq=1):
        self.neq = neq
        self.domain_start = 0.0
        self.domain_end = 1.0

class MockDiscretization:
    def __init__(self, n_elements=4):
        self.n_elements = n_elements
        self.nodes = np.linspace(0, 1, n_elements + 1)
        self.element_length = 1.0 / n_elements

class MockStaticCondensation:
    def __init__(self, neq=1):
        self.neq = neq
        
    def static_condensation(self, local_trace, local_source=None):
        """Mock static condensation with predictable results."""
        # Returns deterministic outputs for testing
        pass
\end{lstlisting}

\textbf{2. Test Cases:}
\begin{itemize}
    \item \textbf{Case 1}: Single equation, 3 elements
    \item \textbf{Case 2}: Two equations, 4 elements  
    \item \textbf{Case 3}: Single equation, 5 elements
\end{itemize}

\textbf{3. Validation Tests:}
\begin{enumerate}
    \item \textbf{Shape Validation}: Verify output array shapes match expected dimensions
    \item \textbf{Finite Value Check}: Ensure no NaN or infinite values in outputs
    \item \textbf{Jacobian Properties}: Check Jacobian matrix properties
    \item \textbf{Edge Case Testing}: Test with zero trace and zero forcing
\end{enumerate}

\textbf{Test Output Format:}
\begin{lstlisting}[language=Python, caption=Sample Test Output]
============================================================
TESTING domain_flux_jump FUNCTION
============================================================

Test Case 1: Single equation, 3 elements
----------------------------------------
  Input shapes:
    trace_solution: (4, 1)
    forcing_term: (2, 3)
    ✓ U shape: (2, 3)
    ✓ F shape: (4, 1)
    ✓ JF shape: (4, 4)
    ✓ U contains only finite values
    ✓ F contains only finite values
    ✓ JF contains only finite values
    ✓ JF has non-zero entries
    ✓ Test case passed
  Results summary:
    |U|_max = 1.234567e+00
    |F|_norm = 2.345678e+00
    JF condition = 1.23e+01

Edge Case Tests:
----------------------------------------
  ✓ Zero trace test passed: |F| = 1.234567e-02
  ✓ Zero forcing test passed: |F| = 2.345678e-01

============================================================
✅ ALL TESTS PASSED
============================================================
\end{lstlisting}

\textbf{Usage:}
\begin{lstlisting}[language=Python, caption=Test Function Usage]
# Run tests with detailed output
success = test_domain_flux_jump(verbose=True)
if success:
    print("All tests passed")
else:
    print("Some tests failed")

# Run tests silently
success = test_domain_flux_jump(verbose=False)

# Command line execution
if __name__ == "__main__":
    success = test_domain_flux_jump(verbose=True)
    exit(0 if success else 1)
\end{lstlisting}

\textbf{Mock Static Condensation Algorithm:}
\begin{lstlisting}[language=Python, caption=Mock Static Condensation Details]
def static_condensation(self, local_trace, local_source=None):
    """Mock implementation with predictable behavior."""
    trace_length = len(local_trace.flatten())
    neq = trace_length // 2
    
    if local_source is None:
        local_source = np.zeros(2 * neq)
    
    local_trace_flat = local_trace.flatten()
    local_source_flat = local_source.flatten()
    
    # Mock local solution: 2*neq coefficients
    coeffs_per_element = 2 * neq
    local_solution = np.zeros(coeffs_per_element)
    
    # Fill with simple pattern for testing
    for i in range(min(coeffs_per_element, len(local_trace_flat))):
        local_solution[i] = 0.8 * local_trace_flat[i] + 0.1 * (i + 1)
    
    # Mock flux: scalar
    flux = np.sum(local_trace_flat) * 0.1
    
    # Mock flux_trace: same length as local_trace
    flux_trace = local_trace_flat * 0.9 + local_source_flat * 0.1
    
    # Mock jacobian
    jacobian = np.eye(len(local_trace_flat)) * 1.1 + 0.1
    
    return local_solution, flux, flux_trace, jacobian
\end{lstlisting}

\subsection{Complete Usage Examples}
\label{subsec:flux_jump_complete_examples}

\subsubsection{Integration with Real Static Condensation}

\begin{lstlisting}[language=Python, caption=Real Static Condensation Integration]
from ooc1d.core.flux_jump import domain_flux_jump
from ooc1d.core.static_condensation_ooc import StaticCondensationOOC
from ooc1d.core.problem import Problem
from ooc1d.core.discretization import Discretization, GlobalDiscretization
from ooc1d.utils.elementary_matrices import ElementaryMatrices
import numpy as np

# Setup OrganOnChip problem (4 equations)
ooc_params = np.array([1.0, 2.0, 1.0, 1.0, 0.0, 1.0, 0.0, 1.0, 1.0])
problem = Problem(
    neq=4,
    domain_start=0.0,
    domain_length=1.0,
    parameters=ooc_params,
    problem_type="organ_on_chip"
)

# Setup discretization
discretization = Discretization(n_elements=10)
global_disc = GlobalDiscretization([discretization])

# Create static condensation
elementary_matrices = ElementaryMatrices()
static_condensation = StaticCondensation(
    problem=problem,
    global_discretization=global_disc,
    elementary_matrices=elementary_matrices,
    domain_index=0
)

# Prepare realistic input data
N = 10  # elements
neq = 4  # equations
n_nodes = N + 1  # 11 nodes

# Create trace solution (4 equations × 11 nodes = 44 values)
trace_solution = np.random.rand(neq * n_nodes, 1)

# Create forcing term (8 coefficients × 10 elements)
forcing_term = np.random.rand(2 * neq, N)

# Compute flux jump
U, F, JF = domain_flux_jump(
    trace_solution=trace_solution,
    forcing_term=forcing_term,
    problem=problem,
    discretization=discretization,
    static_condensation=static_condensation
)

print(f"OrganOnChip flux jump computation:")
print(f"  Input: {neq} equations, {N} elements, {n_nodes} nodes")
print(f"  Output shapes:")
print(f"    Bulk solutions U: {U.shape}")     # (8, 10)
print(f"    Flux jumps F: {F.shape}")        # (44, 1)
print(f"    Jacobian JF: {JF.shape}")        # (44, 44)
print(f"  Solution statistics:")
print(f"    |U|_max = {np.max(np.abs(U)):.6e}")
print(f"    |F|_norm = {np.linalg.norm(F):.6e}")
print(f"    JF condition = {np.linalg.cond(JF):.2e}")
\end{lstlisting}



\subsubsection{Error Handling and Debugging}

\begin{lstlisting}[language=Python, caption=Error Handling Example]
def robust_flux_jump_computation(trace_solution, forcing_term, 
                                problem, discretization, static_condensation):
    """
    Robust wrapper for domain_flux_jump with comprehensive error handling.
    """
    try:
        # Validate inputs
        if trace_solution.size == 0:
            raise ValueError("Empty trace solution")
        
        if forcing_term.size == 0:
            raise ValueError("Empty forcing term")
        
        # Check for NaN or infinite values in inputs
        if np.any(np.isnan(trace_solution)) or np.any(np.isinf(trace_solution)):
            raise ValueError("Trace solution contains NaN or infinite values")
        
        if np.any(np.isnan(forcing_term)) or np.any(np.isinf(forcing_term)):
            raise ValueError("Forcing term contains NaN or infinite values")
        
        # Compute flux jump
        U, F, JF = domain_flux_jump(
            trace_solution, forcing_term, problem, discretization, static_condensation
        )
        
        # Validate outputs
        if np.any(np.isnan(U)) or np.any(np.isinf(U)):
            raise RuntimeError("Bulk solution contains NaN or infinite values")
        
        if np.any(np.isnan(F)) or np.any(np.isinf(F)):
            raise RuntimeError("Flux jump contains NaN or infinite values")
        
        if np.any(np.isnan(JF)) or np.any(np.isinf(JF)):
            raise RuntimeError("Jacobian contains NaN or infinite values")
        
        # Check Jacobian conditioning
        jf_condition = np.linalg.cond(JF)
        if jf_condition > 1e12:
            print(f"Warning: Jacobian is poorly conditioned (cond = {jf_condition:.2e})")
        
        return U, F, JF, {"status": "success", "condition": jf_condition}
        
    except Exception as e:
        error_info = {
            "status": "error",
            "error_type": type(e).__name__,
            "error_message": str(e),
            "trace_shape": trace_solution.shape if hasattr(trace_solution, 'shape') else None,
            "forcing_shape": forcing_term.shape if hasattr(forcing_term, 'shape') else None
        }
        
        print(f"Flux jump computation failed: {e}")
        print(f"Error details: {error_info}")
        
        # Return zeros with error info
        neq = forcing_term.shape[0] // 2 if hasattr(forcing_term, 'shape') and forcing_term.size > 0 else 1
        N = forcing_term.shape[1] if hasattr(forcing_term, 'shape') and len(forcing_term.shape) > 1 else 1
        n_nodes = N + 1
        
        U_zero = np.zeros((2 * neq, N))
        F_zero = np.zeros((neq * n_nodes, 1))
        JF_zero = np.eye(neq * n_nodes)  # Identity to avoid singular matrix
        
        return U_zero, F_zero, JF_zero, error_info

# Usage with error handling
U, F, JF, status = robust_flux_jump_computation(
    trace_solution, forcing_term, problem, discretization, static_condensation
)

if status["status"] == "success":
    print(f"✓ Computation successful (condition = {status['condition']:.2e})")
else:
    print(f"✗ Computation failed: {status['error_message']}")
\end{lstlisting}

\subsection{Method and Function Summary}
\label{subsec:flux_jump_summary}

\begin{longtable}{|p{4.3cm}|p{3.5cm}|p{6cm}|}
\hline
\textbf{Function} & \textbf{Returns} & \textbf{Purpose} \\
\hline
\endhead

\texttt{domain\_flux\_jump} & \texttt{Tuple[3×np.ndarray]} & Compute local flux balance for HDG method \\
\hline

\texttt{test\_domain\_flux\_jump} & \texttt{bool} & Comprehensive testing with mock objects \\
\hline

\end{longtable}

\subsection{Key Features and Capabilities}

\begin{itemize}
    \item \textbf{MATLAB Compatibility}: Direct Python equivalent of MATLAB \texttt{fluxJump.m}
    \item \textbf{Multi-Equation Support}: Handles arbitrary number of equations (\texttt{neq})
    \item \textbf{Flexible Element Count}: Works with any number of elements (N)
    \item \textbf{Static Condensation Integration}: Compatible with all static condensation implementations
    \item \textbf{Comprehensive Testing}: Built-in test suite with mock objects and edge cases
    \item \textbf{Error Handling}: Robust error reporting with element-specific debugging
    \item \textbf{Shape Validation}: Automatic input/output shape verification
    \item \textbf{Jacobian Assembly}: Full Jacobian matrix construction for Newton solvers
\end{itemize}

This documentation provides an exact reference for the flux jump module, emphasizing its role in HDG static condensation and integration with the broader BioNetFlux framework.

% End of flux jump module API documentation


\section{Flux Jump Module API Reference (Accurate Analysis)}
\label{sec:flux_jump_module_api}

This section provides an exact reference for the flux jump module (\texttt{ooc1d.core.flux\_jump}) based on detailed analysis of the actual implementation. The module computes local flux balance contributions for HDG methods, serving as the Python equivalent of MATLAB \texttt{fluxJump.m}.

\subsection{Module Overview}

The flux jump module provides:
\begin{itemize}
    \item Local flux balance computation for HDG static condensation
    \item Element-wise bulk solution reconstruction
    \item Flux jump assembly across mesh points
    \item Jacobian computation for Newton solvers
    \item Comprehensive testing infrastructure
\end{itemize}

\subsection{Module Imports and Dependencies}

\begin{lstlisting}[language=Python, caption=Module Dependencies]
import numpy as np
from typing import Tuple
\end{lstlisting}

\subsection{Main Function}
\label{subsec:domain_flux_jump_function}

The primary function for computing local flux balance contributions.

\paragraph{domain\_flux\_jump()}
\begin{lstlisting}[language=Python, caption=Domain Flux Jump Function]
def domain_flux_jump(
    trace_solution: np.ndarray,
    forcing_term: np.ndarray,
    problem, # dummy placeholder for backwards compatibility
    discretization, # dummy placeholder for backwards compatibility
    static_condensation
) -> Tuple[np.ndarray, np.ndarray, np.ndarray]
\end{lstlisting}

\textbf{Purpose:} Python equivalent of MATLAB \texttt{fluxJump.m} function for local flux balance computation

\textbf{Parameters:}
\begin{itemize}
    \item \texttt{trace\_solution}: Trial trace solution vector of length \texttt{neq*(N+1)} where N is number of elements
    \item \texttt{forcing\_term}: \texttt{(2*neq)×N} matrix accounting for previous timestep and forcing terms (stacked for all equations)
    \item \texttt{problem}: Problem instance containing boundary conditions (placeholder for backwards compatibility)
    \item \texttt{discretization}: Discretization parameters (placeholder for backwards compatibility)
    \item \texttt{static\_condensation}: Static condensation implementation with \texttt{static\_condensation()} method
\end{itemize}

\textbf{Returns:} \texttt{Tuple[np.ndarray, np.ndarray, np.ndarray]} - (U, F, JF) where:
\begin{itemize}
    \item \texttt{U}: \texttt{(2*neq)×N} matrix of bulk solutions for each element
    \item \texttt{F}: \texttt{neq*(N+1)} vector of flux jumps at mesh points
    \item \texttt{JF}: \texttt{neq*(N+1)×neq*(N+1)} Jacobian matrix
\end{itemize}

\textbf{Algorithm Description:}
\begin{enumerate}
    \item \textbf{Input Deduction}: Extract N (elements) and neq (equations) from \texttt{forcing\_term.shape}
    \item \textbf{Initialization}: Create output arrays U, F, JF with appropriate dimensions
    \item \textbf{Element Loop}: For each element k = 0, ..., N-1:
    \begin{itemize}
        \item Extract local indices for element nodes
        \item Extract local forcing term and trace values
        \item Apply static condensation to get local solution and flux contributions
        \item Update global bulk solution matrix U
        \item Accumulate flux jump contributions in F
        \item Accumulate Jacobian contributions in JF
    \end{itemize}
\end{enumerate}

\textbf{Input Shape Requirements:}
\begin{itemize}
    \item \texttt{trace\_solution}: Shape \texttt{(neq*(N+1), 1)} or \texttt{(neq*(N+1),)}
    \item \texttt{forcing\_term}: Shape \texttt{(2*neq, N)}
    \item Static condensation must provide method: \texttt{static\_condensation(local\_trace, local\_forcing)}
\end{itemize}

\textbf{Output Shape Guarantees:}
\begin{itemize}
    \item \texttt{U}: Shape \texttt{(2*neq, N)} - bulk solutions for all elements
    \item \texttt{F}: Shape \texttt{(neq*(N+1), 1)} - flux jumps at all nodes
    \item \texttt{JF}: Shape \texttt{(neq*(N+1), neq*(N+1))} - full Jacobian matrix
\end{itemize}

\textbf{Usage Examples:}
\begin{lstlisting}[language=Python, caption=Domain Flux Jump Usage Examples]
import numpy as np
from ooc1d.core.flux_jump import domain_flux_jump

# Example 1: Single equation (neq=1), 4 elements
N = 4
neq = 1
n_nodes = N + 1  # 5 nodes

# Create input data
trace_solution = np.random.rand(neq * n_nodes, 1)  # Shape: (5, 1)
forcing_term = np.random.rand(2 * neq, N)          # Shape: (2, 4)

# Assume static_condensation is available
U, F, JF = domain_flux_jump(
    trace_solution=trace_solution,
    forcing_term=forcing_term,
    problem=None,  # Placeholder
    discretization=None,  # Placeholder
    static_condensation=static_condensation_instance
)

print(f"Bulk solutions U: {U.shape}")    # (2, 4)
print(f"Flux jumps F: {F.shape}")       # (5, 1)
print(f"Jacobian JF: {JF.shape}")       # (5, 5)

# Example 2: Two equations (neq=2), 3 elements
N = 3
neq = 2
n_nodes = N + 1  # 4 nodes

trace_solution = np.random.rand(neq * n_nodes, 1)  # Shape: (8, 1)
forcing_term = np.random.rand(2 * neq, N)          # Shape: (4, 3)

U, F, JF = domain_flux_jump(
    trace_solution, forcing_term, None, None, static_condensation_instance
)

print(f"Bulk solutions U: {U.shape}")    # (4, 3)
print(f"Flux jumps F: {F.shape}")       # (8, 1)  
print(f"Jacobian JF: {JF.shape}")       # (8, 8)
\end{lstlisting}

\textbf{Local Index Construction:} For element k, the function constructs local indices as:
\begin{lstlisting}[language=Python, caption=Local Index Construction Algorithm]
# Initialize local_indices as vector of length 2*neq
local_indices = np.zeros(2 * neq, dtype=int)

# Fill indices for each equation
for ieq in range(neq):
    # For equation ieq: indices k and k+1 in the corresponding block
    local_indices[ieq * 2] = ieq * n_nodes + k      # Left node for equation ieq
    local_indices[ieq * 2 + 1] = ieq * n_nodes + (k + 1)  # Right node for equation ieq

# Example for neq=2, k=1, n_nodes=5:
# local_indices = [1, 2, 6, 7]  # [u_left, u_right, phi_left, phi_right]
\end{lstlisting}

\textbf{Static Condensation Interface:} The function expects the static condensation object to provide:
\begin{lstlisting}[language=Python, caption=Required Static Condensation Interface]
def static_condensation(self, local_trace, local_source):
    """
    Required interface for static condensation objects.
    
    Args:
        local_trace: Local trace values, shape (2*neq, 1) or (2*neq,)
        local_source: Local forcing term, shape (2*neq, 1) or (2*neq,)
        
    Returns:
        Tuple[np.ndarray, np.ndarray, np.ndarray, np.ndarray]:
            - local_solution: Bulk coefficients, shape (2*neq,)
            - flux: Scalar flux value
            - flux_trace: Flux contributions, shape (2*neq,)
            - jacobian: Local Jacobian, shape (2*neq, 2*neq)
    """
    pass
\end{lstlisting}

\textbf{Error Handling:}
\begin{itemize}
    \item \texttt{RuntimeError}: Raised if static condensation fails for any element
    \item Includes element number in error message for debugging
\end{itemize}

\textbf{Integration with BioNetFlux:}
\begin{lstlisting}[language=Python, caption=BioNetFlux Integration Example]
# Integration with GlobalAssembler
class GlobalAssembler:
    def compute_element_contributions(self, trace_solutions, forcing_terms):
        """Example integration in GlobalAssembler."""
        all_bulk_solutions = []
        all_flux_jumps = []
        all_jacobians = []
        
        for domain_idx in range(self.n_domains):
            # Get domain-specific data
            trace_sol = trace_solutions[domain_idx]
            forcing = forcing_terms[domain_idx]
            static_cond = self.static_condensations[domain_idx]
            
            # Compute domain flux jump
            U, F, JF = domain_flux_jump(
                trace_solution=trace_sol,
                forcing_term=forcing,
                problem=self.problems[domain_idx],
                discretization=self.discretizations[domain_idx],
                static_condensation=static_cond
            )
            
            all_bulk_solutions.append(U)
            all_flux_jumps.append(F)
            all_jacobians.append(JF)
        
        return all_bulk_solutions, all_flux_jumps, all_jacobians
\end{lstlisting}

\subsection{Testing Infrastructure}
\label{subsec:flux_jump_testing}

The module provides comprehensive testing infrastructure for validation.

\paragraph{test\_domain\_flux\_jump()}
\begin{lstlisting}[language=Python, caption=Test Function]
def test_domain_flux_jump(verbose=True) -> bool
\end{lstlisting}

\textbf{Parameters:}
\begin{itemize}
    \item \texttt{verbose}: If True, print detailed test information (default: True)
\end{itemize}

\textbf{Returns:} \texttt{bool} - True if all tests pass, False otherwise

\textbf{Test Suite Components:}

\textbf{1. Mock Object Classes:}
\begin{lstlisting}[language=Python, caption=Mock Object Definitions]
class MockProblem:
    def __init__(self, neq=1):
        self.neq = neq
        self.domain_start = 0.0
        self.domain_end = 1.0

class MockDiscretization:
    def __init__(self, n_elements=4):
        self.n_elements = n_elements
        self.nodes = np.linspace(0, 1, n_elements + 1)
        self.element_length = 1.0 / n_elements

class MockStaticCondensation:
    def __init__(self, neq=1):
        self.neq = neq
        
    def static_condensation(self, local_trace, local_source=None):
        """Mock static condensation with predictable results."""
        # Returns deterministic outputs for testing
        pass
\end{lstlisting}

\textbf{2. Test Cases:}
\begin{itemize}
    \item \textbf{Case 1}: Single equation, 3 elements
    \item \textbf{Case 2}: Two equations, 4 elements  
    \item \textbf{Case 3}: Single equation, 5 elements
\end{itemize}

\textbf{3. Validation Tests:}
\begin{enumerate}
    \item \textbf{Shape Validation}: Verify output array shapes match expected dimensions
    \item \textbf{Finite Value Check}: Ensure no NaN or infinite values in outputs
    \item \textbf{Jacobian Properties}: Check Jacobian matrix properties
    \item \textbf{Edge Case Testing}: Test with zero trace and zero forcing
\end{enumerate}

\textbf{Test Output Format:}
\begin{lstlisting}[language=Python, caption=Sample Test Output]
============================================================
TESTING domain_flux_jump FUNCTION
============================================================

Test Case 1: Single equation, 3 elements
----------------------------------------
  Input shapes:
    trace_solution: (4, 1)
    forcing_term: (2, 3)
    ✓ U shape: (2, 3)
    ✓ F shape: (4, 1)
    ✓ JF shape: (4, 4)
    ✓ U contains only finite values
    ✓ F contains only finite values
    ✓ JF contains only finite values
    ✓ JF has non-zero entries
    ✓ Test case passed
  Results summary:
    |U|_max = 1.234567e+00
    |F|_norm = 2.345678e+00
    JF condition = 1.23e+01

Edge Case Tests:
----------------------------------------
  ✓ Zero trace test passed: |F| = 1.234567e-02
  ✓ Zero forcing test passed: |F| = 2.345678e-01

============================================================
✅ ALL TESTS PASSED
============================================================
\end{lstlisting}

\textbf{Usage:}
\begin{lstlisting}[language=Python, caption=Test Function Usage]
# Run tests with detailed output
success = test_domain_flux_jump(verbose=True)
if success:
    print("All tests passed")
else:
    print("Some tests failed")

# Run tests silently
success = test_domain_flux_jump(verbose=False)

# Command line execution
if __name__ == "__main__":
    success = test_domain_flux_jump(verbose=True)
    exit(0 if success else 1)
\end{lstlisting}

\textbf{Mock Static Condensation Algorithm:}
\begin{lstlisting}[language=Python, caption=Mock Static Condensation Details]
def static_condensation(self, local_trace, local_source=None):
    """Mock implementation with predictable behavior."""
    trace_length = len(local_trace.flatten())
    neq = trace_length // 2
    
    if local_source is None:
        local_source = np.zeros(2 * neq)
    
    local_trace_flat = local_trace.flatten()
    local_source_flat = local_source.flatten()
    
    # Mock local solution: 2*neq coefficients
    coeffs_per_element = 2 * neq
    local_solution = np.zeros(coeffs_per_element)
    
    # Fill with simple pattern for testing
    for i in range(min(coeffs_per_element, len(local_trace_flat))):
        local_solution[i] = 0.8 * local_trace_flat[i] + 0.1 * (i + 1)
    
    # Mock flux: scalar
    flux = np.sum(local_trace_flat) * 0.1
    
    # Mock flux_trace: same length as local_trace
    flux_trace = local_trace_flat * 0.9 + local_source_flat * 0.1
    
    # Mock jacobian
    jacobian = np.eye(len(local_trace_flat)) * 1.1 + 0.1
    
    return local_solution, flux, flux_trace, jacobian
\end{lstlisting}

\subsection{Complete Usage Examples}
\label{subsec:flux_jump_complete_examples}

\subsubsection{Integration with Real Static Condensation}

\begin{lstlisting}[language=Python, caption=Real Static Condensation Integration]
from ooc1d.core.flux_jump import domain_flux_jump
from ooc1d.core.static_condensation_ooc import StaticCondensationOOC
from ooc1d.core.problem import Problem
from ooc1d.core.discretization import Discretization, GlobalDiscretization
from ooc1d.utils.elementary_matrices import ElementaryMatrices
import numpy as np

# Setup OrganOnChip problem (4 equations)
ooc_params = np.array([1.0, 2.0, 1.0, 1.0, 0.0, 1.0, 0.0, 1.0, 1.0])
problem = Problem(
    neq=4,
    domain_start=0.0,
    domain_length=1.0,
    parameters=ooc_params,
    problem_type="organ_on_chip"
)

# Setup discretization
discretization = Discretization(n_elements=10)
global_disc = GlobalDiscretization([discretization])

# Create static condensation
elementary_matrices = ElementaryMatrices()
static_condensation = StaticCondensation(
    problem=problem,
    global_discretization=global_disc,
    elementary_matrices=elementary_matrices,
    domain_index=0
)

# Prepare realistic input data
N = 10  # elements
neq = 4  # equations
n_nodes = N + 1  # 11 nodes

# Create trace solution (4 equations × 11 nodes = 44 values)
trace_solution = np.random.rand(neq * n_nodes, 1)

# Create forcing term (8 coefficients × 10 elements)
forcing_term = np.random.rand(2 * neq, N)

# Compute flux jump
U, F, JF = domain_flux_jump(
    trace_solution=trace_solution,
    forcing_term=forcing_term,
    problem=problem,
    discretization=discretization,
    static_condensation=static_condensation
)

print(f"OrganOnChip flux jump computation:")
print(f"  Input: {neq} equations, {N} elements, {n_nodes} nodes")
print(f"  Output shapes:")
print(f"    Bulk solutions U: {U.shape}")     # (8, 10)
print(f"    Flux jumps F: {F.shape}")        # (44, 1)
print(f"    Jacobian JF: {JF.shape}")        # (44, 44)
print(f"  Solution statistics:")
print(f"    |U|_max = {np.max(np.abs(U)):.6e}")
print(f"    |F|_norm = {np.linalg.norm(F):.6e}")
print(f"    JF condition = {np.linalg.cond(JF):.2e}")
\end{lstlisting}



\subsubsection{Error Handling and Debugging}

\begin{lstlisting}[language=Python, caption=Error Handling Example]
def robust_flux_jump_computation(trace_solution, forcing_term, 
                                problem, discretization, static_condensation):
    """
    Robust wrapper for domain_flux_jump with comprehensive error handling.
    """
    try:
        # Validate inputs
        if trace_solution.size == 0:
            raise ValueError("Empty trace solution")
        
        if forcing_term.size == 0:
            raise ValueError("Empty forcing term")
        
        # Check for NaN or infinite values in inputs
        if np.any(np.isnan(trace_solution)) or np.any(np.isinf(trace_solution)):
            raise ValueError("Trace solution contains NaN or infinite values")
        
        if np.any(np.isnan(forcing_term)) or np.any(np.isinf(forcing_term)):
            raise ValueError("Forcing term contains NaN or infinite values")
        
        # Compute flux jump
        U, F, JF = domain_flux_jump(
            trace_solution, forcing_term, problem, discretization, static_condensation
        )
        
        # Validate outputs
        if np.any(np.isnan(U)) or np.any(np.isinf(U)):
            raise RuntimeError("Bulk solution contains NaN or infinite values")
        
        if np.any(np.isnan(F)) or np.any(np.isinf(F)):
            raise RuntimeError("Flux jump contains NaN or infinite values")
        
        if np.any(np.isnan(JF)) or np.any(np.isinf(JF)):
            raise RuntimeError("Jacobian contains NaN or infinite values")
        
        # Check Jacobian conditioning
        jf_condition = np.linalg.cond(JF)
        if jf_condition > 1e12:
            print(f"Warning: Jacobian is poorly conditioned (cond = {jf_condition:.2e})")
        
        return U, F, JF, {"status": "success", "condition": jf_condition}
        
    except Exception as e:
        error_info = {
            "status": "error",
            "error_type": type(e).__name__,
            "error_message": str(e),
            "trace_shape": trace_solution.shape if hasattr(trace_solution, 'shape') else None,
            "forcing_shape": forcing_term.shape if hasattr(forcing_term, 'shape') else None
        }
        
        print(f"Flux jump computation failed: {e}")
        print(f"Error details: {error_info}")
        
        # Return zeros with error info
        neq = forcing_term.shape[0] // 2 if hasattr(forcing_term, 'shape') and forcing_term.size > 0 else 1
        N = forcing_term.shape[1] if hasattr(forcing_term, 'shape') and len(forcing_term.shape) > 1 else 1
        n_nodes = N + 1
        
        U_zero = np.zeros((2 * neq, N))
        F_zero = np.zeros((neq * n_nodes, 1))
        JF_zero = np.eye(neq * n_nodes)  # Identity to avoid singular matrix
        
        return U_zero, F_zero, JF_zero, error_info

# Usage with error handling
U, F, JF, status = robust_flux_jump_computation(
    trace_solution, forcing_term, problem, discretization, static_condensation
)

if status["status"] == "success":
    print(f"✓ Computation successful (condition = {status['condition']:.2e})")
else:
    print(f"✗ Computation failed: {status['error_message']}")
\end{lstlisting}

\subsection{Method and Function Summary}
\label{subsec:flux_jump_summary}

\begin{longtable}{|p{4.3cm}|p{3.5cm}|p{6cm}|}
\hline
\textbf{Function} & \textbf{Returns} & \textbf{Purpose} \\
\hline
\endhead

\texttt{domain\_flux\_jump} & \texttt{Tuple[3×np.ndarray]} & Compute local flux balance for HDG method \\
\hline

\texttt{test\_domain\_flux\_jump} & \texttt{bool} & Comprehensive testing with mock objects \\
\hline

\end{longtable}

\subsection{Key Features and Capabilities}

\begin{itemize}
    \item \textbf{MATLAB Compatibility}: Direct Python equivalent of MATLAB \texttt{fluxJump.m}
    \item \textbf{Multi-Equation Support}: Handles arbitrary number of equations (\texttt{neq})
    \item \textbf{Flexible Element Count}: Works with any number of elements (N)
    \item \textbf{Static Condensation Integration}: Compatible with all static condensation implementations
    \item \textbf{Comprehensive Testing}: Built-in test suite with mock objects and edge cases
    \item \textbf{Error Handling}: Robust error reporting with element-specific debugging
    \item \textbf{Shape Validation}: Automatic input/output shape verification
    \item \textbf{Jacobian Assembly}: Full Jacobian matrix construction for Newton solvers
\end{itemize}

This documentation provides an exact reference for the flux jump module, emphasizing its role in HDG static condensation and integration with the broader BioNetFlux framework.

% End of flux jump module API documentation


\section{Flux Jump Module API Reference (Accurate Analysis)}
\label{sec:flux_jump_module_api}

This section provides an exact reference for the flux jump module (\texttt{ooc1d.core.flux\_jump}) based on detailed analysis of the actual implementation. The module computes local flux balance contributions for HDG methods, serving as the Python equivalent of MATLAB \texttt{fluxJump.m}.

\subsection{Module Overview}

The flux jump module provides:
\begin{itemize}
    \item Local flux balance computation for HDG static condensation
    \item Element-wise bulk solution reconstruction
    \item Flux jump assembly across mesh points
    \item Jacobian computation for Newton solvers
    \item Comprehensive testing infrastructure
\end{itemize}

\subsection{Module Imports and Dependencies}

\begin{lstlisting}[language=Python, caption=Module Dependencies]
import numpy as np
from typing import Tuple
\end{lstlisting}

\subsection{Main Function}
\label{subsec:domain_flux_jump_function}

The primary function for computing local flux balance contributions.

\paragraph{domain\_flux\_jump()}
\begin{lstlisting}[language=Python, caption=Domain Flux Jump Function]
def domain_flux_jump(
    trace_solution: np.ndarray,
    forcing_term: np.ndarray,
    problem, # dummy placeholder for backwards compatibility
    discretization, # dummy placeholder for backwards compatibility
    static_condensation
) -> Tuple[np.ndarray, np.ndarray, np.ndarray]
\end{lstlisting}

\textbf{Purpose:} Python equivalent of MATLAB \texttt{fluxJump.m} function for local flux balance computation

\textbf{Parameters:}
\begin{itemize}
    \item \texttt{trace\_solution}: Trial trace solution vector of length \texttt{neq*(N+1)} where N is number of elements
    \item \texttt{forcing\_term}: \texttt{(2*neq)×N} matrix accounting for previous timestep and forcing terms (stacked for all equations)
    \item \texttt{problem}: Problem instance containing boundary conditions (placeholder for backwards compatibility)
    \item \texttt{discretization}: Discretization parameters (placeholder for backwards compatibility)
    \item \texttt{static\_condensation}: Static condensation implementation with \texttt{static\_condensation()} method
\end{itemize}

\textbf{Returns:} \texttt{Tuple[np.ndarray, np.ndarray, np.ndarray]} - (U, F, JF) where:
\begin{itemize}
    \item \texttt{U}: \texttt{(2*neq)×N} matrix of bulk solutions for each element
    \item \texttt{F}: \texttt{neq*(N+1)} vector of flux jumps at mesh points
    \item \texttt{JF}: \texttt{neq*(N+1)×neq*(N+1)} Jacobian matrix
\end{itemize}

\textbf{Algorithm Description:}
\begin{enumerate}
    \item \textbf{Input Deduction}: Extract N (elements) and neq (equations) from \texttt{forcing\_term.shape}
    \item \textbf{Initialization}: Create output arrays U, F, JF with appropriate dimensions
    \item \textbf{Element Loop}: For each element k = 0, ..., N-1:
    \begin{itemize}
        \item Extract local indices for element nodes
        \item Extract local forcing term and trace values
        \item Apply static condensation to get local solution and flux contributions
        \item Update global bulk solution matrix U
        \item Accumulate flux jump contributions in F
        \item Accumulate Jacobian contributions in JF
    \end{itemize}
\end{enumerate}

\textbf{Input Shape Requirements:}
\begin{itemize}
    \item \texttt{trace\_solution}: Shape \texttt{(neq*(N+1), 1)} or \texttt{(neq*(N+1),)}
    \item \texttt{forcing\_term}: Shape \texttt{(2*neq, N)}
    \item Static condensation must provide method: \texttt{static\_condensation(local\_trace, local\_forcing)}
\end{itemize}

\textbf{Output Shape Guarantees:}
\begin{itemize}
    \item \texttt{U}: Shape \texttt{(2*neq, N)} - bulk solutions for all elements
    \item \texttt{F}: Shape \texttt{(neq*(N+1), 1)} - flux jumps at all nodes
    \item \texttt{JF}: Shape \texttt{(neq*(N+1), neq*(N+1))} - full Jacobian matrix
\end{itemize}

\textbf{Usage Examples:}
\begin{lstlisting}[language=Python, caption=Domain Flux Jump Usage Examples]
import numpy as np
from ooc1d.core.flux_jump import domain_flux_jump

# Example 1: Single equation (neq=1), 4 elements
N = 4
neq = 1
n_nodes = N + 1  # 5 nodes

# Create input data
trace_solution = np.random.rand(neq * n_nodes, 1)  # Shape: (5, 1)
forcing_term = np.random.rand(2 * neq, N)          # Shape: (2, 4)

# Assume static_condensation is available
U, F, JF = domain_flux_jump(
    trace_solution=trace_solution,
    forcing_term=forcing_term,
    problem=None,  # Placeholder
    discretization=None,  # Placeholder
    static_condensation=static_condensation_instance
)

print(f"Bulk solutions U: {U.shape}")    # (2, 4)
print(f"Flux jumps F: {F.shape}")       # (5, 1)
print(f"Jacobian JF: {JF.shape}")       # (5, 5)

# Example 2: Two equations (neq=2), 3 elements
N = 3
neq = 2
n_nodes = N + 1  # 4 nodes

trace_solution = np.random.rand(neq * n_nodes, 1)  # Shape: (8, 1)
forcing_term = np.random.rand(2 * neq, N)          # Shape: (4, 3)

U, F, JF = domain_flux_jump(
    trace_solution, forcing_term, None, None, static_condensation_instance
)

print(f"Bulk solutions U: {U.shape}")    # (4, 3)
print(f"Flux jumps F: {F.shape}")       # (8, 1)  
print(f"Jacobian JF: {JF.shape}")       # (8, 8)
\end{lstlisting}

\textbf{Local Index Construction:} For element k, the function constructs local indices as:
\begin{lstlisting}[language=Python, caption=Local Index Construction Algorithm]
# Initialize local_indices as vector of length 2*neq
local_indices = np.zeros(2 * neq, dtype=int)

# Fill indices for each equation
for ieq in range(neq):
    # For equation ieq: indices k and k+1 in the corresponding block
    local_indices[ieq * 2] = ieq * n_nodes + k      # Left node for equation ieq
    local_indices[ieq * 2 + 1] = ieq * n_nodes + (k + 1)  # Right node for equation ieq

# Example for neq=2, k=1, n_nodes=5:
# local_indices = [1, 2, 6, 7]  # [u_left, u_right, phi_left, phi_right]
\end{lstlisting}

\textbf{Static Condensation Interface:} The function expects the static condensation object to provide:
\begin{lstlisting}[language=Python, caption=Required Static Condensation Interface]
def static_condensation(self, local_trace, local_source):
    """
    Required interface for static condensation objects.
    
    Args:
        local_trace: Local trace values, shape (2*neq, 1) or (2*neq,)
        local_source: Local forcing term, shape (2*neq, 1) or (2*neq,)
        
    Returns:
        Tuple[np.ndarray, np.ndarray, np.ndarray, np.ndarray]:
            - local_solution: Bulk coefficients, shape (2*neq,)
            - flux: Scalar flux value
            - flux_trace: Flux contributions, shape (2*neq,)
            - jacobian: Local Jacobian, shape (2*neq, 2*neq)
    """
    pass
\end{lstlisting}

\textbf{Error Handling:}
\begin{itemize}
    \item \texttt{RuntimeError}: Raised if static condensation fails for any element
    \item Includes element number in error message for debugging
\end{itemize}

\textbf{Integration with BioNetFlux:}
\begin{lstlisting}[language=Python, caption=BioNetFlux Integration Example]
# Integration with GlobalAssembler
class GlobalAssembler:
    def compute_element_contributions(self, trace_solutions, forcing_terms):
        """Example integration in GlobalAssembler."""
        all_bulk_solutions = []
        all_flux_jumps = []
        all_jacobians = []
        
        for domain_idx in range(self.n_domains):
            # Get domain-specific data
            trace_sol = trace_solutions[domain_idx]
            forcing = forcing_terms[domain_idx]
            static_cond = self.static_condensations[domain_idx]
            
            # Compute domain flux jump
            U, F, JF = domain_flux_jump(
                trace_solution=trace_sol,
                forcing_term=forcing,
                problem=self.problems[domain_idx],
                discretization=self.discretizations[domain_idx],
                static_condensation=static_cond
            )
            
            all_bulk_solutions.append(U)
            all_flux_jumps.append(F)
            all_jacobians.append(JF)
        
        return all_bulk_solutions, all_flux_jumps, all_jacobians
\end{lstlisting}

\subsection{Testing Infrastructure}
\label{subsec:flux_jump_testing}

The module provides comprehensive testing infrastructure for validation.

\paragraph{test\_domain\_flux\_jump()}
\begin{lstlisting}[language=Python, caption=Test Function]
def test_domain_flux_jump(verbose=True) -> bool
\end{lstlisting}

\textbf{Parameters:}
\begin{itemize}
    \item \texttt{verbose}: If True, print detailed test information (default: True)
\end{itemize}

\textbf{Returns:} \texttt{bool} - True if all tests pass, False otherwise

\textbf{Test Suite Components:}

\textbf{1. Mock Object Classes:}
\begin{lstlisting}[language=Python, caption=Mock Object Definitions]
class MockProblem:
    def __init__(self, neq=1):
        self.neq = neq
        self.domain_start = 0.0
        self.domain_end = 1.0

class MockDiscretization:
    def __init__(self, n_elements=4):
        self.n_elements = n_elements
        self.nodes = np.linspace(0, 1, n_elements + 1)
        self.element_length = 1.0 / n_elements

class MockStaticCondensation:
    def __init__(self, neq=1):
        self.neq = neq
        
    def static_condensation(self, local_trace, local_source=None):
        """Mock static condensation with predictable results."""
        # Returns deterministic outputs for testing
        pass
\end{lstlisting}

\textbf{2. Test Cases:}
\begin{itemize}
    \item \textbf{Case 1}: Single equation, 3 elements
    \item \textbf{Case 2}: Two equations, 4 elements  
    \item \textbf{Case 3}: Single equation, 5 elements
\end{itemize}

\textbf{3. Validation Tests:}
\begin{enumerate}
    \item \textbf{Shape Validation}: Verify output array shapes match expected dimensions
    \item \textbf{Finite Value Check}: Ensure no NaN or infinite values in outputs
    \item \textbf{Jacobian Properties}: Check Jacobian matrix properties
    \item \textbf{Edge Case Testing}: Test with zero trace and zero forcing
\end{enumerate}

\textbf{Test Output Format:}
\begin{lstlisting}[language=Python, caption=Sample Test Output]
============================================================
TESTING domain_flux_jump FUNCTION
============================================================

Test Case 1: Single equation, 3 elements
----------------------------------------
  Input shapes:
    trace_solution: (4, 1)
    forcing_term: (2, 3)
    ✓ U shape: (2, 3)
    ✓ F shape: (4, 1)
    ✓ JF shape: (4, 4)
    ✓ U contains only finite values
    ✓ F contains only finite values
    ✓ JF contains only finite values
    ✓ JF has non-zero entries
    ✓ Test case passed
  Results summary:
    |U|_max = 1.234567e+00
    |F|_norm = 2.345678e+00
    JF condition = 1.23e+01

Edge Case Tests:
----------------------------------------
  ✓ Zero trace test passed: |F| = 1.234567e-02
  ✓ Zero forcing test passed: |F| = 2.345678e-01

============================================================
✅ ALL TESTS PASSED
============================================================
\end{lstlisting}

\textbf{Usage:}
\begin{lstlisting}[language=Python, caption=Test Function Usage]
# Run tests with detailed output
success = test_domain_flux_jump(verbose=True)
if success:
    print("All tests passed")
else:
    print("Some tests failed")

# Run tests silently
success = test_domain_flux_jump(verbose=False)

# Command line execution
if __name__ == "__main__":
    success = test_domain_flux_jump(verbose=True)
    exit(0 if success else 1)
\end{lstlisting}

\textbf{Mock Static Condensation Algorithm:}
\begin{lstlisting}[language=Python, caption=Mock Static Condensation Details]
def static_condensation(self, local_trace, local_source=None):
    """Mock implementation with predictable behavior."""
    trace_length = len(local_trace.flatten())
    neq = trace_length // 2
    
    if local_source is None:
        local_source = np.zeros(2 * neq)
    
    local_trace_flat = local_trace.flatten()
    local_source_flat = local_source.flatten()
    
    # Mock local solution: 2*neq coefficients
    coeffs_per_element = 2 * neq
    local_solution = np.zeros(coeffs_per_element)
    
    # Fill with simple pattern for testing
    for i in range(min(coeffs_per_element, len(local_trace_flat))):
        local_solution[i] = 0.8 * local_trace_flat[i] + 0.1 * (i + 1)
    
    # Mock flux: scalar
    flux = np.sum(local_trace_flat) * 0.1
    
    # Mock flux_trace: same length as local_trace
    flux_trace = local_trace_flat * 0.9 + local_source_flat * 0.1
    
    # Mock jacobian
    jacobian = np.eye(len(local_trace_flat)) * 1.1 + 0.1
    
    return local_solution, flux, flux_trace, jacobian
\end{lstlisting}

\subsection{Complete Usage Examples}
\label{subsec:flux_jump_complete_examples}

\subsubsection{Integration with Real Static Condensation}

\begin{lstlisting}[language=Python, caption=Real Static Condensation Integration]
from ooc1d.core.flux_jump import domain_flux_jump
from ooc1d.core.static_condensation_ooc import StaticCondensationOOC
from ooc1d.core.problem import Problem
from ooc1d.core.discretization import Discretization, GlobalDiscretization
from ooc1d.utils.elementary_matrices import ElementaryMatrices
import numpy as np

# Setup OrganOnChip problem (4 equations)
ooc_params = np.array([1.0, 2.0, 1.0, 1.0, 0.0, 1.0, 0.0, 1.0, 1.0])
problem = Problem(
    neq=4,
    domain_start=0.0,
    domain_length=1.0,
    parameters=ooc_params,
    problem_type="organ_on_chip"
)

# Setup discretization
discretization = Discretization(n_elements=10)
global_disc = GlobalDiscretization([discretization])

# Create static condensation
elementary_matrices = ElementaryMatrices()
static_condensation = StaticCondensation(
    problem=problem,
    global_discretization=global_disc,
    elementary_matrices=elementary_matrices,
    domain_index=0
)

# Prepare realistic input data
N = 10  # elements
neq = 4  # equations
n_nodes = N + 1  # 11 nodes

# Create trace solution (4 equations × 11 nodes = 44 values)
trace_solution = np.random.rand(neq * n_nodes, 1)

# Create forcing term (8 coefficients × 10 elements)
forcing_term = np.random.rand(2 * neq, N)

# Compute flux jump
U, F, JF = domain_flux_jump(
    trace_solution=trace_solution,
    forcing_term=forcing_term,
    problem=problem,
    discretization=discretization,
    static_condensation=static_condensation
)

print(f"OrganOnChip flux jump computation:")
print(f"  Input: {neq} equations, {N} elements, {n_nodes} nodes")
print(f"  Output shapes:")
print(f"    Bulk solutions U: {U.shape}")     # (8, 10)
print(f"    Flux jumps F: {F.shape}")        # (44, 1)
print(f"    Jacobian JF: {JF.shape}")        # (44, 44)
print(f"  Solution statistics:")
print(f"    |U|_max = {np.max(np.abs(U)):.6e}")
print(f"    |F|_norm = {np.linalg.norm(F):.6e}")
print(f"    JF condition = {np.linalg.cond(JF):.2e}")
\end{lstlisting}



\subsubsection{Error Handling and Debugging}

\begin{lstlisting}[language=Python, caption=Error Handling Example]
def robust_flux_jump_computation(trace_solution, forcing_term, 
                                problem, discretization, static_condensation):
    """
    Robust wrapper for domain_flux_jump with comprehensive error handling.
    """
    try:
        # Validate inputs
        if trace_solution.size == 0:
            raise ValueError("Empty trace solution")
        
        if forcing_term.size == 0:
            raise ValueError("Empty forcing term")
        
        # Check for NaN or infinite values in inputs
        if np.any(np.isnan(trace_solution)) or np.any(np.isinf(trace_solution)):
            raise ValueError("Trace solution contains NaN or infinite values")
        
        if np.any(np.isnan(forcing_term)) or np.any(np.isinf(forcing_term)):
            raise ValueError("Forcing term contains NaN or infinite values")
        
        # Compute flux jump
        U, F, JF = domain_flux_jump(
            trace_solution, forcing_term, problem, discretization, static_condensation
        )
        
        # Validate outputs
        if np.any(np.isnan(U)) or np.any(np.isinf(U)):
            raise RuntimeError("Bulk solution contains NaN or infinite values")
        
        if np.any(np.isnan(F)) or np.any(np.isinf(F)):
            raise RuntimeError("Flux jump contains NaN or infinite values")
        
        if np.any(np.isnan(JF)) or np.any(np.isinf(JF)):
            raise RuntimeError("Jacobian contains NaN or infinite values")
        
        # Check Jacobian conditioning
        jf_condition = np.linalg.cond(JF)
        if jf_condition > 1e12:
            print(f"Warning: Jacobian is poorly conditioned (cond = {jf_condition:.2e})")
        
        return U, F, JF, {"status": "success", "condition": jf_condition}
        
    except Exception as e:
        error_info = {
            "status": "error",
            "error_type": type(e).__name__,
            "error_message": str(e),
            "trace_shape": trace_solution.shape if hasattr(trace_solution, 'shape') else None,
            "forcing_shape": forcing_term.shape if hasattr(forcing_term, 'shape') else None
        }
        
        print(f"Flux jump computation failed: {e}")
        print(f"Error details: {error_info}")
        
        # Return zeros with error info
        neq = forcing_term.shape[0] // 2 if hasattr(forcing_term, 'shape') and forcing_term.size > 0 else 1
        N = forcing_term.shape[1] if hasattr(forcing_term, 'shape') and len(forcing_term.shape) > 1 else 1
        n_nodes = N + 1
        
        U_zero = np.zeros((2 * neq, N))
        F_zero = np.zeros((neq * n_nodes, 1))
        JF_zero = np.eye(neq * n_nodes)  # Identity to avoid singular matrix
        
        return U_zero, F_zero, JF_zero, error_info

# Usage with error handling
U, F, JF, status = robust_flux_jump_computation(
    trace_solution, forcing_term, problem, discretization, static_condensation
)

if status["status"] == "success":
    print(f"✓ Computation successful (condition = {status['condition']:.2e})")
else:
    print(f"✗ Computation failed: {status['error_message']}")
\end{lstlisting}

\subsection{Method and Function Summary}
\label{subsec:flux_jump_summary}

\begin{longtable}{|p{4.3cm}|p{3.5cm}|p{6cm}|}
\hline
\textbf{Function} & \textbf{Returns} & \textbf{Purpose} \\
\hline
\endhead

\texttt{domain\_flux\_jump} & \texttt{Tuple[3×np.ndarray]} & Compute local flux balance for HDG method \\
\hline

\texttt{test\_domain\_flux\_jump} & \texttt{bool} & Comprehensive testing with mock objects \\
\hline

\end{longtable}

\subsection{Key Features and Capabilities}

\begin{itemize}
    \item \textbf{MATLAB Compatibility}: Direct Python equivalent of MATLAB \texttt{fluxJump.m}
    \item \textbf{Multi-Equation Support}: Handles arbitrary number of equations (\texttt{neq})
    \item \textbf{Flexible Element Count}: Works with any number of elements (N)
    \item \textbf{Static Condensation Integration}: Compatible with all static condensation implementations
    \item \textbf{Comprehensive Testing}: Built-in test suite with mock objects and edge cases
    \item \textbf{Error Handling}: Robust error reporting with element-specific debugging
    \item \textbf{Shape Validation}: Automatic input/output shape verification
    \item \textbf{Jacobian Assembly}: Full Jacobian matrix construction for Newton solvers
\end{itemize}

This documentation provides an exact reference for the flux jump module, emphasizing its role in HDG static condensation and integration with the broader BioNetFlux framework.

% End of flux jump module API documentation
