% Domain Geometry Module API Documentation
% To be included in master LaTeX document
%
% Usage: % Domain Geometry Module API Documentation
% To be included in master LaTeX document
%
% Usage: % Domain Geometry Module API Documentation
% To be included in master LaTeX document
%
% Usage: % Domain Geometry Module API Documentation
% To be included in master LaTeX document
%
% Usage: \input{docs/domain_geometry_api}

\section{Domain Geometry Module API Reference}
\label{sec:domain_geometry_api}

This section provides a comprehensive reference for the domain geometry module (\texttt{domain\_geometry.py}), which provides lean geometry management for multi-domain BioNetFlux problems.

\subsection{Module Overview}

The module contains two main classes:
\begin{itemize}
    \item \texttt{DomainInfo}: Data container for individual domain properties
    \item \texttt{DomainGeometry}: Main geometry manager for collections of domains
\end{itemize}

\subsection{DomainInfo Class}
\label{subsec:domaininfo_class}

A dataclass container for domain geometric information, representing individual segments in the network geometry.

\subsubsection{Attributes}

\begin{longtable}{|p{3cm}|p{3.5cm}|p{6cm}|}
\hline
\textbf{Attribute} & \textbf{Type} & \textbf{Description} \\
\hline
\endhead

\texttt{domain\_id} & \texttt{int} & Unique identifier for the domain within the geometry \\
\hline

\texttt{extrema\_start} & \texttt{Tuple[float, float]} & Physical coordinates (x₁, y₁) of domain start point \\
\hline

\texttt{extrema\_end} & \texttt{Tuple[float, float]} & Physical coordinates (x₂, y₂) of domain end point \\
\hline

\texttt{domain\_start} & \texttt{float} & Parameter space start coordinate (default: 0.0) \\
\hline

\texttt{domain\_length} & \texttt{float} & Parameter space length (default: Euclidean distance) \\
\hline

\texttt{name} & \texttt{Optional[str]} & Human-readable name for the domain \\
\hline

\texttt{metadata} & \texttt{Dict[str, Any]} & Additional domain-specific data storage \\
\hline

\end{longtable}

\subsubsection{Methods}

\paragraph{Constructor}
\begin{lstlisting}[language=Python, caption=DomainInfo Constructor]
def __init__(self, domain_id: int, 
             extrema_start: Tuple[float, float],
             extrema_end: Tuple[float, float],
             domain_start: float = 0.0,
             domain_length: float = 1.0,
             name: Optional[str] = None,
             metadata: Dict[str, Any] = None)
\end{lstlisting}

\textbf{Parameters:}
\begin{itemize}
    \item \texttt{domain\_id}: Unique domain identifier
    \item \texttt{extrema\_start}: Start point (x₁, y₁) in physical coordinates
    \item \texttt{extrema\_end}: End point (x₂, y₂) in physical coordinates
    \item \texttt{domain\_start}: Parameter space origin (optional, default: 0.0)
    \item \texttt{domain\_length}: Parameter space length (optional, default: calculated)
    \item \texttt{name}: Domain name (optional)
    \item \texttt{metadata}: Additional properties (optional)
\end{itemize}

\textbf{Usage Example:}
\begin{lstlisting}[language=Python, caption=DomainInfo Usage]
# Create a domain from (0,0) to (1,1)
domain = DomainInfo(
    domain_id=0,
    extrema_start=(0.0, 0.0),
    extrema_end=(1.0, 1.0),
    name="diagonal_segment"
)
# domain_length will be automatically set to sqrt(2)
\end{lstlisting}

\paragraph{euclidean\_length()}
\begin{lstlisting}[language=Python, caption=Euclidean Length Calculation]
def euclidean_length(self) -> float
\end{lstlisting}

\textbf{Returns:} \texttt{float} - Euclidean distance between \texttt{extrema\_start} and \texttt{extrema\_end}

\textbf{Formula:} $\sqrt{(x_2-x_1)^2 + (y_2-y_1)^2}$

\textbf{Usage:}
\begin{lstlisting}[language=Python]
length = domain.euclidean_length()  # Returns geometric length
\end{lstlisting}

\paragraph{center\_point()}
\begin{lstlisting}[language=Python, caption=Center Point Calculation]
def center_point(self) -> Tuple[float, float]
\end{lstlisting}

\textbf{Returns:} \texttt{Tuple[float, float]} - Midpoint coordinates $(x_c, y_c)$

\textbf{Formula:} $x_c = \frac{x_1 + x_2}{2}$, $y_c = \frac{y_1 + y_2}{2}$

\textbf{Usage:}
\begin{lstlisting}[language=Python]
center = domain.center_point()  # Returns (x_center, y_center)
\end{lstlisting}

\paragraph{direction\_vector()}
\begin{lstlisting}[language=Python, caption=Direction Vector Calculation]
def direction_vector(self) -> Tuple[float, float]
\end{lstlisting}

\textbf{Returns:} \texttt{Tuple[float, float]} - Unit direction vector from start to end

\textbf{Formula:} $\vec{u} = \frac{(x_2-x_1, y_2-y_1)}{||(x_2-x_1, y_2-y_1)||}$

\textbf{Usage:}
\begin{lstlisting}[language=Python]
direction = domain.direction_vector()  # Returns (u_x, u_y)
\end{lstlisting}

\subsection{DomainGeometry Class}
\label{subsec:domaingeometry_class}

Main geometry manager class for handling collections of domains and providing interface methods for BioNetFlux integration.

\subsubsection{Attributes}

\begin{longtable}{|p{3cm}|p{2.5cm}|p{7cm}|}
\hline
\textbf{Attribute} & \textbf{Type} & \textbf{Description} \\
\hline
\endhead

\texttt{name} & \texttt{str} & Descriptive name for the geometry \\
\hline

\texttt{domains} & \texttt{List[DomainInfo]} & Collection of domain information objects \\
\hline

\texttt{\_next\_id} & \texttt{int} & Internal counter for generating unique domain IDs \\
\hline

\texttt{\_global\_metadata} & \texttt{Dict[str, Any]} & Geometry-wide metadata storage \\
\hline

\end{longtable}

\subsubsection{Constructor and Basic Operations}

\paragraph{Constructor}
\begin{lstlisting}[language=Python, caption=DomainGeometry Constructor]
def __init__(self, name: str = "unnamed_geometry")
\end{lstlisting}

\textbf{Parameters:}
\begin{itemize}
    \item \texttt{name}: Descriptive name for the geometry (optional, default: "unnamed\_geometry")
\end{itemize}

\textbf{Usage:}
\begin{lstlisting}[language=Python]
# Create new geometry
geometry = DomainGeometry(name="vascular_network")
\end{lstlisting}

\paragraph{add\_domain()}
\begin{lstlisting}[language=Python, caption=Add Domain Method]
def add_domain(self, 
               extrema_start: Tuple[float, float],
               extrema_end: Tuple[float, float],
               domain_start: Optional[float] = None,
               domain_length: Optional[float] = None,
               name: Optional[str] = None,
               **metadata) -> int
\end{lstlisting}

\textbf{Parameters:}
\begin{itemize}
    \item \texttt{extrema\_start}: Start point (x₁, y₁) in physical space
    \item \texttt{extrema\_end}: End point (x₂, y₂) in physical space
    \item \texttt{domain\_start}: Parameter space start (optional, default: 0.0)
    \item \texttt{domain\_length}: Parameter space length (optional, default: Euclidean distance)
    \item \texttt{name}: Domain name (optional, auto-generated if None)
    \item \texttt{**metadata}: Additional domain-specific data
\end{itemize}

\textbf{Returns:} \texttt{int} - Domain ID of the newly added domain

\textbf{Usage:}
\begin{lstlisting}[language=Python, caption=Adding Domains Example]
# Add horizontal segment
domain_id_1 = geometry.add_domain(
    extrema_start=(0.0, 0.0),
    extrema_end=(1.0, 0.0),
    name="main_vessel",
    vessel_type="artery",
    diameter=0.1
)

# Add vertical segment
domain_id_2 = geometry.add_domain(
    extrema_start=(1.0, 0.0),
    extrema_end=(1.0, 1.0),
    name="branch_vessel"
)
\end{lstlisting}

\subsubsection{Domain Access and Query Methods}

\paragraph{get\_domain()}
\begin{lstlisting}[language=Python, caption=Get Domain Method]
def get_domain(self, domain_id: int) -> DomainInfo
\end{lstlisting}

\textbf{Parameters:}
\begin{itemize}
    \item \texttt{domain\_id}: Domain index to retrieve
\end{itemize}

\textbf{Returns:} \texttt{DomainInfo} - Complete domain information object

\textbf{Raises:} \texttt{IndexError} - If domain\_id is out of range

\textbf{Usage:}
\begin{lstlisting}[language=Python]
domain = geometry.get_domain(0)  # Get first domain
print(f"Domain name: {domain.name}")
print(f"Length: {domain.euclidean_length()}")
\end{lstlisting}

\paragraph{get\_all\_domains()}
\begin{lstlisting}[language=Python, caption=Get All Domains Method]
def get_all_domains(self) -> List[DomainInfo]
\end{lstlisting}

\textbf{Returns:} \texttt{List[DomainInfo]} - Copy of all domains in the geometry

\textbf{Usage:}
\begin{lstlisting}[language=Python]
all_domains = geometry.get_all_domains()
for domain in all_domains:
    print(f"Domain {domain.domain_id}: {domain.name}")
\end{lstlisting}

\paragraph{num\_domains()}
\begin{lstlisting}[language=Python, caption=Number of Domains Method]
def num_domains(self) -> int
\end{lstlisting}

\textbf{Returns:} \texttt{int} - Total number of domains in the geometry

\textbf{Usage:}
\begin{lstlisting}[language=Python]
n_domains = geometry.num_domains()
print(f"Geometry contains {n_domains} domains")
\end{lstlisting}

\paragraph{find\_domain\_by\_name()}
\begin{lstlisting}[language=Python, caption=Find Domain by Name Method]
def find_domain_by_name(self, name: str) -> Optional[int]
\end{lstlisting}

\textbf{Parameters:}
\begin{itemize}
    \item \texttt{name}: Domain name to search for
\end{itemize}

\textbf{Returns:} \texttt{Optional[int]} - Domain ID if found, None otherwise

\textbf{Usage:}
\begin{lstlisting}[language=Python]
domain_id = geometry.find_domain_by_name("main_vessel")
if domain_id is not None:
    domain = geometry.get_domain(domain_id)
\end{lstlisting}

\paragraph{get\_domain\_names()}
\begin{lstlisting}[language=Python, caption=Get Domain Names Method]
def get_domain_names(self) -> List[str]
\end{lstlisting}

\textbf{Returns:} \texttt{List[str]} - List of all domain names

\textbf{Usage:}
\begin{lstlisting}[language=Python]
names = geometry.get_domain_names()
print("Available domains:", ", ".join(names))
\end{lstlisting}

\subsubsection{Geometric Analysis Methods}

\paragraph{get\_bounding\_box()}
\begin{lstlisting}[language=Python, caption=Bounding Box Calculation]
def get_bounding_box(self) -> Dict[str, float]
\end{lstlisting}

\textbf{Returns:} \texttt{Dict[str, float]} - Dictionary with keys: \texttt{x\_min}, \texttt{x\_max}, \texttt{y\_min}, \texttt{y\_max}

\textbf{Usage:}
\begin{lstlisting}[language=Python]
bbox = geometry.get_bounding_box()
width = bbox['x_max'] - bbox['x_min']
height = bbox['y_max'] - bbox['y_min']
print(f"Geometry bounding box: {width} × {height}")
\end{lstlisting}

\paragraph{total\_length()}
\begin{lstlisting}[language=Python, caption=Total Length Calculation]
def total_length(self) -> float
\end{lstlisting}

\textbf{Returns:} \texttt{float} - Sum of Euclidean lengths of all domains

\textbf{Usage:}
\begin{lstlisting}[language=Python]
total_len = geometry.total_length()
avg_len = total_len / geometry.num_domains()
print(f"Total network length: {total_len:.3f}")
print(f"Average segment length: {avg_len:.3f}")
\end{lstlisting}

\subsubsection{Metadata Management}

\paragraph{set\_global\_metadata()}
\begin{lstlisting}[language=Python, caption=Set Global Metadata Method]
def set_global_metadata(self, **metadata)
\end{lstlisting}

\textbf{Parameters:}
\begin{itemize}
    \item \texttt{**metadata}: Key-value pairs for geometry-wide metadata
\end{itemize}

\textbf{Usage:}
\begin{lstlisting}[language=Python]
geometry.set_global_metadata(
    problem_type="organ_on_chip",
    fluid_viscosity=0.001,
    temperature=37.0,
    units="mm"
)
\end{lstlisting}

\paragraph{get\_global\_metadata()}
\begin{lstlisting}[language=Python, caption=Get Global Metadata Method]
def get_global_metadata(self) -> Dict[str, Any]
\end{lstlisting}

\textbf{Returns:} \texttt{Dict[str, Any]} - Copy of all global metadata

\textbf{Usage:}
\begin{lstlisting}[language=Python]
metadata = geometry.get_global_metadata()
if "fluid_viscosity" in metadata:
    viscosity = metadata["fluid_viscosity"]
\end{lstlisting}

\subsubsection{Utility and Maintenance Methods}

\paragraph{remove\_domain()}
\begin{lstlisting}[language=Python, caption=Remove Domain Method]
def remove_domain(self, domain_id: int)
\end{lstlisting}

\textbf{Parameters:}
\begin{itemize}
    \item \texttt{domain\_id}: Domain ID to remove
\end{itemize}

\textbf{Raises:} \texttt{IndexError} - If domain\_id is invalid

\textbf{Note:} Automatically renumbers remaining domains to maintain consistency

\textbf{Usage:}
\begin{lstlisting}[language=Python]
# Remove domain and automatically renumber others
geometry.remove_domain(1)
\end{lstlisting}

\paragraph{summary()}
\begin{lstlisting}[language=Python, caption=Summary Generation Method]
def summary(self) -> str
\end{lstlisting}

\textbf{Returns:} \texttt{str} - Multi-line summary of geometry contents

\textbf{Usage:}
\begin{lstlisting}[language=Python]
print(geometry.summary())
# Output:
# Geometry: vascular_network
# Number of domains: 3
# Total length: 4.236
# Domains:
#   0: main_vessel
#     Extrema: (0.0, 0.0) → (1.0, 0.0)
#     Parameter: [0.000, 1.000]
#     Length: 1.000
# ...
\end{lstlisting}

\subsubsection{Special Methods (Python Magic Methods)}

\paragraph{\_\_len\_\_()}
\begin{lstlisting}[language=Python, caption=Length Support]
def __len__(self) -> int
\end{lstlisting}

\textbf{Returns:} \texttt{int} - Number of domains (enables \texttt{len(geometry)})

\textbf{Usage:}
\begin{lstlisting}[language=Python]
num_domains = len(geometry)  # Equivalent to geometry.num_domains()
\end{lstlisting}

\paragraph{\_\_getitem\_\_()}
\begin{lstlisting}[language=Python, caption=Indexing Support]
def __getitem__(self, domain_id: int) -> DomainInfo
\end{lstlisting}

\textbf{Parameters:}
\begin{itemize}
    \item \texttt{domain\_id}: Domain index
\end{itemize}

\textbf{Returns:} \texttt{DomainInfo} - Domain at specified index (enables \texttt{geometry[i]})

\textbf{Usage:}
\begin{lstlisting}[language=Python]
first_domain = geometry[0]  # Equivalent to geometry.get_domain(0)
\end{lstlisting}

\paragraph{\_\_iter\_\_()}
\begin{lstlisting}[language=Python, caption=Iteration Support]
def __iter__(self)
\end{lstlisting}

\textbf{Returns:} Iterator over all \texttt{DomainInfo} objects

\textbf{Usage:}
\begin{lstlisting}[language=Python]
for domain in geometry:
    print(f"Processing domain {domain.domain_id}: {domain.name}")
    length = domain.euclidean_length()
    center = domain.center_point()
\end{lstlisting}

\subsection{Complete Usage Example}
\label{subsec:complete_example}

\begin{lstlisting}[language=Python, caption=Complete Geometry Usage Example]
from ooc1d.geometry.domain_geometry import DomainGeometry

# Create geometry for Y-junction network
geometry = DomainGeometry(name="y_junction_network")

# Add main vessel (horizontal segment)
main_id = geometry.add_domain(
    extrema_start=(0.0, 0.0),
    extrema_end=(2.0, 0.0),
    name="main_vessel",
    vessel_type="parent",
    diameter=1.0
)

# Add upper branch
upper_id = geometry.add_domain(
    extrema_start=(2.0, 0.0),
    extrema_end=(3.0, 1.0),
    name="upper_branch",
    vessel_type="daughter",
    diameter=0.7
)

# Add lower branch  
lower_id = geometry.add_domain(
    extrema_start=(2.0, 0.0),
    extrema_end=(3.0, -1.0),
    name="lower_branch",
    vessel_type="daughter",
    diameter=0.7
)

# Set global properties
geometry.set_global_metadata(
    fluid_type="blood",
    viscosity=0.004,  # Pa·s
    density=1060,     # kg/m³
    problem_type="organ_on_chip"
)

# Analyze geometry
print(geometry.summary())
print(f"\nTotal network length: {geometry.total_length():.3f}")

# Access individual domains
for domain in geometry:
    center = domain.center_point()
    direction = domain.direction_vector()
    print(f"Domain {domain.name}:")
    print(f"  Center: ({center[0]:.2f}, {center[1]:.2f})")
    print(f"  Direction: ({direction[0]:.2f}, {direction[1]:.2f})")
    
    # Access metadata
    if "diameter" in domain.metadata:
        print(f"  Diameter: {domain.metadata['diameter']}")

# Find specific domain
main_domain_id = geometry.find_domain_by_name("main_vessel")
if main_domain_id is not None:
    main_domain = geometry[main_domain_id]
    print(f"\nMain vessel length: {main_domain.euclidean_length():.3f}")
\end{lstlisting}

\subsection{Integration with BioNetFlux}
\label{subsec:bionetflux_integration}

The \texttt{DomainGeometry} class is designed for seamless integration with BioNetFlux components:

\begin{lstlisting}[language=Python, caption=BioNetFlux Integration Example]
from ooc1d.core.problem import Problem
from ooc1d.core.discretization import Discretization, GlobalDiscretization

# Create BioNetFlux problems from geometry
def geometry_to_problems(geometry, neq=4, parameters=None):
    """Convert geometry domains to BioNetFlux Problem instances."""
    problems = []
    
    for domain in geometry:
        problem = Problem(
            neq=neq,
            domain_start=domain.domain_start,
            domain_length=domain.domain_length,
            parameters=parameters or np.ones(9),  # OrganOnChip parameters
            problem_type="organ_on_chip",
            name=domain.name
        )
        problems.append(problem)
    
    return problems

# Create discretizations from geometry
def geometry_to_discretizations(geometry, n_elements=10):
    """Convert geometry domains to BioNetFlux Discretization instances."""
    discretizations = []
    
    for domain in geometry:
        disc = Discretization(
            n_elements=n_elements,
            domain_start=domain.domain_start,
            domain_length=domain.domain_length,
            stab_constant=1.0
        )
        discretizations.append(disc)
    
    return GlobalDiscretization(discretizations)

# Usage
problems = geometry_to_problems(geometry)
global_disc = geometry_to_discretizations(geometry)
\end{lstlisting}

This API provides a comprehensive interface for managing complex multi-domain geometries while maintaining compatibility with the existing BioNetFlux solver framework.

% End of domain geometry API documentation


\section{Domain Geometry Module API Reference}
\label{sec:domain_geometry_api}

This section provides a comprehensive reference for the domain geometry module (\texttt{domain\_geometry.py}), which provides lean geometry management for multi-domain BioNetFlux problems.

\subsection{Module Overview}

The module contains two main classes:
\begin{itemize}
    \item \texttt{DomainInfo}: Data container for individual domain properties
    \item \texttt{DomainGeometry}: Main geometry manager for collections of domains
\end{itemize}

\subsection{DomainInfo Class}
\label{subsec:domaininfo_class}

A dataclass container for domain geometric information, representing individual segments in the network geometry.

\subsubsection{Attributes}

\begin{longtable}{|p{3cm}|p{3.5cm}|p{6cm}|}
\hline
\textbf{Attribute} & \textbf{Type} & \textbf{Description} \\
\hline
\endhead

\texttt{domain\_id} & \texttt{int} & Unique identifier for the domain within the geometry \\
\hline

\texttt{extrema\_start} & \texttt{Tuple[float, float]} & Physical coordinates (x₁, y₁) of domain start point \\
\hline

\texttt{extrema\_end} & \texttt{Tuple[float, float]} & Physical coordinates (x₂, y₂) of domain end point \\
\hline

\texttt{domain\_start} & \texttt{float} & Parameter space start coordinate (default: 0.0) \\
\hline

\texttt{domain\_length} & \texttt{float} & Parameter space length (default: Euclidean distance) \\
\hline

\texttt{name} & \texttt{Optional[str]} & Human-readable name for the domain \\
\hline

\texttt{metadata} & \texttt{Dict[str, Any]} & Additional domain-specific data storage \\
\hline

\end{longtable}

\subsubsection{Methods}

\paragraph{Constructor}
\begin{lstlisting}[language=Python, caption=DomainInfo Constructor]
def __init__(self, domain_id: int, 
             extrema_start: Tuple[float, float],
             extrema_end: Tuple[float, float],
             domain_start: float = 0.0,
             domain_length: float = 1.0,
             name: Optional[str] = None,
             metadata: Dict[str, Any] = None)
\end{lstlisting}

\textbf{Parameters:}
\begin{itemize}
    \item \texttt{domain\_id}: Unique domain identifier
    \item \texttt{extrema\_start}: Start point (x₁, y₁) in physical coordinates
    \item \texttt{extrema\_end}: End point (x₂, y₂) in physical coordinates
    \item \texttt{domain\_start}: Parameter space origin (optional, default: 0.0)
    \item \texttt{domain\_length}: Parameter space length (optional, default: calculated)
    \item \texttt{name}: Domain name (optional)
    \item \texttt{metadata}: Additional properties (optional)
\end{itemize}

\textbf{Usage Example:}
\begin{lstlisting}[language=Python, caption=DomainInfo Usage]
# Create a domain from (0,0) to (1,1)
domain = DomainInfo(
    domain_id=0,
    extrema_start=(0.0, 0.0),
    extrema_end=(1.0, 1.0),
    name="diagonal_segment"
)
# domain_length will be automatically set to sqrt(2)
\end{lstlisting}

\paragraph{euclidean\_length()}
\begin{lstlisting}[language=Python, caption=Euclidean Length Calculation]
def euclidean_length(self) -> float
\end{lstlisting}

\textbf{Returns:} \texttt{float} - Euclidean distance between \texttt{extrema\_start} and \texttt{extrema\_end}

\textbf{Formula:} $\sqrt{(x_2-x_1)^2 + (y_2-y_1)^2}$

\textbf{Usage:}
\begin{lstlisting}[language=Python]
length = domain.euclidean_length()  # Returns geometric length
\end{lstlisting}

\paragraph{center\_point()}
\begin{lstlisting}[language=Python, caption=Center Point Calculation]
def center_point(self) -> Tuple[float, float]
\end{lstlisting}

\textbf{Returns:} \texttt{Tuple[float, float]} - Midpoint coordinates $(x_c, y_c)$

\textbf{Formula:} $x_c = \frac{x_1 + x_2}{2}$, $y_c = \frac{y_1 + y_2}{2}$

\textbf{Usage:}
\begin{lstlisting}[language=Python]
center = domain.center_point()  # Returns (x_center, y_center)
\end{lstlisting}

\paragraph{direction\_vector()}
\begin{lstlisting}[language=Python, caption=Direction Vector Calculation]
def direction_vector(self) -> Tuple[float, float]
\end{lstlisting}

\textbf{Returns:} \texttt{Tuple[float, float]} - Unit direction vector from start to end

\textbf{Formula:} $\vec{u} = \frac{(x_2-x_1, y_2-y_1)}{||(x_2-x_1, y_2-y_1)||}$

\textbf{Usage:}
\begin{lstlisting}[language=Python]
direction = domain.direction_vector()  # Returns (u_x, u_y)
\end{lstlisting}

\subsection{DomainGeometry Class}
\label{subsec:domaingeometry_class}

Main geometry manager class for handling collections of domains and providing interface methods for BioNetFlux integration.

\subsubsection{Attributes}

\begin{longtable}{|p{3cm}|p{2.5cm}|p{7cm}|}
\hline
\textbf{Attribute} & \textbf{Type} & \textbf{Description} \\
\hline
\endhead

\texttt{name} & \texttt{str} & Descriptive name for the geometry \\
\hline

\texttt{domains} & \texttt{List[DomainInfo]} & Collection of domain information objects \\
\hline

\texttt{\_next\_id} & \texttt{int} & Internal counter for generating unique domain IDs \\
\hline

\texttt{\_global\_metadata} & \texttt{Dict[str, Any]} & Geometry-wide metadata storage \\
\hline

\end{longtable}

\subsubsection{Constructor and Basic Operations}

\paragraph{Constructor}
\begin{lstlisting}[language=Python, caption=DomainGeometry Constructor]
def __init__(self, name: str = "unnamed_geometry")
\end{lstlisting}

\textbf{Parameters:}
\begin{itemize}
    \item \texttt{name}: Descriptive name for the geometry (optional, default: "unnamed\_geometry")
\end{itemize}

\textbf{Usage:}
\begin{lstlisting}[language=Python]
# Create new geometry
geometry = DomainGeometry(name="vascular_network")
\end{lstlisting}

\paragraph{add\_domain()}
\begin{lstlisting}[language=Python, caption=Add Domain Method]
def add_domain(self, 
               extrema_start: Tuple[float, float],
               extrema_end: Tuple[float, float],
               domain_start: Optional[float] = None,
               domain_length: Optional[float] = None,
               name: Optional[str] = None,
               **metadata) -> int
\end{lstlisting}

\textbf{Parameters:}
\begin{itemize}
    \item \texttt{extrema\_start}: Start point (x₁, y₁) in physical space
    \item \texttt{extrema\_end}: End point (x₂, y₂) in physical space
    \item \texttt{domain\_start}: Parameter space start (optional, default: 0.0)
    \item \texttt{domain\_length}: Parameter space length (optional, default: Euclidean distance)
    \item \texttt{name}: Domain name (optional, auto-generated if None)
    \item \texttt{**metadata}: Additional domain-specific data
\end{itemize}

\textbf{Returns:} \texttt{int} - Domain ID of the newly added domain

\textbf{Usage:}
\begin{lstlisting}[language=Python, caption=Adding Domains Example]
# Add horizontal segment
domain_id_1 = geometry.add_domain(
    extrema_start=(0.0, 0.0),
    extrema_end=(1.0, 0.0),
    name="main_vessel",
    vessel_type="artery",
    diameter=0.1
)

# Add vertical segment
domain_id_2 = geometry.add_domain(
    extrema_start=(1.0, 0.0),
    extrema_end=(1.0, 1.0),
    name="branch_vessel"
)
\end{lstlisting}

\subsubsection{Domain Access and Query Methods}

\paragraph{get\_domain()}
\begin{lstlisting}[language=Python, caption=Get Domain Method]
def get_domain(self, domain_id: int) -> DomainInfo
\end{lstlisting}

\textbf{Parameters:}
\begin{itemize}
    \item \texttt{domain\_id}: Domain index to retrieve
\end{itemize}

\textbf{Returns:} \texttt{DomainInfo} - Complete domain information object

\textbf{Raises:} \texttt{IndexError} - If domain\_id is out of range

\textbf{Usage:}
\begin{lstlisting}[language=Python]
domain = geometry.get_domain(0)  # Get first domain
print(f"Domain name: {domain.name}")
print(f"Length: {domain.euclidean_length()}")
\end{lstlisting}

\paragraph{get\_all\_domains()}
\begin{lstlisting}[language=Python, caption=Get All Domains Method]
def get_all_domains(self) -> List[DomainInfo]
\end{lstlisting}

\textbf{Returns:} \texttt{List[DomainInfo]} - Copy of all domains in the geometry

\textbf{Usage:}
\begin{lstlisting}[language=Python]
all_domains = geometry.get_all_domains()
for domain in all_domains:
    print(f"Domain {domain.domain_id}: {domain.name}")
\end{lstlisting}

\paragraph{num\_domains()}
\begin{lstlisting}[language=Python, caption=Number of Domains Method]
def num_domains(self) -> int
\end{lstlisting}

\textbf{Returns:} \texttt{int} - Total number of domains in the geometry

\textbf{Usage:}
\begin{lstlisting}[language=Python]
n_domains = geometry.num_domains()
print(f"Geometry contains {n_domains} domains")
\end{lstlisting}

\paragraph{find\_domain\_by\_name()}
\begin{lstlisting}[language=Python, caption=Find Domain by Name Method]
def find_domain_by_name(self, name: str) -> Optional[int]
\end{lstlisting}

\textbf{Parameters:}
\begin{itemize}
    \item \texttt{name}: Domain name to search for
\end{itemize}

\textbf{Returns:} \texttt{Optional[int]} - Domain ID if found, None otherwise

\textbf{Usage:}
\begin{lstlisting}[language=Python]
domain_id = geometry.find_domain_by_name("main_vessel")
if domain_id is not None:
    domain = geometry.get_domain(domain_id)
\end{lstlisting}

\paragraph{get\_domain\_names()}
\begin{lstlisting}[language=Python, caption=Get Domain Names Method]
def get_domain_names(self) -> List[str]
\end{lstlisting}

\textbf{Returns:} \texttt{List[str]} - List of all domain names

\textbf{Usage:}
\begin{lstlisting}[language=Python]
names = geometry.get_domain_names()
print("Available domains:", ", ".join(names))
\end{lstlisting}

\subsubsection{Geometric Analysis Methods}

\paragraph{get\_bounding\_box()}
\begin{lstlisting}[language=Python, caption=Bounding Box Calculation]
def get_bounding_box(self) -> Dict[str, float]
\end{lstlisting}

\textbf{Returns:} \texttt{Dict[str, float]} - Dictionary with keys: \texttt{x\_min}, \texttt{x\_max}, \texttt{y\_min}, \texttt{y\_max}

\textbf{Usage:}
\begin{lstlisting}[language=Python]
bbox = geometry.get_bounding_box()
width = bbox['x_max'] - bbox['x_min']
height = bbox['y_max'] - bbox['y_min']
print(f"Geometry bounding box: {width} × {height}")
\end{lstlisting}

\paragraph{total\_length()}
\begin{lstlisting}[language=Python, caption=Total Length Calculation]
def total_length(self) -> float
\end{lstlisting}

\textbf{Returns:} \texttt{float} - Sum of Euclidean lengths of all domains

\textbf{Usage:}
\begin{lstlisting}[language=Python]
total_len = geometry.total_length()
avg_len = total_len / geometry.num_domains()
print(f"Total network length: {total_len:.3f}")
print(f"Average segment length: {avg_len:.3f}")
\end{lstlisting}

\subsubsection{Metadata Management}

\paragraph{set\_global\_metadata()}
\begin{lstlisting}[language=Python, caption=Set Global Metadata Method]
def set_global_metadata(self, **metadata)
\end{lstlisting}

\textbf{Parameters:}
\begin{itemize}
    \item \texttt{**metadata}: Key-value pairs for geometry-wide metadata
\end{itemize}

\textbf{Usage:}
\begin{lstlisting}[language=Python]
geometry.set_global_metadata(
    problem_type="organ_on_chip",
    fluid_viscosity=0.001,
    temperature=37.0,
    units="mm"
)
\end{lstlisting}

\paragraph{get\_global\_metadata()}
\begin{lstlisting}[language=Python, caption=Get Global Metadata Method]
def get_global_metadata(self) -> Dict[str, Any]
\end{lstlisting}

\textbf{Returns:} \texttt{Dict[str, Any]} - Copy of all global metadata

\textbf{Usage:}
\begin{lstlisting}[language=Python]
metadata = geometry.get_global_metadata()
if "fluid_viscosity" in metadata:
    viscosity = metadata["fluid_viscosity"]
\end{lstlisting}

\subsubsection{Utility and Maintenance Methods}

\paragraph{remove\_domain()}
\begin{lstlisting}[language=Python, caption=Remove Domain Method]
def remove_domain(self, domain_id: int)
\end{lstlisting}

\textbf{Parameters:}
\begin{itemize}
    \item \texttt{domain\_id}: Domain ID to remove
\end{itemize}

\textbf{Raises:} \texttt{IndexError} - If domain\_id is invalid

\textbf{Note:} Automatically renumbers remaining domains to maintain consistency

\textbf{Usage:}
\begin{lstlisting}[language=Python]
# Remove domain and automatically renumber others
geometry.remove_domain(1)
\end{lstlisting}

\paragraph{summary()}
\begin{lstlisting}[language=Python, caption=Summary Generation Method]
def summary(self) -> str
\end{lstlisting}

\textbf{Returns:} \texttt{str} - Multi-line summary of geometry contents

\textbf{Usage:}
\begin{lstlisting}[language=Python]
print(geometry.summary())
# Output:
# Geometry: vascular_network
# Number of domains: 3
# Total length: 4.236
# Domains:
#   0: main_vessel
#     Extrema: (0.0, 0.0) → (1.0, 0.0)
#     Parameter: [0.000, 1.000]
#     Length: 1.000
# ...
\end{lstlisting}

\subsubsection{Special Methods (Python Magic Methods)}

\paragraph{\_\_len\_\_()}
\begin{lstlisting}[language=Python, caption=Length Support]
def __len__(self) -> int
\end{lstlisting}

\textbf{Returns:} \texttt{int} - Number of domains (enables \texttt{len(geometry)})

\textbf{Usage:}
\begin{lstlisting}[language=Python]
num_domains = len(geometry)  # Equivalent to geometry.num_domains()
\end{lstlisting}

\paragraph{\_\_getitem\_\_()}
\begin{lstlisting}[language=Python, caption=Indexing Support]
def __getitem__(self, domain_id: int) -> DomainInfo
\end{lstlisting}

\textbf{Parameters:}
\begin{itemize}
    \item \texttt{domain\_id}: Domain index
\end{itemize}

\textbf{Returns:} \texttt{DomainInfo} - Domain at specified index (enables \texttt{geometry[i]})

\textbf{Usage:}
\begin{lstlisting}[language=Python]
first_domain = geometry[0]  # Equivalent to geometry.get_domain(0)
\end{lstlisting}

\paragraph{\_\_iter\_\_()}
\begin{lstlisting}[language=Python, caption=Iteration Support]
def __iter__(self)
\end{lstlisting}

\textbf{Returns:} Iterator over all \texttt{DomainInfo} objects

\textbf{Usage:}
\begin{lstlisting}[language=Python]
for domain in geometry:
    print(f"Processing domain {domain.domain_id}: {domain.name}")
    length = domain.euclidean_length()
    center = domain.center_point()
\end{lstlisting}

\subsection{Complete Usage Example}
\label{subsec:complete_example}

\begin{lstlisting}[language=Python, caption=Complete Geometry Usage Example]
from ooc1d.geometry.domain_geometry import DomainGeometry

# Create geometry for Y-junction network
geometry = DomainGeometry(name="y_junction_network")

# Add main vessel (horizontal segment)
main_id = geometry.add_domain(
    extrema_start=(0.0, 0.0),
    extrema_end=(2.0, 0.0),
    name="main_vessel",
    vessel_type="parent",
    diameter=1.0
)

# Add upper branch
upper_id = geometry.add_domain(
    extrema_start=(2.0, 0.0),
    extrema_end=(3.0, 1.0),
    name="upper_branch",
    vessel_type="daughter",
    diameter=0.7
)

# Add lower branch  
lower_id = geometry.add_domain(
    extrema_start=(2.0, 0.0),
    extrema_end=(3.0, -1.0),
    name="lower_branch",
    vessel_type="daughter",
    diameter=0.7
)

# Set global properties
geometry.set_global_metadata(
    fluid_type="blood",
    viscosity=0.004,  # Pa·s
    density=1060,     # kg/m³
    problem_type="organ_on_chip"
)

# Analyze geometry
print(geometry.summary())
print(f"\nTotal network length: {geometry.total_length():.3f}")

# Access individual domains
for domain in geometry:
    center = domain.center_point()
    direction = domain.direction_vector()
    print(f"Domain {domain.name}:")
    print(f"  Center: ({center[0]:.2f}, {center[1]:.2f})")
    print(f"  Direction: ({direction[0]:.2f}, {direction[1]:.2f})")
    
    # Access metadata
    if "diameter" in domain.metadata:
        print(f"  Diameter: {domain.metadata['diameter']}")

# Find specific domain
main_domain_id = geometry.find_domain_by_name("main_vessel")
if main_domain_id is not None:
    main_domain = geometry[main_domain_id]
    print(f"\nMain vessel length: {main_domain.euclidean_length():.3f}")
\end{lstlisting}

\subsection{Integration with BioNetFlux}
\label{subsec:bionetflux_integration}

The \texttt{DomainGeometry} class is designed for seamless integration with BioNetFlux components:

\begin{lstlisting}[language=Python, caption=BioNetFlux Integration Example]
from ooc1d.core.problem import Problem
from ooc1d.core.discretization import Discretization, GlobalDiscretization

# Create BioNetFlux problems from geometry
def geometry_to_problems(geometry, neq=4, parameters=None):
    """Convert geometry domains to BioNetFlux Problem instances."""
    problems = []
    
    for domain in geometry:
        problem = Problem(
            neq=neq,
            domain_start=domain.domain_start,
            domain_length=domain.domain_length,
            parameters=parameters or np.ones(9),  # OrganOnChip parameters
            problem_type="organ_on_chip",
            name=domain.name
        )
        problems.append(problem)
    
    return problems

# Create discretizations from geometry
def geometry_to_discretizations(geometry, n_elements=10):
    """Convert geometry domains to BioNetFlux Discretization instances."""
    discretizations = []
    
    for domain in geometry:
        disc = Discretization(
            n_elements=n_elements,
            domain_start=domain.domain_start,
            domain_length=domain.domain_length,
            stab_constant=1.0
        )
        discretizations.append(disc)
    
    return GlobalDiscretization(discretizations)

# Usage
problems = geometry_to_problems(geometry)
global_disc = geometry_to_discretizations(geometry)
\end{lstlisting}

This API provides a comprehensive interface for managing complex multi-domain geometries while maintaining compatibility with the existing BioNetFlux solver framework.

% End of domain geometry API documentation


\section{Domain Geometry Module API Reference}
\label{sec:domain_geometry_api}

This section provides a comprehensive reference for the domain geometry module (\texttt{domain\_geometry.py}), which provides lean geometry management for multi-domain BioNetFlux problems.

\subsection{Module Overview}

The module contains two main classes:
\begin{itemize}
    \item \texttt{DomainInfo}: Data container for individual domain properties
    \item \texttt{DomainGeometry}: Main geometry manager for collections of domains
\end{itemize}

\subsection{DomainInfo Class}
\label{subsec:domaininfo_class}

A dataclass container for domain geometric information, representing individual segments in the network geometry.

\subsubsection{Attributes}

\begin{longtable}{|p{3cm}|p{3.5cm}|p{6cm}|}
\hline
\textbf{Attribute} & \textbf{Type} & \textbf{Description} \\
\hline
\endhead

\texttt{domain\_id} & \texttt{int} & Unique identifier for the domain within the geometry \\
\hline

\texttt{extrema\_start} & \texttt{Tuple[float, float]} & Physical coordinates (x₁, y₁) of domain start point \\
\hline

\texttt{extrema\_end} & \texttt{Tuple[float, float]} & Physical coordinates (x₂, y₂) of domain end point \\
\hline

\texttt{domain\_start} & \texttt{float} & Parameter space start coordinate (default: 0.0) \\
\hline

\texttt{domain\_length} & \texttt{float} & Parameter space length (default: Euclidean distance) \\
\hline

\texttt{name} & \texttt{Optional[str]} & Human-readable name for the domain \\
\hline

\texttt{metadata} & \texttt{Dict[str, Any]} & Additional domain-specific data storage \\
\hline

\end{longtable}

\subsubsection{Methods}

\paragraph{Constructor}
\begin{lstlisting}[language=Python, caption=DomainInfo Constructor]
def __init__(self, domain_id: int, 
             extrema_start: Tuple[float, float],
             extrema_end: Tuple[float, float],
             domain_start: float = 0.0,
             domain_length: float = 1.0,
             name: Optional[str] = None,
             metadata: Dict[str, Any] = None)
\end{lstlisting}

\textbf{Parameters:}
\begin{itemize}
    \item \texttt{domain\_id}: Unique domain identifier
    \item \texttt{extrema\_start}: Start point (x₁, y₁) in physical coordinates
    \item \texttt{extrema\_end}: End point (x₂, y₂) in physical coordinates
    \item \texttt{domain\_start}: Parameter space origin (optional, default: 0.0)
    \item \texttt{domain\_length}: Parameter space length (optional, default: calculated)
    \item \texttt{name}: Domain name (optional)
    \item \texttt{metadata}: Additional properties (optional)
\end{itemize}

\textbf{Usage Example:}
\begin{lstlisting}[language=Python, caption=DomainInfo Usage]
# Create a domain from (0,0) to (1,1)
domain = DomainInfo(
    domain_id=0,
    extrema_start=(0.0, 0.0),
    extrema_end=(1.0, 1.0),
    name="diagonal_segment"
)
# domain_length will be automatically set to sqrt(2)
\end{lstlisting}

\paragraph{euclidean\_length()}
\begin{lstlisting}[language=Python, caption=Euclidean Length Calculation]
def euclidean_length(self) -> float
\end{lstlisting}

\textbf{Returns:} \texttt{float} - Euclidean distance between \texttt{extrema\_start} and \texttt{extrema\_end}

\textbf{Formula:} $\sqrt{(x_2-x_1)^2 + (y_2-y_1)^2}$

\textbf{Usage:}
\begin{lstlisting}[language=Python]
length = domain.euclidean_length()  # Returns geometric length
\end{lstlisting}

\paragraph{center\_point()}
\begin{lstlisting}[language=Python, caption=Center Point Calculation]
def center_point(self) -> Tuple[float, float]
\end{lstlisting}

\textbf{Returns:} \texttt{Tuple[float, float]} - Midpoint coordinates $(x_c, y_c)$

\textbf{Formula:} $x_c = \frac{x_1 + x_2}{2}$, $y_c = \frac{y_1 + y_2}{2}$

\textbf{Usage:}
\begin{lstlisting}[language=Python]
center = domain.center_point()  # Returns (x_center, y_center)
\end{lstlisting}

\paragraph{direction\_vector()}
\begin{lstlisting}[language=Python, caption=Direction Vector Calculation]
def direction_vector(self) -> Tuple[float, float]
\end{lstlisting}

\textbf{Returns:} \texttt{Tuple[float, float]} - Unit direction vector from start to end

\textbf{Formula:} $\vec{u} = \frac{(x_2-x_1, y_2-y_1)}{||(x_2-x_1, y_2-y_1)||}$

\textbf{Usage:}
\begin{lstlisting}[language=Python]
direction = domain.direction_vector()  # Returns (u_x, u_y)
\end{lstlisting}

\subsection{DomainGeometry Class}
\label{subsec:domaingeometry_class}

Main geometry manager class for handling collections of domains and providing interface methods for BioNetFlux integration.

\subsubsection{Attributes}

\begin{longtable}{|p{3cm}|p{2.5cm}|p{7cm}|}
\hline
\textbf{Attribute} & \textbf{Type} & \textbf{Description} \\
\hline
\endhead

\texttt{name} & \texttt{str} & Descriptive name for the geometry \\
\hline

\texttt{domains} & \texttt{List[DomainInfo]} & Collection of domain information objects \\
\hline

\texttt{\_next\_id} & \texttt{int} & Internal counter for generating unique domain IDs \\
\hline

\texttt{\_global\_metadata} & \texttt{Dict[str, Any]} & Geometry-wide metadata storage \\
\hline

\end{longtable}

\subsubsection{Constructor and Basic Operations}

\paragraph{Constructor}
\begin{lstlisting}[language=Python, caption=DomainGeometry Constructor]
def __init__(self, name: str = "unnamed_geometry")
\end{lstlisting}

\textbf{Parameters:}
\begin{itemize}
    \item \texttt{name}: Descriptive name for the geometry (optional, default: "unnamed\_geometry")
\end{itemize}

\textbf{Usage:}
\begin{lstlisting}[language=Python]
# Create new geometry
geometry = DomainGeometry(name="vascular_network")
\end{lstlisting}

\paragraph{add\_domain()}
\begin{lstlisting}[language=Python, caption=Add Domain Method]
def add_domain(self, 
               extrema_start: Tuple[float, float],
               extrema_end: Tuple[float, float],
               domain_start: Optional[float] = None,
               domain_length: Optional[float] = None,
               name: Optional[str] = None,
               **metadata) -> int
\end{lstlisting}

\textbf{Parameters:}
\begin{itemize}
    \item \texttt{extrema\_start}: Start point (x₁, y₁) in physical space
    \item \texttt{extrema\_end}: End point (x₂, y₂) in physical space
    \item \texttt{domain\_start}: Parameter space start (optional, default: 0.0)
    \item \texttt{domain\_length}: Parameter space length (optional, default: Euclidean distance)
    \item \texttt{name}: Domain name (optional, auto-generated if None)
    \item \texttt{**metadata}: Additional domain-specific data
\end{itemize}

\textbf{Returns:} \texttt{int} - Domain ID of the newly added domain

\textbf{Usage:}
\begin{lstlisting}[language=Python, caption=Adding Domains Example]
# Add horizontal segment
domain_id_1 = geometry.add_domain(
    extrema_start=(0.0, 0.0),
    extrema_end=(1.0, 0.0),
    name="main_vessel",
    vessel_type="artery",
    diameter=0.1
)

# Add vertical segment
domain_id_2 = geometry.add_domain(
    extrema_start=(1.0, 0.0),
    extrema_end=(1.0, 1.0),
    name="branch_vessel"
)
\end{lstlisting}

\subsubsection{Domain Access and Query Methods}

\paragraph{get\_domain()}
\begin{lstlisting}[language=Python, caption=Get Domain Method]
def get_domain(self, domain_id: int) -> DomainInfo
\end{lstlisting}

\textbf{Parameters:}
\begin{itemize}
    \item \texttt{domain\_id}: Domain index to retrieve
\end{itemize}

\textbf{Returns:} \texttt{DomainInfo} - Complete domain information object

\textbf{Raises:} \texttt{IndexError} - If domain\_id is out of range

\textbf{Usage:}
\begin{lstlisting}[language=Python]
domain = geometry.get_domain(0)  # Get first domain
print(f"Domain name: {domain.name}")
print(f"Length: {domain.euclidean_length()}")
\end{lstlisting}

\paragraph{get\_all\_domains()}
\begin{lstlisting}[language=Python, caption=Get All Domains Method]
def get_all_domains(self) -> List[DomainInfo]
\end{lstlisting}

\textbf{Returns:} \texttt{List[DomainInfo]} - Copy of all domains in the geometry

\textbf{Usage:}
\begin{lstlisting}[language=Python]
all_domains = geometry.get_all_domains()
for domain in all_domains:
    print(f"Domain {domain.domain_id}: {domain.name}")
\end{lstlisting}

\paragraph{num\_domains()}
\begin{lstlisting}[language=Python, caption=Number of Domains Method]
def num_domains(self) -> int
\end{lstlisting}

\textbf{Returns:} \texttt{int} - Total number of domains in the geometry

\textbf{Usage:}
\begin{lstlisting}[language=Python]
n_domains = geometry.num_domains()
print(f"Geometry contains {n_domains} domains")
\end{lstlisting}

\paragraph{find\_domain\_by\_name()}
\begin{lstlisting}[language=Python, caption=Find Domain by Name Method]
def find_domain_by_name(self, name: str) -> Optional[int]
\end{lstlisting}

\textbf{Parameters:}
\begin{itemize}
    \item \texttt{name}: Domain name to search for
\end{itemize}

\textbf{Returns:} \texttt{Optional[int]} - Domain ID if found, None otherwise

\textbf{Usage:}
\begin{lstlisting}[language=Python]
domain_id = geometry.find_domain_by_name("main_vessel")
if domain_id is not None:
    domain = geometry.get_domain(domain_id)
\end{lstlisting}

\paragraph{get\_domain\_names()}
\begin{lstlisting}[language=Python, caption=Get Domain Names Method]
def get_domain_names(self) -> List[str]
\end{lstlisting}

\textbf{Returns:} \texttt{List[str]} - List of all domain names

\textbf{Usage:}
\begin{lstlisting}[language=Python]
names = geometry.get_domain_names()
print("Available domains:", ", ".join(names))
\end{lstlisting}

\subsubsection{Geometric Analysis Methods}

\paragraph{get\_bounding\_box()}
\begin{lstlisting}[language=Python, caption=Bounding Box Calculation]
def get_bounding_box(self) -> Dict[str, float]
\end{lstlisting}

\textbf{Returns:} \texttt{Dict[str, float]} - Dictionary with keys: \texttt{x\_min}, \texttt{x\_max}, \texttt{y\_min}, \texttt{y\_max}

\textbf{Usage:}
\begin{lstlisting}[language=Python]
bbox = geometry.get_bounding_box()
width = bbox['x_max'] - bbox['x_min']
height = bbox['y_max'] - bbox['y_min']
print(f"Geometry bounding box: {width} × {height}")
\end{lstlisting}

\paragraph{total\_length()}
\begin{lstlisting}[language=Python, caption=Total Length Calculation]
def total_length(self) -> float
\end{lstlisting}

\textbf{Returns:} \texttt{float} - Sum of Euclidean lengths of all domains

\textbf{Usage:}
\begin{lstlisting}[language=Python]
total_len = geometry.total_length()
avg_len = total_len / geometry.num_domains()
print(f"Total network length: {total_len:.3f}")
print(f"Average segment length: {avg_len:.3f}")
\end{lstlisting}

\subsubsection{Metadata Management}

\paragraph{set\_global\_metadata()}
\begin{lstlisting}[language=Python, caption=Set Global Metadata Method]
def set_global_metadata(self, **metadata)
\end{lstlisting}

\textbf{Parameters:}
\begin{itemize}
    \item \texttt{**metadata}: Key-value pairs for geometry-wide metadata
\end{itemize}

\textbf{Usage:}
\begin{lstlisting}[language=Python]
geometry.set_global_metadata(
    problem_type="organ_on_chip",
    fluid_viscosity=0.001,
    temperature=37.0,
    units="mm"
)
\end{lstlisting}

\paragraph{get\_global\_metadata()}
\begin{lstlisting}[language=Python, caption=Get Global Metadata Method]
def get_global_metadata(self) -> Dict[str, Any]
\end{lstlisting}

\textbf{Returns:} \texttt{Dict[str, Any]} - Copy of all global metadata

\textbf{Usage:}
\begin{lstlisting}[language=Python]
metadata = geometry.get_global_metadata()
if "fluid_viscosity" in metadata:
    viscosity = metadata["fluid_viscosity"]
\end{lstlisting}

\subsubsection{Utility and Maintenance Methods}

\paragraph{remove\_domain()}
\begin{lstlisting}[language=Python, caption=Remove Domain Method]
def remove_domain(self, domain_id: int)
\end{lstlisting}

\textbf{Parameters:}
\begin{itemize}
    \item \texttt{domain\_id}: Domain ID to remove
\end{itemize}

\textbf{Raises:} \texttt{IndexError} - If domain\_id is invalid

\textbf{Note:} Automatically renumbers remaining domains to maintain consistency

\textbf{Usage:}
\begin{lstlisting}[language=Python]
# Remove domain and automatically renumber others
geometry.remove_domain(1)
\end{lstlisting}

\paragraph{summary()}
\begin{lstlisting}[language=Python, caption=Summary Generation Method]
def summary(self) -> str
\end{lstlisting}

\textbf{Returns:} \texttt{str} - Multi-line summary of geometry contents

\textbf{Usage:}
\begin{lstlisting}[language=Python]
print(geometry.summary())
# Output:
# Geometry: vascular_network
# Number of domains: 3
# Total length: 4.236
# Domains:
#   0: main_vessel
#     Extrema: (0.0, 0.0) → (1.0, 0.0)
#     Parameter: [0.000, 1.000]
#     Length: 1.000
# ...
\end{lstlisting}

\subsubsection{Special Methods (Python Magic Methods)}

\paragraph{\_\_len\_\_()}
\begin{lstlisting}[language=Python, caption=Length Support]
def __len__(self) -> int
\end{lstlisting}

\textbf{Returns:} \texttt{int} - Number of domains (enables \texttt{len(geometry)})

\textbf{Usage:}
\begin{lstlisting}[language=Python]
num_domains = len(geometry)  # Equivalent to geometry.num_domains()
\end{lstlisting}

\paragraph{\_\_getitem\_\_()}
\begin{lstlisting}[language=Python, caption=Indexing Support]
def __getitem__(self, domain_id: int) -> DomainInfo
\end{lstlisting}

\textbf{Parameters:}
\begin{itemize}
    \item \texttt{domain\_id}: Domain index
\end{itemize}

\textbf{Returns:} \texttt{DomainInfo} - Domain at specified index (enables \texttt{geometry[i]})

\textbf{Usage:}
\begin{lstlisting}[language=Python]
first_domain = geometry[0]  # Equivalent to geometry.get_domain(0)
\end{lstlisting}

\paragraph{\_\_iter\_\_()}
\begin{lstlisting}[language=Python, caption=Iteration Support]
def __iter__(self)
\end{lstlisting}

\textbf{Returns:} Iterator over all \texttt{DomainInfo} objects

\textbf{Usage:}
\begin{lstlisting}[language=Python]
for domain in geometry:
    print(f"Processing domain {domain.domain_id}: {domain.name}")
    length = domain.euclidean_length()
    center = domain.center_point()
\end{lstlisting}

\subsection{Complete Usage Example}
\label{subsec:complete_example}

\begin{lstlisting}[language=Python, caption=Complete Geometry Usage Example]
from ooc1d.geometry.domain_geometry import DomainGeometry

# Create geometry for Y-junction network
geometry = DomainGeometry(name="y_junction_network")

# Add main vessel (horizontal segment)
main_id = geometry.add_domain(
    extrema_start=(0.0, 0.0),
    extrema_end=(2.0, 0.0),
    name="main_vessel",
    vessel_type="parent",
    diameter=1.0
)

# Add upper branch
upper_id = geometry.add_domain(
    extrema_start=(2.0, 0.0),
    extrema_end=(3.0, 1.0),
    name="upper_branch",
    vessel_type="daughter",
    diameter=0.7
)

# Add lower branch  
lower_id = geometry.add_domain(
    extrema_start=(2.0, 0.0),
    extrema_end=(3.0, -1.0),
    name="lower_branch",
    vessel_type="daughter",
    diameter=0.7
)

# Set global properties
geometry.set_global_metadata(
    fluid_type="blood",
    viscosity=0.004,  # Pa·s
    density=1060,     # kg/m³
    problem_type="organ_on_chip"
)

# Analyze geometry
print(geometry.summary())
print(f"\nTotal network length: {geometry.total_length():.3f}")

# Access individual domains
for domain in geometry:
    center = domain.center_point()
    direction = domain.direction_vector()
    print(f"Domain {domain.name}:")
    print(f"  Center: ({center[0]:.2f}, {center[1]:.2f})")
    print(f"  Direction: ({direction[0]:.2f}, {direction[1]:.2f})")
    
    # Access metadata
    if "diameter" in domain.metadata:
        print(f"  Diameter: {domain.metadata['diameter']}")

# Find specific domain
main_domain_id = geometry.find_domain_by_name("main_vessel")
if main_domain_id is not None:
    main_domain = geometry[main_domain_id]
    print(f"\nMain vessel length: {main_domain.euclidean_length():.3f}")
\end{lstlisting}

\subsection{Integration with BioNetFlux}
\label{subsec:bionetflux_integration}

The \texttt{DomainGeometry} class is designed for seamless integration with BioNetFlux components:

\begin{lstlisting}[language=Python, caption=BioNetFlux Integration Example]
from ooc1d.core.problem import Problem
from ooc1d.core.discretization import Discretization, GlobalDiscretization

# Create BioNetFlux problems from geometry
def geometry_to_problems(geometry, neq=4, parameters=None):
    """Convert geometry domains to BioNetFlux Problem instances."""
    problems = []
    
    for domain in geometry:
        problem = Problem(
            neq=neq,
            domain_start=domain.domain_start,
            domain_length=domain.domain_length,
            parameters=parameters or np.ones(9),  # OrganOnChip parameters
            problem_type="organ_on_chip",
            name=domain.name
        )
        problems.append(problem)
    
    return problems

# Create discretizations from geometry
def geometry_to_discretizations(geometry, n_elements=10):
    """Convert geometry domains to BioNetFlux Discretization instances."""
    discretizations = []
    
    for domain in geometry:
        disc = Discretization(
            n_elements=n_elements,
            domain_start=domain.domain_start,
            domain_length=domain.domain_length,
            stab_constant=1.0
        )
        discretizations.append(disc)
    
    return GlobalDiscretization(discretizations)

# Usage
problems = geometry_to_problems(geometry)
global_disc = geometry_to_discretizations(geometry)
\end{lstlisting}

This API provides a comprehensive interface for managing complex multi-domain geometries while maintaining compatibility with the existing BioNetFlux solver framework.

% End of domain geometry API documentation


\section{Domain Geometry Module API Reference}
\label{sec:domain_geometry_api}

This section provides a comprehensive reference for the domain geometry module (\texttt{domain\_geometry.py}), which provides lean geometry management for multi-domain BioNetFlux problems.

\subsection{Module Overview}

The module contains two main classes:
\begin{itemize}
    \item \texttt{DomainInfo}: Data container for individual domain properties
    \item \texttt{DomainGeometry}: Main geometry manager for collections of domains
\end{itemize}

\subsection{DomainInfo Class}
\label{subsec:domaininfo_class}

A dataclass container for domain geometric information, representing individual segments in the network geometry.

\subsubsection{Attributes}

\begin{longtable}{|p{3cm}|p{3.5cm}|p{6cm}|}
\hline
\textbf{Attribute} & \textbf{Type} & \textbf{Description} \\
\hline
\endhead

\texttt{domain\_id} & \texttt{int} & Unique identifier for the domain within the geometry \\
\hline

\texttt{extrema\_start} & \texttt{Tuple[float, float]} & Physical coordinates (x₁, y₁) of domain start point \\
\hline

\texttt{extrema\_end} & \texttt{Tuple[float, float]} & Physical coordinates (x₂, y₂) of domain end point \\
\hline

\texttt{domain\_start} & \texttt{float} & Parameter space start coordinate (default: 0.0) \\
\hline

\texttt{domain\_length} & \texttt{float} & Parameter space length (default: Euclidean distance) \\
\hline

\texttt{name} & \texttt{Optional[str]} & Human-readable name for the domain \\
\hline

\texttt{metadata} & \texttt{Dict[str, Any]} & Additional domain-specific data storage \\
\hline

\end{longtable}

\subsubsection{Methods}

\paragraph{Constructor}
\begin{lstlisting}[language=Python, caption=DomainInfo Constructor]
def __init__(self, domain_id: int, 
             extrema_start: Tuple[float, float],
             extrema_end: Tuple[float, float],
             domain_start: float = 0.0,
             domain_length: float = 1.0,
             name: Optional[str] = None,
             metadata: Dict[str, Any] = None)
\end{lstlisting}

\textbf{Parameters:}
\begin{itemize}
    \item \texttt{domain\_id}: Unique domain identifier
    \item \texttt{extrema\_start}: Start point (x₁, y₁) in physical coordinates
    \item \texttt{extrema\_end}: End point (x₂, y₂) in physical coordinates
    \item \texttt{domain\_start}: Parameter space origin (optional, default: 0.0)
    \item \texttt{domain\_length}: Parameter space length (optional, default: calculated)
    \item \texttt{name}: Domain name (optional)
    \item \texttt{metadata}: Additional properties (optional)
\end{itemize}

\textbf{Usage Example:}
\begin{lstlisting}[language=Python, caption=DomainInfo Usage]
# Create a domain from (0,0) to (1,1)
domain = DomainInfo(
    domain_id=0,
    extrema_start=(0.0, 0.0),
    extrema_end=(1.0, 1.0),
    name="diagonal_segment"
)
# domain_length will be automatically set to sqrt(2)
\end{lstlisting}

\paragraph{euclidean\_length()}
\begin{lstlisting}[language=Python, caption=Euclidean Length Calculation]
def euclidean_length(self) -> float
\end{lstlisting}

\textbf{Returns:} \texttt{float} - Euclidean distance between \texttt{extrema\_start} and \texttt{extrema\_end}

\textbf{Formula:} $\sqrt{(x_2-x_1)^2 + (y_2-y_1)^2}$

\textbf{Usage:}
\begin{lstlisting}[language=Python]
length = domain.euclidean_length()  # Returns geometric length
\end{lstlisting}

\paragraph{center\_point()}
\begin{lstlisting}[language=Python, caption=Center Point Calculation]
def center_point(self) -> Tuple[float, float]
\end{lstlisting}

\textbf{Returns:} \texttt{Tuple[float, float]} - Midpoint coordinates $(x_c, y_c)$

\textbf{Formula:} $x_c = \frac{x_1 + x_2}{2}$, $y_c = \frac{y_1 + y_2}{2}$

\textbf{Usage:}
\begin{lstlisting}[language=Python]
center = domain.center_point()  # Returns (x_center, y_center)
\end{lstlisting}

\paragraph{direction\_vector()}
\begin{lstlisting}[language=Python, caption=Direction Vector Calculation]
def direction_vector(self) -> Tuple[float, float]
\end{lstlisting}

\textbf{Returns:} \texttt{Tuple[float, float]} - Unit direction vector from start to end

\textbf{Formula:} $\vec{u} = \frac{(x_2-x_1, y_2-y_1)}{||(x_2-x_1, y_2-y_1)||}$

\textbf{Usage:}
\begin{lstlisting}[language=Python]
direction = domain.direction_vector()  # Returns (u_x, u_y)
\end{lstlisting}

\subsection{DomainGeometry Class}
\label{subsec:domaingeometry_class}

Main geometry manager class for handling collections of domains and providing interface methods for BioNetFlux integration.

\subsubsection{Attributes}

\begin{longtable}{|p{3cm}|p{2.5cm}|p{7cm}|}
\hline
\textbf{Attribute} & \textbf{Type} & \textbf{Description} \\
\hline
\endhead

\texttt{name} & \texttt{str} & Descriptive name for the geometry \\
\hline

\texttt{domains} & \texttt{List[DomainInfo]} & Collection of domain information objects \\
\hline

\texttt{\_next\_id} & \texttt{int} & Internal counter for generating unique domain IDs \\
\hline

\texttt{\_global\_metadata} & \texttt{Dict[str, Any]} & Geometry-wide metadata storage \\
\hline

\end{longtable}

\subsubsection{Constructor and Basic Operations}

\paragraph{Constructor}
\begin{lstlisting}[language=Python, caption=DomainGeometry Constructor]
def __init__(self, name: str = "unnamed_geometry")
\end{lstlisting}

\textbf{Parameters:}
\begin{itemize}
    \item \texttt{name}: Descriptive name for the geometry (optional, default: "unnamed\_geometry")
\end{itemize}

\textbf{Usage:}
\begin{lstlisting}[language=Python]
# Create new geometry
geometry = DomainGeometry(name="vascular_network")
\end{lstlisting}

\paragraph{add\_domain()}
\begin{lstlisting}[language=Python, caption=Add Domain Method]
def add_domain(self, 
               extrema_start: Tuple[float, float],
               extrema_end: Tuple[float, float],
               domain_start: Optional[float] = None,
               domain_length: Optional[float] = None,
               name: Optional[str] = None,
               **metadata) -> int
\end{lstlisting}

\textbf{Parameters:}
\begin{itemize}
    \item \texttt{extrema\_start}: Start point (x₁, y₁) in physical space
    \item \texttt{extrema\_end}: End point (x₂, y₂) in physical space
    \item \texttt{domain\_start}: Parameter space start (optional, default: 0.0)
    \item \texttt{domain\_length}: Parameter space length (optional, default: Euclidean distance)
    \item \texttt{name}: Domain name (optional, auto-generated if None)
    \item \texttt{**metadata}: Additional domain-specific data
\end{itemize}

\textbf{Returns:} \texttt{int} - Domain ID of the newly added domain

\textbf{Usage:}
\begin{lstlisting}[language=Python, caption=Adding Domains Example]
# Add horizontal segment
domain_id_1 = geometry.add_domain(
    extrema_start=(0.0, 0.0),
    extrema_end=(1.0, 0.0),
    name="main_vessel",
    vessel_type="artery",
    diameter=0.1
)

# Add vertical segment
domain_id_2 = geometry.add_domain(
    extrema_start=(1.0, 0.0),
    extrema_end=(1.0, 1.0),
    name="branch_vessel"
)
\end{lstlisting}

\subsubsection{Domain Access and Query Methods}

\paragraph{get\_domain()}
\begin{lstlisting}[language=Python, caption=Get Domain Method]
def get_domain(self, domain_id: int) -> DomainInfo
\end{lstlisting}

\textbf{Parameters:}
\begin{itemize}
    \item \texttt{domain\_id}: Domain index to retrieve
\end{itemize}

\textbf{Returns:} \texttt{DomainInfo} - Complete domain information object

\textbf{Raises:} \texttt{IndexError} - If domain\_id is out of range

\textbf{Usage:}
\begin{lstlisting}[language=Python]
domain = geometry.get_domain(0)  # Get first domain
print(f"Domain name: {domain.name}")
print(f"Length: {domain.euclidean_length()}")
\end{lstlisting}

\paragraph{get\_all\_domains()}
\begin{lstlisting}[language=Python, caption=Get All Domains Method]
def get_all_domains(self) -> List[DomainInfo]
\end{lstlisting}

\textbf{Returns:} \texttt{List[DomainInfo]} - Copy of all domains in the geometry

\textbf{Usage:}
\begin{lstlisting}[language=Python]
all_domains = geometry.get_all_domains()
for domain in all_domains:
    print(f"Domain {domain.domain_id}: {domain.name}")
\end{lstlisting}

\paragraph{num\_domains()}
\begin{lstlisting}[language=Python, caption=Number of Domains Method]
def num_domains(self) -> int
\end{lstlisting}

\textbf{Returns:} \texttt{int} - Total number of domains in the geometry

\textbf{Usage:}
\begin{lstlisting}[language=Python]
n_domains = geometry.num_domains()
print(f"Geometry contains {n_domains} domains")
\end{lstlisting}

\paragraph{find\_domain\_by\_name()}
\begin{lstlisting}[language=Python, caption=Find Domain by Name Method]
def find_domain_by_name(self, name: str) -> Optional[int]
\end{lstlisting}

\textbf{Parameters:}
\begin{itemize}
    \item \texttt{name}: Domain name to search for
\end{itemize}

\textbf{Returns:} \texttt{Optional[int]} - Domain ID if found, None otherwise

\textbf{Usage:}
\begin{lstlisting}[language=Python]
domain_id = geometry.find_domain_by_name("main_vessel")
if domain_id is not None:
    domain = geometry.get_domain(domain_id)
\end{lstlisting}

\paragraph{get\_domain\_names()}
\begin{lstlisting}[language=Python, caption=Get Domain Names Method]
def get_domain_names(self) -> List[str]
\end{lstlisting}

\textbf{Returns:} \texttt{List[str]} - List of all domain names

\textbf{Usage:}
\begin{lstlisting}[language=Python]
names = geometry.get_domain_names()
print("Available domains:", ", ".join(names))
\end{lstlisting}

\subsubsection{Geometric Analysis Methods}

\paragraph{get\_bounding\_box()}
\begin{lstlisting}[language=Python, caption=Bounding Box Calculation]
def get_bounding_box(self) -> Dict[str, float]
\end{lstlisting}

\textbf{Returns:} \texttt{Dict[str, float]} - Dictionary with keys: \texttt{x\_min}, \texttt{x\_max}, \texttt{y\_min}, \texttt{y\_max}

\textbf{Usage:}
\begin{lstlisting}[language=Python]
bbox = geometry.get_bounding_box()
width = bbox['x_max'] - bbox['x_min']
height = bbox['y_max'] - bbox['y_min']
print(f"Geometry bounding box: {width} × {height}")
\end{lstlisting}

\paragraph{total\_length()}
\begin{lstlisting}[language=Python, caption=Total Length Calculation]
def total_length(self) -> float
\end{lstlisting}

\textbf{Returns:} \texttt{float} - Sum of Euclidean lengths of all domains

\textbf{Usage:}
\begin{lstlisting}[language=Python]
total_len = geometry.total_length()
avg_len = total_len / geometry.num_domains()
print(f"Total network length: {total_len:.3f}")
print(f"Average segment length: {avg_len:.3f}")
\end{lstlisting}

\subsubsection{Metadata Management}

\paragraph{set\_global\_metadata()}
\begin{lstlisting}[language=Python, caption=Set Global Metadata Method]
def set_global_metadata(self, **metadata)
\end{lstlisting}

\textbf{Parameters:}
\begin{itemize}
    \item \texttt{**metadata}: Key-value pairs for geometry-wide metadata
\end{itemize}

\textbf{Usage:}
\begin{lstlisting}[language=Python]
geometry.set_global_metadata(
    problem_type="organ_on_chip",
    fluid_viscosity=0.001,
    temperature=37.0,
    units="mm"
)
\end{lstlisting}

\paragraph{get\_global\_metadata()}
\begin{lstlisting}[language=Python, caption=Get Global Metadata Method]
def get_global_metadata(self) -> Dict[str, Any]
\end{lstlisting}

\textbf{Returns:} \texttt{Dict[str, Any]} - Copy of all global metadata

\textbf{Usage:}
\begin{lstlisting}[language=Python]
metadata = geometry.get_global_metadata()
if "fluid_viscosity" in metadata:
    viscosity = metadata["fluid_viscosity"]
\end{lstlisting}

\subsubsection{Utility and Maintenance Methods}

\paragraph{remove\_domain()}
\begin{lstlisting}[language=Python, caption=Remove Domain Method]
def remove_domain(self, domain_id: int)
\end{lstlisting}

\textbf{Parameters:}
\begin{itemize}
    \item \texttt{domain\_id}: Domain ID to remove
\end{itemize}

\textbf{Raises:} \texttt{IndexError} - If domain\_id is invalid

\textbf{Note:} Automatically renumbers remaining domains to maintain consistency

\textbf{Usage:}
\begin{lstlisting}[language=Python]
# Remove domain and automatically renumber others
geometry.remove_domain(1)
\end{lstlisting}

\paragraph{summary()}
\begin{lstlisting}[language=Python, caption=Summary Generation Method]
def summary(self) -> str
\end{lstlisting}

\textbf{Returns:} \texttt{str} - Multi-line summary of geometry contents

\textbf{Usage:}
\begin{lstlisting}[language=Python]
print(geometry.summary())
# Output:
# Geometry: vascular_network
# Number of domains: 3
# Total length: 4.236
# Domains:
#   0: main_vessel
#     Extrema: (0.0, 0.0) → (1.0, 0.0)
#     Parameter: [0.000, 1.000]
#     Length: 1.000
# ...
\end{lstlisting}

\subsubsection{Special Methods (Python Magic Methods)}

\paragraph{\_\_len\_\_()}
\begin{lstlisting}[language=Python, caption=Length Support]
def __len__(self) -> int
\end{lstlisting}

\textbf{Returns:} \texttt{int} - Number of domains (enables \texttt{len(geometry)})

\textbf{Usage:}
\begin{lstlisting}[language=Python]
num_domains = len(geometry)  # Equivalent to geometry.num_domains()
\end{lstlisting}

\paragraph{\_\_getitem\_\_()}
\begin{lstlisting}[language=Python, caption=Indexing Support]
def __getitem__(self, domain_id: int) -> DomainInfo
\end{lstlisting}

\textbf{Parameters:}
\begin{itemize}
    \item \texttt{domain\_id}: Domain index
\end{itemize}

\textbf{Returns:} \texttt{DomainInfo} - Domain at specified index (enables \texttt{geometry[i]})

\textbf{Usage:}
\begin{lstlisting}[language=Python]
first_domain = geometry[0]  # Equivalent to geometry.get_domain(0)
\end{lstlisting}

\paragraph{\_\_iter\_\_()}
\begin{lstlisting}[language=Python, caption=Iteration Support]
def __iter__(self)
\end{lstlisting}

\textbf{Returns:} Iterator over all \texttt{DomainInfo} objects

\textbf{Usage:}
\begin{lstlisting}[language=Python]
for domain in geometry:
    print(f"Processing domain {domain.domain_id}: {domain.name}")
    length = domain.euclidean_length()
    center = domain.center_point()
\end{lstlisting}

\subsection{Complete Usage Example}
\label{subsec:complete_example}

\begin{lstlisting}[language=Python, caption=Complete Geometry Usage Example]
from ooc1d.geometry.domain_geometry import DomainGeometry

# Create geometry for Y-junction network
geometry = DomainGeometry(name="y_junction_network")

# Add main vessel (horizontal segment)
main_id = geometry.add_domain(
    extrema_start=(0.0, 0.0),
    extrema_end=(2.0, 0.0),
    name="main_vessel",
    vessel_type="parent",
    diameter=1.0
)

# Add upper branch
upper_id = geometry.add_domain(
    extrema_start=(2.0, 0.0),
    extrema_end=(3.0, 1.0),
    name="upper_branch",
    vessel_type="daughter",
    diameter=0.7
)

# Add lower branch  
lower_id = geometry.add_domain(
    extrema_start=(2.0, 0.0),
    extrema_end=(3.0, -1.0),
    name="lower_branch",
    vessel_type="daughter",
    diameter=0.7
)

# Set global properties
geometry.set_global_metadata(
    fluid_type="blood",
    viscosity=0.004,  # Pa·s
    density=1060,     # kg/m³
    problem_type="organ_on_chip"
)

# Analyze geometry
print(geometry.summary())
print(f"\nTotal network length: {geometry.total_length():.3f}")

# Access individual domains
for domain in geometry:
    center = domain.center_point()
    direction = domain.direction_vector()
    print(f"Domain {domain.name}:")
    print(f"  Center: ({center[0]:.2f}, {center[1]:.2f})")
    print(f"  Direction: ({direction[0]:.2f}, {direction[1]:.2f})")
    
    # Access metadata
    if "diameter" in domain.metadata:
        print(f"  Diameter: {domain.metadata['diameter']}")

# Find specific domain
main_domain_id = geometry.find_domain_by_name("main_vessel")
if main_domain_id is not None:
    main_domain = geometry[main_domain_id]
    print(f"\nMain vessel length: {main_domain.euclidean_length():.3f}")
\end{lstlisting}

\subsection{Integration with BioNetFlux}
\label{subsec:bionetflux_integration}

The \texttt{DomainGeometry} class is designed for seamless integration with BioNetFlux components:

\begin{lstlisting}[language=Python, caption=BioNetFlux Integration Example]
from ooc1d.core.problem import Problem
from ooc1d.core.discretization import Discretization, GlobalDiscretization

# Create BioNetFlux problems from geometry
def geometry_to_problems(geometry, neq=4, parameters=None):
    """Convert geometry domains to BioNetFlux Problem instances."""
    problems = []
    
    for domain in geometry:
        problem = Problem(
            neq=neq,
            domain_start=domain.domain_start,
            domain_length=domain.domain_length,
            parameters=parameters or np.ones(9),  # OrganOnChip parameters
            problem_type="organ_on_chip",
            name=domain.name
        )
        problems.append(problem)
    
    return problems

# Create discretizations from geometry
def geometry_to_discretizations(geometry, n_elements=10):
    """Convert geometry domains to BioNetFlux Discretization instances."""
    discretizations = []
    
    for domain in geometry:
        disc = Discretization(
            n_elements=n_elements,
            domain_start=domain.domain_start,
            domain_length=domain.domain_length,
            stab_constant=1.0
        )
        discretizations.append(disc)
    
    return GlobalDiscretization(discretizations)

# Usage
problems = geometry_to_problems(geometry)
global_disc = geometry_to_discretizations(geometry)
\end{lstlisting}

This API provides a comprehensive interface for managing complex multi-domain geometries while maintaining compatibility with the existing BioNetFlux solver framework.

% End of domain geometry API documentation
