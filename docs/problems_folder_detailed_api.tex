% Problems Folder Detailed API Documentation (Accurate Analysis)
% To be included in master LaTeX document
%
% Usage: % Problems Folder Detailed API Documentation (Accurate Analysis)
% To be included in master LaTeX document
%
% Usage: % Problems Folder Detailed API Documentation (Accurate Analysis)
% To be included in master LaTeX document
%
% Usage: % Problems Folder Detailed API Documentation (Accurate Analysis)
% To be included in master LaTeX document
%
% Usage: \input{docs/problems_folder_detailed_api}

\section{Problems Folder Detailed API Reference}
\label{sec:problems_folder_detailed_api}

This section provides detailed documentation specifically for the problem modules in the BioNetFlux problems folder, focusing on the mathematical formulations, parameter specifications, and `create\_global\_framework` method implementations based on the actual MATLAB reference files and Python implementations.

\subsection{Problems Folder Structure}

The problems folder contains problem-specific modules that implement the mathematical models for various biological transport phenomena:

\begin{itemize}
    \item \textbf{MATLAB Reference Files}: Original problem definitions and parameters
    \item \textbf{Python Implementation Modules}: BioNetFlux-compatible problem setups
    \item \textbf{Mathematical Model Specifications}: 4-equation OrganOnChip system parameters
    \item \textbf{Boundary Condition Configurations}: Dirichlet and Neumann boundary setups
\end{itemize}

\subsection{Mathematical Model: OrganOnChip System}
\label{subsec:ooc_mathematical_system}

All problems in the folder implement variants of the 4-equation OrganOnChip system:

\subsubsection{Governing System}

\begin{align}
\frac{\partial u}{\partial t} &= \nu \nabla^2 u - \chi \nabla \cdot (u \nabla \phi) + a u + f_u(x,t) \label{eq:u_equation}\\
\frac{\partial \omega}{\partial t} &= \epsilon \nabla^2 \omega + c \omega + d u + f_\omega(x,t) \label{eq:omega_equation}\\
\frac{\partial v}{\partial t} &= \sigma \nabla^2 v + \lambda(\bar{\omega}) v + f_v(x,t) \label{eq:v_equation}\\
\frac{\partial \phi}{\partial t} &= \mu \nabla^2 \phi + a \phi + b v + f_\phi(x,t) \label{eq:phi_equation}
\end{align}

where $\bar{\omega} = \frac{1}{|K|} \int_K \omega \, dx$ is the cell-averaged value of $\omega$.

\subsubsection{Parameter Vector Structure}

All problem modules use the standardized parameter vector:
\begin{align}
\text{parameters} = [\nu, \mu, \epsilon, \sigma, a, b, c, d, \chi]
\end{align}

\subsection{MATLAB Reference Problems}
\label{subsec:matlab_reference_detailed}

\subsubsection{TestProblem.m Analysis}

\paragraph{Complete Parameter Configuration}
\begin{lstlisting}[language=Matlab, caption=TestProblem.m Full Parameter Setup]
% Viscosity parameters (diffusion coefficients)
nu = 1.0;      % Cell diffusion coefficient
mu = 2.0;      % Potential diffusion coefficient  
epsilon = 1.0; % Chemical diffusion coefficient
sigma = 1.0;   % Velocity diffusion coefficient

% Reaction parameters
a = 0.0;       % Growth/decay rate (u and phi equations)
c = 0.0;       % Chemical reaction rate (omega equation)

% Domain specification
A = 0;         % Domain start coordinate
L = 1.0;       % Domain length

% Coupling parameters
b = 1.0;       % Velocity-potential coupling strength
d = 1.0;       % Cell-chemical coupling strength
chi = 1.0;     % Chemotactic sensitivity parameter

% Nonlinearity specification
lambda = @(x) constant_function(x);  % Constant nonlinear function
\end{lstlisting}

\paragraph{OoC\_pbParameters Function}
\begin{lstlisting}[language=Matlab, caption=Parameter Structure Creation]
function problem = OoC_pbParameters(a,b,c,d,chi,lambda,mu,nu,epsilon,sigma,A,L)
    s = whos; % Get all workspace variables
    problem = cell2struct({s.name}.', {s.name}); % Create structure
    eval(structvars(problem,0).'); % Evaluate structure variables
end
\end{lstlisting}

\paragraph{Initial Conditions Specification}
\begin{lstlisting}[language=Matlab, caption=TestProblem.m Initial Conditions]
% Cell density u: Non-trivial sinusoidal profile
problem.u0{1} = @(x,t) sin(2*pi*x);  

% Chemical concentration omega: Zero initial condition
problem.u0{2} = @(x,t) zeros(size(x));

% Velocity field v: Zero initial condition  
problem.u0{3} = @(x,t) zeros(size(x));

% Potential field phi: Zero initial condition
problem.u0{4} = @(x,t) zeros(size(x));

% Gradient initial condition
problem.phix0 = @(x,t) zeros(size(x));
\end{lstlisting}

\paragraph{Source Term Configuration}
\begin{lstlisting}[language=Matlab, caption=TestProblem.m Source Terms]
% All source terms are zero - homogeneous problem
force{1} = @(x,t) zeros(size(x));  % f_u = 0
force{2} = @(x,t) zeros(size(x));  % f_omega = 0  
force{3} = @(x,t) zeros(size(x));  % f_v = 0
force{4} = @(x,t) zeros(size(x));  % f_phi = 0
\end{lstlisting}

\paragraph{Boundary Condition Setup}
\begin{lstlisting}[language=Matlab, caption=TestProblem.m Boundary Conditions]
% Neumann flux conditions at both boundaries
% Left boundary (x = A = 0)
problem.fluxu0{1} = @(t) 0.;  % Zero flux for u
problem.fluxu0{2} = @(t) 0.;  % Zero flux for omega
problem.fluxu0{3} = @(t) 0.;  % Zero flux for v
problem.fluxu0{4} = @(t) 0.;  % Zero flux for phi

% Right boundary (x = A + L = 1)
problem.fluxu1{1} = @(t) 0.;  % Zero flux for u
problem.fluxu1{2} = @(t) 0.;  % Zero flux for omega
problem.fluxu1{3} = @(t) 0.;  % Zero flux for v
problem.fluxu1{4} = @(t) 0.;  % Zero flux for phi

% Neumann data vector (8×1: 2 boundaries × 4 equations)
problem.NeumannData = zeros(8,1);
\end{lstlisting}

\subsubsection{EmptyProblem.m Analysis}

\paragraph{Parameter Differences from TestProblem.m}
\begin{lstlisting}[language=Matlab, caption=EmptyProblem.m Parameter Configuration]
% IDENTICAL to TestProblem.m:
nu = 1.0; mu = 2.0; epsilon = 1.0; sigma = 1.0;
a = 0.0; c = 0.0;
A = 0; L = 1.0;
b = 1.0; d = 1.0; chi = 1.0;
lambda = @(x) constant_function(x);
\end{lstlisting}

\paragraph{Key Difference: Initial Conditions}
\begin{lstlisting}[language=Matlab, caption=EmptyProblem.m Initial Conditions]
% ALL initial conditions are zero (unlike TestProblem.m)
problem.u0{1} = @(x,t) zeros(size(x));  % Zero cell density
problem.u0{2} = @(x,t) zeros(size(x));  % Zero chemical
problem.u0{3} = @(x,t) zeros(size(x));  % Zero velocity
problem.u0{4} = @(x,t) zeros(size(x));  % Zero potential

problem.phix0 = @(x,t) zeros(size(x));  % Zero gradient
\end{lstlisting}

\paragraph{Source Terms and Boundary Conditions}
\begin{lstlisting}[language=Matlab, caption=EmptyProblem.m Additional Configuration]
% Source terms - also all zero
force_u = @(x,t) zeros(size(x));
force_phi = @(x,t) zeros(size(x));  
force_v = @(x,t) zeros(size(x));
force_omega = @(x,t) zeros(size(x));

% Boundary conditions - identical to TestProblem.m
% All zero Neumann flux conditions
problem.fluxu0{1-4} = @(t) 0.;  % Left boundary
problem.fluxu1{1-4} = @(t) 0.;  % Right boundary
problem.NeumannData = zeros(8,1);
\end{lstlisting}

\subsubsection{MATLAB Problems Comparison}

\begin{longtable}{|p{3cm}|p{4cm}|p{4cm}|p{2cm}|}
\hline
\textbf{Aspect} & \textbf{TestProblem.m} & \textbf{EmptyProblem.m} & \textbf{Purpose} \\
\hline
\endhead

Parameters & $\nu=1, \mu=2, \epsilon=1, \sigma=1$ & Identical & Reference values \\
\hline

Reaction & $a=0, c=0$ & Identical & No reaction terms \\
\hline

Coupling & $b=1, d=1, \chi=1$ & Identical & Unit coupling \\
\hline

Domain & $[0, 1]$ & Identical & Unit interval \\
\hline

Nonlinearity & $\lambda(x) = \text{constant}$ & Identical & Linear case \\
\hline

Initial u & $\sin(2\pi x)$ & $0$ & Non-trivial vs zero \\
\hline

Initial $\omega,v,\phi$ & $0$ & $0$ & Both zero \\
\hline

Source terms & All zero & All zero & Homogeneous \\
\hline

Boundary conditions & Zero Neumann & Zero Neumann & Natural boundaries \\
\hline

Use case & Non-trivial initial test & Baseline/template & Testing framework \\
\hline

\end{longtable}

\subsection{Python Implementation: ooc\_test\_problem.py}
\label{subsec:python_ooc_implementation}

\subsubsection{create\_global\_framework Method Analysis}

\paragraph{Method Signature and Overview}
\begin{lstlisting}[language=Python, caption=create\_global\_framework Method Signature]
def create_global_framework():
    """
    OrganOnChip test problem - Python port from MATLAB TestProblem.m
    
    Returns:
        Tuple: (problems, global_discretization, constraint_manager, problem_name)
            - problems: List[Problem] - Single domain problem instance
            - global_discretization: GlobalDiscretization - Time and space discretization
            - constraint_manager: ConstraintManager - Boundary condition manager
            - problem_name: str - Descriptive problem name
    """
\end{lstlisting}

\paragraph{Mesh and Global Configuration}
\begin{lstlisting}[language=Python, caption=Python Global Configuration]
# Mesh parameters
n_elements = 20  # Spatial discretization fineness

# Global problem structure  
ndom = 1        # Single domain
neq = 4         # Four coupled equations
T = 1.0         # Final simulation time
dt = 0.1        # Time step size

problem_name = "OrganOnChip Test Problem"
\end{lstlisting}

\paragraph{Parameter Configuration Differences}
\begin{lstlisting}[language=Python, caption=Python Parameter Setup vs MATLAB]
# Physical parameters - DIFFERENCES from MATLAB highlighted
nu = 1.0      # SAME as MATLAB
mu = 1.0      # DIFFERENT: MATLAB uses mu = 2.0
epsilon = 1.0 # SAME as MATLAB  
sigma = 1.0   # SAME as MATLAB

# Reaction parameters - DIFFERENCES from MATLAB
a = 1.0       # DIFFERENT: MATLAB uses a = 0.0
c = 1.0       # DIFFERENT: MATLAB uses c = 0.0

# Coupling parameters - SAME as MATLAB
b = 1.0       # SAME as MATLAB
d = 1.0       # SAME as MATLAB  
chi = 1.0     # SAME as MATLAB

# Domain configuration - DIFFERENT from MATLAB
domain_start = 1.0    # DIFFERENT: MATLAB uses A = 0
domain_length = 1.0   # SAME: MATLAB uses L = 1.0

# Parameter vector assembly [nu, mu, epsilon, sigma, a, b, c, d, chi]
parameters = np.array([nu, mu, epsilon, sigma, a, b, c, d, chi])
\end{lstlisting}

\paragraph{Nonlinear Function Configuration}
\begin{lstlisting}[language=Python, caption=Python Nonlinear Function Setup]
# DIFFERENT from MATLAB: Non-constant nonlinearity
lambda_function = lambda x: 1.0/(1.0 + x**2)      # Saturation-type
dlambda_function = lambda x: -2.0*x/(1.0 + x**2)**2  # Analytical derivative

# MATLAB equivalent (for comparison):
# lambda = @(x) constant_function(x);  % Returns ones(size(x))
\end{lstlisting}

\paragraph{Initial Condition Functions}
\begin{lstlisting}[language=Python, caption=Python Initial Conditions - Complex]
def initial_u(s, t=0.0):
    """Cell density initial condition"""
    s = np.asarray(s)
    return 0.0 * s  # Zero initial condition (like EmptyProblem.m)

def initial_omega(s, t=0.0):
    """Chemical concentration - TIME-DEPENDENT initial condition"""
    s = np.asarray(s)
    return np.sin(2 * np.pi * s + np.pi * t)  # MATLAB: zeros(size(x))

def initial_v(s, t=0.0):
    """Velocity field - SPACE-TIME coupled initial condition"""
    s = np.asarray(s)
    return t * s  # MATLAB: zeros(size(x))

def initial_phi(s, t=0.0):
    """Potential field - QUADRATIC spatial profile"""
    return s ** 2  # MATLAB: zeros(size(x))
\end{lstlisting}

\paragraph{Source Term Functions}
\begin{lstlisting}[language=Python, caption=Python Source Terms - Non-Zero]
def force_u(s, t):
    """Cell density source - remains zero"""
    s = np.asarray(s)
    return np.zeros_like(s)  # SAME as MATLAB

def force_omega(s, t):
    """Chemical source - COMPLEX analytical form"""
    s = np.asarray(s)
    x = 2 * np.pi * s + np.pi * t
    return np.sin(x) + 4 * np.pi**2 * np.sin(x) + np.pi * np.cos(x)
    # MATLAB: zeros(size(x))

def force_v(s, t):
    """Velocity source - NONLINEAR coupling"""
    omega_val = initial_omega(s, t)
    lambda_val = lambda_function(omega_val)
    s = np.asarray(s)
    return s + lambda_val * t * s
    # MATLAB: zeros(size(x))

def force_phi(s, t):
    """Potential source - MULTI-TERM expression"""
    s = np.asarray(s)
    return - mu * 2.0 * np.ones_like(s) + a * s**2 - b * t * s
    # MATLAB: zeros(size(x))
\end{lstlisting}

\paragraph{Problem Instance Creation}
\begin{lstlisting}[language=Python, caption=Python Problem Instance Setup]
# Create Problem instance with BioNetFlux framework
problem = Problem(
    neq=neq,                    # 4 equations
    domain_start=domain_start,  # 1.0 (vs MATLAB A=0) 
    domain_length=domain_length, # 1.0 (same as MATLAB L)
    parameters=parameters,       # 9-element parameter vector
    problem_type="organ_on_chip", # Problem classification
    name="ooc_test"             # Instance identifier
)

# Set nonlinear functions using flexible interface
problem.set_function('lambda_function', lambda_function)  
problem.set_function('dlambda_function', dlambda_function)

# Set source terms for all 4 equations
problem.set_force(0, lambda s, t: force_u(s, t))      # u equation
problem.set_force(1, lambda s, t: force_omega(s, t))  # omega equation  
problem.set_force(2, lambda s, t: force_v(s, t))      # v equation
problem.set_force(3, lambda s, t: force_phi(s, t))    # phi equation

# Set initial conditions for all 4 equations
problem.set_initial_condition(0, initial_u)   # u initial condition
problem.set_initial_condition(1, initial_omega) # omega initial condition
problem.set_initial_condition(2, initial_v)   # v initial condition
problem.set_initial_condition(3, initial_phi) # phi initial condition
\end{lstlisting}

\paragraph{Discretization Configuration}
\begin{lstlisting}[language=Python, caption=Python Discretization Setup]
# Spatial discretization
discretization = Discretization(
    n_elements=n_elements,        # 20 elements
    domain_start=domain_start,    # 1.0
    domain_length=domain_length,  # 1.0
    stab_constant=1.0            # Stabilization constant
)

# Stabilization parameters for HDG method
tau_u = 1.0 / discretization.element_length     # Scaled by mesh size
tau_omega = 1.0                                 # Constant
tau_v = 1.0                                     # Constant  
tau_phi = 1.0                                   # Constant

discretization.set_tau([tau_u, tau_omega, tau_v, tau_phi])

# Global discretization with time parameters
global_disc = GlobalDiscretization([discretization])
global_disc.set_time_parameters(dt, T)  # dt=0.1, T=1.0
\end{lstlisting}

\paragraph{Boundary Condition Configuration}
\begin{lstlisting}[language=Python, caption=Python Boundary Condition Setup]
# Analytical flux functions for Neumann boundary conditions
flux_u = lambda s, t: 0.0  # SAME as MATLAB: zero flux
flux_omega = lambda s, t: 2 * np.pi * np.cos(2 * np.pi * s + np.pi * t)  # NON-ZERO
flux_v = lambda s, t: t    # TIME-DEPENDENT flux  
flux_phi = lambda s, t: 2 * s  # SPACE-DEPENDENT flux

# Create constraint manager
constraint_manager = ConstraintManager()
domain_end = domain_start + domain_length  # 2.0

# Add Neumann boundary conditions for all equations
# Left boundary (x = domain_start = 1.0)
constraint_manager.add_neumann(0, 0, domain_start, lambda t: -flux_u(domain_start, t))
constraint_manager.add_neumann(1, 0, domain_start, lambda t: -flux_omega(domain_start, t))
constraint_manager.add_neumann(2, 0, domain_start, lambda t: -flux_v(domain_start, t))
constraint_manager.add_neumann(3, 0, domain_start, lambda t: -flux_phi(domain_start, t))

# Right boundary (x = domain_end = 2.0)  
constraint_manager.add_neumann(0, 0, domain_end, lambda t: flux_u(domain_end, t))
constraint_manager.add_neumann(1, 0, domain_end, lambda t: flux_omega(domain_end, t))
constraint_manager.add_neumann(2, 0, domain_end, lambda t: flux_v(domain_end, t))
constraint_manager.add_neumann(3, 0, domain_end, lambda t: flux_phi(domain_end, t))

# Map constraints to discretizations
constraint_manager.map_to_discretizations([discretization])
\end{lstlisting}

\paragraph{Return Value Assembly}
\begin{lstlisting}[language=Python, caption=create\_global\_framework Return]
# Return framework components as expected by BioNetFlux
return [problem], global_disc, constraint_manager, problem_name

# Return structure:
# - [problem]: List containing single Problem instance
# - global_disc: GlobalDiscretization with time and space parameters  
# - constraint_manager: ConstraintManager with mapped boundary conditions
# - problem_name: String identifier for the problem configuration
\end{lstlisting}

\subsection{MATLAB vs Python Implementation Comparison}
\label{subsec:comprehensive_comparison}

\subsubsection{Parameter Value Differences}

\begin{longtable}{|p{2.5cm}|p{2cm}|p{2cm}|p{6.5cm}|}
\hline
\textbf{Parameter} & \textbf{MATLAB} & \textbf{Python} & \textbf{Mathematical Impact} \\
\hline
\endhead

$\mu$ (potential diffusion) & 2.0 & 1.0 & Reduced potential diffusion in Python \\
\hline

$a$ (growth rate) & 0.0 & 1.0 & Added growth terms: $+u$ in eq.~\eqref{eq:u_equation}, $+\phi$ in eq.~\eqref{eq:phi_equation} \\
\hline

$c$ (chemical reaction) & 0.0 & 1.0 & Added reaction term: $+\omega$ in eq.~\eqref{eq:omega_equation} \\
\hline

Domain start & 0.0 & 1.0 & Shifted domain from $[0,1]$ to $[1,2]$ \\
\hline

$\lambda(x)$ function & $\text{constant}$ & $\frac{1}{1+x^2}$ & Nonlinear saturation vs linear behavior \\
\hline

\end{longtable}

\subsubsection{Initial Condition Complexity}

\begin{longtable}{|p{2cm}|p{3.5cm}|p{3.5cm}|p{4cm}|}
\hline
\textbf{Variable} & \textbf{MATLAB TestProblem.m} & \textbf{Python ooc\_test\_problem.py} & \textbf{Complexity Level} \\
\hline
\endhead

$u(x,0)$ & $\sin(2\pi x)$ & $0$ & MATLAB: Spatial oscillation \\
\hline

$\omega(x,0)$ & $0$ & $\sin(2\pi x + \pi t)$ & Python: Spatio-temporal \\
\hline

$v(x,0)$ & $0$ & $t \cdot x$ & Python: Space-time product \\
\hline

$\phi(x,0)$ & $0$ & $x^2$ & Python: Quadratic profile \\
\hline

\end{longtable}

\subsubsection{Source Term Complexity}

\begin{longtable}{|p{2cm}|p{5cm}|p{6cm}|}
\hline
\textbf{Equation} & \textbf{MATLAB (All Problems)} & \textbf{Python Implementation} \\
\hline
\endhead

$f_u$ & $0$ & $0$ \\
\hline

$f_\omega$ & $0$ & $\sin(x) + 4\pi^2\sin(x) + \pi\cos(x)$ where $x = 2\pi s + \pi t$ \\
\hline

$f_v$ & $0$ & $s + \lambda(\omega(s,t)) \cdot t \cdot s$ \\
\hline

$f_\phi$ & $0$ & $-\mu \cdot 2 + a s^2 - b t s$ \\
\hline

\end{longtable}

\subsubsection{Boundary Condition Differences}

\begin{longtable}{|p{2.5cm}|p{3cm}|p{3cm}|p{4.5cm}|}
\hline
\textbf{Flux Type} & \textbf{MATLAB} & \textbf{Python} & \textbf{Mathematical Form} \\
\hline
\endhead

$u$ flux & Zero & Zero & No difference \\
\hline

$\omega$ flux & Zero & $2\pi\cos(2\pi s + \pi t)$ & Time-dependent analytical flux \\
\hline

$v$ flux & Zero & $t$ & Linear time dependence \\
\hline

$\phi$ flux & Zero & $2s$ & Linear spatial dependence \\
\hline

\end{longtable}

\subsection{Framework Integration Patterns}
\label{subsec:framework_integration_patterns}

\subsubsection{Problem Instance Configuration}

\begin{lstlisting}[language=Python, caption=Standard Problem Configuration Pattern]
def create_global_framework():
    """Standard pattern for problem module implementation"""
    
    # 1. Parameter Definition
    # Define all physical, reaction, and coupling parameters
    parameters = np.array([nu, mu, epsilon, sigma, a, b, c, d, chi])
    
    # 2. Domain Configuration  
    # Set domain start, length, and mesh parameters
    domain_start, domain_length, n_elements = ...
    
    # 3. Function Definitions
    # Define initial conditions, source terms, and nonlinear functions
    def initial_condition_i(s, t): ...
    def force_function_i(s, t): ...
    lambda_function = lambda x: ...
    
    # 4. Problem Creation
    problem = Problem(neq, domain_start, domain_length, parameters, ...)
    problem.set_initial_condition(i, initial_condition_i)
    problem.set_force(i, force_function_i)
    problem.set_function('lambda_function', lambda_function)
    
    # 5. Discretization Setup
    discretization = Discretization(n_elements, domain_start, domain_length, ...)
    discretization.set_tau([tau_0, tau_1, tau_2, tau_3])
    
    # 6. Time Integration Configuration
    global_disc = GlobalDiscretization([discretization])
    global_disc.set_time_parameters(dt, T)
    
    # 7. Boundary Condition Setup
    constraint_manager = ConstraintManager()
    constraint_manager.add_neumann/add_dirichlet(...)
    constraint_manager.map_to_discretizations([discretization])
    
    # 8. Return Framework Components
    return [problem], global_disc, constraint_manager, problem_name
\end{lstlisting}

\subsubsection{Multi-Domain Extension Pattern}

\begin{lstlisting}[language=Python, caption=Expected Multi-Domain Problem Pattern]
def create_multi_domain_framework():
    """Expected pattern for multi-domain problems"""
    
    problems = []
    discretizations = []
    
    # Create multiple domains
    for domain_idx, domain_config in enumerate(domain_configurations):
        # Domain-specific parameters
        domain_parameters = modify_parameters_for_domain(base_parameters, domain_config)
        
        # Create domain problem
        problem = Problem(
            neq=neq,
            domain_start=domain_config['start'],
            domain_length=domain_config['length'],
            parameters=domain_parameters
        )
        
        # Set domain-specific functions
        problem.set_initial_condition(i, lambda s, t: domain_initial_i(s, t, domain_idx))
        problem.set_force(i, lambda s, t: domain_force_i(s, t, domain_idx))
        
        problems.append(problem)
        
        # Create domain discretization
        discretization = Discretization(...)
        discretizations.append(discretization)
    
    # Global discretization and time setup
    global_disc = GlobalDiscretization(discretizations)
    global_disc.set_time_parameters(dt, T)
    
    # Inter-domain constraints (junctions)
    constraint_manager = ConstraintManager()
    
    # Add junction constraints between domains
    for junction in junction_configurations:
        constraint_manager.add_trace_continuity(...)
        constraint_manager.add_flux_continuity(...)
        constraint_manager.add_kedem_katchalsky(...)
    
    # Add boundary constraints for exterior boundaries
    constraint_manager.add_dirichlet/add_neumann(...)
    
    constraint_manager.map_to_discretizations(discretizations)
    
    return problems, global_disc, constraint_manager, problem_name
\end{lstlisting}

\subsection{Parameter Sensitivity Analysis Methods}
\label{subsec:parameter_sensitivity_methods}

\subsubsection{Parameter Variation Utilities}

\begin{lstlisting}[language=Python, caption=Parameter Sensitivity Analysis Pattern]
def create_parameter_sweep(base_problem_module, parameter_variations):
    """
    Create multiple problem instances for parameter sensitivity studies
    
    Args:
        base_problem_module: String path to base problem module
        parameter_variations: Dict mapping parameter names to value lists
        
    Returns:
        List of (parameter_set, framework_components) tuples
    """
    
    # Load base framework
    base_problems, base_global_disc, base_constraints, base_name = \
        importlib.import_module(base_problem_module).create_global_framework()
    
    base_problem = base_problems[0]
    base_parameters = base_problem.parameters.copy()
    
    # Parameter index mapping
    param_indices = {
        'nu': 0, 'mu': 1, 'epsilon': 2, 'sigma': 3,
        'a': 4, 'b': 5, 'c': 6, 'd': 7, 'chi': 8
    }
    
    results = []
    
    # Generate parameter combinations
    for param_name, param_values in parameter_variations.items():
        param_idx = param_indices[param_name]
        
        for param_value in param_values:
            # Create modified parameters
            modified_params = base_parameters.copy()
            modified_params[param_idx] = param_value
            
            # Create new problem with modified parameters
            modified_problem = Problem(
                neq=base_problem.neq,
                domain_start=base_problem.domain_start,
                domain_length=base_problem.domain_length,
                parameters=modified_params,
                problem_type=base_problem.type,
                name=f"{base_problem.name}_{param_name}_{param_value}"
            )
            
            # Copy functions and conditions from base problem
            for i in range(base_problem.neq):
                if base_problem.initial_conditions[i] is not None:
                    modified_problem.set_initial_condition(i, base_problem.initial_conditions[i])
                if base_problem.force_functions[i] is not None:
                    modified_problem.set_force(i, base_problem.force_functions[i])
            
            # Copy additional functions
            for func_name, func in base_problem.additional_functions.items():
                modified_problem.set_function(func_name, func)
            
            # Create modified framework components
            modified_framework = (
                [modified_problem],
                base_global_disc,  # Reuse discretization and time setup
                base_constraints,  # Reuse boundary conditions
                f"{base_name} - {param_name}={param_value}"
            )
            
            results.append(((param_name, param_value), modified_framework))
    
    return results

# Usage example
parameter_variations = {
    'chi': [0.1, 0.5, 1.0, 2.0, 5.0],  # Chemotactic sensitivity
    'mu': [0.5, 1.0, 2.0, 4.0],        # Potential diffusion
    'a': [0.0, 0.5, 1.0, 1.5]          # Growth rate
}

sensitivity_study = create_parameter_sweep(
    'ooc1d.problems.ooc_test_problem',
    parameter_variations
)
\end{lstlisting}

\subsection{Problem Module Summary}
\label{subsec:problem_module_summary_detailed}

\subsubsection{Implemented Problems Overview}

\begin{longtable}{|p{4.5cm}|p{2cm}|p{3cm}|p{4.5cm}|}
\hline
\textbf{Module/File} & \textbf{Domain} & \textbf{Complexity} & \textbf{Key Features} \\
\hline
\endhead

\texttt{TestProblem.m} & $[0,1]$ & Medium & Non-zero $u$ initial condition, constant $\lambda$ \\
\hline

\texttt{EmptyProblem.m} & $[0,1]$ & Low & All zero initial conditions, template structure \\
\hline

\texttt{ooc\_test\_problem.py} & $[1,2]$ & High & Rich analytical test case, nonlinear $\lambda$ \\
\hline

\end{longtable}

\subsubsection{Framework Integration Summary}

\begin{longtable}{|p{5cm}|p{8cm}|}
\hline
\textbf{Component} & \textbf{Implementation Details} \\
\hline
\endhead

\texttt{create\_global\_framework} & Returns (problems, global\_discretization, constraint\_manager, problem\_name) \\
\hline

Parameter vector & Standardized 9-element array: $[\nu, \mu, \epsilon, \sigma, a, b, c, d, \chi]$ \\
\hline

Initial conditions & 4 functions for $(u, \omega, v, \phi)$ with optional time dependence \\
\hline

Source terms & 4 functions for $(f_u, f_\omega, f_v, f_\phi)$ with spatio-temporal dependence \\
\hline

Nonlinear functions & $\lambda(x)$ and $\lambda'(x)$ for velocity equation nonlinearity \\
\hline

Boundary conditions & Neumann flux specifications for all 4 equations at both boundaries \\
\hline

Time discretization & Uniform time stepping with configurable $dt$ and final time $T$ \\
\hline

Space discretization & Uniform mesh with configurable element count and stabilization parameters \\
\hline

\end{longtable}

\subsection{Expected Extensions and Variations}
\label{subsec:expected_extensions}

\subsubsection{Alternative Nonlinear Functions}

\begin{lstlisting}[language=Python, caption=Expected Alternative Lambda Functions]
# Michaelis-Menten kinetics
lambda_mm = lambda x, K=1.0, V_max=2.0: V_max * x / (K + x)
dlambda_mm = lambda x, K=1.0, V_max=2.0: V_max * K / (K + x)**2

# Hill function (cooperative binding)  
lambda_hill = lambda x, K=1.0, n=2.0: x**n / (K**n + x**n)
dlambda_hill = lambda x, K=1.0, n=2.0: n * x**(n-1) * K**n / (K**n + x**n)**2

# Exponential saturation
lambda_exp = lambda x, alpha=1.0, beta=2.0: alpha * (1.0 - np.exp(-beta * x))
dlambda_exp = lambda x, alpha=1.0, beta=2.0: alpha * beta * np.exp(-beta * x)
\end{lstlisting}

\subsubsection{Expected Multi-Domain Problem Structure}

\begin{lstlisting}[language=Python, caption=Expected Multi-Domain Problem Template]
def create_vascular_network_framework():
    """Expected vascular network problem with 3 domains"""
    
    # Domain configurations
    domains = [
        {"start": 0.0, "length": 1.0, "chi": 1.0, "name": "vessel_1"},
        {"start": 1.0, "length": 0.5, "chi": 2.0, "name": "junction"},  
        {"start": 1.5, "length": 1.0, "chi": 0.5, "name": "vessel_2"}
    ]
    
    problems = []
    discretizations = []
    
    for i, domain_config in enumerate(domains):
        # Domain-specific parameter modifications
        parameters = base_parameters.copy()
        parameters[8] = domain_config["chi"]  # Modify chemotactic sensitivity
        
        # Create domain problem
        problem = Problem(
            neq=4,
            domain_start=domain_config["start"],
            domain_length=domain_config["length"],
            parameters=parameters,
            problem_type="vascular_network",
            name=domain_config["name"]
        )
        
        # Domain-specific initial conditions and source terms
        problem.set_initial_condition(0, lambda s, t: domain_initial_u(s, t, i))
        # ... set other conditions ...
        
        problems.append(problem)
        discretizations.append(create_domain_discretization(domain_config))
    
    # Junction constraints
    constraint_manager = ConstraintManager()
    
    # Continuity at domain interfaces
    constraint_manager.add_trace_continuity(0, 0, 1, -1, 1.0)  # u continuity
    constraint_manager.add_trace_continuity(1, 0, 2, -1, 1.5)  # omega continuity
    
    # Flux continuity with permeability effects
    constraint_manager.add_kedem_katchalsky(2, 0, 1, -1, 1.0, P=0.1)  # v coupling
    constraint_manager.add_kedem_katchalsky(3, 0, 2, -1, 1.5, P=0.2)  # phi coupling
    
    # Boundary conditions at network ends
    constraint_manager.add_dirichlet(0, 0, 0.0, lambda t: 1.0)  # u inlet
    constraint_manager.add_neumann(0, 2, 2.5, lambda t: 0.0)    # u outlet
    
    global_disc = GlobalDiscretization(discretizations)
    global_disc.set_time_parameters(dt=0.01, T=2.0)
    
    constraint_manager.map_to_discretizations(discretizations)
    
    return problems, global_disc, constraint_manager, "Vascular Network Model"
\end{lstlisting}

This comprehensive documentation provides exact details of the problems folder modules, focusing specifically on the 'create\_global\_framework' implementations and their parameter configurations, without duplicating information about other BioNetFlux components that are documented separately.

% End of problems folder detailed API documentation


\section{Problems Folder Detailed API Reference}
\label{sec:problems_folder_detailed_api}

This section provides detailed documentation specifically for the problem modules in the BioNetFlux problems folder, focusing on the mathematical formulations, parameter specifications, and `create\_global\_framework` method implementations based on the actual MATLAB reference files and Python implementations.

\subsection{Problems Folder Structure}

The problems folder contains problem-specific modules that implement the mathematical models for various biological transport phenomena:

\begin{itemize}
    \item \textbf{MATLAB Reference Files}: Original problem definitions and parameters
    \item \textbf{Python Implementation Modules}: BioNetFlux-compatible problem setups
    \item \textbf{Mathematical Model Specifications}: 4-equation OrganOnChip system parameters
    \item \textbf{Boundary Condition Configurations}: Dirichlet and Neumann boundary setups
\end{itemize}

\subsection{Mathematical Model: OrganOnChip System}
\label{subsec:ooc_mathematical_system}

All problems in the folder implement variants of the 4-equation OrganOnChip system:

\subsubsection{Governing System}

\begin{align}
\frac{\partial u}{\partial t} &= \nu \nabla^2 u - \chi \nabla \cdot (u \nabla \phi) + a u + f_u(x,t) \label{eq:u_equation}\\
\frac{\partial \omega}{\partial t} &= \epsilon \nabla^2 \omega + c \omega + d u + f_\omega(x,t) \label{eq:omega_equation}\\
\frac{\partial v}{\partial t} &= \sigma \nabla^2 v + \lambda(\bar{\omega}) v + f_v(x,t) \label{eq:v_equation}\\
\frac{\partial \phi}{\partial t} &= \mu \nabla^2 \phi + a \phi + b v + f_\phi(x,t) \label{eq:phi_equation}
\end{align}

where $\bar{\omega} = \frac{1}{|K|} \int_K \omega \, dx$ is the cell-averaged value of $\omega$.

\subsubsection{Parameter Vector Structure}

All problem modules use the standardized parameter vector:
\begin{align}
\text{parameters} = [\nu, \mu, \epsilon, \sigma, a, b, c, d, \chi]
\end{align}

\subsection{MATLAB Reference Problems}
\label{subsec:matlab_reference_detailed}

\subsubsection{TestProblem.m Analysis}

\paragraph{Complete Parameter Configuration}
\begin{lstlisting}[language=Matlab, caption=TestProblem.m Full Parameter Setup]
% Viscosity parameters (diffusion coefficients)
nu = 1.0;      % Cell diffusion coefficient
mu = 2.0;      % Potential diffusion coefficient  
epsilon = 1.0; % Chemical diffusion coefficient
sigma = 1.0;   % Velocity diffusion coefficient

% Reaction parameters
a = 0.0;       % Growth/decay rate (u and phi equations)
c = 0.0;       % Chemical reaction rate (omega equation)

% Domain specification
A = 0;         % Domain start coordinate
L = 1.0;       % Domain length

% Coupling parameters
b = 1.0;       % Velocity-potential coupling strength
d = 1.0;       % Cell-chemical coupling strength
chi = 1.0;     % Chemotactic sensitivity parameter

% Nonlinearity specification
lambda = @(x) constant_function(x);  % Constant nonlinear function
\end{lstlisting}

\paragraph{OoC\_pbParameters Function}
\begin{lstlisting}[language=Matlab, caption=Parameter Structure Creation]
function problem = OoC_pbParameters(a,b,c,d,chi,lambda,mu,nu,epsilon,sigma,A,L)
    s = whos; % Get all workspace variables
    problem = cell2struct({s.name}.', {s.name}); % Create structure
    eval(structvars(problem,0).'); % Evaluate structure variables
end
\end{lstlisting}

\paragraph{Initial Conditions Specification}
\begin{lstlisting}[language=Matlab, caption=TestProblem.m Initial Conditions]
% Cell density u: Non-trivial sinusoidal profile
problem.u0{1} = @(x,t) sin(2*pi*x);  

% Chemical concentration omega: Zero initial condition
problem.u0{2} = @(x,t) zeros(size(x));

% Velocity field v: Zero initial condition  
problem.u0{3} = @(x,t) zeros(size(x));

% Potential field phi: Zero initial condition
problem.u0{4} = @(x,t) zeros(size(x));

% Gradient initial condition
problem.phix0 = @(x,t) zeros(size(x));
\end{lstlisting}

\paragraph{Source Term Configuration}
\begin{lstlisting}[language=Matlab, caption=TestProblem.m Source Terms]
% All source terms are zero - homogeneous problem
force{1} = @(x,t) zeros(size(x));  % f_u = 0
force{2} = @(x,t) zeros(size(x));  % f_omega = 0  
force{3} = @(x,t) zeros(size(x));  % f_v = 0
force{4} = @(x,t) zeros(size(x));  % f_phi = 0
\end{lstlisting}

\paragraph{Boundary Condition Setup}
\begin{lstlisting}[language=Matlab, caption=TestProblem.m Boundary Conditions]
% Neumann flux conditions at both boundaries
% Left boundary (x = A = 0)
problem.fluxu0{1} = @(t) 0.;  % Zero flux for u
problem.fluxu0{2} = @(t) 0.;  % Zero flux for omega
problem.fluxu0{3} = @(t) 0.;  % Zero flux for v
problem.fluxu0{4} = @(t) 0.;  % Zero flux for phi

% Right boundary (x = A + L = 1)
problem.fluxu1{1} = @(t) 0.;  % Zero flux for u
problem.fluxu1{2} = @(t) 0.;  % Zero flux for omega
problem.fluxu1{3} = @(t) 0.;  % Zero flux for v
problem.fluxu1{4} = @(t) 0.;  % Zero flux for phi

% Neumann data vector (8×1: 2 boundaries × 4 equations)
problem.NeumannData = zeros(8,1);
\end{lstlisting}

\subsubsection{EmptyProblem.m Analysis}

\paragraph{Parameter Differences from TestProblem.m}
\begin{lstlisting}[language=Matlab, caption=EmptyProblem.m Parameter Configuration]
% IDENTICAL to TestProblem.m:
nu = 1.0; mu = 2.0; epsilon = 1.0; sigma = 1.0;
a = 0.0; c = 0.0;
A = 0; L = 1.0;
b = 1.0; d = 1.0; chi = 1.0;
lambda = @(x) constant_function(x);
\end{lstlisting}

\paragraph{Key Difference: Initial Conditions}
\begin{lstlisting}[language=Matlab, caption=EmptyProblem.m Initial Conditions]
% ALL initial conditions are zero (unlike TestProblem.m)
problem.u0{1} = @(x,t) zeros(size(x));  % Zero cell density
problem.u0{2} = @(x,t) zeros(size(x));  % Zero chemical
problem.u0{3} = @(x,t) zeros(size(x));  % Zero velocity
problem.u0{4} = @(x,t) zeros(size(x));  % Zero potential

problem.phix0 = @(x,t) zeros(size(x));  % Zero gradient
\end{lstlisting}

\paragraph{Source Terms and Boundary Conditions}
\begin{lstlisting}[language=Matlab, caption=EmptyProblem.m Additional Configuration]
% Source terms - also all zero
force_u = @(x,t) zeros(size(x));
force_phi = @(x,t) zeros(size(x));  
force_v = @(x,t) zeros(size(x));
force_omega = @(x,t) zeros(size(x));

% Boundary conditions - identical to TestProblem.m
% All zero Neumann flux conditions
problem.fluxu0{1-4} = @(t) 0.;  % Left boundary
problem.fluxu1{1-4} = @(t) 0.;  % Right boundary
problem.NeumannData = zeros(8,1);
\end{lstlisting}

\subsubsection{MATLAB Problems Comparison}

\begin{longtable}{|p{3cm}|p{4cm}|p{4cm}|p{2cm}|}
\hline
\textbf{Aspect} & \textbf{TestProblem.m} & \textbf{EmptyProblem.m} & \textbf{Purpose} \\
\hline
\endhead

Parameters & $\nu=1, \mu=2, \epsilon=1, \sigma=1$ & Identical & Reference values \\
\hline

Reaction & $a=0, c=0$ & Identical & No reaction terms \\
\hline

Coupling & $b=1, d=1, \chi=1$ & Identical & Unit coupling \\
\hline

Domain & $[0, 1]$ & Identical & Unit interval \\
\hline

Nonlinearity & $\lambda(x) = \text{constant}$ & Identical & Linear case \\
\hline

Initial u & $\sin(2\pi x)$ & $0$ & Non-trivial vs zero \\
\hline

Initial $\omega,v,\phi$ & $0$ & $0$ & Both zero \\
\hline

Source terms & All zero & All zero & Homogeneous \\
\hline

Boundary conditions & Zero Neumann & Zero Neumann & Natural boundaries \\
\hline

Use case & Non-trivial initial test & Baseline/template & Testing framework \\
\hline

\end{longtable}

\subsection{Python Implementation: ooc\_test\_problem.py}
\label{subsec:python_ooc_implementation}

\subsubsection{create\_global\_framework Method Analysis}

\paragraph{Method Signature and Overview}
\begin{lstlisting}[language=Python, caption=create\_global\_framework Method Signature]
def create_global_framework():
    """
    OrganOnChip test problem - Python port from MATLAB TestProblem.m
    
    Returns:
        Tuple: (problems, global_discretization, constraint_manager, problem_name)
            - problems: List[Problem] - Single domain problem instance
            - global_discretization: GlobalDiscretization - Time and space discretization
            - constraint_manager: ConstraintManager - Boundary condition manager
            - problem_name: str - Descriptive problem name
    """
\end{lstlisting}

\paragraph{Mesh and Global Configuration}
\begin{lstlisting}[language=Python, caption=Python Global Configuration]
# Mesh parameters
n_elements = 20  # Spatial discretization fineness

# Global problem structure  
ndom = 1        # Single domain
neq = 4         # Four coupled equations
T = 1.0         # Final simulation time
dt = 0.1        # Time step size

problem_name = "OrganOnChip Test Problem"
\end{lstlisting}

\paragraph{Parameter Configuration Differences}
\begin{lstlisting}[language=Python, caption=Python Parameter Setup vs MATLAB]
# Physical parameters - DIFFERENCES from MATLAB highlighted
nu = 1.0      # SAME as MATLAB
mu = 1.0      # DIFFERENT: MATLAB uses mu = 2.0
epsilon = 1.0 # SAME as MATLAB  
sigma = 1.0   # SAME as MATLAB

# Reaction parameters - DIFFERENCES from MATLAB
a = 1.0       # DIFFERENT: MATLAB uses a = 0.0
c = 1.0       # DIFFERENT: MATLAB uses c = 0.0

# Coupling parameters - SAME as MATLAB
b = 1.0       # SAME as MATLAB
d = 1.0       # SAME as MATLAB  
chi = 1.0     # SAME as MATLAB

# Domain configuration - DIFFERENT from MATLAB
domain_start = 1.0    # DIFFERENT: MATLAB uses A = 0
domain_length = 1.0   # SAME: MATLAB uses L = 1.0

# Parameter vector assembly [nu, mu, epsilon, sigma, a, b, c, d, chi]
parameters = np.array([nu, mu, epsilon, sigma, a, b, c, d, chi])
\end{lstlisting}

\paragraph{Nonlinear Function Configuration}
\begin{lstlisting}[language=Python, caption=Python Nonlinear Function Setup]
# DIFFERENT from MATLAB: Non-constant nonlinearity
lambda_function = lambda x: 1.0/(1.0 + x**2)      # Saturation-type
dlambda_function = lambda x: -2.0*x/(1.0 + x**2)**2  # Analytical derivative

# MATLAB equivalent (for comparison):
# lambda = @(x) constant_function(x);  % Returns ones(size(x))
\end{lstlisting}

\paragraph{Initial Condition Functions}
\begin{lstlisting}[language=Python, caption=Python Initial Conditions - Complex]
def initial_u(s, t=0.0):
    """Cell density initial condition"""
    s = np.asarray(s)
    return 0.0 * s  # Zero initial condition (like EmptyProblem.m)

def initial_omega(s, t=0.0):
    """Chemical concentration - TIME-DEPENDENT initial condition"""
    s = np.asarray(s)
    return np.sin(2 * np.pi * s + np.pi * t)  # MATLAB: zeros(size(x))

def initial_v(s, t=0.0):
    """Velocity field - SPACE-TIME coupled initial condition"""
    s = np.asarray(s)
    return t * s  # MATLAB: zeros(size(x))

def initial_phi(s, t=0.0):
    """Potential field - QUADRATIC spatial profile"""
    return s ** 2  # MATLAB: zeros(size(x))
\end{lstlisting}

\paragraph{Source Term Functions}
\begin{lstlisting}[language=Python, caption=Python Source Terms - Non-Zero]
def force_u(s, t):
    """Cell density source - remains zero"""
    s = np.asarray(s)
    return np.zeros_like(s)  # SAME as MATLAB

def force_omega(s, t):
    """Chemical source - COMPLEX analytical form"""
    s = np.asarray(s)
    x = 2 * np.pi * s + np.pi * t
    return np.sin(x) + 4 * np.pi**2 * np.sin(x) + np.pi * np.cos(x)
    # MATLAB: zeros(size(x))

def force_v(s, t):
    """Velocity source - NONLINEAR coupling"""
    omega_val = initial_omega(s, t)
    lambda_val = lambda_function(omega_val)
    s = np.asarray(s)
    return s + lambda_val * t * s
    # MATLAB: zeros(size(x))

def force_phi(s, t):
    """Potential source - MULTI-TERM expression"""
    s = np.asarray(s)
    return - mu * 2.0 * np.ones_like(s) + a * s**2 - b * t * s
    # MATLAB: zeros(size(x))
\end{lstlisting}

\paragraph{Problem Instance Creation}
\begin{lstlisting}[language=Python, caption=Python Problem Instance Setup]
# Create Problem instance with BioNetFlux framework
problem = Problem(
    neq=neq,                    # 4 equations
    domain_start=domain_start,  # 1.0 (vs MATLAB A=0) 
    domain_length=domain_length, # 1.0 (same as MATLAB L)
    parameters=parameters,       # 9-element parameter vector
    problem_type="organ_on_chip", # Problem classification
    name="ooc_test"             # Instance identifier
)

# Set nonlinear functions using flexible interface
problem.set_function('lambda_function', lambda_function)  
problem.set_function('dlambda_function', dlambda_function)

# Set source terms for all 4 equations
problem.set_force(0, lambda s, t: force_u(s, t))      # u equation
problem.set_force(1, lambda s, t: force_omega(s, t))  # omega equation  
problem.set_force(2, lambda s, t: force_v(s, t))      # v equation
problem.set_force(3, lambda s, t: force_phi(s, t))    # phi equation

# Set initial conditions for all 4 equations
problem.set_initial_condition(0, initial_u)   # u initial condition
problem.set_initial_condition(1, initial_omega) # omega initial condition
problem.set_initial_condition(2, initial_v)   # v initial condition
problem.set_initial_condition(3, initial_phi) # phi initial condition
\end{lstlisting}

\paragraph{Discretization Configuration}
\begin{lstlisting}[language=Python, caption=Python Discretization Setup]
# Spatial discretization
discretization = Discretization(
    n_elements=n_elements,        # 20 elements
    domain_start=domain_start,    # 1.0
    domain_length=domain_length,  # 1.0
    stab_constant=1.0            # Stabilization constant
)

# Stabilization parameters for HDG method
tau_u = 1.0 / discretization.element_length     # Scaled by mesh size
tau_omega = 1.0                                 # Constant
tau_v = 1.0                                     # Constant  
tau_phi = 1.0                                   # Constant

discretization.set_tau([tau_u, tau_omega, tau_v, tau_phi])

# Global discretization with time parameters
global_disc = GlobalDiscretization([discretization])
global_disc.set_time_parameters(dt, T)  # dt=0.1, T=1.0
\end{lstlisting}

\paragraph{Boundary Condition Configuration}
\begin{lstlisting}[language=Python, caption=Python Boundary Condition Setup]
# Analytical flux functions for Neumann boundary conditions
flux_u = lambda s, t: 0.0  # SAME as MATLAB: zero flux
flux_omega = lambda s, t: 2 * np.pi * np.cos(2 * np.pi * s + np.pi * t)  # NON-ZERO
flux_v = lambda s, t: t    # TIME-DEPENDENT flux  
flux_phi = lambda s, t: 2 * s  # SPACE-DEPENDENT flux

# Create constraint manager
constraint_manager = ConstraintManager()
domain_end = domain_start + domain_length  # 2.0

# Add Neumann boundary conditions for all equations
# Left boundary (x = domain_start = 1.0)
constraint_manager.add_neumann(0, 0, domain_start, lambda t: -flux_u(domain_start, t))
constraint_manager.add_neumann(1, 0, domain_start, lambda t: -flux_omega(domain_start, t))
constraint_manager.add_neumann(2, 0, domain_start, lambda t: -flux_v(domain_start, t))
constraint_manager.add_neumann(3, 0, domain_start, lambda t: -flux_phi(domain_start, t))

# Right boundary (x = domain_end = 2.0)  
constraint_manager.add_neumann(0, 0, domain_end, lambda t: flux_u(domain_end, t))
constraint_manager.add_neumann(1, 0, domain_end, lambda t: flux_omega(domain_end, t))
constraint_manager.add_neumann(2, 0, domain_end, lambda t: flux_v(domain_end, t))
constraint_manager.add_neumann(3, 0, domain_end, lambda t: flux_phi(domain_end, t))

# Map constraints to discretizations
constraint_manager.map_to_discretizations([discretization])
\end{lstlisting}

\paragraph{Return Value Assembly}
\begin{lstlisting}[language=Python, caption=create\_global\_framework Return]
# Return framework components as expected by BioNetFlux
return [problem], global_disc, constraint_manager, problem_name

# Return structure:
# - [problem]: List containing single Problem instance
# - global_disc: GlobalDiscretization with time and space parameters  
# - constraint_manager: ConstraintManager with mapped boundary conditions
# - problem_name: String identifier for the problem configuration
\end{lstlisting}

\subsection{MATLAB vs Python Implementation Comparison}
\label{subsec:comprehensive_comparison}

\subsubsection{Parameter Value Differences}

\begin{longtable}{|p{2.5cm}|p{2cm}|p{2cm}|p{6.5cm}|}
\hline
\textbf{Parameter} & \textbf{MATLAB} & \textbf{Python} & \textbf{Mathematical Impact} \\
\hline
\endhead

$\mu$ (potential diffusion) & 2.0 & 1.0 & Reduced potential diffusion in Python \\
\hline

$a$ (growth rate) & 0.0 & 1.0 & Added growth terms: $+u$ in eq.~\eqref{eq:u_equation}, $+\phi$ in eq.~\eqref{eq:phi_equation} \\
\hline

$c$ (chemical reaction) & 0.0 & 1.0 & Added reaction term: $+\omega$ in eq.~\eqref{eq:omega_equation} \\
\hline

Domain start & 0.0 & 1.0 & Shifted domain from $[0,1]$ to $[1,2]$ \\
\hline

$\lambda(x)$ function & $\text{constant}$ & $\frac{1}{1+x^2}$ & Nonlinear saturation vs linear behavior \\
\hline

\end{longtable}

\subsubsection{Initial Condition Complexity}

\begin{longtable}{|p{2cm}|p{3.5cm}|p{3.5cm}|p{4cm}|}
\hline
\textbf{Variable} & \textbf{MATLAB TestProblem.m} & \textbf{Python ooc\_test\_problem.py} & \textbf{Complexity Level} \\
\hline
\endhead

$u(x,0)$ & $\sin(2\pi x)$ & $0$ & MATLAB: Spatial oscillation \\
\hline

$\omega(x,0)$ & $0$ & $\sin(2\pi x + \pi t)$ & Python: Spatio-temporal \\
\hline

$v(x,0)$ & $0$ & $t \cdot x$ & Python: Space-time product \\
\hline

$\phi(x,0)$ & $0$ & $x^2$ & Python: Quadratic profile \\
\hline

\end{longtable}

\subsubsection{Source Term Complexity}

\begin{longtable}{|p{2cm}|p{5cm}|p{6cm}|}
\hline
\textbf{Equation} & \textbf{MATLAB (All Problems)} & \textbf{Python Implementation} \\
\hline
\endhead

$f_u$ & $0$ & $0$ \\
\hline

$f_\omega$ & $0$ & $\sin(x) + 4\pi^2\sin(x) + \pi\cos(x)$ where $x = 2\pi s + \pi t$ \\
\hline

$f_v$ & $0$ & $s + \lambda(\omega(s,t)) \cdot t \cdot s$ \\
\hline

$f_\phi$ & $0$ & $-\mu \cdot 2 + a s^2 - b t s$ \\
\hline

\end{longtable}

\subsubsection{Boundary Condition Differences}

\begin{longtable}{|p{2.5cm}|p{3cm}|p{3cm}|p{4.5cm}|}
\hline
\textbf{Flux Type} & \textbf{MATLAB} & \textbf{Python} & \textbf{Mathematical Form} \\
\hline
\endhead

$u$ flux & Zero & Zero & No difference \\
\hline

$\omega$ flux & Zero & $2\pi\cos(2\pi s + \pi t)$ & Time-dependent analytical flux \\
\hline

$v$ flux & Zero & $t$ & Linear time dependence \\
\hline

$\phi$ flux & Zero & $2s$ & Linear spatial dependence \\
\hline

\end{longtable}

\subsection{Framework Integration Patterns}
\label{subsec:framework_integration_patterns}

\subsubsection{Problem Instance Configuration}

\begin{lstlisting}[language=Python, caption=Standard Problem Configuration Pattern]
def create_global_framework():
    """Standard pattern for problem module implementation"""
    
    # 1. Parameter Definition
    # Define all physical, reaction, and coupling parameters
    parameters = np.array([nu, mu, epsilon, sigma, a, b, c, d, chi])
    
    # 2. Domain Configuration  
    # Set domain start, length, and mesh parameters
    domain_start, domain_length, n_elements = ...
    
    # 3. Function Definitions
    # Define initial conditions, source terms, and nonlinear functions
    def initial_condition_i(s, t): ...
    def force_function_i(s, t): ...
    lambda_function = lambda x: ...
    
    # 4. Problem Creation
    problem = Problem(neq, domain_start, domain_length, parameters, ...)
    problem.set_initial_condition(i, initial_condition_i)
    problem.set_force(i, force_function_i)
    problem.set_function('lambda_function', lambda_function)
    
    # 5. Discretization Setup
    discretization = Discretization(n_elements, domain_start, domain_length, ...)
    discretization.set_tau([tau_0, tau_1, tau_2, tau_3])
    
    # 6. Time Integration Configuration
    global_disc = GlobalDiscretization([discretization])
    global_disc.set_time_parameters(dt, T)
    
    # 7. Boundary Condition Setup
    constraint_manager = ConstraintManager()
    constraint_manager.add_neumann/add_dirichlet(...)
    constraint_manager.map_to_discretizations([discretization])
    
    # 8. Return Framework Components
    return [problem], global_disc, constraint_manager, problem_name
\end{lstlisting}

\subsubsection{Multi-Domain Extension Pattern}

\begin{lstlisting}[language=Python, caption=Expected Multi-Domain Problem Pattern]
def create_multi_domain_framework():
    """Expected pattern for multi-domain problems"""
    
    problems = []
    discretizations = []
    
    # Create multiple domains
    for domain_idx, domain_config in enumerate(domain_configurations):
        # Domain-specific parameters
        domain_parameters = modify_parameters_for_domain(base_parameters, domain_config)
        
        # Create domain problem
        problem = Problem(
            neq=neq,
            domain_start=domain_config['start'],
            domain_length=domain_config['length'],
            parameters=domain_parameters
        )
        
        # Set domain-specific functions
        problem.set_initial_condition(i, lambda s, t: domain_initial_i(s, t, domain_idx))
        problem.set_force(i, lambda s, t: domain_force_i(s, t, domain_idx))
        
        problems.append(problem)
        
        # Create domain discretization
        discretization = Discretization(...)
        discretizations.append(discretization)
    
    # Global discretization and time setup
    global_disc = GlobalDiscretization(discretizations)
    global_disc.set_time_parameters(dt, T)
    
    # Inter-domain constraints (junctions)
    constraint_manager = ConstraintManager()
    
    # Add junction constraints between domains
    for junction in junction_configurations:
        constraint_manager.add_trace_continuity(...)
        constraint_manager.add_flux_continuity(...)
        constraint_manager.add_kedem_katchalsky(...)
    
    # Add boundary constraints for exterior boundaries
    constraint_manager.add_dirichlet/add_neumann(...)
    
    constraint_manager.map_to_discretizations(discretizations)
    
    return problems, global_disc, constraint_manager, problem_name
\end{lstlisting}

\subsection{Parameter Sensitivity Analysis Methods}
\label{subsec:parameter_sensitivity_methods}

\subsubsection{Parameter Variation Utilities}

\begin{lstlisting}[language=Python, caption=Parameter Sensitivity Analysis Pattern]
def create_parameter_sweep(base_problem_module, parameter_variations):
    """
    Create multiple problem instances for parameter sensitivity studies
    
    Args:
        base_problem_module: String path to base problem module
        parameter_variations: Dict mapping parameter names to value lists
        
    Returns:
        List of (parameter_set, framework_components) tuples
    """
    
    # Load base framework
    base_problems, base_global_disc, base_constraints, base_name = \
        importlib.import_module(base_problem_module).create_global_framework()
    
    base_problem = base_problems[0]
    base_parameters = base_problem.parameters.copy()
    
    # Parameter index mapping
    param_indices = {
        'nu': 0, 'mu': 1, 'epsilon': 2, 'sigma': 3,
        'a': 4, 'b': 5, 'c': 6, 'd': 7, 'chi': 8
    }
    
    results = []
    
    # Generate parameter combinations
    for param_name, param_values in parameter_variations.items():
        param_idx = param_indices[param_name]
        
        for param_value in param_values:
            # Create modified parameters
            modified_params = base_parameters.copy()
            modified_params[param_idx] = param_value
            
            # Create new problem with modified parameters
            modified_problem = Problem(
                neq=base_problem.neq,
                domain_start=base_problem.domain_start,
                domain_length=base_problem.domain_length,
                parameters=modified_params,
                problem_type=base_problem.type,
                name=f"{base_problem.name}_{param_name}_{param_value}"
            )
            
            # Copy functions and conditions from base problem
            for i in range(base_problem.neq):
                if base_problem.initial_conditions[i] is not None:
                    modified_problem.set_initial_condition(i, base_problem.initial_conditions[i])
                if base_problem.force_functions[i] is not None:
                    modified_problem.set_force(i, base_problem.force_functions[i])
            
            # Copy additional functions
            for func_name, func in base_problem.additional_functions.items():
                modified_problem.set_function(func_name, func)
            
            # Create modified framework components
            modified_framework = (
                [modified_problem],
                base_global_disc,  # Reuse discretization and time setup
                base_constraints,  # Reuse boundary conditions
                f"{base_name} - {param_name}={param_value}"
            )
            
            results.append(((param_name, param_value), modified_framework))
    
    return results

# Usage example
parameter_variations = {
    'chi': [0.1, 0.5, 1.0, 2.0, 5.0],  # Chemotactic sensitivity
    'mu': [0.5, 1.0, 2.0, 4.0],        # Potential diffusion
    'a': [0.0, 0.5, 1.0, 1.5]          # Growth rate
}

sensitivity_study = create_parameter_sweep(
    'ooc1d.problems.ooc_test_problem',
    parameter_variations
)
\end{lstlisting}

\subsection{Problem Module Summary}
\label{subsec:problem_module_summary_detailed}

\subsubsection{Implemented Problems Overview}

\begin{longtable}{|p{4.5cm}|p{2cm}|p{3cm}|p{4.5cm}|}
\hline
\textbf{Module/File} & \textbf{Domain} & \textbf{Complexity} & \textbf{Key Features} \\
\hline
\endhead

\texttt{TestProblem.m} & $[0,1]$ & Medium & Non-zero $u$ initial condition, constant $\lambda$ \\
\hline

\texttt{EmptyProblem.m} & $[0,1]$ & Low & All zero initial conditions, template structure \\
\hline

\texttt{ooc\_test\_problem.py} & $[1,2]$ & High & Rich analytical test case, nonlinear $\lambda$ \\
\hline

\end{longtable}

\subsubsection{Framework Integration Summary}

\begin{longtable}{|p{5cm}|p{8cm}|}
\hline
\textbf{Component} & \textbf{Implementation Details} \\
\hline
\endhead

\texttt{create\_global\_framework} & Returns (problems, global\_discretization, constraint\_manager, problem\_name) \\
\hline

Parameter vector & Standardized 9-element array: $[\nu, \mu, \epsilon, \sigma, a, b, c, d, \chi]$ \\
\hline

Initial conditions & 4 functions for $(u, \omega, v, \phi)$ with optional time dependence \\
\hline

Source terms & 4 functions for $(f_u, f_\omega, f_v, f_\phi)$ with spatio-temporal dependence \\
\hline

Nonlinear functions & $\lambda(x)$ and $\lambda'(x)$ for velocity equation nonlinearity \\
\hline

Boundary conditions & Neumann flux specifications for all 4 equations at both boundaries \\
\hline

Time discretization & Uniform time stepping with configurable $dt$ and final time $T$ \\
\hline

Space discretization & Uniform mesh with configurable element count and stabilization parameters \\
\hline

\end{longtable}

\subsection{Expected Extensions and Variations}
\label{subsec:expected_extensions}

\subsubsection{Alternative Nonlinear Functions}

\begin{lstlisting}[language=Python, caption=Expected Alternative Lambda Functions]
# Michaelis-Menten kinetics
lambda_mm = lambda x, K=1.0, V_max=2.0: V_max * x / (K + x)
dlambda_mm = lambda x, K=1.0, V_max=2.0: V_max * K / (K + x)**2

# Hill function (cooperative binding)  
lambda_hill = lambda x, K=1.0, n=2.0: x**n / (K**n + x**n)
dlambda_hill = lambda x, K=1.0, n=2.0: n * x**(n-1) * K**n / (K**n + x**n)**2

# Exponential saturation
lambda_exp = lambda x, alpha=1.0, beta=2.0: alpha * (1.0 - np.exp(-beta * x))
dlambda_exp = lambda x, alpha=1.0, beta=2.0: alpha * beta * np.exp(-beta * x)
\end{lstlisting}

\subsubsection{Expected Multi-Domain Problem Structure}

\begin{lstlisting}[language=Python, caption=Expected Multi-Domain Problem Template]
def create_vascular_network_framework():
    """Expected vascular network problem with 3 domains"""
    
    # Domain configurations
    domains = [
        {"start": 0.0, "length": 1.0, "chi": 1.0, "name": "vessel_1"},
        {"start": 1.0, "length": 0.5, "chi": 2.0, "name": "junction"},  
        {"start": 1.5, "length": 1.0, "chi": 0.5, "name": "vessel_2"}
    ]
    
    problems = []
    discretizations = []
    
    for i, domain_config in enumerate(domains):
        # Domain-specific parameter modifications
        parameters = base_parameters.copy()
        parameters[8] = domain_config["chi"]  # Modify chemotactic sensitivity
        
        # Create domain problem
        problem = Problem(
            neq=4,
            domain_start=domain_config["start"],
            domain_length=domain_config["length"],
            parameters=parameters,
            problem_type="vascular_network",
            name=domain_config["name"]
        )
        
        # Domain-specific initial conditions and source terms
        problem.set_initial_condition(0, lambda s, t: domain_initial_u(s, t, i))
        # ... set other conditions ...
        
        problems.append(problem)
        discretizations.append(create_domain_discretization(domain_config))
    
    # Junction constraints
    constraint_manager = ConstraintManager()
    
    # Continuity at domain interfaces
    constraint_manager.add_trace_continuity(0, 0, 1, -1, 1.0)  # u continuity
    constraint_manager.add_trace_continuity(1, 0, 2, -1, 1.5)  # omega continuity
    
    # Flux continuity with permeability effects
    constraint_manager.add_kedem_katchalsky(2, 0, 1, -1, 1.0, P=0.1)  # v coupling
    constraint_manager.add_kedem_katchalsky(3, 0, 2, -1, 1.5, P=0.2)  # phi coupling
    
    # Boundary conditions at network ends
    constraint_manager.add_dirichlet(0, 0, 0.0, lambda t: 1.0)  # u inlet
    constraint_manager.add_neumann(0, 2, 2.5, lambda t: 0.0)    # u outlet
    
    global_disc = GlobalDiscretization(discretizations)
    global_disc.set_time_parameters(dt=0.01, T=2.0)
    
    constraint_manager.map_to_discretizations(discretizations)
    
    return problems, global_disc, constraint_manager, "Vascular Network Model"
\end{lstlisting}

This comprehensive documentation provides exact details of the problems folder modules, focusing specifically on the 'create\_global\_framework' implementations and their parameter configurations, without duplicating information about other BioNetFlux components that are documented separately.

% End of problems folder detailed API documentation


\section{Problems Folder Detailed API Reference}
\label{sec:problems_folder_detailed_api}

This section provides detailed documentation specifically for the problem modules in the BioNetFlux problems folder, focusing on the mathematical formulations, parameter specifications, and `create\_global\_framework` method implementations based on the actual MATLAB reference files and Python implementations.

\subsection{Problems Folder Structure}

The problems folder contains problem-specific modules that implement the mathematical models for various biological transport phenomena:

\begin{itemize}
    \item \textbf{MATLAB Reference Files}: Original problem definitions and parameters
    \item \textbf{Python Implementation Modules}: BioNetFlux-compatible problem setups
    \item \textbf{Mathematical Model Specifications}: 4-equation OrganOnChip system parameters
    \item \textbf{Boundary Condition Configurations}: Dirichlet and Neumann boundary setups
\end{itemize}

\subsection{Mathematical Model: OrganOnChip System}
\label{subsec:ooc_mathematical_system}

All problems in the folder implement variants of the 4-equation OrganOnChip system:

\subsubsection{Governing System}

\begin{align}
\frac{\partial u}{\partial t} &= \nu \nabla^2 u - \chi \nabla \cdot (u \nabla \phi) + a u + f_u(x,t) \label{eq:u_equation}\\
\frac{\partial \omega}{\partial t} &= \epsilon \nabla^2 \omega + c \omega + d u + f_\omega(x,t) \label{eq:omega_equation}\\
\frac{\partial v}{\partial t} &= \sigma \nabla^2 v + \lambda(\bar{\omega}) v + f_v(x,t) \label{eq:v_equation}\\
\frac{\partial \phi}{\partial t} &= \mu \nabla^2 \phi + a \phi + b v + f_\phi(x,t) \label{eq:phi_equation}
\end{align}

where $\bar{\omega} = \frac{1}{|K|} \int_K \omega \, dx$ is the cell-averaged value of $\omega$.

\subsubsection{Parameter Vector Structure}

All problem modules use the standardized parameter vector:
\begin{align}
\text{parameters} = [\nu, \mu, \epsilon, \sigma, a, b, c, d, \chi]
\end{align}

\subsection{MATLAB Reference Problems}
\label{subsec:matlab_reference_detailed}

\subsubsection{TestProblem.m Analysis}

\paragraph{Complete Parameter Configuration}
\begin{lstlisting}[language=Matlab, caption=TestProblem.m Full Parameter Setup]
% Viscosity parameters (diffusion coefficients)
nu = 1.0;      % Cell diffusion coefficient
mu = 2.0;      % Potential diffusion coefficient  
epsilon = 1.0; % Chemical diffusion coefficient
sigma = 1.0;   % Velocity diffusion coefficient

% Reaction parameters
a = 0.0;       % Growth/decay rate (u and phi equations)
c = 0.0;       % Chemical reaction rate (omega equation)

% Domain specification
A = 0;         % Domain start coordinate
L = 1.0;       % Domain length

% Coupling parameters
b = 1.0;       % Velocity-potential coupling strength
d = 1.0;       % Cell-chemical coupling strength
chi = 1.0;     % Chemotactic sensitivity parameter

% Nonlinearity specification
lambda = @(x) constant_function(x);  % Constant nonlinear function
\end{lstlisting}

\paragraph{OoC\_pbParameters Function}
\begin{lstlisting}[language=Matlab, caption=Parameter Structure Creation]
function problem = OoC_pbParameters(a,b,c,d,chi,lambda,mu,nu,epsilon,sigma,A,L)
    s = whos; % Get all workspace variables
    problem = cell2struct({s.name}.', {s.name}); % Create structure
    eval(structvars(problem,0).'); % Evaluate structure variables
end
\end{lstlisting}

\paragraph{Initial Conditions Specification}
\begin{lstlisting}[language=Matlab, caption=TestProblem.m Initial Conditions]
% Cell density u: Non-trivial sinusoidal profile
problem.u0{1} = @(x,t) sin(2*pi*x);  

% Chemical concentration omega: Zero initial condition
problem.u0{2} = @(x,t) zeros(size(x));

% Velocity field v: Zero initial condition  
problem.u0{3} = @(x,t) zeros(size(x));

% Potential field phi: Zero initial condition
problem.u0{4} = @(x,t) zeros(size(x));

% Gradient initial condition
problem.phix0 = @(x,t) zeros(size(x));
\end{lstlisting}

\paragraph{Source Term Configuration}
\begin{lstlisting}[language=Matlab, caption=TestProblem.m Source Terms]
% All source terms are zero - homogeneous problem
force{1} = @(x,t) zeros(size(x));  % f_u = 0
force{2} = @(x,t) zeros(size(x));  % f_omega = 0  
force{3} = @(x,t) zeros(size(x));  % f_v = 0
force{4} = @(x,t) zeros(size(x));  % f_phi = 0
\end{lstlisting}

\paragraph{Boundary Condition Setup}
\begin{lstlisting}[language=Matlab, caption=TestProblem.m Boundary Conditions]
% Neumann flux conditions at both boundaries
% Left boundary (x = A = 0)
problem.fluxu0{1} = @(t) 0.;  % Zero flux for u
problem.fluxu0{2} = @(t) 0.;  % Zero flux for omega
problem.fluxu0{3} = @(t) 0.;  % Zero flux for v
problem.fluxu0{4} = @(t) 0.;  % Zero flux for phi

% Right boundary (x = A + L = 1)
problem.fluxu1{1} = @(t) 0.;  % Zero flux for u
problem.fluxu1{2} = @(t) 0.;  % Zero flux for omega
problem.fluxu1{3} = @(t) 0.;  % Zero flux for v
problem.fluxu1{4} = @(t) 0.;  % Zero flux for phi

% Neumann data vector (8×1: 2 boundaries × 4 equations)
problem.NeumannData = zeros(8,1);
\end{lstlisting}

\subsubsection{EmptyProblem.m Analysis}

\paragraph{Parameter Differences from TestProblem.m}
\begin{lstlisting}[language=Matlab, caption=EmptyProblem.m Parameter Configuration]
% IDENTICAL to TestProblem.m:
nu = 1.0; mu = 2.0; epsilon = 1.0; sigma = 1.0;
a = 0.0; c = 0.0;
A = 0; L = 1.0;
b = 1.0; d = 1.0; chi = 1.0;
lambda = @(x) constant_function(x);
\end{lstlisting}

\paragraph{Key Difference: Initial Conditions}
\begin{lstlisting}[language=Matlab, caption=EmptyProblem.m Initial Conditions]
% ALL initial conditions are zero (unlike TestProblem.m)
problem.u0{1} = @(x,t) zeros(size(x));  % Zero cell density
problem.u0{2} = @(x,t) zeros(size(x));  % Zero chemical
problem.u0{3} = @(x,t) zeros(size(x));  % Zero velocity
problem.u0{4} = @(x,t) zeros(size(x));  % Zero potential

problem.phix0 = @(x,t) zeros(size(x));  % Zero gradient
\end{lstlisting}

\paragraph{Source Terms and Boundary Conditions}
\begin{lstlisting}[language=Matlab, caption=EmptyProblem.m Additional Configuration]
% Source terms - also all zero
force_u = @(x,t) zeros(size(x));
force_phi = @(x,t) zeros(size(x));  
force_v = @(x,t) zeros(size(x));
force_omega = @(x,t) zeros(size(x));

% Boundary conditions - identical to TestProblem.m
% All zero Neumann flux conditions
problem.fluxu0{1-4} = @(t) 0.;  % Left boundary
problem.fluxu1{1-4} = @(t) 0.;  % Right boundary
problem.NeumannData = zeros(8,1);
\end{lstlisting}

\subsubsection{MATLAB Problems Comparison}

\begin{longtable}{|p{3cm}|p{4cm}|p{4cm}|p{2cm}|}
\hline
\textbf{Aspect} & \textbf{TestProblem.m} & \textbf{EmptyProblem.m} & \textbf{Purpose} \\
\hline
\endhead

Parameters & $\nu=1, \mu=2, \epsilon=1, \sigma=1$ & Identical & Reference values \\
\hline

Reaction & $a=0, c=0$ & Identical & No reaction terms \\
\hline

Coupling & $b=1, d=1, \chi=1$ & Identical & Unit coupling \\
\hline

Domain & $[0, 1]$ & Identical & Unit interval \\
\hline

Nonlinearity & $\lambda(x) = \text{constant}$ & Identical & Linear case \\
\hline

Initial u & $\sin(2\pi x)$ & $0$ & Non-trivial vs zero \\
\hline

Initial $\omega,v,\phi$ & $0$ & $0$ & Both zero \\
\hline

Source terms & All zero & All zero & Homogeneous \\
\hline

Boundary conditions & Zero Neumann & Zero Neumann & Natural boundaries \\
\hline

Use case & Non-trivial initial test & Baseline/template & Testing framework \\
\hline

\end{longtable}

\subsection{Python Implementation: ooc\_test\_problem.py}
\label{subsec:python_ooc_implementation}

\subsubsection{create\_global\_framework Method Analysis}

\paragraph{Method Signature and Overview}
\begin{lstlisting}[language=Python, caption=create\_global\_framework Method Signature]
def create_global_framework():
    """
    OrganOnChip test problem - Python port from MATLAB TestProblem.m
    
    Returns:
        Tuple: (problems, global_discretization, constraint_manager, problem_name)
            - problems: List[Problem] - Single domain problem instance
            - global_discretization: GlobalDiscretization - Time and space discretization
            - constraint_manager: ConstraintManager - Boundary condition manager
            - problem_name: str - Descriptive problem name
    """
\end{lstlisting}

\paragraph{Mesh and Global Configuration}
\begin{lstlisting}[language=Python, caption=Python Global Configuration]
# Mesh parameters
n_elements = 20  # Spatial discretization fineness

# Global problem structure  
ndom = 1        # Single domain
neq = 4         # Four coupled equations
T = 1.0         # Final simulation time
dt = 0.1        # Time step size

problem_name = "OrganOnChip Test Problem"
\end{lstlisting}

\paragraph{Parameter Configuration Differences}
\begin{lstlisting}[language=Python, caption=Python Parameter Setup vs MATLAB]
# Physical parameters - DIFFERENCES from MATLAB highlighted
nu = 1.0      # SAME as MATLAB
mu = 1.0      # DIFFERENT: MATLAB uses mu = 2.0
epsilon = 1.0 # SAME as MATLAB  
sigma = 1.0   # SAME as MATLAB

# Reaction parameters - DIFFERENCES from MATLAB
a = 1.0       # DIFFERENT: MATLAB uses a = 0.0
c = 1.0       # DIFFERENT: MATLAB uses c = 0.0

# Coupling parameters - SAME as MATLAB
b = 1.0       # SAME as MATLAB
d = 1.0       # SAME as MATLAB  
chi = 1.0     # SAME as MATLAB

# Domain configuration - DIFFERENT from MATLAB
domain_start = 1.0    # DIFFERENT: MATLAB uses A = 0
domain_length = 1.0   # SAME: MATLAB uses L = 1.0

# Parameter vector assembly [nu, mu, epsilon, sigma, a, b, c, d, chi]
parameters = np.array([nu, mu, epsilon, sigma, a, b, c, d, chi])
\end{lstlisting}

\paragraph{Nonlinear Function Configuration}
\begin{lstlisting}[language=Python, caption=Python Nonlinear Function Setup]
# DIFFERENT from MATLAB: Non-constant nonlinearity
lambda_function = lambda x: 1.0/(1.0 + x**2)      # Saturation-type
dlambda_function = lambda x: -2.0*x/(1.0 + x**2)**2  # Analytical derivative

# MATLAB equivalent (for comparison):
# lambda = @(x) constant_function(x);  % Returns ones(size(x))
\end{lstlisting}

\paragraph{Initial Condition Functions}
\begin{lstlisting}[language=Python, caption=Python Initial Conditions - Complex]
def initial_u(s, t=0.0):
    """Cell density initial condition"""
    s = np.asarray(s)
    return 0.0 * s  # Zero initial condition (like EmptyProblem.m)

def initial_omega(s, t=0.0):
    """Chemical concentration - TIME-DEPENDENT initial condition"""
    s = np.asarray(s)
    return np.sin(2 * np.pi * s + np.pi * t)  # MATLAB: zeros(size(x))

def initial_v(s, t=0.0):
    """Velocity field - SPACE-TIME coupled initial condition"""
    s = np.asarray(s)
    return t * s  # MATLAB: zeros(size(x))

def initial_phi(s, t=0.0):
    """Potential field - QUADRATIC spatial profile"""
    return s ** 2  # MATLAB: zeros(size(x))
\end{lstlisting}

\paragraph{Source Term Functions}
\begin{lstlisting}[language=Python, caption=Python Source Terms - Non-Zero]
def force_u(s, t):
    """Cell density source - remains zero"""
    s = np.asarray(s)
    return np.zeros_like(s)  # SAME as MATLAB

def force_omega(s, t):
    """Chemical source - COMPLEX analytical form"""
    s = np.asarray(s)
    x = 2 * np.pi * s + np.pi * t
    return np.sin(x) + 4 * np.pi**2 * np.sin(x) + np.pi * np.cos(x)
    # MATLAB: zeros(size(x))

def force_v(s, t):
    """Velocity source - NONLINEAR coupling"""
    omega_val = initial_omega(s, t)
    lambda_val = lambda_function(omega_val)
    s = np.asarray(s)
    return s + lambda_val * t * s
    # MATLAB: zeros(size(x))

def force_phi(s, t):
    """Potential source - MULTI-TERM expression"""
    s = np.asarray(s)
    return - mu * 2.0 * np.ones_like(s) + a * s**2 - b * t * s
    # MATLAB: zeros(size(x))
\end{lstlisting}

\paragraph{Problem Instance Creation}
\begin{lstlisting}[language=Python, caption=Python Problem Instance Setup]
# Create Problem instance with BioNetFlux framework
problem = Problem(
    neq=neq,                    # 4 equations
    domain_start=domain_start,  # 1.0 (vs MATLAB A=0) 
    domain_length=domain_length, # 1.0 (same as MATLAB L)
    parameters=parameters,       # 9-element parameter vector
    problem_type="organ_on_chip", # Problem classification
    name="ooc_test"             # Instance identifier
)

# Set nonlinear functions using flexible interface
problem.set_function('lambda_function', lambda_function)  
problem.set_function('dlambda_function', dlambda_function)

# Set source terms for all 4 equations
problem.set_force(0, lambda s, t: force_u(s, t))      # u equation
problem.set_force(1, lambda s, t: force_omega(s, t))  # omega equation  
problem.set_force(2, lambda s, t: force_v(s, t))      # v equation
problem.set_force(3, lambda s, t: force_phi(s, t))    # phi equation

# Set initial conditions for all 4 equations
problem.set_initial_condition(0, initial_u)   # u initial condition
problem.set_initial_condition(1, initial_omega) # omega initial condition
problem.set_initial_condition(2, initial_v)   # v initial condition
problem.set_initial_condition(3, initial_phi) # phi initial condition
\end{lstlisting}

\paragraph{Discretization Configuration}
\begin{lstlisting}[language=Python, caption=Python Discretization Setup]
# Spatial discretization
discretization = Discretization(
    n_elements=n_elements,        # 20 elements
    domain_start=domain_start,    # 1.0
    domain_length=domain_length,  # 1.0
    stab_constant=1.0            # Stabilization constant
)

# Stabilization parameters for HDG method
tau_u = 1.0 / discretization.element_length     # Scaled by mesh size
tau_omega = 1.0                                 # Constant
tau_v = 1.0                                     # Constant  
tau_phi = 1.0                                   # Constant

discretization.set_tau([tau_u, tau_omega, tau_v, tau_phi])

# Global discretization with time parameters
global_disc = GlobalDiscretization([discretization])
global_disc.set_time_parameters(dt, T)  # dt=0.1, T=1.0
\end{lstlisting}

\paragraph{Boundary Condition Configuration}
\begin{lstlisting}[language=Python, caption=Python Boundary Condition Setup]
# Analytical flux functions for Neumann boundary conditions
flux_u = lambda s, t: 0.0  # SAME as MATLAB: zero flux
flux_omega = lambda s, t: 2 * np.pi * np.cos(2 * np.pi * s + np.pi * t)  # NON-ZERO
flux_v = lambda s, t: t    # TIME-DEPENDENT flux  
flux_phi = lambda s, t: 2 * s  # SPACE-DEPENDENT flux

# Create constraint manager
constraint_manager = ConstraintManager()
domain_end = domain_start + domain_length  # 2.0

# Add Neumann boundary conditions for all equations
# Left boundary (x = domain_start = 1.0)
constraint_manager.add_neumann(0, 0, domain_start, lambda t: -flux_u(domain_start, t))
constraint_manager.add_neumann(1, 0, domain_start, lambda t: -flux_omega(domain_start, t))
constraint_manager.add_neumann(2, 0, domain_start, lambda t: -flux_v(domain_start, t))
constraint_manager.add_neumann(3, 0, domain_start, lambda t: -flux_phi(domain_start, t))

# Right boundary (x = domain_end = 2.0)  
constraint_manager.add_neumann(0, 0, domain_end, lambda t: flux_u(domain_end, t))
constraint_manager.add_neumann(1, 0, domain_end, lambda t: flux_omega(domain_end, t))
constraint_manager.add_neumann(2, 0, domain_end, lambda t: flux_v(domain_end, t))
constraint_manager.add_neumann(3, 0, domain_end, lambda t: flux_phi(domain_end, t))

# Map constraints to discretizations
constraint_manager.map_to_discretizations([discretization])
\end{lstlisting}

\paragraph{Return Value Assembly}
\begin{lstlisting}[language=Python, caption=create\_global\_framework Return]
# Return framework components as expected by BioNetFlux
return [problem], global_disc, constraint_manager, problem_name

# Return structure:
# - [problem]: List containing single Problem instance
# - global_disc: GlobalDiscretization with time and space parameters  
# - constraint_manager: ConstraintManager with mapped boundary conditions
# - problem_name: String identifier for the problem configuration
\end{lstlisting}

\subsection{MATLAB vs Python Implementation Comparison}
\label{subsec:comprehensive_comparison}

\subsubsection{Parameter Value Differences}

\begin{longtable}{|p{2.5cm}|p{2cm}|p{2cm}|p{6.5cm}|}
\hline
\textbf{Parameter} & \textbf{MATLAB} & \textbf{Python} & \textbf{Mathematical Impact} \\
\hline
\endhead

$\mu$ (potential diffusion) & 2.0 & 1.0 & Reduced potential diffusion in Python \\
\hline

$a$ (growth rate) & 0.0 & 1.0 & Added growth terms: $+u$ in eq.~\eqref{eq:u_equation}, $+\phi$ in eq.~\eqref{eq:phi_equation} \\
\hline

$c$ (chemical reaction) & 0.0 & 1.0 & Added reaction term: $+\omega$ in eq.~\eqref{eq:omega_equation} \\
\hline

Domain start & 0.0 & 1.0 & Shifted domain from $[0,1]$ to $[1,2]$ \\
\hline

$\lambda(x)$ function & $\text{constant}$ & $\frac{1}{1+x^2}$ & Nonlinear saturation vs linear behavior \\
\hline

\end{longtable}

\subsubsection{Initial Condition Complexity}

\begin{longtable}{|p{2cm}|p{3.5cm}|p{3.5cm}|p{4cm}|}
\hline
\textbf{Variable} & \textbf{MATLAB TestProblem.m} & \textbf{Python ooc\_test\_problem.py} & \textbf{Complexity Level} \\
\hline
\endhead

$u(x,0)$ & $\sin(2\pi x)$ & $0$ & MATLAB: Spatial oscillation \\
\hline

$\omega(x,0)$ & $0$ & $\sin(2\pi x + \pi t)$ & Python: Spatio-temporal \\
\hline

$v(x,0)$ & $0$ & $t \cdot x$ & Python: Space-time product \\
\hline

$\phi(x,0)$ & $0$ & $x^2$ & Python: Quadratic profile \\
\hline

\end{longtable}

\subsubsection{Source Term Complexity}

\begin{longtable}{|p{2cm}|p{5cm}|p{6cm}|}
\hline
\textbf{Equation} & \textbf{MATLAB (All Problems)} & \textbf{Python Implementation} \\
\hline
\endhead

$f_u$ & $0$ & $0$ \\
\hline

$f_\omega$ & $0$ & $\sin(x) + 4\pi^2\sin(x) + \pi\cos(x)$ where $x = 2\pi s + \pi t$ \\
\hline

$f_v$ & $0$ & $s + \lambda(\omega(s,t)) \cdot t \cdot s$ \\
\hline

$f_\phi$ & $0$ & $-\mu \cdot 2 + a s^2 - b t s$ \\
\hline

\end{longtable}

\subsubsection{Boundary Condition Differences}

\begin{longtable}{|p{2.5cm}|p{3cm}|p{3cm}|p{4.5cm}|}
\hline
\textbf{Flux Type} & \textbf{MATLAB} & \textbf{Python} & \textbf{Mathematical Form} \\
\hline
\endhead

$u$ flux & Zero & Zero & No difference \\
\hline

$\omega$ flux & Zero & $2\pi\cos(2\pi s + \pi t)$ & Time-dependent analytical flux \\
\hline

$v$ flux & Zero & $t$ & Linear time dependence \\
\hline

$\phi$ flux & Zero & $2s$ & Linear spatial dependence \\
\hline

\end{longtable}

\subsection{Framework Integration Patterns}
\label{subsec:framework_integration_patterns}

\subsubsection{Problem Instance Configuration}

\begin{lstlisting}[language=Python, caption=Standard Problem Configuration Pattern]
def create_global_framework():
    """Standard pattern for problem module implementation"""
    
    # 1. Parameter Definition
    # Define all physical, reaction, and coupling parameters
    parameters = np.array([nu, mu, epsilon, sigma, a, b, c, d, chi])
    
    # 2. Domain Configuration  
    # Set domain start, length, and mesh parameters
    domain_start, domain_length, n_elements = ...
    
    # 3. Function Definitions
    # Define initial conditions, source terms, and nonlinear functions
    def initial_condition_i(s, t): ...
    def force_function_i(s, t): ...
    lambda_function = lambda x: ...
    
    # 4. Problem Creation
    problem = Problem(neq, domain_start, domain_length, parameters, ...)
    problem.set_initial_condition(i, initial_condition_i)
    problem.set_force(i, force_function_i)
    problem.set_function('lambda_function', lambda_function)
    
    # 5. Discretization Setup
    discretization = Discretization(n_elements, domain_start, domain_length, ...)
    discretization.set_tau([tau_0, tau_1, tau_2, tau_3])
    
    # 6. Time Integration Configuration
    global_disc = GlobalDiscretization([discretization])
    global_disc.set_time_parameters(dt, T)
    
    # 7. Boundary Condition Setup
    constraint_manager = ConstraintManager()
    constraint_manager.add_neumann/add_dirichlet(...)
    constraint_manager.map_to_discretizations([discretization])
    
    # 8. Return Framework Components
    return [problem], global_disc, constraint_manager, problem_name
\end{lstlisting}

\subsubsection{Multi-Domain Extension Pattern}

\begin{lstlisting}[language=Python, caption=Expected Multi-Domain Problem Pattern]
def create_multi_domain_framework():
    """Expected pattern for multi-domain problems"""
    
    problems = []
    discretizations = []
    
    # Create multiple domains
    for domain_idx, domain_config in enumerate(domain_configurations):
        # Domain-specific parameters
        domain_parameters = modify_parameters_for_domain(base_parameters, domain_config)
        
        # Create domain problem
        problem = Problem(
            neq=neq,
            domain_start=domain_config['start'],
            domain_length=domain_config['length'],
            parameters=domain_parameters
        )
        
        # Set domain-specific functions
        problem.set_initial_condition(i, lambda s, t: domain_initial_i(s, t, domain_idx))
        problem.set_force(i, lambda s, t: domain_force_i(s, t, domain_idx))
        
        problems.append(problem)
        
        # Create domain discretization
        discretization = Discretization(...)
        discretizations.append(discretization)
    
    # Global discretization and time setup
    global_disc = GlobalDiscretization(discretizations)
    global_disc.set_time_parameters(dt, T)
    
    # Inter-domain constraints (junctions)
    constraint_manager = ConstraintManager()
    
    # Add junction constraints between domains
    for junction in junction_configurations:
        constraint_manager.add_trace_continuity(...)
        constraint_manager.add_flux_continuity(...)
        constraint_manager.add_kedem_katchalsky(...)
    
    # Add boundary constraints for exterior boundaries
    constraint_manager.add_dirichlet/add_neumann(...)
    
    constraint_manager.map_to_discretizations(discretizations)
    
    return problems, global_disc, constraint_manager, problem_name
\end{lstlisting}

\subsection{Parameter Sensitivity Analysis Methods}
\label{subsec:parameter_sensitivity_methods}

\subsubsection{Parameter Variation Utilities}

\begin{lstlisting}[language=Python, caption=Parameter Sensitivity Analysis Pattern]
def create_parameter_sweep(base_problem_module, parameter_variations):
    """
    Create multiple problem instances for parameter sensitivity studies
    
    Args:
        base_problem_module: String path to base problem module
        parameter_variations: Dict mapping parameter names to value lists
        
    Returns:
        List of (parameter_set, framework_components) tuples
    """
    
    # Load base framework
    base_problems, base_global_disc, base_constraints, base_name = \
        importlib.import_module(base_problem_module).create_global_framework()
    
    base_problem = base_problems[0]
    base_parameters = base_problem.parameters.copy()
    
    # Parameter index mapping
    param_indices = {
        'nu': 0, 'mu': 1, 'epsilon': 2, 'sigma': 3,
        'a': 4, 'b': 5, 'c': 6, 'd': 7, 'chi': 8
    }
    
    results = []
    
    # Generate parameter combinations
    for param_name, param_values in parameter_variations.items():
        param_idx = param_indices[param_name]
        
        for param_value in param_values:
            # Create modified parameters
            modified_params = base_parameters.copy()
            modified_params[param_idx] = param_value
            
            # Create new problem with modified parameters
            modified_problem = Problem(
                neq=base_problem.neq,
                domain_start=base_problem.domain_start,
                domain_length=base_problem.domain_length,
                parameters=modified_params,
                problem_type=base_problem.type,
                name=f"{base_problem.name}_{param_name}_{param_value}"
            )
            
            # Copy functions and conditions from base problem
            for i in range(base_problem.neq):
                if base_problem.initial_conditions[i] is not None:
                    modified_problem.set_initial_condition(i, base_problem.initial_conditions[i])
                if base_problem.force_functions[i] is not None:
                    modified_problem.set_force(i, base_problem.force_functions[i])
            
            # Copy additional functions
            for func_name, func in base_problem.additional_functions.items():
                modified_problem.set_function(func_name, func)
            
            # Create modified framework components
            modified_framework = (
                [modified_problem],
                base_global_disc,  # Reuse discretization and time setup
                base_constraints,  # Reuse boundary conditions
                f"{base_name} - {param_name}={param_value}"
            )
            
            results.append(((param_name, param_value), modified_framework))
    
    return results

# Usage example
parameter_variations = {
    'chi': [0.1, 0.5, 1.0, 2.0, 5.0],  # Chemotactic sensitivity
    'mu': [0.5, 1.0, 2.0, 4.0],        # Potential diffusion
    'a': [0.0, 0.5, 1.0, 1.5]          # Growth rate
}

sensitivity_study = create_parameter_sweep(
    'ooc1d.problems.ooc_test_problem',
    parameter_variations
)
\end{lstlisting}

\subsection{Problem Module Summary}
\label{subsec:problem_module_summary_detailed}

\subsubsection{Implemented Problems Overview}

\begin{longtable}{|p{4.5cm}|p{2cm}|p{3cm}|p{4.5cm}|}
\hline
\textbf{Module/File} & \textbf{Domain} & \textbf{Complexity} & \textbf{Key Features} \\
\hline
\endhead

\texttt{TestProblem.m} & $[0,1]$ & Medium & Non-zero $u$ initial condition, constant $\lambda$ \\
\hline

\texttt{EmptyProblem.m} & $[0,1]$ & Low & All zero initial conditions, template structure \\
\hline

\texttt{ooc\_test\_problem.py} & $[1,2]$ & High & Rich analytical test case, nonlinear $\lambda$ \\
\hline

\end{longtable}

\subsubsection{Framework Integration Summary}

\begin{longtable}{|p{5cm}|p{8cm}|}
\hline
\textbf{Component} & \textbf{Implementation Details} \\
\hline
\endhead

\texttt{create\_global\_framework} & Returns (problems, global\_discretization, constraint\_manager, problem\_name) \\
\hline

Parameter vector & Standardized 9-element array: $[\nu, \mu, \epsilon, \sigma, a, b, c, d, \chi]$ \\
\hline

Initial conditions & 4 functions for $(u, \omega, v, \phi)$ with optional time dependence \\
\hline

Source terms & 4 functions for $(f_u, f_\omega, f_v, f_\phi)$ with spatio-temporal dependence \\
\hline

Nonlinear functions & $\lambda(x)$ and $\lambda'(x)$ for velocity equation nonlinearity \\
\hline

Boundary conditions & Neumann flux specifications for all 4 equations at both boundaries \\
\hline

Time discretization & Uniform time stepping with configurable $dt$ and final time $T$ \\
\hline

Space discretization & Uniform mesh with configurable element count and stabilization parameters \\
\hline

\end{longtable}

\subsection{Expected Extensions and Variations}
\label{subsec:expected_extensions}

\subsubsection{Alternative Nonlinear Functions}

\begin{lstlisting}[language=Python, caption=Expected Alternative Lambda Functions]
# Michaelis-Menten kinetics
lambda_mm = lambda x, K=1.0, V_max=2.0: V_max * x / (K + x)
dlambda_mm = lambda x, K=1.0, V_max=2.0: V_max * K / (K + x)**2

# Hill function (cooperative binding)  
lambda_hill = lambda x, K=1.0, n=2.0: x**n / (K**n + x**n)
dlambda_hill = lambda x, K=1.0, n=2.0: n * x**(n-1) * K**n / (K**n + x**n)**2

# Exponential saturation
lambda_exp = lambda x, alpha=1.0, beta=2.0: alpha * (1.0 - np.exp(-beta * x))
dlambda_exp = lambda x, alpha=1.0, beta=2.0: alpha * beta * np.exp(-beta * x)
\end{lstlisting}

\subsubsection{Expected Multi-Domain Problem Structure}

\begin{lstlisting}[language=Python, caption=Expected Multi-Domain Problem Template]
def create_vascular_network_framework():
    """Expected vascular network problem with 3 domains"""
    
    # Domain configurations
    domains = [
        {"start": 0.0, "length": 1.0, "chi": 1.0, "name": "vessel_1"},
        {"start": 1.0, "length": 0.5, "chi": 2.0, "name": "junction"},  
        {"start": 1.5, "length": 1.0, "chi": 0.5, "name": "vessel_2"}
    ]
    
    problems = []
    discretizations = []
    
    for i, domain_config in enumerate(domains):
        # Domain-specific parameter modifications
        parameters = base_parameters.copy()
        parameters[8] = domain_config["chi"]  # Modify chemotactic sensitivity
        
        # Create domain problem
        problem = Problem(
            neq=4,
            domain_start=domain_config["start"],
            domain_length=domain_config["length"],
            parameters=parameters,
            problem_type="vascular_network",
            name=domain_config["name"]
        )
        
        # Domain-specific initial conditions and source terms
        problem.set_initial_condition(0, lambda s, t: domain_initial_u(s, t, i))
        # ... set other conditions ...
        
        problems.append(problem)
        discretizations.append(create_domain_discretization(domain_config))
    
    # Junction constraints
    constraint_manager = ConstraintManager()
    
    # Continuity at domain interfaces
    constraint_manager.add_trace_continuity(0, 0, 1, -1, 1.0)  # u continuity
    constraint_manager.add_trace_continuity(1, 0, 2, -1, 1.5)  # omega continuity
    
    # Flux continuity with permeability effects
    constraint_manager.add_kedem_katchalsky(2, 0, 1, -1, 1.0, P=0.1)  # v coupling
    constraint_manager.add_kedem_katchalsky(3, 0, 2, -1, 1.5, P=0.2)  # phi coupling
    
    # Boundary conditions at network ends
    constraint_manager.add_dirichlet(0, 0, 0.0, lambda t: 1.0)  # u inlet
    constraint_manager.add_neumann(0, 2, 2.5, lambda t: 0.0)    # u outlet
    
    global_disc = GlobalDiscretization(discretizations)
    global_disc.set_time_parameters(dt=0.01, T=2.0)
    
    constraint_manager.map_to_discretizations(discretizations)
    
    return problems, global_disc, constraint_manager, "Vascular Network Model"
\end{lstlisting}

This comprehensive documentation provides exact details of the problems folder modules, focusing specifically on the 'create\_global\_framework' implementations and their parameter configurations, without duplicating information about other BioNetFlux components that are documented separately.

% End of problems folder detailed API documentation


\section{Problems Folder Detailed API Reference}
\label{sec:problems_folder_detailed_api}

This section provides detailed documentation specifically for the problem modules in the BioNetFlux problems folder, focusing on the mathematical formulations, parameter specifications, and `create\_global\_framework` method implementations based on the actual MATLAB reference files and Python implementations.

\subsection{Problems Folder Structure}

The problems folder contains problem-specific modules that implement the mathematical models for various biological transport phenomena:

\begin{itemize}
    \item \textbf{MATLAB Reference Files}: Original problem definitions and parameters
    \item \textbf{Python Implementation Modules}: BioNetFlux-compatible problem setups
    \item \textbf{Mathematical Model Specifications}: 4-equation OrganOnChip system parameters
    \item \textbf{Boundary Condition Configurations}: Dirichlet and Neumann boundary setups
\end{itemize}

\subsection{Mathematical Model: OrganOnChip System}
\label{subsec:ooc_mathematical_system}

All problems in the folder implement variants of the 4-equation OrganOnChip system:

\subsubsection{Governing System}

\begin{align}
\frac{\partial u}{\partial t} &= \nu \nabla^2 u - \chi \nabla \cdot (u \nabla \phi) + a u + f_u(x,t) \label{eq:u_equation}\\
\frac{\partial \omega}{\partial t} &= \epsilon \nabla^2 \omega + c \omega + d u + f_\omega(x,t) \label{eq:omega_equation}\\
\frac{\partial v}{\partial t} &= \sigma \nabla^2 v + \lambda(\bar{\omega}) v + f_v(x,t) \label{eq:v_equation}\\
\frac{\partial \phi}{\partial t} &= \mu \nabla^2 \phi + a \phi + b v + f_\phi(x,t) \label{eq:phi_equation}
\end{align}

where $\bar{\omega} = \frac{1}{|K|} \int_K \omega \, dx$ is the cell-averaged value of $\omega$.

\subsubsection{Parameter Vector Structure}

All problem modules use the standardized parameter vector:
\begin{align}
\text{parameters} = [\nu, \mu, \epsilon, \sigma, a, b, c, d, \chi]
\end{align}

\subsection{MATLAB Reference Problems}
\label{subsec:matlab_reference_detailed}

\subsubsection{TestProblem.m Analysis}

\paragraph{Complete Parameter Configuration}
\begin{lstlisting}[language=Matlab, caption=TestProblem.m Full Parameter Setup]
% Viscosity parameters (diffusion coefficients)
nu = 1.0;      % Cell diffusion coefficient
mu = 2.0;      % Potential diffusion coefficient  
epsilon = 1.0; % Chemical diffusion coefficient
sigma = 1.0;   % Velocity diffusion coefficient

% Reaction parameters
a = 0.0;       % Growth/decay rate (u and phi equations)
c = 0.0;       % Chemical reaction rate (omega equation)

% Domain specification
A = 0;         % Domain start coordinate
L = 1.0;       % Domain length

% Coupling parameters
b = 1.0;       % Velocity-potential coupling strength
d = 1.0;       % Cell-chemical coupling strength
chi = 1.0;     % Chemotactic sensitivity parameter

% Nonlinearity specification
lambda = @(x) constant_function(x);  % Constant nonlinear function
\end{lstlisting}

\paragraph{OoC\_pbParameters Function}
\begin{lstlisting}[language=Matlab, caption=Parameter Structure Creation]
function problem = OoC_pbParameters(a,b,c,d,chi,lambda,mu,nu,epsilon,sigma,A,L)
    s = whos; % Get all workspace variables
    problem = cell2struct({s.name}.', {s.name}); % Create structure
    eval(structvars(problem,0).'); % Evaluate structure variables
end
\end{lstlisting}

\paragraph{Initial Conditions Specification}
\begin{lstlisting}[language=Matlab, caption=TestProblem.m Initial Conditions]
% Cell density u: Non-trivial sinusoidal profile
problem.u0{1} = @(x,t) sin(2*pi*x);  

% Chemical concentration omega: Zero initial condition
problem.u0{2} = @(x,t) zeros(size(x));

% Velocity field v: Zero initial condition  
problem.u0{3} = @(x,t) zeros(size(x));

% Potential field phi: Zero initial condition
problem.u0{4} = @(x,t) zeros(size(x));

% Gradient initial condition
problem.phix0 = @(x,t) zeros(size(x));
\end{lstlisting}

\paragraph{Source Term Configuration}
\begin{lstlisting}[language=Matlab, caption=TestProblem.m Source Terms]
% All source terms are zero - homogeneous problem
force{1} = @(x,t) zeros(size(x));  % f_u = 0
force{2} = @(x,t) zeros(size(x));  % f_omega = 0  
force{3} = @(x,t) zeros(size(x));  % f_v = 0
force{4} = @(x,t) zeros(size(x));  % f_phi = 0
\end{lstlisting}

\paragraph{Boundary Condition Setup}
\begin{lstlisting}[language=Matlab, caption=TestProblem.m Boundary Conditions]
% Neumann flux conditions at both boundaries
% Left boundary (x = A = 0)
problem.fluxu0{1} = @(t) 0.;  % Zero flux for u
problem.fluxu0{2} = @(t) 0.;  % Zero flux for omega
problem.fluxu0{3} = @(t) 0.;  % Zero flux for v
problem.fluxu0{4} = @(t) 0.;  % Zero flux for phi

% Right boundary (x = A + L = 1)
problem.fluxu1{1} = @(t) 0.;  % Zero flux for u
problem.fluxu1{2} = @(t) 0.;  % Zero flux for omega
problem.fluxu1{3} = @(t) 0.;  % Zero flux for v
problem.fluxu1{4} = @(t) 0.;  % Zero flux for phi

% Neumann data vector (8×1: 2 boundaries × 4 equations)
problem.NeumannData = zeros(8,1);
\end{lstlisting}

\subsubsection{EmptyProblem.m Analysis}

\paragraph{Parameter Differences from TestProblem.m}
\begin{lstlisting}[language=Matlab, caption=EmptyProblem.m Parameter Configuration]
% IDENTICAL to TestProblem.m:
nu = 1.0; mu = 2.0; epsilon = 1.0; sigma = 1.0;
a = 0.0; c = 0.0;
A = 0; L = 1.0;
b = 1.0; d = 1.0; chi = 1.0;
lambda = @(x) constant_function(x);
\end{lstlisting}

\paragraph{Key Difference: Initial Conditions}
\begin{lstlisting}[language=Matlab, caption=EmptyProblem.m Initial Conditions]
% ALL initial conditions are zero (unlike TestProblem.m)
problem.u0{1} = @(x,t) zeros(size(x));  % Zero cell density
problem.u0{2} = @(x,t) zeros(size(x));  % Zero chemical
problem.u0{3} = @(x,t) zeros(size(x));  % Zero velocity
problem.u0{4} = @(x,t) zeros(size(x));  % Zero potential

problem.phix0 = @(x,t) zeros(size(x));  % Zero gradient
\end{lstlisting}

\paragraph{Source Terms and Boundary Conditions}
\begin{lstlisting}[language=Matlab, caption=EmptyProblem.m Additional Configuration]
% Source terms - also all zero
force_u = @(x,t) zeros(size(x));
force_phi = @(x,t) zeros(size(x));  
force_v = @(x,t) zeros(size(x));
force_omega = @(x,t) zeros(size(x));

% Boundary conditions - identical to TestProblem.m
% All zero Neumann flux conditions
problem.fluxu0{1-4} = @(t) 0.;  % Left boundary
problem.fluxu1{1-4} = @(t) 0.;  % Right boundary
problem.NeumannData = zeros(8,1);
\end{lstlisting}

\subsubsection{MATLAB Problems Comparison}

\begin{longtable}{|p{3cm}|p{4cm}|p{4cm}|p{2cm}|}
\hline
\textbf{Aspect} & \textbf{TestProblem.m} & \textbf{EmptyProblem.m} & \textbf{Purpose} \\
\hline
\endhead

Parameters & $\nu=1, \mu=2, \epsilon=1, \sigma=1$ & Identical & Reference values \\
\hline

Reaction & $a=0, c=0$ & Identical & No reaction terms \\
\hline

Coupling & $b=1, d=1, \chi=1$ & Identical & Unit coupling \\
\hline

Domain & $[0, 1]$ & Identical & Unit interval \\
\hline

Nonlinearity & $\lambda(x) = \text{constant}$ & Identical & Linear case \\
\hline

Initial u & $\sin(2\pi x)$ & $0$ & Non-trivial vs zero \\
\hline

Initial $\omega,v,\phi$ & $0$ & $0$ & Both zero \\
\hline

Source terms & All zero & All zero & Homogeneous \\
\hline

Boundary conditions & Zero Neumann & Zero Neumann & Natural boundaries \\
\hline

Use case & Non-trivial initial test & Baseline/template & Testing framework \\
\hline

\end{longtable}

\subsection{Python Implementation: ooc\_test\_problem.py}
\label{subsec:python_ooc_implementation}

\subsubsection{create\_global\_framework Method Analysis}

\paragraph{Method Signature and Overview}
\begin{lstlisting}[language=Python, caption=create\_global\_framework Method Signature]
def create_global_framework():
    """
    OrganOnChip test problem - Python port from MATLAB TestProblem.m
    
    Returns:
        Tuple: (problems, global_discretization, constraint_manager, problem_name)
            - problems: List[Problem] - Single domain problem instance
            - global_discretization: GlobalDiscretization - Time and space discretization
            - constraint_manager: ConstraintManager - Boundary condition manager
            - problem_name: str - Descriptive problem name
    """
\end{lstlisting}

\paragraph{Mesh and Global Configuration}
\begin{lstlisting}[language=Python, caption=Python Global Configuration]
# Mesh parameters
n_elements = 20  # Spatial discretization fineness

# Global problem structure  
ndom = 1        # Single domain
neq = 4         # Four coupled equations
T = 1.0         # Final simulation time
dt = 0.1        # Time step size

problem_name = "OrganOnChip Test Problem"
\end{lstlisting}

\paragraph{Parameter Configuration Differences}
\begin{lstlisting}[language=Python, caption=Python Parameter Setup vs MATLAB]
# Physical parameters - DIFFERENCES from MATLAB highlighted
nu = 1.0      # SAME as MATLAB
mu = 1.0      # DIFFERENT: MATLAB uses mu = 2.0
epsilon = 1.0 # SAME as MATLAB  
sigma = 1.0   # SAME as MATLAB

# Reaction parameters - DIFFERENCES from MATLAB
a = 1.0       # DIFFERENT: MATLAB uses a = 0.0
c = 1.0       # DIFFERENT: MATLAB uses c = 0.0

# Coupling parameters - SAME as MATLAB
b = 1.0       # SAME as MATLAB
d = 1.0       # SAME as MATLAB  
chi = 1.0     # SAME as MATLAB

# Domain configuration - DIFFERENT from MATLAB
domain_start = 1.0    # DIFFERENT: MATLAB uses A = 0
domain_length = 1.0   # SAME: MATLAB uses L = 1.0

# Parameter vector assembly [nu, mu, epsilon, sigma, a, b, c, d, chi]
parameters = np.array([nu, mu, epsilon, sigma, a, b, c, d, chi])
\end{lstlisting}

\paragraph{Nonlinear Function Configuration}
\begin{lstlisting}[language=Python, caption=Python Nonlinear Function Setup]
# DIFFERENT from MATLAB: Non-constant nonlinearity
lambda_function = lambda x: 1.0/(1.0 + x**2)      # Saturation-type
dlambda_function = lambda x: -2.0*x/(1.0 + x**2)**2  # Analytical derivative

# MATLAB equivalent (for comparison):
# lambda = @(x) constant_function(x);  % Returns ones(size(x))
\end{lstlisting}

\paragraph{Initial Condition Functions}
\begin{lstlisting}[language=Python, caption=Python Initial Conditions - Complex]
def initial_u(s, t=0.0):
    """Cell density initial condition"""
    s = np.asarray(s)
    return 0.0 * s  # Zero initial condition (like EmptyProblem.m)

def initial_omega(s, t=0.0):
    """Chemical concentration - TIME-DEPENDENT initial condition"""
    s = np.asarray(s)
    return np.sin(2 * np.pi * s + np.pi * t)  # MATLAB: zeros(size(x))

def initial_v(s, t=0.0):
    """Velocity field - SPACE-TIME coupled initial condition"""
    s = np.asarray(s)
    return t * s  # MATLAB: zeros(size(x))

def initial_phi(s, t=0.0):
    """Potential field - QUADRATIC spatial profile"""
    return s ** 2  # MATLAB: zeros(size(x))
\end{lstlisting}

\paragraph{Source Term Functions}
\begin{lstlisting}[language=Python, caption=Python Source Terms - Non-Zero]
def force_u(s, t):
    """Cell density source - remains zero"""
    s = np.asarray(s)
    return np.zeros_like(s)  # SAME as MATLAB

def force_omega(s, t):
    """Chemical source - COMPLEX analytical form"""
    s = np.asarray(s)
    x = 2 * np.pi * s + np.pi * t
    return np.sin(x) + 4 * np.pi**2 * np.sin(x) + np.pi * np.cos(x)
    # MATLAB: zeros(size(x))

def force_v(s, t):
    """Velocity source - NONLINEAR coupling"""
    omega_val = initial_omega(s, t)
    lambda_val = lambda_function(omega_val)
    s = np.asarray(s)
    return s + lambda_val * t * s
    # MATLAB: zeros(size(x))

def force_phi(s, t):
    """Potential source - MULTI-TERM expression"""
    s = np.asarray(s)
    return - mu * 2.0 * np.ones_like(s) + a * s**2 - b * t * s
    # MATLAB: zeros(size(x))
\end{lstlisting}

\paragraph{Problem Instance Creation}
\begin{lstlisting}[language=Python, caption=Python Problem Instance Setup]
# Create Problem instance with BioNetFlux framework
problem = Problem(
    neq=neq,                    # 4 equations
    domain_start=domain_start,  # 1.0 (vs MATLAB A=0) 
    domain_length=domain_length, # 1.0 (same as MATLAB L)
    parameters=parameters,       # 9-element parameter vector
    problem_type="organ_on_chip", # Problem classification
    name="ooc_test"             # Instance identifier
)

# Set nonlinear functions using flexible interface
problem.set_function('lambda_function', lambda_function)  
problem.set_function('dlambda_function', dlambda_function)

# Set source terms for all 4 equations
problem.set_force(0, lambda s, t: force_u(s, t))      # u equation
problem.set_force(1, lambda s, t: force_omega(s, t))  # omega equation  
problem.set_force(2, lambda s, t: force_v(s, t))      # v equation
problem.set_force(3, lambda s, t: force_phi(s, t))    # phi equation

# Set initial conditions for all 4 equations
problem.set_initial_condition(0, initial_u)   # u initial condition
problem.set_initial_condition(1, initial_omega) # omega initial condition
problem.set_initial_condition(2, initial_v)   # v initial condition
problem.set_initial_condition(3, initial_phi) # phi initial condition
\end{lstlisting}

\paragraph{Discretization Configuration}
\begin{lstlisting}[language=Python, caption=Python Discretization Setup]
# Spatial discretization
discretization = Discretization(
    n_elements=n_elements,        # 20 elements
    domain_start=domain_start,    # 1.0
    domain_length=domain_length,  # 1.0
    stab_constant=1.0            # Stabilization constant
)

# Stabilization parameters for HDG method
tau_u = 1.0 / discretization.element_length     # Scaled by mesh size
tau_omega = 1.0                                 # Constant
tau_v = 1.0                                     # Constant  
tau_phi = 1.0                                   # Constant

discretization.set_tau([tau_u, tau_omega, tau_v, tau_phi])

# Global discretization with time parameters
global_disc = GlobalDiscretization([discretization])
global_disc.set_time_parameters(dt, T)  # dt=0.1, T=1.0
\end{lstlisting}

\paragraph{Boundary Condition Configuration}
\begin{lstlisting}[language=Python, caption=Python Boundary Condition Setup]
# Analytical flux functions for Neumann boundary conditions
flux_u = lambda s, t: 0.0  # SAME as MATLAB: zero flux
flux_omega = lambda s, t: 2 * np.pi * np.cos(2 * np.pi * s + np.pi * t)  # NON-ZERO
flux_v = lambda s, t: t    # TIME-DEPENDENT flux  
flux_phi = lambda s, t: 2 * s  # SPACE-DEPENDENT flux

# Create constraint manager
constraint_manager = ConstraintManager()
domain_end = domain_start + domain_length  # 2.0

# Add Neumann boundary conditions for all equations
# Left boundary (x = domain_start = 1.0)
constraint_manager.add_neumann(0, 0, domain_start, lambda t: -flux_u(domain_start, t))
constraint_manager.add_neumann(1, 0, domain_start, lambda t: -flux_omega(domain_start, t))
constraint_manager.add_neumann(2, 0, domain_start, lambda t: -flux_v(domain_start, t))
constraint_manager.add_neumann(3, 0, domain_start, lambda t: -flux_phi(domain_start, t))

# Right boundary (x = domain_end = 2.0)  
constraint_manager.add_neumann(0, 0, domain_end, lambda t: flux_u(domain_end, t))
constraint_manager.add_neumann(1, 0, domain_end, lambda t: flux_omega(domain_end, t))
constraint_manager.add_neumann(2, 0, domain_end, lambda t: flux_v(domain_end, t))
constraint_manager.add_neumann(3, 0, domain_end, lambda t: flux_phi(domain_end, t))

# Map constraints to discretizations
constraint_manager.map_to_discretizations([discretization])
\end{lstlisting}

\paragraph{Return Value Assembly}
\begin{lstlisting}[language=Python, caption=create\_global\_framework Return]
# Return framework components as expected by BioNetFlux
return [problem], global_disc, constraint_manager, problem_name

# Return structure:
# - [problem]: List containing single Problem instance
# - global_disc: GlobalDiscretization with time and space parameters  
# - constraint_manager: ConstraintManager with mapped boundary conditions
# - problem_name: String identifier for the problem configuration
\end{lstlisting}

\subsection{MATLAB vs Python Implementation Comparison}
\label{subsec:comprehensive_comparison}

\subsubsection{Parameter Value Differences}

\begin{longtable}{|p{2.5cm}|p{2cm}|p{2cm}|p{6.5cm}|}
\hline
\textbf{Parameter} & \textbf{MATLAB} & \textbf{Python} & \textbf{Mathematical Impact} \\
\hline
\endhead

$\mu$ (potential diffusion) & 2.0 & 1.0 & Reduced potential diffusion in Python \\
\hline

$a$ (growth rate) & 0.0 & 1.0 & Added growth terms: $+u$ in eq.~\eqref{eq:u_equation}, $+\phi$ in eq.~\eqref{eq:phi_equation} \\
\hline

$c$ (chemical reaction) & 0.0 & 1.0 & Added reaction term: $+\omega$ in eq.~\eqref{eq:omega_equation} \\
\hline

Domain start & 0.0 & 1.0 & Shifted domain from $[0,1]$ to $[1,2]$ \\
\hline

$\lambda(x)$ function & $\text{constant}$ & $\frac{1}{1+x^2}$ & Nonlinear saturation vs linear behavior \\
\hline

\end{longtable}

\subsubsection{Initial Condition Complexity}

\begin{longtable}{|p{2cm}|p{3.5cm}|p{3.5cm}|p{4cm}|}
\hline
\textbf{Variable} & \textbf{MATLAB TestProblem.m} & \textbf{Python ooc\_test\_problem.py} & \textbf{Complexity Level} \\
\hline
\endhead

$u(x,0)$ & $\sin(2\pi x)$ & $0$ & MATLAB: Spatial oscillation \\
\hline

$\omega(x,0)$ & $0$ & $\sin(2\pi x + \pi t)$ & Python: Spatio-temporal \\
\hline

$v(x,0)$ & $0$ & $t \cdot x$ & Python: Space-time product \\
\hline

$\phi(x,0)$ & $0$ & $x^2$ & Python: Quadratic profile \\
\hline

\end{longtable}

\subsubsection{Source Term Complexity}

\begin{longtable}{|p{2cm}|p{5cm}|p{6cm}|}
\hline
\textbf{Equation} & \textbf{MATLAB (All Problems)} & \textbf{Python Implementation} \\
\hline
\endhead

$f_u$ & $0$ & $0$ \\
\hline

$f_\omega$ & $0$ & $\sin(x) + 4\pi^2\sin(x) + \pi\cos(x)$ where $x = 2\pi s + \pi t$ \\
\hline

$f_v$ & $0$ & $s + \lambda(\omega(s,t)) \cdot t \cdot s$ \\
\hline

$f_\phi$ & $0$ & $-\mu \cdot 2 + a s^2 - b t s$ \\
\hline

\end{longtable}

\subsubsection{Boundary Condition Differences}

\begin{longtable}{|p{2.5cm}|p{3cm}|p{3cm}|p{4.5cm}|}
\hline
\textbf{Flux Type} & \textbf{MATLAB} & \textbf{Python} & \textbf{Mathematical Form} \\
\hline
\endhead

$u$ flux & Zero & Zero & No difference \\
\hline

$\omega$ flux & Zero & $2\pi\cos(2\pi s + \pi t)$ & Time-dependent analytical flux \\
\hline

$v$ flux & Zero & $t$ & Linear time dependence \\
\hline

$\phi$ flux & Zero & $2s$ & Linear spatial dependence \\
\hline

\end{longtable}

\subsection{Framework Integration Patterns}
\label{subsec:framework_integration_patterns}

\subsubsection{Problem Instance Configuration}

\begin{lstlisting}[language=Python, caption=Standard Problem Configuration Pattern]
def create_global_framework():
    """Standard pattern for problem module implementation"""
    
    # 1. Parameter Definition
    # Define all physical, reaction, and coupling parameters
    parameters = np.array([nu, mu, epsilon, sigma, a, b, c, d, chi])
    
    # 2. Domain Configuration  
    # Set domain start, length, and mesh parameters
    domain_start, domain_length, n_elements = ...
    
    # 3. Function Definitions
    # Define initial conditions, source terms, and nonlinear functions
    def initial_condition_i(s, t): ...
    def force_function_i(s, t): ...
    lambda_function = lambda x: ...
    
    # 4. Problem Creation
    problem = Problem(neq, domain_start, domain_length, parameters, ...)
    problem.set_initial_condition(i, initial_condition_i)
    problem.set_force(i, force_function_i)
    problem.set_function('lambda_function', lambda_function)
    
    # 5. Discretization Setup
    discretization = Discretization(n_elements, domain_start, domain_length, ...)
    discretization.set_tau([tau_0, tau_1, tau_2, tau_3])
    
    # 6. Time Integration Configuration
    global_disc = GlobalDiscretization([discretization])
    global_disc.set_time_parameters(dt, T)
    
    # 7. Boundary Condition Setup
    constraint_manager = ConstraintManager()
    constraint_manager.add_neumann/add_dirichlet(...)
    constraint_manager.map_to_discretizations([discretization])
    
    # 8. Return Framework Components
    return [problem], global_disc, constraint_manager, problem_name
\end{lstlisting}

\subsubsection{Multi-Domain Extension Pattern}

\begin{lstlisting}[language=Python, caption=Expected Multi-Domain Problem Pattern]
def create_multi_domain_framework():
    """Expected pattern for multi-domain problems"""
    
    problems = []
    discretizations = []
    
    # Create multiple domains
    for domain_idx, domain_config in enumerate(domain_configurations):
        # Domain-specific parameters
        domain_parameters = modify_parameters_for_domain(base_parameters, domain_config)
        
        # Create domain problem
        problem = Problem(
            neq=neq,
            domain_start=domain_config['start'],
            domain_length=domain_config['length'],
            parameters=domain_parameters
        )
        
        # Set domain-specific functions
        problem.set_initial_condition(i, lambda s, t: domain_initial_i(s, t, domain_idx))
        problem.set_force(i, lambda s, t: domain_force_i(s, t, domain_idx))
        
        problems.append(problem)
        
        # Create domain discretization
        discretization = Discretization(...)
        discretizations.append(discretization)
    
    # Global discretization and time setup
    global_disc = GlobalDiscretization(discretizations)
    global_disc.set_time_parameters(dt, T)
    
    # Inter-domain constraints (junctions)
    constraint_manager = ConstraintManager()
    
    # Add junction constraints between domains
    for junction in junction_configurations:
        constraint_manager.add_trace_continuity(...)
        constraint_manager.add_flux_continuity(...)
        constraint_manager.add_kedem_katchalsky(...)
    
    # Add boundary constraints for exterior boundaries
    constraint_manager.add_dirichlet/add_neumann(...)
    
    constraint_manager.map_to_discretizations(discretizations)
    
    return problems, global_disc, constraint_manager, problem_name
\end{lstlisting}

\subsection{Parameter Sensitivity Analysis Methods}
\label{subsec:parameter_sensitivity_methods}

\subsubsection{Parameter Variation Utilities}

\begin{lstlisting}[language=Python, caption=Parameter Sensitivity Analysis Pattern]
def create_parameter_sweep(base_problem_module, parameter_variations):
    """
    Create multiple problem instances for parameter sensitivity studies
    
    Args:
        base_problem_module: String path to base problem module
        parameter_variations: Dict mapping parameter names to value lists
        
    Returns:
        List of (parameter_set, framework_components) tuples
    """
    
    # Load base framework
    base_problems, base_global_disc, base_constraints, base_name = \
        importlib.import_module(base_problem_module).create_global_framework()
    
    base_problem = base_problems[0]
    base_parameters = base_problem.parameters.copy()
    
    # Parameter index mapping
    param_indices = {
        'nu': 0, 'mu': 1, 'epsilon': 2, 'sigma': 3,
        'a': 4, 'b': 5, 'c': 6, 'd': 7, 'chi': 8
    }
    
    results = []
    
    # Generate parameter combinations
    for param_name, param_values in parameter_variations.items():
        param_idx = param_indices[param_name]
        
        for param_value in param_values:
            # Create modified parameters
            modified_params = base_parameters.copy()
            modified_params[param_idx] = param_value
            
            # Create new problem with modified parameters
            modified_problem = Problem(
                neq=base_problem.neq,
                domain_start=base_problem.domain_start,
                domain_length=base_problem.domain_length,
                parameters=modified_params,
                problem_type=base_problem.type,
                name=f"{base_problem.name}_{param_name}_{param_value}"
            )
            
            # Copy functions and conditions from base problem
            for i in range(base_problem.neq):
                if base_problem.initial_conditions[i] is not None:
                    modified_problem.set_initial_condition(i, base_problem.initial_conditions[i])
                if base_problem.force_functions[i] is not None:
                    modified_problem.set_force(i, base_problem.force_functions[i])
            
            # Copy additional functions
            for func_name, func in base_problem.additional_functions.items():
                modified_problem.set_function(func_name, func)
            
            # Create modified framework components
            modified_framework = (
                [modified_problem],
                base_global_disc,  # Reuse discretization and time setup
                base_constraints,  # Reuse boundary conditions
                f"{base_name} - {param_name}={param_value}"
            )
            
            results.append(((param_name, param_value), modified_framework))
    
    return results

# Usage example
parameter_variations = {
    'chi': [0.1, 0.5, 1.0, 2.0, 5.0],  # Chemotactic sensitivity
    'mu': [0.5, 1.0, 2.0, 4.0],        # Potential diffusion
    'a': [0.0, 0.5, 1.0, 1.5]          # Growth rate
}

sensitivity_study = create_parameter_sweep(
    'ooc1d.problems.ooc_test_problem',
    parameter_variations
)
\end{lstlisting}

\subsection{Problem Module Summary}
\label{subsec:problem_module_summary_detailed}

\subsubsection{Implemented Problems Overview}

\begin{longtable}{|p{4.5cm}|p{2cm}|p{3cm}|p{4.5cm}|}
\hline
\textbf{Module/File} & \textbf{Domain} & \textbf{Complexity} & \textbf{Key Features} \\
\hline
\endhead

\texttt{TestProblem.m} & $[0,1]$ & Medium & Non-zero $u$ initial condition, constant $\lambda$ \\
\hline

\texttt{EmptyProblem.m} & $[0,1]$ & Low & All zero initial conditions, template structure \\
\hline

\texttt{ooc\_test\_problem.py} & $[1,2]$ & High & Rich analytical test case, nonlinear $\lambda$ \\
\hline

\end{longtable}

\subsubsection{Framework Integration Summary}

\begin{longtable}{|p{5cm}|p{8cm}|}
\hline
\textbf{Component} & \textbf{Implementation Details} \\
\hline
\endhead

\texttt{create\_global\_framework} & Returns (problems, global\_discretization, constraint\_manager, problem\_name) \\
\hline

Parameter vector & Standardized 9-element array: $[\nu, \mu, \epsilon, \sigma, a, b, c, d, \chi]$ \\
\hline

Initial conditions & 4 functions for $(u, \omega, v, \phi)$ with optional time dependence \\
\hline

Source terms & 4 functions for $(f_u, f_\omega, f_v, f_\phi)$ with spatio-temporal dependence \\
\hline

Nonlinear functions & $\lambda(x)$ and $\lambda'(x)$ for velocity equation nonlinearity \\
\hline

Boundary conditions & Neumann flux specifications for all 4 equations at both boundaries \\
\hline

Time discretization & Uniform time stepping with configurable $dt$ and final time $T$ \\
\hline

Space discretization & Uniform mesh with configurable element count and stabilization parameters \\
\hline

\end{longtable}

\subsection{Expected Extensions and Variations}
\label{subsec:expected_extensions}

\subsubsection{Alternative Nonlinear Functions}

\begin{lstlisting}[language=Python, caption=Expected Alternative Lambda Functions]
# Michaelis-Menten kinetics
lambda_mm = lambda x, K=1.0, V_max=2.0: V_max * x / (K + x)
dlambda_mm = lambda x, K=1.0, V_max=2.0: V_max * K / (K + x)**2

# Hill function (cooperative binding)  
lambda_hill = lambda x, K=1.0, n=2.0: x**n / (K**n + x**n)
dlambda_hill = lambda x, K=1.0, n=2.0: n * x**(n-1) * K**n / (K**n + x**n)**2

# Exponential saturation
lambda_exp = lambda x, alpha=1.0, beta=2.0: alpha * (1.0 - np.exp(-beta * x))
dlambda_exp = lambda x, alpha=1.0, beta=2.0: alpha * beta * np.exp(-beta * x)
\end{lstlisting}

\subsubsection{Expected Multi-Domain Problem Structure}

\begin{lstlisting}[language=Python, caption=Expected Multi-Domain Problem Template]
def create_vascular_network_framework():
    """Expected vascular network problem with 3 domains"""
    
    # Domain configurations
    domains = [
        {"start": 0.0, "length": 1.0, "chi": 1.0, "name": "vessel_1"},
        {"start": 1.0, "length": 0.5, "chi": 2.0, "name": "junction"},  
        {"start": 1.5, "length": 1.0, "chi": 0.5, "name": "vessel_2"}
    ]
    
    problems = []
    discretizations = []
    
    for i, domain_config in enumerate(domains):
        # Domain-specific parameter modifications
        parameters = base_parameters.copy()
        parameters[8] = domain_config["chi"]  # Modify chemotactic sensitivity
        
        # Create domain problem
        problem = Problem(
            neq=4,
            domain_start=domain_config["start"],
            domain_length=domain_config["length"],
            parameters=parameters,
            problem_type="vascular_network",
            name=domain_config["name"]
        )
        
        # Domain-specific initial conditions and source terms
        problem.set_initial_condition(0, lambda s, t: domain_initial_u(s, t, i))
        # ... set other conditions ...
        
        problems.append(problem)
        discretizations.append(create_domain_discretization(domain_config))
    
    # Junction constraints
    constraint_manager = ConstraintManager()
    
    # Continuity at domain interfaces
    constraint_manager.add_trace_continuity(0, 0, 1, -1, 1.0)  # u continuity
    constraint_manager.add_trace_continuity(1, 0, 2, -1, 1.5)  # omega continuity
    
    # Flux continuity with permeability effects
    constraint_manager.add_kedem_katchalsky(2, 0, 1, -1, 1.0, P=0.1)  # v coupling
    constraint_manager.add_kedem_katchalsky(3, 0, 2, -1, 1.5, P=0.2)  # phi coupling
    
    # Boundary conditions at network ends
    constraint_manager.add_dirichlet(0, 0, 0.0, lambda t: 1.0)  # u inlet
    constraint_manager.add_neumann(0, 2, 2.5, lambda t: 0.0)    # u outlet
    
    global_disc = GlobalDiscretization(discretizations)
    global_disc.set_time_parameters(dt=0.01, T=2.0)
    
    constraint_manager.map_to_discretizations(discretizations)
    
    return problems, global_disc, constraint_manager, "Vascular Network Model"
\end{lstlisting}

This comprehensive documentation provides exact details of the problems folder modules, focusing specifically on the 'create\_global\_framework' implementations and their parameter configurations, without duplicating information about other BioNetFlux components that are documented separately.

% End of problems folder detailed API documentation
